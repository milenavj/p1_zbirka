\renewcommand{\chaptermark}[1]{\markboth{\thechapter\ #1}{#1}}
\renewcommand{\sectionmark}[1]{\markright{\thesection\ #1}}


\chapter{Uvodni zadaci}

\iffalse
\pagestyle{fancy}
\fancyhf{}
\fancyhead[RE]{\bfseries\slshape\leftmark}
\fancyhead[LO]{\bfseries\slshape\rightmark}
\fancyfoot[RO,LE]{\thepage}
\fi

\pagestyle{fancyplain}
\renewcommand{\chaptermark}[1]{\markboth{\thechapter\ #1}{#1}}
\renewcommand{\sectionmark}[1]{\markright{\thesection\ #1}}
%\lhead[\fancyplain{}{\bfseries\slshape\thepage}]{\fancyplain{}{\bfseries\slshape\rightmark}}
%\rhead[\fancyplain{}{\bfseries\slshape\leftmark}]{\fancyplain{}{\bfseries\slshape\thepage }}
\lhead[\fancyplain{}{\bfseries\slshape\leftmark}]{\fancyplain{}{}}
\rhead[\fancyplain{}{}]{{\fancyplain{}{\bfseries\slshape\rightmark}}}
%\rhead[]{\fancyplain{}{\bfseries\slshape\rightmark}}
%\lhead[]{\fancyplain{}{\bfseries\slshape\leftmark}}
\cfoot{}

%\fancyhead[LE,RO]{\fancyplain{}{\bfseries\slshape\rightmark}
%\fancyhead[RE,LO]{\fancyplain{}{\bfseries\slshape\leftmark}}
\fancyfoot[LE]{\thepage} 
\fancyfoot[RO]{\thepage} 
 

\pagenumbering{arabic}
\setcounter{page}{1}


%\section{Samo ispis}

\begin{Exercise}[label=v1.1_01] 
Napisati program koji na standardni izlaz ispisuje tekst \kckod{Zdravo svima!}.

\begin{miditest}
\begin{upotreba}{1}
#\naslovInt#
#\izlaz{Zdravo svima!}#
\end{upotreba}
\end{miditest}

\linkresenje{v1.1_01}
\end{Exercise}
\begin{Answer}[ref=v1.1_01]
\includecode{resenja/1_KontrolaToka/1.1_UvodniZadaci/1_01.c}
\end{Answer}


%\section{Celi brojevi}

%\komentarM{Tamo gde je pretpostavka da je pitanju pozitivan ceo broj, treba staviti da je tip unsigned.}

\begin{Exercise}[label=p1.1_01] 
Napisati program za uneti ceo broj ispisuje taj broj, njegov kvadrat i njegov kub. 

\begin{miditest}
\begin{upotreba}{1}
#\naslovInt#
#\izlaz{Unesite ceo broj:}\ulaz{4}#
#\izlaz{Kvadrat: 16}#
#\izlaz{Kub: 64}#
\end{upotreba}
\end{miditest}
\begin{miditest}
\begin{upotreba}{2}
#\naslovInt#
#\izlaz{Unesite ceo broj:}\ulaz{-14}#
#\izlaz{Kvadrat: 196}#
#\izlaz{Kub: -2744}#
\end{upotreba}
\end{miditest}

\linkresenje{p1.1_01}
\end{Exercise}
\begin{Answer}[ref=p1.1_01]
\includecode{resenja/1_KontrolaToka/1.1_UvodniZadaci/1_02.c}
\end{Answer}


\begin{Exercise}[label=v1.1_02] 
Napisati program koji za uneta dva cela broja ispisuje najpre unete vrednosti, a zatim i njihov zbir, razliku, proizvod, ceo deo pri deljenju prvog broja drugim brojem 
i ostatak pri deljenju prvog broja drugim brojem. 
\napomena{Pretpostaviti da je unos korektan, tj.~da druga uneta vrednost nije 0.}

\begin{miditest}
\begin{upotreba}{1}
#\naslovInt#
#\izlaz{Unesi vrednost celobrojne promenljive x:}\ulaz{7}#
#\izlaz{Unesi vrednost celobrojne promenljive y:}\ulaz{2}#
#\izlaz{7 + 2 = 9}#
#\izlaz{7 - 2 = 5}#
#\izlaz{7 * 2 = 14}#
#\izlaz{7 / 2 = 3}#
#\izlaz{7 \% 2 = 1}#
\end{upotreba}
\end{miditest}
\begin{miditest}
\begin{upotreba}{2}
#\naslovInt#
#\izlaz{Unesi vrednost celobrojne promenljive x:}\ulaz{-3}#
#\izlaz{Unesi vrednost celobrojne promenljive y:}\ulaz{8}#
#\izlaz{-3 + 8 = 5}#
#\izlaz{-3 - 8 = -11}#
#\izlaz{-3 * 8 = -24}#
#\izlaz{-3 / 8 = 0}#
#\izlaz{-3 \% 8 = -3}#
\end{upotreba}
\end{miditest}

\linkresenje{v1.1_02}
\end{Exercise}
\begin{Answer}[ref=v1.1_02]
\includecode{resenja/1_KontrolaToka/1.1_UvodniZadaci/1_03.c}
\end{Answer}

%\subsection{Prodavnica}


\begin{Exercise}[label=p1_005]
Napisati program koji pomaže kasirki da izračuna ukupan račun ako su poznate cene dva kupljena artikla. 
\napomena{Pretpostaviti da su cene artikala pozitivni celi brojevi i da je unos korektan.}

\begin{miditest}
\begin{upotreba}{1}
#\naslovInt#
#\izlaz{Unesi cenu prvog artikla:}\ulaz{173}#
#\izlaz{Unesi cenu drugog artikla:}\ulaz{2024}#
#\izlaz{Ukupna cena iznosi 2197}#
\end{upotreba}
\end{miditest}
\begin{miditest}
\begin{upotreba}{2}
#\naslovInt#
#\izlaz{Unesi cenu prvog artikla:}\ulaz{384}#
#\izlaz{Unesi cenu drugog artikla:}\ulaz{555}#
#\izlaz{Ukupna cena iznosi 940}#
\end{upotreba}
\end{miditest}

\linkresenje{p1_005}
\end{Exercise}
\begin{Answer}[ref=p1_005]

Rešenje ovog zadatka svodi se na rešenje zadatka \ref{v1.1_02}, na deo koji se odnosi na izračunavanje zbira dva broja. Zbog pretpostavke da su cene artikala pozitivni celi brojevi, tip promenljivih za artikle treba da bude \kckod{unsigned int}.
\end{Answer}

\begin{Exercise}[label=p1_05]
Napisati program koji za unetu količinu jabuka u kilogramima i unetu
cenu po kilogramu ispisuje ukupnu vrednost date količine jabuka. \napomena{Pretpostaviti da je cena jabuka pozitivan ceo broj i da je unos korektan.} 

\begin{miditest}
\begin{upotreba}{1}
#\naslovInt#
#\izlaz{Unesite kolicinu jabuka (u kg):}\ulaz{6}#
#\izlaz{Unesite cenu (u dinarima):}\ulaz{82}#
#\izlaz{Molimo platite 492 dinara.}#
\end{upotreba}
\end{miditest}
\begin{miditest}
\begin{upotreba}{1}
#\naslovInt#
#\izlaz{Unesite kolicinu jabuka (u kg):}\ulaz{10}#
#\izlaz{Unesite cenu (u dinarima):}\ulaz{93}#
#\izlaz{Molimo platite 930 dinara.}#
\end{upotreba}
\end{miditest}
\linkresenje{p1_05}
\end{Exercise}
\begin{Answer}[ref=p1_05]

Rešenje ovog zadatka svodi se na rešenje zadatka \ref{v1.1_02}, na deo koji se odnosi na izračunavanje proizvoda dva broja. Zbog pretpostavke da su cene artikala pozitivni celi brojevi, tip promenljivih za artikle treba da bude \kckod{unsigned int}.
\end{Answer}


\begin{Exercise}[label=p1.1_06] 
Napisati program koji pomaže kasirki da obračuna kusur koji treba da vrati kupcu. Za unetu cenu artikla, količinu artikla i iznos koji je kupac dao, program treba da ispiše vrednost kusura. \napomena{Pretpostaviti da su cene svih artikala pozitivni celi brojevi, kao i da su unete vrednosti ispravne, tj.~da se može vratiti kusur.}


\begin{miditest}
\begin{upotreba}{1}
#\naslovInt#
#\izlaz{Unesite cenu, kolicinu i iznos:}\ulaz{132 2 500}#
#\izlaz{Kusur je 236 dinara.}#
\end{upotreba}
\end{miditest}
\begin{miditest}
\begin{upotreba}{2}
#\naslovInt#
#\izlaz{Unesite cenu, kolicinu i iznos:}\ulaz{59 6 2000}#
#\izlaz{Kusur je 1646 dinara.}#
\end{upotreba}
\end{miditest}

\linkresenje{p1.1_06}
\end{Exercise}
\begin{Answer}[ref=p1.1_06]
\includecode{resenja/1_KontrolaToka/1.1_UvodniZadaci/1_06.c}
\end{Answer}




\begin{Exercise}[label=p1.1_11] 
Napisati program koji za uneta vremena poletanja i sletanja aviona  ispisuje dužinu trajanja leta. \napomena{Pretpostaviti da su poletanje i sletanje u istom danu kao i da su sve vrednosti ispravno unete.}

%TODO U nekom trenutku povezati sa odgovarajucim IF-om i videti sta i kako

\begin{miditest}
\begin{upotreba}{1}
#\naslovInt#
#\izlaz{Unesite vreme poletanja:}\ulaz{8 5}#
#\izlaz{Unesite vreme sletanja:}\ulaz{12 41}#
#\izlaz{Duzina trajanja leta je 4 h i 36 min}#
\end{upotreba}
\end{miditest}
\begin{miditest}
\begin{upotreba}{2}
#\naslovInt#
#\izlaz{Unesite vreme poletanja:}\ulaz{13 20}#
#\izlaz{Unesite vreme sletanja:}\ulaz{18 45}#
#\izlaz{Duzina trajanja leta je 5 h i 25 min}#
\end{upotreba}
\end{miditest}

\linkresenje{p1.1_11}
\end{Exercise}
\begin{Answer}[ref=p1.1_11]
\includecode{resenja/1_KontrolaToka/1.1_UvodniZadaci/1_07.c}
\end{Answer}


%\subsection{Naredba dodele}
\begin{Exercise}[label=v1.1_10] 
Date su dve celobrojne promenljive. Napisati program koji razmenjuje njihove vrednosti.

\begin{miditest}
\begin{upotreba}{1}
#\naslovInt#
#\izlaz{Unesi dve celobrojne vrednosti:}\ulaz{5 7}#
#\izlaz{pre zamene: x=5, y=7}#
#\izlaz{posle zamene: x=7, y=5}#
\end{upotreba}
\end{miditest}
\begin{miditest}
\begin{upotreba}{2}
#\naslovInt#
#\izlaz{Unesi dve celobrojne vrednosti:}\ulaz{237 -592}#
#\izlaz{pre zamene: x=237, y=-592}#
#\izlaz{posle zamene: x=-592, y=237}#
\end{upotreba}
\end{miditest}

\linkresenje{v1.1_10}
\end{Exercise}
\begin{Answer}[ref=v1.1_10]
\includecode{resenja/1_KontrolaToka/1.1_UvodniZadaci/1_08.c}
\end{Answer}

\begin{Exercise}[label=p1_14]
Date su dve celobrojene promenljive $a$ i $b$. Napisati program koji promenljivoj $a$ dodeljuje
njihovu sumu, a promenljivoj $b$ njihovu razliku. \napomena{Ne koristiti pomoćne
promenljive}. 

\begin{miditest}
\begin{upotreba}{1}
#\naslovInt#
#\izlaz{Unesi dve celobrojne vrednosti:}\ulaz{5 7}#
#\izlaz{Nove vrednosti su: a=12, b=-2}#
\end{upotreba}
\end{miditest}
\begin{miditest}
\begin{upotreba}{2}
#\naslovInt#
#\izlaz{Unesi dve celobrojne vrednosti:}\ulaz{237 -592}#
#\izlaz{Nove vrednosti su: a=-355, b=829}#
\end{upotreba}
\end{miditest}

\end{Exercise}

%\subsection{Cifre}


%\komentarM{Da, imalo bi smisla to ujednaciti. Mozda prirodan broj ako se podrazumeva da moze da bude i nula? Pozitivan broj moze da bude realan, i zato je bolje reci pozitivan prirodan broj, ukoliko nam je za ulaz bitno da nije nula. Dakle, rekla bih prirodan ili pozitivan prirodan, nikako samo pozitivan!}


\begin{Exercise}[label=v1.1_05] 
Napisati program koji za uneti pozitivan trocifreni broj na standardni izlaz ispisuje njegove cifre jedinica, desetica i stotina. \napomena{Pretpostaviti da je unos ispravan.}

\begin{miditest}
\begin{upotreba}{1}
#\naslovInt#
#\izlaz{Unesi trocifreni broj:}\ulaz{697}#
#\izlaz{jedinica 7, desetica 9, stotina 6}#
\end{upotreba}
\end{miditest}
\begin{miditest}
\begin{upotreba}{2}
#\naslovInt#
#\izlaz{Unesi trocifreni broj:}\ulaz{504}#
#\izlaz{jedinica 4, desetica 0, stotina 5}#
\end{upotreba}
\end{miditest}
\linkresenje{v1.1_05}
\end{Exercise}
\begin{Answer}[ref=v1.1_05]
\includecode{resenja/1_KontrolaToka/1.1_UvodniZadaci/1_10.c}
\end{Answer}


\begin{Exercise}[label=v1.1_08] 
Napisati program koji za unetu cenu proizvoda ispisuje najmanji broj novčanica koje je potrebno izdvojiti prilikom plaćanja proizvoda. Na raspolaganju su novčanice od 5000, 2000, 1000, 500, 200, 100, 50, 20, 10 i 1 dinar. \napomena{Pretpostaviti da je cena proizvoda pozitivan ceo broj.}

\begin{maxitest}
\begin{upotreba}{1}
#\naslovInt#
#\izlaz{Unesite cenu proizvoda:}\ulaz{8367}#
#\izlaz{8367=1*5000+ 1*2000 +1*1000 +0*500 +1*200 +1*100 +1*50 +0*20 +1*10 +7*1}#
\end{upotreba}
\begin{upotreba}{2}
#\naslovInt#
#\izlaz{Unesite cenu proizvoda:}\ulaz{934}#
#\izlaz{934=0*5000+ 0*2000 +0*1000 +1*500 +2*200 +0*100 +0*50 +1*20 +1*10 +4*1}#
\end{upotreba}
\end{maxitest}
\linkresenje{v1.1_08}
\end{Exercise}
\begin{Answer}[ref=v1.1_08]
\includecode{resenja/1_KontrolaToka/1.1_UvodniZadaci/1_11.c}
\end{Answer}


\begin{Exercise}[label=v1.1_06] 
Napisati program koji učitava pozitivan trocifreni broj sa standardnog ulaza i ispisuje broj dobijen obrtanjem njegovih cifara. \napomena{Pretpostaviti da je unos ispravan.}

\begin{miditest}
\begin{upotreba}{1}
#\naslovInt#
#\izlaz{Unesi trocifreni broj:}\ulaz{892}#
#\izlaz{Obrnuto: 298}#
\end{upotreba}
\end{miditest}
\begin{miditest}
\begin{upotreba}{2}
#\naslovInt#
#\izlaz{Unesi trocifreni broj:}\ulaz{230}#
#\izlaz{Obrnuto: 32}#
\end{upotreba}
\end{miditest}

\linkresenje{v1.1_06}

\end{Exercise}
\begin{Answer}[ref=v1.1_06]
\includecode{resenja/1_KontrolaToka/1.1_UvodniZadaci/1_12.c}
\end{Answer}


\begin{Exercise}[label=p1.1_07] 
Napisati program koji za uneti pozitivan četvorocifreni broj:
\begin{enumerate}
\item izračunava proizvod cifara
\item izračunava razliku sume krajnjih i srednjih cifara 
\item izračunava sumu kvadrata cifara
\item izračunava broj koji se dobija ispisom cifara u obrnutom poretku
\item izračunava broj koji se dobija zamenom cifre jedinice i cifre stotine
\end{enumerate}
\napomena{Pretpostaviti da je unos ispravan.}

\begin{maxitest}
\begin{upotreba}{1}
#\naslovInt#
#\izlaz{Unesite cetvorocifreni broj:}\ulaz{2371}#
#\izlaz{Proizvod cifara: 42}#
#\izlaz{Razlika sume krajnjih i srednjih: -7}#
#\izlaz{Suma kvadrata cifara: 63}#
#\izlaz{Broj u obrnutom poretku: 1732}#
#\izlaz{Broj sa zamenjenom cifrom jedinica i stotina: 2173}#
\end{upotreba}
\begin{upotreba}{2}
#\naslovInt#
#\izlaz{Unesite cetvorocifreni broj:}\ulaz{3570}#
#\izlaz{Proizvod cifara: 0}#
#\izlaz{Razlika sume krajnjih i srednjih: -9}#
#\izlaz{Suma kvadrata cifara: 83}#
#\izlaz{Broj u obrnutom poretku: 753}#
#\izlaz{Broj sa zamenjenom cifrom jedinica i stotina: 3075}#
\end{upotreba}
\end{maxitest}
\linkresenje{p1.1_07}
\end{Exercise}
\begin{Answer}[ref=p1.1_07]
\includecode{resenja/1_KontrolaToka/1.1_UvodniZadaci/1_13.c}
\end{Answer}


\begin{Exercise}[label=p1.1_08] 
Napisati program koji ispisuje broj koji se dobija izbacivanjem cifre desetica u unetom prirodnom broju.

\begin{miditest}
\begin{upotreba}{1}
#\naslovInt#
#\izlaz{Unesite broj:}\ulaz{1349}#
#\izlaz{Rezultat je: 139}#
\end{upotreba}
\end{miditest}
\begin{miditest}
\begin{upotreba}{2}
#\naslovInt#
#\izlaz{Unesite broj:}\ulaz{825}#
#\izlaz{Rezultat je: 85}#
\end{upotreba}
\end{miditest}

%\linkresenje{p1.1_08}
\end{Exercise}
%\begin{Answer}[ref=p1.1_08]
%\includecode{resenja/1_KontrolaToka/1.1_UvodniZadaci/p1_08.c}
%\end{Answer}



\begin{Exercise}[label=p1_16]
Sa standardnog unosa se unosi pozitivan prirodan broj $n$ i pozitivan dvocifreni broj $m$. Napisati program ispisuje broj dobijen umetanjem broja $m$ između cifre stotina i cifre hiljada broja $n$. 
\napomena{Za neke ulazne podatke može se dobiti neočekivan rezultat zbog prekoračenja, što ilustruje test primer broj 2.}

\begin{miditest}
\begin{upotreba}{1}
#\naslovInt#
#\izlaz{Unesite pozitivan prirodan broj:}\ulaz{12345}#
#\izlaz{Unesite pozitivan dvocifreni broj:}\ulaz{67}#
#\izlaz{Novi broj je 1267345}#
\end{upotreba}
\end{miditest}  
\begin{miditest}
\begin{upotreba}{2}
#\naslovInt#
#\izlaz{Unesite pozitivan prirodan broj:}\ulaz{50000000}#
#\izlaz{Unesite pozitivan dvocifreni broj:}\ulaz{12}#
#\izlaz{Novi broj je 705044704}#
\end{upotreba}
\end{miditest}   

\linkresenje{p1_16}
\end{Exercise}
\begin{Answer}[ref=p1_16]
\includecode{resenja/1_KontrolaToka/1.1_UvodniZadaci/1_15.c}
\end{Answer}



%\section{Realni brojevi}

\begin{Exercise}[label=v1.1_03] 
Napisati program koji učitava realnu vrednost izraženu
   u inčima, konvertuje tu vrednost u centimetre i ispisuje je zaokruženu na dve decimale. \uputstvo{Jedan inč ima $2.54$ centimetra.}
   
\begin{miditest}
\begin{upotreba}{1}
#\naslovInt#
#\izlaz{Unesi broj inca:}\ulaz{4.69}#
#\izlaz{4.69 in = 11.91 cm}#
\end{upotreba}
\end{miditest}  
\begin{miditest}
\begin{upotreba}{2}
#\naslovInt#
#\izlaz{Unesi broj inca:}\ulaz{71.426}#
#\izlaz{71.43 in = 181.42 cm}#
\end{upotreba}
\end{miditest}   

\linkresenje{v1.1_03}
\end{Exercise}
\begin{Answer}[ref=v1.1_03]
\includecode{resenja/1_KontrolaToka/1.1_UvodniZadaci/1_16.c}
\end{Answer}


\begin{Exercise}[label=p1.1_10a] 
Napisati program koji učitava dužinu izraženu
   u miljama, konvertuje tu vrednost u kilometre i ispisuje je zaokruženu na dve decimale. \uputstvo{Jedna milja ima $1.609344$ kilometara.}
   
\begin{miditest}
\begin{upotreba}{1}
#\naslovInt#
#\izlaz{Unesi broj milja:}\ulaz{50.42}#
#\izlaz{50.42 mi = 81.14 km}#
\end{upotreba}
\end{miditest}  
\begin{miditest}
\begin{upotreba}{2}
#\naslovInt#
#\izlaz{Unesi broj milja:}\ulaz{327.128}#
#\izlaz{327.128 mi = 526.46 km}#
\end{upotreba}
\end{miditest}   

\linkresenje{p1.1_10a}
\end{Exercise}
\begin{Answer}[ref=p1.1_10a]
Zadatak se rešava analogno zadatku \ref{v1.1_03}.
\end{Answer}


\begin{Exercise}[label=p1.1_10b] 
Napisati program koji učitava težinu izraženu
   u funtama, konvertuje tu vrednost u kilograme i ispisuje je zaokruženu na dve decimale. \uputstvo{Jedna funta ima $0.45359237$ kilograma.}

\begin{miditest}
\begin{upotreba}{1}
#\naslovInt#
#\izlaz{Unesi broj funti:}\ulaz{2.78}#
#\izlaz{2.78 lb = 1.26 kg}#
\end{upotreba}
\end{miditest}  
\begin{miditest}
\begin{upotreba}{2}
#\naslovInt#
#\izlaz{Unesi broj funti:}\ulaz{89.437}#
#\izlaz{89.437 lb = 40.57 kg}#
\end{upotreba}
\end{miditest}   
\linkresenje{p1.1_10b}
\end{Exercise}
\begin{Answer}[ref=p1.1_10b]
Zadatak se rešava analogno zadatku \ref{v1.1_03}.
\end{Answer}


\begin{Exercise}[label=p1.1_10c] 
Napisati program koji učitava temperaturu izraženu
   u farenhajtima, konvertuje tu vrednost u celzijuse i ispisuje je zaokruženu na dve decimale. \uputstvo{Veza između farenhajta i celzijusa je zadata narednom formulom $F=\frac{9\cdot C}{5}+32$}
   
\begin{miditest}
\begin{upotreba}{1}
#\naslovInt#
#\izlaz{Unesi temperaturu u F:}\ulaz{100.93}#
#\izlaz{100.93 F = 38.29 C}#
\end{upotreba}
\end{miditest}  
\begin{miditest}
\begin{upotreba}{2}
#\naslovInt#
#\izlaz{Unesi temperaturu u F:}\ulaz{25.562}#
#\izlaz{25.562 F = -3.58 C}#
\end{upotreba}
\end{miditest}
\linkresenje{p1.1_10c}
\end{Exercise}
\begin{Answer}[ref=p1.1_10c]
Zadatak se rešava analogno zadatku \ref{v1.1_03}.
\end{Answer}





\begin{Exercise}[label=p1.10_]
Napisati program koji za unete realne vrednosti $a_{11}$, $a_{12}$, $a_{21}$, $a_{22}$  ispisuje vrednost determinante matrice:
\[
 \begin{bmatrix}
  a_{11} & a_{12} \\
  a_{21} & a_{22} \\
 \end{bmatrix}
\]
Pri ispisu vrednost zaokružiti na $4$ decimale.

\begin{miditest}
\begin{upotreba}{1}
#\naslovInt#
#\izlaz{Unesite brojeve:}\ulaz{1 2 3 4}#
#\izlaz{-2.0000}#
\end{upotreba}
\end{miditest}
\begin{miditest}
\begin{upotreba}{2}
#\naslovInt#
#\izlaz{Unesite brojeve:}\ulaz{-1 0 0 1}#
#\izlaz{-1.0000}#
\end{upotreba}
\end{miditest}

\begin{miditest}
\begin{upotreba}{3}
#\naslovInt#
#\izlaz{Unesite brojeve:}\ulaz{1.5 -2 3 4.5}#
#\izlaz{12.7500}#
\end{upotreba}
\end{miditest}
\begin{miditest}
\begin{upotreba}{4}
#\naslovInt#
#\izlaz{Unesite brojeve:}\ulaz{0.01 0.01 0.5 7}#
#\izlaz{0.0650}#
\end{upotreba}
\end{miditest}
%\linkresenje{p1.10_}
\end{Exercise}
%\begin{Answer}[ref=p1.10_]
%\includecode{resenja/1_KontrolaToka/1.2_NaredbeGrananja/1_14.c}
%\end{Answer}

%\subsection{Geometrijski zadaci}

\begin{Exercise}[label=p1.1_02] 
Napisati program koji za unete realne vrednosti dužina stranica pravougaonika ispisuje njegov obim i površinu. Ispisati tražene vrednosti zaokružene na dve decimale.
\napomena{Pretpostaviti da je unos ispravan.} 

\begin{miditest}
\begin{upotreba}{1}
#\naslovInt#
#\izlaz{Unesite duzine stranica:}\ulaz{4.3 9.4}#
#\izlaz{Obim: 27.40}#
#\izlaz{Povrsina: 40.42}#
\end{upotreba}
\end{miditest}
\begin{miditest}
\begin{upotreba}{2}
#\naslovInt#
#\izlaz{Unesite duzine stranica:}\ulaz{10.756 36.2}#
#\izlaz{Obim: 93.91}#
#\izlaz{Povrsina: 389.37}#
\end{upotreba}
\end{miditest}

\linkresenje{p1.1_02}
\end{Exercise}
\begin{Answer}[ref=p1.1_02]
\includecode{resenja/1_KontrolaToka/1.1_UvodniZadaci/1_21.c}
\end{Answer}

\begin{Exercise}[label=v1.1_04] 
Napisati program koji za unetu realnu vrednost dužine poluprečnika kruga ispisuje njegov obim i površinu zaokružene na dve decimale. \napomena{Pretpostaviti da je unos ispravan.}
      
\begin{miditest}
\begin{upotreba}{1}
#\naslovInt#
#\izlaz{Unesite duzinu poluprecnika kruga:}\ulaz{4.2}#
#\izlaz{Obim: 26.39, povrsina: 55.42}#
\end{upotreba}
\end{miditest}
\begin{miditest}
\begin{upotreba}{2}
#\naslovInt#
#\izlaz{Unesite duzinu poluprecnika kruga:}\ulaz{14.932}#
#\izlaz{Obim: 93.82, povrsina: 700.46}#
\end{upotreba}
\end{miditest}   
   
\linkresenje{v1.1_04}
\end{Exercise}
\begin{Answer}[ref=v1.1_04]
\includecode{resenja/1_KontrolaToka/1.1_UvodniZadaci/1_22.c}
\end{Answer}


\begin{Exercise}[label=p1.1_03a] 
Napisati program koji za unetu realnu vrednost dužine stranice jednakostraničnog trougla ispisuje njegov obim i površinu zaokružene na dve decimale. \napomena{Pretpostaviti da je unos ispravan.}

\begin{miditest}
\begin{upotreba}{1}
#\naslovInt#
#\izlaz{Unesite duzinu stranice trougla:}\ulaz{5}#
#\izlaz{Obim: 15.00}#
#\izlaz{Povrsina: 10.82}#
\end{upotreba}
\end{miditest}
\begin{miditest}
\begin{upotreba}{2}
#\naslovInt#
#\izlaz{Unesite duzinu stranice trougla:}\ulaz{2}#
#\izlaz{Obim:  6.00}#
#\izlaz{Povrsina: 1.73}#
\end{upotreba}
\end{miditest}
\linkresenje{p1.1_03a}
\end{Exercise}
\begin{Answer}[ref=p1.1_03a]
\includecode{resenja/1_KontrolaToka/1.1_UvodniZadaci/1_23.c}
\end{Answer}

\begin{Exercise}[label=p1.1_03] 
Napisati program koji za unete realne vrednosti dužina stranica trougla ispisuje njegov obim i površinu zaokružene na dve decimale. \napomena{Pretpostaviti da je unos ispravan.}

\begin{miditest}
\begin{upotreba}{1}
#\naslovInt#
#\izlaz{Unesite duzine stranica trougla:}\ulaz{3 4 5}#
#\izlaz{Obim: 12.00}#
#\izlaz{Povrsina: 6.00}#
\end{upotreba}
\end{miditest}
\begin{miditest}
\begin{upotreba}{2}
#\naslovInt#
#\izlaz{Unesite duzine stranica trougla:}\ulaz{4.3 9.7 8.8}#
#\izlaz{Obim:  22.80}#
#\izlaz{Povrsina: 18.91}#
\end{upotreba}
\end{miditest}
\linkresenje{p1.1_03}
\end{Exercise}
\begin{Answer}[ref=p1.1_03]
\includecode{resenja/1_KontrolaToka/1.1_UvodniZadaci/1_24.c}
\end{Answer}


\begin{Exercise}[label=p1_13] 
Pravougaonik čije su stranice paralelne koordinatnim osama zadat je svojim realnim koordinatama suprotnih temena (gornje levo i donje desno teme). Napisati program koji ispisuje njegov obim i površinu zaokružene na dve decimale. 

\begin{maxitest}
\begin{upotreba}{1}
#\naslovInt#
#\izlaz{Unesite koordinate gornjeg levog temena:}\ulaz{4.3 5.8}#
#\izlaz{Unesite koordinate donjeg desnog temena:}\ulaz{6.7 2.3}#
#\izlaz{Obim: 5.90}#
#\izlaz{Povrsina: 8.40}#
\end{upotreba}
\end{maxitest}

\begin{maxitest}
\begin{upotreba}{2}
#\naslovInt#
#\izlaz{Unesite koordinate gornjeg levog temena:}\ulaz{-3.7 8.23}#
#\izlaz{Unesite koordinate donjeg desnog temena:}\ulaz{-0.56 2}#
#\izlaz{Obim: 9.37}#
#\izlaz{Povrsina: 19.56}#
\end{upotreba}
\end{maxitest}
%\linkresenje{p1_13}
\end{Exercise}
%\begin{Answer}[ref=p1_13]
%\includecode{resenja/1_UvodniZadaci/1_13.c}
%\end{Answer}

%\section{Mesano celi i realni (kastovanje)}

\begin{Exercise}[label=v1.1_09] 
Napisati program koji za tri uneta cela broja ispisuje njihovu artimetičku sredinu zaokruženu na dve decimale.

\begin{miditest}
\begin{upotreba}{1}
#\naslovInt#
#\izlaz{Unesite tri cela broja:}\ulaz{11 5 4}#
#\izlaz{Aritmeticka sredina unetih brojeva je 6.67}#
\end{upotreba}
\end{miditest}
\begin{miditest}
\begin{upotreba}{2}
#\naslovInt#
#\izlaz{Unesite tri cela broja:}\ulaz{3 -8 13}#
#\izlaz{Aritmeticka sredina unetih brojeva je 2.67}#
\end{upotreba}
\end{miditest}

\linkresenje{v1.1_09}
\end{Exercise}
\begin{Answer}[ref=v1.1_09]
\includecode{resenja/1_KontrolaToka/1.1_UvodniZadaci/1_26.c}
\end{Answer}







\begin{Exercise}[label=p1.1_04] 
Napisati program koji pomaže moleru da izračuna površinu zidova prostorije koju treba da okreči. Za unete dimenzije sobe u metrima (dužinu, širinu i visinu), program treba da  ispiše površinu zidova za krečenje pod pretpostavkom da na vrata i prozore otpada oko 20\%. Omogućiti i da na osnovu unete cene usluge po kvadratnom metru program izračuna ukupnu cenu krečenja. Sve realne vrednosti ispisati zaokružene na dve decimale.

\begin{miditest}
\begin{upotreba}{1}
#\naslovInt#
#\izlaz{Unesite dimenzije sobe:}\ulaz{4 4 3}#
#\izlaz{Unesite cenu po m2:}\ulaz{500}#
#\izlaz{Moler treba da okreci 51.20 m2}#
#\izlaz{Cena krecenja je 25600.00}#
\end{upotreba}
\end{miditest}
\begin{miditest}
\begin{upotreba}{2}
#\naslovInt#
#\izlaz{Unesite dimenzije sobe:}\ulaz{13 17 3}#
#\izlaz{Unesite cenu po m2:}\ulaz{475}#
#\izlaz{Moler treba da okreci 320.80 m2}#
#\izlaz{Cena krecenja je 152380.00}#
\end{upotreba}
\end{miditest}
\linkresenje{p1.1_04}
\end{Exercise}
\begin{Answer}[ref=p1.1_04]
\includecode{resenja/1_KontrolaToka/1.1_UvodniZadaci/1_27.c}
\end{Answer}




\begin{Exercise}[label=p1.1_09] 
Napisati program koji za unete pozitivne prirodne brojeve $x$, $c$ i $p$ ispisuje broj koji se dobija ubacivanjem cifre $c$  u broj $x$ 
na poziciju $p$. \napomena{Podrazumevati da je unos ispravan, tj.~da je broj $p$ manji od ukupnog broja cifara broja $x$. Numeracija cifara počinje od nule, odnosno cifra najmanje težine nalazi se na nultoj poziciji.} \uputstvo{Koristiti funkciju \kckod{pow} iz \kckod{math.h} biblioteke.}

\begin{miditest}
\begin{upotreba}{1}
#\naslovInt#
#\izlaz{Unesite redom x, c i p:}\ulaz{140 2 1}#
#\izlaz{Rezultat je: 1420}#
\end{upotreba}
\end{miditest}
\begin{miditest}
\begin{upotreba}{1}
#\naslovInt#
#\izlaz{Unesite redom x, c i p:}\ulaz{12345 9 2}#
#\izlaz{Rezultat je: 129345}#
\end{upotreba}
\end{miditest}

\linkresenje{p1.1_09}
\end{Exercise}
\begin{Answer}[ref=p1.1_09]
\includecode{resenja/1_KontrolaToka/1.1_UvodniZadaci/1_28.c}
\end{Answer}

\begin{Exercise}[label=p1_18]
Napisati program koji za uneta dva cela broja $a$ i $b$ dodeljuje promenljivoj $rezultat$ vrednost 1
ako važi uslov:
\begin{description}
\item{a)} $a$ i $b$ su različiti brojevi
\item{b)} $a$ i $b$ su parni brojevi
\item{c)} $a$ i $b$ su pozitivni brojevi, ne veći od 100
\end{description} 
U suprotnom, promenljivoj $rezultat$ dodeliti vrednost 0. Ispisati vrednost promenljive $rezultat$. 

\begin{miditest}
\begin{upotreba}{1}
#\naslovInt#
#\izlaz{Unesite dva cela broja:}\ulaz{4 8}#
#\izlaz{a) rezultat=1}#
#\izlaz{b) rezultat=1}#
#\izlaz{c) rezultat=1}#
\end{upotreba}
\end{miditest}
\begin{miditest}
\begin{upotreba}{2}
#\naslovInt#
#\izlaz{Unesite dva cela broja:}\ulaz{3 -11}#
#\izlaz{a) rezultat=1}#
#\izlaz{b) rezultat=0}#
#\izlaz{c) rezultat=0}#
\end{upotreba}
\end{miditest}

\linkresenje{p1_18}
\end{Exercise}
\begin{Answer}[ref=p1_18]
\includecode{resenja/1_KontrolaToka/1.1_UvodniZadaci/1_30.c}
\end{Answer}

\begin{Exercise}[label=p1_17]
Napisati program koji za uneta dva cela broja ispisuje njihov maksimum. 

\begin{miditest}
\begin{upotreba}{1}
#\naslovInt#
#\izlaz{Unesite dva cela broja:}\ulaz{19 256}#
#\izlaz{Maksimum je 256}#
\end{upotreba}
\end{miditest}
\begin{miditest}
\begin{upotreba}{2}
#\naslovInt#
#\izlaz{Unesite dva cela broja:}\ulaz{-39 57}#
#\izlaz{Maksimum je 57}#
\end{upotreba}
\end{miditest}

\linkresenje{p1_17}
\end{Exercise}
\begin{Answer}[ref=p1_17]
\includecode{resenja/1_KontrolaToka/1.1_UvodniZadaci/1_29.c}
\end{Answer}

\begin{Exercise}[label=p1_21]
Napisati program koji za uneta dva cela broja ispisuje njihov minimum. 

\begin{miditest}
\begin{upotreba}{1}
#\naslovInt#
#\izlaz{Unesite dva cela broja:}\ulaz{4 8}#
#\izlaz{Minimum je 4}#
\end{upotreba}
\end{miditest}
\begin{miditest}
\begin{upotreba}{2}
#\naslovInt#
#\izlaz{Unesite dva cela broja:}\ulaz{-3 -110}#
#\izlaz{Minimum je -110}#
\end{upotreba}
\end{miditest}

\linkresenje{p1_21}
\end{Exercise}
\begin{Answer}[ref=p1_21]
Zadatak se rešava analogno zadatku \ref{p1_21}
\end{Answer}

\begin{Exercise}[label=p1_22]
Napisati program koji za unete realne vrednosti promenljivih $x$ i
$y$ ispisuje vrednost sledećeg izraza:
$$rez = \frac{\min(x, y) + 0.5}{1 + \max^2(x, y)}$$ zaokruženu na dve decimale. 

\begin{miditest}
\begin{upotreba}{1}
#\naslovInt#
#\izlaz{Unesite dva realna broja:}\ulaz{5.7 11.2}#
#\izlaz{Rezultat je: 0.05}#
\end{upotreba}
\end{miditest}
\begin{miditest}
\begin{upotreba}{1}
#\naslovInt#
#\izlaz{Unesite dva realna broja:}\ulaz{-9.34 8.99}#
#\izlaz{Rezultat je: -0.11}#
\end{upotreba}
\end{miditest}

\linkresenje{p1_22}
\end{Exercise}
\begin{Answer}[ref=p1_22]
\includecode{resenja/1_KontrolaToka/1.1_UvodniZadaci/1_31.c}
\end{Answer}


\ifresenja
\section{Rešenja}
\shipoutAnswer
\fi