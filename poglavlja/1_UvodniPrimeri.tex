\renewcommand{\chaptermark}[1]{\markboth{\thechapter\ #1}{#1}}
\renewcommand{\sectionmark}[1]{\markright{\thesection\ #1}}


\chapter{Uvodni zadaci}

\iffalse
\pagestyle{fancy}
\fancyhf{}
\fancyhead[RE]{\bfseries\slshape\leftmark}
\fancyhead[LO]{\bfseries\slshape\rightmark}
\fancyfoot[RO,LE]{\thepage}
\fi

\pagestyle{fancyplain}
\renewcommand{\chaptermark}[1]{\markboth{\thechapter\ #1}{#1}}
\renewcommand{\sectionmark}[1]{\markright{\thesection\ #1}}
%\lhead[\fancyplain{}{\bfseries\slshape\thepage}]{\fancyplain{}{\bfseries\slshape\rightmark}}
%\rhead[\fancyplain{}{\bfseries\slshape\leftmark}]{\fancyplain{}{\bfseries\slshape\thepage }}
\lhead[\fancyplain{}{\bfseries\slshape\leftmark}]{\fancyplain{}{}}
\rhead[\fancyplain{}{}]{{\fancyplain{}{\bfseries\slshape\rightmark}}}
%\rhead[]{\fancyplain{}{\bfseries\slshape\rightmark}}
%\lhead[]{\fancyplain{}{\bfseries\slshape\leftmark}}
\cfoot{}

%\fancyhead[LE,RO]{\fancyplain{}{\bfseries\slshape\rightmark}
%\fancyhead[RE,LO]{\fancyplain{}{\bfseries\slshape\leftmark}}
\fancyfoot[LE]{\thepage} 
\fancyfoot[RO]{\thepage} 
 

\pagenumbering{arabic}
\setcounter{page}{1}


\section{Naredba izraza}

\begin{Exercise}[label=UZ_NI_01] 
Napisati program koji na standardni izlaz ispisuje tekst \kckod{Zdravo svima!}.

\begin{miditest}
\begin{upotreba}{1}
#\naslovInt#
#\izlaz{Zdravo svima!}#
\end{upotreba}
\end{miditest}

\linkresenje{UZ_NI_01}
\end{Exercise}
\ifresenja
\begin{Answer}[ref=UZ_NI_01]
\includecode{resenja/1_UvodniZadaci/1.1_NaredbaIzraza/1_01.c}
\end{Answer}
\fi

%\section{Celi brojevi}


\begin{Exercise}[label=UZ_NI_02] 
Napisati program za uneti ceo broj ispisuje njegov kvadrat i njegov kub. 

\begin{miditest}
\begin{upotreba}{1}
#\naslovInt#
#\izlaz{Unesite ceo broj:}\ulaz{4}#
#\izlaz{Kvadrat: 16}#
#\izlaz{Kub: 64}#
\end{upotreba}
\end{miditest}
\begin{miditest}
\begin{upotreba}{2}
#\naslovInt#
#\izlaz{Unesite ceo broj:}\ulaz{-14}#
#\izlaz{Kvadrat: 196}#
#\izlaz{Kub: -2744}#
\end{upotreba}
\end{miditest}

\linkresenje{UZ_NI_02}
\end{Exercise}
\ifresenja
\begin{Answer}[ref=UZ_NI_02]
\includecode{resenja/1_UvodniZadaci/1.1_NaredbaIzraza/1_02.c}
\end{Answer}
\fi


\begin{Exercise}[label=UZ_NI_03] 
Napisati program koji za uneta dva cela broja $x$ i $y$ ispisuje njihov zbir, razliku, proizvod, 
ceo deo pri deljenju prvog broja drugim brojem i ostatak pri deljenju prvog broja drugim brojem. 
\napomena{Pretpostaviti da je unos ispravan.}

\begin{miditest}
\begin{upotreba}{1}
#\naslovInt#
#\izlaz{Unesite vrednost promenljive x:}\ulaz{7}#
#\izlaz{Unesite vrednost promenljive y:}\ulaz{2}#
#\izlaz{7 + 2 = 9}#
#\izlaz{7 - 2 = 5}#
#\izlaz{7 * 2 = 14}#
#\izlaz{7 / 2 = 3}#
#\izlaz{7 \% 2 = 1}#
\end{upotreba}
\end{miditest}
\begin{miditest}
\begin{upotreba}{2}
#\naslovInt#
#\izlaz{Unesite vrednost promenljive x:}\ulaz{-3}#
#\izlaz{Unesite vrednost promenljive y:}\ulaz{8}#
#\izlaz{-3 + 8 = 5}#
#\izlaz{-3 - 8 = -11}#
#\izlaz{-3 * 8 = -24}#
#\izlaz{-3 / 8 = 0}#
#\izlaz{-3 \% 8 = -3}#
\end{upotreba}
\end{miditest}

\linkresenje{UZ_NI_03}
\end{Exercise}
\ifresenja
\begin{Answer}[ref=UZ_NI_03]
\includecode{resenja/1_UvodniZadaci/1.1_NaredbaIzraza/1_03.c}
\end{Answer}
\fi

%\subsection{Prodavnica}


\begin{Exercise}[label=UZ_NI_04]
Napisati program koji pomaže kasirki da izračuna ukupan račun ako su poznate cene dva kupljena artikla. Cene artikala su pozitivni celi brojevi.
\napomena{Pretpostaviti da je unos ispravan.}

\begin{miditest}
\begin{upotreba}{1}
#\naslovInt#
#\izlaz{Unesite cenu prvog artikla:}\ulaz{173}#
#\izlaz{Unesite cenu drugog artikla:}\ulaz{2024}#
#\izlaz{Ukupna cena iznosi 2197}#
\end{upotreba}
\end{miditest}
\begin{miditest}
\begin{upotreba}{2}
#\naslovInt#
#\izlaz{Unesite cenu prvog artikla:}\ulaz{384}#
#\izlaz{Unesite cenu drugog artikla:}\ulaz{555}#
#\izlaz{Ukupna cena iznosi 939}#
\end{upotreba}
\end{miditest}

\linkresenje{UZ_NI_04}
\end{Exercise}
\ifresenja
\begin{Answer}[ref=UZ_NI_04]

Rešenje ovog zadatka svodi se na rešenje zadatka \ref{UZ_NI_03}, na deo koji se odnosi na 
izračunavanje zbira dva broja. Zbog pretpostavke da su cene artikala pozitivni celi brojevi, 
tip promenljivih za artikle treba da bude \kckod{unsigned int}.
\end{Answer}
\fi


\begin{Exercise}[label=UZ_NI_05]
Napisati program koji za unetu količinu jabuka u kilogramima i unetu
cenu po kilogramu ispisuje ukupnu vrednost date količine jabuka. 
Obe ulazne vrednosti su pozitivni celi brojevi.
\napomena{Pretpostaviti da je unos ispravan.} 

\begin{miditest}
\begin{upotreba}{1}
#\naslovInt#
#\izlaz{Unesite kolicinu jabuka (u kg):}\ulaz{6}#
#\izlaz{Unesite cenu (u dinarima):}\ulaz{82}#
#\izlaz{Molimo platite 492 dinara.}#
\end{upotreba}
\end{miditest}
\begin{miditest}
\begin{upotreba}{1}
#\naslovInt#
#\izlaz{Unesite kolicinu jabuka (u kg):}\ulaz{10}#
#\izlaz{Unesite cenu (u dinarima):}\ulaz{93}#
#\izlaz{Molimo platite 930 dinara.}#
\end{upotreba}
\end{miditest}
\linkresenje{UZ_NI_05}
\end{Exercise}
\ifresenja
\begin{Answer}[ref=UZ_NI_05]

Rešenje ovog zadatka svodi se na rešenje zadatka \ref{UZ_NI_03}, na deo koji se odnosi na izračunavanje proizvoda dva broja. 
Zbog pretpostavke da su cene artikala pozitivni celi brojevi, tip promenljivih za artikle treba da bude \kckod{unsigned int}.
\end{Answer}
\fi


\begin{Exercise}[label=UZ_NI_06] 
Napisati program koji pomaže kasirki da obračuna kusur koji treba da vrati kupcu. 
Za unetu cenu artikla, količinu artikla i iznos koji je kupac dao, program treba da 
ispiše vrednost kusura. Sve ulazne vrednosti su pozitivni celi brojevi. 
\napomena{Pretpostaviti da je unos ispravan.}

\begin{miditest}
\begin{upotreba}{1}
#\naslovInt#
#\izlaz{Unesite cenu, kolicinu i iznos:}#
#\ulaz{132 2 500}#
#\izlaz{Kusur je 236 dinara.}#
\end{upotreba}
\end{miditest}
\begin{miditest}
\begin{upotreba}{2}
#\naslovInt#
#\izlaz{Unesite cenu, kolicinu i iznos:}#
#\ulaz{59 6 2000}#
#\izlaz{Kusur je 1646 dinara.}#
\end{upotreba}
\end{miditest}

\linkresenje{UZ_NI_06}
\end{Exercise}
\ifresenja
\begin{Answer}[ref=UZ_NI_06]
\includecode{resenja/1_UvodniZadaci/1.1_NaredbaIzraza/1_06.c}
\end{Answer}
\fi


\begin{Exercise}[label=UZ_NI_07] 
Napisati program koji za uneta vremena poletanja i sletanja aviona ispisuje dužinu trajanja leta. 
\napomena{Pretpostaviti da su poletanje i sletanje u istom danu kao i da su sve vrednosti ispravno unete.}

\begin{miditest}
\begin{upotreba}{1}
#\naslovInt#
#\izlaz{Unesite vreme poletanja:}\ulaz{8 5}#
#\izlaz{Unesite vreme sletanja:}\ulaz{12 41}#
#\izlaz{Duzina trajanja leta je 4 h i 36 min}#
\end{upotreba}
\end{miditest}
\begin{miditest}
\begin{upotreba}{2}
#\naslovInt#
#\izlaz{Unesite vreme poletanja:}\ulaz{13 20}#
#\izlaz{Unesite vreme sletanja:}\ulaz{18 45}#
#\izlaz{Duzina trajanja leta je 5 h i 25 min}#
\end{upotreba}
\end{miditest}

\linkresenje{UZ_NI_07}
\end{Exercise}
\ifresenja
\begin{Answer}[ref=UZ_NI_07]
\includecode{resenja/1_UvodniZadaci/1.1_NaredbaIzraza/1_07.c}
\end{Answer}
\fi

%\subsection{Naredba dodele}


\begin{Exercise}[label=UZ_NI_08] 
Date su dve celobrojne promenljive $x$ i $y$. Napisati program koji razmenjuje njihove vrednosti.

\begin{miditest}
\begin{upotreba}{1}
#\naslovInt#
#\izlaz{Unesite vrednosti x i y:}\ulaz{5 7}#
#\izlaz{Pre zamene: x=5, y=7}#
#\izlaz{Posle zamene: x=7, y=5}#
\end{upotreba}
\end{miditest}
\begin{miditest}
\begin{upotreba}{2}
#\naslovInt#
#\izlaz{Unesite vrednosti x i y:}\ulaz{237 -592}#
#\izlaz{Pre zamene: x=237, y=-592}#
#\izlaz{Posle zamene: x=-592, y=237}#
\end{upotreba}
\end{miditest}

\linkresenje{UZ_NI_08}
\end{Exercise}
\ifresenja
\begin{Answer}[ref=UZ_NI_08]
\includecode{resenja/1_UvodniZadaci/1.1_NaredbaIzraza/1_08.c}
\end{Answer}
\fi


\begin{Exercise}[label=UZ_NI_09]
Date su dve celobrojene promenljive $a$ i $b$. Napisati program koji promenljivoj $a$ dodeljuje
njihovu sumu, a promenljivoj $b$ njihovu razliku. \napomena{Ne koristiti pomoćne
promenljive}. 

\begin{miditest}
\begin{upotreba}{1}
#\naslovInt#
#\izlaz{Unesite vrednosti a i b:}\ulaz{5 7}#
#\izlaz{Nove vrednosti su: a=12, b=-2}#
\end{upotreba}
\end{miditest}
\begin{miditest}
\begin{upotreba}{2}
#\naslovInt#
#\izlaz{Unesite vrednosti a i b:}\ulaz{237 -592}#
#\izlaz{Nove vrednosti su: a=-355, b=829}#
\end{upotreba}
\end{miditest}

\linkresenje{UZ_NI_09}
\end{Exercise}
\ifresenja
\begin{Answer}[ref=UZ_NI_09]
\includecode{resenja/1_UvodniZadaci/1.1_NaredbaIzraza/1_09.c}
\end{Answer}
\fi

%\subsection{Cifre}


\begin{Exercise}[label=UZ_NI_10] 
Napisati program koji za uneti pozitivan trocifreni broj ispisuje njegove cifre jedinica, 
desetica i stotina. \napomena{Pretpostaviti da je unos ispravan.}

\begin{miditest}
\begin{upotreba}{1}
#\naslovInt#
#\izlaz{Unesite trocifreni broj:}\ulaz{697}#
#\izlaz{jedinica 7, desetica 9, stotina 6}#
\end{upotreba}
\end{miditest}
\begin{miditest}
\begin{upotreba}{2}
#\naslovInt#
#\izlaz{Unesite trocifreni broj:}\ulaz{504}#
#\izlaz{jedinica 4, desetica 0, stotina 5}#
\end{upotreba}
\end{miditest}
\linkresenje{UZ_NI_10}
\end{Exercise}
\ifresenja
\begin{Answer}[ref=UZ_NI_10]
\includecode{resenja/1_UvodniZadaci/1.1_NaredbaIzraza/1_10.c}
\end{Answer}
\fi


\begin{Exercise}[label=UZ_NI_11] 
Napisati program koji za unetu cenu proizvoda ispisuje najmanji broj novčanica koje je potrebno izdvojiti
prilikom plaćanja proizvoda. Na raspolaganju su novčanice od 5000, 2000, 1000, 500, 200, 100, 50, 20, 10 i 1 dinar. 
Cena proizvoda je pozitivan ceo broj. \napomena{Pretpostaviti da je unos ispravan.}

\begin{maxitest}
\begin{upotreba}{1}
#\naslovInt#
#\izlaz{Unesite cenu proizvoda:}\ulaz{8367}#
#\izlaz{8367 = 1*5000 + 1*2000 + 1*1000 + 0*500 + 1*200 + 1*100 + 1*50 + 0*20 + 1*10 + 7*1}#
\end{upotreba}
\begin{upotreba}{2}
#\naslovInt#
#\izlaz{Unesite cenu proizvoda:}\ulaz{934}#
#\izlaz{934 = 0*5000 + 0*2000 + 0*1000 + 1*500 + 2*200 + 0*100 + 0*50 + 1*20 + 1*10 + 4*1}#
\end{upotreba}
\end{maxitest}
\linkresenje{UZ_NI_11}
\end{Exercise}
\ifresenja
\begin{Answer}[ref=UZ_NI_11]
\includecode{resenja/1_UvodniZadaci/1.1_NaredbaIzraza/1_11.c}
\end{Answer}
\fi

\begin{Exercise}[label=UZ_NI_12] 
Napisati program koji učitava pozitivan trocifreni broj i ispisuje broj dobijen obrtanjem njegovih cifara. 
\napomena{Pretpostaviti da je unos ispravan.}

\begin{miditest}
\begin{upotreba}{1}
#\naslovInt#
#\izlaz{Unesite trocifreni broj:}\ulaz{892}#
#\izlaz{Obrnuto: 298}#
\end{upotreba}
\end{miditest}
\begin{miditest}
\begin{upotreba}{2}
#\naslovInt#
#\izlaz{Unesite trocifreni broj:}\ulaz{230}#
#\izlaz{Obrnuto: 32}#
\end{upotreba}
\end{miditest}

\linkresenje{UZ_NI_12}

\end{Exercise}
\ifresenja
\begin{Answer}[ref=UZ_NI_12]
\includecode{resenja/1_UvodniZadaci/1.1_NaredbaIzraza/1_12.c}
\end{Answer}
\fi


\begin{Exercise}[label=UZ_NI_13] 
Napisati program koji za uneti pozitivan četvorocifreni broj:
\begin{enumerate}
\item izračunava proizvod cifara
\item izračunava razliku sume krajnjih i srednjih cifara 
\item izračunava sumu kvadrata cifara
\item izračunava broj koji se dobija ispisom cifara u obrnutom poretku
\item izračunava broj koji se dobija zamenom cifre jedinice i cifre stotine
\end{enumerate}
\napomena{Pretpostaviti da je unos ispravan.}

\begin{maxitest}
\begin{upotreba}{1}
#\naslovInt#
#\izlaz{Unesite cetvorocifreni broj:}\ulaz{2371}#
#\izlaz{Proizvod cifara: 42}#
#\izlaz{Razlika sume krajnjih i srednjih: -7}#
#\izlaz{Suma kvadrata cifara: 63}#
#\izlaz{Broj u obrnutom poretku: 1732}#
#\izlaz{Broj sa zamenjenom cifrom jedinica i stotina: 2173}#
\end{upotreba}
\begin{upotreba}{2}
#\naslovInt#
#\izlaz{Unesite cetvorocifreni broj:}\ulaz{3570}#
#\izlaz{Proizvod cifara: 0}#
#\izlaz{Razlika sume krajnjih i srednjih: -9}#
#\izlaz{Suma kvadrata cifara: 83}#
#\izlaz{Broj u obrnutom poretku: 753}#
#\izlaz{Broj sa zamenjenom cifrom jedinica i stotina: 3075}#
\end{upotreba}
\end{maxitest}
\linkresenje{UZ_NI_13}
\end{Exercise}
\ifresenja
\begin{Answer}[ref=UZ_NI_13]
\includecode{resenja/1_UvodniZadaci/1.1_NaredbaIzraza/1_13.c}
\end{Answer}
\fi


\begin{Exercise}[label=UZ_NI_14] 
Napisati program koji ispisuje broj koji se dobija izbacivanjem cifre desetica u unetom pozitivnom celom broju. 
\napomena{Pretpostaviti da je unos ispravan.}

\begin{miditest}
\begin{upotreba}{1}
#\naslovInt#
#\izlaz{Unesite broj:}\ulaz{1349}#
#\izlaz{Rezultat je: 139}#
\end{upotreba}
\end{miditest}
\begin{miditest}
\begin{upotreba}{2}
#\naslovInt#
#\izlaz{Unesite broj:}\ulaz{825}#
#\izlaz{Rezultat je: 85}#
\end{upotreba}
\end{miditest}

\linkresenje{UZ_NI_14}
\end{Exercise}
\ifresenja
\begin{Answer}[ref=UZ_NI_14]
\includecode{resenja/1_UvodniZadaci/1.1_NaredbaIzraza/1_14.c}
\end{Answer}
\fi


\begin{Exercise}[label=UZ_NI_15]
Napisati program koji učitava pozitivan ceo broj $n$ i pozitivan dvocifreni broj $m$ i ispisuje broj 
dobijen umetanjem broja $m$ između cifre stotina i cifre hiljada broja $n$. 
\napomena{Za neke ulazne podatke može se dobiti neočekivan rezultat zbog prekoračenja, što ilustruje test primer broj 2.}

\begin{miditest}
\begin{upotreba}{1}
#\naslovInt#
#\izlaz{Unesite pozitivan ceo broj:}\ulaz{12345}#
#\izlaz{Unesite pozitivan dvocifreni broj:}\ulaz{67}#
#\izlaz{Novi broj je 1267345}#
\end{upotreba}
\end{miditest}  
\begin{miditest}
\begin{upotreba}{2}
#\naslovInt#
#\izlaz{Unesite pozitivan ceo broj:}\ulaz{50000000}#
#\izlaz{Unesite pozitivan dvocifreni broj:}\ulaz{12}#
#\izlaz{Novi broj je 705044704}#
\end{upotreba}
\end{miditest}   

\linkresenje{UZ_NI_15}
\end{Exercise}
\ifresenja
\begin{Answer}[ref=UZ_NI_15]
\includecode{resenja/1_UvodniZadaci/1.1_NaredbaIzraza/1_15.c}
\end{Answer}
\fi

%\section{Realni brojevi}


\begin{Exercise}[label=UZ_NI_16] 
Napisati program koji učitava realnu vrednost izraženu u inčima, konvertuje tu vrednost u centimetre 
i ispisuje je zaokruženu na dve decimale. \uputstvo{Jedan inč ima $2.54$ centimetra.}
   
\begin{miditest}
\begin{upotreba}{1}
#\naslovInt#
#\izlaz{Unesite broj inca:}\ulaz{4.69}#
#\izlaz{4.69 in = 11.91 cm}#
\end{upotreba}
\end{miditest}  
\begin{miditest}
\begin{upotreba}{2}
#\naslovInt#
#\izlaz{Unesite broj inca:}\ulaz{71.426}#
#\izlaz{71.43 in = 181.42 cm}#
\end{upotreba}
\end{miditest}   

\linkresenje{UZ_NI_16}
\end{Exercise}
\ifresenja
\begin{Answer}[ref=UZ_NI_16]
\includecode{resenja/1_UvodniZadaci/1.1_NaredbaIzraza/1_16.c}
\end{Answer}
\fi


\begin{Exercise}[label=UZ_NI_17] 
Napisati program koji učitava dužinu izraženu u miljama, konvertuje tu vrednost u kilometre i ispisuje 
je zaokruženu na dve decimale. \uputstvo{Jedna milja ima $1.609344$ kilometara.}
   
\begin{miditest}
\begin{upotreba}{1}
#\naslovInt#
#\izlaz{Unesite broj milja:}\ulaz{50.42}#
#\izlaz{50.42 mi = 81.14 km}#
\end{upotreba}
\end{miditest}  
\begin{miditest}
\begin{upotreba}{2}
#\naslovInt#
#\izlaz{Unesite broj milja:}\ulaz{327.128}#
#\izlaz{327.128 mi = 526.46 km}#
\end{upotreba}
\end{miditest}   

\linkresenje{UZ_NI_17}
\end{Exercise}
\ifresenja
\begin{Answer}[ref=UZ_NI_17]

Zadatak se rešava analogno zadatku \ref{UZ_NI_16}.
\end{Answer}
\fi


\begin{Exercise}[label=UZ_NI_18] 
Napisati program koji učitava težinu izraženu u funtama, konvertuje tu vrednost u kilograme i ispisuje 
je zaokruženu na dve decimale. \uputstvo{Jedna funta ima $0.45359237$ kilograma.}

\begin{miditest}
\begin{upotreba}{1}
#\naslovInt#
#\izlaz{Unesite broj funti:}\ulaz{2.78}#
#\izlaz{2.78 lb = 1.26 kg}#
\end{upotreba}
\end{miditest}  
\begin{miditest}
\begin{upotreba}{2}
#\naslovInt#
#\izlaz{Unesite broj funti:}\ulaz{89.437}#
#\izlaz{89.437 lb = 40.57 kg}#
\end{upotreba}
\end{miditest}   
\linkresenje{UZ_NI_18}
\end{Exercise}
\ifresenja
\begin{Answer}[ref=UZ_NI_18]

Zadatak se rešava analogno zadatku \ref{UZ_NI_16}.
\end{Answer}
\fi


\begin{Exercise}[label=UZ_NI_19] 
Napisati program koji učitava temperaturu izraženu u farenhajtima, konvertuje tu vrednost u celzijuse 
i ispisuje je zaokruženu na dve decimale. \napomena{Pretpostaviti da je unos ispravan.}
\uputstvo{Veza između farenhajta i celzijusa je zadata narednom formulom $F=\frac{9\cdot C}{5}+32$}
   
\begin{miditest}
\begin{upotreba}{1}
#\naslovInt#
#\izlaz{Unesite temperaturu u F:}\ulaz{100.93}#
#\izlaz{100.93 F = 38.29 C}#
\end{upotreba}
\end{miditest}  
\begin{miditest}
\begin{upotreba}{2}
#\naslovInt#
#\izlaz{Unesite temperaturu u F:}\ulaz{25.562}#
#\izlaz{25.562 F = -3.58 C}#
\end{upotreba}
\end{miditest}
\linkresenje{UZ_NI_19}
\end{Exercise}
\ifresenja
\begin{Answer}[ref=UZ_NI_19]

Zadatak se rešava analogno zadatku \ref{UZ_NI_16}.
\end{Answer}
\fi


\begin{Exercise}[label=UZ_NI_20]
Napisati program koji za unete realne vrednosti $a_{11}$, $a_{12}$, $a_{21}$, $a_{22}$  ispisuje vrednost determinante matrice:
\[
 \begin{bmatrix}
  a_{11} & a_{12} \\
  a_{21} & a_{22} \\
 \end{bmatrix}
\]
Pri ispisu vrednost zaokružiti na $4$ decimale.

\begin{miditest}
\begin{upotreba}{1}
#\naslovInt#
#\izlaz{Unesite brojeve:}\ulaz{1 2 3 4}#
#\izlaz{Determinanta: -2.0000}#
\end{upotreba}
\end{miditest}
\begin{miditest}
\begin{upotreba}{2}
#\naslovInt#
#\izlaz{Unesite brojeve:}\ulaz{-1 0 0 1}#
#\izlaz{Determinanta: -1.0000}#
\end{upotreba}
\end{miditest}

\begin{miditest}
\begin{upotreba}{3}
#\naslovInt#
#\izlaz{Unesite brojeve:}\ulaz{1.5 -2 3 4.5}#
#\izlaz{Determinanta: 12.7500}#
\end{upotreba}
\end{miditest}
\begin{miditest}
\begin{upotreba}{4}
#\naslovInt#
#\izlaz{Unesite brojeve:}\ulaz{0.01 0.01 0.5 7}#
#\izlaz{Determinanta: 0.0650}#
\end{upotreba}
\end{miditest}
\linkresenje{UZ_NI_20}
\end{Exercise}
\ifresenja
\begin{Answer}[ref=UZ_NI_20]
\includecode{resenja/1_UvodniZadaci/1.1_NaredbaIzraza/1_20.c}
\end{Answer}
\fi

%\subsection{Geometrijski zadaci}

\begin{Exercise}[label=UZ_NI_21] 
Napisati program koji za unete realne vrednosti dužina stranica pravougaonika ispisuje 
njegov obim i površinu. Ispisati tražene vrednosti zaokružene na dve decimale.
\napomena{Pretpostaviti da je unos ispravan.} 

\begin{miditest}
\begin{upotreba}{1}
#\naslovInt#
#\izlaz{Unesite duzine stranica:}\ulaz{4.3 9.4}#
#\izlaz{Obim: 27.40}#
#\izlaz{Povrsina: 40.42}#
\end{upotreba}
\end{miditest}
\begin{miditest}
\begin{upotreba}{2}
#\naslovInt#
#\izlaz{Unesite duzine stranica:}\ulaz{10.756 36.2}#
#\izlaz{Obim: 93.91}#
#\izlaz{Povrsina: 389.37}#
\end{upotreba}
\end{miditest}

\linkresenje{UZ_NI_21}
\end{Exercise}
\ifresenja
\begin{Answer}[ref=UZ_NI_21]
\includecode{resenja/1_UvodniZadaci/1.1_NaredbaIzraza/1_21.c}
\end{Answer}
\fi


\begin{Exercise}[label=UZ_NI_22] 
Napisati program koji za unetu realnu vrednost dužine poluprečnika kruga ispisuje njegov obim i površinu zaokružene na dve decimale. 
\napomena{Pretpostaviti da je unos ispravan.}
      
\begin{miditest}
\begin{upotreba}{1}
#\naslovInt#
#\izlaz{Unesite poluprecnik:}\ulaz{4.2}#
#\izlaz{Obim: 26.39}#
#\izlaz{Povrsina: 55.42}#
\end{upotreba}
\end{miditest}
\begin{miditest}
\begin{upotreba}{2}
#\naslovInt#
#\izlaz{Unesite poluprecnik:}\ulaz{14.932}#
#\izlaz{Obim: 93.82}#
#\izlaz{Povrsina: 700.46}#
\end{upotreba}
\end{miditest}   
   
\linkresenje{UZ_NI_22}
\end{Exercise}
\ifresenja
\begin{Answer}[ref=UZ_NI_22]
\includecode{resenja/1_UvodniZadaci/1.1_NaredbaIzraza/1_22.c}
\end{Answer}
\fi


\begin{Exercise}[label=UZ_NI_23] 
Napisati program koji za unetu realnu vrednost dužine stranice jednakostraničnog trougla 
ispisuje njegov obim i površinu zaokružene na dve decimale. 
\napomena{Pretpostaviti da je unos ispravan.}
\uputstvo{Za računanje korena broja koristiti funkciju \kckod{sqrt} čija se deklaracija nalazi u zaglavlju \kckod{math.h}.}

\begin{miditest}
\begin{upotreba}{1}
#\naslovInt#
#\izlaz{Unesite duzinu stranice trougla:}\ulaz{5}#
#\izlaz{Obim: 15.00}#
#\izlaz{Povrsina: 10.82}#
\end{upotreba}
\end{miditest}
\begin{miditest}
\begin{upotreba}{2}
#\naslovInt#
#\izlaz{Unesite duzinu stranice trougla:}\ulaz{2}#
#\izlaz{Obim:  6.00}#
#\izlaz{Povrsina: 1.73}#
\end{upotreba}
\end{miditest}
\linkresenje{UZ_NI_23}
\end{Exercise}
\ifresenja
\begin{Answer}[ref=UZ_NI_23]
\includecode{resenja/1_UvodniZadaci/1.1_NaredbaIzraza/1_23.c}
\end{Answer}
\fi


\begin{Exercise}[label=UZ_NI_24] 
Napisati program koji za unete realne vrednosti dužina stranica trougla ispisuje 
njegov obim i površinu zaokružene na dve decimale. 
\napomena{Pretpostaviti da je unos ispravan.}

\begin{miditest}
\begin{upotreba}{1}
#\naslovInt#
#\izlaz{Unesite duzine stranica trougla:}#
#\ulaz{3 4 5}#
#\izlaz{Obim: 12.00}#
#\izlaz{Povrsina: 6.00}#
\end{upotreba}
\end{miditest}
\begin{miditest}
\begin{upotreba}{2}
#\naslovInt#
#\izlaz{Unesite duzine stranica trougla:}#
#\ulaz{4.3 9.7 8.8}#
#\izlaz{Obim:  22.80}#
#\izlaz{Povrsina: 18.91}#
\end{upotreba}
\end{miditest}
\linkresenje{UZ_NI_24}
\end{Exercise}
\ifresenja
\begin{Answer}[ref=UZ_NI_24]
\includecode{resenja/1_UvodniZadaci/1.1_NaredbaIzraza/1_24.c}
\end{Answer}
\fi


\begin{Exercise}[label=UZ_NI_25] 
Pravougaonik čije su stranice paralelne koordinatnim osama zadat je svojim realnim 
koordinatama suprotnih temena (gornje levo i donje desno teme). Napisati program koji 
ispisuje njegov obim i površinu zaokružene na dve decimale. 
\napomena{Pretpostaviti da je unos ispravan.}

\begin{maxitest}
\begin{upotreba}{1}
#\naslovInt#
#\izlaz{Unesite koordinate gornjeg levog temena:}\ulaz{4.3 5.8}#
#\izlaz{Unesite koordinate donjeg desnog temena:}\ulaz{6.7 2.3}#
#\izlaz{Obim: 11.80}#
#\izlaz{Povrsina: 8.40}#
\end{upotreba}
\end{maxitest}

\begin{maxitest}
\begin{upotreba}{2}
#\naslovInt#
#\izlaz{Unesite koordinate gornjeg levog temena:}\ulaz{-3.7 8.23}#
#\izlaz{Unesite koordinate donjeg desnog temena:}\ulaz{-0.56 2}#
#\izlaz{Obim: 18.74}#
#\izlaz{Povrsina: 19.56}#
\end{upotreba}
\end{maxitest}
\linkresenje{UZ_NI_25}
\end{Exercise}
\ifresenja
\begin{Answer}[ref=UZ_NI_25]

Nakon ispravnog izračunavanja dužina stranica, zadatak se rešava analogno zadatku \ref{UZ_NI_21}.
\end{Answer}
\fi

%\section{Mesano celi i realni (kastovanje)}


\begin{Exercise}[label=UZ_NI_26] 
Napisati program koji za tri uneta cela broja ispisuje njihovu artimetičku sredinu zaokruženu na dve decimale.

\begin{miditest}
\begin{upotreba}{1}
#\naslovInt#
#\izlaz{Unesite tri cela broja:}\ulaz{11 5 4}#
#\izlaz{Aritmeticka sredina: 6.67}#
\end{upotreba}
\end{miditest}
\begin{miditest}
\begin{upotreba}{2}
#\naslovInt#
#\izlaz{Unesite tri cela broja:}\ulaz{3 -8 13}#
#\izlaz{Aritmeticka sredina: 2.67}#
\end{upotreba}
\end{miditest}

\linkresenje{UZ_NI_26}
\end{Exercise}
\ifresenja
\begin{Answer}[ref=UZ_NI_26]
\includecode{resenja/1_UvodniZadaci/1.1_NaredbaIzraza/1_26.c}
\end{Answer}
\fi


\begin{Exercise}[label=UZ_NI_27] 
Napisati program koji pomaže moleru da izračuna površinu zidova prostorije koju treba da okreči. 
Za unete celobrojne vrednosti dimenzije sobe u metrima (dužinu, širinu i visinu), program treba da ispiše površinu zidova 
za krečenje pod pretpostavkom da na vrata i prozore otpada oko 20\%. 
Omogućiti i da na osnovu unete celobrojene cene usluge po kvadratnom metru program izračuna ukupnu cenu krečenja. 
Sve realne vrednosti ispisati zaokružene na dve decimale. 
\napomena{Pretpostaviti da je unos ispravan.}

\begin{miditest}
\begin{upotreba}{1}
#\naslovInt#
#\izlaz{Unesite dimenzije sobe:}\ulaz{4 4 3}#
#\izlaz{Unesite cenu po m2:}\ulaz{500}#
#\izlaz{Moler treba da okreci 51.20 m2}#
#\izlaz{Cena krecenja je 25600.00}#
\end{upotreba}
\end{miditest}
\begin{miditest}
\begin{upotreba}{2}
#\naslovInt#
#\izlaz{Unesite dimenzije sobe:}\ulaz{13 17 3}#
#\izlaz{Unesite cenu po m2:}\ulaz{475}#
#\izlaz{Moler treba da okreci 320.80 m2}#
#\izlaz{Cena krecenja je 152380.00}#
\end{upotreba}
\end{miditest}
\linkresenje{UZ_NI_27}
\end{Exercise}
\ifresenja
\begin{Answer}[ref=UZ_NI_27]
\includecode{resenja/1_UvodniZadaci/1.1_NaredbaIzraza/1_27.c}
\end{Answer}
\fi


\begin{Exercise}[label=UZ_NI_28] 
Napisati program koji za unete pozitivne cele brojeve $x$, $p$ i $c$ ispisuje broj koji se dobija ubacivanjem cifre $c$  u broj $x$ na poziciju $p$. Pretpostaviti da numeracija cifara počinje od nule, odnosno da se cifra najmanje težine nalazi se na nultoj poziciji.
\napomena{Pretpostaviti da je unos ispravan.} 
\uputstvo{Koristiti funkciju \kckod{pow} čija se deklaracija nalazi u zaglavlju \kckod{math.h}.}

\begin{miditest}
\begin{upotreba}{1}
#\naslovInt#
#\izlaz{Unesite redom x, p i c:}\ulaz{140 1 2}#
#\izlaz{Rezultat je: 1420}#
\end{upotreba}
\end{miditest}
\begin{miditest}
\begin{upotreba}{1}
#\naslovInt#
#\izlaz{Unesite redom x, p i c:}\ulaz{12345 2 9}#
#\izlaz{Rezultat je: 123945}#
\end{upotreba}
\end{miditest}

\linkresenje{UZ_NI_28}
\end{Exercise}
\ifresenja
\begin{Answer}[ref=UZ_NI_28]
\includecode{resenja/1_UvodniZadaci/1.1_NaredbaIzraza/1_28.c}
\end{Answer}
\fi

\begin{Exercise}[label=UZ_NI_29]
Napisati program koji za uneta dva cela broja $a$ i $b$ dodeljuje promenljivoj $rezultat$ vrednost 1
ako važi uslov:
\begin{description}
\item{a)} $a$ i $b$ su različiti brojevi
\item{b)} $a$ i $b$ su parni brojevi
\item{c)} $a$ i $b$ su pozitivni brojevi, ne veći od 100
\end{description} 
U suprotnom, promenljivoj $rezultat$ dodeliti vrednost 0. Ispisati vrednost promenljive $rezultat$. 

\begin{miditest}
\begin{upotreba}{1}
#\naslovInt#
#\izlaz{Unesite dva cela broja:}\ulaz{4 8}#
#\izlaz{a) rezultat=1}#
#\izlaz{b) rezultat=1}#
#\izlaz{c) rezultat=1}#
\end{upotreba}
\end{miditest}
\begin{miditest}
\begin{upotreba}{2}
#\naslovInt#
#\izlaz{Unesite dva cela broja:}\ulaz{3 -11}#
#\izlaz{a) rezultat=1}#
#\izlaz{b) rezultat=0}#
#\izlaz{c) rezultat=0}#
\end{upotreba}
\end{miditest}

\linkresenje{UZ_NI_29}
\end{Exercise}
\ifresenja
\begin{Answer}[ref=UZ_NI_29]
\includecode{resenja/1_UvodniZadaci/1.1_NaredbaIzraza/1_29.c}
\end{Answer}
\fi


\begin{Exercise}[label=UZ_NI_30]
Napisati program koji za uneta dva cela broja ispisuje njihov maksimum. 

\begin{miditest}
\begin{upotreba}{1}
#\naslovInt#
#\izlaz{Unesite dva cela broja:}\ulaz{19 256}#
#\izlaz{Maksimum je 256}#
\end{upotreba}
\end{miditest}
\begin{miditest}
\begin{upotreba}{2}
#\naslovInt#
#\izlaz{Unesite dva cela broja:}\ulaz{-39 57}#
#\izlaz{Maksimum je 57}#
\end{upotreba}
\end{miditest}

\linkresenje{UZ_NI_30}
\end{Exercise}
\ifresenja
\begin{Answer}[ref=UZ_NI_30]
\includecode{resenja/1_UvodniZadaci/1.1_NaredbaIzraza/1_30.c}
\end{Answer}
\fi


\begin{Exercise}[label=UZ_NI_31]
Napisati program koji za uneta dva cela broja ispisuje njihov minimum. 

\begin{miditest}
\begin{upotreba}{1}
#\naslovInt#
#\izlaz{Unesite dva cela broja:}\ulaz{4 8}#
#\izlaz{Minimum je 4}#
\end{upotreba}
\end{miditest}
\begin{miditest}
\begin{upotreba}{2}
#\naslovInt#
#\izlaz{Unesite dva cela broja:}\ulaz{-3 -110}#
#\izlaz{Minimum je -110}#
\end{upotreba}
\end{miditest}

\linkresenje{UZ_NI_31}
\end{Exercise}
\ifresenja
\begin{Answer}[ref=UZ_NI_31]

Zadatak se rešava analogno zadatku \ref{UZ_NI_30}
\end{Answer}
\fi


\begin{Exercise}[label=UZ_NI_32]
Napisati program koji za unete realne vrednosti promenljivih $x$ i
$y$ ispisuje vrednost sledećeg izraza:
$$rez = \frac{\min(x, y) + 0.5}{1 + \max^2(x, y)}$$ zaokruženu na dve decimale. 

\begin{miditest}
\begin{upotreba}{1}
#\naslovInt#
#\izlaz{Unesite dva realna broja:}\ulaz{5.7 11.2}#
#\izlaz{Rezultat je: 0.05}#
\end{upotreba}
\end{miditest}
\begin{miditest}
\begin{upotreba}{1}
#\naslovInt#
#\izlaz{Unesite dva realna broja:}\ulaz{-9.34 8.99}#
#\izlaz{Rezultat je: -0.11}#
\end{upotreba}
\end{miditest}

\linkresenje{UZ_NI_32}
\end{Exercise}
\ifresenja
\begin{Answer}[ref=UZ_NI_32]
\includecode{resenja/1_UvodniZadaci/1.1_NaredbaIzraza/1_32.c}
\end{Answer}
\fi


\ifresenja
\section{Rešenja}
\shipoutAnswer
\fi
