\renewcommand{\chaptermark}[1]{\markboth{\thechapter\ #1}{#1}}
\renewcommand{\sectionmark}[1]{\markright{\thesection\ #1}}


\chapter{Uvodni zadaci}

\iffalse
\pagestyle{fancy}
\fancyhf{}
\fancyhead[RE]{\bfseries\slshape\leftmark}
\fancyhead[LO]{\bfseries\slshape\rightmark}
\fancyfoot[RO,LE]{\thepage}
\fi

\pagestyle{fancyplain}
\renewcommand{\chaptermark}[1]{\markboth{\thechapter\ #1}{#1}}
\renewcommand{\sectionmark}[1]{\markright{\thesection\ #1}}
%\lhead[\fancyplain{}{\bfseries\slshape\thepage}]{\fancyplain{}{\bfseries\slshape\rightmark}}
%\rhead[\fancyplain{}{\bfseries\slshape\leftmark}]{\fancyplain{}{\bfseries\slshape\thepage }}
\lhead[\fancyplain{}{\bfseries\slshape\leftmark}]{\fancyplain{}{}}
\rhead[\fancyplain{}{}]{{\fancyplain{}{\bfseries\slshape\rightmark}}}
%\rhead[]{\fancyplain{}{\bfseries\slshape\rightmark}}
%\lhead[]{\fancyplain{}{\bfseries\slshape\leftmark}}
\cfoot{}

%\fancyhead[LE,RO]{\fancyplain{}{\bfseries\slshape\rightmark}
%\fancyhead[RE,LO]{\fancyplain{}{\bfseries\slshape\leftmark}}
\fancyfoot[LE]{\thepage} 
\fancyfoot[RO]{\thepage} 
 

\pagenumbering{arabic}
\setcounter{page}{1}



\begin{Exercise}[label=v1.1_01] 
Tekst
\linkresenje{v1.1_01}
\end{Exercise}
\begin{Answer}[ref=v1.1_01]
\includecode{resenja/1_KontrolaToka/1.1_UvodniZadaci/1_01.c}
\end{Answer}

\begin{Exercise}[label=v1.1_02] 
Tekst
\linkresenje{v1.1_02}
\end{Exercise}
\begin{Answer}[ref=v1.1_02]
\includecode{resenja/1_KontrolaToka/1.1_UvodniZadaci/1_02.c}
\end{Answer}

\begin{Exercise}[label=v1.1_03] 
Tekst
\linkresenje{v1.1_03}
\end{Exercise}
\begin{Answer}[ref=v1.1_03]
\includecode{resenja/1_KontrolaToka/1.1_UvodniZadaci/1_03.c}
\end{Answer}

\begin{Exercise}[label=v1.1_04] 
Tekst
\linkresenje{v1.1_04}
\end{Exercise}
\begin{Answer}[ref=v1.1_04]
\includecode{resenja/1_KontrolaToka/1.1_UvodniZadaci/1_04.c}
\end{Answer}

\begin{Exercise}[label=v1.1_05] 
Tekst
\linkresenje{v1.1_05}
\end{Exercise}
\begin{Answer}[ref=v1.1_05]
\includecode{resenja/1_KontrolaToka/1.1_UvodniZadaci/1_05.c}
\end{Answer}

\begin{Exercise}[label=v1.1_06] 
Tekst
\linkresenje{v1.1_06}
\end{Exercise}
\begin{Answer}[ref=v1.1_06]
\includecode{resenja/1_KontrolaToka/1.1_UvodniZadaci/1_06.c}
\end{Answer}

\begin{Exercise}[label=v1.1_07] 
Tekst
\linkresenje{v1.1_07}
\end{Exercise}
\begin{Answer}[ref=v1.1_07]
\includecode{resenja/1_KontrolaToka/1.1_UvodniZadaci/1_07.c}
\end{Answer}

\begin{Exercise}[label=v1.1_08] 
Tekst
\linkresenje{v1.1_08}
\end{Exercise}
\begin{Answer}[ref=v1.1_08]
\includecode{resenja/1_KontrolaToka/1.1_UvodniZadaci/1_08.c}
\end{Answer}

\begin{Exercise}[label=v1.1_09] 
Tekst
\linkresenje{v1.1_09}
\end{Exercise}
\begin{Answer}[ref=v1.1_09]
\includecode{resenja/1_KontrolaToka/1.1_UvodniZadaci/1_09.c}
\end{Answer}

\begin{Exercise}[label=v1.1_10] 
Tekst
\linkresenje{v1.1_10}
\end{Exercise}
\begin{Answer}[ref=v1.1_10]
\includecode{resenja/1_KontrolaToka/1.1_UvodniZadaci/1_10.c}
\end{Answer}




\begin{Exercise}[label=p1_01] 
Napisati program koji omogućava korisniku da unese ceo broj, a zatim ispisuje njegov kvadrat i kub. \\
\begin{miditest}
\begin{upotreba}{1}
#\naslovInt#
#\izlaz{Unesite ceo broj:}\ulaz{4}#
#\izlaz{Kvadrat:16}#
#\izlaz{Kub: 64}#
\end{upotreba}
\end{miditest}
\linkresenje{p1_01}
\end{Exercise}
\begin{Answer}[ref=p1_01]
%\includecode{resenja/1_UvodniZadaci/1_01.c}
\end{Answer}

\begin{Exercise}[label=p1_02] 
Napisati program koji za unete stranice pravougaonika ispisuje njegov obim i površinu.\\
\begin{miditest}
\begin{upotreba}{1}
#\naslovInt#
#\izlaz{Unesite duzine stranica pravougaonika:}\ulaz{2 8}#
#\izlaz{Obim: 20}#
#\izlaz{Povrsina: 16}#
\end{upotreba}
\end{miditest}
\linkresenje{p1_02}
\end{Exercise}
\begin{Answer}[ref=p1_02]
%\includecode{resenja/1_UvodniZadaci/1_01.c}
\end{Answer}



\begin{Exercise}[label=p1_03] 
Napisati program koji za unete stranice trougla ispisuje njegov obim i površinu.\\
\begin{miditest}
\begin{upotreba}{1}
#\naslovInt#
#\izlaz{Unesite duzine stranica trougla:}\ulaz{3 4 5}#
#\izlaz{Obim: 12.00}#
#\izlaz{Povrsina: 6.00}#
\end{upotreba}
\end{miditest}
\linkresenje{p1_03}
\end{Exercise}
\begin{Answer}[ref=p1_03]
%\includecode{resenja/1_UvodniZadaci/1_01.c}
\end{Answer}


\begin{Exercise}[label=p1_04] 
Napisati program koji za unete dimenzije sobe u metrima (dužinu,
širinu i visinu) ispisuje koju površinu treba da okreči
moler. Uračunati da na vrata i prozore otpada oko 20\%. Omogućiti i
unos cene usluge po kvadratnom metru i izračunati zaradu koju
ostvaruje moler.\\
\begin{miditest}
\begin{upotreba}{1}
#\naslovInt#
#\izlaz{Unesite dimenzije sobe:}\ulaz{4 4 3}#
#\izlaz{Unesite cenu po kvadratnom metru:}\ulaz{500}#
#\izlaz{Moler treba da okreci 51.2 kvadratna metra}#
#\izlaz{Cena krecenja je 25600}#
\end{upotreba}
\end{miditest}
\linkresenje{p1_04}
\end{Exercise}
\begin{Answer}[ref=p1_04]
%\includecode{resenja/1_UvodniZadaci/1_01.c}
\end{Answer}



\begin{Exercise}[label=p1_05] 
Napisati program koji za unetu količinu jabuka u kilogramima i unetu
cenu po kilogramu ispisuje ukupan iznos koji treba platiti.\\
\begin{miditest}
\begin{upotreba}{1}
#\naslovInt#
#\izlaz{Unesite kolicinu jabuka (u kg):}\ulaz{6}#
#\izlaz{Unesite cenu (u dinarima):}\ulaz{82}#
#\izlaz{Molimo platite 492 dinara.}#
\end{upotreba}
\end{miditest}
\linkresenje{p1_05}
\end{Exercise}
\begin{Answer}[ref=p1_05]
%\includecode{resenja/1_UvodniZadaci/1_01.c}
\end{Answer}


\begin{Exercise}[label=p1_06] 
Napisati program koji pomaže kasirki da obračuna kusur tako što od nje traži da unese cenu artikla, količinu artikla i iznos koji je dobila od kupca. \\
\begin{miditest}
\begin{upotreba}{1}
#\naslovInt#
#\izlaz{Unesite redom cenu, kolicinu i iznos:}\ulaz{132 2 500}#
#\izlaz{Kusur je 236 dinara.}#
\end{upotreba}
\end{miditest}
\linkresenje{p1_06}
\end{Exercise}
\begin{Answer}[ref=p1_06]
%\includecode{resenja/1_UvodniZadaci/1_01.c}
\end{Answer}


\begin{Exercise}[label=p1_07] 
Napisati program koji prirodnom četvorocifrenom broju koji se unosi sa standardnog ulaza:
\begin{itemize}
\item izračunava proizvod cifara
\item izračunava razliku sume krajnjih i srednjih cifara 
\item izračunava sumu kvadrata cifara
\item određuje broj koji se dobija ispisom cifara u obrnutom poretku
\item određuje broj koji se dobija zamenom cifre jedinice i cifre stotine
\end{itemize}

\begin{maxitest}
\begin{upotreba}{1}
#\naslovInt#
#\izlaz{Unesite cetvorocifreni broj:}\ulaz{2371}#
#\izlaz{Proizvod cifara: 42}#
#\izlaz{Razlika sume krajnjih i srednjih: -7}#
#\izlaz{Suma kvadrata cifara: 63}#
#\izlaz{Broj u obrnutom poretku: 1732}#
#\izlaz{Broj sa zamenjenom cifrom jedinica i stotina: 2173}#
\end{upotreba}
\end{maxitest}
\linkresenje{p1_07}
\end{Exercise}
\begin{Answer}[ref=p1_07]
%\includecode{resenja/1_UvodniZadaci/1_01.c}
\end{Answer}


\begin{Exercise}[label=p1_08] 
Napisati program koji izbacuje cifru desetica datom prirodnom broju. \\
\begin{miditest}
\begin{upotreba}{1}
#\naslovInt#
#\izlaz{Unesite broj:}\ulaz{1349}#
#\izlaz{Rezultat je: 139}#
\end{upotreba}
\end{miditest}
\linkresenje{p1_08}
\end{Exercise}
\begin{Answer}[ref=p1_08]
%\includecode{resenja/1_UvodniZadaci/1_01.c}
\end{Answer}


\begin{Exercise}[label=p1_09] 
Napisati program koji u datom prirodnom broju x ubacuje cifru c
na poziciju p i rezultat ispisuje na standardni izlaz. Brojevi x, c i p se unose sa standardnog ulaza. Podrazumeva se da je broj p manji od ukupnog broja cifara broja i da numeracija cifara počinje od 1. Uputstvo: koristiti funkciju $pow$ iz $math.h$ biblioteke.\\
\begin{miditest}
\begin{upotreba}{1}
#\naslovInt#
#\izlaz{Unesite redom x, c i p:}\ulaz{140 2 2}#
#\izlaz{Rezultat je: 1420}#
\end{upotreba}
\end{miditest}
\linkresenje{p1_09}
\end{Exercise}
\begin{Answer}[ref=p1_09]
%\includecode{resenja/1_UvodniZadaci/1_01.c}
\end{Answer}


\begin{Exercise}[label=p1_10] 
Napisati program koji:
\begin{itemize}
\item unetu dužinu u miljama konvertuje u kilometre (1 mi = 1.609344 km)
\item unetu težinu u funtama konvertuje u kilograme ( 1 lb = 0.45359237 kg)
\item unetu temperaturu u celzijusima konvertuje u farenhajte ($F=\frac{9\cdot C}{5}+32$)
\end{itemize}

\begin{maxitest}
\begin{upotreba}{1}
#\naslovInt#
#\izlaz{Unesite duzinu u miljama:}\ulaz{1.8}#
#\izlaz{Vrednost duzine u kilometrima je: 2.896819}#
#\izlaz{Unesite tezinu u funtama:}\ulaz{10}#
#\izlaz{Vrednost tezine u kilogramima je: 4.535923}#
#\izlaz{Unesite temperaturu u celzjusima:}\ulaz{37.2}#
#\izlaz{Vrednost temperature u farenhajtima je: 98.960007}#
\end{upotreba}
\end{maxitest}
\linkresenje{p1_10}
\end{Exercise}
\begin{Answer}[ref=p1_10]
%\includecode{resenja/1_UvodniZadaci/1_01.c}
\end{Answer}

\begin{Exercise}[label=p1_11] 
Napisati program koji učitava sa standardnog ulaza vreme poletanja i vreme sletanja aviona, a potom ispisuje dužinu trajanja leta. Možemo pretpostaviti da su poletanje i sletanje u istom danu.\\
\begin{miditest}
\begin{upotreba}{1}
#\naslovInt#
#\izlaz{Unesite vreme poletanja:}\ulaz{8 5 0}#
#\izlaz{Unesite vreme sletanja:}\ulaz{12 41 30}#
#\izlaz{Duzina trajanja leta: 4 h 36 min 30 sec}#
\end{upotreba}
\end{miditest}
\linkresenje{p1_11}
\end{Exercise}
\begin{Answer}[ref=p1_11]
%\includecode{resenja/1_UvodniZadaci/1_01.c}
\end{Answer}

\begin{Exercise}[label=p1_12]
Sa standarnog ulaza se učitavaju dve realne promenljive. Razmeniti vrednosti
promenljivima i nove vrednosti ispisati na standarni izlaz. \\
\linkresenje{p1_12}
\end{Exercise}
\begin{Answer}[ref=p1_12]
%\includecode{resenja/1_UvodniZadaci/1_01.c}
\end{Answer}


\begin{Exercise}[label=p1_13] 
Unose se koordinate suprotnih temena pravougaonika (gornje levo i
donje desno teme).  Pretpostaviti da su stranice pravougaonika
paralelene koordinatnim osama.  Odrediti obim i površinu
pravougaonika.\\
\linkresenje{p1_13}
\end{Exercise}
\begin{Answer}[ref=p1_13]
%\includecode{resenja/1_UvodniZadaci/1_13.c}
\end{Answer}


\begin{Exercise}[label=p1_14]
Date su dve celobrojene promenljive a i b. Promenljivoj a dodeliti
njihovu sumu, a promenljivoj b njihovu razliku bez korišćenja pomoćne
promenljive. \\
\linkresenje{p1_14}
\end{Exercise}
\begin{Answer}[ref=p1_14]
%\includecode{resenja/1_UvodniZadaci/1_01.c}
\end{Answer}

\begin{Exercise}[label=p1_15]
Napisati program koji na standarni izlaz ispisuje sledeći tekst: \\
\begin{miditest}
\begin{upotreba}{1}
#\naslovInt#
#\izlaz{Karakteri : \" \% \{ * + = a}#
#\izlaz{Brojevi: 43, -56, 455}#
\end{upotreba}
\end{miditest}
\linkresenje{p1_15}
\end{Exercise}
\begin{Answer}[ref=p1_15]
%\includecode{resenja/1_UvodniZadaci/1_01.c}
\end{Answer}

\begin{Exercise}[label=p1_16]
Napisati program koji na mesto stotina i hiljada umeće cifre c1 i c2. 
Da li se može desiti da za neke ulazne podatke dodje do prekoračenja?
Obrazložiti. \\
\linkresenje{p1_16}
\end{Exercise}
\begin{Answer}[ref=p1_16]
%\includecode{resenja/1_UvodniZadaci/1_01.c}
\end{Answer}

\begin{Exercise}[label=p1_17]
Sa standarnog ulaza se unose dva cela broja. Na stadarni izlaz ispisati 
maksimum ova dva broja. \\
\linkresenje{p1_17}
\end{Exercise}
\begin{Answer}[ref=p1_17]
%\includecode{resenja/1_UvodniZadaci/1_01.c}
\end{Answer}

\begin{Exercise}[label=p1_18]
Data su 3 cela broja a, b, c. Dodeliti promenljivoj rez vrednost 1
ako:
\begin{description}
\item{a)} a, b, c su različiti brojevi
\item{b)} a, b, c su parni brojevi
\item{c)} a, b, c su pozitivni brojevi, ne veći od 100
\end{description}
U suprotnom promenljivoj dodeliti vrednost 0. Proveriti ispisom na
standarni izlaz. \\
\linkresenje{p1_18}
\end{Exercise}
\begin{Answer}[ref=p1_18]
%\includecode{resenja/1_UvodniZadaci/1_01.c}
\end{Answer}

\begin{Exercise}[label=p1_19]
Program treba da proveri da li se ta\v cke $A(x_1, y_1)$ i $B(x_2,
y_2)$ nalaze u istom kvadrantu. Na standarni izlaz ispisati odgovor
\verb|DA| ili \verb|NE|. \\
\linkresenje{p1_19}
\end{Exercise}
\begin{Answer}[ref=p1_19]
%\includecode{resenja/1_UvodniZadaci/1_01.c}
\end{Answer}

\begin{Exercise}[label=p1_20]
Program treba da proveri da li se ta\v cke $A(x_1, y_1)$, $B(x_2,
y_2)$ i $C(x_3, y_3)$ nalaze na istoj pravi. Na standarni izlaz
ispisati odgovor \verb|DA| ili \verb|NE|. \\
\linkresenje{p1_20}
\end{Exercise}
\begin{Answer}[ref=p1_20]
%\includecode{resenja/1_UvodniZadaci/1_01.c}
\end{Answer}

\begin{Exercise}[label=p1_21]
Polje \v sahovske table se defini\v se parom prirodnih brojeva ne ve\'
cih od 8: prvi se odnosi na red, drugi na kolonu. Ako su dati takvi
parovi, napisati program koji proverava: \\
\begin{description}
\item[a)] da li su polja (k, m) i (l, n) iste boje
\item[b)] da li kraljica sa (k, l) ugrozava polje (m, n)
\item[c)] da li konj sa (k, l) ugrozava polje (m, n)
\end{description}
\linkresenje{p1_21}
\end{Exercise}
\begin{Answer}[ref=p1_21]
%\includecode{resenja/1_UvodniZadaci/1_01.c}
\end{Answer}


\begin{Exercise}[label=p1_22]
Sa standarnog ulaza unose se dve promeljive \verb|x| i
\verb|y|. Izra\v cunati vrednost izraza:
 $$rez = \frac{\min(x, y) + 0.5}{1 + \max^2(x, y)}$$
Rezultat ispisati na standardni izlaz. \\
\linkresenje{p1_22}
\end{Exercise}
\begin{Answer}[ref=p1_22]
%\includecode{resenja/1_UvodniZadaci/1_01.c}
\end{Answer}

\begin{Exercise}[label=p1_23]
Tekst \\
\linkresenje{p1_23}
\end{Exercise}
\begin{Answer}[ref=p1_23]
\includecode{resenja/1_KontrolaToka/1.1_UvodniZadaci/1_11.c}
\end{Answer}

\begin{Exercise}[label=p1.10_]
Napisati program koji za unete brojeve a11, a12, a21, a22 tipa
\verb|float| izra\v cunava i ispisuje na standardni izlaz determinantu
matrice:
\begin{verbatim}
  a11 a12 
  a21 a22
\end{verbatim}
Pri ispisu vrednosti se zaokru\v{z}uju na 4 decimale. \\
\begin{miditest}
\begin{upotreba}{1}
#\naslovInt#
#\izlaz{Unesite brojeve:}\ulaz{1 2 3 4}#
#\izlaz{-2.0000}#
\end{upotreba}
\end{miditest}
\begin{miditest}
\begin{upotreba}{2}
#\naslovInt#
#\izlaz{Unesite brojeve:}\ulaz{-1 0 0 1}#
#\izlaz{-1.0000}#
\end{upotreba}
\end{miditest}
\begin{miditest}
\begin{upotreba}{3}
#\naslovInt#
#\izlaz{Unesite brojeve:}\ulaz{1.5 -2 3 4.5}#
#\izlaz{12.7500}#
\end{upotreba}
\end{miditest}
\begin{miditest}
\begin{upotreba}{4}
#\naslovInt#
#\izlaz{Unesite brojeve:}\ulaz{0.01 0.01 0.5 7}#
#\izlaz{0.0650}#
\end{upotreba}
\end{miditest}
\linkresenje{p1.10_}
\end{Exercise}
\begin{Answer}[ref=p1.10_]
%\includecode{resenja/1_KontrolaToka/1.2_NaredbeGrananja/1_14.c}
\end{Answer}

\section{Rešenja}
\shipoutAnswer
