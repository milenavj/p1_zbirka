\chapter{Ulaz i izlaz programa}

\section{Argumenti komandne linije}
 
\begin{Exercise}[label=v2.2_04] 
    Napisati program koji ispisuje broj navedenih argumenata komandne linije,
    a zatim i same argumenate i njihove redne brojeve.
    
\begin{miditest}
\begin{upotreba}{1}
#\poziv{./a.out abcde 123 -5 3.7}#

#\naslovIzlaz#
#\izlaz{Broj argumenata je 5:}#
#\izlaz{0: ./a.out}#
#\izlaz{1: abcde}#
#\izlaz{2: 123}#
#\izlaz{3: -5}#
#\izlaz{4: 3.7}#
\end{upotreba}
\end{miditest}
\begin{miditest}
\begin{upotreba}{2}
#\poziv{./a.out}#

#\naslovIzlaz#
#\izlaz{Broj argumenata je 1:}#
#\izlaz{0: ./a.out}#
\end{upotreba}
\end{miditest}

\linkresenje{v2.2_04}
\end{Exercise}
\ifresenja
\begin{Answer}[ref=v2.2_04]
\includecode{resenja/4a_ArgumentiKomandneLinije/1.c}
\end{Answer}
 \fi


\begin{Exercise}[label=p2.6_01] 
Napisati program koji ispisuje zbir numeričkih argumenata komandne linije. 
\uputstvo{Koristiti funkciju \kckod{atoi}}.

\begin{miditest}
\begin{upotreba}{1}
#\poziv{./a.out 5 mkp 9 -2 11 a 4 2}#

#\naslovIzlaz#
#\izlaz{Zbir numerickih argumenata: 29}#
\end{upotreba}
\end{miditest}
\begin{miditest}
\begin{upotreba}{2}
#\poziv{./a.out ab u f hj}#

#\naslovIzlaz#
#\izlaz{Zbir numerickih argumenata: 0}#
\end{upotreba}
\end{miditest}

\begin{miditest}
\begin{upotreba}{3}
#\poziv{./a.out 33 1 p 44}#

#\naslovIzlaz#
#\izlaz{Zbir numerickih argumenata: 78}#
\end{upotreba}
\end{miditest}
\begin{miditest}
\begin{upotreba}{4}
#\poziv{./a.out}#

#\naslovIzlaz#
#\izlaz{Zbir numerickih argumenata: 0}#
\end{upotreba}
\end{miditest}
\linkresenje{p2.6_01}
\end{Exercise}
\ifresenja
\begin{Answer}[ref=p2.6_01]
\includecode{resenja/4a_ArgumentiKomandneLinije/2.c}
\end{Answer}
 \fi

 
\begin{Exercise}[label=p2.6_04] 
 Napisati program koji na osnovu broja $n$, koji se zadaje kao argument komandne 
 linije, ispisuje cele brojeve iz intervala $[-n,\ n]$.
 U slučaju greške ispisati odgovarajuću poruku o grešci.
 
\begin{minitest}
\begin{upotreba}{1}
#\poziv{./a.out 2}#

#\naslovIzlaz#
#\izlaz{-2 -1 0 1 2}#
\end{upotreba}
\end{minitest}
\begin{minitest}
\begin{upotreba}{2}
#\poziv{./a.out}#

#\naslovIzlaz#
#\izlaz{Greska: neispravan poziv.}#
\end{upotreba}
\end{minitest}
\begin{minitest}
\begin{upotreba}{3}
#\poziv{./a.out 0}#

#\naslovIzlaz#
#\izlaz{0}#
\end{upotreba}
\end{minitest}

\linkresenje{p2.6_04}
\end{Exercise}
\ifresenja
\begin{Answer}[ref=p2.6_04]
\includecode{resenja/4a_ArgumentiKomandneLinije/3.c}
\end{Answer}
 \fi 
 

\begin{Exercise}[label=p2.6_02] 
Napisati program koji ispisuje argumente komandne linije koji počinju karakterom \textit{@}.

\begin{miditest}
\begin{upotreba}{1}
#\poziv{./a.out @marija @milena \#zvezda}#

#\naslovIzlaz#
#\izlaz{Argumenti koji pocinju sa @:}#
#\izlaz{@marija @milena}#
\end{upotreba}
\end{miditest}
\begin{miditest}
\begin{upotreba}{2}
#\poziv{./a.out bundeva pomorandza}#

#\naslovIzlaz#
#\izlaz{Nema argumenata koji pocinju sa @.}#
\end{upotreba}
\end{miditest}

\begin{miditest}
\begin{upotreba}{3}
#\poziv{./a.out sanke @zapad zujanje}#

#\naslovIzlaz#
#\izlaz{Argumenti koji pocinju sa @:}#
#\izlaz{@zapad}#
\end{upotreba}
\end{miditest}
\begin{miditest}
\begin{upotreba}{4}
#\poziv{./a.out}#

#\naslovIzlaz#
#\izlaz{Nema argumenata koji pocinju sa @.}#
\end{upotreba}
\end{miditest}
\linkresenje{p2.6_02}
\end{Exercise}
\ifresenja
\begin{Answer}[ref=p2.6_02]
\includecode{resenja/4a_ArgumentiKomandneLinije/4.c}
\end{Answer}
 \fi


\begin{Exercise}[label=p2.6_03] 
Napisati program koji ispisuje broj argumenata komandne linije koji sadrže karakter \textit{@}.

\begin{miditest}
\begin{upotreba}{1}
#\poziv{./a.out pera@gmail.com www.math.rs}#

#\naslovIzlaz#
#\izlaz{1}#
\end{upotreba}
\end{miditest}
\begin{miditest}
\begin{upotreba}{2}
#\poziv{./a.out math.rs pera@math.rs }#

#\naslovIzlaz#
#\izlaz{2}#
\end{upotreba}
\end{miditest}

\begin{miditest}
\begin{upotreba}{3}
#\poziv{./a.out japan caj}#

#\naslovIzlaz#
#\izlaz{0}#
\end{upotreba}
\end{miditest}
\begin{miditest}
\begin{upotreba}{4}
#\poziv{./a.out}#

#\naslovIzlaz#
#\izlaz{0}#
\end{upotreba}
\end{miditest}


\linkresenje{p2.6_03}
\end{Exercise}
\ifresenja
\begin{Answer}[ref=p2.6_03]
\includecode{resenja/4a_ArgumentiKomandneLinije/5.c}
\end{Answer}
 \fi


\begin{Exercise}[label=p2.6_05] 
 Napisati program koji proverava da li se među zadatim argumentima 
 komandne linije nalaze barem dva ista.
 
\begin{miditest}
\begin{upotreba}{1}
#\poziv{./a.out pec zima deda mraz pec}#

#\naslovIzlaz#
#\izlaz{Medju argumentima ima istih.}#
\end{upotreba}
\end{miditest}
\begin{miditest}
\begin{upotreba}{2}
#\poziv{./a.out xyz abc abc abc efgh}#

#\naslovIzlaz#
#\izlaz{Medju argumentima ima istih.}#
\end{upotreba}
\end{miditest}

\begin{miditest}
\begin{upotreba}{3}
#\poziv{./a.out 11 15 abc 888}#

#\naslovIzlaz#
#\izlaz{Medju argumentima nema istih.}#
\end{upotreba}
\end{miditest}
\begin{miditest}
\begin{upotreba}{4}
#\poziv{./a.out}#

#\naslovIzlaz#
#\izlaz{Medju argumentima nema istih.}#
\end{upotreba}
\end{miditest}

\linkresenje{p2.6_05}
\end{Exercise}
\ifresenja
\begin{Answer}[ref=p2.6_05]
\includecode{resenja/4a_ArgumentiKomandneLinije/6.c}
\end{Answer}
 \fi


% \begin{Exercise}[label=v2.2_05] 
% Napisati funkciju koja za dva data stringa određuje koliko se uzastopnih 
% karaktera prvog stringa nalazi u drugom stringu počev od početka. Napisati 
%    program koji testira napisanu funkciju  za dva stringa 
%    koji se unose kao argumenti komandne linije.   \\
% \begin{miditest}
% \begin{upotreba}{1}
% #\poziv{./a.out aladin bal}#
% #\naslovInt#
% #\izlaz{3}#
% \end{upotreba}
% \end{miditest}
% \begin{miditest}
% \begin{upotreba}{2}
% #\poziv{./a.out aladin lad}#
% #\naslovInt#
% #\izlaz{4}#
% \end{upotreba}
% \end{miditest}
% \begin{miditest}
% \begin{upotreba}{3}
% #\poziv{./a.out Aladin ala}#
% #\naslovInt#
% #\izlaz{0}#
% \end{upotreba}
% \end{miditest}
% \begin{miditest}
% \begin{upotreba}{4}
% #\poziv{./a.out aladin}#
% #\naslovInt#
% #\izlaz{Nekorektan poziv}#
% #\izlaz{Program treba pozvati sa ./a.out arg1 arg2}#
% \end{upotreba}
% \end{miditest}
% 
% \linkresenje{v2.2_05}
% \end{Exercise}
% \ifresenja
% \begin{Answer}[ref=v2.2_05]
% \includecode{resenja/4a_ArgumentiKomandneLinije/1_05.c}
% \end{Answer}
%  \fi



\begin{Exercise}[label=v2.2_01] 
 Napisati program koji ispisuje sve opcije koje su navedene u komandnoj liniji.
 \uputstvo{Opcije su karakteri koji se nalaze nakon karaktera \kckod{-}.}
 
\begin{miditest}
\begin{upotreba}{1}
#\poziv{./a.out -abc in.txt -d -Fg out}#

#\naslovIzlaz#
#\izlaz{Opcije su: a b c d F g}#
\end{upotreba}
\end{miditest}
\begin{miditest}
\begin{upotreba}{2}
#\poziv{./a.out}#

#\naslovIzlaz#
#\izlaz{Medju argumentima nema opcija.}#
\end{upotreba}
\end{miditest}

\begin{miditest}
\begin{upotreba}{3}
#\poziv{./a.out ulaz.txt  }#

#\naslovIzlaz#
#\izlaz{Medju argumentima nema opcija.}#
\end{upotreba}
\end{miditest}
\begin{miditest}
\begin{upotreba}{4}
#\poziv{./a.out in.txt -x -yZ -g out}#

#\naslovIzlaz#
#\izlaz{Opcije su: x y Z g}#
\end{upotreba}
\end{miditest}

\linkresenje{v2.2_01}
\end{Exercise}
\ifresenja
\begin{Answer}[ref=v2.2_01]
\includecode{resenja/4a_ArgumentiKomandneLinije/7.c}
\end{Answer}
 \fi


% \begin{Exercise}[label=v2.2_06] 
% Napisati funkciju \kckod{void sifruj(char s[], char c, int k)} koja šifruje
%    string s na sledeći način: svako malo i veliko slovo stringa s konvertuje u
%    slovo koje je u abecedi od njega udaljeno k pozicija, i to 
%    k pozicija ulevo, ako je karakter c jednak karakteru 'L' ili udesno
%    ako je karakter c jednak karakteru 'D'. Šifrovanje treba da bude kružno. Ako string
%    s sadrži karakter koji nije alfanumerički, ostaviti ga nešifriranog. Napisati program koji testira napisanu funkciju za string i prirodan
%    broj koji se unose kao argumenti komandne linije dok se pravac šifrovanja unosi
%    kao opcija -p koja može imati vrednosti 'L' ili 'D'. Ukoliko opcija -p nije 
%    navedena, podrazumevani pravac je udesno. \napomena{Možemo podrazumevati da string sadrži najviše 30 karaktera}.\\
% \begin{miditest}
% \begin{upotreba}{1}
% #\poziv{./a.out abcd 2}#
% #\naslovInt#
% #\izlaz{cdef}#
% \end{upotreba}
% \end{miditest}
% \begin{miditest}
% \begin{upotreba}{2}
% #\poziv{./a.out abcd 2 -p D}#
% #\naslovInt#
% #\izlaz{cdef}#
% \end{upotreba}
% \end{miditest}
% \begin{miditest}
% \begin{upotreba}{3}
% #\poziv{./a.out abcd 2 -p L}#
% #\naslovInt#
% #\izlaz{yzab}#
% \end{upotreba}
% \end{miditest}
% \begin{miditest}
% \begin{upotreba}{4}
% #\poziv{./a.out abcd -3 -p L}#
% #\naslovInt#
% #\izlaz{Nekorektan unos}#
% \end{upotreba}
% \end{miditest}
% \begin{miditest}
% \begin{upotreba}{5}
% #\poziv{./a.out abcd 3 -p X}#
% #\naslovInt#
% #\izlaz{Nekorektan unos}#
% \end{upotreba}
% \end{miditest}
% \begin{miditest}
% \begin{upotreba}{6}
% #\poziv{./a.out ab12cd 2 -p D}#
% #\naslovInt#
% #\izlaz{Nekorektan unos}#
% \end{upotreba}
% \end{miditest}
% 
% \linkresenje{v2.2_06}
% \end{Exercise}
% \ifresenja
% \begin{Answer}[ref=v2.2_06]
% \includecode{resenja/4a_ArgumentiKomandneLinije/1_06.c}
% \end{Answer}
%  \fi
% 
% \iffalse
% \begin{Exercise}[label=v2.2_01] 
% Tekst
% %\komentarJ{Ovaj zadatak nema mnogo smisla, predlazem da ga izbacimo.}
% \linkresenje{v2.2_01}
% \end{Exercise}
% \ifresenja
% \begin{Answer}[ref=v2.2_01]
% \includecode{resenja/4a_ArgumentiKomandneLinije/1_07.c}
% \end{Answer}
%  \fi
% \fi
% 
% \iffalse
% \begin{Exercise}[label=v2.2_01] 
% %\komentarJ{Da li je ovo zadatak iz pokazivaca? Link ka resenju je pogresan, da nije doslo do greske?}
% Ako su celi brojevi \verb|a| i \verb|b| argumenti komandne linije
% napraviti niz \verb|A[0] = a, A[1] = a+1,|
% \verb|A[2] = a+2, ..., A[b-a] = b| i ispisati ga. Pretpostaviti da je
% maksimalna du\v zina niza 200 elemenata. Proveriti da li $a < b$ i
% $b-a < 200$ i ako ovi uslovi nisu ispunjeni ispisati poruku da je do\v
% slo do gre\v ske. U slu\v caju da je dato manje ili vi\v se argumenata
% komandne linije ispisati poruku o gre\v sci. \\ 
% \begin{miditest}
% \begin{upotreba}{1}
% #\poziv{./a.out 34}#
% #\naslovInt#
% #\izlaz{greska}#
% \end{upotreba}
% \end{miditest}
% \begin{miditest}
% \begin{upotreba}{2}
% #\poziv{./a.out 12 20}#
% #\naslovInt#
% #\izlaz{12 13 14 15 16 17 18 19 20}#
% \end{upotreba}
% \end{miditest}
% \begin{miditest}
% \begin{upotreba}{3}
% #\poziv{./a.out 30 8}#
% #\naslovInt#
% #\izlaz{greska}#
% \end{upotreba}
% \end{miditest}
% \begin{miditest}
% \begin{upotreba}{4}
% #\poziv{./a.out -4 -1}#
% #\naslovInt#
% #\izlaz{-4 -3 -2 -1}#
% \end{upotreba}
% \end{miditest}
% \linkresenje{v2.2_01}
% \end{Exercise}
% \ifresenja
% \begin{Answer}[ref=v2.2_01]
% \includecode{resenja/4a_ArgumentiKomandneLinije/1_07.c}
% \end{Answer}
%  \fi
% 
% \fi
% 
% \begin{Exercise}[label=v2.2_01] 
% %\komentarJ{Da li je ovo zadatak iz pokazivaca? Link ka resenju je pogresan, da nije doslo do greske?}
% Parametri komandne linije su $n, a$ i $b$ ($a < b$). Treba popuniti
% prvih {\tt n} elemenata niza {\tt A} celim slu\v cajnim brojevima koji
% su između {\tt a} i {\tt b}. I\v stampati niz {\tt A} na standarni
% izlaz. Maksimalan broj elemenata niza {\tt A} je 200. Ukoliko nisu
% zadati svi argumenti komandne linije ili ne zadovoljavaju potrebna
% svojstva ispisati poruku o gre\v sci. 
% 
% %\linkresenje{v2.2_01}
% \end{Exercise}
% %\ifresenja
% %\begin{Answer}[ref=v2.2_01]
% %\includecode{resenja/4a_ArgumentiKomandneLinije/1_07.c}
% %\end{Answer}
% % \fi

\ifresenja
\section{Rešenja}
\shipoutAnswer
\fi
