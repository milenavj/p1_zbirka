\chapter{Ulaz i izlaz programa}

\section{Argumenti komandne linije}
 
\begin{Exercise}[label=v2.2_04] 
    Napisati program koji ispisuje broj navedenih argumenata komandne linije,
    a zatim i same argumenate i njihove redne brojeve.
    
\begin{miditest}
\begin{upotreba}{1}
#\poziv{./a.out d1.txt 10 13.5 d2.txt}#

#\naslovIzlaz#
#\izlaz{Broj argumenata je 5.}#
#\izlaz{0: ./a.out}#
#\izlaz{1: d1.txt}#
#\izlaz{2: 10}#
#\izlaz{3: 13.5}#
#\izlaz{4: d2.txt}#
\end{upotreba}
\end{miditest}
\begin{miditest}
\begin{upotreba}{2}
#\poziv{./a.out}#

#\naslovIzlaz#
#\izlaz{Broj argumenata je 1.}#
#\izlaz{0: ./a.out}#
\end{upotreba}
\end{miditest}

\linkresenje{v2.2_04}
\end{Exercise}
\ifresenja
\begin{Answer}[ref=v2.2_04]
\includecode{resenja/4a_ArgumentiKomandneLinije/1.c}
\end{Answer}
 \fi


\begin{Exercise}[label=p2.6_01] 
Napisati program koji ispisuje zbir celobrojnih argumenata komandne linije. 
\uputstvo{Koristiti funkciju \kckod{atoi}}.

\begin{minitest}
\begin{upotreba}{1}
#\naslovPokretanje#
#\argumenti{./a.out 5 ana 9 -2 11 4 +2}#

#\naslovIzlaz#
#\izlaz{Zbir celobrojnih argumenata:}#
#\izlaz{29}#
\end{upotreba}
\end{minitest}
\begin{minitest}
\begin{upotreba}{2}
#\naslovPokretanje#
#\argumenti{./a.out a1 b1 1a 1b}#

#\naslovIzlaz#
#\izlaz{Zbir celobrojnih argumenata:}#
#\izlaz{0}#
\end{upotreba}
\end{minitest}
\begin{minitest}
\begin{upotreba}{3}
#\naslovPokretanje#
#\argumenti{./a.out 33 1 @matf 44 22.56}#

#\naslovIzlaz#
#\izlaz{Zbir celobrojnih argumenata:}#
#\izlaz{78}#
\end{upotreba}
\end{minitest}
\linkresenje{p2.6_01}
\end{Exercise}
\ifresenja
\begin{Answer}[ref=p2.6_01]
\includecode{resenja/4a_ArgumentiKomandneLinije/2.c}
\end{Answer}
 \fi

 
\begin{Exercise}[label=p2.6_04] 
 Napisati program koji na osnovu broja $n$ koji se zadaje kao argument komandne 
 linije, ispisuje cele brojeve iz intervala $[-n,\ n]$.
 U slučaju neispravnog pokretanja programa ispisati odgovarajuću poruku o grešci.
 
\begin{minitest}
\begin{upotreba}{1}
#\poziv{./a.out 2}#

#\naslovIzlaz#
#\izlaz{-2 -1 0 1 2}#
\end{upotreba}
\end{minitest}
\begin{minitest}
\begin{upotreba}{2}
#\poziv{./a.out}#

#\naslovIzlaz#
#\izlaz{Greska: neispravan poziv.}#
\end{upotreba}
\end{minitest}
\begin{minitest}
\begin{upotreba}{3}
#\poziv{./a.out 0}#

#\naslovIzlaz#
#\izlaz{0}#
\end{upotreba}
\end{minitest}

\linkresenje{p2.6_04}
\end{Exercise}
\ifresenja
\begin{Answer}[ref=p2.6_04]
\includecode{resenja/4a_ArgumentiKomandneLinije/3.c}
\end{Answer}
 \fi 
 

\begin{Exercise}[label=p2.6_02] 
Napisati program koji ispisuje argumente komandne linije koji počinju karakterom \textit{@}.

\begin{minitest}
\begin{upotreba}{1}
#\naslovPokretanje#
#\argumenti{./a.out @ana @aca \#zvezda}#

#\naslovIzlaz#
#\izlaz{Argumenti koji pocinju sa @:}#
#\izlaz{@ana @aca}#
\end{upotreba}
\end{minitest}
\begin{minitest}
\begin{upotreba}{2}
#\naslovPokretanje#
#\argumenti{./a.out sanke @zapad zujanje}#

#\naslovIzlaz#
#\izlaz{Argumenti koji pocinju sa @:}#
#\izlaz{@zapad}#
\end{upotreba}
\end{minitest}
\begin{minitest}
\begin{upotreba}{3}
#\naslovPokretanje#
#\argumenti{./a.out bundeva pomorandza}#

#\naslovIzlaz#
#\izlaz{Nema argumenata koji}#
#\izlaz{pocinju sa @.}#
\end{upotreba}
\end{minitest}

\linkresenje{p2.6_02}
\end{Exercise}
\ifresenja
\begin{Answer}[ref=p2.6_02]
\includecode{resenja/4a_ArgumentiKomandneLinije/4.c}
\end{Answer}
 \fi


\begin{Exercise}[label=p2.6_03] 
Napisati program koji ispisuje broj argumenata komandne linije koji sadrže karakter \textit{@}.

\begin{minitest}
\begin{upotreba}{1}
#\naslovPokretanje#
#\argumenti{./a.out pera@gmail.com @}#

#\naslovIzlaz#
#\izlaz{Rezultat: 2}#
\end{upotreba}
\end{minitest}
\begin{minitest}
\begin{upotreba}{2}
#\naslovPokretanje#
#\argumenti{./a.out japan caj}#

#\naslovIzlaz#
#\izlaz{0}#
\end{upotreba}
\end{minitest}
\begin{minitest}
\begin{upotreba}{3}
#\naslovPokretanje#
#\argumenti{./a.out}#

#\naslovIzlaz#
#\izlaz{Rezultat: 0}#
\end{upotreba}
\end{minitest}


\linkresenje{p2.6_03}
\end{Exercise}
\ifresenja
\begin{Answer}[ref=p2.6_03]
\includecode{resenja/4a_ArgumentiKomandneLinije/5.c}
\end{Answer}
 \fi


\begin{Exercise}[label=p2.6_05] 
 Napisati program koji proverava da li se među zadatim argumentima 
 komandne linije nalaze barem dva ista.
 
\begin{miditest}
\begin{upotreba}{1}
#\naslovPokretanje#
#\argumenti{./a.out ulaz.txt izlaz.txt ulaz.txt}#

#\naslovIzlaz#
#\izlaz{Medju argumentima ima istih.}#
\end{upotreba}
\end{miditest}
\begin{miditest}
\begin{upotreba}{2}
#\naslovPokretanje#
#\argumenti{./a.out srce pik tref tref}#

#\naslovIzlaz#
#\izlaz{Medju argumentima ima istih.}#
\end{upotreba}
\end{miditest}

\begin{miditest}
\begin{upotreba}{3}
#\naslovPokretanje#
#\argumenti{./a.out Riba ribi grize rep.}#

#\naslovIzlaz#
#\izlaz{Medju argumentima nema istih.}#
\end{upotreba}
\end{miditest}
\begin{miditest}
\begin{upotreba}{4}
#\naslovPokretanje#
#\argumenti{./a.out}#

#\naslovIzlaz#
#\izlaz{Medju argumentima nema istih.}#
\end{upotreba}
\end{miditest}

\linkresenje{p2.6_05}
\end{Exercise}
\ifresenja
\begin{Answer}[ref=p2.6_05]
\includecode{resenja/4a_ArgumentiKomandneLinije/6.c}
\end{Answer}
 \fi


\begin{Exercise}[label=v2.2_01] 
 Napisati program koji ispisuje sve opcije koje su navedene u komandnoj liniji. Opcije su karakteri argumenata koji se nalaze nakon karaktera \kckod{-}.
 
\begin{miditest}
\begin{upotreba}{1}
#\naslovPokretanje#
#\argumenti{./a.out -rf in.txt}#

#\naslovIzlaz#
#\izlaz{Opcije su: r f}#
\end{upotreba}
\end{miditest}
\begin{miditest}
\begin{upotreba}{2}
#\naslovPokretanje#
#\argumenti{./a.out}#

#\naslovIzlaz#
#\izlaz{Medju argumentima nema opcija.}#
\end{upotreba}
\end{miditest}

\begin{miditest}
\begin{upotreba}{3}
#\naslovPokretanje#
#\argumenti{./a.out ulaz.txt}#

#\naslovIzlaz#
#\izlaz{Medju argumentima nema opcija.}#
\end{upotreba}
\end{miditest}
\begin{miditest}
\begin{upotreba}{4}
#\naslovPokretanje#
#\argumenti{./a.out in.txt -l -n 10 -fi out.txt}#

#\naslovIzlaz#
#\izlaz{Opcije su: l n f i}#
\end{upotreba}
\end{miditest}

\linkresenje{v2.2_01}
\end{Exercise}
\ifresenja
\begin{Answer}[ref=v2.2_01]
\includecode{resenja/4a_ArgumentiKomandneLinije/7.c}
\end{Answer}
 \fi


\ifresenja
\section{Rešenja}
\shipoutAnswer
\fi
