\section{Petlje}

%--------------------------------------------------------------------
%--------------------------------------------------------------------
% \subsection{Ispis podataka}
%--------------------------------------------------------------------
%--------------------------------------------------------------------


\begin{Exercise}[label=PET_01] 
Napisati program koji pet puta ispisuje tekst \kckod{Mi volimo da programiramo}.  

\begin{miditest}
\begin{upotreba}{1}
#\naslovInt#
#\izlaz{Mi volimo da programiramo.}#
#\izlaz{Mi volimo da programiramo.}#
#\izlaz{Mi volimo da programiramo.}#
#\izlaz{Mi volimo da programiramo.}#
#\izlaz{Mi volimo da programiramo.}#
\end{upotreba}
\end{miditest}

\linkresenje{PET_01}
\end{Exercise}
\ifresenja
\begin{Answer}[ref=PET_01]
\includecode{resenja/2_KontrolaToka/1.3_Petlje/1.3_1.c}
\end{Answer}
\fi


\begin{Exercise}[label=PET_02] 
Napisati program koji učitava pozitivan ceo broj $n$ i $n$ puta ispisuje tekst
\kckod{Mi volimo da programiramo}. 
U slučaju neispravnog unosa, ispisati odgovarajuću poruku o grešci. 

\begin{miditest}
\begin{upotreba}{1}
#\naslovInt#
#\izlaz{Unesite broj n:}\ulaz{6}#
#\izlaz{Mi volimo da programiramo.}#
#\izlaz{Mi volimo da programiramo.}#
#\izlaz{Mi volimo da programiramo.}#
#\izlaz{Mi volimo da programiramo.}#
#\izlaz{Mi volimo da programiramo.}#
#\izlaz{Mi volimo da programiramo.}#
\end{upotreba}
\end{miditest}
\begin{miditest}
\begin{upotreba}{2}
#\naslovInt#
#\izlaz{Unesite broj n:}\ulaz{0}#
#\izlaz{Greska: pogresan unos broja n.}#
\end{upotreba}
\end{miditest}

\begin{miditest}
\begin{upotreba}{3}
#\naslovInt#
#\izlaz{Unesite broj n:}\ulaz{-5}#
#\izlaz{Greska: pogresan unos broja n.}#
\end{upotreba}
\end{miditest}
\begin{miditest}
\begin{upotreba}{4}
#\naslovInt#
#\izlaz{Unesite broj n:}\ulaz{1}#
#\izlaz{Mi volimo da programiramo.}#
\end{upotreba}
\end{miditest}

\linkresenje{PET_02}
\end{Exercise}
\ifresenja
\begin{Answer}[ref=PET_02]
\includecode{resenja/2_KontrolaToka/1.3_Petlje/1.3_2.c}
\end{Answer}
\fi


\begin{Exercise}[label=PET_03] 
Napisati program koji učitava nenegativan ceo broj $n$
a potom ispisuje sve cele brojeve od $0$ do $n$. 
U slučaju neispravnog unosa, ispisati odgovarajuću poruku o grešci. 

\begin{miditest}
\begin{upotreba}{1}
#\naslovInt#
#\izlaz{Unesite broj n:}\ulaz{4}#
#\izlaz{0 1 2 3 4}#
\end{upotreba}
\end{miditest}
\begin{miditest}
\begin{upotreba}{2}
#\naslovInt#
#\izlaz{Unesite broj n:}\ulaz{-10}#
#\izlaz{Greska: pogresan unos broja n.}#
\end{upotreba}
\end{miditest}
\linkresenje{PET_03}
\end{Exercise}
\ifresenja
\begin{Answer}[ref=PET_03]
\includecode{resenja/2_KontrolaToka/1.3_Petlje/1.3_3.c}
\end{Answer}
\fi


\begin{Exercise}[label=PET_04] 
Napisati program koji učitava dva cela broja $n$ i $m$ $(n \leq m)$ i ispisuje sve
cele brojeve iz intervala $[n,m]$. Pri rešavanju zadatka: 
\begin{enumerate}
\item koristiti \kckod{while} petlju
\item koristiti \kckod{for} petlju
\item koristiti \kckod{do-while} petlju
 \end{enumerate}
U slučaju neispravnog unosa, ispisati odgovarajuću poruku o grešci. 

\begin{miditest}
\begin{upotreba}{1}
#\naslovInt#
#\izlaz{Unesite granice intervala:}\ulaz{-2 4}#
#\izlaz{-2 -1 0 1 2 3 4}#
\end{upotreba}
\end{miditest}
\begin{miditest}
\begin{upotreba}{2}
#\naslovInt#
#\izlaz{Unesite granice intervala:}\ulaz{10 6}#
#\izlaz{Greska: pogresan unos granica. }#
\end{upotreba}
\end{miditest}
\linkresenje{PET_04}
\end{Exercise}
\ifresenja
\begin{Answer}[ref=PET_04]
% \includecode{resenja/2_KontrolaToka/1.3_Petlje/1.3_4a.c}
% \includecode{resenja/2_KontrolaToka/1.3_Petlje/1.3_4b.c}
% \includecode{resenja/2_KontrolaToka/1.3_Petlje/1.3_4c.c}
\includecode{resenja/2_KontrolaToka/1.3_Petlje/1.3_4.c}
\end{Answer}
\fi

%--------------------------------------------------------------------
%--------------------------------------------------------------------
% \subsection{Obrada celih brojeva, rad sa ciframa broja}
%--------------------------------------------------------------------
%--------------------------------------------------------------------

\begin{Exercise}[label=PET_05] 
 Napisati program koji učitava nenegativan ceo broj $n$ i izračunava njegov
 faktorijel. 
 U slučaju neispravnog unosa, ispisati odgovarajuću poruku o grešci. 

\begin{miditest}
\begin{upotreba}{1}
#\naslovInt#
#\izlaz{Unesite broj n:}\ulaz{18}#
#\izlaz{18! = 6402373705728000}#
\end{upotreba}
\end{miditest}
\begin{miditest}
\begin{upotreba}{2}
#\naslovInt#
#\izlaz{Unesite broj n:}\ulaz{8}#
#\izlaz{8! = 40320}#
\end{upotreba}
\end{miditest}

\begin{miditest}
\begin{upotreba}{3}
#\naslovInt#
#\izlaz{Unesite broj n:}\ulaz{40}#
#\izlaz{Pri racunanju 40! ce doci do prekoracenja.}#
\end{upotreba}
\end{miditest}
\begin{miditest}
\begin{upotreba}{4}
#\naslovInt#
#\izlaz{Unesite broj n:}\ulaz{-5}#
#\izlaz{Greska: neispravan unos. }#
\end{upotreba}
\end{miditest}

 \linkresenje{PET_05}
\end{Exercise}
\ifresenja
\begin{Answer}[ref=PET_05]
\includecode{resenja/2_KontrolaToka/1.3_Petlje/1.3_5.c}
\end{Answer}
\fi


\begin{Exercise}[label=PET_06] 
Napisati program koji učitava realan broj $x$ i ceo nenegativan broj
$n$ i izračunava $n$-ti stepen broja $x$, tj.~$x^n$.
U slučaju neispravnog unosa, ispisati odgovarajuću poruku o grešci.
 
\begin{miditest}
\begin{upotreba}{1}
#\naslovInt#
#\izlaz{Unesite redom brojeve x i n:}\ulaz{4 3}#
#\izlaz{Rezultat: 64.00000}#
\end{upotreba}
\end{miditest}
\begin{miditest}
\begin{upotreba}{2}
#\naslovInt#
#\izlaz{Unesite redom brojeve x i n:}\ulaz{5.8 5}#
#\izlaz{Rezultat: 6563.56768}#
\end{upotreba}
\end{miditest}

\begin{miditest}
\begin{upotreba}{3}
#\naslovInt#
#\izlaz{Unesite redom brojeve x i n:}\ulaz{11.43 -6}#
#\izlaz{Greska: neispravan unos broja n.}#
\end{upotreba}
\end{miditest}
\begin{miditest}
\begin{upotreba}{4}
#\naslovInt#
#\izlaz{Unesite redom brojeve x i n:}\ulaz{11.43 0}#
#\izlaz{Rezultat: 1.00000}#
\end{upotreba}
\end{miditest}

\linkresenje{PET_06}
\end{Exercise}
\ifresenja
\begin{Answer}[ref=PET_06]
  \includecode{resenja/2_KontrolaToka/1.3_Petlje/1.3_6.c}
\end{Answer}
\fi

\begin{Exercise}[label=PET_07]
Napisati program koji učitava realan broj $x$ i ceo broj
$n$ i izračunava $n$-ti stepen broja $x$. 
 
\begin{miditest}
\begin{upotreba}{1}
#\naslovInt#
#\izlaz{Unesite redom brojeve x i n:}\ulaz{2 -3}#
#\izlaz{Rezultat: 0.125}#
\end{upotreba}
\end{miditest}
\begin{miditest}
\begin{upotreba}{2}
#\naslovInt#
#\izlaz{Unesite redom brojeve x i n:}\ulaz{-3 2}#
#\izlaz{Rezultat: 9.000}#
\end{upotreba}
\end{miditest}
\linkresenje{PET_07}
\end{Exercise}
\ifresenja
\begin{Answer}[ref=PET_07]
\includecode{resenja/2_KontrolaToka/1.3_Petlje/1.3_7.c}
\end{Answer}
\fi


\begin{Exercise}[label=PET_08] 
Pravi delioci celog broja su svi delioci sem jedinice i samog tog
broja. Napisati program za uneti pozitivan ceo broj $n$
ispisuje sve njegove prave delioce.
U slučaju neispravnog unosa, ispisati odgovarajuću poruku o grešci.

\begin{miditest}
\begin{upotreba}{1}
#\naslovInt#
#\izlaz{Unesite broj n:}\ulaz{100}#
#\izlaz{Pravi delioci: 2 4 5 10 20 25 50}#
\end{upotreba}
\end{miditest}
\begin{miditest}
\begin{upotreba}{2}
#\naslovInt#
#\izlaz{Unesite broj n:}\ulaz{-6}#
#\izlaz{Greska: neispravan unos.}#
\end{upotreba}
\end{miditest}
\linkresenje{PET_08}
\end{Exercise}
\ifresenja
\begin{Answer}[ref=PET_08]
\includecode{resenja/2_KontrolaToka/1.3_Petlje/1.3_8.c}
\end{Answer}
\fi


\begin{Exercise}[label=PET_09] 
Napisati program koji za uneti ceo broj ispisuje broj dobijen
uklanjanjem svih nula sa desne strane unetog broja.
 
\begin{miditest}
\begin{upotreba}{1}
#\naslovInt#
#\izlaz{Unesite broj:}\ulaz{12000}#
#\izlaz{Rezultat: 12}#
\end{upotreba}
\end{miditest}
\begin{miditest}
\begin{upotreba}{2}
#\naslovInt#
#\izlaz{Unesite broj:}\ulaz{0}#
#\izlaz{Rezultat: 0}#
\end{upotreba}
\end{miditest}

\begin{miditest}
\begin{upotreba}{3}
#\naslovInt#
#\izlaz{Unesite broj:}\ulaz{-1400}#
#\izlaz{Rezultat: -14}#
\end{upotreba}
\end{miditest}
\begin{miditest}
\begin{upotreba}{4}
#\naslovInt#
#\izlaz{Unesite broj:}\ulaz{147}#
#\izlaz{Rezultat: 147}#
\end{upotreba}
\end{miditest}

\linkresenje{PET_09}
\end{Exercise}
\ifresenja
\begin{Answer}[ref=PET_09]
\includecode{resenja/2_KontrolaToka/1.3_Petlje/1.3_9.c}
\end{Answer}
\fi


\begin{Exercise}[label=PET_10] 
Napisati program koji učitava ceo broj i ispisuje njegove cifre u
obrnutom poretku. 

\begin{miditest}
\begin{upotreba}{1}
#\naslovInt#
#\izlaz{Unesite ceo broj:}\ulaz{6789}#
#\izlaz{Rezultat: 9 8 7 6}#
\end{upotreba}
\end{miditest}
\begin{miditest}
\begin{upotreba}{2}
#\naslovInt#
#\izlaz{Unesite ceo broj:}\ulaz{-892345}#
#\izlaz{Rezultat: 5 4 3 2 9 8}#
\end{upotreba}
\end{miditest}
\linkresenje{PET_10}
\end{Exercise}
\ifresenja
\begin{Answer}[ref=PET_10]
\includecode{resenja/2_KontrolaToka/1.3_Petlje/1.3_10.c}
\end{Answer}
\fi


\begin{Exercise}[label=PET_11] 
Napisati program koji za uneti pozitivan ceo broj ispisuje da li je on deljiv
sumom svojih cifara. 
U slučaju neispravnog unosa, ispisati odgovarajuću poruku o grešci. 

\begin{miditest}
\begin{upotreba}{1}
#\naslovInt#
#\izlaz{Unesite broj:}\ulaz{12}#
#\izlaz{Broj 12 je deljiv sa 3.}#
\end{upotreba}
\end{miditest}
\begin{miditest}
\begin{upotreba}{2}
#\naslovInt#
#\izlaz{Unesite broj:}\ulaz{2564}#
#\izlaz{Broj 2564 nije deljiv sa 17.}#
\end{upotreba}
\end{miditest}

\begin{miditest}
\begin{upotreba}{3}
#\naslovInt#
#\izlaz{Unesite broj:}\ulaz{-4}#
#\izlaz{Greska: neispravan ulaz.}#
\end{upotreba}
\end{miditest}
\begin{miditest}
\begin{upotreba}{4}
#\naslovInt#
#\izlaz{Unesite broj:}\ulaz{0}#
#\izlaz{Greska: neispravan ulaz.}#
\end{upotreba}
\end{miditest}
\linkresenje{PET_11}
\end{Exercise}
\ifresenja
\begin{Answer}[ref=PET_11]
\includecode{resenja/2_KontrolaToka/1.3_Petlje/1.3_11.c}
\end{Answer}
\fi


\begin{Exercise}[label=PET_12] 
Knjigovođa vodi evidenciju o transakcijama jedne firme i treba da napiše izveštaj
o godišnjem poslovanju te firme. Firma je tokom godine imala $t$ transakcija. Transakcije
su predstavljene celim brojevima i u slučaju da je vrednost transakcije pozitivna, ta transakcija označava
prihod firme, a u slučaju da je negativna rashod. 
Napisati program koji učitava nenegativan ceo broj $t$ i podatke o $t$ transakcija i
zatim izračunava i ispisuje ukupan prihod, ukupan rashod i zaradu, odnosno gubitak, koji je firma ostvarila
tokom godine. 
U slučaju neispravnog unosa, ispisati odgovarajuću poruku o grešci. 

\begin{minitest}
\begin{upotreba}{1}
#\naslovInt#
#\izlaz{Unesite broj t:}\ulaz{7}#
#\izlaz{Unesite transakcije:}#
#\ulaz{8 -50 45 2007 -67 -123 14}#
#\izlaz{Prihod: 2074}#
#\izlaz{Rashod: -240}#
#\izlaz{Zarada: 1834}#
\end{upotreba}
\end{minitest}
\begin{minitest}
\begin{upotreba}{2}
#\naslovInt#
#\izlaz{Unesite broj t:}\ulaz{5}#
#\izlaz{Unesite transakcije:}#
#\ulaz{-5 -20 -4 -200 -8}#
#\izlaz{Prihod: 0}#
#\izlaz{Rashod: -237}#
#\izlaz{Gubitak: 237}#
\end{upotreba}
\end{minitest}
\begin{minitest}
\begin{upotreba}{3}
#\naslovInt#
#\izlaz{Unesite broj t:}\ulaz{2}#
#\izlaz{Unesite transakcije:}#
#\ulaz{120 -120}#
#\izlaz{Prihod: 120}#
#\izlaz{Rashod: -120}#
#\izlaz{Zarada: 0}#
\end{upotreba}
\end{minitest}

\begin{minitest}
\begin{upotreba}{4}
#\naslovInt#
#\izlaz{Unesite broj t:}\ulaz{-6}#
#\izlaz{Greska: neispravan unos.}#
\end{upotreba}
\end{minitest}
\begin{minitest}
\begin{upotreba}{5}
#\naslovInt#
#\izlaz{Unesite broj n:}\ulaz{0}#
#\izlaz{Nema evidentiranih}#
#\izlaz{transakcija.}#
\end{upotreba}
\end{minitest}

\linkresenje{PET_12}
\end{Exercise}
\ifresenja
\begin{Answer}[ref=PET_12]
\includecode{resenja/2_KontrolaToka/1.3_Petlje/1.3_12.c}
\end{Answer}
\fi


\begin{Exercise}[label=PET_13] 
Napisati program koji učitava pozitivan ceo broj $n$, a potom i $n$ celih
brojeva. Izračunati i ispisati zbir onih brojeva koji su istovremeno neparni i
negativni.
U slučaju neispravnog unosa, ispisati odgovarajuću poruku o grešci.

\begin{miditest}
\begin{upotreba}{1}
#\naslovInt#
#\izlaz{Unesite broj n:}\ulaz{5}#
#\izlaz{Unesite n brojeva:}#
#\ulaz{1 -5 -6 3 -11}#
#\izlaz{Zbir neparnih i negativnih: -16}#
\end{upotreba}
\end{miditest}
\begin{miditest}
\begin{upotreba}{2}
#\naslovInt#
#\izlaz{Unesite broj n:}\ulaz{-4}#
#\izlaz{Greska: neispravan unos.}#
\end{upotreba}
\end{miditest}

\begin{miditest}
\begin{upotreba}{3}
#\naslovInt#
#\izlaz{Unesite broj n:}\ulaz{4}#
#\izlaz{Unesite n brojeva:}#
#\ulaz{5 8 13 17}#
#\izlaz{Zbir neparnih i negativnih: 0}#
\end{upotreba}
\end{miditest}
\begin{miditest}
\begin{upotreba}{4}
#\naslovInt#
#\izlaz{Unesite broj n:}\ulaz{0}#
#\izlaz{Greska: neispravan unos.}#
\end{upotreba}
\end{miditest}

\linkresenje{PET_13}
\end{Exercise}
\ifresenja
\begin{Answer}[ref=PET_13]
\includecode{resenja/2_KontrolaToka/1.3_Petlje/1.3_13.c}
\end{Answer}
\fi


\begin{Exercise}[label=PET_14] 
Napisati program koji učitava pozitivan ceo broj $n$, a potom $n$ celih brojeva i
računa i ispisuje sumu brojeva koji su deljivi sa $5$, a nisu deljivi sa $7$.
U slučaju neispravnog unosa, ispisati odgovarajuću poruku o grešci.

\begin{miditest}
\begin{upotreba}{1}
#\naslovInt#
#\izlaz{Unesite broj n:}\ulaz{5}#
#\izlaz{Unesite n brojeva:}\ulaz{2 35 5 -175 -20 }#
#\izlaz{Suma: -15}#
\end{upotreba}
\end{miditest}
\begin{miditest}
\begin{upotreba}{2}
#\naslovInt#
#\izlaz{Unesite broj n:}\ulaz{-3}#
#\izlaz{Greska: neispravan unos.}#
\end{upotreba}
\end{miditest}

\begin{miditest}
\begin{upotreba}{3}
#\naslovInt#
#\izlaz{Unesite broj n:}\ulaz{10}#
#\izlaz{Unesite n brojeva:}#
#\ulaz{-5 6 175 -20 -25 -8 42 245 1 6}#
#\izlaz{Suma: -50}#
\end{upotreba}
\end{miditest}
\begin{miditest}
\begin{upotreba}{4}
#\naslovInt#
#\izlaz{Unesite broj n:}\ulaz{6}#
#\izlaz{Unesite brojeve:}#
#\ulaz{2205 -1904 2 7 -540 5}#
#\izlaz{Suma: -535}#
\end{upotreba}
\end{miditest}

\linkresenje{PET_14}
\end{Exercise}
\ifresenja
\begin{Answer}[ref=PET_14]
\includecode{resenja/2_KontrolaToka/1.3_Petlje/1.3_14.c}
\end{Answer}
\fi


\begin{Exercise}[label=PET_16] 
Napisati program koji učitava cele brojeve sve do unosa broja nula i ispisuje proizvod onih unetih brojeva koji su
pozitivni.  

\begin{miditest}
\begin{upotreba}{1}
#\naslovInt#
#\izlaz{Unesite brojeve:}#
#\ulaz{-87 12 -108 -13 56 0}#
#\izlaz{Proizvod pozitivnih brojeva je 672.}#
\end{upotreba}
\end{miditest}
\begin{miditest}
\begin{upotreba}{2}
#\naslovInt#
#\izlaz{Unesite brojeve:}\ulaz{0}#
#\izlaz{Nije unet nijedan broj.}#
\end{upotreba}
\end{miditest}

\begin{miditest}
\begin{upotreba}{3}
#\naslovInt#
#\izlaz{Unesite brojeve:}#
#\ulaz{-5 -200 -43 0}#
#\izlaz{Medju unetim brojevima nema pozitivnih.}#
\end{upotreba}
\end{miditest}
\begin{miditest}
\begin{upotreba}{4}
#\naslovInt#
#\izlaz{Unesite brojeve:}\ulaz{1 0}#
#\izlaz{Proizvod pozitivnih brojeva je 1.}#
\end{upotreba}
\end{miditest}

\linkresenje{PET_16}
\end{Exercise}
\ifresenja
\begin{Answer}[ref=PET_16]
\includecode{resenja/2_KontrolaToka/1.3_Petlje/1.3_16.c}
\end{Answer}
\fi


\begin{Exercise}[label=PET_17] 
 Napisati program koji za uneti ceo broj proverava i ispisuje da
 li se cifra $5$ nalazi u njegovom zapisu.

\begin{minitest}
\begin{upotreba}{1}
#\naslovInt#
#\izlaz{Unesite broj:}\ulaz{1857}#
#\izlaz{Broj 1857 sadrzi cifru 5.}#
\end{upotreba}
\end{minitest}
\begin{minitest}
\begin{upotreba}{2}
#\naslovInt#
#\izlaz{Unesite broj:}\ulaz{84}#
#\izlaz{Broj 84 ne sadrzi cifru 5.}#
\end{upotreba}
\end{minitest}
\begin{minitest}
\begin{upotreba}{3}
#\naslovInt#
#\izlaz{Unesite broj:}\ulaz{-2515}#
#\izlaz{Broj -2515 sadrzi cifru 5.}#
\end{upotreba}
\end{minitest}
\linkresenje{PET_17}
\end{Exercise}
\ifresenja
\begin{Answer}[ref=PET_17]
\includecode{resenja/2_KontrolaToka/1.3_Petlje/1.3_17.c}
\end{Answer}
\fi


\begin{Exercise}[label=PET_18] 
Napisati program koji učitava cele brojeve sve do unosa broja nula, a zatim
izračunava i ispisuje aritmetičku sredinu unetih brojeva
na četiri decimale.

\begin{minitest}
\begin{upotreba}{1}
#\naslovInt#
#\izlaz{Unesite brojeve:}#
#\ulaz{8 5 6 3 0}#
#\izlaz{Aritmeticka sredina:}#
#\izlaz{5.5000}#
\end{upotreba}
\end{minitest}
\begin{minitest}
\begin{upotreba}{2}
#\naslovInt#
#\izlaz{Unesite brojeve:}\ulaz{0}#
#\izlaz{Nisu uneti brojevi.}#
\end{upotreba}
\end{minitest}
\begin{minitest}
\begin{upotreba}{3}
#\naslovInt#
#\izlaz{Unesite brojeve:}#
#\ulaz{762 -12 800 2010}#
#\ulaz{-356 899 -101 0}#
#\izlaz{Aritmeticka sredina:}#
#\izlaz{571.7143}#
\end{upotreba}
\end{minitest}

\linkresenje{PET_18}
\end{Exercise}
\ifresenja
\begin{Answer}[ref=PET_18]
\includecode{resenja/2_KontrolaToka/1.3_Petlje/1.3_18.c}
\end{Answer}
\fi


\begin{Exercise}[label=PET_19] 
U prodavnici se nalaze artikli čije su cene pozitivni realni
brojevi. Napisati program koji učitava cene artikala sve do unosa broja nula i
izračunava i ispisuje prosečnu vrednost cena u radnji.
U slučaju neispravnog unosa, ispisati odgovarajuću poruku o grešci.

\begin{minitest}
\begin{upotreba}{1}
#\naslovInt#
#\izlaz{Unesite cene:}#
#\ulaz{8 5.2 6.11 3 0}#
#\izlaz{Prosecna cena: 5.5775}#
\end{upotreba}
\end{minitest}
\begin{minitest}
\begin{upotreba}{2}
#\naslovInt#
#\izlaz{Unesite cene:}#
#\ulaz{6.32 -9}#
#\izlaz{Greska: neispravan unos}#
#\izlaz{cene.}#
\end{upotreba}
\end{minitest}
\begin{minitest}
\begin{upotreba}{3}
#\naslovInt#
#\izlaz{Unesite cene:}#
#\ulaz{0}#
#\izlaz{Nisu unete cene.}#
\end{upotreba}
\end{minitest}
\linkresenje{PET_19}
\end{Exercise}
\ifresenja
\begin{Answer}[ref=PET_19]
\includecode{resenja/2_KontrolaToka/1.3_Petlje/1.3_19.c}
\end{Answer}
\fi


\begin{Exercise}[label=PET_20] 
Napisati program koji učitava pozitivan ceo broj $n$, potom $n$ realnih
brojeva, a zatim određuje i ispisuje koliko puta je prilikom unosa došlo 
do promene znaka.
U slučaju neispravnog unosa, ispisati odgovarajuću poruku o grešci.

\begin{miditest}
\begin{upotreba}{1}
#\naslovInt#
#\izlaz{Unesite broj n:}\ulaz{9}#
#\izlaz{Unesite brojeve:}#
#\ulaz{7.82 4.3 -1.2 56.8 -3.4 -72.1 8.9 11.2 -11.2}#
#\izlaz{Broj promena je 5.}#
\end{upotreba}
\end{miditest}
\begin{miditest}
\begin{upotreba}{2}
#\naslovInt#
#\izlaz{Unesite broj n:}\ulaz{5}#
#\izlaz{Unesite brojeve:}#
#\ulaz{-23.8 -11.2 0 5.6 7.2}#
#\izlaz{Broj promena je 1.}#
\end{upotreba}
\end{miditest}

\begin{miditest}
\begin{upotreba}{3}
#\naslovInt#
#\izlaz{Unesite broj n:}\ulaz{-6}#
#\izlaz{Greska: neispravan unos.}#
\end{upotreba}
\end{miditest}
\begin{miditest}
\begin{upotreba}{4}
#\naslovInt#
#\izlaz{Unesite broj n:}\ulaz{0}#
#\izlaz{Greska: neispravan unos.}#
\end{upotreba}
\end{miditest}

\linkresenje{PET_20}
\end{Exercise}
\ifresenja
\begin{Answer}[ref=PET_20]
\includecode{resenja/2_KontrolaToka/1.3_Petlje/1.3_20.c}
\end{Answer}
\fi


\begin{Exercise}[label=PET_21] 
U prodavnici se nalazi $n$ artikala čije su cene pozitivni realni
brojevi. Napisati program koji učitava $n$, a potom i cenu svakog od
$n$ artikala i određuje i ispisuje najmanju cenu.
U slučaju neispravnog unosa, ispisati odgovarajuću poruku o grešci.

\begin{minitest}
\begin{upotreba}{1}
#\naslovInt#
#\izlaz{Unesite broj artikla:}\ulaz{6}#
#\izlaz{Unesite cene artikala:}#
#\ulaz{12 3.4 90 100.53 53.2 12.8}#
#\izlaz{Najmanja cena: 3.400000}#
\end{upotreba}
\end{minitest}
\begin{minitest}
\begin{upotreba}{2}
#\naslovInt#
#\izlaz{Unesite broj artikla:}\ulaz{3}#
#\izlaz{Unesite cene artikala:}#
#\ulaz{4 -8 92}#
#\izlaz{Greska: neispravan unos}#
#\izlaz{cene.}#
\end{upotreba}
\end{minitest}
\begin{minitest}
\begin{upotreba}{3}
#\naslovInt#
#\izlaz{Unesite broj artikla:}\ulaz{-9}#
#\izlaz{Greska: neispravan unos.}#
\end{upotreba}
\end{minitest}

\linkresenje{PET_21}
\end{Exercise}
\ifresenja
\begin{Answer}[ref=PET_21]
\includecode{resenja/2_KontrolaToka/1.3_Petlje/1.3_21.c}
\end{Answer}
\fi


\begin{Exercise}[label=PET_15] 
Nikola želi da obraduje baku i da joj kupi jedan poklon u radnji. On
na raspolaganju ima $m$ dinara. U radnji se nalazi $n$ artikala i
zanima ga koliko ima artikala u radnji čija cena je manja ili
jednaka $m$. Napisati program koji pomaže Nikoli da brzo odredi
broj atikala. Program učitava realan nenegativan broj $m$, ceo
nenegativan broj $n$ i $n$ pozitivnih realnih brojeva. 
Ispisati koliko artikala ima cenu čija je vrednost manja ili jednaka $m$. 
\napomena{Pretpostaviti da je unos ispravan.}

\begin{miditest}
\begin{upotreba}{1}
#\naslovInt#
#\izlaz{Nikolin budzet:}\ulaz{12.37}#
#\izlaz{Unesite broj artikala:}\ulaz{5}#
#\izlaz{Unesite cene artikala:}\ulaz{11 54.13 6 13 8}#
#\izlaz{Ukupno artikala: 3}#
\end{upotreba}
\end{miditest}
\begin{miditest}
\begin{upotreba}{2}
#\naslovInt#
#\izlaz{Nikolin budzet:}\ulaz{2}#
#\izlaz{Unesite broj artikala:}\ulaz{4}#
#\izlaz{Unesite cene artikala:}\ulaz{1 11 4.32 3}#
#\izlaz{Ukupno artikala: 1}#
\end{upotreba}
\end{miditest}

\linkresenje{PET_15}
\end{Exercise}
\ifresenja
\begin{Answer}[ref=PET_15]
\includecode{resenja/2_KontrolaToka/1.3_Petlje/1.3_15.c}
\end{Answer}
\fi


\begin{Exercise}[label=PET_22] 
Napisati program koji učitava ceo nenegativan broj $n$, $n$ celih
brojeva i zatim izračunava i ispisuje tražene vrednosti.
U slučaju neispravnog unosa, ispisati odgovarajuću poruku o grešci.

\begin{enumerate}
\item Broj sa najvećom cifrom desetica. Ukoliko ima više takvih, ispisati prvi.

\begin{miditest}
\begin{upotreba}{1}
#\naslovInt#
#\izlaz{Unesite broj n:}\ulaz{5}#
#\izlaz{Unesite brojeve:}#
#\ulaz{18 365 25 1 78}#
#\izlaz{Broj sa najvecom cifrom desetica: 78.}#
\end{upotreba}
\end{miditest}
\begin{miditest}
\begin{upotreba}{2}
#\naslovInt#
#\izlaz{Unesite broj n:}\ulaz{8}#
#\izlaz{Unesite brojeve:}#
#\ulaz{14 1576 -1267 -89 109 122 306 918}#
#\izlaz{Broj sa najvecom cifrom desetica: -89.}#
\end{upotreba}
\end{miditest}

\begin{miditest}
\begin{upotreba}{3}
#\naslovInt#
#\izlaz{Unesite broj n:}\ulaz{4}#
#\izlaz{Unesite brojeve:}#
#\ulaz{100 200 300 400}#
#\izlaz{Broj sa najvecom cifrom desetica: 100.}#
\end{upotreba}
\end{miditest}
\begin{miditest}
\begin{upotreba}{4}
#\naslovInt#
#\izlaz{Unesite broj n:}\ulaz{-12}#
#\izlaz{Greska: neispravan unos.}#
\end{upotreba}
\end{miditest}

\item Broj sa najvećim brojem cifara. Ukoliko ima više takvih, ispisati prvi.

\begin{miditest}
\begin{upotreba}{1}
#\naslovInt#
#\izlaz{Unesite broj n:}\ulaz{5}#
#\izlaz{Unesite n brojeva:}\ulaz{18 -365 251 1 78}#
#\izlaz{Najvise cifara ima broj -365.}#
\end{upotreba}
\end{miditest}
\begin{miditest}
\begin{upotreba}{2}
#\naslovInt#
#\izlaz{Unesite broj n:}\ulaz{7}#
#\izlaz{Unesite n brojeva:}#
#\ulaz{3 892 18 21 639 742 85}#
#\izlaz{Najvise cifara ima broj 892.}#
\end{upotreba}
\end{miditest}

\begin{miditest}
\begin{upotreba}{3}
#\naslovInt#
#\izlaz{Unesite broj n:}\ulaz{0}#
#\izlaz{Nisu uneti brojevi.}
\end{upotreba}
\end{miditest}
\begin{miditest}
\begin{upotreba}{4}
#\naslovInt#
#\izlaz{Unesite broj n:}\ulaz{-7}#
#\izlaz{Greska: neispravan unos.}#
\end{upotreba}
\end{miditest}

\begin{miditest}
\begin{upotreba}{5}
#\naslovInt#
#\izlaz{Unesite broj n:}\ulaz{5}#
#\izlaz{Unesite n brojeva:}\ulaz{0 1 2 -3 4}#
#\izlaz{Najvise cifara ima broj 0.}#
\end{upotreba}
\end{miditest}
\begin{miditest}
\begin{upotreba}{6}
#\naslovInt#
#\izlaz{Unesite broj n:}\ulaz{5}#
#\izlaz{Unesite n brojeva:}\ulaz{-5 4 -3 2 1}#
#\izlaz{Najvise cifara ima broj -5.}#
\end{upotreba}
\end{miditest}

\item Broj sa najvećom vodećom cifrom. Vodeća cifra je cifra najveće težine u zapisu broja. Ukoliko
ima više takvih, ispisati prvi.

\begin{miditest}
\begin{upotreba}{1}
#\naslovInt#
#\izlaz{Unesite broj n:}\ulaz{5}#
#\izlaz{Unesite n brojeva:}\ulaz{8 964 -32 511 27}#
#\izlaz{Broj sa najvecom vodecom cifrom je 964.}#
\end{upotreba}
\end{miditest}
\begin{miditest}
\begin{upotreba}{2}
#\naslovInt#
#\izlaz{Unesite broj n:}\ulaz{3}#
#\izlaz{Unesite n brojeva:}\ulaz{0 0 0}#
#\izlaz{Broj sa najvecom vodecom cifrom je 0.}#
\end{upotreba}
\end{miditest}

\begin{miditest}
\begin{upotreba}{3}
#\naslovInt#
#\izlaz{Unesite broj n:}\ulaz{3}#
#\izlaz{Unesite n brojeva:}\ulaz{41 669 -8}#
#\izlaz{Broj sa najvecom vodecom cifrom je -8.}#
\end{upotreba}
\end{miditest}
\begin{miditest}
\begin{upotreba}{4}
#\naslovInt#
#\izlaz{Unesite broj n:}\ulaz{0}#
#\izlaz{Nisu uneti brojevi.}#
\end{upotreba}
\end{miditest}

\end{enumerate}

\linkresenje{PET_22}
\end{Exercise}
\ifresenja
\begin{Answer}[ref=PET_22]
\includecodeLib{resenja/2_KontrolaToka/1.3_Petlje/1.3_22.c}{Rešenje (a)}
\includecodeLib{resenja/2_KontrolaToka/1.3_Petlje/1.3_23.c}{Rešenje (b)}
\includecodeLib{resenja/2_KontrolaToka/1.3_Petlje/1.3_24.c}{Rešenje (c)}
\end{Answer}
\fi


\begin{Exercise}[label=PET_25] 
Vršena su merenja nadmorskih visina na određenom delu teritorije i
naučnike zanima razlika između najveće i najmanje nadmorske
visine. Napisati program koji učitava cele brojeve koji predstavljaju nadmorske visine 
sve do unosa broja nula i ispisuje razliku najveće i najmanje nadmorske visine.

\begin{minitest}
\begin{upotreba}{1}
#\naslovInt#
#\izlaz{Unesite brojeve:}#
#\ulaz{8 6 5 2 11 7 0}#
#\izlaz{Razlika: 9}#
\end{upotreba}
\end{minitest}
\begin{minitest}
\begin{upotreba}{2}
#\naslovInt#
#\izlaz{Unesite brojeve:}#
#\ulaz{8 -1 8 6 0}#
#\izlaz{Razlika: 9}#
\end{upotreba}
\end{minitest}
\begin{minitest}
\begin{upotreba}{3}
#\naslovInt#
#\izlaz{Unesite brojeve:}#
#\ulaz{0}#
#\izlaz{Nisu unete nadmorske}#
#\izlaz{visine.}#
\end{upotreba}
\end{minitest}

\begin{minitest}
\begin{upotreba}{4}
#\naslovInt#
#\izlaz{Unesite brojeve:}#
#\ulaz{-500 0}#
#\izlaz{Razlika: 0}#
\end{upotreba}
\end{minitest}
\begin{minitest}
\begin{upotreba}{5}
#\naslovInt#
#\izlaz{Unesite brojeve:}#
#\ulaz{-500 -300 -5000 0}#
#\izlaz{Razlika: 4700}#
\end{upotreba}
\end{minitest}

\linkresenje{PET_25}
\end{Exercise}
\ifresenja
\begin{Answer}[ref=PET_25]
\includecode{resenja/2_KontrolaToka/1.3_Petlje/1.3_25.c}
\end{Answer}
\fi


\begin{Exercise}[label=PET_26] 
Napisati program koji učitava ceo broj $n$ ($n>1$), nenegativan ceo broj $d$, a zatim i
$n$ celih brojeva i izračunava i ispisuje koliko ima parova
uzastopnih brojeva među unetim brojevima koji se nalaze na rastojanju
$d$. Rastojanje između brojeva je definisano sa $d(x,y)=|y-x|$.
U slučaju neispravnog unosa, ispisati odgovarajuću poruku o grešci.

\begin{miditest}
\begin{upotreba}{1}
#\naslovInt#
#\izlaz{Unesite brojeve n i d:}\ulaz{5 2}#
#\izlaz{Unesite n brojeva:}\ulaz{2 3 5 1 -1}#
#\izlaz{Broj parova: 2}#
\end{upotreba}
\end{miditest}
\begin{miditest}
\begin{upotreba}{2}
#\naslovInt#
#\izlaz{Unesite brojeve n i d:}\ulaz{10 5}#
#\izlaz{Unesite n brojeva:}#
#\ulaz{-3 6 11 -20 -25 -8 42 37 1 6}#
#\izlaz{Broj parova: 4}#
\end{upotreba}
\end{miditest}

\begin{miditest}
\begin{upotreba}{3}
#\naslovInt#
#\izlaz{Unesite brojeve n i d:}\ulaz{5 0}#
#\izlaz{Unesite n brojeva:}\ulaz{1 1 1 1 1}#
#\izlaz{Broj parova: 4}#
\end{upotreba}
\end{miditest}
\begin{miditest}
\begin{upotreba}{4}
#\naslovInt#
#\izlaz{Unesite brojeve n i d:}\ulaz{1 3}#
#\izlaz{Greska: neispravan unos.}#
\end{upotreba}
\end{miditest}

\linkresenje{PET_26}
\end{Exercise}
\ifresenja
\begin{Answer}[ref=PET_26]
\includecode{resenja/2_KontrolaToka/1.3_Petlje/1.3_26.c}
\end{Answer}
\fi


\begin{Exercise}[label=PET_27] 
Napisati program koji uneti pozitivan ceo broj transformiše tako što svaku
parnu cifru u zapisu broja uveća za jedan. Ispisati dobijeni broj.
U slučaju neispravnog unosa, ispisati odgovarajuću poruku o grešci.

\begin{minitest}
\begin{upotreba}{1}
#\naslovInt#
#\izlaz{Unesite broj:}\ulaz{2417}#
#\izlaz{Rezultat: 3517}#
\end{upotreba}
\end{minitest}
\begin{minitest}
\begin{upotreba}{2}
#\naslovInt#
#\izlaz{Unesite broj:}\ulaz{138}#
#\izlaz{Rezultat: 139}#
\end{upotreba}
\end{minitest}
\begin{minitest}
\begin{upotreba}{3}
#\naslovInt#
#\izlaz{Unesite broj:}\ulaz{59}#
#\izlaz{Rezultat: 59}#
\end{upotreba}
\end{minitest}

\linkresenje{PET_27}
\end{Exercise}
\ifresenja
\begin{Answer}[ref=PET_27]
\includecode{resenja/2_KontrolaToka/1.3_Petlje/1.3_27.c}
\end{Answer}
\fi


\begin{Exercise}[label=PET_28]
Napisati program koji učitava jedan ceo broj i zatim formira i ispisuje broj 
koji se dobija izbacivanjem svake druge cifre polaznog broja, idući sa desna na levo.
 
\begin{minitest}
\begin{upotreba}{1}
#\naslovInt#
#\izlaz{Unesite broj:}\ulaz{21854}#
#\izlaz{Rezultat: 284}#
\end{upotreba}
\end{minitest}
\begin{minitest}
\begin{upotreba}{2}
#\naslovInt#
#\izlaz{Unesite broj:}\ulaz{-18}#
#\izlaz{Rezultat: -8}#
\end{upotreba}
\end{minitest}
\begin{minitest}
\begin{upotreba}{3}
#\naslovInt#
#\izlaz{Unesite broj:}\ulaz{1}#
#\izlaz{Rezultat: 1}#
\end{upotreba}
\end{minitest}

\linkresenje{PET_28}
\end{Exercise}
\ifresenja
\begin{Answer}[ref=PET_28]
\includecode{resenja/2_KontrolaToka/1.3_Petlje/1.3_28.c}
\end{Answer}
\fi


\begin{Exercise}[difficulty=1, label=PET_29] 
 Napisati program koji na osnovu unetog pozitivnog celog broja formira i
 ispisuje broj koji se dobija izbacivanjem cifara koje su u polaznom broju
 jednake zbiru svojih suseda.
 U slučaju neispravnog unosa, ispisati odgovarajuću poruku o grešci.

\begin{minitest}
\begin{upotreba}{1}
#\naslovInt#
#\izlaz{Unesite broj:}\ulaz{28631}#
#\izlaz{Rezultat: 2631}#
\end{upotreba}
\end{minitest}
\begin{minitest}
\begin{upotreba}{2}
#\naslovInt#
#\izlaz{Unesite broj:}\ulaz{440}#
#\izlaz{Rezultat: 40}#
\end{upotreba}
\end{minitest}
\begin{minitest}
\begin{upotreba}{3}
#\naslovInt#
#\izlaz{Unesite broj:}\ulaz{-5}#
#\izlaz{Greska: neispravan unos.}#
\end{upotreba}
\end{minitest}
\linkresenje{PET_29}
\end{Exercise}
\ifresenja
\begin{Answer}[ref=PET_29]
\includecode{resenja/2_KontrolaToka/1.3_Petlje/1.3_29.c}
\end{Answer}
\fi


\begin{Exercise}[difficulty=1, label=PET_30] 
Broj je \textit{palindrom} ukoliko se isto čita i sa leve i sa desne
strane. Napisati program koji učitava pozitivan ceo broj i proverava da li
je učitani broj palindrom.
U slučaju neispravnog unosa, ispisati odgovarajuću poruku o grešci.

\begin{minitest}
\begin{upotreba}{1}
#\naslovInt#
#\izlaz{Unesite broj:}\ulaz{25452}#
#\izlaz{Broj je palindrom.}#
\end{upotreba}
\end{minitest}
\begin{minitest}
\begin{upotreba}{2}
#\naslovInt#
#\izlaz{Unesite broj:}\ulaz{895}#
#\izlaz{Broj nije palindrom.}#
\end{upotreba}
\end{minitest}
\begin{minitest}
\begin{upotreba}{3}
#\naslovInt#
#\izlaz{Unesite broj:}\ulaz{5}#
#\izlaz{Broj je palindrom.}#
\end{upotreba}
\end{minitest}
\linkresenje{PET_30}
\end{Exercise}
\ifresenja
\begin{Answer}[ref=PET_30]
\includecode{resenja/2_KontrolaToka/1.3_Petlje/1.3_30.c}
\end{Answer}
\fi


\begin{Exercise}[label=PET_31] 
Fibonačijev niz počinje članovima $0$ i $1$, a svaki naredni član se dobija
kao zbir prethodna dva. Napisati program koji učitava nenegativan ceo broj
$n$ i određuje i ispisuje $n$-ti član Fibonačijevog niza. Niz se indeksira
počevši od nule.
U slučaju neispravnog unosa, ispisati odgovarajuću poruku o grešci.

\begin{miditest}
\begin{upotreba}{1}
#\naslovInt#
#\izlaz{Unesite broj n:}\ulaz{10}#
#\izlaz{F[10] = 55}#
\end{upotreba}
\end{miditest}
\begin{miditest}
\begin{upotreba}{2}
#\naslovInt#
#\izlaz{Unesite broj n:}\ulaz{-100}#
#\izlaz{Greska: neispravan unos.}#
\end{upotreba}
\end{miditest}

\begin{miditest}
\begin{upotreba}{3}
#\naslovInt#
#\izlaz{Unesite broj n:}\ulaz{40}#
#\izlaz{F[40] = 102334155}#
\end{upotreba}
\end{miditest}
\begin{miditest}
\begin{upotreba}{4}
#\naslovInt#
#\izlaz{Unesite broj n:}\ulaz{20}#
#\izlaz{F[20] = 6765}#
\end{upotreba}
\end{miditest}

\linkresenje{PET_31}
\end{Exercise}
\ifresenja
\begin{Answer}[ref=PET_31]
\includecode{resenja/2_KontrolaToka/1.3_Petlje/1.3_31.c}
\end{Answer}
\fi


\begin{Exercise}[label=PET_32] 
Niz prirodnih brojeva formira se prema sledećem pravilu:
\begin{equation*}
a_{n+1} = \left\{
\begin{array}{cl}
\frac{a_n}{2} & \text{ako je } a_n \text{ parno}\\
\frac{3\cdot a_n + 1}{2} & \text{ako je } a_n \text{ neparno}\\
\end{array} \right.
\end{equation*}
Napisati program koji za uneti početni član niza $a_0$ (pozitivan ceo
broj) štampa niz brojeva sve do onog člana niza koji je jednak $1$. 
U slučaju neispravnog unosa, ispisati odgovarajuću poruku o grešci.

\begin{minitest}
\begin{upotreba}{1}
#\naslovInt#
#\izlaz{Unesite prvi clan:}\ulaz{56}#
#\izlaz{Clanovi niza:}#
#\izlaz{56 28 14 7 11 17 26 13}#
#\izlaz{20 10 5 8 4 2 1}#
\end{upotreba}
\end{minitest}
\begin{minitest}
\begin{upotreba}{2}
#\naslovInt#
#\izlaz{Unesite prvi clan:}\ulaz{-48}#
#\izlaz{Greska: neispravan unos.}#
\end{upotreba}
\end{minitest}
\begin{minitest}
\begin{upotreba}{3}
#\naslovInt#
#\izlaz{Unesite prvi clan:}\ulaz{67}#
#\izlaz{Clanovi niza:}#
#\izlaz{67 101 152 76 38 19 29}#
#\izlaz{44 22 11 17 26 13 20 10}#
#\izlaz{5 8 4 2 1}#
\end{upotreba}
\end{minitest}
\linkresenje{PET_32}
\end{Exercise}
\ifresenja
\begin{Answer}[ref=PET_32]
\includecode{resenja/2_KontrolaToka/1.3_Petlje/1.3_32.c}
\end{Answer}
\fi


\begin{Exercise}[difficulty=1, label=PET_33] 
Papir $A_0$ ima površinu 1$m^2$ i odnos stranica
$1:\sqrt{2}$. Papir $A_1$ dobija se podelom papira $A_0$ po dužoj
ivici. Papir $A_2$ dobija se podelom $A_1$ papira po dužoj ivici
itd. Napisati program koji za uneti nenegativan broj $k$ ispisuje 
dimenzije papira $A_k$ u milimetrima. Rezultat ispisati kao celobrojne
vrednosti. U slučaju neispravnog unosa, ispisati odgovarajuću poruku o grešci.
  
\begin{miditest}
\begin{upotreba}{1}
#\naslovInt#
#\izlaz{Unesite format papira:}\ulaz{4}#
#\izlaz{Dimenzije papira: 210 297}#
\end{upotreba}
\end{miditest}
\begin{miditest}
\begin{upotreba}{2}
#\naslovInt#
#\izlaz{Unesite format papira:}\ulaz{0}#
#\izlaz{Dimenzije papira: 840 1189}#
\end{upotreba}
\end{miditest}

\begin{miditest}
\begin{upotreba}{3}
#\naslovInt#
#\izlaz{Unesite format papira:}\ulaz{-7}#
#\izlaz{Greska: neispravan unos.}#
\end{upotreba}
\end{miditest}
\begin{miditest}
\begin{upotreba}{4}
#\naslovInt#
#\izlaz{Unesite format papira:}\ulaz{9}#
#\izlaz{Dimenzije papira: 37 52}#
\end{upotreba}
\end{miditest}
\linkresenje{PET_33}
\end{Exercise}
\ifresenja
\begin{Answer}[ref=PET_33]
\includecode{resenja/2_KontrolaToka/1.3_Petlje/1.3_33.c}
\end{Answer}
\fi


%--------------------------------------------------------------------
%--------------------------------------------------------------------
% \subsection{Rad sa karakterima}
%--------------------------------------------------------------------
%--------------------------------------------------------------------

\begin{Exercise}[label=PET_34] 
Napisati program koji učitava karaktere dok se ne unese karakter tačka,
i ako je karakter malo slovo ispisuje odgovarajuće veliko, ako je
karakter veliko slovo ispisuje odgovarajuće malo, a u suprotnom
ispisuje isti karakter kao i uneti.

\begin{miditest}
\begin{upotreba}{1}
#\naslovInt#
#\ulaz{Danas je Veoma Lep DAN.}#
#\izlaz{dANAS JE vEOMA lEP dan}#
\end{upotreba}
\end{miditest}
\begin{miditest}
\begin{upotreba}{2}
#\naslovInt#
#\ulaz{PROGRAMIRANJE 1 je zanimljivo!.}#
#\izlaz{programiranje 1 JE ZANIMLJIVO!}#
\end{upotreba}
\end{miditest}

\linkresenje{PET_34}
\end{Exercise}
\ifresenja
\begin{Answer}[ref=PET_34]
\includecode{resenja/2_KontrolaToka/1.3_Petlje/1.3_34.c}
\end{Answer}
\fi


\begin{Exercise}[label=PET_35] 
Napisati program koji učitava karaktere sve do kraja ulaza, a potom
ispisuje broj velikih slova, broj malih slova, broj cifara, broj
belina i zbir unetih cifara.

\begin{miditest}
\begin{upotreba}{1}
#\naslovInt#
#\izlaz{Unesite tekst:}#
#\ulaz{Beograd - Nis 230km}#
#\ulaz{Uzice - Cacak 56.3km}#
#\ulaz{Subotica - Ruma 139km}#
#\izlaz{Velika slova: 6}#
#\izlaz{Mala slova: 32}#
#\izlaz{Cifre: 9}#
#\izlaz{Beline: 12}#
#\izlaz{Suma cifara: 32}#
\end{upotreba}
\end{miditest}
\begin{miditest}
\begin{upotreba}{2}
#\naslovInt#
#\izlaz{Unesite tekst:}#
#\ulaz{Isli smo u Afriku da sadimo papriku.}#
#\izlaz{Velika slova: 2}#
#\izlaz{Mala slova: 27}#
#\izlaz{Cifre: 0}#
#\izlaz{Beline: 7}#
#\izlaz{Suma cifara: 0}#
\end{upotreba}
\end{miditest}
\linkresenje{PET_35}
\end{Exercise}
\ifresenja
\begin{Answer}[ref=PET_35]
\includecode{resenja/2_KontrolaToka/1.3_Petlje/1.3_35.c}
\end{Answer}
\fi


\begin{Exercise}[label=PET_36] 
 Program učitava pozitivan ceo broj $n$, a potom i $n$ karaktera. Za
 svaki od samoglasnika ispisati koliko puta se pojavio među unetim
 karakterima. Ne praviti razliku između malih i velikih slova.
 U slučaju neispravnog unosa, ispisati odgovarajuću poruku o grešci.
 
\begin{minitest}
\begin{upotreba}{1}
#\naslovInt#
#\izlaz{Unesite broj n:}\ulaz{5}#
#\izlaz{Unesite n karaktera:}#
#\ulaz{uAbao}#
#\izlaz{Samoglasnik a: 2}#
#\izlaz{Samoglasnik e: 0}#
#\izlaz{Samoglasnik i: 0}#
#\izlaz{Samoglasnik o: 1}#
#\izlaz{Samoglasnik u: 1}#
\end{upotreba}
\end{minitest}
\begin{minitest}
\begin{upotreba}{2}
#\naslovInt#
#\izlaz{Unesite broj n:}\ulaz{7}#
#\izlaz{Unesite n karaktera:}#
#\ulaz{jk+EEae}#
#\izlaz{Samoglasnik a: 1}#
#\izlaz{Samoglasnik e: 3}#
#\izlaz{Samoglasnik i: 0}#
#\izlaz{Samoglasnik o: 0}#
#\izlaz{Samoglasnik u: 0}#
\end{upotreba}
\end{minitest}
\begin{minitest}
\begin{upotreba}{3}
#\naslovInt#
#\izlaz{Unesite broj n:}\ulaz{-7}#
#\izlaz{Greska: neispravan unos.}#
\end{upotreba}
\end{minitest}

\linkresenje{PET_36}
\end{Exercise}
\ifresenja
\begin{Answer}[ref=PET_36]
\includecode{resenja/2_KontrolaToka/1.3_Petlje/1.3_36.c}
\end{Answer}
\fi


\begin{Exercise}[label=PET_37] 
Program učitava pozitivan ceo broj $n$, a zatim i $n$ karaktera. Napisati
program koji proverava da li se od unetih karaktera može napisati reč
\textit{Zima}.
U slučaju neispravnog unosa, ispisati odgovarajuću poruku o grešci.

\begin{miditest}
\begin{upotreba}{1}
#\naslovInt#
#\izlaz{Unesite broj n:}\ulaz{4}#
#\izlaz{Unestite 1. karakter: }\ulaz{+}#
#\izlaz{Unestite 2. karakter: }\ulaz{o}#
#\izlaz{Unestite 3. karakter: }\ulaz{Z}#
#\izlaz{Unestite 4. karakter: }\ulaz{j}#
#\izlaz{Ne moze se napisati rec Zima.}#
\end{upotreba}
\end{miditest}
\begin{miditest}
\begin{upotreba}{2}
#\naslovInt#
#\izlaz{Unesite broj n:}\ulaz{10}#
#\izlaz{Unestite 1. karakter: }\ulaz{i}#
#\izlaz{Unestite 2. karakter: }\ulaz{9}#
#\izlaz{Unestite 3. karakter: }\ulaz{0}#
#\izlaz{Unestite 4. karakter: }\ulaz{p}#
#\izlaz{Unestite 5. karakter: }\ulaz{a}#
#\izlaz{Unestite 6. karakter: }\ulaz{Z}#
#\izlaz{Unestite 7. karakter: }\ulaz{o}#
#\izlaz{Unestite 8. karakter: }\ulaz{m}#
#\izlaz{Unestite 9. karakter: }\ulaz{M}#
#\izlaz{Unestite 10. karakter: }\ulaz{-}#
#\izlaz{Moze se napisati rec Zima.}#
\end{upotreba}
\end{miditest}

\begin{miditest}
\begin{upotreba}{3}
#\naslovInt#
#\izlaz{Unesite broj n:}\ulaz{0}#
#\izlaz{Greska: neispravan unos.}#
\end{upotreba}
\end{miditest}

\linkresenje{PET_37}
\end{Exercise}
\ifresenja
\begin{Answer}[ref=PET_37]
\includecode{resenja/2_KontrolaToka/1.3_Petlje/1.3_37.c}
\end{Answer}
\fi


%--------------------------------------------------------------------
%--------------------------------------------------------------------
% \subsection{Računanje sume i proizvoda}
%--------------------------------------------------------------------
%--------------------------------------------------------------------


\begin{Exercise}[label=PET_38] 
Napisati program koji učitava pozitivan ceo broj $n$ i ispisuje
vrednost sume kubova brojeva od $1$ do $n$, odnosno $s = 1+2^3+3^3+
\ldots +n^3$. 
U slučaju neispravnog unosa, ispisati odgovarajuću poruku o grešci.

\begin{minitest}
\begin{upotreba}{1}
#\naslovInt#
#\izlaz{Unesite broj n:}\ulaz{14}#
#\izlaz{Suma kubova: 11025}#
\end{upotreba}
\end{minitest}
\begin{minitest}
\begin{upotreba}{2}
#\naslovInt#
#\izlaz{Unesite broj n:}\ulaz{25}#
#\izlaz{Suma kubova: 105625}#
\end{upotreba}
\end{minitest}
\begin{minitest}
\begin{upotreba}{3}
#\naslovInt#
#\izlaz{Unesite broj n:}\ulaz{0}#
#\izlaz{Greska: neispravan unos.}#
\end{upotreba}
\end{minitest}

\linkresenje{PET_38}
\end{Exercise}
\ifresenja
\begin{Answer}[ref=PET_38]
\includecode{resenja/2_KontrolaToka/1.3_Petlje/1.3_38.c}
\end{Answer}
\fi

\begin{Exercise}[label=PET_39] 
Napisati program koji učitava pozitivan ceo broj $n$ i ispisuje sumu
kubova, $s = 1+2^3+3^3+ \ldots +k^3$, za svaku vrednost $k = 1,
\ldots, n$.
U slučaju neispravnog unosa, ispisati odgovarajuću poruku o grešci.

\begin{minitest}
\begin{upotreba}{1}
#\naslovInt#
#\izlaz{Unesite broj n:}\ulaz{5}#
#\izlaz{[k=1] Suma kubova: 1}#
#\izlaz{[k=2] Suma kubova: 9}#
#\izlaz{[k=3] Suma kubova: 36}#
#\izlaz{[k=4] Suma kubova: 100}#
#\izlaz{[k=5] Suma kubova: 225}#
\end{upotreba}
\end{minitest}
\begin{minitest}
\begin{upotreba}{2}
#\naslovInt#
#\izlaz{Unesite broj n:}\ulaz{8}#
#\izlaz{[k=1] Suma kubova: 1}#
#\izlaz{[k=2] Suma kubova: 9}#
#\izlaz{[k=3] Suma kubova: 36}#
#\izlaz{[k=4] Suma kubova: 100}#
#\izlaz{[k=5] Suma kubova: 225}#
#\izlaz{[k=6] Suma kubova: 441}#
#\izlaz{[k=7] Suma kubova: 784}#
#\izlaz{[k=8] Suma kubova: 1296}#
\end{upotreba}
\end{minitest}
\begin{minitest}
\begin{upotreba}{3}
#\naslovInt#
#\izlaz{Unesite broj n:}\ulaz{-5}#
#\izlaz{Greska: neispravan unos.}#
\end{upotreba}
\end{minitest}

\linkresenje{PET_39}
\end{Exercise}
\ifresenja
\begin{Answer}[ref=PET_39]
Rešenje je analogno rešenju zadatka \ref{PET_38}.
\end{Answer}
\fi


\begin{Exercise}[label=PET_40]
 Napisati program koji učitava realan broj $x$ i pozitivan ceo broj $n$ i 
 izračunava i ispisuje sumu $S=x+2\cdot x^2+3\cdot
 x^3+\ldots+n\cdot x^n$.
 U slučaju neispravnog unosa, ispisati odgovarajuću poruku o grešci.
 
\begin{miditest}
\begin{upotreba}{1}
#\naslovInt#
#\izlaz{Unesite redom brojeve x i n:}\ulaz{2 3}#
#\izlaz{S = 34.000000}#
\end{upotreba}
\end{miditest}
\begin{miditest}
\begin{upotreba}{2}
#\naslovInt#
#\izlaz{Unesite redom brojeve x i n:}\ulaz{1.5 5}#
#\izlaz{S = 74.343750}#
\end{upotreba}
\end{miditest}

\begin{miditest}
\begin{upotreba}{3}
#\naslovInt#
#\izlaz{Unesite redom brojeve x i n:}\ulaz{5.5 0}#
#\izlaz{Greska: neispravan unos.}#
\end{upotreba}
\end{miditest}
\begin{miditest}
\begin{upotreba}{4}
#\naslovInt#
#\izlaz{Unesite redom brojeve x i n:}\ulaz{-0.5 -5}#
#\izlaz{Greska: neispravan unos.}#
\end{upotreba}
\end{miditest}

\linkresenje{PET_40}
\end{Exercise}
\ifresenja
\begin{Answer}[ref=PET_40]
\includecode{resenja/2_KontrolaToka/1.3_Petlje/1.3_40.c}
\end{Answer}
\fi


\begin{Exercise}[label=PET_41]
Napisati program koji učitava realan broj $x$ i pozitivan ceo broj $n$ i 
izračunava i ispisuje sumu
$S=1+\frac{1}{x}+\frac{1}{x^2}+\ldots\frac{1}{x^n}$.
U slučaju neispravnog unosa, ispisati odgovarajuću poruku o grešci.

\begin{miditest}
\begin{upotreba}{1}
#\naslovInt#
#\izlaz{Unesite redom brojeve x i n:}\ulaz{2 4}#
#\izlaz{S = 1.937500}#
\end{upotreba}
\end{miditest}
\begin{miditest}
\begin{upotreba}{2}
#\naslovInt#
#\izlaz{Unesite redom brojeve x i n:}\ulaz{1.8 6}#
#\izlaz{S = 2.213249}#
\end{upotreba}
\end{miditest}

\begin{miditest}
\begin{upotreba}{3}
#\naslovInt#
#\izlaz{Unesite redom brojeve x i n:}\ulaz{5.5 0}#
#\izlaz{Greska: neispravan unos.}#
\end{upotreba}
\end{miditest}
\begin{miditest}
\begin{upotreba}{4}
#\naslovInt#
#\izlaz{Unesite redom brojeve x i n:}\ulaz{-0.5 -5}#
#\izlaz{Greska: neispravan unos.}#
\end{upotreba}
\end{miditest}

\linkresenje{PET_41}
\end{Exercise}
\ifresenja
\begin{Answer}[ref=PET_41]
\includecode{resenja/2_KontrolaToka/1.3_Petlje/1.3_41.c}
\end{Answer}
\fi


\begin{Exercise}[difficulty=1, label=PET_42] 
Napisati program koji učitava realne brojeve $x$ i $eps$ i sa tačnošću $eps$ izračunava i ispisuje sumu
$S=1+x+\frac{x^2}{2!}+\frac{x^3}{3!}+\ldots$.  Za sumu S se kaže da je izračunata sa 
tačnošću $eps$ ako je apsolutna vrednost poslednjeg člana sume manja od $eps$.   
\uputstvo{Prilikom računanja sume koristiti prethodni
izračunati član sume u računanju sledećeg člana sume. Naime, ako je
izračunat član sume $\frac{x^n}{n!}$ na osnovu njega se lako može
dobiti član $\frac{x^{n+1}}{(n+1)!}$. Nikako ne računati stepen i
faktorijel odvojeno zbog neefikasnosti takvog rešenja i zbog
mogućnosti prekoračenja.}
  
\begin{miditest}
\begin{upotreba}{1}
#\naslovInt#
#\izlaz{Unesite x:}\ulaz{2}#
#\izlaz{Unesite tacnost eps:}\ulaz{0.001}#
#\izlaz{S = 7.388713}#
\end{upotreba}
\end{miditest}
\begin{miditest}
\begin{upotreba}{2}
#\naslovInt#
#\izlaz{Unesite x:}\ulaz{3}#
#\izlaz{Unesite tacnost eps:}\ulaz{0.01}#
#\izlaz{S = 20.079666}#
\end{upotreba}
\end{miditest}

\linkresenje{PET_42}
\end{Exercise}
\ifresenja
\begin{Answer}[ref=PET_42]
\includecode{resenja/2_KontrolaToka/1.3_Petlje/1.3_42.c}
\end{Answer}
\fi


\begin{Exercise}[difficulty=1, label=PET_43]
Napisati program koji učitava realne brojeve $x$ i $eps$ i sa zadatom
tačnošću $eps$ izračunava i ispisuje sumu
$S=1-x+\frac{x^2}{2!}-\frac{x^3}{3!}+\frac{x^4}{4!}-\frac{x^5}{5!}\ldots$.
\napomena{Voditi računa o efikasnosti rešenja i o mogućnosti prekoračenja.}
  
\begin{miditest}
\begin{upotreba}{1}
#\naslovInt#
#\izlaz{Unesite x:}\ulaz{3}#
#\izlaz{Unesite tacnost eps:}\ulaz{0.000001}#
#\izlaz{S = 0.049787}#
\end{upotreba}
\end{miditest}
\begin{miditest}
\begin{upotreba}{2}
#\naslovInt#
#\izlaz{Unesite x:}\ulaz{3.14}#
#\izlaz{Unesite tacnost eps:}\ulaz{0.01}#
#\izlaz{S = 0.049072}#
\end{upotreba}
\end{miditest}

\linkresenje{PET_43}
\end{Exercise}
\ifresenja
\begin{Answer}[ref=PET_43]
\includecode{resenja/2_KontrolaToka/1.3_Petlje/1.3_43.c}
\end{Answer}
\fi


\begin{Exercise}[label=PET_44] 
Napisati program koji učitava realan broj $x$ i pozitivan ceo broj $n$ i
izračunava proizvod $P = (1 + \cos(x))\cdot(1 + \cos(x^2))\cdot \ldots
\cdot(1 + \cos(x^n))$. 
U slučaju neispravnog unosa, ispisati odgovarajuću poruku o grešci.
\napomena{Voditi računa o efikasnosti rešenja}.  

\begin{miditest}
\begin{upotreba}{1}
#\naslovInt#
#\izlaz{Unesite redom brojeve x i n:}\ulaz{3.4 5}#
#\izlaz{P = 0.026817}#
\end{upotreba}
\end{miditest}
\begin{miditest}
\begin{upotreba}{2}
#\naslovInt#
#\izlaz{Unesite redom brojeve x i n:}\ulaz{12 8}#
#\izlaz{P = 2.640565}#
\end{upotreba}
\end{miditest}

\begin{miditest}
\begin{upotreba}{3}
#\naslovInt#
#\izlaz{Unesite redom brojeve x i n:}\ulaz{12 0}#
#\izlaz{Greska: neispravan unos.}#
\end{upotreba}
\end{miditest}
\begin{miditest}
\begin{upotreba}{4}
#\naslovInt#
#\izlaz{Unesite redom brojeve x i n:}\ulaz{12 -6}#
#\izlaz{Greska: neispravan unos.}#
\end{upotreba}
\end{miditest}

\linkresenje{PET_44}
\end{Exercise}
\ifresenja
\begin{Answer}[ref=PET_44]
\includecode{resenja/2_KontrolaToka/1.3_Petlje/1.3_44.c}
\end{Answer}
\fi


\begin{Exercise}[difficulty=1, label=PET_45] 
Napisati program koji učitava pozitivan ceo broj n i ispisuje vrednost
razlomka  \\
\[
  \frac{1}{1 + \frac{1}{2 + \frac{1}{3 + \frac{1}{4 + \frac{1}{\ldots + \frac{1}{(n-1) + \frac{1}{n}}}}}}}.
\]
U slučaju neispravnog unosa, ispisati odgovarajuću poruku o grešci.

\begin{minitest}
\begin{upotreba}{1}
#\naslovInt#
#\izlaz{Unesite broj n:}\ulaz{4}#
#\izlaz{R = 0.697674}#
\end{upotreba}
\end{minitest}
\begin{minitest}
\begin{upotreba}{2}
#\naslovInt#
#\izlaz{Unesite broj n:}\ulaz{20}#
#\izlaz{R = 0.697775}#
\end{upotreba}
\end{minitest}
\begin{minitest}
\begin{upotreba}{3}
#\naslovInt#
#\izlaz{Unesite broj n:}\ulaz{0}#
#\izlaz{Greska: neispravan unos.}#
\end{upotreba}
\end{minitest}

\linkresenje{PET_45}
\end{Exercise}
\ifresenja
\begin{Answer}[ref=PET_45]
\includecode{resenja/2_KontrolaToka/1.3_Petlje/1.3_45.c}
\end{Answer}
\fi


\begin{Exercise}[difficulty=1, label=PET_46] 
Napisati program koji učitava realan broj $x$ i pozitivan ceo broj $n$ i računa sumu
$$1 - \frac{x^{2}}{2!} + \frac{x^{4}}{4!} - \ldots +
(-1)^{n}\frac{x^{2n}}{(2n)!}.$$
U slučaju neispravnog unosa, ispisati odgovarajuću poruku o grešci.
\napomena{Voditi računa o efikasnosti rešenja i o mogućnosti prekoračenja.} 

\begin{minitest}
\begin{upotreba}{1}
#\naslovInt#
#\izlaz{Unesite x i n:}\ulaz{5.6 8}#
#\izlaz{S = 0.779792}#
\end{upotreba}
\end{minitest}
\begin{minitest}
\begin{upotreba}{2}
#\naslovInt#
#\izlaz{Unesite x i n:}\ulaz{14.32 11}#
#\izlaz{S = -6714.066406}#
\end{upotreba}
\end{minitest}
\begin{minitest}
\begin{upotreba}{3}
#\naslovInt#
#\izlaz{Unesite x i n:}\ulaz{2 -6}#
#\izlaz{Greska: neispravan unos.}#
\end{upotreba}
\end{minitest}

\linkresenje{PET_46}
\end{Exercise}
\ifresenja
\begin{Answer}[ref=PET_46]
\includecode{resenja/2_KontrolaToka/1.3_Petlje/1.3_46.c}
\end{Answer}
\fi


\begin{Exercise}[difficulty=1, label=PET_47] 
Napisati program koji učitava pozitivan ceo broj $n$ i
koji računa proizvod
$$P = (1 + \frac{1}{2!})(1 + \frac{1}{3!})\ldots(1 +
\frac{1}{n!}).$$ 
U slučaju neispravnog unosa, ispisati odgovarajuću poruku o grešci.
\napomena{Voditi računa o efikasnosti rešenja i o mogućnosti prekoračenja.} 
  
\begin{miditest}
\begin{upotreba}{1}
#\naslovInt#
#\izlaz{Unesite broj n:}\ulaz{5}#
#\izlaz{P = 1.838108}#
\end{upotreba}
\end{miditest}
\begin{miditest}
\begin{upotreba}{2}
#\naslovInt#
#\izlaz{Unesite broj n:}\ulaz{7}#
#\izlaz{P = 1.841026}#
\end{upotreba}
\end{miditest}

\begin{miditest}
\begin{upotreba}{3}
#\naslovInt#
#\izlaz{Unesite broj n:}\ulaz{0}#
#\izlaz{Greska: neispravan unos.}#
\end{upotreba}
\end{miditest}
\begin{miditest}
\begin{upotreba}{4}
#\naslovInt#
#\izlaz{Unesite broj n:}\ulaz{10}#
#\izlaz{P = 1.841077}#
\end{upotreba}
\end{miditest}
\linkresenje{PET_47}
\end{Exercise}
\ifresenja
\begin{Answer}[ref=PET_47]
\includecode{resenja/2_KontrolaToka/1.3_Petlje/1.3_47.c}
\end{Answer}
\fi


\begin{Exercise}[difficulty=1, label=PET_48] 
Napisati program koji učitava neparan ceo broj $n$ ($n\geq5$) i izračunava
i ispisuje sumu 
$$S = 1\cdot3\cdot5 - 1\cdot3\cdot5\cdot7 + 1\cdot3\cdot5\cdot7\cdot9
- 1\cdot3\cdot5\cdot7\cdot9\cdot11 + \ldots
(-1)^{\frac{n-1}{2}+1}\cdot1\cdot3\cdot \ldots \cdot n.$$ U slučaju
greške pri unosu podataka ispisati odgovarajuću poruku. 
\napomena{Voditi računa o efikasnosti rešenja i o
  mogućnosti prekoračenja.} 
  
\begin{miditest}
\begin{upotreba}{1}
#\naslovInt#
#\izlaz{Unesite broj n:}\ulaz{9}#
#\izlaz{S = 855}#
\end{upotreba}
\end{miditest}
\begin{miditest}
\begin{upotreba}{2}
#\naslovInt#
#\izlaz{Unesite broj n:}\ulaz{11}#
#\izlaz{S = -9540}#
\end{upotreba}
\end{miditest}

\begin{miditest}
\begin{upotreba}{3}
#\naslovInt#
#\izlaz{Unesite broj n:}\ulaz{20}#
#\izlaz{Greska: neispravan unos.}#
\end{upotreba}
\end{miditest}
\begin{miditest}
\begin{upotreba}{4}
#\naslovInt#
#\izlaz{Unesite broj n:}\ulaz{-3}#
#\izlaz{Greska: neispravan unos.}#
\end{upotreba}
\end{miditest}
\linkresenje{PET_48}
\end{Exercise}
\ifresenja
\begin{Answer}[ref=PET_48]
\includecode{resenja/2_KontrolaToka/1.3_Petlje/1.3_48.c}
\end{Answer}
\fi


\begin{Exercise}[label=PET_49] 
Napisati program koji učitava realne brojeve $x$ i $a$ i pozitivan ceo broj $n$ i
zatim izračunava i ispisuje vrednost izraza
 $$((\ldots \underbrace{(((x+a)^2 + a)^2 + a)^2 + \ldots a)^2}_n.$$
U slučaju neispravnog unosa, ispisati odgovarajuću poruku o grešci.

\begin{miditest}
\begin{upotreba}{1}
#\naslovInt#
#\izlaz{Unesite brojeve x i a:}\ulaz{3.2 0.2}#
#\izlaz{Unesite broj n:}\ulaz{5}#
#\izlaz{Izraz = 135380494030332048.000000}#
\end{upotreba}
\end{miditest}
\begin{miditest}
\begin{upotreba}{2}
#\naslovInt#
#\izlaz{Unesite brojeve x i a:}\ulaz{2 1}#
#\izlaz{Unesite broj n:}\ulaz{3}#
#\izlaz{Izraz = 10201.000000}#
\end{upotreba}
\end{miditest}

\begin{miditest}
\begin{upotreba}{3}
#\naslovInt#
#\izlaz{Unesite brojeve x i a:}\ulaz{2.6 0.3}#
#\izlaz{Unesite broj n:}\ulaz{3}#
#\izlaz{Izraz = 5800.970129}#
\end{upotreba}
\end{miditest}
\begin{miditest}
\begin{upotreba}{4}
#\naslovInt#
#\izlaz{Unesite brojeve x i a:}\ulaz{5.4 7}#
#\izlaz{Unesite broj n:}\ulaz{-2}#
#\izlaz{Greska: neispravan unos.}#
\end{upotreba}
\end{miditest}
\linkresenje{PET_49}
\end{Exercise}
\ifresenja
\begin{Answer}[ref=PET_49]
\includecode{resenja/2_KontrolaToka/1.3_Petlje/1.3_49.c}
\end{Answer}
\fi


%--------------------------------------------------------------------
%--------------------------------------------------------------------
% \subsection{Dvostruka petlja i ispisivanje slike}
%--------------------------------------------------------------------
%--------------------------------------------------------------------
\begin{Exercise}[label=PET_50] 
Napisati programe koji za unetu pozitivnu celobrojnu vrednost $n$ ispisuju
tražene tablice. \napomena{Pretpostaviti da je unos ispravan.}


\begin{enumerate}
\item Napisati program koji za unetu vrednost $n$ ispisuje tablicu množenja. 

\begin{minitest}
\begin{upotreba}{1}
#\naslovInt#
#\izlaz{Unesite broj n:}\ulaz{1}#
#\izlaz{1}#
\end{upotreba}
\end{minitest}
\begin{minitest}
\begin{upotreba}{2}
#\naslovInt#
#\izlaz{Unesite broj n:}\ulaz{2}#
#\izlaz{1 \ \ 2}#
#\izlaz{2 \ \ 4}#
\end{upotreba}
\end{minitest}
\begin{minitest}
\begin{upotreba}{3}
#\naslovInt#
#\izlaz{Unesite broj n:}\ulaz{4}#
#\izlaz{1 \ \ 2 \ \ 3 \ \ 4 }#
#\izlaz{2 \ \ 4 \ \ 6 \ \ 8 }#
#\izlaz{3 \ \ 6 \ \ 9 \ \ 12}#
#\izlaz{4 \ \ 8 \ \ 12 \ 16}#
\end{upotreba}
\end{minitest}

%\linkresenje{1.3_50a}


\item Napisati program koji za uneto $n$ ispisuje sve brojeve od 1 do $n^2$ pri čemu se ispisuje po $n$ brojeva u jednoj vrsti.

\begin{miditest}
\begin{upotreba}{1}
#\naslovInt#
#\izlaz{Unesite broj n:}\ulaz{3}#
#\izlaz{1 \ \ 2 \ \ 3 }#
#\izlaz{4 \ \ 5 \ \ 6 }#
#\izlaz{7 \ \ 8 \ \ 9}#
\end{upotreba}
\end{miditest}
\begin{miditest}
\begin{upotreba}{2}
#\naslovInt#
#\izlaz{Unesite broj n:}\ulaz{4}#
#\izlaz{1 \ \ 2 \ \ 3 \ \ 4 }#
#\izlaz{5 \ \ 6 \ \ 7 \ \ 8}#
#\izlaz{9 \ \ 10 \ 11 \ 12}#
#\izlaz{13 \ 14 \ 15 \ 16}#
\end{upotreba}
\end{miditest}

\item Napisati program koji za uneto $n$ ispisuje tablicu brojeva tako 
da su u prvoj vrsti svi brojevi od $1$ do $n$, a svaka naredna vrsta 
dobija se rotiranjem prethodne vrste za jedno mesto u levo. 

\begin{miditest}
\begin{upotreba}{1}
#\naslovInt#
#\izlaz{Unesite broj n:}\ulaz{3}#
#\izlaz{1 2 3 }#
#\izlaz{2 3 1 }#
#\izlaz{3 1 2}#
\end{upotreba}
\end{miditest}
\begin{miditest}
\begin{upotreba}{2}
#\naslovInt#
#\izlaz{Unesite broj n:}\ulaz{4}#
#\izlaz{1 2 3 4 }#
#\izlaz{2 3 4 1}#
#\izlaz{3 4 1 2}#
#\izlaz{4 1 2 3 }#
\end{upotreba}
\end{miditest}


\item Napisati program koji za uneto
$n$ iscrtava pravougli ,,trougao'' sačinjen od ,,koordinata'' svojih
tačaka. ,,Koordinata'' tačke je oblika $(i,j)$ pri čemu $i,\ j = 0,
\ldots, n$. Prav ugao se nalazi u gornjem levom uglu slike i njegova
koordinata je $(0, 0)$. Koordinata $i$ se uvećava po vrsti, a
koordinata $j$ po koloni, pa je zato koordinata tačke koja je ispod
tačke $(0,0)$ jednaka $(1, 0)$, a koordinata tačke koja je desno od
tačke $(0,0)$ jednaka $(0,1)$.

\begin{minitest}
\begin{upotreba}{1}
#\naslovInt#
#\izlaz{Unesite broj n:}\ulaz{1}#
#\izlaz{(0,0)}#
\end{upotreba}
\end{minitest}
\begin{minitest}
\begin{upotreba}{2}
#\naslovInt#
#\izlaz{Unesite broj n:}\ulaz{2}#
#\izlaz{(0,0) (0,1)}#
#\izlaz{(1,0)}#
\end{upotreba}
\end{minitest}
\begin{minitest}
\begin{upotreba}{3}
#\naslovInt#
#\izlaz{Unesite broj n:}\ulaz{4}#
#\izlaz{(0,0) (0,1) (0,2) (0,3)}#
#\izlaz{(1,0) (1,1) (1,2)}#
#\izlaz{(2,0) (2,1)}#
#\izlaz{(3,0)}#
\end{upotreba}
\end{minitest}
%\linkresenje{1.3_50d}
\end{enumerate}
\end{Exercise}

\linkresenje{PET_50}

\ifresenja
\begin{Answer}[ref=PET_50]
\includecodeLib{resenja/2_KontrolaToka/1.3_Petlje/1.3_50a.c}{Rešenje (a)}
\includecodeLib{resenja/2_KontrolaToka/1.3_Petlje/1.3_50b.c}{Rešenje (b)}
\includecodeLib{resenja/2_KontrolaToka/1.3_Petlje/1.3_50c.c}{Rešenje (c)}
\includecodeLib{resenja/2_KontrolaToka/1.3_Petlje/1.3_50d.c}{Rešenje (d)}
\end{Answer}
\fi

\begin{Exercise}[label=PET_51] 
Napisati program koji za uneti pozitivan ceo broj $n$ zvezdicama iscrtava
odgovarajuću sliku. \napomena{Pretpostaviti da je unos ispravan.}

\begin{enumerate}
\item Slika predstavlja kvadrat stranice $n$ sastavljen od zvezdica. 

\begin{miditest}
\begin{upotreba}{1}
#\naslovInt#
#\izlaz{Unesite broj n:}\ulaz{3}#
#\izlaz{***}#
#\izlaz{***}#
#\izlaz{***}#
\end{upotreba}
\end{miditest}
\begin{miditest}
\begin{upotreba}{2}
#\naslovInt#
#\izlaz{Unesite broj n:}\ulaz{4}#
#\izlaz{****}#
#\izlaz{****}#
#\izlaz{****}#
#\izlaz{****}#
\end{upotreba}
\end{miditest}
%\linkresenje{1.3_51a}


\item Slika predstavlja rub kvadrata dimenzije $n$. 

\begin{miditest}
\begin{upotreba}{1}
#\naslovInt#
#\izlaz{Unesite broj n:}\ulaz{5}#
#\izlaz{*****}#
#\izlaz{*\ \ \ *}#
#\izlaz{*\ \ \ *}#
#\izlaz{*\ \ \ *}#
#\izlaz{*****}#
\end{upotreba}
\end{miditest}
\begin{miditest}
\begin{upotreba}{2}
#\naslovInt#
#\izlaz{Unesite broj n:}\ulaz{2}#
#\izlaz{**}#
#\izlaz{**}#
\end{upotreba}
\end{miditest}
%\linkresenje{1.3_51b}


\item Slika predstavlja rub kvadrata dimenzije $n$ koji i na glavnoj dijagonali ima
  zvezdice.
  
\begin{miditest}
\begin{upotreba}{1}
#\naslovInt#
#\izlaz{Unesite broj n:}\ulaz{5}#
#\izlaz{*****}#
#\izlaz{**\ \ *}#
#\izlaz{*\ *\ *}#
#\izlaz{*\ \ **}#
#\izlaz{*****}#
\end{upotreba}
\end{miditest}
\begin{miditest}
\begin{upotreba}{1}
#\naslovInt#
#\izlaz{Unesite broj n:}\ulaz{4}#
#\izlaz{****}#
#\izlaz{**\ *}#
#\izlaz{*\ **}#
#\izlaz{****}#
\end{upotreba}
\end{miditest}
\end{enumerate}
\linkresenje{PET_51}
\end{Exercise}
\ifresenja
\begin{Answer}[ref=PET_51]
\includecodeLib{resenja/2_KontrolaToka/1.3_Petlje/1.3_51a.c}{Rešenje (a)}
\includecodeLib{resenja/2_KontrolaToka/1.3_Petlje/1.3_51b.c}{Rešenje (b)}
\includecodeLib{resenja/2_KontrolaToka/1.3_Petlje/1.3_51c.c}{Rešenje (c)}
\end{Answer}
\fi

\begin{Exercise}[difficulty=1, label=PET_52]
 Napisati program koji za uneti pozitivan ceo broj $n$ zvezdicama iscrtava
 slovo \textit{X} dimenzije $n$. \napomena{Pretpostaviti da je unos ispravan.}


\begin{miditest}
\begin{upotreba}{1}
#\naslovInt#
#\izlaz{Unesite broj n:}\ulaz{5}#
#\izlaz{*\ \ \ *}#
#\izlaz{\ *\ *\ }#
#\izlaz{\ \ *\ \ }#
#\izlaz{\ *\ *\ }#
#\izlaz{*\ \ \ *}#
\end{upotreba}
\end{miditest}
\begin{miditest}
\begin{upotreba}{2}
#\naslovInt#
#\izlaz{Unesite broj n:}\ulaz{3}#
#\izlaz{*\ *}#
#\izlaz{\ *\ }#
#\izlaz{*\ *}#
\end{upotreba}
\end{miditest}
\linkresenje{PET_52}
\end{Exercise}
\ifresenja
\begin{Answer}[ref=PET_52]
\includecode{resenja/2_KontrolaToka/1.3_Petlje/1.3_52.c}
\end{Answer}
\fi


\begin{Exercise}[difficulty=1, label=PET_53]
 Napisati program koji za uneti neparan pozitivan broj $n$ korišćenjem
 znaka $+$ iscrtava veliko $+$ dimenzije $n$. 
 \napomena{Pretpostaviti da je unos ispravan.}
 
\begin{miditest}
\begin{upotreba}{1}
#\naslovInt#
#\izlaz{Unesite broj n:}\ulaz{5}#
#\izlaz{\ \ +}#
#\izlaz{\ \ +}#
#\izlaz{+++++}#
#\izlaz{\ \ +}#
#\izlaz{\ \ +}#
\end{upotreba}
\end{miditest}
\begin{miditest}
\begin{upotreba}{2}
#\naslovInt#
#\izlaz{Unesite broj n:}\ulaz{3}#
#\izlaz{\ +}#
#\izlaz{+++}#
#\izlaz{\ +}#
\end{upotreba}
\end{miditest}

\linkresenje{PET_53}
\end{Exercise}
\ifresenja
\begin{Answer}[ref=PET_53]
\includecode{resenja/2_KontrolaToka/1.3_Petlje/1.3_53.c}
\end{Answer}
\fi


\begin{Exercise}[label=PET_54] 
Napisati program koji učitava pozitivan ceo broj $n$, a potom iscrtava
odgovarajuću sliku. \napomena{Pretpostaviti da je unos ispravan.}

\begin{enumerate}
\item Slika predstavlja pravougli trougao sastavljen od zvezdica. Kateta
  trougla je dužine $n$, a prav ugao se nalazi u gornjem levom uglu
  slike.

\begin{miditest}
\begin{upotreba}{1}
#\naslovInt#
#\izlaz{Unesite broj n:}\ulaz{3}#
#\izlaz{***}#
#\izlaz{**}#
#\izlaz{*}#
\end{upotreba}
\end{miditest}
\begin{miditest}
\begin{upotreba}{1}
#\naslovInt#
#\izlaz{Unesite broj n:}\ulaz{4}#
#\izlaz{****}#
#\izlaz{***}#
#\izlaz{**}#
#\izlaz{*}#
\end{upotreba}
\end{miditest}
%\linkresenje{1.3_54a}

\item  Slika predstavlja pravougli trougao sastavljen od zvezdica. Kateta trougla je
  dužine $n$, a prav ugao se nalazi u donjem levom uglu slike. 

\begin{miditest}
\begin{upotreba}{1}
#\naslovInt#
#\izlaz{Unesite broj n:}\ulaz{3}#
#\izlaz{*}#
#\izlaz{**}#
#\izlaz{***}#
\end{upotreba}
\end{miditest}
\begin{miditest}
\begin{upotreba}{2}
#\naslovInt#
#\izlaz{Unesite broj n:}\ulaz{4}#
#\izlaz{*}#
#\izlaz{**}#
#\izlaz{***}#
#\izlaz{****}#
\end{upotreba}
\end{miditest}
%\linkresenje{1.3_54b}

\item  Slika predstavlja pravougli trougao sastavljen od zvezdica. Kateta trougla je
  dužine $n$, a prav ugao se nalazi u gornjem desnom uglu slike. 

\begin{miditest}
\begin{upotreba}{1}
#\naslovInt#
#\izlaz{Unesite broj n:}\ulaz{3}#
#\izlaz{***}#
#\izlaz{\ **}#
#\izlaz{\ \ *}#
\end{upotreba}
\end{miditest}
\begin{miditest}
\begin{upotreba}{1}
#\naslovInt#
#\izlaz{Unesite broj n:}\ulaz{4}#
#\izlaz{****}#
#\izlaz{\ ***}#
#\izlaz{\ \ **}#
#\izlaz{\ \ \ *}#
\end{upotreba}
\end{miditest}
%\linkresenje{1.3_54c}

\item  Slika predstavlja pravougli trougao sastavljen od zvezdica. Kateta trougla je
  dužine $n$, a prav ugao se nalazi u donjem desnom uglu slike. 

\begin{miditest}
\begin{upotreba}{1}
#\naslovInt#
#\izlaz{Unesite broj n:}\ulaz{3}#
#\izlaz{\ \ *}#
#\izlaz{\ **}#
#\izlaz{***}#
\end{upotreba}
\end{miditest}
\begin{miditest}
\begin{upotreba}{2}
#\naslovInt#
#\izlaz{Unesite broj n:}\ulaz{4}#
#\izlaz{\ \ \ *}#
#\izlaz{\ \ **}#
#\izlaz{\ ***}#
#\izlaz{****}#
\end{upotreba}
\end{miditest}
%\linkresenje{1.3_54d}

\item
 Slika predstavlja trougao sastavljen od zvezdica. Trougao se dobija spajanjem
  dva pravougla trougla kateta dužine $n$, pri čemu je prav
  ugao prvog trougla u njegovom donjem levom uglu, dok je prav ugao
  drugog trougla u njegovom gornjem levom uglu, a spajanje se vrši po
  horiznotalnoj kateti. 
  
\begin{miditest}
\begin{upotreba}{1}
#\naslovInt#
#\izlaz{Unesite broj n:}\ulaz{3}#
#\izlaz{*}#
#\izlaz{**}#
#\izlaz{***}#
#\izlaz{**}#
#\izlaz{*}#
\end{upotreba}
\end{miditest}
\begin{miditest}
\begin{upotreba}{2}
#\naslovInt#
#\izlaz{Unesite broj n:}\ulaz{4}#
#\izlaz{*}#
#\izlaz{**}#
#\izlaz{***}#
#\izlaz{****}#
#\izlaz{***}#
#\izlaz{**}#
#\izlaz{*}#
\end{upotreba}
\end{miditest}
%\linkresenje{1.3_54e}

\item Slika predstavlja rub jednakokrakog pravouglog trougla čije su katete dužine
  $n$. Program učitava karakter $c$ i taj karakter koristi za
  iscrtavanje ruba trougla. 
  
\begin{miditest}
\begin{upotreba}{1}
#\naslovInt#
#\izlaz{Unesite broj n:}\ulaz{4}#
#\izlaz{Unesite karakter c:}\ulaz{*}#
#\izlaz{*}#
#\izlaz{**}#
#\izlaz{*\ *}#
#\izlaz{****}#
\end{upotreba}
\end{miditest}
\begin{miditest}
\begin{upotreba}{2}
#\naslovInt#
#\izlaz{Unesite broj n:}\ulaz{5}#
#\izlaz{Unesite karakter c:}\ulaz{+}#
#\izlaz{+}#
#\izlaz{++}#
#\izlaz{+\ +}#
#\izlaz{+\ \ +}#
#\izlaz{+++++}#
\end{upotreba}
\end{miditest}
\end{enumerate}
\linkresenje{PET_54}
\end{Exercise}
\ifresenja
\begin{Answer}[ref=PET_54]
\includecodeLib{resenja/2_KontrolaToka/1.3_Petlje/1.3_54a.c}{Rešenje (a)}
\includecodeLib{resenja/2_KontrolaToka/1.3_Petlje/1.3_54b.c}{Rešenje (b)}
\includecodeLib{resenja/2_KontrolaToka/1.3_Petlje/1.3_54c.c}{Rešenje (c)}
\includecodeLib{resenja/2_KontrolaToka/1.3_Petlje/1.3_54d.c}{Rešenje (d)}
\includecodeLib{resenja/2_KontrolaToka/1.3_Petlje/1.3_54e.c}{Rešenje (e)}
\includecodeLib{resenja/2_KontrolaToka/1.3_Petlje/1.3_54f.c}{Rešenje (f)}
\end{Answer}
\fi


\begin{Exercise}[label=PET_55] 
Napisati program koji učitava pozitivan ceo broj $n$, a potom iscrtava odgovarajuću sliku.
\napomena{Pretpostaviti da je unos ispravan.}

\begin{enumerate}
\item  Slika predstavlja jednakostranični trougao stranice $n$ koji je sastavljen od
  zvezdica.  
  
\begin{miditest}
\begin{upotreba}{1}
#\naslovInt#
#\izlaz{Unesite broj n:}\ulaz{3}#
#\izlaz{\ \ *}#
#\izlaz{\ ***}#
#\izlaz{*****}#
\end{upotreba}
\end{miditest}
\begin{miditest}
\begin{upotreba}{2}
#\naslovInt#
#\izlaz{Unesite broj n:}\ulaz{4}#
#\izlaz{\ \ \ *}#
#\izlaz{\ \ ***}#
#\izlaz{\ *****}#
#\izlaz{*******}#
\end{upotreba}
\end{miditest}
%\linkresenje{1.3_55a}

\item  Slika predstavlja jednakostranični trougao stranice $n$ koji je sastavljen od
  zvezdica pri čemu je vrh trougla na dnu slike.  
  
\begin{miditest}
\begin{upotreba}{1}
#\naslovInt#
#\izlaz{Unesite broj n:}\ulaz{3}#
#\izlaz{*****}#
#\izlaz{\ ***}#
#\izlaz{\ \ *}#
\end{upotreba}
\end{miditest}
\begin{miditest}
\begin{upotreba}{2}
#\naslovInt#
#\izlaz{Unesite broj n:}\ulaz{4}#
#\izlaz{*******}#
#\izlaz{\ *****}#
#\izlaz{\ \ ***}#
#\izlaz{\ \ \ *}#
\end{upotreba}
\end{miditest}
%\linkresenje{1.3_55b}

\item Slika predstavlja trougao koji se dobija spajanjem dva jednakostranična
  trougla stranice $n$ koji su sastavljeni od zvezdica. 
  
\begin{miditest}
\begin{upotreba}{1}
#\naslovInt#
#\izlaz{Unesite broj n:}\ulaz{3}#
#\izlaz{\ \ *}#
#\izlaz{\ ***}#
#\izlaz{*****}#
#\izlaz{\ ***}#
#\izlaz{\ \ *}#
\end{upotreba}
\end{miditest}
\begin{miditest}
\begin{upotreba}{2}
#\naslovInt#
#\izlaz{Unesite broj n:}\ulaz{5}#
#\izlaz{\ \ \ \ *}#
#\izlaz{\ \ \ ***}#
#\izlaz{\ \ *****}#
#\izlaz{\ *******}#
#\izlaz{*********}#
#\izlaz{\ *******}#
#\izlaz{\ \ *****}#
#\izlaz{\ \ \ ***}#
#\izlaz{\ \ \ \ *}#
\end{upotreba}
\end{miditest}
%\linkresenje{1.3_55c}

\item Slika predstavlja rub jednakostraničnog trougla čija stranica je dužine $n$. 

\begin{miditest}
\begin{upotreba}{1}
#\naslovInt#
#\izlaz{Unesite broj n:}\ulaz{3}#
#\izlaz{\ \ *}#
#\izlaz{\ *\ *}#
#\izlaz{*\ *\ *}#
\end{upotreba}
\end{miditest}
\begin{miditest}
\begin{upotreba}{1}
#\naslovInt#
#\izlaz{Unesite broj n:}\ulaz{5}#
#\izlaz{\ \ \ \ *}#
#\izlaz{\ \ \ *\ *}#
#\izlaz{\ \ *\ \ \ *}#
#\izlaz{\ *\ \ \ \ \ *}#
#\izlaz{*\ *\ *\ *\ *}#
\end{upotreba}
\end{miditest}
%\linkresenje{1.3_55d}

\item  Slika se dobija spajanjem dva jednakostranična trougla
  čija stranica je dužine $n$. Iscrtavati samo rub trouglova.
  
\begin{miditest}
\begin{upotreba}{1}
#\naslovInt#
#\izlaz{Unesite broj n:}\ulaz{3}#
#\izlaz{\ \ *}#
#\izlaz{\ *\ *}#
#\izlaz{*\ *\ *}#
#\izlaz{\ *\ *}#
#\izlaz{\ \ *}#
\end{upotreba}
\end{miditest}
\begin{miditest}
\begin{upotreba}{2}
#\naslovInt#
#\izlaz{Unesite broj n:}\ulaz{5}#
#\izlaz{\ \ \ \ *}#
#\izlaz{\ \ \ *\ *}#
#\izlaz{\ \ *\ \ \ *}#
#\izlaz{\ *\ \ \ \ \ *}#
#\izlaz{*\ *\ *\ *\ *}#
#\izlaz{\ *\ \ \ \ \ *}#
#\izlaz{\ \ *\ \ \ *}#
#\izlaz{\ \ \ *\ *}#
#\izlaz{\ \ \ \ *}#
\end{upotreba}
\end{miditest}
\end{enumerate}
%\linkresenje{1.3_55e}
\end{Exercise}

\ifresenja
\begin{Answer}[ref=PET_55]
\includecodeLib{resenja/2_KontrolaToka/1.3_Petlje/1.3_55a.c}{Rešenje (a)}
\includecodeLib{resenja/2_KontrolaToka/1.3_Petlje/1.3_55b.c}{Rešenje (b)}
\includecodeLib{resenja/2_KontrolaToka/1.3_Petlje/1.3_55c.c}{Rešenje (c)}
\includecodeLib{resenja/2_KontrolaToka/1.3_Petlje/1.3_55d.c}{Rešenje (d)}
\includecodeLib{resenja/2_KontrolaToka/1.3_Petlje/1.3_55e.c}{Rešenje (c)}
\end{Answer}
\fi



\begin{Exercise}[difficulty=1, label=PET_56] 
 Napisati program koji za uneti pozitivan ceo broj $n$ iscrtava strelice
 dimenzije $n$. \napomena{Pretpostaviti da je unos ispravan.}

 
\begin{miditest}
\begin{upotreba}{1}
#\naslovInt#
#\izlaz{Unesite broj n:}\ulaz{3}#
#\izlaz{*}#
#\izlaz{\ *}#
#\izlaz{***}#
#\izlaz{\ *}#
#\izlaz{*}#
\end{upotreba}
\end{miditest}
\begin{miditest}
\begin{upotreba}{2}
#\naslovInt#
#\izlaz{Unesite broj n:}\ulaz{5}#
#\izlaz{*}#
#\izlaz{\ *}#
#\izlaz{\ \ *}#
#\izlaz{\ \ \ *}#
#\izlaz{*****}#
#\izlaz{\ \ \ *}#
#\izlaz{\ \ *}#
#\izlaz{\ *}#
#\izlaz{*}#
\end{upotreba}
\end{miditest} 
\linkresenje{PET_56}
\end{Exercise}
\ifresenja
\begin{Answer}[ref=PET_56]
\includecode{resenja/2_KontrolaToka/1.3_Petlje/1.3_56.c}
\end{Answer}
\fi

\begin{Exercise}[difficulty=1, label=PET_57] 
Napisati program koji učitava pozitivan ceo broj $n$ i iscrtava sliku koja se
dobija na sledeći način: u prvom redu je jedna zvezdica, u drugom redu
su dve zvezdice razdvojene razmakom, treći red je sastavljen od
zvezdica i iste je dužine kao i drugi red, četvrti red se sastoji od
tri zvezdice razdvojene razmakom, a peti red je sastavljen od zvezdica
i iste je dužine kao i četvrti red itd. Ukupna visina slike je $n$.
\napomena{Pretpostaviti da je unos ispravan.}


\begin{miditest}
\begin{upotreba}{1}
#\naslovInt#
#\izlaz{Unesite broj n:}\ulaz{7}#
#\izlaz{*}#
#\izlaz{*\ *}#
#\izlaz{***}#
#\izlaz{*\ *\ *}#
#\izlaz{*****}#
#\izlaz{*\ *\ *\ *}#
#\izlaz{*******}#
\end{upotreba}
\end{miditest}
\linkresenje{PET_57}
\end{Exercise}
\ifresenja
\begin{Answer}[ref=PET_57]
\includecode{resenja/2_KontrolaToka/1.3_Petlje/1.3_57.c}
\end{Answer}
\fi

\begin{Exercise}[difficulty=1, label=PET_58] 
Napisati program koji učitava pozitivne cele brojeve $m$ i $n$ i
iscrtava jedan do drugog $n$ kvadrata čija je
svaka strana sastavljena od $m$ zvezdica razdvojenih prazninom.
\napomena{Pretpostaviti da je unos ispravan.}

\begin{miditest}
\begin{upotreba}{1}
#\naslovInt#
#\izlaz{Unesite brojeve n i m:}\ulaz{5 3}#
#\izlaz{*\ *\ *\ *\ *\ *\ *\ *\ *\ *\ *\ *\ *}#         
#\izlaz{*\ \ \ \ \ \ \ *\ \ \ \ \ \ \ *\ \ \ \ \ \ \ *}#           
#\izlaz{*\ \ \ \ \ \ \ *\ \ \ \ \ \ \ *\ \ \ \ \ \ \ *}#             
#\izlaz{*\ \ \ \ \ \ \ *\ \ \ \ \ \ \ *\ \ \ \ \ \ \ *}#
#\izlaz{*\ *\ *\ *\ *\ *\ *\ *\ *\ *\ *\ *\ *}#
\end{upotreba}
\end{miditest}
\begin{miditest}
\begin{upotreba}{2}
#\naslovInt#
#\izlaz{Unesite brojeve n i m:}\ulaz{4 4}#
#\izlaz{*\ *\ *\ *\ *\ *\ *\ *\ *\ *\ *\ *\ *}#
#\izlaz{*\ \ \ \ \ *\ \ \ \ \ *\ \ \ \ \ *\ \ \ \ \ *}#
#\izlaz{*\ \ \ \ \ *\ \ \ \ \ *\ \ \ \ \ *\ \ \ \ \ *}#
#\izlaz{*\ *\ *\ *\ *\ *\ *\ *\ *\ *\ *\ *\ *}#
\end{upotreba}
\end{miditest}
\linkresenje{PET_58}
\end{Exercise}
\ifresenja
\begin{Answer}[ref=PET_58]
\includecode{resenja/2_KontrolaToka/1.3_Petlje/1.3_58.c}
\end{Answer}
\fi

\begin{Exercise}[difficulty=1, label=PET_59] 
Napisati program koji učitava pozitivan ceo broj $n$ i iscrtava romb
sastavljen od minusa u pravougaoniku sastavljenom od zvezdica.
\napomena{Pretpostaviti da je unos ispravan.}


\begin{miditest}
\begin{upotreba}{1}
#\naslovInt#
#\izlaz{Unesite broj n:}\ulaz{6}#
#\izlaz{************}#
#\izlaz{*****--*****}#
#\izlaz{****----****}#
#\izlaz{***------***}#
#\izlaz{**--------**}#
#\izlaz{*----------*}#
#\izlaz{**--------**}#
#\izlaz{***------***}#
#\izlaz{****----****}#
#\izlaz{*****--*****}#
#\izlaz{************}#
\end{upotreba}
\end{miditest}
\begin{miditest}
\begin{upotreba}{2}
#\naslovInt#
#\izlaz{Unesite broj n:}\ulaz{2}#
#\izlaz{****}#
#\izlaz{*--*}#
#\izlaz{****}#
\end{upotreba}
\end{miditest}
\linkresenje{PET_59}
\end{Exercise}
\ifresenja
\begin{Answer}[ref=PET_59]
\includecode{resenja/2_KontrolaToka/1.3_Petlje/1.3_59.c}
\end{Answer}
\fi

\begin{Exercise}[label=PET_60] 
Napisati program koji učitava ceo broj $n$ ($n \geq 2$) i koji
iscrtava sliku kuće sa krovom: kuća je kvadrat stranice $n$, a krov
jednakostranični trougao stranice $n$. Pretpostaviti da je unos
korektan.

\begin{miditest}
\begin{upotreba}{1}
#\naslovInt#
#\izlaz{Unesite broj n:}\ulaz{4}#
#\izlaz{\ \ \ *}#
#\izlaz{\ \ *\ *}#
#\izlaz{\ *\ \ \ *}#
#\izlaz{*\ *\ *\ *}#
#\izlaz{*\ \ \ \ \ *}#
#\izlaz{*\ \ \ \ \ *}#
#\izlaz{*\ *\ *\ *}#
\end{upotreba}
\end{miditest}
\begin{miditest}
\begin{upotreba}{2}
#\naslovInt#
#\izlaz{Unesite broj n:}\ulaz{3}#
#\izlaz{\ \ *}#
#\izlaz{\ *\ *}#
#\izlaz{*\ *\ *}#
#\izlaz{*\ \ \ *}#
#\izlaz{*\ *\ *}#
\end{upotreba}
\end{miditest}

\linkresenje{PET_60}
\end{Exercise}
\ifresenja
\begin{Answer}[ref=PET_60]
\includecode{resenja/2_KontrolaToka/1.3_Petlje/1.3_60.c}
\end{Answer}
\fi


\begin{Exercise}[difficulty=1, label=PET_61] 
Napisati program koji učitava pozitivan ceo broj $n$ i ispisuje
brojeve od $1$ do $n$, zatim od $2$ do $n-1$, $3$ do $n-2$, itd. Ispis
se završava kada nije moguće ispisati ni jedan broj.
\napomena{Pretpostaviti da je unos ispravan.}

\begin{miditest}
\begin{upotreba}{1}
#\naslovInt#
#\izlaz{Unesite broj n:}\ulaz{5}#
#\izlaz{1 2 3 4 5 2 3 4 3}#
\end{upotreba}
\end{miditest}
\begin{miditest}
\begin{upotreba}{2}
#\naslovInt#
#\izlaz{Unesite broj n:}\ulaz{6}#
#\izlaz{1 2 3 4 5 6 2 3 4 5 3 4}#
\end{upotreba}
\end{miditest}

\begin{miditest}
\begin{upotreba}{3}
#\naslovInt#
#\izlaz{Unesite broj n:}\ulaz{7}#
#\izlaz{1 2 3 4 5 6 7 2 3 4 5 6 3 4 5 4}#
\end{upotreba}
\end{miditest}
\begin{miditest}
\begin{upotreba}{4}
#\naslovInt#
#\izlaz{Unesite broj n:}\ulaz{3}#
#\izlaz{1 2 3 2}#
\end{upotreba}
\end{miditest}
\linkresenje{PET_61}
\end{Exercise}
\ifresenja
\begin{Answer}[ref=PET_61]
\includecode{resenja/2_KontrolaToka/1.3_Petlje/1.3_61.c}
\end{Answer}
\fi


\begin{Exercise}[difficulty=1, label=PET_62] 
Napisati program koji učitava pozitivan ceo broj $n$ i ispisuje sve
brojeve od $1$ do $n$, zatim svaki drugi broj od $1$ do $n$, zatim
svaki treći broj od $1$ do $n$ itd., završavajući sa svakim $n$-tim
(tj. samo sa $1$).
\napomena{Pretpostaviti da je unos ispravan.}

\begin{miditest}
\begin{upotreba}{1}
#\naslovInt#
#\izlaz{Unesite broj n:}\ulaz{3}#
#\izlaz{1 2 3}#
#\izlaz{1 3}#
#\izlaz{1}#
\end{upotreba}
\end{miditest}
\begin{miditest}
\begin{upotreba}{2}
#\naslovInt#
#\izlaz{Unesite broj n:}\ulaz{7}#
#\izlaz{1 2 3 4 5 6 7}#
#\izlaz{1 3 5 7}#
#\izlaz{1 4 7}#
#\izlaz{1 5}#
#\izlaz{1 6}#
#\izlaz{1 7}#
#\izlaz{1}#
\end{upotreba}
\end{miditest}

\begin{miditest}
\begin{upotreba}{3}
#\naslovInt#
#\izlaz{Unesite broj n:}\ulaz{1}#
#\izlaz{1}#
\end{upotreba}
\end{miditest}

\linkresenje{PET_62}
\end{Exercise}
\ifresenja
\begin{Answer}[ref=PET_62]
\includecode{resenja/2_KontrolaToka/1.3_Petlje/1.3_62.c}
\end{Answer}
\fi


\ifresenja
\section{Rešenja}
\shipoutAnswer
\fi




