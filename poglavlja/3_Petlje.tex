\section{Petlje}

%--------------------------------------------------------------------
%--------------------------------------------------------------------
\subsection{Ispis podataka}
%--------------------------------------------------------------------
%--------------------------------------------------------------------

\komentar{REDOSLED: Petlje se sustinski koriste za tri stvari: map, filter i reduce, kao i za kombinaciju te tri stvari.\\
Map --- preslikavanje, dakle ceo niz necega se preslikava na neki nacin u neki novi niz (dupliranje vrednosti svih elemenata niza, dupliranje svake cifre broja, dodavanje prefiksa svim recima...)\\
Filter --- iz niza necega biraju se neki koji zadovoljavaju neki kriterijum (svi parni brojevi, svi koji sadrze karakter "a", svi prosti brojevi, svi savrseni brojevi...)\\
Reduce --- ceo niz se svodi na jednu vrednost (zbir svih vrednost, proizvod svih vrednosti, nadovezane sve vrednosti...)\\
Kombinacija --- dve tehnike od prethodne tri (npr filter-reduce: zbir svih parnih brojeva) ili od svake po malo (zbir svih dupliranih brojeva koji su savrseni)\\
Sustinski, studenti treba da usvoje najpre ove tri tehnike, pa onda da idu ne njihove kombinacije, i to najpre na kombinacije dve od tri, pa na kraju na zadatke koje kombinuju sve to. Ove tehnike nisu vezane za nizove, mogu se primeniti i na prirodne brojeve posmatrane kao niz brojeva ili na prirodni broj posmatran kao niz cifara... \\
Danijela:: ok neka bude redosled koji si predlo\v zila}

\komentar{TODO Treba smisliti odgovarajuci redosled. Mislim da ce to biti lakse kada budu svi 
test primeri i resenja. \\
Danijela: Onda menjam redosled kad budem sredjivala re\v senja i test primere.}

\komentar{TODO Ima zadataka koji se posle ponavljaju u funkcijama. Resenja mozda ne bih duplirala svaki put vec bih negde dala resenje ovde, negde u funkcijama, a ukoliko nisu data oba, onda bih dala odgovarajuci link u resenju da se pogleda resenje tog drugog zadatka. Takodje, voditi racuna da su formulacije u takvim zadacima slicne, mozda cak i test primeri da se poklapaju? \\
Danijela: Ovo je osetljivo, dogovoriti se sa Jovanom kakko uskladiti takve zadatke...}

\begin{Exercise}[label=v1.3_01] 
Napisati program koji $5$ puta ispisuje tekst \kckod{Mi volimo
  da programiramo}.  
  \linkresenje{v1.3_01}
\end{Exercise}
\begin{Answer}[ref=v1.3_01]
\includecode{resenja/1_KontrolaToka/1.3_Petlje/1_01.c}
\end{Answer}

\begin{Exercise}[label=v1.3_02] 
Napisati program koji učitava pozitivan ceo broj $n$
a potom ispisuje sve cele brojeve od $0$ do $n-1$.
\linkresenje{v1.3_02}
\end{Exercise}
\begin{Answer}[ref=v1.3_02]
\includecode{resenja/1_KontrolaToka/1.3_Petlje/1_02.c}
\end{Answer}


\begin{Exercise}[label=v1.3_10] 
\komentar{Nisam sigurnada odmah treba da krenemo sa sve tri petlje. Mozda je dovoljno da prvih par zadataka koriste while?}
Napisati program koji učitava ceo broj $n$ i ispisuje $n$ puta tekst
\kckod{Mi volimo da programiramo}.
\begin{enumerate}
\item Koristiti \kckod{while} petlju.
\item Koristiti \kckod{for} petlju.
\item Koristiti \kckod{fo-while} petlju.
 \end{enumerate}
\komentar{Ne svidja mi se resenje jer je razlicitio za ove tri varijante za n 
manje od nule a trebalo bi da 
isti test primeri odgovaraju razlicitim resenjima. Zato je potrebno resenje podesiti 
tako da je stampanje uvek isto, a u komentarima objasniti zasto su i gde i kakve izmene potrebne.
Predlazem da to budu tri programa za svaki od delova a, b i c \\
Danijela: ok, kada budem sredjivala resenje, usvojicu tvoj komentar}

\linkresenje{v1.3_10}
\end{Exercise}
\begin{Answer}[ref=v1.3_10]
\includecode{resenja/1_KontrolaToka/1.3_Petlje/1_10.c}
\end{Answer}

\begin{Exercise}[label=v1.3_11] 
Napisati program koji učitava dva cela broja $n$ i $m$ ispisuje sve
cele brojeve iz intervala $[n,m]$.
\begin{enumerate}
\item Koristiti \kckod{while} petlju.
\item Koristiti \kckod{for} petlju.
\item Koristiti \kckod{fo-while} petlju.
 \end{enumerate}
\komentar{Isti komentar kao i malopre, moraju sva tri resenja da budu ispravna.}
\linkresenje{v1.3_11}
\end{Exercise}
\begin{Answer}[ref=v1.3_11]
\includecode{resenja/1_KontrolaToka/1.3_Petlje/1_11.c}
\end{Answer}


\begin{Exercise}[label=p1.7_] 
Napisati program koji učitava dva realna broja dvostruke tačnosti, $a$
i $b$, za koje važi $a<b$. Napisati program koji ispisuje vrednost
funkcije $cos(x)$ u 10 ravnomerno razmaknutih tačaka intervala $[a,
  b]$ Pri ispisu vrednosti se zaokružuju na četiri decimale. Za neispravan
unos, program ispisuje odgovarajuću poruku.

\komentar{Uskladiti test primer sa tekstom zadatka. Ovakav isti
  zadatak ali sa sinusom imamo u delu sa funkcijama, ne znam da li je
  to na neki nacin redudantno, tj da li je na oba mesta potrebno dati
  resenje. Ukoliko se odlucimo da jeste, onda treba voditi racunada
  resenja budu sustinski ista i da budu na slican nacin
  iskomentarisana tako da se jasno primeti da su isti koncepti u
  pitanju.}

\begin{miditest}
\begin{upotreba}{1}
#\naslovInt#
#\izlaz{Unesite brojeve a i b:}\ulaz{ 1 10}#
#\izlaz{0.5403   -0.4161   -0.9900   -0.6536    0.2837    0.9602    0.7539   -0.1455   -0.9111   -0.8391}#
\end{upotreba}
\end{miditest}
\begin{miditest}
\begin{upotreba}{2}
#\naslovInt#
#\izlaz{Unesite brojeve a i b:}\ulaz{0 28.274}#
#\izlaz{1.0000  -1.0000  1.0000  -1.0000  1.0000  -1.0000  1.0000  -1.0000  1.0000  -1.0000}#
\end{upotreba}
\end{miditest}

\begin{miditest}
\begin{upotreba}{3}
#\naslovInt#
#\izlaz{Unesite brojeve a i b:}\ulaz{ 1 -3}#
#\izlaz{-1}#
\end{upotreba}
\end{miditest}
\begin{miditest}
\begin{upotreba}{4}
#\naslovInt#
#\izlaz{Unesite brojeve a i b:}\ulaz{0 1}#
#\izlaz{1.0000    0.9938    0.9754    0.9450    0.9028    0.8496    0.7859    0.7125    0.6303    0.5403}#
\end{upotreba}
\end{miditest}
\linkresenje{p1.7_}
\end{Exercise}
\begin{Answer}[ref=p1.7_]
%\includecode{resenja/1_KontrolaToka/1.3_Petlje/1_14.c}
\end{Answer}


\begin{Exercise}[label=p1.4_] 
\komentar{Uskladiti formulaciju zadatka sa odgovarajućom formulacijom kod nizova.}
Fibonačijev niz počinje ciframa $1$ i $1$, a svaki član se dobija zbirom
prethodna dva. Napisati program koji učitava ceo neoznačen broj $n$ i
određuje i na standardni izlaz ispisuje $n$-ti član Fibonačijevog niza. \\
\linkresenje{p1.4_}
\end{Exercise}
\begin{Answer}[ref=p1.4_]
%\includecode{resenja/1_KontrolaToka/1.3_Petlje/1_14.c}
\end{Answer}


\begin{Exercise}[difficulty=1, label=v1.3_09] 
Niz prirodnih brojeva formira se prema sledećem pravilu:
\begin{equation*}
a_{n+1} = \left\{
\begin{array}{rl}
\frac{a_n}{2} & \text{ako je } a_n \text{ parno}\\
\frac{3\cdot a_n + 1}{2} & \text{ako je } a_n \text{ neparno}\\
\end{array} \right.
\end{equation*}
Napisati program koji za uneti početni član niza $a_0$ (ceo broj)
štampa niz brojeva sve do onog člana niza koji je jednak $1$.
\linkresenje{v1.3_09}
\end{Exercise}
\begin{Answer}[ref=v1.3_09]
\includecode{resenja/1_KontrolaToka/1.3_Petlje/1_09.c}
\end{Answer}


\begin{Exercise}[difficulty=1, label=p1.7_] 
Papir $A_0$ ima povr\v sinu 1$m^2$ i odnos stranica
  $1:\sqrt{2}$. Papir $A_1$ dobija se podelom papira $A_0$ po dužoj
  ivici. Papir $A_2$ dobija se podelom $A_1$ papira po dužoj ivici
  itd. Napisati program koji za uneti neoznačen broj $k$ ispisuje
  dimenzije papira $A_k$ u milimetrima. 
  
\begin{miditest}
\begin{upotreba}{1}
#\naslovInt#
#\izlaz{Unesite broj n:}\ulaz{4}#
#\izlaz{297 210}#
\end{upotreba}
\end{miditest}
\begin{miditest}
\begin{upotreba}{2}
#\naslovInt#
#\izlaz{Unesite broj n:}\ulaz{3}#
#\izlaz{297 420}#
\end{upotreba}
\end{miditest}

\begin{miditest}
\begin{upotreba}{3}
#\naslovInt#
#\izlaz{Unesite broj n:}\ulaz{7}#
#\izlaz{74 105}#
\end{upotreba}
\end{miditest}
\begin{miditest}
\begin{upotreba}{4}
#\naslovInt#
#\izlaz{Unesite broj n:}\ulaz{9}#
#\izlaz{37 52}#
\end{upotreba}
\end{miditest}
\linkresenje{p1.7_}
\end{Exercise}
\begin{Answer}[ref=p1.7_]
%\includecode{resenja/1_KontrolaToka/1.3_Petlje/1_14.c}
\end{Answer}


%--------------------------------------------------------------------
%--------------------------------------------------------------------
\subsection{Obrada celih brojeva, rad sa ciframa broja}
%--------------------------------------------------------------------
%--------------------------------------------------------------------

\begin{Exercise}[label=v1.3_06] 
Napisati program koji učitava ceo broj i ispisuje njegove cifre u
obrnutom poretku.
\linkresenje{v1.3_06}
\end{Exercise}
\begin{Answer}[ref=v1.3_06]
\includecode{resenja/1_KontrolaToka/1.3_Petlje/1_06.c}
\end{Answer}


\begin{Exercise}[label=v1.3_14] 
Pravi delioci celog broja su svi delioci sem jedinice i samog tog
broja.  Napisati program koji učitava ceo pozitivan broj $n$ i
ispisuje sve prave delioce unetog broja. U slučaju greške pri unosu
podataka ispisati odgovarajuću poruku.  \\
\linkresenje{v1.3_14}
\end{Exercise}
\begin{Answer}[ref=v1.3_14]
\includecode{resenja/1_KontrolaToka/1.3_Petlje/1_14.c}
\end{Answer}


\begin{Exercise}[label=p1.3_04] 
 Sa standardnog ulaza unosi se ceo neoznačen broj. Napisati program
 koji proverava i ispisuje da li se cifra 5 nalazi u njegovom zapisu.
 
\begin{miditest}
\begin{upotreba}{1}
#\naslovInt#
#\izlaz{Unesite broj:}\ulaz{1857}#
#\izlaz{Cifra 5 se nalazi u zapisu!}#
\end{upotreba}
\end{miditest}
\begin{miditest}
\begin{upotreba}{2}
#\naslovInt#
#\izlaz{Unesite broj:}\ulaz{84}#
#\izlaz{Cifra 5 se ne nalazi u zapisu!}#
\end{upotreba}
\end{miditest}
\linkresenje{p1.3_04}
\end{Exercise}
\begin{Answer}[ref=p1.3_04]
\includecode{resenja/1_KontrolaToka/1.3_Petlje/praktikumi6/3_04.c}
\end{Answer}


\begin{Exercise}[label=p1.7_] 
Sa standarnog ulaza unosi se ceo broj. Napisati program koji na
standardni izlaz ispisuje odgovor da li je uneti prirodan broj deljiv
sumom svojih cifara.\\ \linkresenje{p1.7_}
\end{Exercise}
\begin{Answer}[ref=p1.7_]
%\includecode{resenja/1_KontrolaToka/1.3_Petlje/1_14.c}
\end{Answer}


\begin{Exercise}[label=p1.3_05] 
 Napisati program koji učitava ceo neoznačen broj i uklanja sve nule
 sa desne strane unetog broja. Novodobijeni broj ispisati na
 standardni izlaz. 
 
\begin{miditest}
\begin{upotreba}{1}
#\naslovInt#
#\izlaz{Unesite broj:}\ulaz{12000}#
#\izlaz{12}#
\end{upotreba}
\end{miditest}
\begin{miditest}
\begin{upotreba}{2}
#\naslovInt#
#\izlaz{Unesite broj:}\ulaz{856}#
#\izlaz{856}#
\end{upotreba}
\end{miditest}

\begin{miditest}
\begin{upotreba}{3}
#\naslovInt#
#\izlaz{Unesite broj:}\ulaz{140}#
#\izlaz{14}#
\end{upotreba}
\end{miditest}
\linkresenje{p1.3_05}
\end{Exercise}
\begin{Answer}[ref=p1.3_05]
\includecode{resenja/1_KontrolaToka/1.3_Petlje/praktikumi6/3_05.c}
\end{Answer}


\begin{Exercise}[label=p1.3_06] 
Napisati program koji učitava neoznačeni ceo broj i transformiše ga
tako što svaku parnu cifru u zapisu broja uveća za 1. Novodobijeni
broj ispisati na standarni izlaz.

\begin{miditest}
\begin{upotreba}{1}
#\naslovInt#
#\izlaz{Unesite broj:}\ulaz{2417}#
#\izlaz{3517}#
\end{upotreba}
\end{miditest}
\begin{miditest}
\begin{upotreba}{2}
#\naslovInt#
#\izlaz{Unesite broj:}\ulaz{138}#
#\izlaz{139}#
\end{upotreba}
\end{miditest}

\begin{miditest}
\begin{upotreba}{3}
#\naslovInt#
#\izlaz{Unesite broj:}\ulaz{59}#
#\izlaz{59}#
\end{upotreba}
\end{miditest}
\linkresenje{p1.3_06}
\end{Exercise}
\begin{Answer}[ref=p1.3_06]
\includecode{resenja/1_KontrolaToka/1.3_Petlje/praktikumi6/3_06.c}
\end{Answer}


\begin{Exercise}[label=p1.3_07]
 Sa standardnog ulaza unosi se neoznačen ceo broj. Napisati program
 koji formira i ispisuje broj koji se dobija izbacivanjem svake druge
 cifre polaznog broja, počevši od krajnje desne cifre.
 
\begin{miditest}
\begin{upotreba}{1}
#\naslovInt#
#\izlaz{Unesite broj:}\ulaz{21854}#
#\izlaz{284}#
\end{upotreba}
\end{miditest}
\begin{miditest}
\begin{upotreba}{2}
#\naslovInt#
#\izlaz{Unesite broj:}\ulaz{18}#
#\izlaz{8}#
\end{upotreba}
\end{miditest}

\begin{miditest}
\begin{upotreba}{3}
#\naslovInt#
#\izlaz{Unesite broj:}\ulaz{1}#
#\izlaz{1}#
\end{upotreba}
\end{miditest}
\linkresenje{p1.3_07}
\end{Exercise}
\begin{Answer}[ref=p1.3_07]
\includecode{resenja/1_KontrolaToka/1.3_Petlje/praktikumi6/3_07.c}
\end{Answer}

\begin{Exercise}[difficulty=1, label=p1.3_14] 
Sa standardnog ulaza unosi se neoznačen ceo broj. Napisati program
koji formira i ispisuje broj koji se dobija izbacivanjem cifara koje
su jednake zbiru svojih suseda. 

\begin{miditest}
\begin{upotreba}{1}
#\naslovInt#
#\izlaz{Unesite broj:}\ulaz{28631}#
#\izlaz{2631}#
\end{upotreba}
\end{miditest}
\begin{miditest}
\begin{upotreba}{2}
#\naslovInt#
#\izlaz{Unesite broj:}\ulaz{440}#
#\izlaz{40}#
\end{upotreba}
\end{miditest}

\begin{miditest}
\begin{upotreba}{3}
#\naslovInt#
#\izlaz{Unesite broj:}\ulaz{242}#
#\izlaz{22}#
\end{upotreba}
\end{miditest}
\linkresenje{p1.3_14}
\end{Exercise}
\begin{Answer}[ref=p1.3_14]
\includecode{resenja/1_KontrolaToka/1.3_Petlje/praktikumi6/3_14.c}
\end{Answer}

\begin{Exercise}[difficulty=1, label=p1.3_15] 
Broj je \textit{palindrom} ukoliko se isto čita i sa leve i sa desne
strane. Napisati program koji učitava ceo neoznačen broj i proverava
da li je učitani broj palindrom. 

\begin{miditest}
\begin{upotreba}{1}
#\naslovInt#
#\izlaz{Unesite broj:}\ulaz{25452}#
#\izlaz{Broj je palindrom!}#
\end{upotreba}
\end{miditest}
\begin{miditest}
\begin{upotreba}{2}
#\naslovInt#
#\izlaz{Unesite broj:}\ulaz{895}#
#\izlaz{Broj nije palindrom!}#
\end{upotreba}
\end{miditest}

\begin{miditest}
\begin{upotreba}{3}
#\naslovInt#
#\izlaz{Unesite broj:}\ulaz{5}#
#\izlaz{Broj je palindrom!}#
\end{upotreba}
\end{miditest}
\linkresenje{p1.3_15}
\end{Exercise}
\begin{Answer}[ref=p1.3_15]
\includecode{resenja/1_KontrolaToka/1.3_Petlje/praktikumi6/3_15.c}
\end{Answer}


%--------------------------------------------------------------------
%--------------------------------------------------------------------
\subsection{Unos i obrada veće količine podatka (\komentar{unos i obrada niza brojeva?, nije sjajno zbog nizova}) }
%--------------------------------------------------------------------
%--------------------------------------------------------------------

\begin{Exercise}[label=v1.3_04] 
Napisati program koji učitava pozitivan ceo broj $n$, a zatim učitava
$n$ celih brojeva i na standarni izlaz ispisuje sumu pozitivnih i sumu
negativnih unetih brojeva.  
\linkresenje{v1.3_04}
\end{Exercise}
\begin{Answer}[ref=v1.3_04]
\includecode{resenja/1_KontrolaToka/1.3_Petlje/1_04.c}
\end{Answer}

\begin{Exercise}[label=p1.3_01] 
Sa standardnog ulaza unosi se ceo pozitivan broj $n$, a potom i $n$
celih brojeva. Izračunati i ispisati zbir onih brojeva koji su neparni
i negativni. 

\begin{miditest}
\begin{upotreba}{1}
#\naslovInt#
#\izlaz{Unesite broj n:}\ulaz{5}#
#\izlaz{Unesite n brojeva:}\ulaz{1 -5 -6 3 -11}#
#\izlaz{-16}#
\end{upotreba}
\end{miditest}
\begin{miditest}
\begin{upotreba}{2}
#\naslovInt#
#\izlaz{Unesite broj n:}\ulaz{4}#
#\izlaz{Unesite n brojeva:}\ulaz{-1 1 0 3}#
#\izlaz{-1}#
\end{upotreba}
\end{miditest}

\begin{miditest}
\begin{upotreba}{3}
#\naslovInt#
#\izlaz{Unesite broj n:}\ulaz{4}#
#\izlaz{Unesite n brojeva:}\ulaz{5 8 13 17}#
#\izlaz{0}#
\end{upotreba}
\end{miditest}
\linkresenje{p1.3_01}
\end{Exercise}
\begin{Answer}[ref=p1.3_01]
\includecode{resenja/1_KontrolaToka/1.3_Petlje/praktikumi6/3_01.c}
\end{Answer}


\begin{Exercise}[label=v1.3_05] 
Napisati program koji učitava cele cele brojeve sve dok se ne unese
nula. Nakon toga ispisati proizvod onih unetih brojeva koji su
pozitivni.  
\linkresenje{v1.3_05}
\end{Exercise}
\begin{Answer}[ref=v1.3_05]
\includecode{resenja/1_KontrolaToka/1.3_Petlje/1_05.c}
\end{Answer}

\begin{Exercise}[label=v1.3_12] 
U prodavnici se nalazi $n$ artikala čije cene su realni
brojevi. Napisati program koji učitava $n$, a potom i cenu svakog od
$n$ artikala i određuje i na standarni izlaz ispisuje najmanju
cenu.  
\linkresenje{v1.3_12}
\end{Exercise}
\begin{Answer}[ref=v1.3_12]
\includecode{resenja/1_KontrolaToka/1.3_Petlje/1_12.c}
\end{Answer}

\begin{Exercise}[label=p1.3_22] 
Sa standardnog ulaza se unose realni brojevi sve do unosa broja nula
$0$. Napisati program koji izračunava i ispisuje aritmetičku sredinu
unetih brojeva. 

\begin{miditest}
\begin{upotreba}{1}
#\naslovInt#
#\izlaz{Unesite brojeve:}\ulaz{8 5.2 6.11 3 0}#
#\izlaz{Aritmeticka sredina: 5.5775}#
\end{upotreba}
\end{miditest}
\linkresenje{p1.3_22}
\end{Exercise}
\begin{Answer}[ref=p1.3_22]
\includecode{resenja/1_KontrolaToka/1.3_Petlje/praktikumi7/3_22.c}
\end{Answer}


\begin{Exercise}[label=p1.3_22] 
U prodavnici se nalaze artikala čije cene su realni pozitivni
brojevi. Cene artikala se unose sa standarnog unosa sve do unosa broja
nula $0$. Napisati program koji izračunava i ispisuje prose\v cnu
vrednost cena u radnji.

\komentar{I ovo bi moglo da se preformulise u cene, tj da se sracuna prosecna vrednost cena u radnji. Cak mislim da bi mogli da stavimo dva zadatka, ovaj i jedan sa cenama, a u resenju da se pozovemo samo na resenje ovog zadatka, tako da se vidi da je to u sustini isti problem. \\
Danijela: dodat jo\v s jedan zadatak, u re\v senju se pozvati na prethodni. \\
Danijela: obratiti pa\v znju da cene mogu biti samo pozitivni brojevi, dok u prethodnom zatku nismo imali takav zahtev -- da li menjati prethodni zadatak ili dati re\v senje i za ovaj?}

\begin{miditest}
\begin{upotreba}{1}
#\naslovInt#
#\izlaz{Unesite cene:}\ulaz{8 5.2 6.11 3 0}#
#\izlaz{Aritmeticka sredina: 5.5775}#
\end{upotreba}
\end{miditest}
\linkresenje{p1.3_22}
\end{Exercise}
\begin{Answer}[ref=p1.3_22]
\includecode{resenja/1_KontrolaToka/1.3_Petlje/praktikumi7/3_22.c}
\end{Answer}




\begin{Exercise}[label=p1.3_02] 
\komentar{U narednim zadacima se u tekstu kaze da se unosi ceo pozitivan broj a posle se u resenju nigde to ne proverava, niti se koristi tip unsigned. Nesto od toga mora, inae resenje nije dobro. \\
Danijela: sredicu re\v senje u odnosu na ovaj komentar. \\
Danijela: obratiti pa\v znju na tekst i Peru -- da li nam se svidja ovako ne\v sto? Ako su cene onda mogu biti samo pozitivni brojevi?}

Pera želi da obraduje baku i da joj kupi jedan poklon u radnji. On
na raspolaganju ima $m$ novaca. U radnji se nalazi $n$ artikala i
zanima ga koliko ima artikala u radnji čija cena je manja ili
jednaka $m$. Napisati program koji pomaže Peri da brzo odrediti
broj atikala. Program učitava realan pozitivan broj $m$, ceo
neoznačen broj $n$ i $n$ realnih pozitivnih brojeva različitih
od $0$. Ispisati koliko artikala ima manju ili jednaku cenu od $m$. U
slučaju greške ispisati odgovarajuću poruku.


\begin{miditest}
\begin{upotreba}{1}
#\naslovInt#
#\izlaz{Unesite broj m:}\ulaz{12.37}#
#\izlaz{Unesite broj n:}\ulaz{5}#
#\izlaz{Unesite n brojeva:}\ulaz{11 54.13 -6 13 8}#
#\izlaz{3}#
\end{upotreba}
\end{miditest}
\begin{miditest}
\begin{upotreba}{2}
#\naslovInt#
#\izlaz{Unesite broj m:}\ulaz{2}#
#\izlaz{Unesite broj n:}\ulaz{4}#
#\izlaz{Unesite n brojeva:}\ulaz{-1 11 4.32 3}#
#\izlaz{1}#
\end{upotreba}
\end{miditest}
\linkresenje{p1.3_02}
\end{Exercise}
\begin{Answer}[ref=p1.3_02]
\includecode{resenja/1_KontrolaToka/1.3_Petlje/praktikumi6/3_02.c}
\end{Answer}


\begin{Exercise}[label=p1.7_] 
Sa standardnog ulaza unosi se ceo pozitivan broj $n$, a potom $n$
celih brojeva.  Naći sumu brojeva koji su deljivi sa $5$, a nisu
deljivi sa $7$. U slučaju greške pri unosu podataka ispisati
odgovarajuću poruku.

\komentar{Ukoliko test primer ne moze da stane u midi onda treba da bude maxi, ali mozda bolje
skratiti ga u midi. \\
Danijela: bice promenjeno kad budem sredjivala test primere.}

\begin{miditest}
\begin{upotreba}{1}
#\naslovInt#
#\izlaz{Unesite broj n:}\ulaz{5 2 35 5 -175 -20 }#
#\izlaz{-15}#
\end{upotreba}
\end{miditest}
\begin{miditest}
\begin{upotreba}{2}
#\naslovInt#
#\izlaz{Unesite broj n:}\ulaz{-3}#
#\izlaz{-1}#
\end{upotreba}
\end{miditest}

\begin{miditest}
\begin{upotreba}{3}
#\naslovInt#
#\izlaz{Unesite broj n:}\ulaz{10 -5 6 175 -20 -25 -8 42 245 1 6}#
#\izlaz{-50}#
\end{upotreba}
\end{miditest}
\begin{miditest}
\begin{upotreba}{4}
#\naslovInt#
#\izlaz{Unesite broj n:}\ulaz{6 2205 -1904 2 7 -540 5}#
#\izlaz{-535}#
\end{upotreba}
\end{miditest}
\linkresenje{p1.7_}
\end{Exercise}
\begin{Answer}[ref=p1.7_]
%\includecode{resenja/1_KontrolaToka/1.3_Petlje/1_14.c}
\end{Answer}


\begin{Exercise}[label=p1.7_] 
Sa standarnog ulaza unosi se ceo broj $n$, a potom $n$ realnih
brojeva. Odrediti koliko puta je prilikom unosa došlo do promene
znaka. Ispisati dobijenu vrednost na standarni izlaz.\\ 
\linkresenje{p1.7_}
\end{Exercise}
\begin{Answer}[ref=p1.7_]
%\includecode{resenja/1_KontrolaToka/1.3_Petlje/1_14.c}
\end{Answer}


\begin{Exercise}[label=p1.3_16] 
Sa standardnog ulaza se unosi ceo pozitivan broj $n$, a zatim i $n$
celih brojeva. Napisati program koji ispisuje broj sa najvećom cifrom
desetica. Ukoliko ima više takvih, ispisati prvi. 

\begin{miditest}
\begin{upotreba}{1}
#\naslovInt#
#\izlaz{Unesite broj n:}\ulaz{5}#
#\izlaz{Unesite n brojeva:}\ulaz{18 365 25 1 78}#
#\izlaz{78}#
\end{upotreba}
\end{miditest}
\linkresenje{p1.3_16}
\end{Exercise}
\begin{Answer}[ref=p1.3_16]
\includecode{resenja/1_KontrolaToka/1.3_Petlje/praktikumi7/3_16.c}
\end{Answer}

\begin{Exercise}[label=p1.3_17] 
Sa standardnog ulaza se unosi ceo pozitivan broj $n$, a zatim i $n$
celih brojeva. Napisati program koji ispisuje broj sa najvećim brojem
cifara. Ukoliko ima više takvih, ispisati prvi. 

\begin{miditest}
\begin{upotreba}{1}
#\naslovInt#
#\izlaz{Unesite broj n:}\ulaz{5}#
#\izlaz{Unesite n brojeva:}\ulaz{18 365 25 1 78}#
#\izlaz{365}#
\end{upotreba}
\end{miditest}
\begin{miditest}
\begin{upotreba}{2}
#\naslovInt#
#\izlaz{Unesite broj n:}\ulaz{7}#
#\izlaz{Unesite n brojeva:}\ulaz{3 892 18 21 639 742 85}#
#\izlaz{892}#
\end{upotreba}
\end{miditest}
\linkresenje{p1.3_17}
\end{Exercise}
\begin{Answer}[ref=p1.3_17]
\includecode{resenja/1_KontrolaToka/1.3_Petlje/praktikumi7/3_17.c}
\end{Answer}

\begin{Exercise}[label=p1.3_18] 
Sa standardnog ulaza se unosi ceo pozitivan broj $n$, a zatim i $n$
celih brojeva. Napisati program koji ispisuje broj sa najvećom vodećom
cifrom. Vodeća cifra je cifra najveće težine u zapisu broja. Ukoliko ima više
takvih, ispisati prvi. 

\begin{miditest}
\begin{upotreba}{1}
#\naslovInt#
#\izlaz{Unesite broj n:}\ulaz{5}#
#\izlaz{Unesite n brojeva:}\ulaz{8 964 32 511 27}#
#\izlaz{964}#
\end{upotreba}
\end{miditest}
\begin{miditest}
\begin{upotreba}{1}
#\naslovInt#
#\izlaz{Unesite broj n:}\ulaz{3}#
#\izlaz{Unesite n brojeva:}\ulaz{41 669 8}#
#\izlaz{8}#
\end{upotreba}
\end{miditest}
\linkresenje{p1.3_18}
\end{Exercise}
\begin{Answer}[ref=p1.3_18]
\includecode{resenja/1_KontrolaToka/1.3_Petlje/praktikumi7/3_18.c}
\end{Answer}


\begin{Exercise}[label=p1.3_19] 
Sa standardnog ulaza se unose celi pozitivni brojevi $n$ ($n>1$) i
$d$, a zatim i $n$ celih brojeva. Napisati program koji izračunava
koliko ima parova uzastopnih brojeva među unetim brojevima koji se
nalaze na rastojanju $d$. Rastojanje između brojeva je definisano sa
$d(x,y)=|y-x|$. Rezultat ispisati na standardni izlaz. 

\begin{miditest}
\begin{upotreba}{1}
#\naslovInt#
#\izlaz{Unesite brojeve n i d:}\ulaz{5 2}#
#\izlaz{Unesite n brojeva:}\ulaz{2 3 5 1 -1}#
#\izlaz{Broj parova: 2}#
\end{upotreba}
\end{miditest}
\begin{miditest}
\begin{upotreba}{2}
#\naslovInt#
#\izlaz{Unesite brojeve n i d:}\ulaz{10 5}#
#\izlaz{Unesite n brojeva:}\ulaz{-3 6 11 -20 -25 -8 42 37 1 6}#
#\izlaz{Broj parova: 4}#
\end{upotreba}
\end{miditest}
\linkresenje{p1.3_19}
\end{Exercise}
\begin{Answer}[ref=p1.3_19]
\includecode{resenja/1_KontrolaToka/1.3_Petlje/praktikumi7/3_19.c}
\end{Answer}


\begin{Exercise}[label=p1.3_21] 
Vršna su merenja nadmorskih visina na određenom delu teritorije i
naučnike zanima razlika između najveće i najmanje nadmorske
visine. Napisati program koji učitava $n$, potom $n$ realnih brojeva
koji označvaju nadmorske visine i ispisuje razliku najveće i najmanje
nadmorske visine.

\begin{miditest}
\begin{upotreba}{1}
#\naslovInt#
#\izlaz{Unesite brojeve:}\ulaz{8 6 5 2 11 7 0}#
#\izlaz{Razlika: 9}#
\end{upotreba}
\end{miditest}
\begin{miditest}
\begin{upotreba}{2}
#\naslovInt#
#\izlaz{Unesite brojeve:}\ulaz{8 -1 8 6 0}#
#\izlaz{Razlika: 9}#
\end{upotreba}
\end{miditest}
\linkresenje{p1.3_21}
\end{Exercise}
\begin{Answer}[ref=p1.3_21]
\includecode{resenja/1_KontrolaToka/1.3_Petlje/praktikumi7/3_21.c}
\end{Answer}


%--------------------------------------------------------------------
%--------------------------------------------------------------------
\subsection{Rad sa karakterima}
%--------------------------------------------------------------------
%--------------------------------------------------------------------

\begin{Exercise}[label=v1.3_07] 
Napisati program koji učitava karaktere dok se ne unese karakter tačka
i ako je karakter malo slovo, ispisuje odgovarajuće veliko, ako je
karakter veliko slovo ispisuje odgovarajuće malo, a u suprotnom
ispisuje isti karakter kao i uneti.
\linkresenje{v1.3_07}
\end{Exercise}
\begin{Answer}[ref=v1.3_07]
\includecode{resenja/1_KontrolaToka/1.3_Petlje/1_07.c}
\end{Answer}

\begin{Exercise}[label=v1.3_08] 
Napisati program koji učitava karaktere sve do kraja ulaza, a
potom ispisuje broj velikih slova, broj malih slova, broj cifara, broj
belina i zbir unetih cifara.  
\linkresenje{v1.3_08}
\end{Exercise}
\begin{Answer}[ref=v1.3_08]
\includecode{resenja/1_KontrolaToka/1.3_Petlje/1_08.c}
\end{Answer}


\begin{Exercise}[label=p1.3_03] 
 Sa standardnog ulaza unosi se ceo pozitivan broj $n$, a potom i $n$
 karaktera. Za svaki od samoglasnika ispisati koliko puta se pojavio
 među unetim karakterima. Ne praviti razliku između malih i velikih
 slova. 
 
\begin{miditest}
\begin{upotreba}{1}
#\naslovInt#
#\izlaz{Unesite broj n:}\ulaz{5}#
#\izlaz{Unesite n karaktera:}\ulaz{u A b a o}#
#\izlaz{Samoglasnik a: 2}#
#\izlaz{Samoglasnik e: 0}#
#\izlaz{Samoglasnik i: 0}#
#\izlaz{Samoglasnik o: 1}#
#\izlaz{Samoglasnik u: 0}#
\end{upotreba}
\end{miditest}
\begin{miditest}
\begin{upotreba}{2}
#\naslovInt#
#\izlaz{Unesite broj n:}\ulaz{7}#
#\izlaz{Unesite n karaktera:}\ulaz{j k + E E a e}#
#\izlaz{Samoglasnik a: 1}#
#\izlaz{Samoglasnik e: 3}#
#\izlaz{Samoglasnik i: 0}#
#\izlaz{Samoglasnik o: 0}#
#\izlaz{Samoglasnik u: 0}#
\end{upotreba}
\end{miditest}
\linkresenje{p1.3_03}
\end{Exercise}
\begin{Answer}[ref=p1.3_03]
%\includecode{resenja/1_KontrolaToka/1.3_Petlje/praktikumi6/3_03.c}
\end{Answer}


\begin{Exercise}[label=p1.3_20] 
Sa standardnog ulaza se unosi ceo broj $n$, a zatim i $n$
karaktera. Napisati program koji proverava da li se od unetih
karaktera može napisati reč \textit{Zima}.

\begin{miditest}
\begin{upotreba}{1}
#\naslovInt#
#\izlaz{Unesite broj n:}\ulaz{4}#
#\izlaz{Unestite 1. karakter: }\ulaz{+}#
#\izlaz{Unestite 2. karakter: }\ulaz{o}#
#\izlaz{Unestite 3. karakter: }\ulaz{Z}#
#\izlaz{Unestite 4. karakter: }\ulaz{j}#
#\izlaz{Ne moze se napisati rec Zima.}#
\end{upotreba}
\end{miditest}
\begin{miditest}
\begin{upotreba}{2}
#\naslovInt#
#\izlaz{Unesite broj n:}\ulaz{10}#
#\izlaz{Unestite 1. karakter: }\ulaz{i}#
#\izlaz{Unestite 2. karakter: }\ulaz{9}#
#\izlaz{Unestite 3. karakter: }\ulaz{0}#
#\izlaz{Unestite 4. karakter: }\ulaz{p}#
#\izlaz{Unestite 5. karakter: }\ulaz{a}#
#\izlaz{Unestite 6. karakter: }\ulaz{Z}#
#\izlaz{Unestite 7. karakter: }\ulaz{o}#
#\izlaz{Unestite 8. karakter: }\ulaz{m}#
#\izlaz{Unestite 9. karakter: }\ulaz{M}#
#\izlaz{Unestite 10. karakter: }\ulaz{-}#
#\izlaz{Moze se napisati rec Zima.}#
\end{upotreba}
\end{miditest}
\linkresenje{p1.3_20}
\end{Exercise}
\begin{Answer}[ref=p1.3_20]
\includecode{resenja/1_KontrolaToka/1.3_Petlje/praktikumi7/3_20.c}
\end{Answer}


%--------------------------------------------------------------------
%--------------------------------------------------------------------
\subsection{Računanje sume i proizvoda}
%--------------------------------------------------------------------
%--------------------------------------------------------------------

\begin{Exercise}[label=v1.3_03] 
\komentar{Prekoracenje se javlja mnooogo ranije. I ovo je jedan od zadataka koji imamo u funkcijama. }
 Napisati program koji učitava ceo pozitivan broj i izračunava njegov
 faktorijel. U slučaju neispravnog unosa ispisati odgovarajuću poruku. 
 \uputstvo{Obratiti pažnju da počev od broja $23$ dolazi do prekoračenja prilikom računanja faktorijela.}
 \linkresenje{v1.3_03}
\end{Exercise}
\begin{Answer}[ref=v1.3_03]
\includecode{resenja/1_KontrolaToka/1.3_Petlje/1_03.c}
\end{Answer}


\begin{Exercise}[label=p1.3_08] 
 Sa standradnog ulaza unose se realan broj $x$ i ceo neoznačen broj
 $n$. Napisati program koji izračunava $n$-ti stepen broja $x$, tj.~$x^n$. 
 
\begin{miditest}
\begin{upotreba}{1}
#\naslovInt#
#\izlaz{Unesite redom brojeve x i n:}\ulaz{4 3}#
#\izlaz{64.00000}#
\end{upotreba}
\end{miditest}
\begin{miditest}
\begin{upotreba}{2}
#\naslovInt#
#\izlaz{Unesite redom brojeve x i n:}\ulaz{5.8 5}#
#\izlaz{6563.56768}#
\end{upotreba}
\end{miditest}

\begin{miditest}
\begin{upotreba}{3}
#\naslovInt#
#\izlaz{Unesite redom brojeve x i n:}\ulaz{11.43 0}#
#\izlaz{1.00000}#
\end{upotreba}
\end{miditest}
\linkresenje{p1.3_08}
\end{Exercise}
\begin{Answer}[ref=p1.3_08]
\includecode{resenja/1_KontrolaToka/1.3_Petlje/praktikumi6/3_08.c}
\end{Answer}

\begin{Exercise}[label=p1.3_09]
 Sa standradnog ulaza unose se realan broj $x$ i ceo broj
 $n$. Napisati program koji izračunava $n$-ti stepen broja $x$. 
 
\begin{miditest}
\begin{upotreba}{1}
#\naslovInt#
#\izlaz{Unesite redom brojeve x i n:}\ulaz{2 -3}#
#\izlaz{0.125}#
\end{upotreba}
\end{miditest}
\begin{miditest}
\begin{upotreba}{2}
#\naslovInt#
#\izlaz{Unesite redom brojeve x i n:}\ulaz{-3 2}#
#\izlaz{9.000}#
\end{upotreba}
\end{miditest}
\linkresenje{p1.3_}
\end{Exercise}
\begin{Answer}[ref=p1.3_09]
\includecode{resenja/1_KontrolaToka/1.3_Petlje/praktikumi6/3_09.c}
\end{Answer}


\begin{Exercise}[label=v1.3_13] 
Napisati program koji učitava ceo pozitivan broj $n$ i ispisuje
vrednost sume kubova brojeva od $1$ do $n$, odnosno $s = 1+2^3+3^3+
\ldots +n^3$. U slučaju greške pri unosu podataka ispisati
odgovarajuću poruku. \\
\linkresenje{v1.3_13}
\end{Exercise}
\begin{Answer}[ref=v1.3_13]
\includecode{resenja/1_KontrolaToka/1.3_Petlje/1_13.c}
\end{Answer}

\begin{Exercise}[label=v1.3_13] 
Napisati program koji učitava ceo pozitivan broj $n$ i ispisuje sumu
kubova, $s = 1+2^3+3^3+ \ldots +k^3$, za svaku vrednost $k = 1,
\ldots, n$.. U slučaju greške pri unosu podataka ispisati odgovarajuću
poruku. \\ 
\linkresenje{v1.3_13}
\end{Exercise}
\begin{Answer}[ref=v1.3_13]
\includecode{resenja/1_KontrolaToka/1.3_Petlje/1_13.c}
\end{Answer}



\begin{Exercise}[label=p1.3_10]
 Sa standardnog ulaza unose se realan broj $x$ i ceo neoznačen broj
 $n$. Napisati program koji izračunava i na standarni izlaz ispisuje
 sumu $S=x+2\cdot x^2+3\cdot x^3+\ldots+n\cdot x^n$.
 
\begin{miditest}
\begin{upotreba}{1}
#\naslovInt#
#\izlaz{Unesite redom brojeve x i n:}\ulaz{2 3}#
#\izlaz{S=34.000000}#
\end{upotreba}
\end{miditest}
\begin{miditest}
\begin{upotreba}{2}
#\naslovInt#
#\izlaz{Unesite redom brojeve x i n:}\ulaz{1.5 5}#
#\izlaz{S=74.343750}#
\end{upotreba}
\end{miditest}
\linkresenje{p1.3_10}
\end{Exercise}
\begin{Answer}[ref=p1.3_10]
\includecode{resenja/1_KontrolaToka/1.3_Petlje/praktikumi6/3_10.c}
\end{Answer}


\begin{Exercise}[label=p1.3_11]
 Sa standardnog ulaza unose se realan broj $x$ i ceo neoznačen broj
 $n$. Napisati program koji izračunava i na standarni izlaz ispisuje
 sumu $S=1+\frac{1}{x}+\frac{1}{x^2}+\ldots\frac{1}{x^n}$.
 
\begin{miditest}
\begin{upotreba}{1}
#\naslovInt#
#\izlaz{Unesite redom brojeve x i n:}\ulaz{2 4}#
#\izlaz{S=1.937500}#
\end{upotreba}
\end{miditest}
\begin{miditest}
\begin{upotreba}{2}
#\naslovInt#
#\izlaz{Unesite redom brojeve x i n:}\ulaz{1.8 6}#
#\izlaz{S=2.213249}#
\end{upotreba}
\end{miditest}
\linkresenje{p1.3_11}
\end{Exercise}
\begin{Answer}[ref=p1.3_11]
\includecode{resenja/1_KontrolaToka/1.3_Petlje/praktikumi6/3_11.c}
\end{Answer}


\begin{Exercise}[difficulty=1, label=p1.3_12] 
\komentar{Mislila sam da se tacnost eps odnosi na to da je razlika dva uzastopna clana manja od eps a ne da je sam clan manji od eps? Nisam sigurna, ali mozda treba proveriti ili preformulisati zadatak tako da se ne definise ovaj pojam.}
Napisati program koji učitava realane brojeve $x$ i $eps$ i sa zadatom
tačnošću $eps$ izračunava i na standarni izlaz ispisuje sumu
$S=1+x+\frac{x^2}{2!}+\frac{x^3}{3!}+\ldots$.   Izračunati sumu u odnosu
  na tačnost $eps$ znači uporediti poslednji član sume sa $eps$ i
  ukoliko je taj poslednji član manji od $eps$ prekinuti dalja
  izračunavanja.
\uputstvo{Prilikom
  računanja sume koristiti prethodni izračunati član sume u računanju
  sledećeg člana sume. Naime, ako je izračunat član sume
  $\frac{x^n}{n!}$ na osnovu njega se lako može dobiti član
  $\frac{x^{n+1}}{(n+1)!}$. Nikako ne računati stepen i faktorijel
  odvojeno zbog neefikasnosti takvog rešenja i zbog 
  mogućnosti prekoračenja.} 
  
\begin{miditest}
\begin{upotreba}{1}
#\naslovInt#
#\izlaz{Unesite x:}\ulaz{2}#
#\izlaz{Unesite tacnost eps:}\ulaz{0.001}#
#\izlaz{S=7.388713}#
\end{upotreba}
\end{miditest}
\begin{miditest}
\begin{upotreba}{2}
#\naslovInt#
#\izlaz{Unesite x:}\ulaz{3}#
#\izlaz{Unesite tacnost eps:}\ulaz{0.01}#
#\izlaz{S=20.079666}#
\end{upotreba}
\end{miditest}
\linkresenje{p1.3_12}
\end{Exercise}
\begin{Answer}[ref=p1.3_12]
\includecode{resenja/1_KontrolaToka/1.3_Petlje/praktikumi6/3_12.c}
\end{Answer}


\begin{Exercise}[difficulty=1, label=p1.3_13]
Napisati program koji učitava realane brojeve $x$ i $eps$ i sa zadatom
tačnošću $eps$ izračunava i na standarni izlaz ispisuje sumu
$S=1-x+\frac{x^2}{2!}-\frac{x^3}{3!}+\frac{x^4}{4!}-\frac{x^5}{5!}\ldots$. 
\napomena{Voditi računa o efikasnosti rešenja i o mogućnosti prekoračenja.} 
  
\begin{miditest}
\begin{upotreba}{1}
#\naslovInt#
#\izlaz{Unesite x:}\ulaz{3}#
#\izlaz{Unesite tacnost eps:}\ulaz{0.001}#
#\izlaz{S=0.049997}#
\end{upotreba}
\end{miditest}
\begin{miditest}
\begin{upotreba}{2}
#\naslovInt#
#\izlaz{Unesite x:}\ulaz{3.14}#
#\izlaz{Unesite tacnost eps:}\ulaz{0.01}#
#\izlaz{S=0.049072}#
\end{upotreba}
\end{miditest}
\linkresenje{p1.3_13}
\end{Exercise}
\begin{Answer}[ref=p1.3_13]
\includecode{resenja/1_KontrolaToka/1.3_Petlje/praktikumi6/3_13.c}
\end{Answer}

\begin{Exercise}[label=p1.7_] 
Napisati program koji učitava realan broj $x$ i prirodan broj $n$
izračunava sumu $S = (1 + \cos(x))\cdot(1 + \cos(x^2))\cdot \ldots
\cdot(1 + \cos(x^n))$. \napomena{Voditi računa o efikasnosti
  rešenja}.  \linkresenje{p1.7_}
\end{Exercise}
\begin{Answer}[ref=p1.7_]
%\includecode{resenja/1_KontrolaToka/1.3_Petlje/1_14.c}
\end{Answer}


\begin{Exercise}[difficulty=1, label=p1.7_] 
Napisati program koji učitava ceo neoznačen broj n, a na standarni
izlaz ispisuje vrednost razlomka\\
	\[
		\frac{1}{1 + \frac{1}{2 + \frac{1}{3 + \frac{1}{4 + \frac{1}{\ldots + \frac{1}{(n-1) + \frac{1}{n}}}}}}}.
	\]
 \\
\linkresenje{p1.7_}
\end{Exercise}
\begin{Answer}[ref=p1.7_]
%\includecode{resenja/1_KontrolaToka/1.3_Petlje/1_14.c}
\end{Answer}

\begin{Exercise}[difficulty=1, label=p1.7_] 
Napisati program koji računa sumu
$$1 - \frac{x^{2}}{2!} + \frac{x^{4}}{4!} - \ldots +
(-1)^{n}\frac{x^{2n}}{(2n)!}.$$ za unete cele brojeve $x$ i $n$. 
\napomena{Voditi računa o efikasnosti rešenja i o mogućnosti prekoračenja.} 
\linkresenje{p1.7_}
\end{Exercise}
\begin{Answer}[ref=p1.7_]
%\includecode{resenja/1_KontrolaToka/1.3_Petlje/1_14.c}
\end{Answer}


\begin{Exercise}[difficulty=1, label=p1.7_] 
Sa standardnog ulaza unosi se ceo pozitivan broj $n$ veći od $0$. 
Napisati program koji računa proizvod
$$S = (1 + \frac{1}{2!})(1 + \frac{1}{3!})\ldots(1 +
\frac{1}{n!}).$$ U slučaju greške pri unosu podataka ispisati 
odgovarajuću  poruku. \napomena{Voditi računa o efikasnosti rešenja i o
  mogućnosti prekoračenja.} 
  
\begin{miditest}
\begin{upotreba}{1}
#\naslovInt#
#\izlaz{Unesite broj n:}\ulaz{5}#
#\izlaz{1.838108}#
\end{upotreba}
\end{miditest}
\begin{miditest}
\begin{upotreba}{2}
#\naslovInt#
#\izlaz{Unesite broj n:}\ulaz{7}#
#\izlaz{1.841026}#
\end{upotreba}
\end{miditest}

\begin{miditest}
\begin{upotreba}{3}
#\naslovInt#
#\izlaz{Unesite broj n:}\ulaz{0}#
#\izlaz{-1}#
\end{upotreba}
\end{miditest}
\begin{miditest}
\begin{upotreba}{4}
#\naslovInt#
#\izlaz{Unesite broj n:}\ulaz{10}#
#\izlaz{1.841077}#
\end{upotreba}
\end{miditest}
\linkresenje{p1.7_}
\end{Exercise}
\begin{Answer}[ref=p1.7_]
%\includecode{resenja/1_KontrolaToka/1.3_Petlje/1_14.c}
\end{Answer}

\begin{Exercise}[difficulty=1, label=p1.7_] 
Sa standardnog ulaza unosi se ceo pozitivan neparan broj $n$.
Napisati program koji za uneto $n$ izračunava:
$$S = 1\cdot3\cdot5 - 1\cdot3\cdot5\cdot7 + 1\cdot3\cdot5\cdot7\cdot9
- 1\cdot3\cdot5\cdot7\cdot9\cdot11 + \ldots
(-1)^{\frac{n-1}{2}+1}\cdot1\cdot3\cdot \ldots \cdot n.$$ U slučaju
greške pri unosu podataka ispisati odgovarajuću poruku. 
\napomena{Voditi računa o efikasnosti rešenja i o
  mogućnosti prekoračenja.} 
  
\begin{miditest}
\begin{upotreba}{1}
#\naslovInt#
#\izlaz{Unesite broj n:}\ulaz{9}#
#\izlaz{855}#
\end{upotreba}
\end{miditest}
\begin{miditest}
\begin{upotreba}{2}
#\naslovInt#
#\izlaz{Unesite broj n:}\ulaz{11}#
#\izlaz{-9540}#
\end{upotreba}
\end{miditest}

\begin{miditest}
\begin{upotreba}{3}
#\naslovInt#
#\izlaz{Unesite broj n:}\ulaz{20}#
#\izlaz{-1}#
\end{upotreba}
\end{miditest}
\begin{miditest}
\begin{upotreba}{4}
#\naslovInt#
#\izlaz{Unesite broj n:}\ulaz{-3}#
#\izlaz{-1}#
\end{upotreba}
\end{miditest}
\linkresenje{p1.7_}
\end{Exercise}
\begin{Answer}[ref=p1.7_]
%\includecode{resenja/1_KontrolaToka/1.3_Petlje/1_14.c}
\end{Answer}

\begin{Exercise}[label=p1.7_] 
Sa standardnog ulaza unose se realni brojevi $x$ i $a$ i ceo
pozitivan broj $n$ veći od $0$.  Napisati program koji izračunava:
 $$((\ldots \underbrace{(((x+a)^2 + a)^2 + a)^2 + \ldots a)^2}_n.$$ U
slučaju greške pri unosu podataka ispisati odgovarajuću poruku. 

\begin{miditest}
\begin{upotreba}{1}
#\naslovInt#
#\izlaz{Unesite broj n:}\ulaz{3.2 0.2 5}#
#\izlaz{367940960.000000}#
\end{upotreba}
\end{miditest}
\begin{miditest}
\begin{upotreba}{2}
#\naslovInt#
#\izlaz{Unesite broj n:}\ulaz{2 1 3}#
#\izlaz{101.000000}#
\end{upotreba}
\end{miditest}

\begin{miditest}
\begin{upotreba}{3}
#\naslovInt#
#\izlaz{Unesite broj n:}\ulaz{2.6 0.3 3}#
#\izlaz{76.164085}#
\end{upotreba}
\end{miditest}
\begin{miditest}
\begin{upotreba}{4}
#\naslovInt#
#\izlaz{Unesite broj n:}\ulaz{5.4 7 -2}#
#\izlaz{-1}#
\end{upotreba}
\end{miditest}
\linkresenje{p1.7_}
\end{Exercise}
\begin{Answer}[ref=p1.7_]
%\includecode{resenja/1_KontrolaToka/1.3_Petlje/1_14.c}
\end{Answer}


%--------------------------------------------------------------------
%--------------------------------------------------------------------
\subsection{Dvostruka petlja i ispisivanje slike}
%--------------------------------------------------------------------
%--------------------------------------------------------------------


\begin{Exercise}[label=p1.7_] 
Sa standardnog ulaza unosi se neoznačen broj $n$. Napisati program
koji za uneto $n$ zvezdicama iscrtava
\begin{description}
\item{a)} kvadrat stranice $n$ sastavljen od zvezdica. \\
\begin{miditest}
\begin{upotreba}{1}
#\naslovInt#
#\izlaz{Unesite broj n:}\ulaz{3}#
#\izlaz{***}#
#\izlaz{***}#
#\izlaz{***}#
\end{upotreba}
\end{miditest}
\item{b)} rub kvadrata dimenzije $n$. \\
\begin{miditest}
\begin{upotreba}{1}
#\naslovInt#
#\izlaz{Unesite broj n:}\ulaz{5}#
#\izlaz{*****}#
#\izlaz{*\ \ \ *}#
#\izlaz{*\ \ \ *}#
#\izlaz{*\ \ \ *}#
#\izlaz{*****}#
\end{upotreba}
\end{miditest}
\begin{miditest}
\begin{upotreba}{2}
#\naslovInt#
#\izlaz{Unesite broj n:}\ulaz{2}#
#\izlaz{**}#
#\izlaz{**}#
\end{upotreba}
\end{miditest}
\item{c)} rub kvadrata dimenzije $n$ koji i na glavnoj dijagonali ima
  zvezdice. \\
\begin{miditest}
\begin{upotreba}{1}
#\naslovInt#
#\izlaz{Unesite broj n:}\ulaz{5}#
#\izlaz{*****}#
#\izlaz{**\ \ *}#
#\izlaz{*\ *\ *}#
#\izlaz{*\ \ **}#
#\izlaz{*****}#
\end{upotreba}
\end{miditest}
\end{description}
\linkresenje{p1.7_}
\end{Exercise}
\begin{Answer}[ref=p1.7_]
%\includecode{resenja/1_KontrolaToka/1.3_Petlje/1_14.c}
\end{Answer}

\begin{Exercise}[difficulty=1, label=p1.3_25]
 Napisati program koji za uneti ceo broj $n$ zvezdicama iscrtava slovo \textit{X}
 dimenzije $n$. 

\begin{miditest}
\begin{upotreba}{1}
#\naslovInt#
#\izlaz{Unesite broj n:}\ulaz{5}#
#\izlaz{*\ \ \ *}#
#\izlaz{\ *\ *\ }#
#\izlaz{\ \ *\ \ }#
#\izlaz{\ *\ *\ }#
#\izlaz{*\ \ \ *}#
\end{upotreba}
\end{miditest}
\begin{miditest}
\begin{upotreba}{2}
#\naslovInt#
#\izlaz{Unesite broj n:}\ulaz{3}#
#\izlaz{*\ *}#
#\izlaz{\ *\ }#
#\izlaz{*\ *}#
\end{upotreba}
\end{miditest}
\linkresenje{p1.3_25}
\end{Exercise}
\begin{Answer}[ref=p1.3_25]
\includecode{resenja/1_KontrolaToka/1.3_Petlje/praktikumi7/3_25.c}
\end{Answer}


\begin{Exercise}[difficulty=1, label=p1.3_25]
 Napisati program koji za uneti ceo broj $n$ korišćenjem znaka $+$
 iscrtava veliko $+$ dimenzije $n$.
 
\begin{miditest}
\begin{upotreba}{1}
#\naslovInt#
#\izlaz{Unesite broj n:}\ulaz{5}#
#\izlaz{\ \ +}#
#\izlaz{\ \ +}#
#\izlaz{+++++}#
#\izlaz{\ \ +}#
#\izlaz{\ \ +}#
\end{upotreba}
\end{miditest}
\begin{miditest}
\begin{upotreba}{2}
#\naslovInt#
#\izlaz{Unesite broj n:}\ulaz{3}#
#\izlaz{\ +}#
#\izlaz{+++}#
#\izlaz{\ +}#
\end{upotreba}
\end{miditest}
\linkresenje{p1.3_25}
\end{Exercise}
\begin{Answer}[ref=p1.3_25]
\includecode{resenja/1_KontrolaToka/1.3_Petlje/praktikumi7/3_25.c}
\end{Answer}



\begin{Exercise}[label=p1.7_] 
Napisati program koji učitava ceo neoznačen broj $n$, a potom iscrtava
\begin{description}
\item{a)} pravougli trougao sastavljen od zvezdica. Kateta trougla je
  dužine $n$, a prav ugao se nalazi u gornjem levom uglu slike. \\
\begin{miditest}
\begin{upotreba}{1}
#\naslovInt#
#\izlaz{Unesite broj n:}\ulaz{3}#
#\izlaz{***}#
#\izlaz{**}#
#\izlaz{*}#
\end{upotreba}
\end{miditest}
\item{b)} pravougli trougao sastavljen od zvezdica. Kateta trougla je
  dužine $n$, a prav ugao se nalazi u donjem levom uglu slike. \\
\begin{miditest}
\begin{upotreba}{1}
#\naslovInt#
#\izlaz{Unesite broj n:}\ulaz{3}#
#\izlaz{*}#
#\izlaz{**}#
#\izlaz{***}#
\end{upotreba}
\end{miditest}
\item{c)} trougao sastavljen od zvezdica. Trougao se dobija spajanjem
  dva pravougla trougla čija kateta je dužine $n$, pri čemu je prav
  ugao prvog trougla u njegovom donjem levom uglu, dok je prav ugao
  drugog trougla u njegovom gornjem levom uglu, a spajanje se vrši po
  horiznotalnoj kateti. \\
\begin{miditest}
\begin{upotreba}{1}
#\naslovInt#
#\izlaz{Unesite broj n:}\ulaz{3}#
#\izlaz{*}#
#\izlaz{**}#
#\izlaz{***}#
#\izlaz{**}#
#\izlaz{*}#
\end{upotreba}
\end{miditest}
\item{d)} rub jednakokrakog pravouglog trougla čije su katete dužine
  $n$. Program učitava karakter $c$ i taj karakter koristi za
  iscrtavanje ruba trougla. \\
\begin{miditest}
\begin{upotreba}{1}
#\naslovInt#
#\izlaz{Unesite broj n:}\ulaz{4}#
#\izlaz{Unesite karakter c:}\ulaz{*}#
#\izlaz{*}#
#\izlaz{**}#
#\izlaz{*\ *}#
#\izlaz{****}#
\end{upotreba}
\end{miditest}
\begin{miditest}
\begin{upotreba}{2}
#\naslovInt#
#\izlaz{Unesite broj n:}\ulaz{5}#
#\izlaz{Unesite karakter c:}\ulaz{+}#
#\izlaz{+}#
#\izlaz{++}#
#\izlaz{+\ +}#
#\izlaz{+\ \ +}#
#\izlaz{+++++}#
\end{upotreba}
\end{miditest}
\end{description}
\linkresenje{p1.7_}
\end{Exercise}
\begin{Answer}[ref=p1.7_]
%\includecode{resenja/1_KontrolaToka/1.3_Petlje/1_14.c}
\end{Answer}


\begin{Exercise}[label=p1.7_] 
Napisati program koji učitava ceo broj $n$, a potom iscrtava
\begin{description}
\item{a)} jednakostranični trougao stranice $n$ koji je sastavljen od
  zvezdica. \\
\begin{miditest}
\begin{upotreba}{1}
#\naslovInt#
#\izlaz{Unesite broj n:}\ulaz{3}#
#\izlaz{\ \ *}#
#\izlaz{\ ***}#
#\izlaz{*****}#
\end{upotreba}
\end{miditest}
\item{b)} trougao koji se dobija spajanjem dva jednakostranični
  trougla stranice $n$ koji su sastavljeni od zvezdica. \\
\begin{miditest}
\begin{upotreba}{1}
#\naslovInt#
#\izlaz{Unesite broj n:}\ulaz{3}#
#\izlaz{\ \ *}#
#\izlaz{\ ***}#
#\izlaz{*****}#
#\izlaz{\ ***}#
#\izlaz{\ \ *}#
\end{upotreba}
\end{miditest}
\item{c)} rub jednakostraničnog trougla čija stranica je dužine $n$. \\
\begin{miditest}
\begin{upotreba}{1}
#\naslovInt#
#\izlaz{Unesite broj n:}\ulaz{3}#
#\izlaz{\ \ *}#
#\izlaz{\ *\ *}#
#\izlaz{*\ *\ *}#
\end{upotreba}
\end{miditest}
\item{d)} sliku koja se dobija spajanjem dva jednakostranična trougla
  čija stranica je dužine $n$. Iscrtavati samo rub trouglova.\\
\begin{miditest}
\begin{upotreba}{1}
#\naslovInt#
#\izlaz{Unesite broj n:}\ulaz{3}#
#\izlaz{\ \ *}#
#\izlaz{\ *\ *}#
#\izlaz{*\ *\ *}#
#\izlaz{\ *\ *}#
#\izlaz{\ \ *}#
\end{upotreba}
\end{miditest}
\end{description}
\linkresenje{p1.7_}
\end{Exercise}
\begin{Answer}[ref=p1.7_]
%\includecode{resenja/1_KontrolaToka/1.3_Petlje/1_14.c}
\end{Answer}



\begin{Exercise}[difficulty=1, label=p1.3_26] 
 Napisati program koji za uneti ceo broj $n$ iscrtava strelice
 dimenzije $n$. 
 
\begin{miditest}
\begin{upotreba}{1}
#\naslovInt#
#\izlaz{Unesite broj n:}\ulaz{3}#
#\izlaz{*}#
#\izlaz{\ *}#
#\izlaz{***}#
#\izlaz{\ *}#
#\izlaz{*}#
\end{upotreba}
\end{miditest}
\begin{miditest}
\begin{upotreba}{2}
#\naslovInt#
#\izlaz{Unesite broj n:}\ulaz{5}#
#\izlaz{*}#
#\izlaz{\ *}#
#\izlaz{\ \ *}#
#\izlaz{\ \ \ *}#
#\izlaz{*****}#
#\izlaz{\ \ \ *}#
#\izlaz{\ \ *}#
#\izlaz{\ *}#
#\izlaz{*}#
\end{upotreba}
\end{miditest} 
\linkresenje{p1.3_26}
\end{Exercise}
\begin{Answer}[ref=p1.3_26]
%\includecode{resenja/1_KontrolaToka/1.3_Petlje/praktikumi7/3_26.c}
\end{Answer}

\begin{Exercise}[difficulty=1, label=p1.7_] 
Napisati program koji učitava ceo broj $n$, i iscrtava sliku koja se
dobija na sledeći način: u prvom redu je jedna zvezdica, u drugom redu
su dve zvezdice razdvojene razmakom, treći red je sastavljen od
zvezdica i iste je dužine kao i drugi red, četvrti red se sastoji od
tri zvezdice razdvojene razmakom, a peti red je sastavljen od zvezdica
i iste je dužine kao i četvrti red itd. Ukupna visina slike je $n$.

\begin{miditest}
\begin{upotreba}{1}
#\naslovInt#
#\izlaz{Unesite broj n:}\ulaz{7}#
#\izlaz{*}#
#\izlaz{*\ *}#
#\izlaz{***}#
#\izlaz{*\ *\ *}#
#\izlaz{*****}#
#\izlaz{*\ *\ *\ *}#
#\izlaz{*******}#
\end{upotreba}
\end{miditest}
\linkresenje{p1.7_}
\end{Exercise}
\begin{Answer}[ref=p1.7_]
%\includecode{resenja/1_KontrolaToka/1.3_Petlje/1_14.c}
\end{Answer}

\begin{Exercise}[difficulty=2, label=p1.7_] 
Sa standarnog ulaza unose se neoznačeni celi brojevi $m$ i
$n$. Napisati program koji iscrtava jedan do drugog stranice $n$
kvadrata čija je svaka strana sastavljena od $m$ zvezdica razdvojenih
prazninom.

\komentar{Tekst nije u skladu sa slikom jer nije jasno da se crtaju samo rubovi a ne popunjeni kvadrati. \\
Danijela: Da li je sada jasnije?}

\begin{miditest}
\begin{upotreba}{1}
#\naslovInt#
#\izlaz{Unesite broj n:}\ulaz{5 3}#
#\izlaz{*\ *\ *\ *\ *\ *\ *\ *\ *\ *\ *\ *\ *}#         
#\izlaz{*\ \ \ \ \ \ \ *\ \ \ \ \ \ \ *\ \ \ \ \ \ \ *}#           
#\izlaz{*\ \ \ \ \ \ \ *\ \ \ \ \ \ \ *\ \ \ \ \ \ \ *}#             
#\izlaz{*\ \ \ \ \ \ \ *\ \ \ \ \ \ \ *\ \ \ \ \ \ \ *}#
#\izlaz{*\ *\ *\ *\ *\ *\ *\ *\ *\ *\ *\ *\ *}#
\end{upotreba}
\end{miditest}
\begin{miditest}
\begin{upotreba}{2}
#\naslovInt#
#\izlaz{Unesite broj n:}\ulaz{4 4}#
#\izlaz{*\ *\ *\ *\ *\ *\ *\ *\ *\ *\ *\ *\ *}#
#\izlaz{*\ \ \ \ \ *\ \ \ \ \ *\ \ \ \ \ *\ \ \ \ \ *}#
#\izlaz{*\ \ \ \ \ *\ \ \ \ \ *\ \ \ \ \ *\ \ \ \ \ *}#
#\izlaz{*\ *\ *\ *\ *\ *\ *\ *\ *\ *\ *\ *\ *}#
\end{upotreba}
\end{miditest}
\linkresenje{p1.7_}
\end{Exercise}
\begin{Answer}[ref=p1.7_]
%\includecode{resenja/1_KontrolaToka/1.3_Petlje/1_14.c}
\end{Answer}

\begin{Exercise}[difficulty=1, label=p1.7_] 
Sa standarnog ulaza unosi se ceo neoznačen broj $n$. Napisati program
koji štampa romb sastavljen od minusa u pravougaoniku sastavljenom od
zvezdica. 

\begin{miditest}
\begin{upotreba}{1}
#\naslovInt#
#\izlaz{Unesite broj n:}\ulaz{6}#
#\izlaz{************}#
#\izlaz{*****--*****}#
#\izlaz{****----****}#
#\izlaz{***------***}#
#\izlaz{**--------**}#
#\izlaz{*----------*}#
#\izlaz{**--------**}#
#\izlaz{***------***}#
#\izlaz{****----****}#
#\izlaz{*****--*****}#
#\izlaz{************}#
\end{upotreba}
\end{miditest}
\begin{miditest}
\begin{upotreba}{2}
#\naslovInt#
#\izlaz{Unesite broj n:}\ulaz{2}#
#\izlaz{****}#
#\izlaz{*--*}#
#\izlaz{****}#
\end{upotreba}
\end{miditest}
\linkresenje{p1.7_}
\end{Exercise}
\begin{Answer}[ref=p1.7_]
%\includecode{resenja/1_KontrolaToka/1.3_Petlje/1_14.c}
\end{Answer}

\begin{Exercise}[label=p1.7_] 
Napisati program koji učitava ceo broj $n$ ($n \geq 2$) i 
koji na standardni izlaz iscrtava sliku kuće sa krovom: kuća
je kocka stranice $n$, a krov jednakostranični trougao stranice
$n$.

\begin{miditest}
\begin{upotreba}{1}
#\naslovInt#
#\izlaz{Unesite broj n:}\ulaz{5}#
#\izlaz{\ \ \ *}#
#\izlaz{\ \ *\ *}#
#\izlaz{\ *\ \ \ *}#
#\izlaz{*\ *\ *\ *}#
#\izlaz{*\ \ \ \ \ *}#
#\izlaz{*\ \ \ \ \ *}#
#\izlaz{*\ *\ *\ *}#
\end{upotreba}
\end{miditest}
\linkresenje{p1.7_}
\end{Exercise}
\begin{Answer}[ref=p1.7_]
%\includecode{resenja/1_KontrolaToka/1.3_Petlje/1_14.c}
\end{Answer}



\begin{Exercise}[label=p1.7_] 
Sa standarnog ulaza učitava se ceo neoznačen broj $n$. Napisati
program koji za uneto $n$ iscrtava pravougli ,,trougao'' sačinjen od
,,koordinata'' svojih tačaka. "Koordinata" tačke je oblika $(i,j)$ pri
čemu $i,\ j = 0, \ldots, n$. Prav ugao se nalazi u gornjem levom uglu
slike i njegova koordinata je $(0, 0)$. Koordinata $i$ se uvećava po
vrsti, a koordinata $j$ po koloni, pa je zato koordinata tačke koja je
ispod tačke $(0,0)$ jednaka $(1, 0)$, a koordinata tačke koja je desno
od tačke $(0,0)$ jednaka $(0,1)$.

\komentar{Ovo treba preformulisati jer je bez test primera skroz nejasno. U test primerima negde ima blanko posle zareza, negde nema, i to treba ujednaciti.}

\komentar{Mene ovaj zadatak zbunjuje i ne svidja mi se. Problem su mi koordinate koje se broje nekako cudno i to od broja 1 a ne od nule. Nije mi jasno zasto u temenu pravog ugla ne bi bila koordinata (0,0)?}

\komentar{Danijela: Izmenila sam test primere i tekst, ali se slazem da zadatak nije nesto, mozemo ga izbrisati.}

\begin{miditest}
\begin{upotreba}{1}
#\naslovInt#
#\izlaz{Unesite broj n:}\ulaz{1}#
#\izlaz{(0,0)}#
\end{upotreba}
\end{miditest}
\begin{miditest}
\begin{upotreba}{2}
#\naslovInt#
#\izlaz{Unesite broj n:}\ulaz{2}#
#\izlaz{(0,0) (0,1)}#
#\izlaz{(1,0)}#
\end{upotreba}
\end{miditest}

\begin{miditest}
\begin{upotreba}{3}
#\naslovInt#
#\izlaz{Unesite broj n:}\ulaz{3}#
#\izlaz{(0,0) (0,1) (0,2)}#
#\izlaz{(1,0) (1,1)}#
#\izlaz{(2,0)}#
\end{upotreba}
\end{miditest}
\begin{miditest}
\begin{upotreba}{4}
#\naslovInt#
#\izlaz{Unesite broj n:}\ulaz{4}#
#\izlaz{(0,0) (0,1) (0,2) (0,3)}#
#\izlaz{(1,0) (1,1) (1,2)}#
#\izlaz{(2,0) (2,1)}#
#\izlaz{(3,0)}#
\end{upotreba}
\end{miditest}
\linkresenje{p1.7_}
\end{Exercise}
\begin{Answer}[ref=p1.7_]
%\includecode{resenja/1_KontrolaToka/1.3_Petlje/1_14.c}
\end{Answer}




\begin{Exercise}[difficulty=1, label=p1.7_] 
Sa standardnog ulaza unosi se ceo pozitivan broj $n$. Napisati program
koji ispisuje brojeve od $1$ do $n$, zatim od $2$ do $n-1$, $3$ do
$n-2$, itd. Ispis se završava kada nije moguće ispisati ni jedan
broj. Za neispravan unos, program ispisuje odgovarajuću  poruku.

\begin{miditest}
\begin{upotreba}{1}
#\naslovInt#
#\izlaz{Unesite broj n:}\ulaz{5}#
#\izlaz{1 2 3 4 5 2 3 4 3}#
\end{upotreba}
\end{miditest}
\begin{miditest}
\begin{upotreba}{2}
#\naslovInt#
#\izlaz{Unesite broj n:}\ulaz{-4}#
#\izlaz{-1}#
\end{upotreba}
\end{miditest}

\begin{miditest}
\begin{upotreba}{3}
#\naslovInt#
#\izlaz{Unesite broj n:}\ulaz{5}#
#\izlaz{1 2 3 4 5 6 7 2 3 4 5 6 3 4 5 4}#
\end{upotreba}
\end{miditest}
\begin{miditest}
\begin{upotreba}{4}
#\naslovInt#
#\izlaz{Unesite broj n:}\ulaz{3}#
#\izlaz{1 2 3 2}#
\end{upotreba}
\end{miditest}
\linkresenje{p1.7_}
\end{Exercise}
\begin{Answer}[ref=p1.7_]
%\includecode{resenja/1_KontrolaToka/1.3_Petlje/1_14.c}
\end{Answer}



\begin{Exercise}[difficulty=1, label=p1.7_] 
Napisati program koji učitava ceo pozitivan broj $n$ i ispisuje sve
brojeve od $1$ do $n$, zatim svaki drugi broj od $1$ do $n$, zatim
svaki treći broj od $1$ do $n$ itd., završavajući sa svakim $n$-tim
(tj. samo sa $1$). U slučaju greške pri unosu podataka odštampati
ogovarajuću poruku.

\begin{miditest}
\begin{upotreba}{1}
#\naslovInt#
#\izlaz{Unesite broj n:}\ulaz{3}#
#\izlaz{1 2 3}#
#\izlaz{1 3}#
#\izlaz{1}#
\end{upotreba}
\end{miditest}
\begin{miditest}
\begin{upotreba}{2}
#\naslovInt#
#\izlaz{Unesite broj n:}\ulaz{1}#
#\izlaz{1}#
\end{upotreba}
\end{miditest}

\begin{miditest}
\begin{upotreba}{3}
#\naslovInt#
#\izlaz{Unesite broj n:}\ulaz{7}#
#\izlaz{1 2 3 4 5 6 7}#
#\izlaz{1 3 5 7}#
#\izlaz{1 4 7}#
#\izlaz{1 5}#
#\izlaz{1 6}#
#\izlaz{1 7}#
#\izlaz{1}#
\end{upotreba}
\end{miditest}
\begin{miditest}
\begin{upotreba}{4}
#\naslovInt#
#\izlaz{Unesite broj n:}\ulaz{-23}#
\end{upotreba}
\end{miditest}
\linkresenje{p1.7_}
\end{Exercise}
\begin{Answer}[ref=p1.7_]
%\includecode{resenja/1_KontrolaToka/1.3_Petlje/1_14.c}
\end{Answer}




\section{Rešenja}
\shipoutAnswer





