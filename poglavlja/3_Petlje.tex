\section{Petlje}

%--------------------------------------------------------------------
%--------------------------------------------------------------------
% \subsection{Ispis podataka}
%--------------------------------------------------------------------
%--------------------------------------------------------------------


\begin{Exercise}[label=1.3_1] 
Napisati program koji $5$ puta ispisuje tekst \kckod{Mi volimo da programiramo}.  
  
\begin{miditest}
\begin{upotreba}{1}
#\naslovInt#
#\izlaz{Mi volimo da programiramo.}#
#\izlaz{Mi volimo da programiramo.}#
#\izlaz{Mi volimo da programiramo.}#
#\izlaz{Mi volimo da programiramo.}#
#\izlaz{Mi volimo da programiramo.}#
\end{upotreba}
\end{miditest}
  \linkresenje{1.3_1}
\end{Exercise}
\ifresenja
\end{Answer}[ref=1.3_1]
\includecode{resenja/1_KontrolaToka/1.3_Petlje/1.3_1.c}
\end{Answer}

\begin{Exercise}[label=1.3_2] 
Napisati program koji učitava ceo broj $n$ i ispisuje $n$ puta tekst
\kckod{Mi volimo da programiramo}. 

\begin{miditest}
\begin{upotreba}{1}
#\naslovInt#
#\izlaz{Unesite ceo broj:}\ulaz{6}#
#\izlaz{Mi volimo da programiramo.}#
#\izlaz{Mi volimo da programiramo.}#
#\izlaz{Mi volimo da programiramo.}#
#\izlaz{Mi volimo da programiramo.}#
#\izlaz{Mi volimo da programiramo.}#
#\izlaz{Mi volimo da programiramo.}#
\end{upotreba}
\end{miditest}
\begin{miditest}
\begin{upotreba}{2}
#\naslovInt#
#\izlaz{Unesite ceo broj}\ulaz{0}#
\end{upotreba}
\end{miditest}
\linkresenje{1.3_2}
\end{Exercise}
\ifresenja
\end{Answer}[ref=1.3_2]
\includecode{resenja/1_KontrolaToka/1.3_Petlje/1.3_2.c}
\end{Answer}
\fi


\begin{Exercise}[label=1.3_3] 
Napisati program koji učitava pozitivan ceo broj $n$
a potom ispisuje sve cele brojeve od $0$ do $n$. 

\begin{miditest}
\begin{upotreba}{1}
#\naslovInt#
#\izlaz{Unesite ceo pozitivan broj:}\ulaz{4}#
#\izlaz{0 1 2 3 4}#
\end{upotreba}
\end{miditest}
\begin{miditest}
\begin{upotreba}{2}
#\naslovInt#
#\izlaz{Unesite ceo pozitivan broj:}\ulaz{-10}#
#\izlaz{Neispravan unos. Promenljiva mora biti pozitivna!}#
\end{upotreba}
\end{miditest}
\linkresenje{1.3_3}
\end{Exercise}
\ifresenja
\end{Answer}[ref=1.3_3]
\includecode{resenja/1_KontrolaToka/1.3_Petlje/1.3_3.c}
\end{Answer}
\fi


\begin{Exercise}[label=1.3_4] 
Napisati program koji učitava dva cela broja $n$ i $m$ ispisuje sve
cele brojeve iz intervala $[n,m]$.
\begin{enumerate}
\item Koristiti \kckod{while} petlju.
\item Koristiti \kckod{for} petlju.
\item Koristiti \kckod{do-while} petlju.
 \end{enumerate}

\begin{miditest}
\begin{upotreba}{1}
#\naslovInt#
#\izlaz{Unesite dva cela broja:}\ulaz{-2 4}#
#\izlaz{-2 -1 0 1 2 3 4}#
\end{upotreba}
\end{miditest}
\begin{miditest}
\begin{upotreba}{2}
#\naslovInt#
#\izlaz{Unesite dva cela broja:}\ulaz{10 6}#
#\izlaz{Neispravan unos. Nisu dobro zadate granice intervala!}#
\end{upotreba}
\end{miditest}
\linkresenje{1.3_4}
\end{Exercise}
\ifresenja
\end{Answer}[ref=1.3_4]
\includecode{resenja/1_KontrolaToka/1.3_Petlje/1.3_4a.c}
\includecode{resenja/1_KontrolaToka/1.3_Petlje/1.3_4b.c}
\includecode{resenja/1_KontrolaToka/1.3_Petlje/1.3_4c.c}
\end{Answer}
\fi

%\komentarM{Komentari u resenjima su uglavnom ok. Jedino smo rekli da stil treba da bude bezlican, tj da umesto "ucitavamo broj" ili "transformisemo vrednost" kazemo "ucitava se broj" ili "vrednost se transformise". Ima i jedne i druge vrste komentara, cak i u istom resenju, i to treba ujednaciti.}

%--------------------------------------------------------------------
%--------------------------------------------------------------------
% \subsection{Obrada celih brojeva, rad sa ciframa broja}
%--------------------------------------------------------------------
%--------------------------------------------------------------------

\begin{Exercise}[label=1.3_5] 
 Napisati program koji učitava ceo pozitivan broj i izračunava njegov
 faktorijel. U slučaju neispravnog unosa ispisati odgovarajuću poruku.

%\komentarD{Ja sam proveravala i kod mene do prekoracenja dolazi od
%22. Prilicno sam sigurna u to jer za svaki broj rezulatat je veci za
%jednu cifru, a onda za 22 vise nije i nema logike u odnosu na
%rezultat za 21!. Ovde se u resenju koristi unsigned long int, pa je
%mozda to razlog zasto kasnije dolazi do prekoracenja.}

\begin{minitest}
\begin{upotreba}{1}
#\naslovInt#
#\izlaz{Unesite pozitivan broj:}\ulaz{18}#
#\izlaz{Faktorijel = 6402373705728000}#
\end{upotreba}
\end{minitest}
\begin{minitest}
\begin{upotreba}{2}
#\naslovInt#
#\izlaz{Unesite pozitivan broj:}\ulaz{8}#
#\izlaz{Faktorijel = 40320}#
\end{upotreba}
\end{minitest}
\begin{minitest}
\begin{upotreba}{3}
#\naslovInt#
#\izlaz{Unesite pozitivan broj:}\ulaz{40}#
#\izlaz{Broj je veliki, dolazi do prekoracenja.}#
\end{upotreba}
\end{minitest}

 \linkresenje{1.3_5}
\end{Exercise}
\ifresenja
\end{Answer}[ref=1.3_5]
\includecode{resenja/1_KontrolaToka/1.3_Petlje/1.3_5.c}
\end{Answer}
\fi


\begin{Exercise}[label=1.3_6] 
 Sa standradnog ulaza unose se realan broj $x$ i ceo pozitivan broj
 $n$. Napisati program koji izračunava $n$-ti stepen broja $x$,
 tj.~$x^n$.
 
\begin{minitest}
\begin{upotreba}{1}
#\naslovInt#
#\izlaz{Unesite redom brojeve x i n:}\ulaz{4 3}#
#\izlaz{64.00000}#
\end{upotreba}
\end{minitest}
\begin{minitest}
\begin{upotreba}{2}
#\naslovInt#
#\izlaz{Unesite redom brojeve x i n:}\ulaz{5.8 5}#
#\izlaz{6563.56768}#
\end{upotreba}
\end{minitest}
\begin{minitest}
\begin{upotreba}{3}
#\naslovInt#
#\izlaz{Unesite redom brojeve x i n:}\ulaz{11.43 0}#
#\izlaz{1.00000}#
\end{upotreba}
\end{minitest}
\linkresenje{1.3_6}
\end{Exercise}
\ifresenja
\end{Answer}[ref=1.3_6]
  \includecode{resenja/1_KontrolaToka/1.3_Petlje/1.3_6.c}
\end{Answer}
\fi

\begin{Exercise}[label=1.3_7]
 Sa standradnog ulaza unose se realan broj $x$ i ceo broj
 $n$. Napisati program koji izračunava $n$-ti stepen broja $x$. 
 
\begin{miditest}
\begin{upotreba}{1}
#\naslovInt#
#\izlaz{Unesite redom brojeve x i n:}\ulaz{2 -3}#
#\izlaz{0.125}#
\end{upotreba}
\end{miditest}
\begin{miditest}
\begin{upotreba}{2}
#\naslovInt#
#\izlaz{Unesite redom brojeve x i n:}\ulaz{-3 2}#
#\izlaz{9.000}#
\end{upotreba}
\end{miditest}
\linkresenje{1.3_7}
\end{Exercise}
\ifresenja
\end{Answer}[ref=1.3_7]
\includecode{resenja/1_KontrolaToka/1.3_Petlje/1.3_7.c}
\end{Answer}
\fi



\begin{Exercise}[label=1.3_8] 
Pravi delioci celog broja su svi delioci sem jedinice i samog tog
broja.  Napisati program za uneti ceo pozitivan broj $x$
ispisuje sve njegove prave delioce. U slučaju greške pri unosu
podataka ispisati odgovarajuću poruku.  

\begin{miditest}
\begin{upotreba}{1}
#\naslovInt#
#\izlaz{Unesite ceo broj veci od 0:}\ulaz{100}#
#\izlaz{2 4 5 10 20 25 50}#
\end{upotreba}
\end{miditest}
\begin{miditest}
\begin{upotreba}{2}
#\naslovInt#
#\izlaz{Unesite ceo broj:}\ulaz{-6}#
#\izlaz{neispravan unos.}#
\end{upotreba}
\end{miditest}
\linkresenje{1.3_8}
\end{Exercise}
\ifresenja
\end{Answer}[ref=1.3_8]
\includecode{resenja/1_KontrolaToka/1.3_Petlje/1.3_8.c}
\end{Answer}
\fi

\begin{Exercise}[label=1.3_9] 
 Napisati program koji za uneti prirodan broj uklanja sve nule
 sa njegove desne strane. Ispisati novodobijeni broj. 
 
\begin{minitest}
\begin{upotreba}{1}
#\naslovInt#
#\izlaz{Unesite broj:}\ulaz{12000}#
#\izlaz{12}#
\end{upotreba}
\end{minitest}
\begin{minitest}
\begin{upotreba}{2}
#\naslovInt#
#\izlaz{Unesite broj:}\ulaz{856}#
#\izlaz{856}#
\end{upotreba}
\end{minitest}
\begin{minitest}
\begin{upotreba}{3}
#\naslovInt#
#\izlaz{Unesite broj:}\ulaz{140}#
#\izlaz{14}#
\end{upotreba}
\end{minitest}
\linkresenje{1.3_9}
\end{Exercise}
\ifresenja
\end{Answer}[ref=1.3_9]
\includecode{resenja/1_KontrolaToka/1.3_Petlje/1.3_9.c}
\end{Answer}
\fi


\begin{Exercise}[label=1.3_10] 
Napisati program koji učitava ceo broj i ispisuje njegove cifre u
obrnutom poretku. 

\begin{miditest}
\begin{upotreba}{1}
#\naslovInt#
#\izlaz{Unesite ceo broj:}\ulaz{6789}#
#\izlaz{9 8 7 6}#
\end{upotreba}
\end{miditest}
\begin{miditest}
\begin{upotreba}{2}
#\naslovInt#
#\izlaz{Unesite ceo broj:}\ulaz{-892345}#
#\izlaz{5 4 3 2 9 8}#
\end{upotreba}
\end{miditest}
\linkresenje{1.3_10}
\end{Exercise}
\ifresenja
\end{Answer}[ref=1.3_10]
\includecode{resenja/1_KontrolaToka/1.3_Petlje/1.3_10.c}
\end{Answer}
\fi



\begin{Exercise}[label=1.3_11] 
Napisati program koji za uneti prirodan broj ispisuje da li je on deljiv
sumom svojih cifara.

\begin{minitest}
\begin{upotreba}{1}
#\naslovInt#
#\izlaz{Unesite broj:}\ulaz{12}#
#\izlaz{Deljiv je sumom svojih cifara.}#
\end{upotreba}
\end{minitest}
\begin{minitest}
\begin{upotreba}{2}
#\naslovInt#
#\izlaz{Unesite broj:}\ulaz{2564}#
#\izlaz{Nije deljiv sumom svojih cifara.}#
\end{upotreba}
\end{minitest}
\begin{minitest}
\begin{upotreba}{3}
#\naslovInt#
#\izlaz{Unesite broj:}\ulaz{-4}#
#\izlaz{Neispravan ulaz.}#
\end{upotreba}
\end{minitest}

\begin{minitest}
\begin{upotreba}{4}
#\naslovInt#
#\izlaz{Unesite broj:}\ulaz{0}#
#\izlaz{Neispravan ulaz.}#
\end{upotreba}
\end{minitest}
\linkresenje{1.3_11}
\end{Exercise}
\ifresenja
\end{Answer}[ref=1.3_11]
\includecode{resenja/1_KontrolaToka/1.3_Petlje/1.3_11.c}
\end{Answer}
\fi


\begin{Exercise}[label=1.3_12] 
Napisati program koji učitava pozitivan ceo broj $n$, a zatim učitava
$n$ celih brojeva i ispisuje sumu pozitivnih i sumu negativnih unetih
brojeva.

\begin{minitest}
\begin{upotreba}{1}
#\naslovInt#
#\izlaz{Unesite broj:}\ulaz{7}#
#\izlaz{Unesite brojeve:}#
#\ulaz{8 -50 45 2007 -67 -123 14}#
#\izlaz{Suma pozitivnih: 2074}#
#\izlaz{Suma negativnih: -240}#
\end{upotreba}
\end{minitest}
\begin{minitest}
\begin{upotreba}{2}
#\naslovInt#
#\izlaz{Unesite broj:}\ulaz{5}#
#\izlaz{Unesite brojeve:}#
#\ulaz{-5 -20 -4 -200 -8}#
#\izlaz{Suma pozitivnih: 0}#
#\izlaz{Suma negativnih: -237}#
\end{upotreba}
\end{minitest}
\begin{minitest}
\begin{upotreba}{3}
#\naslovInt#
#\izlaz{Unesite broj:}\ulaz{-6}#
#\izlaz{Neispravan unos.}#
\end{upotreba}
\end{minitest}
\linkresenje{1.3_12}
\end{Exercise}
\ifresenja
\end{Answer}[ref=1.3_12]
\includecode{resenja/1_KontrolaToka/1.3_Petlje/1.3_12.c}
\end{Answer}
\fi

\begin{Exercise}[label=1.3_13] 
Program unosi ceo pozitivan broj $n$, a potom i $n$ celih
brojeva. Izračunati i ispisati zbir onih brojeva koji su neparni i
negativni.

\begin{minitest}
\begin{upotreba}{1}
#\naslovInt#
#\izlaz{Unesite broj n:}\ulaz{5}#
#\izlaz{Unesite n brojeva:}#
#\ulaz{1 -5 -6 3 -11}#
#\izlaz{-16}#
\end{upotreba}
\end{minitest}
\begin{minitest}
\begin{upotreba}{2}
#\naslovInt#
#\izlaz{Unesite broj n:}\ulaz{4}#
#\izlaz{Unesite n brojeva:}#
#\ulaz{5 8 13 17}#
#\izlaz{0}#
\end{upotreba}
\end{minitest}
\begin{minitest}
\begin{upotreba}{3}
#\naslovInt#
#\izlaz{Unesite broj n:}\ulaz{-4}#
#\izlaz{Neispravan unos.}#
\end{upotreba}
\end{minitest}
\linkresenje{1.3_13}
\end{Exercise}
\ifresenja
\end{Answer}[ref=1.3_13]
\includecode{resenja/1_KontrolaToka/1.3_Petlje/1.3_13.c}
\end{Answer}
\fi

\begin{Exercise}[label=1.3_14] 
Program učitava ceo pozitivan broj $n$, a potom $n$ celih brojeva.
Naći sumu brojeva koji su deljivi sa $5$, a nisu deljivi sa $7$. U
slučaju greške pri unosu podataka ispisati odgovarajuću poruku.

\begin{miditest}
\begin{upotreba}{1}
#\naslovInt#
#\izlaz{Unesite broj n:}\ulaz{5}#
#\izlaz{Unesite brojeve:}\ulaz:{2 35 5 -175 -20 }#
#\izlaz{Suma je -15.}#
\end{upotreba}
\end{miditest}
\begin{miditest}
\begin{upotreba}{2}
#\naslovInt#
#\izlaz{Unesite broj n:}\ulaz{-3}#
#\izlaz{Neispravan unos.}#
\end{upotreba}
\end{miditest}

\begin{miditest}
\begin{upotreba}{3}
#\naslovInt#
#\izlaz{Unesite broj n:}\ulaz{10}#
#\izlaz{Unesite brojeve:}#
#\ulaz{-5 6 175 -20 -25 -8 42 245 1 6}#
#\izlaz{Suma je -50.}#
\end{upotreba}
\end{miditest}
\begin{miditest}
\begin{upotreba}{4}
#\naslovInt#
#\izlaz{Unesite broj n:}\ulaz{6}#
#\izlaz{Unesite brojeve:}#
#\ulaz{2205 -1904 2 7 -540 5}#
#\izlaz{Suma je -535.}#
\end{upotreba}
\end{miditest}
\linkresenje{1.3_14}
\end{Exercise}
\ifresenja
\end{Answer}[ref=1.3_14]
\includecode{resenja/1_KontrolaToka/1.3_Petlje/1.3_14.c}
\end{Answer}
\fi

\iffalse
\begin{Exercise}[label=1.3_14_5] 
Broj je prost ako je deljiv samo sa $1$ i sa samim sobom.
Program učitava ceo broj $n$, a potom proverava da li je broj 
prost i ispisuje odgovarajuću poruku.

\begin{miditest}
\begin{upotreba}{1}
#\naslovInt#
#\izlaz{Unesite broj n:}\ulaz{5}#
#\izlaz{Broj je prost.}#
\end{upotreba}
\end{miditest}
\begin{miditest}
\begin{upotreba}{2}
#\naslovInt#
#\izlaz{Unesite broj n:}\ulaz{100}#
#\izlaz{Broj nije prost.}#
\end{upotreba}
\end{miditest}

\begin{miditest}
\begin{upotreba}{3}
#\naslovInt#
#\izlaz{Unesite broj n:}\ulaz{-17}#
#\izlaz{Broj nije prost.}#
\end{upotreba}
\end{miditest}
\begin{miditest}
\begin{upotreba}{4}
#\naslovInt#
#\izlaz{Unesite broj n:}\ulaz{101}#
#\izlaz{Broj je prost.}#
\end{upotreba}
\end{miditest}
%\linkresenje{1.3_14_5}
\end{Exercise}
%\ifresenja
\end{Answer}[ref=1.3_14_5]
%\includecode{resenja/1_KontrolaToka/1.3_Petlje/1.3_14_5.c}
%\end{Answer}
\fi
\fi

\begin{Exercise}[label=1.3_15] 
Nikola želi da obraduje baku i da joj kupi jedan poklon u radnji. On
na raspolaganju ima $m$ novaca. U radnji se nalazi $n$ artikala i
zanima ga koliko ima artikala u radnji čija cena je manja ili
jednaka $m$. Napisati program koji pomaže Nikoli da brzo odrediti
broj atikala. Program učitava realan pozitivan broj $m$, ceo
pozitivan broj $n$ i $n$ realnih pozitivnih brojeva različitih
od $0$. Ispisati koliko artikala ima manju ili jednaku cenu od $m$. U
slučaju greške ispisati odgovarajuću poruku. 

\begin{miditest}
\begin{upotreba}{1}
#\naslovInt#
#\izlaz{Unesite broj m:}\ulaz{12.37}#
#\izlaz{Unesite broj n:}\ulaz{5}#
#\izlaz{Unesite n brojeva:}\ulaz{11 54.13 6 13 8}#
#\izlaz{3}#
\end{upotreba}
\end{miditest}
\begin{miditest}
\begin{upotreba}{2}
#\naslovInt#
#\izlaz{Unesite broj m:}\ulaz{2}#
#\izlaz{Unesite broj n:}\ulaz{4}#
#\izlaz{Unesite n brojeva:}\ulaz{1 11 4.32 3}#
#\izlaz{1}#
\end{upotreba}
\end{miditest}

\begin{miditest}
\begin{upotreba}{3}
#\naslovInt#
#\izlaz{Unesite broj m:}\ulaz{2}#
#\izlaz{Unesite broj n:}\ulaz{-4}#
#\izlaz{Broj artikala ne moze biti negativan.}#
\end{upotreba}
\end{miditest}
\begin{miditest}
\begin{upotreba}{4}
#\naslovInt#
#\izlaz{Unesite broj m:}\ulaz{30}#
#\izlaz{Unesite broj n:}\ulaz{4}#
#\izlaz{Unesite n brojeva:}\ulaz{67 -100 23 98}#
#\izlaz{Cena ne moze biti negativna.}# 
\end{upotreba}
\end{miditest}
\linkresenje{1.3_15}
\end{Exercise}
\ifresenja
\end{Answer}[ref=1.3_15]
\includecode{resenja/1_KontrolaToka/1.3_Petlje/1.3_15.c}
\end{Answer}
\fi



\begin{Exercise}[label=1.3_16] 
Napisati program koji učitava cele brojeve sve dok se ne unese
nula. Nakon toga ispisati proizvod onih unetih brojeva koji su
pozitivni.  

\begin{minitest}
\begin{upotreba}{1}
#\naslovInt#
#\izlaz{Unesite brojeve:}#
#\ulaz{-87 12 -108 -13 56 0}#
#\izlaz{Proizvod pozitivnih unetih brojeva je 672.}#
\end{upotreba}
\end{minitest}
\begin{minitest}
\begin{upotreba}{2}
#\naslovInt#
#\izlaz{Unesite brojeve:}#
#\ulaz{-5 -200 -43 0}#
#\izlaz{Nisu uneseni pozitivni brojevi.}#
\end{upotreba}
\end{minitest}
\begin{minitest}
\begin{upotreba}{2}
#\naslovInt#
#\izlaz{Unesite brojeve:}\ulaz{0}#
#\izlaz{Nisu uneseni brojevi.}#
\end{upotreba}
\end{minitest}
\linkresenje{1.3_16}
\end{Exercise}
\ifresenja
\end{Answer}[ref=1.3_16]
\includecode{resenja/1_KontrolaToka/1.3_Petlje/1.3_16.c}
\end{Answer}
\fi

\begin{Exercise}[label=1.3_17] 
 Napisati program koji za pozitivan ceo broj proverava i ispisuje da
 li se cifra 5 nalazi u njegovom zapisu.

\begin{minitest}
\begin{upotreba}{1}
#\naslovInt#
#\izlaz{Unesite broj:}\ulaz{1857}#
#\izlaz{Cifra 5 se nalazi u zapisu!}#
\end{upotreba}
\end{minitest}
\begin{minitest}
\begin{upotreba}{2}
#\naslovInt#
#\izlaz{Unesite broj:}\ulaz{84}#
#\izlaz{Cifra 5 se ne nalazi u zapisu!}#
\end{upotreba}
\end{minitest}
\begin{minitest}
\begin{upotreba}{3}
#\naslovInt#
#\izlaz{Unesite broj:}\ulaz{-235515}#
#\izlaz{Cifra 5 se nalazi u zapisu!}#
\end{upotreba}
\end{minitest}
\linkresenje{1.3_17}
\end{Exercise}
\ifresenja
\end{Answer}[ref=1.3_17]
\includecode{resenja/1_KontrolaToka/1.3_Petlje/1.3_17.c}
\end{Answer}
\fi



\begin{Exercise}[label=1.3_18] 
Program učitava cele brojeve sve do unosa broja nula $0$. Napisati
program koji izračunava i ispisuje aritmetičku sredinu unetih brojeva
na četiri decimale.

\begin{miditest}
\begin{upotreba}{1}
#\naslovInt#
#\izlaz{Unesite brojeve:}\ulaz{8 5 6 3 0}#
#\izlaz{Aritmeticka sredina: 5.5000}#
\end{upotreba}
\end{miditest}
\begin{miditest}
\begin{upotreba}{2}
#\naslovInt#
#\izlaz{Unesite brojeve:}\ulaz{762 -12 800 2010 -356 899 -101 0}#
#\izlaz{Aritmeticka sredina: 571.7143}#
\end{upotreba}
\end{miditest}

\begin{miditest}
\begin{upotreba}{3}
#\naslovInt#
#\izlaz{Unesite brojeve:}\ulaz{0}#
#\izlaz{Nisu uneseni brojevi.}#
\end{upotreba}
\end{miditest}
\linkresenje{1.3_18}
\end{Exercise}
\ifresenja
\end{Answer}[ref=1.3_18]
\includecode{resenja/1_KontrolaToka/1.3_Petlje/1.3_18.c}
\end{Answer}
\fi


\begin{Exercise}[label=1.3_19] 
U prodavnici se nalaze artikala čije cene su realni pozitivni
brojevi. Program unosi cene artikala sve do unosa broja nula
$0$. Napisati program koji izračunava i ispisuje prose\v cnu vrednost
cena u radnji.

\begin{miditest}
\begin{upotreba}{1}
#\naslovInt#
#\izlaz{Unesite cene:}\ulaz{8 5.2 6.11 3 0}#
#\izlaz{Prosecna cena je: 5.5775}#
\end{upotreba}
\end{miditest}
\begin{miditest}
\begin{upotreba}{2}
#\naslovInt#
#\izlaz{Unesite cene:}\ulaz{6.32 -9}#
#\izlaz{Cena ne moze biti negativana.}#
\end{upotreba}
\end{miditest}

\begin{miditest}
\begin{upotreba}{3}
#\naslovInt#
#\izlaz{Unesite cene:}\ulaz{0}#
#\izlaz{Nisu unesene cene.}#
\end{upotreba}
\end{miditest}
\linkresenje{1.3_19}
\end{Exercise}
\ifresenja
\end{Answer}[ref=1.3_19]
\includecode{resenja/1_KontrolaToka/1.3_Petlje/1.3_19.c}
\end{Answer}
\fi









\begin{Exercise}[label=1.3_20] 
Program učitava ceo pozitivan broj $n$, a potom $n$ realnih
brojeva. Odrediti koliko puta je prilikom unosa došlo do promene
znaka. Ispisati dobijenu vrednost.

\begin{miditest}
\begin{upotreba}{1}
#\naslovInt#
#\izlaz{Unesite broj n:}\ulaz{10}#
#\izlaz{Unesite brojeve:}#
#\ulaz{7.82 4.3 -1.2 56.8 -3.4 -72.1 8.9 11.2 -11.2 -102.4}#
#\izlaz{Broj promena je 5.}#
\end{upotreba}
\end{miditest}
\begin{miditest}
\begin{upotreba}{2}
#\naslovInt#
#\izlaz{Unesite broj n:}\ulaz{5}#
#\izlaz{Unesite brojeve:}#
#\ulaz{-23.8 -11.2 0 5.6 7.2}#
#\izlaz{Broj promena je 1.}#
\end{upotreba}
\end{miditest}

\begin{miditest}
\begin{upotreba}{3}
#\naslovInt#
#\izlaz{Unesite broj n:}\ulaz{-6}#
#\izlaz{Neispravan unos.}#
\end{upotreba}
\end{miditest}
\begin{miditest}
\begin{upotreba}{4}
#\naslovInt#
#\izlaz{Unesite broj n:}\ulaz{0}#
#\izlaz{Broj promena je 0.}#
\end{upotreba}
\end{miditest}
\linkresenje{1.3_20}
\end{Exercise}
\ifresenja
\end{Answer}[ref=1.3_20]
\includecode{resenja/1_KontrolaToka/1.3_Petlje/1.3_20.c}
\end{Answer}
\fi

\begin{Exercise}[label=1.3_21] 
U prodavnici se nalazi $n$ artikala čije cene su realni
brojevi. Napisati program koji učitava $n$, a potom i cenu svakog od
$n$ artikala i određuje i ispisuje najmanju cenu.

\begin{minitest}
\begin{upotreba}{1}
#\naslovInt#
#\izlaz{Unesite broj artikla:}\ulaz{6}#
#\izlaz{Unesite artikle:}#
#\ulaz{12 3.4 90 100.53 53.2 12.8}#
#\izlaz{Minimalna cena je: 3.400000}#
\end{upotreba}
\end{minitest}
\begin{minitest}
\begin{upotreba}{2}
#\naslovInt#
#\izlaz{Unesite broj artikla:}\ulaz{3}#
#\izlaz{Unesite artikle:}\ulaz{4 -8 92}#
#\izlaz{Cena ne moze biti negativna.}#
\end{upotreba}
\end{minitest}
\begin{minitest}

\begin{upotreba}{3}
#\naslovInt#
#\izlaz{Unesite broj artikla:}\ulaz{-9}#
#\izlaz{Neispravan unos.}#
\end{upotreba}
\end{minitest}
\linkresenje{1.3_21}
\end{Exercise}
\ifresenja
\end{Answer}[ref=1.3_21]
\includecode{resenja/1_KontrolaToka/1.3_Petlje/1.3_21.c}
\end{Answer}
\fi

\begin{Exercise}[label=1.3_22] 
Program učitava ceo pozitivan broj $n$, a zatim i $n$ celih
brojeva. Napisati program koji ispisuje broj sa najvećom cifrom
desetica. Ukoliko ima više takvih, ispisati prvi.

\begin{miditest}
\begin{upotreba}{1}
#\naslovInt#
#\izlaz{Unesite broj n:}\ulaz{5}#
#\izlaz{Unesite brojeve:}#
#\ulaz{18 365 25 1 78}#
#\izlaz{Broj sa najvecom cifrom desetica je 78.}#
\end{upotreba}
\end{miditest}
\begin{miditest}
\begin{upotreba}{2}
#\naslovInt#
#\izlaz{Unesite broj n:}\ulaz{8}#
#\izlaz{Unesite brojeve:}#
#\ulaz{14 1576 -1267 -89 109 122 306 918}#
#\izlaz{Broj sa najvecom cifrom desetica je -89.}#
\end{upotreba}
\end{miditest}

\begin{miditest}
\begin{upotreba}{3}
#\naslovInt#
#\izlaz{Unesite broj n:}\ulaz{-12}#
#\izlaz{Neispravan unos.}#
\end{upotreba}
\end{miditest}
\linkresenje{1.3_22}
\end{Exercise}
\ifresenja
\end{Answer}[ref=1.3_22]
\includecode{resenja/1_KontrolaToka/1.3_Petlje/1.3_22.c}
\end{Answer}
\fi

\begin{Exercise}[label=1.3_23] 
Program učitava ceo pozitivan broj $n$, a zatim i $n$ celih
brojeva. Napisati program koji ispisuje broj sa najvećim brojem
cifara. Ukoliko ima više takvih, ispisati prvi.

\begin{miditest}
\begin{upotreba}{1}
#\naslovInt#
#\izlaz{Unesite broj n:}\ulaz{5}#
#\izlaz{Unesite n brojeva:}\ulaz{18 365 25 1 78}#
#\izlaz{Najvise cifara ima broj 365.}#
\end{upotreba}
\end{miditest}
\begin{miditest}
\begin{upotreba}{2}
#\naslovInt#
#\izlaz{Unesite broj n:}\ulaz{7}#
#\izlaz{Unesite n brojeva:}#
#\ulaz{3 892 18 21 639 742 85}#
#\izlaz{Najvise cifara ima broj 892.}#
\end{upotreba}
\end{miditest}
\linkresenje{1.3_23}
\end{Exercise}
\ifresenja
\end{Answer}[ref=1.3_23]
\includecode{resenja/1_KontrolaToka/1.3_Petlje/1.3_23.c}
\end{Answer}
\fi

\begin{Exercise}[label=1.3_24] 
Program učitava ceo pozitivan broj $n$, a zatim i $n$ celih
brojeva. Napisati program koji ispisuje broj sa najvećom vodećom
cifrom. Vodeća cifra je cifra najveće težine u zapisu broja. Ukoliko
ima više takvih, ispisati prvi.

\begin{miditest}
\begin{upotreba}{1}
#\naslovInt#
#\izlaz{Unesite broj n:}\ulaz{5}#
#\izlaz{Unesite n brojeva:}\ulaz{8 964 32 511 27}#
#\izlaz{Broj sa najvecom vodecom cifrom je 964.}#
\end{upotreba}
\end{miditest}
\begin{miditest}
\begin{upotreba}{1}
#\naslovInt#
#\izlaz{Unesite broj n:}\ulaz{3}#
#\izlaz{Unesite n brojeva:}\ulaz{41 669 8}#
#\izlaz{Broj sa najvecom vodecom cifrom je 8.}#
\end{upotreba}
\end{miditest}
\linkresenje{1.3_24}
\end{Exercise}
\ifresenja
\end{Answer}[ref=1.3_24]
\includecode{resenja/1_KontrolaToka/1.3_Petlje/1.3_24.c}
\end{Answer}
\fi


\begin{Exercise}[label=1.3_25] 
Vršna su merenja nadmorskih visina na određenom delu teritorije i
naučnike zanima razlika između najveće i najmanje nadmorske
visine. Napisati program koji učitava realne brojeve sve do unosa $0$
koji označavaju nadmorske visine i ispisuje razliku najveće i najmanje
nadmorske visine.

\begin{miditest}
\begin{upotreba}{1}
#\naslovInt#
#\izlaz{Unesite brojeve:}\ulaz{8 6 5 2 11 7 0}#
#\izlaz{Razlika: 9}#
\end{upotreba}
\end{miditest}
\begin{miditest}
\begin{upotreba}{2}
#\naslovInt#
#\izlaz{Unesite brojeve:}\ulaz{8 -1 8 6 0}#
#\izlaz{Razlika: 9}#
\end{upotreba}
\end{miditest}
\linkresenje{1.3_25}
\end{Exercise}
\ifresenja
\end{Answer}[ref=1.3_25]
\includecode{resenja/1_KontrolaToka/1.3_Petlje/1.3_25.c}
\end{Answer}
\fi



\begin{Exercise}[label=1.3_26] 
Program učitava cele pozitivane brojeve $n$ ($n>1$) i $d$, a zatim i
$n$ celih brojeva. Napisati program koji izračunava koliko ima parova
uzastopnih brojeva među unetim brojevima koji se nalaze na rastojanju
$d$. Rastojanje između brojeva je definisano sa
$d(x,y)=|y-x|$. Ispisati rezultat.

\begin{miditest}
\begin{upotreba}{1}
#\naslovInt#
#\izlaz{Unesite brojeve n i d:}\ulaz{5 2}#
#\izlaz{Unesite n brojeva:}\ulaz{2 3 5 1 -1}#
#\izlaz{Broj parova: 2}#
\end{upotreba}
\end{miditest}
\begin{miditest}
\begin{upotreba}{2}
#\naslovInt#
#\izlaz{Unesite brojeve n i d:}\ulaz{10 5}#
#\izlaz{Unesite n brojeva:}#
#\ulaz{-3 6 11 -20 -25 -8 42 37 1 6}#
#\izlaz{Broj parova: 4}#
\end{upotreba}
\end{miditest}
\linkresenje{1.3_26}
\end{Exercise}
\ifresenja
\end{Answer}[ref=1.3_26]
\includecode{resenja/1_KontrolaToka/1.3_Petlje/1.3_26.c}
\end{Answer}
\fi



\begin{Exercise}[label=1.3_27] 
Napisati program koji uneti prirodan broj transformiše tako što svaku
parnu cifru u zapisu broja uveća za jedan. Ispisati novodobijeni broj.

\begin{minitest}
\begin{upotreba}{1}
#\naslovInt#
#\izlaz{Unesite broj:}\ulaz{2417}#
#\izlaz{3517}#
\end{upotreba}
\end{minitest}
\begin{minitest}
\begin{upotreba}{2}
#\naslovInt#
#\izlaz{Unesite broj:}\ulaz{138}#
#\izlaz{139}#
\end{upotreba}
\end{minitest}
\begin{minitest}
\begin{upotreba}{3}
#\naslovInt#
#\izlaz{Unesite broj:}\ulaz{59}#
#\izlaz{59}#
\end{upotreba}
\end{minitest}
\linkresenje{1.3_27}
\end{Exercise}
\ifresenja
\end{Answer}[ref=1.3_27]
\includecode{resenja/1_KontrolaToka/1.3_Petlje/1.3_27.c}
\end{Answer}
\fi


\begin{Exercise}[label=1.3_28]
  Napisati program koji formira i ispisuje broj koji se dobija
  izbacivanjem svake druge cifre polaznog celog broja, počevši od
  krajnje desne cifre.
 
\begin{miditest}
\begin{upotreba}{1}
#\naslovInt#
#\izlaz{Unesite broj:}\ulaz{21854}#
#\izlaz{284}#
\end{upotreba}
\end{miditest}
\begin{miditest}
\begin{upotreba}{2}
#\naslovInt#
#\izlaz{Unesite broj:}\ulaz{18}#
#\izlaz{8}#
\end{upotreba}
\end{miditest}

\begin{miditest}
\begin{upotreba}{3}
#\naslovInt#
#\izlaz{Unesite broj:}\ulaz{1}#
#\izlaz{1}#
\end{upotreba}
\end{miditest}
\begin{miditest}
\begin{upotreba}{4}
#\naslovInt#
#\izlaz{Unesite broj:}\ulaz{-67123}#
#\izlaz{-613}#
\end{upotreba}
\end{miditest}
\linkresenje{1.3_28}
\end{Exercise}
\ifresenja
\end{Answer}[ref=1.3_28]
\includecode{resenja/1_KontrolaToka/1.3_Petlje/1.3_28.c}
\end{Answer}
\fi

\begin{Exercise}[difficulty=1, label=1.3_29] 
 Napisati program koji na osnovu unetog prirodnog broja formira i
 ispisuje broj koji se dobija izbacivanjem cifara koje su jednake
 zbiru svojih suseda.

\begin{minitest}
\begin{upotreba}{1}
#\naslovInt#
#\izlaz{Unesite broj:}\ulaz{28631}#
#\izlaz{2631}#
\end{upotreba}
\end{minitest}
\begin{minitest}
\begin{upotreba}{2}
#\naslovInt#
#\izlaz{Unesite broj:}\ulaz{440}#
#\izlaz{40}#
\end{upotreba}
\end{minitest}
\begin{minitest}
\begin{upotreba}{3}
#\naslovInt#
#\izlaz{Unesite broj:}\ulaz{-5}#
#\izlaz{Neispravan unos.}#
\end{upotreba}
\end{minitest}
\linkresenje{1.3_29}
\end{Exercise}
\ifresenja
\end{Answer}[ref=1.3_29]
\includecode{resenja/1_KontrolaToka/1.3_Petlje/1.3_29.c}
\end{Answer}
\fi

\begin{Exercise}[difficulty=1, label=1.3_30] 
Broj je \textit{palindrom} ukoliko se isto čita i sa leve i sa desne
strane. Napisati program koji učitava prirodan broj i proverava da li
je učitani broj palindrom.

\begin{minitest}
\begin{upotreba}{1}
#\naslovInt#
#\izlaz{Unesite broj:}\ulaz{25452}#
#\izlaz{Broj je palindrom!}#
\end{upotreba}
\end{minitest}
\begin{minitest}
\begin{upotreba}{2}
#\naslovInt#
#\izlaz{Unesite broj:}\ulaz{895}#
#\izlaz{Broj nije palindrom!}#
\end{upotreba}
\end{minitest}
\begin{minitest}
\begin{upotreba}{3}
#\naslovInt#
#\izlaz{Unesite broj:}\ulaz{5}#
#\izlaz{Broj je palindrom!}#
\end{upotreba}
\end{minitest}
\linkresenje{1.3_30}
\end{Exercise}
\ifresenja
\end{Answer}[ref=1.3_30]
\includecode{resenja/1_KontrolaToka/1.3_Petlje/1.3_30.c}
\end{Answer}
\fi


\begin{Exercise}[label=1.3_31] 
%\komentar{Uskladiti formulaciju zadatka sa odgovarajućom formulacijom kod nizova.}
Fibonačijev niz počinje ciframa $1$ i $1$, a svaki član se dobija
zbirom prethodna dva. Napisati program koji učitava ceo prirodan broj
$n$ i određuje i ispisuje $n$-ti član Fibonačijevog niza.

\begin{miditest}
\begin{upotreba}{1}
#\naslovInt#
#\izlaz{Unesite ceo broj:}\ulaz{10}#
#\izlaz{Trazeni broj je: 55}#
\end{upotreba}
\end{miditest}
\begin{miditest}
\begin{upotreba}{2}
#\naslovInt#
#\izlaz{Unesite ceo broj:}\ulaz{-100}#
#\izlaz{Neispravan unos. Pozicija u Fibonacijevom}#
#\izlaz{nizu mora biti pozitivan broj koji nije 0!}#
\end{upotreba}
\end{miditest}

\begin{miditest}
\begin{upotreba}{3}
#\naslovInt#
#\izlaz{Unesite ceo broj:}\ulaz{78}#
#\izlaz{Trazeni broj je: 375819880}#
\end{upotreba}
\end{miditest}
\begin{miditest}
\begin{upotreba}{4}
#\naslovInt#
#\izlaz{Unesite ceo broj:}\ulaz{20}#
#\izlaz{Trazeni broj je: 6765}#
\end{upotreba}
\end{miditest}
\linkresenje{1.3_31}
\end{Exercise}
\ifresenja
\end{Answer}[ref=1.3_31]
\includecode{resenja/1_KontrolaToka/1.3_Petlje/1.3_31.c}
\end{Answer}
\fi


\begin{Exercise}[difficulty=1, label=1.3_32] 
Niz prirodnih brojeva formira se prema sledećem pravilu:
\begin{equation*}
a_{n+1} = \left\{
\begin{array}{rl}
\frac{a_n}{2} & \text{ako je } a_n \text{ parno}\\
\frac{3\cdot a_n + 1}{2} & \text{ako je } a_n \text{ neparno}\\
\end{array} \right.
\end{equation*}
Napisati program koji za uneti početni član niza $a_0$ (ceo pozitivan
broj) štampa niz brojeva sve do onog člana niza koji je jednak $1$. 


\begin{miditest}
\begin{upotreba}{1}
#\naslovInt#
#\izlaz{Unesite ceo broj:}\ulaz{56}#
#\izlaz{56 28 14 7 11 17 26 13 20 10}#
#\izlaz{5 8 4 2 1}#
\end{upotreba}
\end{miditest}
\begin{miditest}
\begin{upotreba}{2}
#\naslovInt#
#\izlaz{Unesite ceo broj:}\ulaz{-48}#
#\izlaz{Nekorektan unos. Broj mora biti pozitivan.}#
\end{upotreba}
\end{miditest}

\begin{miditest}
\begin{upotreba}{3}
#\naslovInt#
#\izlaz{Unesite ceo broj:}\ulaz{67}#
#\izlaz{67 101 152 76 38 19 29 44 22 11}#
#\izlaz{17 26 13 20 10 5 8 4 2 1}#
\end{upotreba}
\end{miditest}
\begin{miditest}
\begin{upotreba}{4}
#\naslovInt#
#\izlaz{Unesite ceo broj:}\ulaz{33}#
#\izlaz{33 50 25 38 19 29 44 22}#
#\izlaz{11 17 26 13 20 10 5 8 4 2 1}#
\end{upotreba}
\end{miditest}
\linkresenje{1.3_32}
\end{Exercise}
\ifresenja
\end{Answer}[ref=1.3_32]
\includecode{resenja/1_KontrolaToka/1.3_Petlje/1.3_32.c}
\end{Answer}
\fi


\begin{Exercise}[difficulty=1, label=1.3_33] 
Papir $A_0$ ima povr\v sinu 1$m^2$ i odnos stranica
$1:\sqrt{2}$. Papir $A_1$ dobija se podelom papira $A_0$ po dužoj
ivici. Papir $A_2$ dobija se podelom $A_1$ papira po dužoj ivici
itd. Napisati program koji za uneti prirodan broj $k$ ispisuje 
dimenzije papira $A_k$ u milimetrima. Rezultat ispisati kao celobrojne
vrednosti.  
  
\begin{miditest}
\begin{upotreba}{1}
#\naslovInt#
#\izlaz{Unesite format papira:}\ulaz{4}#
#\izlaz{210 297}#
\end{upotreba}
\end{miditest}
\begin{miditest}
\begin{upotreba}{2}
#\naslovInt#
#\izlaz{Unesite format papira:}\ulaz{3}#
#\izlaz{297 420}#
\end{upotreba}
\end{miditest}

\begin{miditest}
\begin{upotreba}{3}
#\naslovInt#
#\izlaz{Unesite format papira:}\ulaz{7}#
#\izlaz{74 105}#
\end{upotreba}
\end{miditest}
\begin{miditest}
\begin{upotreba}{4}
#\naslovInt#
#\izlaz{Unesite format papira:}\ulaz{9}#
#\izlaz{37 52}#
\end{upotreba}
\end{miditest}
\linkresenje{1.3_33}
\end{Exercise}
\ifresenja
\end{Answer}[ref=1.3_33]
\includecode{resenja/1_KontrolaToka/1.3_Petlje/1.3_33.c}
\end{Answer}
\fi


%--------------------------------------------------------------------
%--------------------------------------------------------------------
% \subsection{Rad sa karakterima}
%--------------------------------------------------------------------
%--------------------------------------------------------------------

\begin{Exercise}[label=1.3_34] 
Napisati program koji učitava karaktere dok se ne unese karakter tačka,
i ako je karakter malo slovo ispisuje odgovarajuće veliko, ako je
karakter veliko slovo ispisuje odgovarajuće malo, a u suprotnom
ispisuje isti karakter kao i uneti.

\begin{miditest}
\begin{upotreba}{1}
#\naslovInt#
#\ulaz{Danas je Veoma Lep DAN.}#
#\izlaz{dANAS JE vEOMA lEP dan}#
\end{upotreba}
\end{miditest}
\begin{miditest}
\begin{upotreba}{2}
#\naslovInt#
#\ulaz{PROGRAMIRANJE 1 je zanimljivo!.}#
#\izlaz{programiranje 1 JE ZANIMLJIVO!}#
\end{upotreba}
\end{miditest}
\linkresenje{1.3_34}
\end{Exercise}
\ifresenja
\end{Answer}[ref=1.3_34]
\includecode{resenja/1_KontrolaToka/1.3_Petlje/1.3_34.c}
\end{Answer}
\fi

\begin{Exercise}[label=1.3_35] 
Napisati program koji učitava karaktere sve do kraja ulaza, a potom
ispisuje broj velikih slova, broj malih slova, broj cifara, broj
belina i zbir unetih cifara.

\begin{miditest}
\begin{upotreba}{1}
#\naslovInt#
#\ulaz{Tekst sa brojevima: 124, -8900, 23...}#
#\izlaz{velika: 1, mala: 15, cifre: 9, beline: 5}#
#\izlaz{suma cifara: 29}#
\end{upotreba}
\end{miditest}
\begin{miditest}
\begin{upotreba}{2}
#\naslovInt#
#\ulaz{NEMA cifara!}#
#\izlaz{velika: 4, mala: 6, cifre: 0, beline: 1 }#
#\izlaz{suma cifara: 0}#
\end{upotreba}
\end{miditest}
\linkresenje{1.3_35}
\end{Exercise}
\ifresenja
\end{Answer}[ref=1.3_35]
\includecode{resenja/1_KontrolaToka/1.3_Petlje/1.3_35.c}
\end{Answer}
\fi


\begin{Exercise}[label=1.3_36] 
 Program učitava ceo pozitivan broj $n$, a potom i $n$ karaktera. Za
 svaki od samoglasnika ispisati koliko puta se pojavio među unetim
 karakterima. Ne praviti razliku između malih i velikih slova.
 
\begin{miditest}
\begin{upotreba}{1}
#\naslovInt#
#\izlaz{Unesite broj n:}\ulaz{5}#
#\izlaz{Unesite n karaktera:}\ulaz{uAbao}#
#\izlaz{Samoglasnik a: 2}#
#\izlaz{Samoglasnik e: 0}#
#\izlaz{Samoglasnik i: 0}#
#\izlaz{Samoglasnik o: 1}#
#\izlaz{Samoglasnik u: 1}#
\end{upotreba}
\end{miditest}
\begin{miditest}
\begin{upotreba}{2}
#\naslovInt#
#\izlaz{Unesite broj n:}\ulaz{7}#
#\izlaz{Unesite n karaktera:}\ulaz{jk+EEae}#
#\izlaz{Samoglasnik a: 1}#
#\izlaz{Samoglasnik e: 3}#
#\izlaz{Samoglasnik i: 0}#
#\izlaz{Samoglasnik o: 0}#
#\izlaz{Samoglasnik u: 0}#
\end{upotreba}
\end{miditest}
\linkresenje{1.3_36}
\end{Exercise}
\ifresenja
\end{Answer}[ref=1.3_36]
\includecode{resenja/1_KontrolaToka/1.3_Petlje/1.3_36.c}
\end{Answer}
\fi


\begin{Exercise}[label=1.3_37] 
Program učitava ceo broj $n$, a zatim i $n$ karaktera. Napisati
program koji proverava da li se od unetih karaktera može napisati reč
\textit{Zima}.

\begin{miditest}
\begin{upotreba}{1}
#\naslovInt#
#\izlaz{Unesite broj n:}\ulaz{4}#
#\izlaz{Unestite 1. karakter: }\ulaz{+}#
#\izlaz{Unestite 2. karakter: }\ulaz{o}#
#\izlaz{Unestite 3. karakter: }\ulaz{Z}#
#\izlaz{Unestite 4. karakter: }\ulaz{j}#
#\izlaz{Ne moze se napisati rec Zima.}#
\end{upotreba}
\end{miditest}
\begin{miditest}
\begin{upotreba}{2}
#\naslovInt#
#\izlaz{Unesite broj n:}\ulaz{10}#
#\izlaz{Unestite 1. karakter: }\ulaz{i}#
#\izlaz{Unestite 2. karakter: }\ulaz{9}#
#\izlaz{Unestite 3. karakter: }\ulaz{0}#
#\izlaz{Unestite 4. karakter: }\ulaz{p}#
#\izlaz{Unestite 5. karakter: }\ulaz{a}#
#\izlaz{Unestite 6. karakter: }\ulaz{Z}#
#\izlaz{Unestite 7. karakter: }\ulaz{o}#
#\izlaz{Unestite 8. karakter: }\ulaz{m}#
#\izlaz{Unestite 9. karakter: }\ulaz{M}#
#\izlaz{Unestite 10. karakter: }\ulaz{-}#
#\izlaz{Moze se napisati rec Zima.}#
\end{upotreba}
\end{miditest}
\linkresenje{1.3_37}
\end{Exercise}
\ifresenja
\end{Answer}[ref=1.3_37]
\includecode{resenja/1_KontrolaToka/1.3_Petlje/1.3_37.c}
\end{Answer}
\fi


%--------------------------------------------------------------------
%--------------------------------------------------------------------
% \subsection{Računanje sume i proizvoda}
%--------------------------------------------------------------------
%--------------------------------------------------------------------


\begin{Exercise}[label=1.3_38] 
Napisati program koji učitava ceo pozitivan broj $n$ i ispisuje
vrednost sume kubova brojeva od $1$ do $n$, odnosno $s = 1+2^3+3^3+
\ldots +n^3$. U slučaju greške pri unosu podataka ispisati
odgovarajuću poruku. 

\begin{miditest}
\begin{upotreba}{1}
#\naslovInt#
#\izlaz{Unesite pozitivan ceo broj:}\ulaz{14}#
#\izlaz{Suma kubova od 1 do 14: 11025}#
\end{upotreba}
\end{miditest}
\begin{miditest}
\begin{upotreba}{2}
#\naslovInt#
#\izlaz{Unesite pozitivan ceo broj}\ulaz{25}#
#\izlaz{Suma kubova od 1 do 25: 105625}#
\end{upotreba}
\end{miditest}
\linkresenje{1.3_38}
\end{Exercise}
\ifresenja
\end{Answer}[ref=1.3_38]
\includecode{resenja/1_KontrolaToka/1.3_Petlje/1.3_38.c}
\end{Answer}
\fi

\begin{Exercise}[label=1.3_39] 
Napisati program koji učitava ceo pozitivan broj $n$ i ispisuje sumu
kubova, $s = 1+2^3+3^3+ \ldots +k^3$, za svaku vrednost $k = 1,
\ldots, n$.. U slučaju greške pri unosu podataka ispisati odgovarajuću
poruku. 

\begin{miditest}
\begin{upotreba}{1}
#\naslovInt#
#\izlaz{Unesite pozitivan ceo broj:}\ulaz{5}#
#\izlaz{i=1, s=1}#
#\izlaz{i=2, s=9}#
#\izlaz{i=3, s=36}#
#\izlaz{i=4, s=100}#
#\izlaz{i=5, s=225}#
\end{upotreba}
\end{miditest}
\begin{miditest}
\begin{upotreba}{2}
#\naslovInt#
#\izlaz{Unesite pozitivan ceo broj}\ulaz{8}#
#\izlaz{i=1, s=1}#
#\izlaz{i=2, s=9}#
#\izlaz{i=3, s=36}#
#\izlaz{i=4, s=100}#
#\izlaz{i=5, s=225}#
#\izlaz{i=6, s=441}#
#\izlaz{i=7, s=784}#
#\izlaz{i=8, s=1296}#
\end{upotreba}
\end{miditest}
\linkresenje{1.3_39}
\end{Exercise}
\ifresenja
\end{Answer}[ref=1.3_39]
\includecode{resenja/1_KontrolaToka/1.3_Petlje/1.3_39.c}
\end{Answer}
\fi



\begin{Exercise}[label=1.3_40]
 Program učitava realan broj $x$ i ceo pozitivan broj $n$. Napisati
 program koji izračunava i ispisuje sumu $S=x+2\cdot x^2+3\cdot
 x^3+\ldots+n\cdot x^n$.
 
\begin{miditest}
\begin{upotreba}{1}
#\naslovInt#
#\izlaz{Unesite redom brojeve x i n:}\ulaz{2 3}#
#\izlaz{S=34.000000}#
\end{upotreba}
\end{miditest}
\begin{miditest}
\begin{upotreba}{2}
#\naslovInt#
#\izlaz{Unesite redom brojeve x i n:}\ulaz{1.5 5}#
#\izlaz{S=74.343750}#
\end{upotreba}
\end{miditest}
\linkresenje{1.3_40}
\end{Exercise}
\ifresenja
\end{Answer}[ref=1.3_40]
\includecode{resenja/1_KontrolaToka/1.3_Petlje/1.3_40.c}
\end{Answer}
\fi


\begin{Exercise}[label=1.3_41]
 Program učitava realan broj $x$ i ceo pozitivan broj $n$. Napisati
 program koji izračunava i ispisuje sumu
 $S=1+\frac{1}{x}+\frac{1}{x^2}+\ldots\frac{1}{x^n}$.
 
\begin{miditest}
\begin{upotreba}{1}
#\naslovInt#
#\izlaz{Unesite redom brojeve x i n:}\ulaz{2 4}#
#\izlaz{S=1.937500}#
\end{upotreba}
\end{miditest}
\begin{miditest}
\begin{upotreba}{2}
#\naslovInt#
#\izlaz{Unesite redom brojeve x i n:}\ulaz{1.8 6}#
#\izlaz{S=2.213249}#
\end{upotreba}
\end{miditest}
\linkresenje{1.3_41}
\end{Exercise}
\ifresenja
\end{Answer}[ref=1.3_41]
\includecode{resenja/1_KontrolaToka/1.3_Petlje/1.3_41.c}
\end{Answer}
\fi


\begin{Exercise}[difficulty=1, label=1.3_42] 
Napisati program koji učitava realane brojeve $x$ i $eps$ i sa zadatom
tačnošću $eps$ izračunava i ispisuje sumu
$S=1+x+\frac{x^2}{2!}+\frac{x^3}{3!}+\ldots$.  Izračunati sumu u
odnosu na tačnost $eps$ znači uporediti poslednji član sume sa $eps$ i
ukoliko je taj poslednji član manji od $eps$ prekinuti dalja
izračunavanja.  \uputstvo{Prilikom računanja sume koristiti prethodni
  izračunati član sume u računanju sledećeg člana sume. Naime, ako je
  izračunat član sume $\frac{x^n}{n!}$ na osnovu njega se lako može
  dobiti član $\frac{x^{n+1}}{(n+1)!}$. Nikako ne računati stepen i
  faktorijel odvojeno zbog neefikasnosti takvog rešenja i zbog
  mogućnosti prekoračenja.}
  
\begin{miditest}
\begin{upotreba}{1}
#\naslovInt#
#\izlaz{Unesite x:}\ulaz{2}#
#\izlaz{Unesite tacnost eps:}\ulaz{0.001}#
#\izlaz{S=7.388713}#
\end{upotreba}
\end{miditest}
\begin{miditest}
\begin{upotreba}{2}
#\naslovInt#
#\izlaz{Unesite x:}\ulaz{3}#
#\izlaz{Unesite tacnost eps:}\ulaz{0.01}#
#\izlaz{S=20.079666}#
\end{upotreba}
\end{miditest}
\linkresenje{1.3_42}
\end{Exercise}
\ifresenja
\end{Answer}[ref=1.3_42]
\includecode{resenja/1_KontrolaToka/1.3_Petlje/1.3_42.c}
\end{Answer}
\fi


\begin{Exercise}[difficulty=1, label=1.3_43]
Napisati program koji učitava realane brojeve $x$ i $eps$ i sa zadatom
tačnošću $eps$ izračunava i ispisuje sumu
$S=1-x+\frac{x^2}{2!}-\frac{x^3}{3!}+\frac{x^4}{4!}-\frac{x^5}{5!}\ldots$.
\napomena{Voditi računa o efikasnosti rešenja i o mogućnosti
  prekoračenja.}
  
\begin{miditest}
\begin{upotreba}{1}
#\naslovInt#
#\izlaz{Unesite x:}\ulaz{3}#
#\izlaz{Unesite tacnost eps:}\ulaz{0.000001}#
#\izlaz{S=0.049787}#
\end{upotreba}
\end{miditest}
\begin{miditest}
\begin{upotreba}{2}
#\naslovInt#
#\izlaz{Unesite x:}\ulaz{3.14}#
#\izlaz{Unesite tacnost eps:}\ulaz{0.01}#
#\izlaz{S=0.049072}#
\end{upotreba}
\end{miditest}
\linkresenje{1.3_43}
\end{Exercise}
\ifresenja
\end{Answer}[ref=1.3_43]
\includecode{resenja/1_KontrolaToka/1.3_Petlje/1.3_43.c}
\end{Answer}
\fi

\begin{Exercise}[label=1.3_44] 
Napisati program koji učitava realan broj $x$ i prirodan broj $n$
izračunava sumu $S = (1 + \cos(x))\cdot(1 + \cos(x^2))\cdot \ldots
\cdot(1 + \cos(x^n))$. \napomena{Voditi računa o efikasnosti
  rešenja}.  

\begin{miditest}
\begin{upotreba}{1}
#\naslovInt#
#\izlaz{Unesite redom brojeve x i n:}\ulaz{3.4 5}#
#\izlaz{Proizvod = 0.026817}#
\end{upotreba}
\end{miditest}
\begin{miditest}
\begin{upotreba}{2}
#\naslovInt#
#\izlaz{Unesite redom brojeve x i n:}\ulaz{12 8}#
#\izlaz{Proizvod = 2.640565}#
\end{upotreba}
\end{miditest}

\linkresenje{1.3_44}
\end{Exercise}
\ifresenja
\end{Answer}[ref=1.3_44]
\includecode{resenja/1_KontrolaToka/1.3_Petlje/1.3_44.c}
\end{Answer}
\fi


\begin{Exercise}[difficulty=1, label=1.3_45] 
Napisati program koji učitava ceo prirodan broj n i ispisuje vrednost
razlomka  \\
	\[
		\frac{1}{1 + \frac{1}{2 + \frac{1}{3 + \frac{1}{4 + \frac{1}{\ldots + \frac{1}{(n-1) + \frac{1}{n}}}}}}}.
	\]
	
\begin{minitest}
\begin{upotreba}{1}
#\naslovInt#
#\izlaz{Unesite prirodan broj:}\ulaz{4}#
#\izlaz{Razlomak = 0.697674}#
\end{upotreba}
\end{minitest}
\begin{minitest}
\begin{upotreba}{2}
#\naslovInt#
#\izlaz{Unesite prirodan broj:}\ulaz{20}#
#\izlaz{Razlomak = 0.697775}#
\end{upotreba}
\end{minitest}
\begin{minitest}
\begin{upotreba}{3}
#\naslovInt#
#\izlaz{Unesite prirodan broj:}\ulaz{0}#
#\izlaz{Neispravan unos.}#
\end{upotreba}
\end{minitest}

\linkresenje{1.3_45}
\end{Exercise}
\ifresenja
\end{Answer}[ref=1.3_45]
\includecode{resenja/1_KontrolaToka/1.3_Petlje/1.3_45.c}
\end{Answer}
\fi

\begin{Exercise}[difficulty=1, label=1.3_46] 
Napisati program koji računa sumu
$$1 - \frac{x^{2}}{2!} + \frac{x^{4}}{4!} - \ldots +
(-1)^{n}\frac{x^{2n}}{(2n)!}.$$ za unete cele brojeve $x$ i $n$. 
\napomena{Voditi računa o efikasnosti rešenja i o mogućnosti prekoračenja.} 

\begin{minitest}
\begin{upotreba}{1}
#\naslovInt#
#\izlaz{Unesite x i n:}\ulaz{5.6 8}#
#\izlaz{S=0.735084}#
\end{upotreba}
\end{minitest}
\begin{minitest}
\begin{upotreba}{2}
#\naslovInt#
#\izlaz{Unesite x i n:}\ulaz{14.32 11}#
#\izlaz{S=17273.136719}#
\end{upotreba}
\end{minitest}

\begin{minitest}
\begin{upotreba}{3}
#\naslovInt#
#\izlaz{Unesite prirodan broj:}\ulaz{-6}#
#\izlaz{Neispravan unos.}#
\end{upotreba}
\end{minitest}

\linkresenje{1.3_46}
\end{Exercise}
\ifresenja
\end{Answer}[ref=1.3_46]
\includecode{resenja/1_KontrolaToka/1.3_Petlje/1.3_46.c}
\end{Answer}
\fi


\begin{Exercise}[difficulty=1, label=1.3_47] 
Program učitava ceo pozitivan broj $n$ veći od $0$.  Napisati program
koji računa proizvod
$$S = (1 + \frac{1}{2!})(1 + \frac{1}{3!})\ldots(1 +
\frac{1}{n!}).$$ U slučaju greške pri unosu podataka ispisati 
odgovarajuću  poruku. \napomena{Voditi računa o efikasnosti rešenja i o
  mogućnosti prekoračenja.} 
  
\begin{miditest}
\begin{upotreba}{1}
#\naslovInt#
#\izlaz{Unesite broj n:}\ulaz{5}#
#\izlaz{1.838108}#
\end{upotreba}
\end{miditest}
\begin{miditest}
\begin{upotreba}{2}
#\naslovInt#
#\izlaz{Unesite broj n:}\ulaz{7}#
#\izlaz{1.841026}#
\end{upotreba}
\end{miditest}

\begin{miditest}
\begin{upotreba}{3}
#\naslovInt#
#\izlaz{Unesite broj n:}\ulaz{0}#
#\izlaz{Neispravan unos.}#
\end{upotreba}
\end{miditest}
\begin{miditest}
\begin{upotreba}{4}
#\naslovInt#
#\izlaz{Unesite broj n:}\ulaz{10}#
#\izlaz{1.841077}#
\end{upotreba}
\end{miditest}
\linkresenje{1.3_47}
\end{Exercise}
\ifresenja
\end{Answer}[ref=1.3_47]
\includecode{resenja/1_KontrolaToka/1.3_Petlje/1.3_47.c}
\end{Answer}
\fi

\begin{Exercise}[difficulty=1, label=1.3_48] 
Program učitava ceo pozitivan neparan broj $n$.  Napisati program koji
za uneto $n$ izračunava:
$$S = 1\cdot3\cdot5 - 1\cdot3\cdot5\cdot7 + 1\cdot3\cdot5\cdot7\cdot9
- 1\cdot3\cdot5\cdot7\cdot9\cdot11 + \ldots
(-1)^{\frac{n-1}{2}+1}\cdot1\cdot3\cdot \ldots \cdot n.$$ U slučaju
greške pri unosu podataka ispisati odgovarajuću poruku. 
\napomena{Voditi računa o efikasnosti rešenja i o
  mogućnosti prekoračenja.} 
  
\begin{miditest}
\begin{upotreba}{1}
#\naslovInt#
#\izlaz{Unesite broj n:}\ulaz{9}#
#\izlaz{855}#
\end{upotreba}
\end{miditest}
\begin{miditest}
\begin{upotreba}{2}
#\naslovInt#
#\izlaz{Unesite broj n:}\ulaz{11}#
#\izlaz{-9540}#
\end{upotreba}
\end{miditest}

\begin{miditest}
\begin{upotreba}{3}
#\naslovInt#
#\izlaz{Unesite broj n:}\ulaz{20}#
#\izlaz{Neispravan unos}#
\end{upotreba}
\end{miditest}
\begin{miditest}
\begin{upotreba}{4}
#\naslovInt#
#\izlaz{Unesite broj n:}\ulaz{-3}#
#\izlaz{Neispravan unos.}#
\end{upotreba}
\end{miditest}
\linkresenje{1.3_48}
\end{Exercise}
\ifresenja
\end{Answer}[ref=1.3_48]
\includecode{resenja/1_KontrolaToka/1.3_Petlje/1.3_48.c}
\end{Answer}
\fi

\begin{Exercise}[label=1.3_49] 
Program učitava realne brojeve $x$ i $a$ i ceo pozitivan broj $n$ veći
od $0$.  Napisati program koji izračunava:
 $$((\ldots \underbrace{(((x+a)^2 + a)^2 + a)^2 + \ldots a)^2}_n.$$ U
slučaju greške pri unosu podataka ispisati odgovarajuću poruku. 

\begin{miditest}
\begin{upotreba}{1}
#\naslovInt#
#\izlaz{Unesite dva relana broja x i a::}\ulaz{3.2 0.2}#
#\izlaz{Unesite prirodan broj:}\ulaz{5}#
#\izlaz{Izraz = 135380494030332048.000000}#
\end{upotreba}
\end{miditest}
\begin{miditest}
\begin{upotreba}{2}
#\naslovInt#
#\izlaz{Unesite dva relana broja x i a::}\ulaz{2 1}#
#\izlaz{Unesite prirodan broj:}\ulaz{3}#
#\izlaz{Izraz = 10201.000000}#
\end{upotreba}
\end{miditest}

\begin{miditest}
\begin{upotreba}{3}
#\naslovInt#
#\izlaz{Unesite dva relana broja x i a::}\ulaz{2.6 0.3}#
#\izlaz{Unesite prirodan broj:}\ulaz{3}#
#\izlaz{Izraz = 5800.970129}#
\end{upotreba}
\end{miditest}
\begin{miditest}
\begin{upotreba}{4}
#\naslovInt#
#\izlaz{Unesite dva relana broja x i a::}\ulaz{5.4 7}#
#\izlaz{Unesite prirodan broj:}\ulaz{-2}#
#\izlaz{Neispravan unos.}#
\end{upotreba}
\end{miditest}
\linkresenje{1.3_49}
\end{Exercise}
\ifresenja
\end{Answer}[ref=1.3_49]
\includecode{resenja/1_KontrolaToka/1.3_Petlje/1.3_49.c}
\end{Answer}
\fi


%--------------------------------------------------------------------
%--------------------------------------------------------------------
% \subsection{Dvostruka petlja i ispisivanje slike}
%--------------------------------------------------------------------
%--------------------------------------------------------------------
\begin{Exercise}[label=1.3_50] 
Za unetu pozitivnu celobrojnu vrednost $n$ napisati programe koji ispisuju odgovarajuće brojeve. Pretpostaviti da je unos korektan.

\begin{enumerate}
\item Napisati program koji za unetu pozitivnu celobrojnu vrednost 
$n$ ispisuje tablicu množenja. 

\begin{miditest}
\begin{upotreba}{1}
#\naslovInt#
#\izlaz{Unesite broj n:}\ulaz{1}#
#\izlaz{1}#
\end{upotreba}
\end{miditest}
\begin{miditest}
\begin{upotreba}{2}
#\naslovInt#
#\izlaz{Unesite broj n:}\ulaz{2}#
#\izlaz{1 \ \ 2}#
#\izlaz{2 \ \ 4}#
\end{upotreba}
\end{miditest}

\begin{miditest}
\begin{upotreba}{3}
#\naslovInt#
#\izlaz{Unesite broj n:}\ulaz{3}#
#\izlaz{1 \ \ 2 \ \ 3 }#
#\izlaz{2 \ \ 4 \ \ 6 }#
#\izlaz{3 \ \ 6 \ \ 9}#
\end{upotreba}
\end{miditest}
\begin{miditest}
\begin{upotreba}{4}
#\naslovInt#
#\izlaz{Unesite broj n:}\ulaz{4}#
#\izlaz{1 \ \ 2 \ \ 3 \ \ 4 }#
#\izlaz{2 \ \ 4 \ \ 6 \ \ 8 }#
#\izlaz{3 \ \ 6 \ \ 9 \ \ 12}#
#\izlaz{4 \ \ 8 \ \ 12 \ 16}#
\end{upotreba}
\end{miditest}

%\linkresenje{1.3_50a}


\item Napisati program koji za uneto $n$ ispisuje sve brojeve od 1 do $n^2$ pri čemu se ispisuje po $n$ brojeva u jednoj vrsti.

\begin{miditest}
\begin{upotreba}{1}
#\naslovInt#
#\izlaz{Unesite broj n:}\ulaz{3}#
#\izlaz{1 \ \ 2 \ \ 3 }#
#\izlaz{4 \ \ 5 \ \ 6 }#
#\izlaz{7 \ \ 8 \ \ 9}#
\end{upotreba}
\end{miditest}
\begin{miditest}
\begin{upotreba}{2}
#\naslovInt#
#\izlaz{Unesite broj n:}\ulaz{4}#
#\izlaz{1 \ \ 2 \ \ 3 \ \ 4 }#
#\izlaz{5 \ \ 6 \ \ 7 \ \ 8}#
#\izlaz{9 \ \ 10 \ 11 \ 12}#
#\izlaz{13 \ 14 \ 15 \ 16}#
\end{upotreba}
\end{miditest}

%\linkresenje{1.3_50b}


\item Napisati program koji za uneto $n$ ispisuje tablicu brojeva tako da su u prvoj vrsti svi brojevi od $1$ do $n$, a svaka naredna vrsta dobija se rotiranjem prethodne vrste za jedno mesto u levo. 

\begin{miditest}
\begin{upotreba}{1}
#\naslovInt#
#\izlaz{Unesite broj n:}\ulaz{3}#
#\izlaz{1 2 3 }#
#\izlaz{2 3 1 }#
#\izlaz{3 1 2}#
\end{upotreba}
\end{miditest}
\begin{miditest}
\begin{upotreba}{2}
#\naslovInt#
#\izlaz{Unesite broj n:}\ulaz{4}#
#\izlaz{1 2 3 4 }#
#\izlaz{2 3 4 1}#
#\izlaz{3 4 1 2}#
#\izlaz{4 1 2 3 }#
\end{upotreba}
\end{miditest}

%\linkresenje{1.3_50c}


\item Napisati program koji za uneto
$n$ iscrtava pravougli ,,trougao'' sačinjen od ,,koordinata'' svojih
tačaka. ,,Koordinata'' tačke je oblika $(i,j)$ pri čemu $i,\ j = 0,
\ldots, n$. Prav ugao se nalazi u gornjem levom uglu slike i njegova
koordinata je $(0, 0)$. Koordinata $i$ se uvećava po vrsti, a
koordinata $j$ po koloni, pa je zato koordinata tačke koja je ispod
tačke $(0,0)$ jednaka $(1, 0)$, a koordinata tačke koja je desno od
tačke $(0,0)$ jednaka $(0,1)$.

\begin{miditest}
\begin{upotreba}{1}
#\naslovInt#
#\izlaz{Unesite broj n:}\ulaz{1}#
#\izlaz{(0,0)}#
\end{upotreba}
\end{miditest}
\begin{miditest}
\begin{upotreba}{2}
#\naslovInt#
#\izlaz{Unesite broj n:}\ulaz{2}#
#\izlaz{(0,0) (0,1)}#
#\izlaz{(1,0)}#
\end{upotreba}
\end{miditest}

\begin{miditest}
\begin{upotreba}{3}
#\naslovInt#
#\izlaz{Unesite broj n:}\ulaz{3}#
#\izlaz{(0,0) (0,1) (0,2)}#
#\izlaz{(1,0) (1,1)}#
#\izlaz{(2,0)}#
\end{upotreba}
\end{miditest}
\begin{miditest}
\begin{upotreba}{4}
#\naslovInt#
#\izlaz{Unesite broj n:}\ulaz{4}#
#\izlaz{(0,0) (0,1) (0,2) (0,3)}#
#\izlaz{(1,0) (1,1) (1,2)}#
#\izlaz{(2,0) (2,1)}#
#\izlaz{(3,0)}#
\end{upotreba}
\end{miditest}
%\linkresenje{1.3_50d}
\end{enumerate}
\end{Exercise}

\linkresenje{1.3_50}

\ifresenja
\end{Answer}[ref=1.3_50]
\includecodeLib{resenja/1_KontrolaToka/1.3_Petlje/1.3_50a.c}{Rešenje (a)}
\includecodeLib{resenja/1_KontrolaToka/1.3_Petlje/1.3_50b.c}{Rešenje (b)}
\includecodeLib{resenja/1_KontrolaToka/1.3_Petlje/1.3_50c.c}{Rešenje (c)}
\includecodeLib{resenja/1_KontrolaToka/1.3_Petlje/1.3_50d.c}{Rešenje (d)}
\end{Answer}
\fi

\begin{Exercise}[label=1.3_51] 
Napisati program koji za unet prirodan broj $n$ zvezdicama iscrtava
odgovarajuću sliku. Pretpostaviti da je unos korektan.
\begin{enumerate}
\item Slika sadrži kvadrat stranice $n$ sastavljen od zvezdica. 

\begin{miditest}
\begin{upotreba}{1}
#\naslovInt#
#\izlaz{Unesite broj n:}\ulaz{3}#
#\izlaz{***}#
#\izlaz{***}#
#\izlaz{***}#
\end{upotreba}
\end{miditest}
\begin{miditest}
\begin{upotreba}{2}
#\naslovInt#
#\izlaz{Unesite broj n:}\ulaz{4}#
#\izlaz{****}#
#\izlaz{****}#
#\izlaz{****}#
#\izlaz{****}#
\end{upotreba}
\end{miditest}
%\linkresenje{1.3_51a}


\item Slika sadrži rub kvadrata dimenzije $n$. 

\begin{miditest}
\begin{upotreba}{1}
#\naslovInt#
#\izlaz{Unesite broj n:}\ulaz{5}#
#\izlaz{*****}#
#\izlaz{*\ \ \ *}#
#\izlaz{*\ \ \ *}#
#\izlaz{*\ \ \ *}#
#\izlaz{*****}#
\end{upotreba}
\end{miditest}
\begin{miditest}
\begin{upotreba}{2}
#\naslovInt#
#\izlaz{Unesite broj n:}\ulaz{2}#
#\izlaz{**}#
#\izlaz{**}#
\end{upotreba}
\end{miditest}
%\linkresenje{1.3_51b}


\item Slika sadrži rub kvadrata dimenzije $n$ koji i na glavnoj dijagonali ima
  zvezdice.
  
\begin{miditest}
\begin{upotreba}{1}
#\naslovInt#
#\izlaz{Unesite broj n:}\ulaz{5}#
#\izlaz{*****}#
#\izlaz{**\ \ *}#
#\izlaz{*\ *\ *}#
#\izlaz{*\ \ **}#
#\izlaz{*****}#
\end{upotreba}
\end{miditest}
\begin{miditest}
\begin{upotreba}{1}
#\naslovInt#
#\izlaz{Unesite broj n:}\ulaz{4}#
#\izlaz{****}#
#\izlaz{**\ *}#
#\izlaz{*\ **}#
#\izlaz{****}#
\end{upotreba}
\end{miditest}
\end{enumerate}
\linkresenje{1.3_51}
\end{Exercise}
\ifresenja
\end{Answer}[ref=1.3_51]
\includecodeLib{resenja/1_KontrolaToka/1.3_Petlje/1.3_51a.c}{Rešenje (a)}
\includecodeLib{resenja/1_KontrolaToka/1.3_Petlje/1.3_51b.c}{Rešenje (b)}
\includecodeLib{resenja/1_KontrolaToka/1.3_Petlje/1.3_51c.c}{Rešenje (c)}
\end{Answer}
\fi

\begin{Exercise}[difficulty=1, label=1.3_52]
 Napisati program koji za uneti prirodan broj $n$ zvezdicama iscrtava
 slovo \textit{X} dimenzije $n$. Pretpostaviti da je unos korektan.

\begin{miditest}
\begin{upotreba}{1}
#\naslovInt#
#\izlaz{Unesite broj n:}\ulaz{5}#
#\izlaz{*\ \ \ *}#
#\izlaz{\ *\ *\ }#
#\izlaz{\ \ *\ \ }#
#\izlaz{\ *\ *\ }#
#\izlaz{*\ \ \ *}#
\end{upotreba}
\end{miditest}
\begin{miditest}
\begin{upotreba}{2}
#\naslovInt#
#\izlaz{Unesite broj n:}\ulaz{3}#
#\izlaz{*\ *}#
#\izlaz{\ *\ }#
#\izlaz{*\ *}#
\end{upotreba}
\end{miditest}
\linkresenje{1.3_52}
\end{Exercise}
\ifresenja
\end{Answer}[ref=1.3_52]
\includecode{resenja/1_KontrolaToka/1.3_Petlje/1.3_52.c}
\end{Answer}
\fi


\begin{Exercise}[difficulty=1, label=1.3_53]
 Napisati program koji za uneti prirodan neparan broj $n$ korišćenjem
 znaka $+$ iscrtava veliko $+$ dimenzije $n$. Pretpostaviti da je unet
 prirodan broj.
 
\begin{miditest}
\begin{upotreba}{1}
#\naslovInt#
#\izlaz{Unesite broj n:}\ulaz{5}#
#\izlaz{\ \ +}#
#\izlaz{\ \ +}#
#\izlaz{+++++}#
#\izlaz{\ \ +}#
#\izlaz{\ \ +}#
\end{upotreba}
\end{miditest}
\begin{miditest}
\begin{upotreba}{2}
#\naslovInt#
#\izlaz{Unesite broj n:}\ulaz{3}#
#\izlaz{\ +}#
#\izlaz{+++}#
#\izlaz{\ +}#
\end{upotreba}
\end{miditest}

\begin{miditest}
\begin{upotreba}{3}
#\naslovInt#
#\izlaz{Unesite broj n:}\ulaz{4}#
#\izlaz{Pogresan unos.}#
\end{upotreba}
\end{miditest}

\linkresenje{1.3_53}
\end{Exercise}
\ifresenja
\end{Answer}[ref=1.3_53]
\includecode{resenja/1_KontrolaToka/1.3_Petlje/1.3_53.c}
\end{Answer}
\fi



\begin{Exercise}[label=1.3_54] 
Napisati program koji učitava prirodan broj $n$, a potom iscrtava
odgovarajuću sliku. Pretpostaviti da je unos korektan.
\begin{enumerate}
\item Slika sadrži pravougli trougao sastavljen od zvezdica. Kateta
  trougla je dužine $n$, a prav ugao se nalazi u gornjem levom uglu
  slike.

\begin{miditest}
\begin{upotreba}{1}
#\naslovInt#
#\izlaz{Unesite broj n:}\ulaz{3}#
#\izlaz{***}#
#\izlaz{**}#
#\izlaz{*}#
\end{upotreba}
\end{miditest}
\begin{miditest}
\begin{upotreba}{1}
#\naslovInt#
#\izlaz{Unesite broj n:}\ulaz{4}#
#\izlaz{****}#
#\izlaz{***}#
#\izlaz{**}#
#\izlaz{*}#
\end{upotreba}
\end{miditest}
%\linkresenje{1.3_54a}

\item  Slika sadrži pravougli trougao sastavljen od zvezdica. Kateta trougla je
  dužine $n$, a prav ugao se nalazi u donjem levom uglu slike. 

\begin{miditest}
\begin{upotreba}{1}
#\naslovInt#
#\izlaz{Unesite broj n:}\ulaz{3}#
#\izlaz{*}#
#\izlaz{**}#
#\izlaz{***}#
\end{upotreba}
\end{miditest}
\begin{miditest}
\begin{upotreba}{2}
#\naslovInt#
#\izlaz{Unesite broj n:}\ulaz{4}#
#\izlaz{*}#
#\izlaz{**}#
#\izlaz{***}#
#\izlaz{****}#
\end{upotreba}
\end{miditest}
%\linkresenje{1.3_54b}

\item  Slika sadrži pravougli trougao sastavljen od zvezdica. Kateta trougla je
  dužine $n$, a prav ugao se nalazi u gornjem desnom uglu slike. 

\begin{miditest}
\begin{upotreba}{1}
#\naslovInt#
#\izlaz{Unesite broj n:}\ulaz{3}#
#\izlaz{***}#
#\izlaz{\ **}#
#\izlaz{\ \ *}#
\end{upotreba}
\end{miditest}
\begin{miditest}
\begin{upotreba}{1}
#\naslovInt#
#\izlaz{Unesite broj n:}\ulaz{4}#
#\izlaz{****}#
#\izlaz{\ ***}#
#\izlaz{\ \ **}#
#\izlaz{\ \ \ *}#
\end{upotreba}
\end{miditest}
%\linkresenje{1.3_54c}

\item  Slika sadrži pravougli trougao sastavljen od zvezdica. Kateta trougla je
  dužine $n$, a prav ugao se nalazi u donjem desnom uglu slike. 

\begin{miditest}
\begin{upotreba}{1}
#\naslovInt#
#\izlaz{Unesite broj n:}\ulaz{3}#
#\izlaz{\ \ *}#
#\izlaz{\ **}#
#\izlaz{***}#
\end{upotreba}
\end{miditest}
\begin{miditest}
\begin{upotreba}{2}
#\naslovInt#
#\izlaz{Unesite broj n:}\ulaz{4}#
#\izlaz{\ \ \ *}#
#\izlaz{\ \ **}#
#\izlaz{\ ***}#
#\izlaz{****}#
\end{upotreba}
\end{miditest}
%\linkresenje{1.3_54d}

\item
 Slika sadrži trougao sastavljen od zvezdica. Trougao se dobija spajanjem
  dva pravougla trougla čija kateta je dužine $n$, pri čemu je prav
  ugao prvog trougla u njegovom donjem levom uglu, dok je prav ugao
  drugog trougla u njegovom gornjem levom uglu, a spajanje se vrši po
  horiznotalnoj kateti. 
  
\begin{miditest}
\begin{upotreba}{1}
#\naslovInt#
#\izlaz{Unesite broj n:}\ulaz{3}#
#\izlaz{*}#
#\izlaz{**}#
#\izlaz{***}#
#\izlaz{**}#
#\izlaz{*}#
\end{upotreba}
\end{miditest}
\begin{miditest}
\begin{upotreba}{2}
#\naslovInt#
#\izlaz{Unesite broj n:}\ulaz{4}#
#\izlaz{*}#
#\izlaz{**}#
#\izlaz{***}#
#\izlaz{****}#
#\izlaz{***}#
#\izlaz{**}#
#\izlaz{*}#
\end{upotreba}
\end{miditest}
%\linkresenje{1.3_54e}

\item Slika sadrži rub jednakokrakog pravouglog trougla čije su katete dužine
  $n$. Program učitava karakter $c$ i taj karakter koristi za
  iscrtavanje ruba trougla. 
  
\begin{miditest}
\begin{upotreba}{1}
#\naslovInt#
#\izlaz{Unesite broj n:}\ulaz{4}#
#\izlaz{Unesite karakter c:}\ulaz{*}#
#\izlaz{*}#
#\izlaz{**}#
#\izlaz{*\ *}#
#\izlaz{****}#
\end{upotreba}
\end{miditest}
\begin{miditest}
\begin{upotreba}{2}
#\naslovInt#
#\izlaz{Unesite broj n:}\ulaz{5}#
#\izlaz{Unesite karakter c:}\ulaz{+}#
#\izlaz{+}#
#\izlaz{++}#
#\izlaz{+\ +}#
#\izlaz{+\ \ +}#
#\izlaz{+++++}#
\end{upotreba}
\end{miditest}
\end{enumerate}
\linkresenje{1.3_54}
\end{Exercise}
\ifresenja
\end{Answer}[ref=1.3_54]
\includecodeLib{resenja/1_KontrolaToka/1.3_Petlje/1.3_54a.c}{Rešenje (a)}
\includecodeLib{resenja/1_KontrolaToka/1.3_Petlje/1.3_54b.c}{Rešenje (b)}
\includecodeLib{resenja/1_KontrolaToka/1.3_Petlje/1.3_54c.c}{Rešenje (c)}
\includecodeLib{resenja/1_KontrolaToka/1.3_Petlje/1.3_54d.c}{Rešenje (d)}
\includecodeLib{resenja/1_KontrolaToka/1.3_Petlje/1.3_54e.c}{Rešenje (e)}
\includecodeLib{resenja/1_KontrolaToka/1.3_Petlje/1.3_54f.c}{Rešenje (f)}
\end{Answer}
\fi


\begin{Exercise}[label=1.3_55] 
Napisati program koji učitava ceo broj $n$, a potom iscrtava odgovarajuću sliku.
\begin{enumerate}
\item  Slika sadrži jednakostranični trougao stranice $n$ koji je sastavljen od
  zvezdica.  
  
\begin{miditest}
\begin{upotreba}{1}
#\naslovInt#
#\izlaz{Unesite broj n:}\ulaz{3}#
#\izlaz{\ \ *}#
#\izlaz{\ ***}#
#\izlaz{*****}#
\end{upotreba}
\end{miditest}
\begin{miditest}
\begin{upotreba}{2}
#\naslovInt#
#\izlaz{Unesite broj n:}\ulaz{4}#
#\izlaz{\ \ \ *}#
#\izlaz{\ \ ***}#
#\izlaz{\ *****}#
#\izlaz{*******}#
\end{upotreba}
\end{miditest}
%\linkresenje{1.3_55a}

\item  Slika sadrži jednakostranični trougao stranice $n$ koji je sastavljen od
  zvezdica pri čemu je vrh trougla na dnu slike.  
  
\begin{miditest}
\begin{upotreba}{1}
#\naslovInt#
#\izlaz{Unesite broj n:}\ulaz{3}#
#\izlaz{*****}#
#\izlaz{\ ***}#
#\izlaz{\ \ *}#
\end{upotreba}
\end{miditest}
\begin{miditest}
\begin{upotreba}{2}
#\naslovInt#
#\izlaz{Unesite broj n:}\ulaz{4}#
#\izlaz{*******}#
#\izlaz{\ *****}#
#\izlaz{\ \ ***}#
#\izlaz{\ \ \ *}#
\end{upotreba}
\end{miditest}
%\linkresenje{1.3_55b}

\item Slika sadrži trougao koji se dobija spajanjem dva jednakostranična
  trougla stranice $n$ koji su sastavljeni od zvezdica. 
  
\begin{miditest}
\begin{upotreba}{1}
#\naslovInt#
#\izlaz{Unesite broj n:}\ulaz{3}#
#\izlaz{\ \ *}#
#\izlaz{\ ***}#
#\izlaz{*****}#
#\izlaz{\ ***}#
#\izlaz{\ \ *}#
\end{upotreba}
\end{miditest}
\begin{miditest}
\begin{upotreba}{2}
#\naslovInt#
#\izlaz{Unesite broj n:}\ulaz{5}#
#\izlaz{\ \ \ \ *}#
#\izlaz{\ \ \ ***}#
#\izlaz{\ \ *****}#
#\izlaz{\ *******}#
#\izlaz{*********}#
#\izlaz{\ *******}#
#\izlaz{\ \ *****}#
#\izlaz{\ \ \ ***}#
#\izlaz{\ \ \ \ *}#
\end{upotreba}
\end{miditest}
%\linkresenje{1.3_55c}

\item Slika sadrži rub jednakostraničnog trougla čija stranica je dužine $n$. 

\begin{miditest}
\begin{upotreba}{1}
#\naslovInt#
#\izlaz{Unesite broj n:}\ulaz{3}#
#\izlaz{\ \ *}#
#\izlaz{\ *\ *}#
#\izlaz{*\ *\ *}#
\end{upotreba}
\end{miditest}
\begin{miditest}
\begin{upotreba}{1}
#\naslovInt#
#\izlaz{Unesite broj n:}\ulaz{5}#
#\izlaz{\ \ \ \ *}#
#\izlaz{\ \ \ *\ *}#
#\izlaz{\ \ *\ \ \ *}#
#\izlaz{\ *\ \ \ \ \ *}#
#\izlaz{*\ *\ *\ *\ *}#
\end{upotreba}
\end{miditest}
%\linkresenje{1.3_55d}

\item  Slika se dobija spajanjem dva jednakostranična trougla
  čija stranica je dužine $n$. Iscrtavati samo rub trouglova.
  
\begin{miditest}
\begin{upotreba}{1}
#\naslovInt#
#\izlaz{Unesite broj n:}\ulaz{3}#
#\izlaz{\ \ *}#
#\izlaz{\ *\ *}#
#\izlaz{*\ *\ *}#
#\izlaz{\ *\ *}#
#\izlaz{\ \ *}#
\end{upotreba}
\end{miditest}
\begin{miditest}
\begin{upotreba}{2}
#\naslovInt#
#\izlaz{Unesite broj n:}\ulaz{5}#
#\izlaz{\ \ \ \ *}#
#\izlaz{\ \ \ *\ *}#
#\izlaz{\ \ *\ \ \ *}#
#\izlaz{\ *\ \ \ \ \ *}#
#\izlaz{*\ *\ *\ *\ *}#
#\izlaz{\ *\ \ \ \ \ *}#
#\izlaz{\ \ *\ \ \ *}#
#\izlaz{\ \ \ *\ *}#
#\izlaz{\ \ \ \ *}#
\end{upotreba}
\end{miditest}
\end{enumerate}
%\linkresenje{1.3_55e}
\end{Exercise}

\ifresenja
\end{Answer}[ref=1.3_55]
\includecodeLib{resenja/1_KontrolaToka/1.3_Petlje/1.3_55a.c}{Rešenje (a)}
\includecodeLib{resenja/1_KontrolaToka/1.3_Petlje/1.3_55b.c}{Rešenje (b)}
\includecodeLib{resenja/1_KontrolaToka/1.3_Petlje/1.3_55c.c}{Rešenje (c)}
\includecodeLib{resenja/1_KontrolaToka/1.3_Petlje/1.3_55d.c}{Rešenje (d)}
\includecodeLib{resenja/1_KontrolaToka/1.3_Petlje/1.3_55e.c}{Rešenje (c)}
\end{Answer}
\fi



\begin{Exercise}[difficulty=1, label=1.3_56] 
 Napisati program koji za uneti prirodan broj $n$ iscrtava strelice
 dimenzije $n$. Pretpostaviti da je unos korektan.
 
\begin{miditest}
\begin{upotreba}{1}
#\naslovInt#
#\izlaz{Unesite broj n:}\ulaz{3}#
#\izlaz{*}#
#\izlaz{\ *}#
#\izlaz{***}#
#\izlaz{\ *}#
#\izlaz{*}#
\end{upotreba}
\end{miditest}
\begin{miditest}
\begin{upotreba}{2}
#\naslovInt#
#\izlaz{Unesite broj n:}\ulaz{5}#
#\izlaz{*}#
#\izlaz{\ *}#
#\izlaz{\ \ *}#
#\izlaz{\ \ \ *}#
#\izlaz{*****}#
#\izlaz{\ \ \ *}#
#\izlaz{\ \ *}#
#\izlaz{\ *}#
#\izlaz{*}#
\end{upotreba}
\end{miditest} 
\linkresenje{1.3_56}
\end{Exercise}
\ifresenja
\end{Answer}[ref=1.3_56]
\includecode{resenja/1_KontrolaToka/1.3_Petlje/1.3_56.c}
\end{Answer}
\fi

\begin{Exercise}[difficulty=1, label=1.3_57] 
Napisati program koji učitava ceo broj $n$, i iscrtava sliku koja se
dobija na sledeći način: u prvom redu je jedna zvezdica, u drugom redu
su dve zvezdice razdvojene razmakom, treći red je sastavljen od
zvezdica i iste je dužine kao i drugi red, četvrti red se sastoji od
tri zvezdice razdvojene razmakom, a peti red je sastavljen od zvezdica
i iste je dužine kao i četvrti red itd. Ukupna visina slike je $n$.
Pretpostaviti da je unos korektan.

\begin{miditest}
\begin{upotreba}{1}
#\naslovInt#
#\izlaz{Unesite broj n:}\ulaz{7}#
#\izlaz{*}#
#\izlaz{*\ *}#
#\izlaz{***}#
#\izlaz{*\ *\ *}#
#\izlaz{*****}#
#\izlaz{*\ *\ *\ *}#
#\izlaz{*******}#
\end{upotreba}
\end{miditest}
\linkresenje{1.3_57}
\end{Exercise}
\ifresenja
\end{Answer}[ref=1.3_57]
\includecode{resenja/1_KontrolaToka/1.3_Petlje/1.3_57.c}
\end{Answer}
\fi

\begin{Exercise}[difficulty=1, label=1.3_58] 
Program učitava prirodne brojeve $m$ i $n$. Napisati
program koji iscrtava jedan do drugog stranice $n$ kvadrata čija je
svaka strana sastavljena od $m$ zvezdica razdvojenih prazninom.
Podrazumevati da je unos korektan.


\begin{miditest}
\begin{upotreba}{1}
#\naslovInt#
#\izlaz{Unesite broj n:}\ulaz{5 3}#
#\izlaz{*\ *\ *\ *\ *\ *\ *\ *\ *\ *\ *\ *\ *}#         
#\izlaz{*\ \ \ \ \ \ \ *\ \ \ \ \ \ \ *\ \ \ \ \ \ \ *}#           
#\izlaz{*\ \ \ \ \ \ \ *\ \ \ \ \ \ \ *\ \ \ \ \ \ \ *}#             
#\izlaz{*\ \ \ \ \ \ \ *\ \ \ \ \ \ \ *\ \ \ \ \ \ \ *}#
#\izlaz{*\ *\ *\ *\ *\ *\ *\ *\ *\ *\ *\ *\ *}#
\end{upotreba}
\end{miditest}
\begin{miditest}
\begin{upotreba}{2}
#\naslovInt#
#\izlaz{Unesite broj n:}\ulaz{4 4}#
#\izlaz{*\ *\ *\ *\ *\ *\ *\ *\ *\ *\ *\ *\ *}#
#\izlaz{*\ \ \ \ \ *\ \ \ \ \ *\ \ \ \ \ *\ \ \ \ \ *}#
#\izlaz{*\ \ \ \ \ *\ \ \ \ \ *\ \ \ \ \ *\ \ \ \ \ *}#
#\izlaz{*\ *\ *\ *\ *\ *\ *\ *\ *\ *\ *\ *\ *}#
\end{upotreba}
\end{miditest}
\linkresenje{1.3_58}
\end{Exercise}
\ifresenja
\end{Answer}[ref=1.3_58]
\includecode{resenja/1_KontrolaToka/1.3_Petlje/1.3_58.c}
\end{Answer}
\fi

\begin{Exercise}[difficulty=1, label=1.3_59] 
Program učitava prirodan broj $n$. Napisati program koji štampa romb
sastavljen od minusa u pravougaoniku sastavljenom od zvezdica.
Podrazumevati da je unos korektan.

\begin{miditest}
\begin{upotreba}{1}
#\naslovInt#
#\izlaz{Unesite broj n:}\ulaz{6}#
#\izlaz{************}#
#\izlaz{*****--*****}#
#\izlaz{****----****}#
#\izlaz{***------***}#
#\izlaz{**--------**}#
#\izlaz{*----------*}#
#\izlaz{**--------**}#
#\izlaz{***------***}#
#\izlaz{****----****}#
#\izlaz{*****--*****}#
#\izlaz{************}#
\end{upotreba}
\end{miditest}
\begin{miditest}
\begin{upotreba}{2}
#\naslovInt#
#\izlaz{Unesite broj n:}\ulaz{2}#
#\izlaz{****}#
#\izlaz{*--*}#
#\izlaz{****}#
\end{upotreba}
\end{miditest}
\linkresenje{1.3_59}
\end{Exercise}
\ifresenja
\end{Answer}[ref=1.3_59]
\includecode{resenja/1_KontrolaToka/1.3_Petlje/1.3_59.c}
\end{Answer}
\fi

\begin{Exercise}[label=1.3_60] 
Napisati program koji učitava ceo broj $n$ ($n \geq 2$) i koji
iscrtava sliku kuće sa krovom: kuća je kocka stranice $n$, a krov
jednakostranični trougao stranice $n$. Pretpostaviti da je unos
korektan.

\begin{miditest}
\begin{upotreba}{1}
#\naslovInt#
#\izlaz{Unesite broj n:}\ulaz{4}#
#\izlaz{\ \ \ *}#
#\izlaz{\ \ *\ *}#
#\izlaz{\ *\ \ \ *}#
#\izlaz{*\ *\ *\ *}#
#\izlaz{*\ \ \ \ \ *}#
#\izlaz{*\ \ \ \ \ *}#
#\izlaz{*\ *\ *\ *}#
\end{upotreba}
\end{miditest}
\begin{miditest}
\begin{upotreba}{1}
#\naslovInt#
#\izlaz{Unesite broj n:}\ulaz{3}#
#\izlaz{\ \ *}#
#\izlaz{\ *\ *}#
#\izlaz{*\ *\ *}#
#\izlaz{*\ \ \ *}#
#\izlaz{*\ *\ *}#
\end{upotreba}
\end{miditest}

\linkresenje{1.3_60}
\end{Exercise}
\ifresenja
\end{Answer}[ref=1.3_60]
\includecode{resenja/1_KontrolaToka/1.3_Petlje/1.3_60.c}
\end{Answer}
\fi



\begin{Exercise}[difficulty=1, label=1.3_61] 
Program učitava ceo pozitivan broj $n$. Napisati program koji ispisuje
brojeve od $1$ do $n$, zatim od $2$ do $n-1$, $3$ do $n-2$, itd. Ispis
se završava kada nije moguće ispisati ni jedan broj. Za neispravan
unos, program ispisuje odgovarajuću poruku. Pretpostaviti da je unos
korektan.

\begin{miditest}
\begin{upotreba}{1}
#\naslovInt#
#\izlaz{Unesite broj n:}\ulaz{5}#
#\izlaz{1 2 3 4 5 2 3 4 3}#
\end{upotreba}
\end{miditest}
\begin{miditest}
\begin{upotreba}{2}
#\naslovInt#
#\izlaz{Unesite broj n:}\ulaz{-4}#
#\izlaz{-1}#
\end{upotreba}
\end{miditest}

\begin{miditest}
\begin{upotreba}{3}
#\naslovInt#
#\izlaz{Unesite broj n:}\ulaz{7}#
#\izlaz{1 2 3 4 5 6 7 2 3 4 5 6 3 4 5 4}#
\end{upotreba}
\end{miditest}
\begin{miditest}
\begin{upotreba}{4}
#\naslovInt#
#\izlaz{Unesite broj n:}\ulaz{3}#
#\izlaz{1 2 3 2}#
\end{upotreba}
\end{miditest}
\linkresenje{1.3_61}
\end{Exercise}
\ifresenja
\end{Answer}[ref=1.3_61]
\includecode{resenja/1_KontrolaToka/1.3_Petlje/1.3_61.c}
\end{Answer}
\fi



\begin{Exercise}[difficulty=1, label=1.3_62] 
Napisati program koji učitava ceo pozitivan broj $n$ i ispisuje sve
brojeve od $1$ do $n$, zatim svaki drugi broj od $1$ do $n$, zatim
svaki treći broj od $1$ do $n$ itd., završavajući sa svakim $n$-tim
(tj. samo sa $1$). U slučaju greške pri unosu podataka odštampati
ogovarajuću poruku.

\begin{miditest}
\begin{upotreba}{1}
#\naslovInt#
#\izlaz{Unesite broj n:}\ulaz{3}#
#\izlaz{1 2 3}#
#\izlaz{1 3}#
#\izlaz{1}#
\end{upotreba}
\end{miditest}
\begin{miditest}
\begin{upotreba}{2}
#\naslovInt#
#\izlaz{Unesite broj n:}\ulaz{7}#
#\izlaz{1 2 3 4 5 6 7}#
#\izlaz{1 3 5 7}#
#\izlaz{1 4 7}#
#\izlaz{1 5}#
#\izlaz{1 6}#
#\izlaz{1 7}#
#\izlaz{1}#
\end{upotreba}
\end{miditest}

\begin{miditest}
\begin{upotreba}{3}
#\naslovInt#
#\izlaz{Unesite broj n:}\ulaz{1}#
#\izlaz{1}#
\end{upotreba}
\end{miditest}
\begin{miditest}
\begin{upotreba}{4}
#\naslovInt#
#\izlaz{Unesite broj n:}\ulaz{-23}#
#\izlaz{Neispravan unos.}#
\end{upotreba}
\end{miditest}
\linkresenje{1.3_62}
\end{Exercise}
\ifresenja
\end{Answer}[ref=1.3_62]
\includecode{resenja/1_KontrolaToka/1.3_Petlje/1.3_62.c}
\end{Answer}
\fi



\ifresenja
\section{Rešenja}
\shipoutAnswer
\fi




