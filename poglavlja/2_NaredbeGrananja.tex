% \chapter{Kontrola toka}


\section{Grananja}

% UVODNI PRIMERI

\begin{Exercise}[label=KT_NG_01] 
Napisati program koji ispisuje najmanji od tri uneta cela broja.

\begin{minitest}
\begin{upotreba}{1}
#\naslovInt#
#\izlaz{Unesite tri cela broja:}#
#\ulaz{5 18 -1}#
#\izlaz{Najmanji: -1}#
\end{upotreba}
\end{minitest}
\begin{minitest}
\begin{upotreba}{2}
#\naslovInt#
#\izlaz{Unesite tri cela broja:}#
#\ulaz{0 43 16}#
#\izlaz{Najmanji: 0}#
\end{upotreba}
\end{minitest}
\begin{minitest}
\begin{upotreba}{3}
#\naslovInt#
#\izlaz{Unesite tri cela broja:}#
#\ulaz{-5 -5 -5}#
#\izlaz{Najmanji: -5}#
\end{upotreba}
\end{minitest}

\linkresenje{KT_NG_01}
\end{Exercise}
\ifresenja
 \begin{Answer}[ref=KT_NG_01]
\includecode{resenja/2_KontrolaToka/1.2_NaredbeGrananja/2_1_01.c}
\end{Answer}
\fi


\begin{Exercise}[label=KT_NG_02] 
Napisati program koji za uneti realan broj ispisuje njegovu apsolutnu vrednost zaokruženu na dve decimale.

\begin{minitest}
\begin{upotreba}{1}
#\naslovInt#
#\izlaz{Unesite jedan realan broj:}#
#\ulaz{7.42}#
#\izlaz{Apsolutna vrednost: 7.42}#
\end{upotreba}
\end{minitest}
\begin{minitest}
\begin{upotreba}{2}
#\naslovInt#
#\izlaz{Unesite jedan realan broj:}#
#\ulaz{-562.428}#
#\izlaz{Apsolutna vrednost: 562.43}#
\end{upotreba}
\end{minitest}
\begin{minitest}
\begin{upotreba}{3}
#\naslovInt#
#\izlaz{Unesite jedan realan broj:}#
#\ulaz{0}#
#\izlaz{Apsolutna vrednost: 0.00}#
\end{upotreba}
\end{minitest}

\linkresenje{KT_NG_02}
\end{Exercise}
\ifresenja
 \begin{Answer}[ref=KT_NG_02]
\includecode{resenja/2_KontrolaToka/1.2_NaredbeGrananja/2_1_02.c}
\end{Answer}
\fi

\begin{Exercise}[label=KT_NG_03] 
Napisati program koji za uneti ceo broj ispisuje njegovu recipročnu vrednost zaokruženu na četiri decimale. 
U slučaju neispravnog unosa, ispisati odgovarajuću poruku o grešci.

\begin{miditest}
\begin{upotreba}{1}
#\naslovInt#
#\izlaz{Unesite jedan ceo broj:}\ulaz{22}#
#\izlaz{Reciprocna vrednost: 0.0455}#
\end{upotreba}
\end{miditest}
\begin{miditest}
\begin{upotreba}{2}
#\naslovInt#
#\izlaz{Unesite jedan ceo broj:}\ulaz{-9}#
#\izlaz{Reciprocna vrednost: -0.1111}#
\end{upotreba}
\end{miditest}

\begin{miditest}
\begin{upotreba}{3}
#\naslovInt#
#\izlaz{Unesite jedan ceo broj:}\ulaz{0}#
#\izlaz{Greska: nedozvoljeno je deljenje nulom.}#
\end{upotreba}
\end{miditest}
\begin{miditest}
\begin{upotreba}{4}
#\naslovInt#
#\izlaz{Unesite jedan ceo broj:}\ulaz{57298}#
#\izlaz{Reciprocna vrednost: 0.0000}#
\end{upotreba}
\end{miditest}

\linkresenje{KT_NG_03}
\end{Exercise}
\ifresenja
 \begin{Answer}[ref=KT_NG_03]
\includecode{resenja/2_KontrolaToka/1.2_NaredbeGrananja/2_1_03.c}
\end{Answer}
\fi

\begin{Exercise}[label=KT_NG_04] 
Napisati program koji učitava tri cela broja i ispisuje zbir pozitivnih.

\begin{minitest}
\begin{upotreba}{1}
#\naslovInt#
#\izlaz{Unesite tri cela broja:}#
#\ulaz{1 3 -6}#
#\izlaz{Zbir pozitivnih: 4}#
\end{upotreba}
\end{minitest}
\begin{minitest}
\begin{upotreba}{2}
#\naslovInt#
#\izlaz{Unesite tri cela broja:}#
#\ulaz{-719 -48 -123}#
#\izlaz{Zbir pozitivnih: 0}#
\end{upotreba}
\end{minitest}
\begin{minitest}
\begin{upotreba}{3}
#\naslovInt#
#\izlaz{Unesite tri cela broja:}#
#\ulaz{16 2 576}#
#\izlaz{Zbir pozitivnih: 594}#
\end{upotreba}
\end{minitest}

\linkresenje{KT_NG_04}
\end{Exercise}
\ifresenja
 \begin{Answer}[ref=KT_NG_04]
\includecode{resenja/2_KontrolaToka/1.2_NaredbeGrananja/2_1_04.c}
\end{Answer}
\fi


\begin{Exercise}[label=KT_NG_05] 
U prodavnici je organizovana akcija da svaki kupac dobije najjeftiniji od tri artikla za jedan dinar. 
Napisati program koji za unete cene tri artikla izračunava ukupnu cenu, kao i koliko dinara se uštedi 
zahvaljujući popustu. Cene artikala su pozitivni celi brojevi. U slučaju neispravnog unosa, ispisati odgovarajuću poruku o grešci.

\begin{miditest}
\begin{upotreba}{1}
#\naslovInt#
#\izlaz{Unesite tri cene:}\ulaz{35 125 97}#
#\izlaz{Cena sa popustom: 223 din}#
#\izlaz{Usteda: 34 din}#
\end{upotreba}
\end{miditest}
\begin{miditest}
\begin{upotreba}{2}
#\naslovInt#
#\izlaz{Unesite tri cene:}\ulaz{1034 15 25}#
#\izlaz{Cena sa popustom: 1060 din}#
#\izlaz{Usteda: 14 din}#
\end{upotreba}
\end{miditest}

\begin{miditest}
\begin{upotreba}{3}
#\naslovInt#
#\izlaz{Unesite tri cene:}\ulaz{500 500 500}#
#\izlaz{Cena sa popustom: 1001 din}#
#\izlaz{Usteda: 499 din}#
\end{upotreba}
\end{miditest}
\begin{miditest}
\begin{upotreba}{4}
#\naslovInt#
#\izlaz{Unesite tri cene:}\ulaz{247 -133 126}#
#\izlaz{Greska: neispravan unos cene.}#
\end{upotreba}
\end{miditest}

\linkresenje{KT_NG_05}
\end{Exercise}
\ifresenja
 \begin{Answer}[ref=KT_NG_05]
\includecode{resenja/2_KontrolaToka/1.2_NaredbeGrananja/2_1_05.c}
\end{Answer}
\fi


\begin{Exercise}[label=KT_NG_06] 
Napisati program koji za uneto vreme u formatu \textit{sat:minut} ispisuje koliko je sati i minuta ostalo do ponoći. Broj sati treba da bude iz intervala $[0,24)$, a broj minuta iz intervala $[0,60)$. 
U slučaju neispravnog unosa, ispisati odgovarajuću poruku o grešci.

\begin{miditest}
\begin{upotreba}{1}
#\naslovInt#
#\izlaz{Unesite vreme:}\ulaz{18:19}#
#\izlaz{Do ponoci: 5 sati i 41 minuta}
\end{upotreba}
\end{miditest}
\begin{miditest}
\begin{upotreba}{2}
#\naslovInt#
#\izlaz{Unesite vreme:}\ulaz{23:7}#
#\izlaz{Do ponoci: 0 sati i 53 minuta}
\end{upotreba}
\end{miditest}

\begin{miditest}
\begin{upotreba}{3}
#\naslovInt#
#\izlaz{Unesite vreme:}\ulaz{24:20}#
#\izlaz{Greska: neispravan unos vremena.}
\end{upotreba}
\end{miditest}
\begin{miditest}
\begin{upotreba}{4}
#\naslovInt#
#\izlaz{Unesite vreme:}\ulaz{14:0}#
#\izlaz{Do ponoci: 10 sati i 0 minuta}
\end{upotreba}
\end{miditest}

\linkresenje{KT_NG_06}
\end{Exercise}
\ifresenja
 \begin{Answer}[ref=KT_NG_06]
\includecode{resenja/2_KontrolaToka/1.2_NaredbeGrananja/2_1_06.c}
\end{Answer}
\fi


\begin{Exercise}[label=KT_NG_07] 
Napisati program koji za unetu godinu ispisuje da li je prestupna. 
Godina je neoznačen ceo broj.

\begin{minitest}
\begin{upotreba}{1}
#\naslovInt#
#\izlaz{Unesite godinu:}\ulaz{2016}#
#\izlaz{Godina je prestupna.}#
\end{upotreba}
\end{minitest}
\begin{minitest}
\begin{upotreba}{2}
#\naslovInt#
#\izlaz{Unesite godinu:}\ulaz{1997}#
#\izlaz{Godina nije prestupna.}#
\end{upotreba}
\end{minitest}
\begin{minitest}
\begin{upotreba}{3}
#\naslovInt#
#\izlaz{Unesite godinu:}\ulaz{1900}#
#\izlaz{Godina nije prestupna.}#
\end{upotreba}
\end{minitest}
\linkresenje{KT_NG_07}
\end{Exercise}
\ifresenja
 \begin{Answer}[ref=KT_NG_07]
\includecode{resenja/2_KontrolaToka/1.2_NaredbeGrananja/2_1_07.c}
\end{Answer}
\fi

% KARAKTERI

\begin{Exercise}[label=KT_NG_08] 
Napisati program koji za učitani karakter ispisuje uneti karakter i njegov ASCII kod. 
Ukoliko je uneti karakter malo (veliko) slovo, ispisati i odgovarajuće veliko (malo) slovo i njegov ASCII kod.

\begin{miditest}
\begin{upotreba}{1}
#\naslovInt#
#\izlaz{Unesite karakter:}\ulaz{0}#
#\izlaz{Uneti karakter: 0} #
#\izlaz{ASCII kod: 48}#
\end{upotreba}
\end{miditest}
\begin{miditest}
\begin{upotreba}{2}
#\naslovInt#
#\izlaz{Unesite karakter:}\ulaz{?}#
#\izlaz{Uneti karakter: ?}#
#\izlaz{ASCII kod: 63}#
\end{upotreba}
\end{miditest}

\begin{miditest}
\begin{upotreba}{3}
#\naslovInt#
#\izlaz{Unesite karakter:}\ulaz{A}#
#\izlaz{Uneti karakter: A}#
#\izlaz{ASCII kod: 65}#
#\izlaz{Odgovarajuce malo slovo: a}#
#\izlaz{ASCII kod: 97}#
\end{upotreba}
\end{miditest}
\begin{miditest}
\begin{upotreba}{4}
#\naslovInt#
#\izlaz{Unesite karakter:}\ulaz{v}#
#\izlaz{Uneti karakter: v}#
#\izlaz{ASCII kod: 118}#
#\izlaz{Odgovarajuce veliko slovo: V}#
#\izlaz{ASCII kod: 86}#
\end{upotreba}
\end{miditest}

\linkresenje{KT_NG_08}
\end{Exercise}
\ifresenja
 \begin{Answer}[ref=KT_NG_08]
\includecode{resenja/2_KontrolaToka/1.2_NaredbeGrananja/2_1_08.c}
\end{Answer}
\fi


% \begin{Exercise}[label=KT_NG_09] 
% Napisati program koji učitava tri karaktera od kojih prvi mora biti cifra i 
% ispisuje zbir ASCII kodova svih unetih karaktera, pomnozen sa vrednoscu unete cifre. 
% U slučaju neispravnog unosa, ispisati odgovarajuću poruku o grešci.
% \napomena{Karakteri koji se unose su razmaknuti belinama.}
% 
% \begin{miditest}
% \begin{upotreba}{1}
% #\naslovInt#
% #\izlaz{Unesite karaktere:}\ulaz{1 A c}#
% #\izlaz{Rezultat: 213}#
% \end{upotreba}
% \end{miditest}
% \begin{miditest}
% \begin{upotreba}{2}
% #\naslovInt#
% #\izlaz{Unesite karaktere:}\ulaz{2 a \_}#
% #\izlaz{Rezultat: 484}#
% \end{upotreba}
% \end{miditest}
% 
% \begin{miditest}
% \begin{upotreba}{3}
% #\naslovInt#
% #\izlaz{Unesite karaktere:}\ulaz{4 3 9}#
% #\izlaz{Rezultat: 640}#
% \end{upotreba}
% \end{miditest}
% \begin{miditest}
% \begin{upotreba}{4}
% #\naslovInt#
% #\izlaz{Unesite karaktere:}\ulaz{B 6 (}#
% #\izlaz{Greska: prvi karakter nije cifra.}#
% \end{upotreba}
% \end{miditest}
% \linkresenje{KT_NG_09}
% \end{Exercise}
% \ifresenja
%  \begin{Answer}[ref=KT_NG_09]
% \includecode{resenja/2_KontrolaToka/1.2_NaredbeGrananja/2_1_09.c}
% \end{Answer}
% \fi


\begin{Exercise}[label=KT_NG_10] 
Napisati program koji učitava tri karaktera. Ispitati da li među unetim karakterima ima cifara i ako je tako
odrediti proizvod tih cifara. 
Ukoliko među unetim karakterima nema cifara, program treba da ispiše odgovarajuću poruku.
\napomena{Karakteri koji se unose su razmaknuti blanko znacima.}

\begin{miditest}
\begin{upotreba}{1}
#\naslovInt#
#\izlaz{Unesite karaktere:}\ulaz{A 5 3}#
#\izlaz{Proizvod cifara: 15}#
\end{upotreba}
\end{miditest}
\begin{miditest}
\begin{upotreba}{2}
#\naslovInt#
#\izlaz{Unesite karaktere:}\ulaz{k ! m}#
#\izlaz{Medju unetim karakterima nema cifara.}#
\end{upotreba}
\end{miditest}

\begin{miditest}
\begin{upotreba}{3}
#\naslovInt#
#\izlaz{Unesite karaktere:}\ulaz{9 9 9}#
#\izlaz{Proizvod cifara: 729}#
\end{upotreba}
\end{miditest}
\begin{miditest}
\begin{upotreba}{4}
#\naslovInt#
#\izlaz{Unesite karaktere:}\ulaz{a 8 0}#
#\izlaz{Proizvod cifara: 0}#
\end{upotreba}
\end{miditest}

\linkresenje{KT_NG_10}
\end{Exercise}
\ifresenja
 \begin{Answer}[ref=KT_NG_10]
\includecode{resenja/2_KontrolaToka/1.2_NaredbeGrananja/2_1_10.c}
\end{Answer}
\fi


\begin{Exercise}[label=KT_NG_11] 
Kasirka unosi šifru artikla koja se zadaje kao tri spojena karaktera koji mogu biti mala slova, velika slova ili cifre. 
U kasi, sve šifre su zapisane malim slovima i ciframa. Napisati program koji kasirkin unos konvertuje u unos
koji je odgovarajući za kasu, tj. koji sva velika slova pretvara u odgovarajuća mala, a ostale karaktere ne menja. 
U slučaju neispravnog unosa šifre, ispisati odgovarajuću poruku o grešci.

\begin{miditest}
\begin{upotreba}{1}
#\naslovInt#
#\izlaz{Unesite sifru:}\ulaz{aBc}#
#\izlaz{Rezultat: abc}#
\end{upotreba}
\end{miditest}
\begin{miditest}
\begin{upotreba}{2}
#\naslovInt#
#\izlaz{Unesite sifru:}\ulaz{a?!}#
#\izlaz{Greška: ? je neispravan karakter.}#
\end{upotreba}
\end{miditest}

\begin{miditest}
\begin{upotreba}{3}
#\naslovInt#
#\izlaz{Unesite karaktere:}\ulaz{5A5}#
#\izlaz{Rezultat: 5a5}#
\end{upotreba}
\end{miditest}
\begin{miditest}
\begin{upotreba}{4}
#\naslovInt#
#\izlaz{Unesite karaktere:}\ulaz{123}#
#\izlaz{Rezultat: 123}#
\end{upotreba}
\end{miditest}
\linkresenje{KT_NG_11}
\end{Exercise}
\ifresenja
 \begin{Answer}[ref=KT_NG_11]
\includecode{resenja/2_KontrolaToka/1.2_NaredbeGrananja/2_1_11.c}
%Pogledajte zadatak \ref{UZ_NG_08}.
\end{Answer}
\fi


% CIFRE 


\begin{Exercise}[label=KT_NG_12] 
Napisati program koji za uneti četvorocifreni broj ispisuje njegovu najveću cifru. 
U slučaju neispravnog unosa, ispisati odgovarajuću poruku o grešci.

\begin{miditest}
\begin{upotreba}{1}
#\naslovInt#
#\izlaz{Unesite cetvorocifreni broj:}\ulaz{6835}#
#\izlaz{Najveca cifra je: 8}#
\end{upotreba}
\end{miditest}
\begin{miditest}
\begin{upotreba}{2}
#\naslovInt#
#\izlaz{Unesite cetvorocifreni broj:}\ulaz{7777}#
#\izlaz{Najveca cifra je: 7}#
\end{upotreba}
\end{miditest}

\begin{miditest}
\begin{upotreba}{3}
#\naslovInt#
#\izlaz{Unesite cetvorocifreni broj:}\ulaz{238}#
#\izlaz{Greska: niste uneli cetvorocifreni broj.}#
\end{upotreba}
\end{miditest}
\begin{miditest}
\begin{upotreba}{4}
#\naslovInt#
#\izlaz{Unesite cetvorocifreni broj:}\ulaz{-2002}#
#\izlaz{Najveca cifra je: 2}#
\end{upotreba}
\end{miditest}

\linkresenje{KT_NG_12}
\end{Exercise}
\ifresenja
 \begin{Answer}[ref=KT_NG_12]
\includecode{resenja/2_KontrolaToka/1.2_NaredbeGrananja/2_1_12.c}
\end{Answer}
\fi


\begin{Exercise}[label=KT_NG_13] 
Trocifreni broj je Armstrongov ako je jednak zbiru kubova svojih cifara. 
Napisati program koji za dati pozitivan trocifreni broj proverava 
da li je Armstrongov.
U slučaju neispravnog unosa, ispisati odgovarajuću poruku o grešci.

\begin{miditest}
\begin{upotreba}{1}
#\naslovInt#
#\izlaz{Unesite pozitivan trocifreni broj:}#
#\ulaz{153}#
#\izlaz{Broj je Armstrongov.}#
\end{upotreba}
\end{miditest}
\begin{miditest}
\begin{upotreba}{2}
#\naslovInt#
#\izlaz{Unesite pozitivan trocifreni broj:}#
#\ulaz{111}#
#\izlaz{Broj nije Armstrongov.}#
\end{upotreba}
\end{miditest}

\begin{miditest}
\begin{upotreba}{3}
#\naslovInt#
#\izlaz{Unesite pozitivan trocifreni broj:}#
#\ulaz{84}#
#\izlaz{Greska: niste uneli pozitivan trocifreni broj.}#
\end{upotreba}
\end{miditest}
\begin{miditest}
\begin{upotreba}{4}
#\naslovInt#
#\izlaz{Unesite pozitivan trocifreni broj:}#
#\ulaz{371}#
#\izlaz{Broj je Armstrongov.}#
\end{upotreba}
\end{miditest}

\linkresenje{KT_NG_13}
\end{Exercise}
\ifresenja
 \begin{Answer}[ref=KT_NG_13]
\includecode{resenja/2_KontrolaToka/1.2_NaredbeGrananja/2_1_13.c}
\end{Answer}
\fi


\begin{Exercise}[label=KT_NG_14] 
Napisati program koji ispisuje proizvod parnih cifara unetog četvorocifrenog broja. 
U slučaju neispravnog unosa, ispisati odgovarajuću poruku o grešci.

\begin{miditest}
\begin{upotreba}{1}
#\naslovInt#
#\izlaz{Unesite cetvorocifreni broj:}\ulaz{8123}#
#\izlaz{Proizvod parnih cifara: 16}#
\end{upotreba}
\end{miditest}
\begin{miditest}
\begin{upotreba}{2}
#\naslovInt#
#\izlaz{Unesite cetvorocifreni broj:}\ulaz{3579}#
#\izlaz{Nema parnih cifara.}#
\end{upotreba}
\end{miditest}

\begin{miditest}
\begin{upotreba}{3}
#\naslovInt#
#\izlaz{Unesite cetvorocifreni broj:}\ulaz{288}#
#\izlaz{Greska: niste uneli cetvorocifreni broj.}#
\end{upotreba}
\end{miditest}
\begin{miditest}
\begin{upotreba}{4}
#\naslovInt#
#\izlaz{Unesite cetvorocifreni broj:}\ulaz{-1234}#
#\izlaz{Proizvod parnih cifara: 8}#
\end{upotreba}
\end{miditest}

\linkresenje{KT_NG_14}
\end{Exercise}
\ifresenja
 \begin{Answer}[ref=KT_NG_14]
\includecode{resenja/2_KontrolaToka/1.2_NaredbeGrananja/2_1_14.c}
\end{Answer}
\fi


\begin{Exercise}[label=KT_NG_15] 
 Napisati program koji učitava četvorocifreni broj i ispisuje broj koji se dobija 
 kada se unetom broju razmene najmanja i najveća cifra.  
 U slučaju da se najmanja ili najveća cifra pojavljuju na više pozicija, 
 uzeti prvo pojavljivanje, gledajući sa desna na levo.
 U slučaju neispravnog unosa, ispisati odgovarajuću poruku o grešci.
 
\begin{miditest}
\begin{upotreba}{1}
#\naslovInt#
#\izlaz{Unesite cetvorocifreni broj:}\ulaz{2863}#
#\izlaz{Rezultat: 8263}#
\end{upotreba}
\end{miditest}
\begin{miditest}
\begin{upotreba}{2}
#\naslovInt#
#\izlaz{Unesite cetvorocifreni broj:}\ulaz{1192}#
#\izlaz{Rezultat: 1912}#
\end{upotreba}
\end{miditest}

\begin{miditest}
\begin{upotreba}{3}
#\naslovInt#
#\izlaz{Unesite cetvorocifreni broj:}\ulaz{247}#
#\izlaz{Greska: niste uneli cetvorocifreni broj.}#
\end{upotreba}
\end{miditest}
\begin{miditest}
\begin{upotreba}{4}
#\naslovInt#
#\izlaz{Unesite cetvorocifreni broj:}\ulaz{-4239}#
#\izlaz{Rezultat: -4932}#
\end{upotreba}
\end{miditest}

\linkresenje{KT_NG_15}
\end{Exercise}
\ifresenja
 \begin{Answer}[ref=KT_NG_15]
\includecode{resenja/2_KontrolaToka/1.2_NaredbeGrananja/2_1_15.c}
\end{Answer}
\fi


\begin{Exercise}[label=KT_NG_16] 
Napisati program koji za uneti četvorocifreni broj proverava
da li su njegove cifre uređene neopadajuće, nerastuće ili nisu
uređene i štampa odgovarajuću poruku.  
U slučaju neispravnog unosa, ispisati odgovarajuću poruku o grešci.

\begin{miditest}
\begin{upotreba}{1}
#\naslovInt#
#\izlaz{Unesite cetvorocifreni broj:}\ulaz{1389}#
#\izlaz{Cifre su uredjene neopadajuce.}#
\end{upotreba}
\end{miditest}
\begin{miditest}
\begin{upotreba}{2}
#\naslovInt#
#\izlaz{Unesite cetvorocifreni broj:}\ulaz{-9622}#
#\izlaz{Cifre su uredjene nerastuce.}#\end{upotreba}
\end{miditest}

\begin{miditest}
\begin{upotreba}{3}
#\naslovInt#
#\izlaz{Unesite cetvorocifreni broj:}\ulaz{88}#
#\izlaz{Greska: niste uneli cetvorocifreni broj.}#
\end{upotreba}
\end{miditest}
\begin{miditest}
\begin{upotreba}{4}
#\naslovInt#
#\izlaz{Unesite cetvorocifreni broj:}\ulaz{6792}#
#\izlaz{Cifre nisu uredjene.}#
\end{upotreba}
\end{miditest}

\linkresenje{KT_NG_16}
\end{Exercise}
\ifresenja
 \begin{Answer}[ref=KT_NG_16]
\includecode{resenja/2_KontrolaToka/1.2_NaredbeGrananja/2_1_16.c}
\end{Answer}
\fi


% MATEMATICKI + UGNJEZDENA GRANANJA


\begin{Exercise}[label=KT_NG_17]
Napisati program koji ispituje da li se tačke $A(x_1, y_1)$ i $B(x_2, y_2)$ nalaze u 
istom kvadrantu. Koordinate tačaka su realni brojevi jednostruke tačnosti.

\begin{miditest}
\begin{upotreba}{1}
#\naslovInt#
#\izlaz{Unesite koordinate tacke A:}\ulaz{1.5 6}#
#\izlaz{Unesite koordinate tacke B:}\ulaz{2.33 9.8}#
#\izlaz{Tacke se nalaze u istom kvadrantu.}#
\end{upotreba}
\end{miditest}
\begin{miditest}
\begin{upotreba}{2}
#\naslovInt#
#\izlaz{Unesite koordinate tacke A:}\ulaz{-3 6}#
#\izlaz{Unesite koordinate tacke B:}\ulaz{0.33 -5}#
#\izlaz{Tacke se ne nalaze u istom kvadrantu.}#
\end{upotreba}
\end{miditest}

\begin{miditest}
\begin{upotreba}{3}
#\naslovInt#
#\izlaz{Unesite koordinate tacke A:}\ulaz{0 -6}#
#\izlaz{Unesite koordinate tacke B:}\ulaz{-1 -99.66}#
#\izlaz{Tacke se nalaze u istom kvadrantu.}#
\end{upotreba}
\end{miditest}
\begin{miditest}
\begin{upotreba}{4}
#\naslovInt#
#\izlaz{Unesite koordinate tacke A:}\ulaz{3 -6}#
#\izlaz{Unesite koordinate tacke B:}\ulaz{-0.33 0}#
#\izlaz{Tacke se ne nalaze u istom kvadrantu.}#
\end{upotreba}
\end{miditest}

\linkresenje{KT_NG_17}
\end{Exercise}
\ifresenja
 \begin{Answer}[ref=KT_NG_17]
\includecode{resenja/2_KontrolaToka/1.2_NaredbeGrananja/2_1_17.c}
\end{Answer}
\fi


\begin{Exercise}[label=KT_NG_18]
Napisati program koji ispituje da li se tačke $A(x_1, y_1)$, $B(x_2, y_2)$ i $C(x_3, y_3)$ 
nalaze na istoj pravoj. \\

\begin{miditest}
\begin{upotreba}{1}
#\naslovInt#
#\izlaz{Unesite koordinate tacke A:}\ulaz{1.5 6}#
#\izlaz{Unesite koordinate tacke B:}\ulaz{-2.5 -10}#
#\izlaz{Unesite koordinate tacke C:}\ulaz{3 12}#
#\izlaz{Tacke se nalaze na istoj pravoj.}#
\end{upotreba}
\end{miditest}
\begin{miditest}
\begin{upotreba}{2}
#\naslovInt#
#\izlaz{Unesite koordinate tacke A:}\ulaz{-1.5 3}#
#\izlaz{Unesite koordinate tacke B:}\ulaz{-0.4 9.8}#
#\izlaz{Unesite koordinate tacke C:}\ulaz{2 3}#
#\izlaz{Tacke se ne nalaze na istoj pravoj.}#
\end{upotreba}
\end{miditest}

\begin{miditest}
\begin{upotreba}{3}
#\naslovInt#
#\izlaz{Unesite koordinate tacke A:}\ulaz{1.55 6}#
#\izlaz{Unesite koordinate tacke B:}\ulaz{-8.4 9.8}#
#\izlaz{Unesite koordinate tacke C:}\ulaz{5 4.682412}#
#\izlaz{Tacke se nalaze na istoj pravoj.}#
\end{upotreba}
\end{miditest}
\begin{miditest}
\begin{upotreba}{4}
#\naslovInt#
#\izlaz{Unesite koordinate tacke A:}\ulaz{5.5 3.5}#
#\izlaz{Unesite koordinate tacke B:}\ulaz{5.5 3.5}#
#\izlaz{Unesite koordinate tacke C:}\ulaz{5.5 3.5}#
#\izlaz{Tacke se nalaze na istoj pravoj.}#
\end{upotreba}
\end{miditest}

\begin{miditest}
\begin{upotreba}{5}
#\naslovInt#
#\izlaz{Unesite koordinate tacke A:}\ulaz{1 2}#
#\izlaz{Unesite koordinate tacke B:}\ulaz{1 2}#
#\izlaz{Unesite koordinate tacke C:}\ulaz{-56 1.3}#
#\izlaz{Tacke se nalaze na istoj pravoj.}#
\end{upotreba}
\end{miditest}
\begin{miditest}
\begin{upotreba}{6}
#\naslovInt#
#\izlaz{Unesite koordinate tacke A:}\ulaz{3.4 3.5}#
#\izlaz{Unesite koordinate tacke B:}\ulaz{-10 -1}#
#\izlaz{Unesite koordinate tacke C:}\ulaz{-10 -1}#
#\izlaz{Tacke se nalaze na istoj pravoj.}#
\end{upotreba}
\end{miditest}

\linkresenje{KT_NG_18}
\end{Exercise}
\ifresenja
 \begin{Answer}[ref=KT_NG_18]
\includecode{resenja/2_KontrolaToka/1.2_NaredbeGrananja/2_1_18.c}
\end{Answer}
\fi


\begin{Exercise}[label=KT_NG_19] 
 Napisati program za rad sa intervalima. Za dva celobrojna intervala $[a_1, b_1]$ i
$[a_2, b_2]$ program treba da odredi:
\begin{itemize}
\item [(a)] dužinu preseka datih intervala
\item [(b)] presečni interval datih intervala
\item [(c)] dužinu prave koju pokrivaju dati intervali
\item [(d)] najmanji interval koji sadrži date intervale.
\end{itemize}

\begin{miditest}
\begin{upotreba}{1}
#\naslovInt#
#\izlaz{Unesite a1, b1, a2 i b2:}\ulaz{2 9 4 11}#
#\izlaz{Duzina preseka: 5}#
#\izlaz{Presecni interval: [4,9]}#
#\izlaz{Duzina koju pokrivaju: 9}#
#\izlaz{Najmanji interval: [2, 11]}#
\end{upotreba}
\end{miditest}
\begin{miditest}
\begin{upotreba}{2}
#\naslovInt#
#\izlaz{Unesite a1, b1, a2 i b2:}\ulaz{1 2 10 13}#
#\izlaz{Duzina preseka: 0}#
#\izlaz{Presecni interval: prazan}#
#\izlaz{Duzina koju pokrivaju: 4}#
#\izlaz{Najmanji interval: [1, 13]}#
\end{upotreba}
\end{miditest}

\linkresenje{KT_NG_19}
\end{Exercise}
\ifresenja
 \begin{Answer}[ref=KT_NG_19]
\includecode{resenja/2_KontrolaToka/1.2_NaredbeGrananja/2_1_19.c}
\end{Answer}
\fi


\begin{Exercise}[label=KT_NG_20] 
Napisati program koji za unete koeficijente kvadratne jednačine ispisuje koliko realnih rešenja jednačina ima i ako ih ima, 
ispisuje ih zaokružene na dve decimale.

\begin{miditest}
\begin{upotreba}{1}
#\naslovInt#
#\izlaz{Unesite koeficijente A, B i C:}\ulaz{1 3 2}#
#\izlaz{Jednacina ima dva razlicita realna resenja:}#
#\izlaz{-1.00 i -2.00}#
\end{upotreba}
\end{miditest}
\begin{miditest}
\begin{upotreba}{2}
#\naslovInt#
#\izlaz{Unesite koeficijente A, B i C:}\ulaz{1 1 1}#
#\izlaz{Jednacina nema resenja.}#
\end{upotreba}
\end{miditest}

\linkresenje{KT_NG_20}
\end{Exercise}
\ifresenja
 \begin{Answer}[ref=KT_NG_20]
\includecode{resenja/2_KontrolaToka/1.2_NaredbeGrananja/2_1_20.c}
\end{Answer}
\fi


\begin{Exercise}[label=KT_NG_21] 
 U nizu 12345678910111213....9899 ispisani su redom brojevi od 1 do 99. Napisati program koji za uneti 
 ceo broj $k$ (1 \leq $k$ \leq 189) ispisuje cifru koja se nalazi na $k$-toj poziciji datog niza.
 U slučaju neispravnog unosa, ispisati odgovarajuću poruku o grešci.
 
\begin{miditest}
\begin{upotreba}{1}
#\naslovInt#
#\izlaz{Unesite k:}\ulaz{13}#
#\izlaz{Na 13-toj poziciji je broj 1.}#
\end{upotreba}
\end{miditest}
\begin{miditest}
\begin{upotreba}{2}
#\naslovInt#
#\izlaz{Unesite k:}\ulaz{105}#
#\izlaz{Na 105-toj poziciji je broj 7.}#
\end{upotreba}
\end{miditest}

\begin{miditest}
\begin{upotreba}{3}
#\naslovInt#
#\izlaz{Unesite k:}\ulaz{200}#
#\izlaz{Greska: neispravan unos pozicije.}#
\end{upotreba}
\end{miditest}
\begin{miditest}
\begin{upotreba}{4}
#\naslovInt#
#\izlaz{Unesite k:}\ulaz{10}#
#\izlaz{Na 10-toj poziciji je broj 1.}#
\end{upotreba}
\end{miditest}

\linkresenje{KT_NG_21}
\end{Exercise}
\ifresenja
 \begin{Answer}[ref=KT_NG_21]
\includecode{resenja/2_KontrolaToka/1.2_NaredbeGrananja/2_1_21.c}
\end{Answer}
\fi


\begin{Exercise}[label=KT_NG_22] 
Data je funkcija $f(x) = 2 \cdot cos(x) - x^3$. Napisati program koji za učitanu 
vrednost realne promenljive $x$ i vrednost celobrojne promenljive $k$ koja može 
biti 1, 2 ili 3 izračunava vrednost funkcije $F(x, k)$ koja se dobija tako što se 
funkcija $f$ primeni $k$-puta ($F(x,1) = f(x), F(x, 2) = f(f(x)), F(x,3) = f(f(f(x)))$).
Dobijenu vrednosti ispisati zaokruženu na dve decimale. 
U slučaju neispravnog unosa, ispisati odgovarajuću poruku o grešci.

\begin{minitest}
\begin{upotreba}{1}
#\naslovInt#
#\izlaz{Unesite redom x i k:}#
#\ulaz{2.31 2}#
#\izlaz{F(2.31, 2) = 2557.52}#
\end{upotreba}
\end{minitest}
\begin{minitest}
\begin{upotreba}{2}
#\naslovInt#
#\izlaz{Unesite redom x i k:}#
#\ulaz{2.31 0}#
#\izlaz{Greska: nedozvoljena}#
#\izlaz{vrednost za k.}#
\end{upotreba}
\end{minitest}
\begin{minitest}
\begin{upotreba}{3}
#\naslovInt#
#\izlaz{Unesite redom x i k:}#
#\ulaz{12 1}#
#\izlaz{F(12, 1) = -1726.31}#
\end{upotreba}
\end{minitest}

\linkresenje{KT_NG_22}
\end{Exercise}
\ifresenja
 \begin{Answer}[ref=KT_NG_22]
\includecode{resenja/2_KontrolaToka/1.2_NaredbeGrananja/2_1_22.c}
\end{Answer}
\fi

%\subsection{Switch-case}


\begin{Exercise}[label=KT_NG_23] 
Napisati program koji za uneti redni broj dana u nedelji ispisuje ime odgovarajućeg dana. 
U slučaju neispravnog unosa, ispisati odgovarajuću poruku o grešci.

\begin{minitest}
\begin{upotreba}{1}
#\naslovInt#
#\izlaz{Unesite broj:}\ulaz{4}#
#\izlaz{U pitanju je: cetvrtak}#
\end{upotreba}
\end{minitest}
\begin{minitest}
\begin{upotreba}{2}
#\naslovInt#
#\izlaz{Unesite broj:}\ulaz{8}#
#\izlaz{Greska: neispravan unos}#
#\izlaz{dana.}#
\end{upotreba}
\end{minitest}
\begin{minitest}
\begin{upotreba}{3}
#\naslovInt#
#\izlaz{Unesite broj:}\ulaz{7}#
#\izlaz{U pitanju je: nedelja}#
\end{upotreba}
\end{minitest}

\linkresenje{KT_NG_23}
\end{Exercise}
\ifresenja
 \begin{Answer}[ref=KT_NG_23]
\includecode{resenja/2_KontrolaToka/1.2_NaredbeGrananja/2_1_23.c}
\end{Answer}
\fi


\begin{Exercise}[label=KT_NG_24] 
Napisati program koji za uneti karakter ispituje da li je samoglasnik ili ne.

\begin{miditest}
\begin{upotreba}{1}
#\naslovInt#
#\izlaz{Unesite jedan karakter:}\ulaz{A}#
#\izlaz{Uneti karakter je samoglasnik.}#
\end{upotreba}
\end{miditest}
\begin{miditest}
\begin{upotreba}{2}
#\naslovInt#
#\izlaz{Unesite jedan karakter:}\ulaz{i}#
#\izlaz{Uneti karakter je samoglasnik.}#
\end{upotreba}
\end{miditest}

\begin{miditest}
\begin{upotreba}{3}
#\naslovInt#
#\izlaz{Unesite jedan karakter:}\ulaz{f}#
#\izlaz{Uneti karakter nije samoglasnik.}#
\end{upotreba}
\end{miditest}
\begin{miditest}
\begin{upotreba}{4}
#\naslovInt#
#\izlaz{Unesite jedan karakter:}\ulaz{4}#
#\izlaz{Uneti karakter nije samoglasnik.}#
\end{upotreba}
\end{miditest}

\linkresenje{KT_NG_24}
\end{Exercise}
\ifresenja
 \begin{Answer}[ref=KT_NG_24]
\includecode{resenja/2_KontrolaToka/1.2_NaredbeGrananja/2_1_24.c}
\end{Answer}
\fi


\begin{Exercise}[label=KT_NG_25] 
Napisatiti program koji učitava dva cela broja i jedan od karaktera +, -, *, / ili \% i ispisuje vrednost izraza 
dobijenog primenom date operacije nad učitanim vrednostima.
U slučaju neispravnog unosa, ispisati odgovarajuću poruku o grešci. 

 \begin{miditest}
\begin{upotreba}{1}
#\naslovInt#
#\izlaz{Unesite izraz:}\ulaz{8 - 11}#
#\izlaz{Rezultat je: -3}#
\end{upotreba}
\end{miditest}
\begin{miditest}
\begin{upotreba}{2}
#\naslovInt#
#\izlaz{Unesite izraz:}\ulaz{14 / 0}#
#\izlaz{Greska: deljenje nulom.}#
\end{upotreba}
\end{miditest}

\begin{miditest}
\begin{upotreba}{3}
#\naslovInt#
#\izlaz{Unesite izraz:}\ulaz{5 ? 7}#
#\izlaz{Greska: nepoznat operator.}#
\end{upotreba}
\end{miditest}
\begin{miditest}
\begin{upotreba}{4}
#\naslovInt#
#\izlaz{Unesite izraz:}\ulaz{19 / 5}#
#\izlaz{Rezultat je: 3}#
\end{upotreba}
\end{miditest}

\linkresenje{KT_NG_25}
\end{Exercise}
\ifresenja
 \begin{Answer}[ref=KT_NG_25]
\includecode{resenja/2_KontrolaToka/1.2_NaredbeGrananja/2_1_25.c}
\end{Answer}
\fi


\begin{Exercise}[label=KT_NG_26] 
Napisati program koji za uneti datum u formatu \textit{dan.mesec.} ispisuje godišnje doba kojem pripadaju. 
\napomena{Pretpostaviti da je unos ispravan.}

\begin{minitest}
\begin{upotreba}{1}
#\naslovInt#
#\izlaz{Unesite dan i mesec:}\ulaz{14.10.}#
#\izlaz{jesen}#
\end{upotreba}
\end{minitest}
\begin{minitest}
\begin{upotreba}{2}
#\naslovInt#
#\izlaz{Unesite dan i mesec:}\ulaz{2.8.}#
#\izlaz{leto}#
\end{upotreba}
\end{minitest}
\begin{minitest}
\begin{upotreba}{3}
#\naslovInt#
#\izlaz{Unesite dan i mesec:}\ulaz{27.2.}#
#\izlaz{zima}#
\end{upotreba}
\end{minitest}

\linkresenje{KT_NG_26}
\end{Exercise}
\ifresenja
 \begin{Answer}[ref=KT_NG_26]
\includecode{resenja/2_KontrolaToka/1.2_NaredbeGrananja/2_1_26.c}
\end{Answer}
\fi


\begin{Exercise}[label=KT_NG_27] 
Napisati program koji za unetu godinu i mesec ispisuje naziv meseca kao i koliko dana ima u tom mesecu te godine.
U slučaju neispravnog unosa, ispisati odgovarajuću poruku o grešci. 

\begin{minitest}
\begin{upotreba}{1}
#\naslovInt#
#\izlaz{Unesite godinu:}\ulaz{2018}#
#\izlaz{Unesite mesec:}\ulaz{1}#
#\izlaz{Januar, 31 dan}#
\end{upotreba}
\end{minitest}
\begin{minitest}
\begin{upotreba}{2}
#\naslovInt#
#\izlaz{Unesite godinu:}\ulaz{2000}#
#\izlaz{Unesite mesec:}\ulaz{2}#
#\izlaz{Februar, 29 dana}#
\end{upotreba}
\end{minitest}
\begin{minitest}
\begin{upotreba}{3}
#\naslovInt#
#\izlaz{Unesite godinu:}\ulaz{2018}#
#\izlaz{Unesite mesec:}\ulaz{13}#
#\izlaz{Greska: neispravan unos}#
#\izlaz{meseca.}#
\end{upotreba}
\end{minitest}

\linkresenje{KT_NG_27}
\end{Exercise}
\ifresenja
 \begin{Answer}[ref=KT_NG_27]
\includecode{resenja/2_KontrolaToka/1.2_NaredbeGrananja/2_1_27.c}
\end{Answer}
\fi


\begin{Exercise}[label=KT_NG_28] 
 Napisati program koji za uneti datum u formatu \textit{dan.mesec.godina.} proverava da li je korektan.
 
\begin{miditest}
\begin{upotreba}{1}
#\naslovInt#
#\izlaz{Unesite datum:}\ulaz{25.11.1983.}#
#\izlaz{Datum je korektan.}#
\end{upotreba}
\end{miditest}
\begin{miditest}
\begin{upotreba}{2}
#\naslovInt#
#\izlaz{Unesite datum:}\ulaz{1.17.2004.}#
#\izlaz{Datum nije korektan.}#
\end{upotreba}
\end{miditest}

\linkresenje{KT_NG_28}
\end{Exercise}
\ifresenja
 \begin{Answer}[ref=KT_NG_28]
\includecode{resenja/2_KontrolaToka/1.2_NaredbeGrananja/2_1_28.c}
\end{Answer}
\fi


\begin{Exercise}[label=KT_NG_29] 
 Napisati program koji za korektno unet datum u formatu \textit{dan.mesec.godina.} ispisuje datum prethodnog dana. 
 
\begin{minitest}
\begin{upotreba}{1}
#\naslovInt#
#\izlaz{Unesite datum:}#
#\ulaz{30.4.2008.}#
#\izlaz{Prethodni datum:}#
#\izlaz{29.4.2008.}#
\end{upotreba}
\end{minitest}
\begin{minitest}
\begin{upotreba}{2}
#\naslovInt#
#\izlaz{Unesite datum:}#
#\ulaz{1.12.2005.}#
#\izlaz{Prethodni datum:}#
#\izlaz{30.11.2005.}#
\end{upotreba}
\end{minitest}
\begin{minitest}
\begin{upotreba}{3}
#\naslovInt#
#\izlaz{Unesite datum:}#
#\ulaz{1.1.2019.}#
#\izlaz{Prethodni datum:}#
#\izlaz{31.12.2018.}#
\end{upotreba}
\end{minitest}

\linkresenje{KT_NG_29}
\end{Exercise}
\ifresenja
 \begin{Answer}[ref=KT_NG_29]
\includecode{resenja/2_KontrolaToka/1.2_NaredbeGrananja/2_1_29.c}
\end{Answer}
\fi


\begin{Exercise}[label=KT_NG_30] 
 Napisati program koji za korektno unet datum u formatu \textit{dan.mesec.godina.} ispisuje datum narednog dana. 
 
\begin{minitest}
\begin{upotreba}{1}
#\naslovInt#
#\izlaz{Unesite datum:}#
#\ulaz{30.4.2008.}#
#\izlaz{Naredni datum:}#
#\izlaz{1.5.2008.}#
\end{upotreba}
\end{minitest}
\begin{minitest}
\begin{upotreba}{2}
#\naslovInt#
#\izlaz{Unesite datum:}#
#\ulaz{1.12.2005.}#
#\izlaz{Naredni datum:}#
#\izlaz{2.12.2005.}#
\end{upotreba}
\end{minitest}
\begin{minitest}
\begin{upotreba}{3}
#\naslovInt#
#\izlaz{Unesite datum:}#
#\ulaz{31.12.2008.}#
#\izlaz{Naredni datum:}#
#\izlaz{1.1.2009.}#
\end{upotreba}
\end{minitest}

\linkresenje{KT_NG_30}
\end{Exercise}
\ifresenja
\begin{Answer}[ref=KT_NG_30]
Pogledajte zadatak \ref{KT_NG_29}.
\end{Answer}
\fi


% \begin{Exercise}[label=KT_NG_31]
% Korisnik unosi tri cela broja $P$, $Q$ i $R$ i dva karaktera, $op1$ i $op2$. 
% Ovi karakteri predstavljaju operacije nad unetim brojevima i imaju naredno značenje:
% \begin{itemize}
% \item karakter \textbf{k} predstavlja  logičku konjukciju
% \item karakter \textbf{d} predstavlja  logičku disjunkciju
% \item karakter \textbf{m} predstavlja  relaciju manje
% \item karakter \textbf{v} predstavlja  relaciju veće
% \end{itemize}
% Program treba da sračuna vrednost izraza \kckod{P op1 Q op2 R} i da ga ispiše.
% U slučaju neispravnog unosa, ispisati odgovarajuću poruku o grešci. 
% 
% \begin{miditest}
% \begin{upotreba}{1}
% #\naslovInt#
% #\izlaz{Unesite izraz:}\ulaz{0 k 1 m 2}#
% #\izlaz{Vrednost izraza: 1}#
% \end{upotreba}
% \end{miditest}
% \begin{miditest}
% \begin{upotreba}{2}
% #\naslovInt#
% #\izlaz{Unesite izraz:}\ulaz{-3 d -1 k 0}#
% #\izlaz{Vrednost izraza: 0}#
% \end{upotreba}
% \end{miditest}
% 
% \begin{miditest}
% \begin{upotreba}{3}
% #\naslovInt#
% #\izlaz{Unesite izraz:}\ulaz{-3 k 1 d 0}#
% #\izlaz{Vrednost izraza: 1}#
% \end{upotreba}
% \end{miditest}
% \begin{miditest}
% \begin{upotreba}{4}
% #\naslovInt#
% #\izlaz{Unesite izraz:}\ulaz{-3 m -1 m 100}#
% #\izlaz{Vrednost izraza: 1}#
% \end{upotreba}
% \end{miditest}
% 
% \linkresenje{KT_NG_31}
% \end{Exercise}
% \ifresenja
% \begin{Answer}[ref=KT_NG_31]
%  
% Rešenje je analogno rešenju zadatka \ref{KT_NG_25}.
% \end{Answer}
% \fi
% 
% 
% \begin{Exercise}[difficulty=1, label=KT_NG_32] 
% Program učitava jedan karakter i osam realnih brojeva koji predstavljaju 
% koordinate četiri tačke: $A(x_1, y_1), B(x_2, y_2), C(x_3, y_3), D(x_4, y_4)$. Na osnovu unetog karaktera 
% ispisuje se odgovarajuća poruka na standardni izlaz:
% \begin{itemize}
% \item ukoliko je uneti karakter $k$ - proverava da li su date tačke temena pravougaonika čije su stranice paralelne koordinatnim osama i 
%     u slučaju da jesu, ispisuje vrednost obima datog pravougaonika. Možemo podrazumevati da će korisnik koordinate tačaka 
%     unosi redom $A,B,C,D$, pri čemu $ABCD$ opisuje pravougaonik čije su stranice $AB,BC,CD,DA$, a dijagonale $AC$ i $BD$. 
%     Na primer, tačke $(1,1),(2,1),(2,2),(1,2)$ čine pravougaonik čije su stranice paralelne koordinatnim osama i čiji je obim 4
%     a tačke $(1,1),(2,2),(3,3),(4,4)$ ne čine pravougaonik. 
% \item ukoliko je uneti karakter $h$ - proverava da li su unete tačke kolinearne i ukoliko jesu, ispisuje jednačinu prave kojoj pripadaju. 
%     Na primer, tačke $(1,2),(2,3),(3,4),(4,5)$ su kolinearne i pripadaju pravoj $y=x+1$, 
%     tačke $(1,1),(1,2),(1,3),(1,4)$ su kolinearne i pripadaju pravoj $x=1$,
%     a tačke $(1,1),(2,1),(2,2),(1,2)$ nisu kolinearne.
% \item ukoliko je uneti karakter $j$ - Kramerovim pravilom proverava da li je sistem jednačina
% $x_1 * p + x_2 * q = x_4 - x_3,y_1 * p + y_2 * q = y_4 - y_3$
%     određen, neodređen ili nema rešenja, i u slučaju da je određen ispisuje rešenja.
% \end{itemize} 
% \linkresenje{KT_NG_32}
% \end{Exercise}
% \ifresenja
%  \begin{Answer}[ref=KT_NG_32]
% \includecode{resenja/2_KontrolaToka/1.2_NaredbeGrananja/2_1_32.c}
% \end{Answer}
% \fi


\begin{Exercise}[difficulty=1, label=KT_NG_33]
Polje šahovske table se definiše parom celih brojeva $(x, y)$, $1 \leq x,y \leq 8$,
gde je $x$ redni broj reda, a $y$ redni broj kolone. Napisati program koji za unete parove
$(k, l)$ i $(m, n)$ proverava
\begin{description}
\item[(a)] da li su polja $(k, l)$ i $(m, n)$ iste boje
\item[(b)] da li kraljica sa $(k, l)$ ugrožava polje $(m, n)$
\item[(c)] da li konj sa $(k, l)$ ugrožava polje $(m, n)$
\end{description}
Pretpostaviti da je polje $(1, 1)$ crno i da predstavlja donji levi ugao šahovske table. 
U slučaju neispravnog unosa, ispisati odgovarajuću poruku o grešci. 

\begin{miditest}
\begin{upotreba}{1}
#\naslovInt#
#\izlaz{Unesite (k,l):}\ulaz{1 1}#
#\izlaz{Unesite (m,n):}\ulaz{2 2}#
#\izlaz{Polja su iste boje.}#
#\izlaz{Kraljica sa (1,1) ugrozava (2,2).}#
#\izlaz{Konj sa (1,1) ne ugrozava (2,2).}#
\end{upotreba}
\end{miditest}
\begin{miditest}
\begin{upotreba}{2}
#\naslovInt#
#\izlaz{Unesite (k,l):}\ulaz{1 1}#
#\izlaz{Unesite (m,n):}\ulaz{3 2}#
#\izlaz{Polja su razlicite boje.}#
#\izlaz{Kraljica sa (1,1) ne ugrozava (3,2).}#
#\izlaz{Konj sa (1,1) ugrozava (3,2).}#
\end{upotreba}
\end{miditest}

\begin{miditest}
\begin{upotreba}{3}
#\naslovInt#
#\izlaz{Unesite (k,l):}\ulaz{5 4}#
#\izlaz{Unesite (m,n):}\ulaz{3 3}#
#\izlaz{Polja su razlicite boje.}#
#\izlaz{Kraljica sa (5,4) ne ugrozava (3,3).}#
#\izlaz{Konj sa (5,4) ugrozava (3,3).}#
\end{upotreba}
\end{miditest}
\begin{miditest}
\begin{upotreba}{4}
#\naslovInt#
#\izlaz{Unesite (k,l):}\ulaz{0 1}#
#\izlaz{Unesite (m,n):}\ulaz{3 9}#
#\izlaz{Greska: neispravna pozicija.}#
\end{upotreba}
\end{miditest}

\linkresenje{KT_NG_33}
\end{Exercise}
\ifresenja
 \begin{Answer}[ref=KT_NG_33]
\includecode{resenja/2_KontrolaToka/1.2_NaredbeGrananja/2_1_33.c}
\end{Answer}
\fi


\ifresenja
\section{Rešenja}
\shipoutAnswer
\fi
