\chapter{Kontrola toka}


\section{Naredbe grananja}

%\subsection{Samo if}

\begin{Exercise}[label=v1.2_05] 
Napisati program koji za dva uneta cela broja  ispisuje njihov minimum.

\begin{miditest}
\begin{upotreba}{1}
#\naslovInt#
#\izlaz{Unesite dva cela broja:}\ulaz{5 18}#
#\izlaz{Minimum je 5.}#
\end{upotreba}
\end{miditest}
\begin{miditest}
\begin{upotreba}{2}
#\naslovInt#
#\izlaz{Unesite dva cela broja:}\ulaz{43 -16}#
#\izlaz{Minimum je -16.}#
\end{upotreba}
\end{miditest}

\linkresenje{v1.2_05}
\end{Exercise}
\ifresenja
 \begin{Answer}[ref=v1.2_05]
\includecode{resenja/1_KontrolaToka/1.2_NaredbeGrananja/2_01.c}
\end{Answer}
\fi

\begin{Exercise}[label=v1.2_05a] 
Napisati program koji za dva uneta cela broja ispisuje njihov maksimum. 

\begin{miditest}
\begin{upotreba}{1}
#\naslovInt#
#\izlaz{Unesite dva cela broja:}\ulaz{141 67}#
#\izlaz{Maksimum je 141.}#
\end{upotreba}
\end{miditest}
\begin{miditest}
\begin{upotreba}{2}
#\naslovInt#
#\izlaz{Unesite dva cela broja:}\ulaz{-893 -54}#
#\izlaz{Maksimum je -54.}#
\end{upotreba}
\end{miditest}

%\linkresenje{v1.2_05a}
\end{Exercise}
\ifresenja
 \begin{Answer}[ref=v1.2_05a]	
Rešenje je analogno rešenju broj \ref{v1.2_05}.
\end{Answer}
\fi


\begin{Exercise}[label=v1.2_10] 
Napisati program koji za uneti realan broj ispisuje njegovu apsolutnu vrednost zaokruženu na dve decimale.

\begin{miditest}
\begin{upotreba}{1}
#\naslovInt#
#\izlaz{Unesite jedan realan broj:}\ulaz{7.42}#
#\izlaz{Njegova apsolutna vrednost je: 7.42}#
\end{upotreba}
\end{miditest}
\begin{miditest}
\begin{upotreba}{2}
#\naslovInt#
#\izlaz{Unesite jedan realan broj:}\ulaz{-562.428}#
#\izlaz{Njegova apsolutna vrednost je: 562.43}#
\end{upotreba}
\end{miditest}

\begin{miditest}
\begin{upotreba}{3}
#\naslovInt#
#\izlaz{Unesite jedan realan broj:}\ulaz{0}#
#\izlaz{Njegova apsolutna vrednost je: 0.00}#
\end{upotreba}
\end{miditest}
\begin{miditest}
\begin{upotreba}{4}
#\naslovInt#
#\izlaz{Unesite jedan realan broj:}\ulaz{52}#
#\izlaz{Njegova apsolutna vrednost je: 52.00}#
\end{upotreba}
\end{miditest}

\linkresenje{v1.2_10}
\end{Exercise}
\ifresenja
 \begin{Answer}[ref=v1.2_10]
\includecode{resenja/1_KontrolaToka/1.2_NaredbeGrananja/2_03.c}
\end{Answer}
\fi

\begin{Exercise}[label=v1.2_02] 
Napisati program koji za uneti ceo broj ispisuje njegovu recipročnu vrednost zaokruženu na četiri decimale.

\begin{miditest}
\begin{upotreba}{1}
#\naslovInt#
#\izlaz{Unesite jedan ceo broj:}\ulaz{22}#
#\izlaz{Reciprocna vrednost unetog broja: 0.0455.}#
\end{upotreba}
\end{miditest}
\begin{miditest}
\begin{upotreba}{2}
#\naslovInt#
#\izlaz{Unesite jedan ceo broj:}\ulaz{-9}#
#\izlaz{Reciprocna vrednost unetog broja: -0.1111.}#
\end{upotreba}
\end{miditest}

\begin{miditest}
\begin{upotreba}{3}
#\naslovInt#
#\izlaz{Unesite jedan ceo broj:}\ulaz{0}#
#\izlaz{Nedozvoljeno deljenje nulom.}#
\end{upotreba}
\end{miditest}
\begin{miditest}
\begin{upotreba}{4}
#\naslovInt#
#\izlaz{Unesite jedan ceo broj:}\ulaz{57298}#
#\izlaz{Reciprocna vrednost unetog broja: 0.0000.}#
\end{upotreba}
\end{miditest}

\linkresenje{v1.2_02}
\end{Exercise}
\ifresenja
 \begin{Answer}[ref=v1.2_02]
\includecode{resenja/1_KontrolaToka/1.2_NaredbeGrananja/2_04.c}
\end{Answer}
\fi

\begin{Exercise}[label=v1.2_09] 
Napisati program koji učitava tri cela broja i ispisuje zbir pozitivnih.

\begin{miditest}
\begin{upotreba}{1}
#\naslovInt#
#\izlaz{Unesite tri cela broja:}\ulaz{1 3 -6}#
#\izlaz{Suma unetih pozitivnih brojeva: 4}#
\end{upotreba}
\end{miditest}
\begin{miditest}
\begin{upotreba}{2}
#\naslovInt#
#\izlaz{Unesite tri cela broja:}\ulaz{-15 81 0}#
#\izlaz{Suma unetih pozitivnih brojeva: 81}#
\end{upotreba}
\end{miditest}

\begin{miditest}
\begin{upotreba}{3}
#\naslovInt#
#\izlaz{Unesite tri cela broja:}\ulaz{-719 -48 -123}#
#\izlaz{Suma unetih pozitivnih brojeva: 0}#
\end{upotreba}
\end{miditest}
\begin{miditest}
\begin{upotreba}{4}
#\naslovInt#
#\izlaz{Unesite tri cela broja:}\ulaz{16 2 576}#
#\izlaz{Suma unetih pozitivnih brojeva: 594}#
\end{upotreba}
\end{miditest}

\linkresenje{v1.2_09}
\end{Exercise}
\ifresenja
 \begin{Answer}[ref=v1.2_09]
\includecode{resenja/1_KontrolaToka/1.2_NaredbeGrananja/2_05.c}
\end{Answer}
\fi






\begin{Exercise}[label=v1.2_06] 
U prodavnici je organizovana akcija da svaki kupac dobije najjeftiniji od tri artikla za jedan dinar. Napisati program koji za unete cene tri artikla izračunava ukupnu cenu, kao i koliko dinara se uštedi zahvaljujući popustu. \napomena{Pretpostaviti da su cene artikala pozitivni celi brojevi.}

\begin{miditest}
\begin{upotreba}{1}
#\naslovInt#
#\izlaz{Unesite cene tri artikla:}\ulaz{35 125 97}#
#\izlaz{Cena sa popustom: 223}#
#\izlaz{Usteda: 34}#
\end{upotreba}
\end{miditest}
\begin{miditest}
\begin{upotreba}{2}
#\naslovInt#
#\izlaz{Unesite cene tri artikla:}\ulaz{1034 15 25}#
#\izlaz{Cena sa popustom: 1060}#
#\izlaz{Usteda: 14}#
\end{upotreba}
\end{miditest}

\begin{miditest}
\begin{upotreba}{3}
#\naslovInt#
#\izlaz{Unesite cene tri artikla:}\ulaz{500 500 500}#
#\izlaz{Cena sa popustom: 1001}#
#\izlaz{Usteda: 499}#
\end{upotreba}
\end{miditest}
\begin{miditest}
\begin{upotreba}{4}
#\naslovInt#
#\izlaz{Unesite cene tri artikla:}\ulaz{247 133 126}#
#\izlaz{Cena sa popustom: 381}#
#\izlaz{Usteda: 125}#
\end{upotreba}
\end{miditest}

\linkresenje{v1.2_06}
\end{Exercise}
\ifresenja
 \begin{Answer}[ref=v1.2_06]
\includecode{resenja/1_KontrolaToka/1.2_NaredbeGrananja/2_06.c}
\end{Answer}
\fi


\begin{Exercise}[label=p1.2_01] 
Napisati program koji za uneti četvorocifreni broj ispisuje njegovu najveću cifru. 

\begin{miditest}
\begin{upotreba}{1}
#\naslovInt#
#\izlaz{Unesite broj:}\ulaz{6835}#
#\izlaz{Najveca cifra je: 8}#
\end{upotreba}
\end{miditest}
\begin{miditest}
\begin{upotreba}{2}
#\naslovInt#
#\izlaz{Unesite broj:}\ulaz{238}#
#\izlaz{Greska: Niste uneli cetvorocifren broj!}#
\end{upotreba}
\end{miditest}

\begin{miditest}
\begin{upotreba}{3}
#\naslovInt#
#\izlaz{Unesite broj:}\ulaz{7777}#
#\izlaz{Najveca cifra je: 7}#
\end{upotreba}
\end{miditest}
\begin{miditest}
\begin{upotreba}{4}
#\naslovInt#
#\izlaz{Unesite broj:}\ulaz{-2002}#
#\izlaz{Najveca cifra je: 2}#
\end{upotreba}
\end{miditest}

\linkresenje{p1.2_01}
\end{Exercise}
\ifresenja
 \begin{Answer}[ref=p1.2_01]
\includecode{resenja/1_KontrolaToka/1.2_NaredbeGrananja/2_07.c}
\end{Answer}
\fi

\begin{Exercise}[label=v1.2_01] 
Napisati program koji za uneto vreme (broj sati iz intervala $[0,24)$ i broj minuta iz intervala $[0,60)$) ispisuje koliko je sati i minuta ostalo do ponoći. 

\begin{miditest}
\begin{upotreba}{1}
#\naslovInt#
#\izlaz{Unesite vreme (broj sati u itervalu [0,24),}#
#\izlaz{broj minuta u intervalu [0,60)):}\ulaz{18 19}#
#\izlaz{Do ponoci je ostalo 5 sati i 41 minuta.}
\end{upotreba}
\end{miditest}
\begin{miditest}
\begin{upotreba}{2}
#\naslovInt#
#\izlaz{Unesite vreme (broj sati u itervalu [0,24),}#
#\izlaz{broj minuta u intervalu [0,60)):}\ulaz{23 7}#
#\izlaz{Do ponoci je ostalo 0 sati i 53 minuta.}
\end{upotreba}
\end{miditest}

\begin{miditest}
\begin{upotreba}{3}
#\naslovInt#
#\izlaz{Unesite vreme (broj sati u itervalu [0,24),}#
#\izlaz{broj minuta u intervalu [0,60)):}\ulaz{24 20}#
#\izlaz{Neispravan unos.}
\end{upotreba}
\end{miditest}
\begin{miditest}
\begin{upotreba}{4}
#\naslovInt#
#\izlaz{Unesite vreme (broj sati u itervalu [0,24),}#
#\izlaz{broj minuta u intervalu [0,60)):}\ulaz{14 0}#
#\izlaz{Do ponoci je ostalo 10 sati i 0 minuta.}
\end{upotreba}
\end{miditest}

\linkresenje{v1.2_01}
\end{Exercise}
\ifresenja
 \begin{Answer}[ref=v1.2_01]
%TODO dodati uresenje komentar na slozeni if uslov
% dodato
\includecode{resenja/1_KontrolaToka/1.2_NaredbeGrananja/2_08.c}
\end{Answer}
\fi







\begin{Exercise}[label=v1.2_08] 
%\komentarJ{Zadatak je preformulisan. Bila je ideja da se nekako pokaze c-'0' ali mislim da u ovom zadatku to nije zgodno. Svakako je vazno da se ovaj koncept vidi u nekom zadatku kasnije. }
Napisati program koji za učitani karakter ispisuje uneti karakter i njegov ASCII kod. Ukoliko je uneti karakter malo (veliko) slovo, ispisati i odgovarajuće veliko (malo) slovo i njegov ASCII kod.

\begin{miditest}
\begin{upotreba}{1}
#\naslovInt#
#\izlaz{Unesite karakter:}\ulaz{0}#
#\izlaz{Uneti karakter: 0, njegov ASCII kod: 48}#
\end{upotreba}
\end{miditest}
\begin{miditest}
\begin{upotreba}{2}
#\naslovInt#
#\izlaz{Unesite karakter:}\ulaz{?}#
#\izlaz{Uneti karakter: ?, njegov ASCII kod: 63}#
\end{upotreba}
\end{miditest}

\begin{maxitest}
\begin{upotreba}{3}
#\naslovInt#
#\izlaz{Unesite karakter:}\ulaz{A}#
#\izlaz{Uneti karakter: A, njegov ASCII kod: 65}#
#\izlaz{odgovarajuce malo slovo: a, njegov ASCII kod: 97}#
\end{upotreba}
\end{maxitest}

\begin{maxitest}
\begin{upotreba}{4}
#\naslovInt#
#\izlaz{Unesite karakter:}\ulaz{v}#
#\izlaz{Uneti karakter: v, njegov ASCII kod: 118}#
#\izlaz{odgovarajuce veliko slovo: V, njegov ASCII kod: 86}#
\end{upotreba}
\end{maxitest}

\linkresenje{v1.2_08}
\end{Exercise}
\ifresenja
 \begin{Answer}[ref=v1.2_08]
\includecode{resenja/1_KontrolaToka/1.2_NaredbeGrananja/2_09.c}
\end{Answer}
\fi


\begin{Exercise}[label=p1.2_07] 
Napisati program koji za unetih pet karaktera ispisuje koliko je među njima malih slova.

\begin{miditest}
\begin{upotreba}{1}
#\naslovInt#
#\izlaz{Unesite karaktere:}\ulaz{A u E f h}#
#\izlaz{Broj malih slova: 3}#
\end{upotreba}
\end{miditest}
\begin{miditest}
\begin{upotreba}{2}
#\naslovInt#
#\izlaz{Unesite karaktere:}\ulaz{k L M 9 o}#
#\izlaz{Broj malih slova: 2}#
\end{upotreba}
\end{miditest}

\linkresenje{p1.2_07}
\end{Exercise}
\ifresenja
 \begin{Answer}[ref=p1.2_07]
\includecode{resenja/1_KontrolaToka/1.2_NaredbeGrananja/2_10.c}
\end{Answer}
\fi

%\komentarJ{Naredna 3 zadatka sa karakterima su sa i smera. Meni se nijedan ne dopada a nisu mnogo ni ilustrativni. Prvi je besmislen. Drugi i treci ucitava karaktere pomocu getchar sto mi ne radimo pre petlji plus koriste tolower i isgigit sto takodje ne spominjemo ovako rano. }
%\komentarM{Prvi sam izbrisala, dva sam ostavila zbog Danijele.}

\begin{Exercise}[label=p1.6_] 
Program učitava pet karaktera. Napisati koliko
se puta pojavilo veliko ili malo slovo \verb|a|. 

\begin{miditest}
\begin{upotreba}{1}
#\naslovInt#
#\izlaz{Unesite karaktere:}\ulaz{aBcAe}#
#\izlaz{2}#
\end{upotreba}
\end{miditest}
\begin{miditest}
\begin{upotreba}{2}
#\naslovInt#
#\izlaz{Unesite karaktere:}\ulaz{aa4A\_}#
#\izlaz{3}#
\end{upotreba}
\end{miditest}

\begin{miditest}
\begin{upotreba}{3}
#\naslovInt#
#\izlaz{Unesite karaktere:}\ulaz{aAaAa}#
#\izlaz{5}#
\end{upotreba}
\end{miditest}
\begin{miditest}
\begin{upotreba}{4}
#\naslovInt#
#\izlaz{Unesite karaktere:}\ulaz{B6(vV}#
#\izlaz{0}#
\end{upotreba}
\end{miditest}
\linkresenje{p1.6_}
\end{Exercise}
\ifresenja
 \begin{Answer}[ref=p1.6_]
%TODO Dodati komentare u resenja
%dodato
\includecode{resenja/1_KontrolaToka/1.2_NaredbeGrananja/2_11.c}
\end{Answer}
\fi


\begin{Exercise}[label=p1.7_] 
Program učitava pet karaktera. Ispisati koliko
puta su se pojavile cifre. 

\begin{miditest}
\begin{upotreba}{1}
#\naslovInt#
#\izlaz{Unesite karaktere:}\ulaz{A1cA3}#
#\izlaz{2}#
\end{upotreba}
\end{miditest}
\begin{miditest}
\begin{upotreba}{2}
#\naslovInt#
#\izlaz{Unesite karaktere:}\ulaz{2a45\_}#
#\izlaz{2}#
\end{upotreba}
\end{miditest}

\begin{miditest}
\begin{upotreba}{3}
#\naslovInt#
#\izlaz{Unesite karaktere:}\ulaz{43986}#
#\izlaz{5}#
\end{upotreba}
\end{miditest}
\begin{miditest}
\begin{upotreba}{4}
#\naslovInt#
#\izlaz{Unesite karaktere:}\ulaz{B6(vV}#
#\izlaz{1}#
\end{upotreba}
\end{miditest}
\linkresenje{p1.7_}
\end{Exercise}
\ifresenja
 \begin{Answer}[ref=p1.7_]
%TODO Dodati komentare u resenja
% dodato
\includecode{resenja/1_KontrolaToka/1.2_NaredbeGrananja/2_12.c}
\end{Answer}
\fi



%\subsection{If-else se jednom pojavljuje}


\begin{Exercise}[label=v1.2_04] 
 Napisati program koji za unetu godinu ispisuje da li je prestupna.
 
\begin{miditest}
\begin{upotreba}{1}
#\naslovInt#
#\izlaz{Unesite godinu:}\ulaz{2016}#
#\izlaz{Godina je prestupna.}#
\end{upotreba}
\end{miditest}
\begin{miditest}
\begin{upotreba}{2}
#\naslovInt#
#\izlaz{Unesite godinu:}\ulaz{1997}#
#\izlaz{Godina nije prestupna.}#
\end{upotreba}
\end{miditest}

\begin{miditest}
\begin{upotreba}{3}
#\naslovInt#
#\izlaz{Unesite godinu:}\ulaz{2000}#
#\izlaz{Godina je prestupna.}#
\end{upotreba}
\end{miditest}
\begin{miditest}
\begin{upotreba}{4}
#\naslovInt#
#\izlaz{Unesite godinu:}\ulaz{1900}#
#\izlaz{Godina nije prestupna.}#
\end{upotreba}
\end{miditest}
\linkresenje{v1.2_04}
\end{Exercise}
\ifresenja
 \begin{Answer}[ref=v1.2_04]
\includecode{resenja/1_KontrolaToka/1.2_NaredbeGrananja/2_13.c}
\end{Answer}
\fi


\begin{Exercise}[label=p1.2_02] 
Broj je Armstrongov ako je jednak zbiru kubova svojih cifara. Napisati program koji za dati trocifren broj proverava da li je Armstrongov.

\begin{miditest}
\begin{upotreba}{1}
#\naslovInt#
#\izlaz{Unesite broj:}\ulaz{153}#
#\izlaz{Broj je Amstrongov.}#
\end{upotreba}
\end{miditest}
\begin{miditest}
\begin{upotreba}{2}
#\naslovInt#
#\izlaz{Unesite broj:}\ulaz{111}#
#\izlaz{Broj nije Amstrongov.}#
\end{upotreba}
\end{miditest}

\begin{miditest}
\begin{upotreba}{3}
#\naslovInt#
#\izlaz{Unesite broj:}\ulaz{84}#
#\izlaz{Greska: Niste uneli trocifren broj!}#
\end{upotreba}
\end{miditest}
\begin{miditest}
\begin{upotreba}{4}
#\naslovInt#
#\izlaz{Unesite broj:}\ulaz{371}#
#\izlaz{Broj je Amstrongov.}#
\end{upotreba}
\end{miditest}

\linkresenje{p1.2_02}
\end{Exercise}
\ifresenja
 \begin{Answer}[ref=p1.2_02]
\includecode{resenja/1_KontrolaToka/1.2_NaredbeGrananja/2_14.c}
\end{Answer}
\fi

\begin{Exercise}[label=p1.2_04] 
Napisati program koji ispisuje proizvod parnih cifara unetog četvorocifrenog broja. 

\begin{miditest}
\begin{upotreba}{1}
#\naslovInt#
#\izlaz{Unesite cetvorocifreni broj:}\ulaz{8123}#
#\izlaz{Proizvod parnih cifara: 16}#
\end{upotreba}
\end{miditest}
\begin{miditest}
\begin{upotreba}{2}
#\naslovInt#
#\izlaz{Unesite cetvorocifreni broj:}\ulaz{3579}#
#\izlaz{Nema parnih cifara.}#
\end{upotreba}
\end{miditest}

\begin{miditest}
\begin{upotreba}{3}
#\naslovInt#
#\izlaz{Unesite cetvorocifreni broj:}\ulaz{-1234}#
#\izlaz{Proizvod parnih cifara: 8}#
\end{upotreba}
\end{miditest}
\begin{miditest}
\begin{upotreba}{4}
#\naslovInt#
#\izlaz{Unesite broj:}\ulaz{288}#
#\izlaz{Broj nije cetvorocifren!}#
\end{upotreba}
\end{miditest}


\linkresenje{p1.2_04}
\end{Exercise}
\ifresenja
 \begin{Answer}[ref=p1.2_04]
\includecode{resenja/1_KontrolaToka/1.2_NaredbeGrananja/2_15.c}
\end{Answer}
\fi






\begin{Exercise}[label=p1.2_08] 
 Napisati program koji učitava četvorocifreni broj i ispisuje broj koji se dobija kada se unetom broju razmene najmanja i najveća cifra. \napomena{U slučaju da se najmanja ili najveća cifra pojavljuju na više pozicija, uzeti prvo pojavljivanje.}

\begin{miditest}
\begin{upotreba}{1}
#\naslovInt#
#\izlaz{Unesite broj:}\ulaz{2863}#
#\izlaz{8263}#
\end{upotreba}
\end{miditest}
\begin{miditest}
\begin{upotreba}{2}
#\naslovInt#
#\izlaz{Unesite broj:}\ulaz{247}#
#\izlaz{Broj nije cetvorocifren!}#
\end{upotreba}
\end{miditest}

\begin{miditest}
\begin{upotreba}{3}
#\naslovInt#
#\izlaz{Unesite broj:}\ulaz{1192}#
#\izlaz{9112}#
\end{upotreba}
\end{miditest}
\begin{miditest}
\begin{upotreba}{4}
#\naslovInt#
#\izlaz{Unesite broj:}\ulaz{-4239}#
#\izlaz{-4932}#
\end{upotreba}
\end{miditest}

\linkresenje{p1.2_08}
\end{Exercise}
\ifresenja
 \begin{Answer}[ref=p1.2_08]
\includecode{resenja/1_KontrolaToka/1.2_NaredbeGrananja/2_16.c}
\end{Answer}
\fi

\begin{Exercise}[label=p1_19]
Napisati program koji ispituje da li se tačke $A(x_1, y_1)$ i $B(x_2, y_2)$ nalaze u istom kvadrantu i ispisuje odgovor
\verb|DA| ili \verb|NE|. 
%\komentarJ{Nemamo resenje za ovaj zadatak.}
%TODO Traziti Danijeli!

\linkresenje{p1_19}
\end{Exercise}
%\ifresenja
% \begin{Answer}[ref=p1_19]
%\includecode{resenja/1_UvodniZadaci/1_01.c}
%\end{Answer}
%\fi


\begin{Exercise}[label=p1_20]
Napisati program koji ispituje da li se tačke $A(x_1, y_1)$, $B(x_2, y_2)$ i $C(x_3, y_3)$ nalaze na istoj pravoj i
ispisuje odgovor \verb|DA| ili \verb|NE|. \\
%\komentarJ{Nemamo resenje za ovaj zadatak.}
%TODO Traziti Danijeli!

%\linkresenje{p1_20}
\end{Exercise}
%\ifresenja
% \begin{Answer}[ref=p1_20]
%\includecode{resenja/1_UvodniZadaci/1_01.c}
%\end{Answer}
%\fi


\begin{Exercise}[label=p1.2_11] 
 Napisati program za rad sa intervalima. Za dva intervala realne prave $[a1, b1]$ i
$[a2, b2]$, program treba da odredi:
\begin{itemize}
\item [a)] dužinu zajedničkog dela ta dva intervala
\item [b)] najveći interval sadržan u datim intervalima (presek), a ako on ne postoji dati
odgovarajuću poruku.
\item [c)] dužinu realne prave koju pokrivaju ta dva intervala
\item [d)] najmanji interval koji sadrži date intervale.
\end{itemize}

\begin{miditest}
\begin{upotreba}{1}
#\naslovInt#
#\izlaz{Unesite redom a1, b1, a2 i b2:}\ulaz{2 9 4 11}#
#\izlaz{Duzina zajednickog dela: 5}#
#\izlaz{Presek intervala: [4,9]}#
#\izlaz{Zajednicka duzina intervala: 9}#
#\izlaz{Najmanji interval: [2, 11]}#
\end{upotreba}
\end{miditest}
\begin{miditest}
\begin{upotreba}{2}
#\naslovInt#
#\izlaz{Unesite redom a1, b1, a2 i b2:}\ulaz{1 2 10 13}#
#\izlaz{Duzina zajednickog dela: 0}#
#\izlaz{Presek intervala: prazan}#
#\izlaz{Zajednicka duzina intervala: 4}#
#\izlaz{Najmanji interval: [1, 13]}#
\end{upotreba}
\end{miditest}

%\linkresenje{p1.2_11}
\end{Exercise}
%\ifresenja
% \begin{Answer}[ref=p1.2_11]
%\includecode{resenja/1_KontrolaToka/1.2_NaredbeGrananja/praktikumi5/2_11.c}
%\end{Answer}
%\fi


%\subsection{If-else ugnjezdeno}


\begin{Exercise}[label=v1.2_03] 
Napisati program koji za uneti ceo broj $x$ ispisuje njegov znak, tj da li je broj jednak nuli, manji od nule ili veći od nule.

\begin{miditest}
\begin{upotreba}{1}
#\naslovInt#
#\izlaz{Unesite jedan ceo broj:}\ulaz{17}#
#\izlaz{Broj je veci od nule.}#
\end{upotreba}
\end{miditest}
\begin{miditest}
\begin{upotreba}{2}
#\naslovInt#
#\izlaz{Unesite jedan ceo broj:}\ulaz{0}#
#\izlaz{Broj je jednak nuli.}#
\end{upotreba}
\end{miditest}

\begin{miditest}
\begin{upotreba}{3}
#\naslovInt#
#\izlaz{Unesite jedan ceo broj:}\ulaz{-586}#
#\izlaz{Broj je manji od nule.}#
\end{upotreba}
\end{miditest}
\begin{miditest}
\begin{upotreba}{4}
#\naslovInt#
#\izlaz{Unesite jedan ceo broj:}\ulaz{62}#
#\izlaz{Broj je veci od nule.}#
\end{upotreba}
\end{miditest}

\linkresenje{v1.2_03}
\end{Exercise}
\ifresenja
 \begin{Answer}[ref=v1.2_03]
\includecode{resenja/1_KontrolaToka/1.2_NaredbeGrananja/2_20.c}
\end{Answer}
\fi


\begin{Exercise}[label=v1.2_07] 
Napisati program koji za unete koeficijente kvadratne jednačine ispisuje koliko realnih rešenja jednačina ima i ako ih ima, ispisuje ih zaokružene na dve decimale.

\begin{miditest}
\begin{upotreba}{1}
#\naslovInt#
#\izlaz{Unesite koeficijente A, B i C:}\ulaz{1 3 2}#
#\izlaz{Jednacina ima dva razlicita realna resenja:\\ -1.00 i -2.00}#
\end{upotreba}
\end{miditest}
\begin{miditest}
\begin{upotreba}{2}
#\naslovInt#
#\izlaz{Unesite koeficijente A, B i C:}\ulaz{1 1 1}#
#\izlaz{Jednacina nema resenja.}#
\end{upotreba}
\end{miditest}

\linkresenje{v1.2_07}
\end{Exercise}
\ifresenja
 \begin{Answer}[ref=v1.2_07]
\includecode{resenja/1_KontrolaToka/1.2_NaredbeGrananja/2_21.c}
\end{Answer}
\fi

\begin{Exercise}[label=v1.2_13] 
Napisati program koji za uneti četvorocifreni broj proverava
da li su njegove cifre uređene rastuće, opadajuće ili nisu
uređene i štampa odgovarajuću poruku.

\begin{miditest}
\begin{upotreba}{1}
#\naslovInt#
#\izlaz{Unesite cetvorocifreni broj:}\ulaz{1389}#
#\izlaz{Cifre su uredjene neopadajuce.}#
\end{upotreba}
\end{miditest}
\begin{miditest}
\begin{upotreba}{2}
#\naslovInt#
#\izlaz{Unesite cetvorocifreni broj:}\ulaz{-9622}#
#\izlaz{Cifre su uredjene nerastuce.}#\end{upotreba}
\end{miditest}

\begin{miditest}
\begin{upotreba}{3}
#\naslovInt#
#\izlaz{Unesite cetvorocifreni broj:}\ulaz{6792}#
#\izlaz{Cifre nisu uredjene.}#\end{upotreba}
\end{miditest}
\begin{miditest}
\begin{upotreba}{4}
#\naslovInt#
#\izlaz{Unesite cetvorocifreni broj:}\ulaz{88}#
#\izlaz{Uneti broj nije cetvorocifren.}#\end{upotreba}
\end{miditest}

\linkresenje{v1.2_13}
\end{Exercise}
\ifresenja
 \begin{Answer}[ref=v1.2_13]
\includecode{resenja/1_KontrolaToka/1.2_NaredbeGrananja/2_22.c}
\end{Answer}
\fi


\begin{Exercise}[label=p1.2_06] 
 Napisati program koji učitava karakter i:
 \begin{description}
\item{a)} ako je $c$ malo slovo, ispisuje odgovarajuće veliko
\item{b)} ako je $c$ veliko slovo, ispisuje odgovarajuće malo
\item{c)} ako je $c$ cifra, ispisuje poruku $cifra$
\item{d)} u ostalim slučajevima, ispisuje karakter $c$ između dve zvezdice.
\end{description}

\begin{miditest}
\begin{upotreba}{1}
#\naslovInt#
#\izlaz{Unesite karakter:}\ulaz{K}#
#\izlaz{k}#
\end{upotreba}
\end{miditest}
\begin{miditest}
\begin{upotreba}{2}
#\naslovInt#
#\izlaz{Unesite karakter:}\ulaz{8}#
#\izlaz{cifra}#
\end{upotreba}
\end{miditest}

\begin{miditest}
\begin{upotreba}{3}
#\naslovInt#
#\izlaz{Unesite karakter:}\ulaz{>}#
#\izlaz{*>*}#
\end{upotreba}
\end{miditest}
\begin{miditest}
\begin{upotreba}{4}
#\naslovInt#
#\izlaz{Unesite karakter:}\ulaz{o}#
#\izlaz{O}#
\end{upotreba}
\end{miditest}

\linkresenje{p1.2_06}
\end{Exercise}
\ifresenja
 \begin{Answer}[ref=p1.2_06]
\includecode{resenja/1_KontrolaToka/1.2_NaredbeGrananja/2_23.c}
\end{Answer}
\fi


\begin{Exercise}[label=p1.2_03] 
 U nizu 12345678910111213....9899 ispisani su redom brojevi od 1 do 99. Napisati program koji za uneti  ceo broj $k$ (1 \geq $k$ \geq 189) ispisuje cifru koja se nalazi na $k$-toj poziciji datog niza.
 
\begin{miditest}
\begin{upotreba}{1}
#\naslovInt#
#\izlaz{Unesite k:}\ulaz{13}#
#\izlaz{Na 13-toj poziciji je broj 1.}#
\end{upotreba}
\end{miditest}
\begin{miditest}
\begin{upotreba}{2}
#\naslovInt#
#\izlaz{Unesite k:}\ulaz{105}#
#\izlaz{Na 105-toj poziciji je broj 7.}#
\end{upotreba}
\end{miditest}

\linkresenje{p1.2_03}
\end{Exercise}
\ifresenja
 \begin{Answer}[ref=p1.2_03]
\includecode{resenja/1_KontrolaToka/1.2_NaredbeGrananja/2_24.c}
\end{Answer}
\fi

\begin{Exercise}[label=p1.2_12] 
Data je funkcija $f(x) = 2 \cdot cos(x) - x^3$. Napisati program koji za učitanu vrednost realne promenljive $x$ i vrednost celobrojne promenljive $k$ koje može biti 1, 2 ili 3 izračunava
vrednost funkcije $F(k, x) = f(f(f(...f(x)))$ gde je funkcija $f$ primenjena $k$-puta i ispisuje je zaokruženu na dve decimale.
U slučaju neispravnog ulaza, odštampati odgovarajuću poruku o grešci.
%TODO \komentar{dodati test primer za neispravan ulaz}
%dodato
\begin{miditest}
\begin{upotreba}{1}
#\naslovInt#
#\izlaz{Unesite redom x i k:}\ulaz{2.31 2}#
#\izlaz{F(2.31, 2)=2557.52}#
\end{upotreba}
\end{miditest}
\begin{miditest}
\begin{upotreba}{2}
#\naslovInt#
#\izlaz{Unesite redom x i k:}\ulaz{12 1}#
#\izlaz{F(12, 1)=-1726.31}#
\end{upotreba}
\end{miditest}

\begin{miditest}
\begin{upotreba}{3}
#\naslovInt#
#\izlaz{Unesite redom x i k:}\ulaz{2.31 0}#
#\izlaz{Greska: nedozvoljena vrednost za k}#
\end{upotreba}
\end{miditest}
\begin{miditest}
\begin{upotreba}{4}
#\naslovInt#
#\izlaz{Unesite redom x i k:}\ulaz{1 3}#
#\izlaz{F(1, 3)=-8.74}#
\end{upotreba}
\end{miditest}

\linkresenje{p1.2_12}
\end{Exercise}
\ifresenja
 \begin{Answer}[ref=p1.2_12]
\includecode{resenja/1_KontrolaToka/1.2_NaredbeGrananja/2_25.c}
\end{Answer}
\fi


%\subsection{Switch-case}



\begin{Exercise}[label=p1.2_13] 
 Napisati program koji za uneti redni broj dana u nedelji ispisuje ime odgovarajućeg dana. U slučaju pogrešnog unosa ispisati odgovarajuću poruku. 
 
\begin{miditest}
\begin{upotreba}{1}
#\naslovInt#
#\izlaz{Unesite broj: }\ulaz{4}#
#\izlaz{U pitanju je: cetvrtak}#
\end{upotreba}
\end{miditest}
\begin{miditest}
\begin{upotreba}{2}
#\naslovInt#
#\izlaz{Unesite broj: }\ulaz{7}#
#\izlaz{U pitanju je: nedelja}#
\end{upotreba}
\end{miditest}

\begin{miditest}
\begin{upotreba}{3}
#\naslovInt#
#\izlaz{Unesite broj: }\ulaz{8}#
#\izlaz{Greska: nedozvoljeni unos!}#
\end{upotreba}
\end{miditest}
\begin{miditest}
\begin{upotreba}{4}
#\naslovInt#
#\izlaz{Unesite broj: }\ulaz{2}#
#\izlaz{U pitanju je: utorak}#
\end{upotreba}
\end{miditest}

\linkresenje{p1.2_13}
\end{Exercise}
\ifresenja
 \begin{Answer}[ref=p1.2_13]
\includecode{resenja/1_KontrolaToka/1.2_NaredbeGrananja/2_26.c}
\end{Answer}
\fi

\begin{Exercise}[label=v1.2_11] 
Napisati program koji za uneti karakter ispituje da li je samoglasnik.

\begin{miditest}
\begin{upotreba}{1}
#\naslovInt#
#\izlaz{Unesite jedan karakter:}\ulaz{A}#
#\izlaz{Uneti karakter je samoglasnik.}#
\end{upotreba}
\end{miditest}
\begin{miditest}
\begin{upotreba}{2}
#\naslovInt#
#\izlaz{Unesite jedan karakter:}\ulaz{i}#
#\izlaz{Uneti karakter je samoglasnik.}#
\end{upotreba}
\end{miditest}

\begin{miditest}
\begin{upotreba}{3}
#\naslovInt#
#\izlaz{Unesite jedan karakter:}\ulaz{f}#
#\izlaz{Uneti karakter nije samoglasnik.}#
\end{upotreba}
\end{miditest}
\begin{miditest}
\begin{upotreba}{4}
#\naslovInt#
#\izlaz{Unesite jedan karakter:}\ulaz{4}#
#\izlaz{Uneti karakter nije samoglasnik.}#
\end{upotreba}
\end{miditest}

\linkresenje{v1.2_11}
\end{Exercise}
\ifresenja
 \begin{Answer}[ref=v1.2_11]
\includecode{resenja/1_KontrolaToka/1.2_NaredbeGrananja/2_27.c}
\end{Answer}
\fi

\begin{Exercise}[label=p1.2_14] 
Napisatiti program koji učitava dva cela broja i jedan od karaktera +, -, *, / ili \% i ispisuje vrednost izraza dobijenog primenom date operacije na date argumente. U slučaju pogrešnog unosa ispisati odgovarajuću poruku. 

 \begin{miditest}
\begin{upotreba}{1}
#\naslovInt#
#\izlaz{Unesite operator i dva cela broja:}\ulaz{- 8 11}#
#\izlaz{Rezultat je: -3}#
\end{upotreba}
\end{miditest}
\begin{miditest}
\begin{upotreba}{2}
#\naslovInt#
#\izlaz{Unesite operator i dva cela broja:}\ulaz{/ 14 0}#
#\izlaz{Greska: deljenje nulom nije dozvoljeno!}#
\end{upotreba}
\end{miditest}

\begin{miditest}
\begin{upotreba}{3}
#\naslovInt#
#\izlaz{Unesite operator i dva cela broja:}\ulaz{? 5 7}#
#\izlaz{Greska: nepoznat operator!}#
\end{upotreba}
\end{miditest}
\begin{miditest}
\begin{upotreba}{4}
#\naslovInt#
#\izlaz{Unesite operator i dva cela broja:}\ulaz{/ 19 5}#
#\izlaz{Rezultat je: 3}#
\end{upotreba}
\end{miditest}

\linkresenje{p1.2_14}
\end{Exercise}
\ifresenja
 \begin{Answer}[ref=p1.2_14]
\includecode{resenja/1_KontrolaToka/1.2_NaredbeGrananja/2_28.c}
\end{Answer}
\fi


\begin{Exercise}[label=v1.2_12] 
Napisati program koji za uneti dan i mesec ispisuje godišnje doba kojem pripadaju. \napomena{Podrazumevati da je unos korektan.}

\begin{miditest}
\begin{upotreba}{1}
#\naslovInt#
#\izlaz{Unesite dan i mesec:}\ulaz{14 10}#
#\izlaz{jesen}#
\end{upotreba}
\end{miditest}
\begin{miditest}
\begin{upotreba}{2}
#\naslovInt#
#\izlaz{Unesite dan i mesec:}\ulaz{2 8}#
#\izlaz{leto}#
\end{upotreba}
\end{miditest}

\begin{miditest}
\begin{upotreba}{3}
#\naslovInt#
#\izlaz{Unesite dan i mesec:}\ulaz{27 2}#
#\izlaz{zima}#
\end{upotreba}
\end{miditest}
\begin{miditest}
\begin{upotreba}{4}
#\naslovInt#
#\izlaz{Unesite dan i mesec:}\ulaz{19 5}#
#\izlaz{prolece}#
\end{upotreba}
\end{miditest}

\linkresenje{v1.2_12}
\end{Exercise}
\ifresenja
 \begin{Answer}[ref=v1.2_12]
\includecode{resenja/1_KontrolaToka/1.2_NaredbeGrananja/2_29.c}
\end{Answer}
\fi





\begin{Exercise}[label=v1.2_17] 
Napisati program koji za unetu godinu i mesec ispisuje naziv meseca kao i koliko dana ima u tom mesecu te godine.

\linkresenje{v1.2_17}
\end{Exercise}
\ifresenja
 \begin{Answer}[ref=v1.2_17]
\includecode{resenja/1_KontrolaToka/1.2_NaredbeGrananja/2_30.c}
\end{Answer}
\fi


%\begin{Exercise}[label=v1.2_18] 
% Za uneti datum odreduje ispisuje se naziv godisnjeg doba kome datum pripada.
% \komentarJ{Da li ovaj tekst imamo negde ranije?}
% \komentarJ{Da, v1.2_12. Zadatak je zakomentarisan.}
%\linkresenje{v1.2_18}
%\end{Exercise}
%\ifresenja
% \begin{Answer}[ref=v1.2_18]
%\includecode{resenja/1_KontrolaToka/1.2_NaredbeGrananja/1_23.c}
%\end{Answer}
%\fi



%\item Napisati program koji za uneti pozitivan petocifreni broj $n$ određuje i ispisuje broj njegovih parnih i broj njegovih neparnih cifara. Za analizu cifara koristiti \textit{switch} naredbu.\\
%\begin{miditest}
%\begin{upotreba}{1}
%#\naslovInt#
%#\izlaz{Unesite broj n:}\ulaz{23456}#
%#\izlaz{Broj parnih cifara: 3}#
%#\izlaz{Broj neparnih cifara: 2}#
%\end{upotreba}
%\end{miditest}


\begin{Exercise}[label=p1.2_15] 
 Napisati program koji za uneti datum u formatu \textit{dan.mesec.godina.} proverava da li je korektan.\\
\begin{miditest}
\begin{upotreba}{1}
#\naslovInt#
#\izlaz{Unesite datum:}\ulaz{25.11.1983.}#
#\izlaz{Datum je korektan!}#
\end{upotreba}
\end{miditest}
\begin{miditest}
\begin{upotreba}{2}
#\naslovInt#
#\izlaz{Unesite datum:}\ulaz{1.17.2004.}#
#\izlaz{Datum nije korektan!}#
\end{upotreba}
\end{miditest}

\linkresenje{p1.2_15}
\end{Exercise}
\ifresenja
 \begin{Answer}[ref=p1.2_15]
\includecode{resenja/1_KontrolaToka/1.2_NaredbeGrananja/2_31.c}
\end{Answer}
\fi



\begin{Exercise}[label=p1.2_16] 
 Napisati program koji za korektno unet datum u formatu \textit{dan.mesec.godina.} ispisuje datum prethodnog dana. 
 
\begin{miditest}
\begin{upotreba}{1}
#\naslovInt#
#\izlaz{Unesite datum:}\ulaz{30.4.2008.}#
#\izlaz{Prethodni datum: 29.4.2008.}#
\end{upotreba}
\end{miditest}
\begin{miditest}
\begin{upotreba}{2}
#\naslovInt#
#\izlaz{Unesite datum:}\ulaz{1.12.2005.}#
#\izlaz{Prethodni datum: 30.11.2005.}#
\end{upotreba}
\end{miditest}

\linkresenje{p1.2_16}
\end{Exercise}
\ifresenja
 \begin{Answer}[ref=p1.2_16]
\includecode{resenja/1_KontrolaToka/1.2_NaredbeGrananja/2_32.c}
\end{Answer}
\fi


\begin{Exercise}[label=p1.2_17] 
 Napisati program koji za korektno unet datum u formatu \textit{dan.mesec.godina.} ispisuje datum narednog dana. 
 
\begin{miditest}
\begin{upotreba}{1}
#\naslovInt#
#\izlaz{Unesite datum:}\ulaz{30.4.2008.}#
#\izlaz{Naredni datum: 1.5.2008.}#
\end{upotreba}
\end{miditest}
\begin{miditest}
\begin{upotreba}{2}
#\naslovInt#
#\izlaz{Unesite datum:}\ulaz{1.12.2005.}#
#\izlaz{Naredni datum: 2.12.2005.}#
\end{upotreba}
\end{miditest}
%\linkresenje{p1.2_17}
\end{Exercise}
\ifresenja
 \begin{Answer}[ref=p1.2_17]
Rešenje je analogno rešenju zadatka \ref{p1.2_16}.
%\includecode{resenja/1_KontrolaToka/1.2_NaredbeGrananja/praktikumi5/2_17.c}
\end{Answer}
\fi




\begin{Exercise}[label=p1.9_]
Korisnik unosi tri cela broja: $P$, $Q$ i $R$.
Nakon toga unosi i dva karaktera, $op1$ i $op2$. Ovi karakteri predstavljaju operacije nad unetim brojevima i imaju naredno značenje:
\begin{itemize}
\item karakter \textbf{k} predstavlja  logičku konjukciju
\item karakter \textbf{d} predstavlja  logičku disjunkciju
\item karakter \textbf{m} predstavlja  relaciju manje
\item karakter \textbf{v} predstavlja  relaciju veće
\end{itemize}
Program treba da sračuna vrednost izraza 
\kckod{P op1 Q op2 R} i da ga ispiše.

\begin{miditest}
\begin{upotreba}{1}
#\naslovInt#
#\izlaz{Unesite tri cela broja:}\ulaz{0 1 2}#
#\izlaz{Unesite dva karaktera cela broja:}\ulaz{k m}#
#\izlaz{1}#
\end{upotreba}
\end{miditest}
\begin{miditest}
\begin{upotreba}{2}
#\naslovInt#
#\izlaz{Unesite tri cela broja:}\ulaz{-3 -1 0}#
#\izlaz{Unesite dva karaktera cela broja:}\ulaz{d k}#
#\izlaz{0}#
\end{upotreba}
\end{miditest}

%\linkresenje{p1.9_}
\end{Exercise}
%\ifresenja
% \begin{Answer}[ref=p1.9_]
%\includecode{resenja/1_KontrolaToka/1.2_NaredbeGrananja/1_14.c}
%\end{Answer}
%\fi

\begin{Exercise}[difficulty=1, label=v1.2_14] 
Program učitava jedan karakter i osam realnih brojeva koji predstavljaju 
koordinate četiri tačke: $A(x_1, y_1), B(x_2, y_2), C(x_3, y_3), D(x_4, y_4)$. Na osnovu unetog karaktera 
ispisuje se odgovarajuća poruka na standardni izlaz:
\begin{itemize}
\item ukoliko je uneti karakter $k$ - proverava da li su date tačke temena pravougaonika čije su stranice paralelne koordinatnim osama i 
    u slučaju da jesu, ispisuje vrednost obima datog pravougaonika. Možemo podrazumevati da će korisnik koordinate tačaka 
    unosi redom $A,B,C,D$, pri čemu $ABCD$ opisuje pravougaonik čije su stranice $AB,BC,CD,DA$, a dijagonale $AC$ i $BD$. 
    Na primer, tačke $(1,1),(2,1),(2,2),(1,2)$ čine pravougaonik čije su stranice paralelne koordinatnim osama i čiji je obim 4
    a tačke $(1,1),(2,2),(3,3),(4,4)$ ne čine pravougaonik. 
\item ukoliko je uneti karakter $h$ - proverava da li su unete tačke kolinearne i ukoliko jesu, ispisuje jednačinu prave kojoj pripadaju. 
    Na primer, tačke $(1,2),(2,3),(3,4),(4,5)$ su kolinearne i pripadaju pravoj $y=x+1$, 
    tačke $(1,1),(1,2),(1,3),(1,4)$ su kolinearne i pripadaju pravoj $x=1$,
    a tačke $(1,1),(2,1),(2,2),(1,2)$ nisu kolinearne.
\item ukoliko je uneti karakter $j$ - Kramerovim pravilom proverava da li je sistem jednačina
$x_1 * p + x_2 * q = x_4 - x_3,y_1 * p + y_2 * q = y_4 - y_3$
    određen, neodređen ili nema rešenja, i u slučaju da je određen ispisuje rešenja.
\end{itemize} 
\linkresenje{v1.2_14}
\end{Exercise}
\ifresenja
 \begin{Answer}[ref=v1.2_14]
\includecode{resenja/1_KontrolaToka/1.2_NaredbeGrananja/2_34.c}
\end{Answer}
\fi


\begin{Exercise}[label=p1_21]
Polje šahovske table se definiše parom prirodnih brojeva ne većih od $8$: prvi se odnosi na red, drugi na kolonu. Ako su dati takvi
parovi, napisati program koji proverava: \\
\begin{description}
\item[a)] da li su polja (k, m) i (l, n) iste boje
\item[b)] da li kraljica sa (k, l) ugrožava polje (m, n)
\item[c)] da li konj sa (k, l) ugrožava polje (m, n)
\end{description}
%\komentarJ{Nemamo resenje za ovaj zadatak.}
%TODO Traziti Danijeli

%\linkresenje{p1_21}
\end{Exercise}
%\ifresenja
% \begin{Answer}[ref=p1_21]
%\includecode{resenja/1_UvodniZadaci/1_01.c}
%\end{Answer}
%\fi




%\begin{Exercise}[label=v1.2_15] 
 %Upotreba funkcija isalpha, isdigit, toupper, tolower

 %isalpha( karakter ) - funkcija vraca vrednost razlicitu od 0 ako je karakter slovo (malo ili veliko), inace 0
 %isdigit( karakter ) - funkcija vraca vrednost razlicitu od 0 ako je karakter cifra, inace 0
 %isupper( karakter ) - funkcija vraca vrednost razlicitu od 0 ako je karakter veliko slovo, inace 0
 %islower( karakter ) - funkcija vraca vrednost razlicitu od 0 ako je karakter malo slovo, inace 0
 %toupper( karakter ) - ukoliko je karakter malo slovo, funkcija vraca odgovarajuce veliko slovo,
%                       inace vraca isti karakter
 %tolower( karakter ) - ukoliko je karakter veliko slovo, funkcija vraca odgovarajuce malo slovo,
 %                      inace vraca isti karakter
 %\komentarJ{Ovo je demonstrativan zadatak, ove funkcije se vide vec vise puta ranije, da li da ostaje?}
%\komentarJ{Zadatak je zakomentarisan.}
%\linkresenje{v1.2_15}
%\end{Exercise}
%\ifresenja
% \begin{Answer}[ref=v1.2_15]
%\includecode{resenja/1_KontrolaToka/1.2_NaredbeGrananja/1_20.c}
%\end{Answer}
%\fi



\ifresenja
\section{Rešenja}
\shipoutAnswer
\fi
