\chapter{Kontrola toka}


\section{Naredbe grananja}

\komentar{TODO Iz svih resenja pobrisati formulaciju zadatka.}

\komentar{TODO U resenjima gde imena promenljivih nisu deskriptivna treba dodati komentare prilikom deklaracija cemu sluze odgovarajuca imena promenljivih.}

\komentar{TODO Da li pominjati stadndardni ulaz/izlaz? Negde se pominju, negde ne, deluje mi da to opterecuje zadatke, ali bi u svakom slucaju to rebalo da je konzistentno.}

\begin{Exercise}[label=v1.2_05] 
Napisati program koji za dva cela broja uneta sa standardnog ulaza ispisuje njihov minimum na standardni izlaz.
\linkresenje{v1.2_05}
\end{Exercise}
\begin{Answer}[ref=v1.2_05]
\includecode{resenja/1_KontrolaToka/1.2_NaredbeGrananja/1_05.c}
\end{Answer}

\begin{Exercise}[label=v1.2_05a] 
Napisati program koji za dva cela broja uneta sa standardnog ulaza ispisuje njihov maksimum na standardni izlaz. \komentar{Ovaj zadatak mozda da ide bez resenja?}
\linkresenje{v1.2_05a}
\end{Exercise}
\begin{Answer}[ref=v1.2_05a]	
%\includecode{resenja/1_KontrolaToka/1.2_NaredbeGrananja/1_05a.c}
\end{Answer}


\begin{Exercise}[label=v1.2_04] 
 Napisati program koji za godinu koja se unosi sa standardnog ulaza na standardni izlaz
  ispisuje da li je prestupna.
\linkresenje{v1.2_04}
\end{Exercise}
\begin{Answer}[ref=v1.2_04]
\includecode{resenja/1_KontrolaToka/1.2_NaredbeGrananja/1_04.c}
\end{Answer}

\begin{Exercise}[label=v1.2_02] 
Napisati program koji za uneti ceo broj ispisuje njegovu recipročnu vrednost.
\napomena{Voditi računa da program radi ispravno za sve unete vrednosti.}
\komentar{TODO U resenje dodati komentar na temu implicitne konverzije kod deljenja}
\linkresenje{v1.2_02}
\end{Exercise}
\begin{Answer}[ref=v1.2_02]
\includecode{resenja/1_KontrolaToka/1.2_NaredbeGrananja/1_02.c}
\end{Answer}

\begin{Exercise}[label=v1.2_03] 
Napisati program koji za uneti ceo broj $x$ ispisuje njegov znak, tj da li je broj jednak nuli, manji od nule ili veći od nule.
\linkresenje{v1.2_03}
\end{Exercise}
\begin{Answer}[ref=v1.2_03]
\includecode{resenja/1_KontrolaToka/1.2_NaredbeGrananja/1_03.c}
\end{Answer}



\begin{Exercise}[label=v1.2_01] 
Napisati program koji za uneto vreme (broj sati iz intervala $[0,24)$ i broj minuta iz intervala $[0,60)$) ispisuje koliko je sati i minuta ostalo do ponoći. \komentar{TODO Dodati u rešenje proveru ispravnosti unetog vremena, tj ako neko unese neispravno vreme.}
\linkresenje{v1.2_01}
\end{Exercise}
\begin{Answer}[ref=v1.2_01]
\includecode{resenja/1_KontrolaToka/1.2_NaredbeGrananja/1_01.c}
\end{Answer}



\begin{Exercise}[label=v1.2_06] 
Sa standardnog ulaza se unose cene tri artikla. Ukoliko se najjeftiniji
  artikal dobija za 1 dinar, napisati program koji izračunava ukupnu cenu, kao i koliko
  dinara se uštedi zahvaljujući popustu.  
\linkresenje{v1.2_06}
\end{Exercise}
\begin{Answer}[ref=v1.2_06]
\includecode{resenja/1_KontrolaToka/1.2_NaredbeGrananja/1_06.c}
\end{Answer}

\begin{Exercise}[label=p1.2_10] 
 Sa standardnog ulaza se učitavaju realni koeficijenti $A$ i $B$ linearne jednačine $Ax+B = 0$. Napisati program koji ispisuje rešenja ove jednačine. Ukoliko jednačina nema rešenja ili ukoliko ima više od jednog rešenja ispisati odgovarajuće poruke.\\
\begin{miditest}
\begin{upotreba}{1}
#\naslovInt#
#\izlaz{Unesite koeficijente A i B:}\ulaz{2 -5}#
#\izlaz{x=2.5}#
\end{upotreba}
\end{miditest}
\begin{miditest}
\begin{upotreba}{2}
#\naslovInt#
#\izlaz{Unesite koeficijente A i B:}\ulaz{0 18.5}#
#\izlaz{Jednacina nema resenja.}#
\end{upotreba}
\end{miditest}
\linkresenje{p1.2_10}
\end{Exercise}
\begin{Answer}[ref=p1.2_10]
%\includecode{resenja/1_KontrolaToka/1.2_NaredbeGrananja/praktikumi5/2_10.c}
\end{Answer}

\begin{Exercise}[label=v1.2_07] 
Napisati program koji za koeficijente kvadratne jednačine,
koji se unose sa standardnog ulaza, 
ispisuje na standardni izlaz koliko realnih rešenja jednačina ima i ako ih ima, ispisuje rešenja jednačine
zaokružena na dve decimale.
\linkresenje{v1.2_07}
\end{Exercise}
\begin{Answer}[ref=v1.2_07]
\includecode{resenja/1_KontrolaToka/1.2_NaredbeGrananja/1_07.c}
\end{Answer}

\begin{Exercise}[label=v1.2_09] 
Napisati program koji učitava tri cela broja i ispisuje zbir onih unetih brojeva
koji su pozitivni.
\linkresenje{v1.2_09}
\end{Exercise}
\begin{Answer}[ref=v1.2_09]
\includecode{resenja/1_KontrolaToka/1.2_NaredbeGrananja/1_09.c}
\end{Answer}

\begin{Exercise}[label=v1.2_10] 
Napisati program koji za realan broj unet sa standardnog ulaza
ispisuje njegovu apsolutnu vrednost.
\linkresenje{v1.2_10}
\end{Exercise}
\begin{Answer}[ref=v1.2_10]
\includecode{resenja/1_KontrolaToka/1.2_NaredbeGrananja/1_10.c}
\end{Answer}

\begin{Exercise}[label=v1.2_11] 
Napisati program koji za karakter unet sa standardnog ulaza ispisuje
da li je samoglasnik.
\linkresenje{v1.2_11}
\end{Exercise}
\begin{Answer}[ref=v1.2_11]
\includecode{resenja/1_KontrolaToka/1.2_NaredbeGrananja/1_11.c}
\end{Answer}

\begin{Exercise}[label=v1.2_12] 
Napisati program koji za uneti dan i mesec ispisuje godišnje doba kojem pripadaju. \napomena{Podrazumevati da je unos korektan.} 
\linkresenje{v1.2_12}
\end{Exercise}
\begin{Answer}[ref=v1.2_12]
\includecode{resenja/1_KontrolaToka/1.2_NaredbeGrananja/1_12.c}
\end{Answer}

\begin{Exercise}[label=v1.2_13] 
Napisati program koji za uneti četvorocifreni broj proverava
da li su njegove cifre uređene rastuće, opadajuće ili nisu
uređene i štampa odgovarajuću poruku na standardni
izlaz. Voditi računa o nekorektnim unosima. \komentar{Mislim da bi uvek rebalo da vode
racuna o nekorektnim unosima, osim kada se stavi napomena da se podrazumeva da je unos korektan?
Zato bi ovde ovo izbrisala?}
\linkresenje{v1.2_13}
\end{Exercise}
\begin{Answer}[ref=v1.2_13]
\includecode{resenja/1_KontrolaToka/1.2_NaredbeGrananja/1_13.c}
\end{Answer}

\begin{Exercise}[difficulty=1, label=v1.2_14] 
\komentar{Zadatke sa swich-om bih grupisala na kraj}
Sa standardnog ulaza unose se jedan karakter i 8 realnih brojeva koji predstavljaju 
koordinate četiri tačke: $A(x_1, y_1), B(x_2, y_2), C(x_3, y_3), D(x_4, y_4)$. Na osnovu unetog karaktera 
ispisuje se odgovarajuća poruka na standardni izlaz:
\begin{itemize}
\item ukoliko je uneti karakter $k$ - proverava da li su date tačke temena pravougaonika čije su stranice paralelne koordinatnim osama i 
    u slučaju da jesu, ispisuje vrednost obima datog pravougaonika. Možemo podrazumevati da će korisnik koordinate tačaka 
    unosi redom $A,B,C,D$, pri čemu $ABCD$ opisuje pravougaonik čije su stranice $AB,BC,CD,DA$, a dijagonale $AC$ i $BD$. 
    Na primer, tačke $(1,1),(2,1),(2,2),(1,2)$ čine pravougaonik čije su stranice paralelne koordinatnim osama i čiji je obim 4
    a tačke $(1,1),(2,2),(3,3),(4,4)$ ne čine pravougaonik. 
\item ukoliko je uneti karakter $h$ - proverava da li su unete tačke kolinearne i ukoliko jesu, ispisuje jednačinu prave kojoj pripadaju. 
    Na primer, tačke $(1,2),(2,3),(3,4),(4,5)$ su kolinearne i pripadaju pravoj $y=x+1$, 
    tačke $(1,1),(1,2),(1,3),(1,4)$ su kolinearne i pripadaju pravoj $x=1$,
    a tačke $(1,1),(2,1),(2,2),(1,2)$ nisu kolinearne.
\item ukoliko je uneti karakter $j$ - Kramerovim pravilom proverava da li je sistem jednačina
$x_1 * p + x_2 * q = x_4 - x_3,y_1 * p + y_2 * q = y_4 - y_3$
    određen, neodređen ili nema rešenja, i u slučaju da je određen ispisuje rešenja.
\end{itemize} 
\linkresenje{v1.2_14}
\end{Exercise}
\begin{Answer}[ref=v1.2_14]
\includecode{resenja/1_KontrolaToka/1.2_NaredbeGrananja/1_14.c}
\end{Answer}


\begin{Exercise}[label=p1.2_01] 
Napisati program koji za uneti četvorocifreni ceo broj
 ispisuje njegovu najveću cifru. \\
\begin{miditest}
\begin{upotreba}{1}
#\naslovInt#
#\izlaz{Unesite broj:}\ulaz{6835}#
#\izlaz{Najveca cifra je: 8}#
\end{upotreba}
\end{miditest}
\begin{miditest}
\begin{upotreba}{2}
#\naslovInt#
#\izlaz{Unesite broj:}\ulaz{238}#
#\izlaz{Greska: Niste uneli cetvorocifren broj!}#
\end{upotreba}
\end{miditest}
\linkresenje{p1.2_01}
\end{Exercise}
\begin{Answer}[ref=p1.2_01]
\includecode{resenja/1_KontrolaToka/1.2_NaredbeGrananja/praktikumi5/2_01.c}
\end{Answer}

\begin{Exercise}[label=p1.2_02] 
Broj je Armstrongov ako je jednak zbiru kubova svojih cifara. Napisati program koji za dati trocifren broj proverava da li je Armstrongov.
\\
\begin{miditest}
\begin{upotreba}{1}
#\naslovInt#
#\izlaz{Unesite broj:}\ulaz{153}#
#\izlaz{Broj je Amstrongov.}#
\end{upotreba}
\end{miditest}
\begin{miditest}
\begin{upotreba}{2}
#\naslovInt#
#\izlaz{Unesite broj:}\ulaz{111}#
#\izlaz{Broj nije Amstrongov.}#
\end{upotreba}
\end{miditest}
\begin{miditest}
\begin{upotreba}{3}
#\naslovInt#
#\izlaz{Unesite broj:}\ulaz{84}#
#\izlaz{Greska: Niste uneli trocifren broj!}#
\end{upotreba}
\end{miditest}

\linkresenje{p1.2_02}
\end{Exercise}
\begin{Answer}[ref=p1.2_02]
\includecode{resenja/1_KontrolaToka/1.2_NaredbeGrananja/praktikumi5/2_02.c}
\end{Answer}

\begin{Exercise}[label=p1.2_03] 
 U nizu 12345678910111213....9899 ispisani su redom brojevi od 1 do 99. Napisati program koji za uneti  ceo broj $k$ (1 \geq $k$ \geq 189) ispisuje cifru koja se nalazi na $k$-toj poziciji datog niza.\\
\begin{miditest}
\begin{upotreba}{1}
#\naslovInt#
#\izlaz{Unesite k:}\ulaz{13}#
#\izlaz{Na 13-toj poziciji je broj 1.}#
\end{upotreba}
\end{miditest}
\begin{miditest}
\begin{upotreba}{2}
#\naslovInt#
#\izlaz{Unesite k:}\ulaz{105}#
#\izlaz{Na 105-toj poziciji je broj 7.}#
\end{upotreba}
\end{miditest}

\linkresenje{p1.2_03}
\end{Exercise}
\begin{Answer}[ref=p1.2_03]
\includecode{resenja/1_KontrolaToka/1.2_NaredbeGrananja/praktikumi5/2_03.c}
\end{Answer}

\begin{Exercise}[label=p1.2_04] 
 Sa standardnog ulaza se unosi četvorocifreni pozitivan broj. Napisati program koji ispisuje proizvod parnih cifara datog broja. 
 \komentar{Izmeniti poruku u resenju!}
\begin{miditest}
\begin{upotreba}{1}
#\naslovInt#
#\izlaz{Unesite broj:}\ulaz{8123}#
#\izlaz{Proizvod parnih cifara: 16}#
\end{upotreba}
\end{miditest}
\begin{miditest}
\begin{upotreba}{2}
#\naslovInt#
#\izlaz{Unesite broj:}\ulaz{3579}#
#\izlaz{Proizvod parnih cifara: 0}#
\end{upotreba}
\end{miditest}
\begin{miditest}
\begin{upotreba}{3}
#\naslovInt#
#\izlaz{Unesite broj:}\ulaz{288}#
#\izlaz{Greska, broj nije cetvorocifren!}#
\end{upotreba}
\end{miditest}


\linkresenje{p1.2_04}
\end{Exercise}
\begin{Answer}[ref=p1.2_04]
\includecode{resenja/1_KontrolaToka/1.2_NaredbeGrananja/praktikumi5/2_04.c}
\end{Answer}

\begin{Exercise}[label=p1.2_05] 
 Sa standarnog ulaza se unosi 5 karaktera. Napisati program koji u slučaju da je prvi karakter veliko ili malo slovo $a$  ispisuje unete karaktere obrnutim redosledom, a u suprotnom ništa ne ispisuje. \komentar{Mozda umesto a da bude o, kao skracenica od obrni? Inace, ovaj zadatak je poprilicno besmislen :-)}
 
\begin{miditest}
\begin{upotreba}{1}
#\naslovInt#
#\izlaz{Unesite karaktere:}\ulaz{A u E f h}#
#\izlaz{h f E u A}#
\end{upotreba}
\end{miditest}
\begin{miditest}
\begin{upotreba}{2}
#\naslovInt#
#\izlaz{Unesite karaktere:}\ulaz{k L M 9 o}#
#\izlaz{}#
\end{upotreba}
\end{miditest}

\linkresenje{p1.2_05}
\end{Exercise}
\begin{Answer}[ref=p1.2_05]
\includecode{resenja/1_KontrolaToka/1.2_NaredbeGrananja/praktikumi5/2_05.c}
\end{Answer}

\begin{Exercise}[label=v1.2_08] 
Napisati program koji za karakter koji učitava:
\begin{itemize}
\item{u slučaju da je uneta cifra, ispisuje nju i njen \textsc{ascii} kod} \komentar{Ovo se ne razlikuje od poslednje stavke: dakle ili ovde treba nesto dodati sto ce ga razlikovati od poslednje stavke, npr da se ispise i broj cifre, tj da vide c-'0'}
\item{u slučaju da je uneto malo slovo, ispisuje njega, njegov \textsc{ascii} kod, odgovarajuće veliko slovo i njegov \textsc{ascii} kod}
\item{u slučaju da je uneto veliko slovo, ispisuje njega, njegov \textsc{ascii} kod, odgovarajuće malo slovo i njegov \textsc{ascii} kod}
\item{u ostalim slučajevima, ispisuje uneti karakter i njegov \textsc{ascii} kod} 
\end{itemize}
\linkresenje{v1.2_08}
\end{Exercise}
\begin{Answer}[ref=v1.2_08]
\includecode{resenja/1_KontrolaToka/1.2_NaredbeGrananja/1_08.c}
\end{Answer}


\begin{Exercise}[label=p1.2_06] 
\komentar{Ovaj zadatak je jako slican sa prethodnim, ne znam da li nam trebaju oba resena. Mozda jedan da bude za vezbe, resen, a drugi za praktikume, neresen?}
 Sa standarnog ulaza se unosi karakter $c$. Napisati program koji:
 \begin{description}
\item{a)} ako je $c$ malo slovo, zamenjuje ga odgovarajućim velikim i ispisuje na standardni izlaz
\item{b)} ako je $c$ veliko slovo, zamenjuje ga odgovarajućim malim i ispisuje na standardni izlaz
\item{c)} ako je $c$ cifra, ispisuje poruku \textit{cifra}
\item{d)} u ostalim slučajevima, ispisuje karakter $c$ između dve zvezdice.
\end{description}
\begin{miditest}
\begin{upotreba}{1}
#\naslovInt#
#\izlaz{Unesite karakter:}\ulaz{K}#
#\izlaz{k}#
\end{upotreba}
\end{miditest}
\begin{miditest}
\begin{upotreba}{2}
#\naslovInt#
#\izlaz{Unesite karakter:}\ulaz{8}#
#\izlaz{cifra}#
\end{upotreba}
\end{miditest}
\begin{miditest}
\begin{upotreba}{3}
#\naslovInt#
#\izlaz{Unesite karakter:}\ulaz{>}#
#\izlaz{*>*}#
\end{upotreba}
\end{miditest}

\linkresenje{p1.2_06}
\end{Exercise}
\begin{Answer}[ref=p1.2_06]
\includecode{resenja/1_KontrolaToka/1.2_NaredbeGrananja/praktikumi5/2_06.c}
\end{Answer}

\begin{Exercise}[label=p1.2_07] 
Napisati program koji za unetih 5 karaktera ispisuje koliko je među njima malih slova.\\
\begin{miditest}
\begin{upotreba}{1}
#\naslovInt#
#\izlaz{Unesite karaktere:}\ulaz{A u E f h}#
#\izlaz{Broj malih slova: 3}#
\end{upotreba}
\end{miditest}
\begin{miditest}
\begin{upotreba}{2}
#\naslovInt#
#\izlaz{Unesite karaktere:}\ulaz{k L M 9 o}#
#\izlaz{Broj malih slova: 2}#
\end{upotreba}
\end{miditest}

\linkresenje{p1.2_07}
\end{Exercise}
\begin{Answer}[ref=p1.2_07]
\includecode{resenja/1_KontrolaToka/1.2_NaredbeGrananja/praktikumi5/2_07.c}
\end{Answer}

\begin{Exercise}[label=p1.2_08] 
 Sa standardnog ulaza se unosi četvorocifren ceo broj. Napisati program koji ispisuje broj koji se dobija kada se unetom broju razmene najmanja i najveća cifra. 
 %Ako uneti broj nije četvorocifren ispisati poruku \textit{Greska!}. \\
\komentar{Izmeniti poruku u resenju.}
 
\begin{miditest}
\begin{upotreba}{1}
#\naslovInt#
#\izlaz{Unesite broj:}\ulaz{2863}#
#\izlaz{Novi broj: 8263}#
\end{upotreba}
\end{miditest}
\begin{miditest}
\begin{upotreba}{2}
#\naslovInt#
#\izlaz{Unesite broj:}\ulaz{247}#
#\izlaz{Greska, broj nije cetvorocifren!}#
\end{upotreba}
\end{miditest}

\linkresenje{p1.2_08}
\end{Exercise}
\begin{Answer}[ref=p1.2_08]
%\includecode{resenja/1_KontrolaToka/1.2_NaredbeGrananja/praktikumi5/2_08.c}
\end{Answer}

\begin{Exercise}[label=p1.2_09] 
Spajanjem cifara dva trocifrena broja dobija se šestocifren broj. Na primer, spajanjem brojeva $321$ i $654$ dobija se broj $321654$. Sa standardnog ulaza se unose tri neoznačena trocifrena broja.  Napisati program koji spaja dva od ta tri trocifrena broja tako da se dobije naveći mogući šestocifren broj. Dobijeni šestocifreni broj ispisati na izlazu. Ako neki od unetih brojeva nije trocifren, smatrati da ulaz nije ispravn.
\komentar{Izmeniti poruku o gresci u resenju}

\begin{miditest}
\begin{upotreba}{1}
#\naslovInt#
#\izlaz{Unesite brojeve:}\ulaz{185 247 311}#
#\izlaz{Trazeni broj je: 311247}#
\end{upotreba}
\end{miditest}
\begin{miditest}
\begin{upotreba}{2}
#\naslovInt#
#\izlaz{Unesite brojeve:}\ulaz{865 11 298}#
#\izlaz{Greska, ulaz nije ispravan!}#
\end{upotreba}
\end{miditest}

\linkresenje{p1.2_09}
\end{Exercise}
\begin{Answer}[ref=p1.2_09]
%\includecode{resenja/1_KontrolaToka/1.2_NaredbeGrananja/praktikumi5/2_09.c}
\end{Answer}


\begin{Exercise}[label=p1.2_11] 
 Napisati program za rad sa intervalima. Za dva intervala realne prave $[a1, b1]$ i
$[a2, b2]$, program treba da odredi:
\begin{itemize}
\item [a)] dužinu zajedničkog dela ta dva intervala
\item [b)] najveći interval sadržan u datim intervalima (presek),a ako on ne postoji dati
odgovarajuću poruku. (?! zar ovo nije isto sto i a?) \komentar{pod a je duzina a ovde je interval, pogledati test primer}
\item [c)] dužinu realne prave koju pokrivaju ta dva intervala
\item [d)] najmanji interval koji sadrži date intervale.
\end{itemize}
\begin{miditest}
\begin{upotreba}{1}
#\naslovInt#
#\izlaz{Unesite redom a1, b1, a2 i b2:}\ulaz{2 9 4 11}#
#\izlaz{Duzina zajednickog dela: 5}#
#\izlaz{Presek intervala: [4,9]}#
#\izlaz{Zajednicka duzina intervala: 9}#
#\izlaz{Najmanji interval: [2, 11]}#
\end{upotreba}
\end{miditest}
\begin{miditest}
\begin{upotreba}{2}
#\naslovInt#
#\izlaz{Unesite redom a1, b1, a2 i b2:}\ulaz{1 2 10 13}#
#\izlaz{Duzina zajednickog dela: 0}#
#\izlaz{Presek intervala: prazan}#
#\izlaz{Zajednicka duzina intervala: 4}#
#\izlaz{Najmanji interval: [1, 13]}#
\end{upotreba}
\end{miditest}

\linkresenje{p1.2_11}
\end{Exercise}
\begin{Answer}[ref=p1.2_11]
\includecode{resenja/1_KontrolaToka/1.2_NaredbeGrananja/praktikumi5/2_11.c}
\end{Answer}

\begin{Exercise}[label=p1.2_12] 
Data je funkcija $f(x) = 2 \cdot cos(x) - x^3$. Sa standarnog ulaza se
unosi realan broj $x$ i broj $k$ koje može biti 1, 2 ili 3. Napisati program koji izračunava
$F(k, x) = f(f(f(...f(x)))$ gde je funkcija $f$ primenjena $k$-puta.
\begin{miditest}
\begin{upotreba}{1}
#\naslovInt#
#\izlaz{Unesite redom x i k:}\ulaz{2.31 2}#
#\izlaz{F(2.31, 2)=2557.516602}#
\end{upotreba}
\end{miditest}
\begin{miditest}
\begin{upotreba}{2}
#\naslovInt#
#\izlaz{Unesite redom x i k:}\ulaz{12 1}#
#\izlaz{F(12, 1)=-1726.312256}#
\end{upotreba}
\end{miditest}

\linkresenje{p1.2_12}
\end{Exercise}
\begin{Answer}[ref=p1.2_12]
\includecode{resenja/1_KontrolaToka/1.2_NaredbeGrananja/praktikumi5/2_12.c}
\end{Answer}

\begin{Exercise}[label=p1.2_13] 
 Napisati program koji za uneti redni broj dana u nedelji ispisuje ime dana. U slučaju pogrešnog unosa ispisati odgovarajuću poruku. \\
\begin{miditest}
\begin{upotreba}{1}
#\naslovInt#
#\izlaz{Unesite broj: }\ulaz{4}#
#\izlaz{U pitanju je: cetvrtak}#
\end{upotreba}
\end{miditest}
\begin{miditest}
\begin{upotreba}{2}
#\naslovInt#
#\izlaz{Unesite broj: }\ulaz{7}#
#\izlaz{U pitanju je: nedelja}#
\end{upotreba}
\end{miditest}
\begin{miditest}
\begin{upotreba}{3}
#\naslovInt#
#\izlaz{Unesite broj: }\ulaz{8}#
#\izlaz{Greska: nedozvoljni unos!}#
\end{upotreba}
\end{miditest}

\linkresenje{p1.2_13}
\end{Exercise}
\begin{Answer}[ref=p1.2_13]
\includecode{resenja/1_KontrolaToka/1.2_NaredbeGrananja/praktikumi5/2_13.c}
\end{Answer}

\begin{Exercise}[label=p1.2_14] 
 Sa standardnog ulaza se učitavaju dva cela broja i jedan od karaktera +, -, *, / ili \% koji predstavlja računsku operaciju. Napisatiti program koji  ispisuje vrednost izraza dobijenog primenom ove operacije na date argumente. Koristiti naredbu \textit{switch}. U slučaju pogrešnog unosa ispisati odgovarajuću poruku. \\
\begin{miditest}
\begin{upotreba}{1}
#\naslovInt#
#\izlaz{Unesite operator i dva cela broja:}\ulaz{- 8 11}#
#\izlaz{Rezultat je: -3}#
\end{upotreba}
\end{miditest}
\begin{miditest}
\begin{upotreba}{2}
#\naslovInt#
#\izlaz{Unesite operator i dva cela broja:}\ulaz{/ 14 0}#
#\izlaz{Greska: deljenje nulom nije dozvoljeno!}#
\end{upotreba}
\end{miditest}
\begin{miditest}
\begin{upotreba}{3}
#\naslovInt#
#\izlaz{Unesite operator i dva cela broja:}\ulaz{? 5 7}#
#\izlaz{Greska: nepoznat operator!}#
\end{upotreba}
\end{miditest}
\linkresenje{p1.2_14}
\end{Exercise}
\begin{Answer}[ref=p1.2_14]
\includecode{resenja/1_KontrolaToka/1.2_NaredbeGrananja/praktikumi5/2_14.c}
\end{Answer}


%\item Napisati program koji za uneti pozitivan petocifreni broj $n$ određuje i ispisuje broj njegovih parnih i broj njegovih neparnih cifara. Za analizu cifara koristiti \textit{switch} naredbu.\\
%\begin{miditest}
%\begin{upotreba}{1}
%#\naslovInt#
%#\izlaz{Unesite broj n:}\ulaz{23456}#
%#\izlaz{Broj parnih cifara: 3}#
%#\izlaz{Broj neparnih cifara: 2}#
%\end{upotreba}
%\end{miditest}


\begin{Exercise}[label=p1.2_15] 
 Napisati program koji za uneti datum u formatu \textit{dan.mesec.godina.} proverava da li je korektan.\\
\begin{miditest}
\begin{upotreba}{1}
#\naslovInt#
#\izlaz{Unesite datum:}\ulaz{25.11.1983.}#
#\izlaz{Datum je korektan!}#
\end{upotreba}
\end{miditest}
\begin{miditest}
\begin{upotreba}{2}
#\naslovInt#
#\izlaz{Unesite datum:}\ulaz{1.17.2004.}#
#\izlaz{Datum nije korektan!}#
\end{upotreba}
\end{miditest}

\linkresenje{p1.2_15}
\end{Exercise}
\begin{Answer}[ref=p1.2_15]
\includecode{resenja/1_KontrolaToka/1.2_NaredbeGrananja/praktikumi5/2_15.c}
\end{Answer}

\begin{Exercise}[label=p1.2_16] 
 Napisati program koji za korektno unet datum u formatu \textit{dan.mesec.godina.} ispisuje datum prethodnog dana. \\
\begin{miditest}
\begin{upotreba}{1}
#\naslovInt#
#\izlaz{Unesite datum:}\ulaz{30.4.2008.}#
#\izlaz{Prethodni datum: 29.4.2008.}#
\end{upotreba}
\end{miditest}
\begin{miditest}
\begin{upotreba}{2}
#\naslovInt#
#\izlaz{Unesite datum:}\ulaz{1.12.2005.}#
#\izlaz{Prethodni datum: 30.11.2005.}#
\end{upotreba}
\end{miditest}

\linkresenje{p1.2_16}
\end{Exercise}
\begin{Answer}[ref=p1.2_16]
\includecode{resenja/1_KontrolaToka/1.2_NaredbeGrananja/praktikumi5/2_16.c}
\end{Answer}

\begin{Exercise}[label=p1.2_17] 
 Napisati program koji za korektno unet datum u formatu \textit{dan.mesec.godina.} ispisuje datum narednog dana.\\
\begin{miditest}
\begin{upotreba}{1}
#\naslovInt#
#\izlaz{Unesite datum:}\ulaz{30.4.2008.}#
#\izlaz{Naredni datum: 1.5.2008.}#
\end{upotreba}
\end{miditest}
\begin{miditest}
\begin{upotreba}{2}
#\naslovInt#
#\izlaz{Unesite datum:}\ulaz{1.12.2005.}#
#\izlaz{Naredni datum: 2.12.2005.}#
\end{upotreba}
\end{miditest}
\linkresenje{p1.2_17}
\end{Exercise}
\begin{Answer}[ref=p1.2_17]
%\includecode{resenja/1_KontrolaToka/1.2_NaredbeGrananja/praktikumi5/2_17.c}
\end{Answer}

% Andjelka stavila komentar zbog problema sa prevodjenjem....
%\begin{Exercise}[label=p1.3_]
%\begin{Answer}[ref=p1.3_]
%\includecode{resenja/1_KontrolaToka/1.2_NaredbeGrananja/1_15.c}
%\end{Answer}

\begin{Exercise}[label=p1.4_] 
Sa standarnog ulaza unosi se jedan karakter. Ako je karakter malo
slovo zameniti ga velikim slovom, ako je veliko slovo zameniti malim
slovom, ako je cifra ispisati \verb|u pitanju je cifra|. Ako je bilo
koji drugi karakter onda ga ispisati na standarni izlaz.
\linkresenje{p1.4_}
\end{Exercise}
\begin{Answer}[ref=p1.4_]
%\includecode{resenja/1_KontrolaToka/1.2_NaredbeGrananja/1_14.c}
\end{Answer}

\begin{Exercise}[label=p1.5_] 
Ovaj zadatak je isti kao 2.22! Kako se brise?! Ima vise test primera nego 2.22.
Sa standardnog ulaza se unosi \v cetvorocifren ceo broj. Napisati
program koji datom broju razmenjuje najmanju i najve\' cu
cifru. Dobijeni broj ispisati na izlaz. Ako broj nije \v cetvorocifren
ispisati -1. \\
\begin{miditest}
\begin{upotreba}{1}
#\naslovInt#
#\izlaz{Unesite broj:}\ulaz{3842}#
#\izlaz{3248}#
\end{upotreba}
\end{miditest}
\begin{miditest}
\begin{upotreba}{2}
#\naslovInt#
#\izlaz{Unesite broj:}\ulaz{-4239}#
#\izlaz{-4932}#
\end{upotreba}
\end{miditest}
\begin{miditest}
\begin{upotreba}{3}
#\naslovInt#
#\izlaz{Unesite broj:}\ulaz{123}#
#\izlaz{-1}#
\end{upotreba}
\end{miditest}
\begin{miditest}
\begin{upotreba}{4}
#\naslovInt#
#\izlaz{Unesite broj:}\ulaz{-45678}#
#\izlaz{-1}#
\end{upotreba}
\end{miditest}
\linkresenje{p1.5_}
\end{Exercise}
\begin{Answer}[ref=p1.5_]
\includecode{resenja/1_KontrolaToka/1.2_NaredbeGrananja/1_16.c}
\end{Answer}

\begin{Exercise}[label=p1.6_] 
Sa standardnog ulaza se unosi 5 karaktera. Napisati program koji ispisuje koliko
se puta pojavilo veliko ili malo slovo \verb|a|. \\
\begin{miditest}
\begin{upotreba}{1}
#\naslovInt#
#\izlaz{Unesite karaktere:}\ulaz{aBcAe}#
#\izlaz{2}#
\end{upotreba}
\end{miditest}
\begin{miditest}
\begin{upotreba}{2}
#\naslovInt#
#\izlaz{Unesite karaktere:}\ulaz{aa4A\_}#
#\izlaz{3}#
\end{upotreba}
\end{miditest}
\begin{miditest}
\begin{upotreba}{3}
#\naslovInt#
#\izlaz{Unesite karaktere:}\ulaz{aAaAa}#
#\izlaz{5}#
\end{upotreba}
\end{miditest}
\begin{miditest}
\begin{upotreba}{4}
#\naslovInt#
#\izlaz{Unesite karaktere:}\ulaz{B6(vV}#
#\izlaz{0}#
\end{upotreba}
\end{miditest}
\linkresenje{p1.6_}
\end{Exercise}
\begin{Answer}[ref=p1.6_]
\includecode{resenja/1_KontrolaToka/1.2_NaredbeGrananja/1_17.c}
\end{Answer}

\begin{Exercise}[label=p1.7_] 
Sa standardnog ulaza se unose 5 karaktera. Napisati program koji ispisuje koliko
puta su se pojavile cifre. \\
\begin{miditest}
\begin{upotreba}{1}
#\naslovInt#
#\izlaz{Unesite karaktere:}\ulaz{A1cA3}#
#\izlaz{2}#
\end{upotreba}
\end{miditest}
\begin{miditest}
\begin{upotreba}{2}
#\naslovInt#
#\izlaz{Unesite karaktere:}\ulaz{2a45\_}#
#\izlaz{2}#
\end{upotreba}
\end{miditest}
\begin{miditest}
\begin{upotreba}{3}
#\naslovInt#
#\izlaz{Unesite karaktere:}\ulaz{43986}#
#\izlaz{5}#
\end{upotreba}
\end{miditest}
\begin{miditest}
\begin{upotreba}{4}
#\naslovInt#
#\izlaz{Unesite karaktere:}\ulaz{B6(vV}#
#\izlaz{0}#
\end{upotreba}
\end{miditest}
\linkresenje{p1.7_}
\end{Exercise}
\begin{Answer}[ref=p1.7_]
\includecode{resenja/1_KontrolaToka/1.2_NaredbeGrananja/1_18.c}
\end{Answer}

\begin{Exercise}[label=p1.8_]
Isti kao 2.23 ali ima bolje test primere!
Sa standardnog ulaza se unose tri neozna\v cena trocifrena
broja. Spojiti dva najve\'ca u \v sestocifren broj. Spajanje izvr\v
siti tako da najve\' ci od trocifrenih brojeva bude na po\v cetku \v
sestocifrenog broja. Dobijeni \v sestocifreni broj ispisati na
izlazu. Ako neki od unetih brojeva nije trocifren, ispisati -1. \\
\begin{miditest}
\begin{upotreba}{1}
#\naslovInt#
#\izlaz{Unesite brojeve:}\ulaz{384 123 245}#
#\izlaz{384245}#
\end{upotreba}
\end{miditest}
\begin{miditest}
\begin{upotreba}{2}
#\naslovInt#
#\izlaz{Unesite brojeve:}\ulaz{123 345 5}#
#\izlaz{-1}#
\end{upotreba}
\end{miditest}
\begin{miditest}
\begin{upotreba}{3}
#\naslovInt#
#\izlaz{Unesite brojeve:}\ulaz{1242 234 324}#
#\izlaz{-1}#
\end{upotreba}
\end{miditest}
\begin{miditest}
\begin{upotreba}{4}
#\naslovInt#
#\izlaz{Unesite brojeve:}\ulaz{374 23 898}#
#\izlaz{-1}#
\end{upotreba}
\end{miditest}
\linkresenje{p1.8_}
\end{Exercise}
\begin{Answer}[ref=p1.8_]
\includecode{resenja/1_KontrolaToka/1.2_NaredbeGrananja/1_19.c}
\end{Answer}


\begin{Exercise}[label=p1.9_]
Korisnik unosi 3 cela broja: \verb|(p), (q) i (r)|.
Nakon toga unosi i dva karaktera, koji imaju
sledeci smisao:
\begin{description}
\item['k']-logi\v cka konjukcija
\item['d']-logi\v cka disjunkcija
\item['m']-relacija manje
\item['v']-relacija ve\' ce
\end{description}
Nakon toga se ra\v cuna vrednost izraza 
\verb|(p) op1 (q) op2 (r)| i ispisuje rezultat.
\linkresenje{p1.9_}
\end{Exercise}
\begin{Answer}[ref=p1.9_]
%\includecode{resenja/1_KontrolaToka/1.2_NaredbeGrananja/1_14.c}
\end{Answer}

\begin{Exercise}[label=v1.2_15] 
Tekst \\
\linkresenje{v1.2_15}
\end{Exercise}
\begin{Answer}[ref=v1.2_15]
\includecode{resenja/1_KontrolaToka/1.2_NaredbeGrananja/1_20.c}
\end{Answer}

\begin{Exercise}[label=v1.2_16] 
Tekst \\
\linkresenje{v1.2_16}
\end{Exercise}
\begin{Answer}[ref=v1.2_16]
\includecode{resenja/1_KontrolaToka/1.2_NaredbeGrananja/1_21.c}
\end{Answer}

\begin{Exercise}[label=v1.2_17] 
Tekst \\
\linkresenje{v1.2_17}
\end{Exercise}
\begin{Answer}[ref=v1.2_17]
\includecode{resenja/1_KontrolaToka/1.2_NaredbeGrananja/1_22.c}
\end{Answer}

\begin{Exercise}[label=v1.2_18] 
Tekst \\
\linkresenje{v1.2_18}
\end{Exercise}
\begin{Answer}[ref=v1.2_18]
\includecode{resenja/1_KontrolaToka/1.2_NaredbeGrananja/1_23.c}
\end{Answer}

\section{Rešenja}
\shipoutAnswer