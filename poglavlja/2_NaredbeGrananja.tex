\chapter{Kontrola toka}


\section{Naredbe grananja}

\begin{Exercise}[label=v1.2_01] 
Tekst
\linkresenje{v1.2_01}
\end{Exercise}
\begin{Answer}[ref=v1.2_01]
\includecode{resenja/1_KontrolaToka/1.2_NaredbeGrananja/1_01.c}
\end{Answer}

\begin{Exercise}[label=v1.2_02] 
Tekst
\linkresenje{v1.2_02}
\end{Exercise}
\begin{Answer}[ref=v1.2_02]
\includecode{resenja/1_KontrolaToka/1.2_NaredbeGrananja/1_02.c}
\end{Answer}

\begin{Exercise}[label=v1.2_03] 
Tekst
\linkresenje{v1.2_03}
\end{Exercise}
\begin{Answer}[ref=v1.2_03]
\includecode{resenja/1_KontrolaToka/1.2_NaredbeGrananja/1_03.c}
\end{Answer}

\begin{Exercise}[label=v1.2_04] 
Tekst
\linkresenje{v1.2_04}
\end{Exercise}
\begin{Answer}[ref=v1.2_04]
\includecode{resenja/1_KontrolaToka/1.2_NaredbeGrananja/1_04.c}
\end{Answer}

\begin{Exercise}[label=v1.2_05] 
Tekst
\linkresenje{v1.2_05}
\end{Exercise}
\begin{Answer}[ref=v1.2_05]
\includecode{resenja/1_KontrolaToka/1.2_NaredbeGrananja/1_05.c}
\end{Answer}

\begin{Exercise}[label=v1.2_06] 
Tekst
\linkresenje{v1.2_06}
\end{Exercise}
\begin{Answer}[ref=v1.2_06]
\includecode{resenja/1_KontrolaToka/1.2_NaredbeGrananja/1_06.c}
\end{Answer}

\begin{Exercise}[label=v1.2_07] 
Tekst
\linkresenje{v1.2_07}
\end{Exercise}
\begin{Answer}[ref=v1.2_07]
\includecode{resenja/1_KontrolaToka/1.2_NaredbeGrananja/1_07.c}
\end{Answer}

\begin{Exercise}[label=v1.2_08] 
Tekst
\linkresenje{v1.2_08}
\end{Exercise}
\begin{Answer}[ref=v1.2_08]
\includecode{resenja/1_KontrolaToka/1.2_NaredbeGrananja/1_08.c}
\end{Answer}

\begin{Exercise}[label=v1.2_09] 
Tekst
\linkresenje{v1.2_09}
\end{Exercise}
\begin{Answer}[ref=v1.2_09]
\includecode{resenja/1_KontrolaToka/1.2_NaredbeGrananja/1_09.c}
\end{Answer}

\begin{Exercise}[label=v1.2_10] 
Tekst
\linkresenje{v1.2_10}
\end{Exercise}
\begin{Answer}[ref=v1.2_10]
\includecode{resenja/1_KontrolaToka/1.2_NaredbeGrananja/1_10.c}
\end{Answer}

\begin{Exercise}[label=v1.2_11] 
Tekst
\linkresenje{v1.2_11}
\end{Exercise}
\begin{Answer}[ref=v1.2_11]
\includecode{resenja/1_KontrolaToka/1.2_NaredbeGrananja/1_11.c}
\end{Answer}

\begin{Exercise}[label=v1.2_12] 
Tekst
\linkresenje{v1.2_12}
\end{Exercise}
\begin{Answer}[ref=v1.2_12]
\includecode{resenja/1_KontrolaToka/1.2_NaredbeGrananja/1_12.c}
\end{Answer}

\begin{Exercise}[label=v1.2_13] 
Tekst
\linkresenje{v1.2_13}
\end{Exercise}
\begin{Answer}[ref=v1.2_13]
\includecode{resenja/1_KontrolaToka/1.2_NaredbeGrananja/1_13.c}
\end{Answer}

\begin{Exercise}[label=v1.2_14] 
Tekst
\linkresenje{v1.2_14}
\end{Exercise}
\begin{Answer}[ref=v1.2_14]
\includecode{resenja/1_KontrolaToka/1.2_NaredbeGrananja/1_14.c}
\end{Answer}


\begin{Exercise}[label=p1.2_] 
Sa standardnog ulaza se unosi ceo četvorocifren broj. Napisati program
koji ispisuje njegovu najveću cifru na standardni izlaz. \\
\begin{miditest}
\begin{upotreba}{1}
#\naslovInt#
#\izlaz{Unesite broj:}\ulaz{6835}#
#\izlaz{Najveca cifra je: 8}#
\end{upotreba}
\end{miditest}
\begin{miditest}
\begin{upotreba}{2}
#\naslovInt#
#\izlaz{Unesite broj:}\ulaz{238}#
#\izlaz{Greska: Niste uneli cetvorocifren broj!}#
\end{upotreba}
\end{miditest}
\linkresenje{p1.2_}
\end{Exercise}
\begin{Answer}[ref=p1.2_]
%\includecode{resenja/1_KontrolaToka/1.2_NaredbeGrananja/1_14.c}
\end{Answer}

\begin{Exercise}[label=p1.2_] 
Napisati program koji za dati trocifren broj proverava da li je Amstrongov.
Broj je Amstrongov ako je jednak zbiru kubova svojih cifara.\\
\begin{miditest}
\begin{upotreba}{1}
#\naslovInt#
#\izlaz{Unesite broj:}\ulaz{153}#
#\izlaz{Broj je Amstrongov.}#
\end{upotreba}
\end{miditest}
\begin{miditest}
\begin{upotreba}{2}
#\naslovInt#
#\izlaz{Unesite broj:}\ulaz{111}#
#\izlaz{Broj nije Amstrongov.}#
\end{upotreba}
\end{miditest}
\begin{miditest}
\begin{upotreba}{3}
#\naslovInt#
#\izlaz{Unesite broj:}\ulaz{84}#
#\izlaz{Greska: Niste uneli trocifren broj!}#
\end{upotreba}
\end{miditest}

\linkresenje{p1.2_}
\end{Exercise}
\begin{Answer}[ref=p1.2_]
%\includecode{resenja/1_KontrolaToka/1.2_NaredbeGrananja/1_14.c}
\end{Answer}

\begin{Exercise}[label=p1.2_] 
 Za ceo broj $k$ između 1 i 189 koji se unosi sa standardnog ulaza, odrediti cifru koja se nalazi na $k$-toj poziciji niza 12345678910111213....9899 u kom su redom ispisani brojevi od 1 do 99. \\
\begin{miditest}
\begin{upotreba}{1}
#\naslovInt#
#\izlaz{Unesite k:}\ulaz{13}#
#\izlaz{Na 13-toj poziciji je broj 1.}#
\end{upotreba}
\end{miditest}
\begin{miditest}
\begin{upotreba}{2}
#\naslovInt#
#\izlaz{Unesite k:}\ulaz{105}#
#\izlaz{Na 105-toj poziciji je broj 7.}#
\end{upotreba}
\end{miditest}

\linkresenje{p1.2_}
\end{Exercise}
\begin{Answer}[ref=p1.2_]
%\includecode{resenja/1_KontrolaToka/1.2_NaredbeGrananja/1_14.c}
\end{Answer}

\begin{Exercise}[label=p1.2_] 
 Sa standardnog ulaza se unosi četvorocifreni pozitivan broj. Napisati program koji računa i ispisuje proizvod parnih cifara datog broja. Ukoliko uneti broj nije pozitivna četvorocifrena vrednost ispisati poruku \textit{Greska!}.\\
\begin{miditest}
\begin{upotreba}{1}
#\naslovInt#
#\izlaz{Unesite broj:}\ulaz{8123}#
#\izlaz{Proizvod parnih cifara: 16}#
\end{upotreba}
\end{miditest}
\begin{miditest}
\begin{upotreba}{2}
#\naslovInt#
#\izlaz{Unesite broj:}\ulaz{3579}#
#\izlaz{Proizvod parnih cifara: 0}#
\end{upotreba}
\end{miditest}
\begin{miditest}
\begin{upotreba}{3}
#\naslovInt#
#\izlaz{Unesite broj:}\ulaz{288}#
#\izlaz{Greska!}#
\end{upotreba}
\end{miditest}


\linkresenje{p1.2_}
\end{Exercise}
\begin{Answer}[ref=p1.2_]
%\includecode{resenja/1_KontrolaToka/1.2_NaredbeGrananja/1_14.c}
\end{Answer}

\begin{Exercise}[label=p1.2_] 
 Sa standarnog ulaza unosi se 5 karaktera. Proveriti da li je prvi karakter veliko ili malo slovo $a$. Ako jeste, ispisati karaktere obrnutim redosledom, a ako nije, ništa ne ispisivati.\\
\begin{miditest}
\begin{upotreba}{1}
#\naslovInt#
#\izlaz{Unesite karaktere:}\ulaz{A u E f h}#
#\izlaz{h f E u A}#
\end{upotreba}
\end{miditest}
\begin{miditest}
\begin{upotreba}{2}
#\naslovInt#
#\izlaz{Unesite karaktere:}\ulaz{k L M 9 o}#
#\izlaz{}#
\end{upotreba}
\end{miditest}

\linkresenje{p1.2_}
\end{Exercise}
\begin{Answer}[ref=p1.2_]
%\includecode{resenja/1_KontrolaToka/1.2_NaredbeGrananja/1_14.c}
\end{Answer}

\begin{Exercise}[label=p1.2_] 
 Sa standarnog ulaza unosi se jedan karakter. Ako je u pitanju malo slovo, zameniti ga odgovarajućim velikim slovom i ispisati na standardni izlaz. Ako je u pitanju veliko slovo, zameniti ga odgovarajućim malim slovom i ispisati ga na standardni izlaz. Ako je u pitanju cifra ispisati poruku \textit{cifra}. Ako je u pitanju bilo koji drugi karakter, onda ga ispisati na standarni izlaz između dveju zvezdica.\\
\begin{miditest}
\begin{upotreba}{1}
#\naslovInt#
#\izlaz{Unesite karakter:}\ulaz{K}#
#\izlaz{k}#
\end{upotreba}
\end{miditest}
\begin{miditest}
\begin{upotreba}{2}
#\naslovInt#
#\izlaz{Unesite karakter:}\ulaz{8}#
#\izlaz{cifra}#
\end{upotreba}
\end{miditest}
\begin{miditest}
\begin{upotreba}{3}
#\naslovInt#
#\izlaz{Unesite karakter:}\ulaz{>}#
#\izlaz{*>*}#
\end{upotreba}
\end{miditest}

\linkresenje{p1.2_}
\end{Exercise}
\begin{Answer}[ref=p1.2_]
%\includecode{resenja/1_KontrolaToka/1.2_NaredbeGrananja/1_14.c}
\end{Answer}

\begin{Exercise}[label=p1.2_] 
 Sa standardnog ulaza se unosi 5 karaktera. Ispisati na izlazu broj unetih malih slova.\\
\begin{miditest}
\begin{upotreba}{1}
#\naslovInt#
#\izlaz{Unesite karaktere:}\ulaz{A u E f h}#
#\izlaz{Broj malih slova: 3}#
\end{upotreba}
\end{miditest}
\begin{miditest}
\begin{upotreba}{2}
#\naslovInt#
#\izlaz{Unesite karaktere:}\ulaz{k L M 9 o}#
#\izlaz{Broj malih slova: 2}#
\end{upotreba}
\end{miditest}

\linkresenje{p1.2_}
\end{Exercise}
\begin{Answer}[ref=p1.2_]
%\includecode{resenja/1_KontrolaToka/1.2_NaredbeGrananja/1_14.c}
\end{Answer}

\begin{Exercise}[label=p1.2_] 
 Sa standardnog ulaza se unosi četvorocifren ceo broj. Napisati program koji datom broju razmenjuje najmanju i najveću cifru. Dobijeni broj ispisati na standardni izlaz. Ako uneti broj nije četvorocifren ispisati poruku \textit{Greska!}. \\
\begin{miditest}
\begin{upotreba}{1}
#\naslovInt#
#\izlaz{Unesite broj:}\ulaz{2863}#
#\izlaz{Novi broj: 8263}#
\end{upotreba}
\end{miditest}
\begin{miditest}
\begin{upotreba}{2}
#\naslovInt#
#\izlaz{Unesite broj:}\ulaz{247}#
#\izlaz{Greska!}#
\end{upotreba}
\end{miditest}

\linkresenje{p1.2_}
\end{Exercise}
\begin{Answer}[ref=p1.2_]
%\includecode{resenja/1_KontrolaToka/1.2_NaredbeGrananja/1_14.c}
\end{Answer}

\begin{Exercise}[label=p1.2_] 
 Sa standardnog ulaza se unose tri neoznačena trocifrena broja. Spojiti
dva najveća u šestocifren broj. Spajanje izvršiti tako da najveći od trocifrenih
brojeva bude na početku šestocifrenog broja. Dobijeni šestocifreni broj ispisati
na izlazu. Ako neki od unetih brojeva nije trocifren, ispisati poruku \textit{Greska!}.\\
\begin{miditest}
\begin{upotreba}{1}
#\naslovInt#
#\izlaz{Unesite brojeve:}\ulaz{185 247 311}#
#\izlaz{Trazeni broj je: 311247}#
\end{upotreba}
\end{miditest}
\begin{miditest}
\begin{upotreba}{2}
#\naslovInt#
#\izlaz{Unesite brojeve:}\ulaz{865 11 298}#
#\izlaz{Greska!}#
\end{upotreba}
\end{miditest}

\linkresenje{p1.2_}
\end{Exercise}
\begin{Answer}[ref=p1.2_]
%\includecode{resenja/1_KontrolaToka/1.2_NaredbeGrananja/1_14.c}
\end{Answer}

\begin{Exercise}[label=p1.2_] 
 Sa standardnog ulaza se učitavaju realni koeficijenti $A$ i $B$ linearne jednačine $Ax+B = 0$. Napisati program koji ispisuje rešenja ove jednačine - ukoliko jednačina nema rešenja ili ukoliko ima više od jednog rešenja ispisati odgovarajuće poruke.\\
\begin{miditest}
\begin{upotreba}{1}
#\naslovInt#
#\izlaz{Unesite koeficijente A i B:}\ulaz{2 -5}#
#\izlaz{x=2.5}#
\end{upotreba}
\end{miditest}
\begin{miditest}
\begin{upotreba}{2}
#\naslovInt#
#\izlaz{Unesite koeficijente A i B:}\ulaz{0 18.5}#
#\izlaz{Jednacina nema resenja.}#
\end{upotreba}
\end{miditest}




\linkresenje{p1.2_}
\end{Exercise}
\begin{Answer}[ref=p1.2_]
%\includecode{resenja/1_KontrolaToka/1.2_NaredbeGrananja/1_14.c}
\end{Answer}

\begin{Exercise}[label=p1.2_] 
 Napisati program koji za dva data intervala realne prave (a1, b1) i
(a2, b2) određuje:
\begin{itemize}
\item [a)] dužinu zajedničkog dela ta dva intervala
\item [b)] najveći interval sadržan u datim intervalima (presek),a ako on ne postoji dati
odgovarajuću poruku.
\item [c)] dužinu realne prave koju pokrivaju ta dva intervala
\item [d)] najmanji interval koji sadrži date intervale
\end{itemize}
\begin{miditest}
\begin{upotreba}{1}
#\naslovInt#
#\izlaz{Unesite redom a1, b1, a2 i b2:}\ulaz{2 9 4 11}#
#\izlaz{Duzina zajednickog dela: 5}#
#\izlaz{Presek intervala: [4,9]}#
#\izlaz{Zajednicka duzina intervala: 9}#
#\izlaz{Najmanji interval: [2, 11]}#
\end{upotreba}
\end{miditest}
\begin{miditest}
\begin{upotreba}{2}
#\naslovInt#
#\izlaz{Unesite redom a1, b1, a2 i b2:}\ulaz{1 2 10 13}#
#\izlaz{Duzina zajednickog dela: 0}#
#\izlaz{Presek intervala: prazan}#
#\izlaz{Zajednicka duzina intervala: 4}#
#\izlaz{Najmanji interval: [1, 13]}#
\end{upotreba}
\end{miditest}

\linkresenje{p1.2_}
\end{Exercise}
\begin{Answer}[ref=p1.2_]
%\includecode{resenja/1_KontrolaToka/1.2_NaredbeGrananja/1_14.c}
\end{Answer}

\begin{Exercise}[label=p1.2_] 
 Data je funkcija $f(x) = 2 \cdot cos(x) - x^3$. Sa standarnog ulaza se
unosi realan broj $x$ i broj $k$ koje može biti 1, 2 ili 3. Napisati program koji izračunava
$F(k, x) = f(f(f(...f(x)))$ gde je funkcija $f$ primenjena $k$-puta.
\begin{miditest}
\begin{upotreba}{1}
#\naslovInt#
#\izlaz{Unesite redom x i k:}\ulaz{2.31 2}#
#\izlaz{F(2.31, 2)=2557.516602}#
\end{upotreba}
\end{miditest}
\begin{miditest}
\begin{upotreba}{2}
#\naslovInt#
#\izlaz{Unesite redom x i k:}\ulaz{12 1}#
#\izlaz{F(12, 1)=-1726.312256}#
\end{upotreba}
\end{miditest}

\linkresenje{p1.2_}
\end{Exercise}
\begin{Answer}[ref=p1.2_]
%\includecode{resenja/1_KontrolaToka/1.2_NaredbeGrananja/1_14.c}
\end{Answer}

\begin{Exercise}[label=p1.2_] 
 Napisati program koji za uneti broj $n\ (1\leq n \leq 7)$ koji predstavlja redni broj dana u nedelji ispisuje ime dana. U slučaju pogrešnog unosa ispisati odgovarajuću poruku. \\
\begin{miditest}
\begin{upotreba}{1}
#\naslovInt#
#\izlaz{Unesite broj: }\ulaz{4}#
#\izlaz{U pitanju je: cetvrtak}#
\end{upotreba}
\end{miditest}
\begin{miditest}
\begin{upotreba}{2}
#\naslovInt#
#\izlaz{Unesite broj: }\ulaz{7}#
#\izlaz{U pitanju je: nedelja}#
\end{upotreba}
\end{miditest}
\begin{miditest}
\begin{upotreba}{3}
#\naslovInt#
#\izlaz{Unesite broj: }\ulaz{8}#
#\izlaz{Greska: nedozvoljni unos!}#
\end{upotreba}
\end{miditest}

\linkresenje{p1.2_}
\end{Exercise}
\begin{Answer}[ref=p1.2_]
%\includecode{resenja/1_KontrolaToka/1.2_NaredbeGrananja/1_14.c}
\end{Answer}

\begin{Exercise}[label=p1.2_] 
 Sa standardnog ulaza se učitavaju dva cela broja i jedan od karaktera +, -, *, / ili \% koji predstavlja operaciju koju treba izvršiti nad unetim brojevima. Napisatiti program koji korišćenjem \textit{switch} naredbe analizira o kom karakteru je reč i na standardni izlaz ispisuje rezultat. U slučaju pogrešnog unosa ispisati odgovarajuću poruku. \\
\begin{miditest}
\begin{upotreba}{1}
#\naslovInt#
#\izlaz{Unesite operator i dva cela broja:}\ulaz{- 8 11}#
#\izlaz{Rezultat je: -3}#
\end{upotreba}
\end{miditest}
\begin{miditest}
\begin{upotreba}{2}
#\naslovInt#
#\izlaz{Unesite operator i dva cela broja:}\ulaz{/ 14 0}#
#\izlaz{Greska: deljenje nulom nije dozvoljeno!}#
\end{upotreba}
\end{miditest}
\begin{miditest}
\begin{upotreba}{3}
#\naslovInt#
#\izlaz{Unesite operator i dva cela broja:}\ulaz{? 5 7}#
#\izlaz{Greska: nepoznat operator!}#
\end{upotreba}
\end{miditest}


%\item Napisati program koji za uneti pozitivan petocifreni broj $n$ određuje i ispisuje broj njegovih parnih i broj njegovih neparnih cifara. Za analizu cifara koristiti \textit{switch} naredbu.\\
%\begin{miditest}
%\begin{upotreba}{1}
%#\naslovInt#
%#\izlaz{Unesite broj n:}\ulaz{23456}#
%#\izlaz{Broj parnih cifara: 3}#
%#\izlaz{Broj neparnih cifara: 2}#
%\end{upotreba}
%\end{miditest}

\linkresenje{p1.2_}
\end{Exercise}
\begin{Answer}[ref=p1.2_]
%\includecode{resenja/1_KontrolaToka/1.2_NaredbeGrananja/1_14.c}
\end{Answer}

\begin{Exercise}[label=p1.2_] 
 Napisati program koji za uneti datum u formatu \textit{dan.mesec.godina.} proverava da li je korektan.\\
\begin{miditest}
\begin{upotreba}{1}
#\naslovInt#
#\izlaz{Unesite datum:}\ulaz{25.11.1983.}#
#\izlaz{Datum je korektan!}#
\end{upotreba}
\end{miditest}
\begin{miditest}
\begin{upotreba}{2}
#\naslovInt#
#\izlaz{Unesite datum:}\ulaz{1.17.2004.}#
#\izlaz{Datum nije korektan!}#
\end{upotreba}
\end{miditest}

\linkresenje{p1.2_}
\end{Exercise}
\begin{Answer}[ref=p1.2_]
%\includecode{resenja/1_KontrolaToka/1.2_NaredbeGrananja/1_14.c}
\end{Answer}

\begin{Exercise}[label=p1.2_] 
 Napisati program koji za korektno unet datum u formatu \textit{dan.mesec.godina.} ispisuje datum prethodnog dana. \\
\begin{miditest}
\begin{upotreba}{1}
#\naslovInt#
#\izlaz{Unesite datum:}\ulaz{30.4.2008.}#
#\izlaz{Prethodni datum: 29.4.2008.}#
\end{upotreba}
\end{miditest}
\begin{miditest}
\begin{upotreba}{2}
#\naslovInt#
#\izlaz{Unesite datum:}\ulaz{1.12.2005.}#
#\izlaz{Prethodni datum: 30.11.2005.}#
\end{upotreba}
\end{miditest}

\linkresenje{p1.2_}
\end{Exercise}
\begin{Answer}[ref=p1.2_]
%\includecode{resenja/1_KontrolaToka/1.2_NaredbeGrananja/1_14.c}
\end{Answer}

\begin{Exercise}[label=p1.2_] 
 Napisati program koji za korektno unet datum u formatu \textit{dan.mesec.godina.} ispisuje datum narednog dana.\\
\begin{miditest}
\begin{upotreba}{1}
#\naslovInt#
#\izlaz{Unesite datum:}\ulaz{30.4.2008.}#
#\izlaz{Naredni datum: 1.5.2008.}#
\end{upotreba}
\end{miditest}
\begin{miditest}
\begin{upotreba}{2}
#\naslovInt#
#\izlaz{Unesite datum:}\ulaz{1.12.2005.}#
#\izlaz{Naredni datum: 2.12.2005.}#
\end{upotreba}
\end{miditest}
\linkresenje{p1.2_}
\end{Exercise}
\begin{Answer}[ref=p1.2_]
%\includecode{resenja/1_KontrolaToka/1.2_NaredbeGrananja/1_14.c}
\end{Answer}


