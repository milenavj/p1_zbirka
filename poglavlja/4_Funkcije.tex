\section{Funkcije}


\komentar{TODO U nekim zadacima pise samo Napisati program koji testira rad ove funkcije, dok u nekim zadacima je detaljnije opisan sam program, npr "Napisati program koji sa standardnog ulaza učitava tri cela broja i ispisuje rezultat poziva funkcije." Nekako bi trebalo to ujednaciti. Meni je lepsi ovaj drugi stil, jer je precizniji.}

\komentar{TODO Smisliti odgovarajući redosled za ove zadatke}

\begin{Exercise}[label=v1.4_01] 
Napisati funkciju \kckod{int kvadrat(int x)} koja računa kvadrat datog
broja. Napisati program koji učitava ceo broj i ispuje reyultat poziva
funkcije.

\linkresenje{v1.4_01}
\end{Exercise}
\begin{Answer}[ref=v1.4_01]
\includecode{resenja/1_KontrolaToka/1.4_Funkcije/1_01.c}
\end{Answer}


\begin{Exercise}[label=v1.4_01] 
Napisati funkciju \kckod{int kub(int x)} koja računa kub datog
broja. Napisati program koji učitava ceo broj i ispuje reyultat poziva
funkcije.

\linkresenje{v1.4_01}
\end{Exercise}
\begin{Answer}[ref=v1.4_01]
\includecode{resenja/1_KontrolaToka/1.4_Funkcije/1_01.c}
\end{Answer}

\begin{Exercise}[label=v1.4_02] 
Napisati funkciju \kckod{float stepen(float x, int n)} koja računa
vrednost $n$-tog stepena realnog broja $x$. Napisati program koji
učitava relan broj $x$ i ceo broj $n$ i ispiuje rezultat rada
funkcije.

\linkresenje{v1.4_02}
\end{Exercise}
\begin{Answer}[ref=v1.4_02]
\includecode{resenja/1_KontrolaToka/1.4_Funkcije/1_02.c}
\end{Answer}

\begin{Exercise}[label=v1.4_03] 
Napisati funkciju \kckod{int euklid(int x, int y)} koja za dva data
cela broja određuje najveći zajednički delilac primenom Euklidovog
algoritma. Napisati program koji učitava dva cela broja i ispisuje
rezultat poziva funkcije.

\linkresenje{v1.4_03}
\end{Exercise}
\begin{Answer}[ref=v1.4_03]
\includecode{resenja/1_KontrolaToka/1.4_Funkcije/1_03.c}
\end{Answer}

\begin{Exercise}[label=v1.4_04] 
Napisati funkciju \kckod{float zbir\_reciprocnih(int n)} koja za dato
$n$ vraća zbir recipročnih vrednosti brojeva od $1$ do $n$. Napisati
program koji učitava ceo broj i ispisuje rezultat rada funkcje
zaokružen na dve decimale.

\linkresenje{v1.4_04}
\end{Exercise}
\begin{Answer}[ref=v1.4_04]
\includecode{resenja/1_KontrolaToka/1.4_Funkcije/1_04.c}
\end{Answer}

\begin{Exercise}[label=v1.4_05] 
 Napisati funkciju \kckod{aritmeticka\_sredina(int n)} koja računa
 aritmetičku sredinu cifara datog broja. Napisati program koji učitava
 ceo broj i ispisuje rezultat rada funkcje zaokružen na tri decimale.

\begin{miditest}
\begin{upotreba}{1}
#\naslovInt#
#\izlaz{Unesite broj:}\ulaz{461}#
#\izlaz{3.667}#
\end{upotreba}
\end{miditest}
\begin{miditest}
\begin{upotreba}{2}
#\naslovInt#
#\izlaz{Unesite broj:}\ulaz{1001}#
#\izlaz{0.500}#
\end{upotreba}
\end{miditest}

\begin{miditest}
\begin{upotreba}{3}
#\naslovInt#
#\izlaz{Unesite broj:}\ulaz{-84723}#
#\izlaz{4.800}#
\end{upotreba}
\end{miditest}
\linkresenje{v1.4_05}
\end{Exercise}
\begin{Answer}[ref=v1.4_05]
\includecode{resenja/1_KontrolaToka/1.4_Funkcije/1_05.c}
\end{Answer}

\begin{Exercise}[label=v1.4_06] 
Napisati funkciju \kckod{void ispis(float x, float y, unsigned n)}
koja za dva realna broja $x$ i $y$ i jedan neoznačeni ceo broj $n$
ispisuje vrednosti sinusne funkcije u $n$ ravnomerno raspoređenih
tačaka intervala $[x,y]$.  Napisati program koji učitava dva realna
broja i jedan ceo pozitivan broj i ispisuje rezultat rada funkcije.

\linkresenje{v1.4_06}
\end{Exercise}
\begin{Answer}[ref=v1.4_06]
\includecode{resenja/1_KontrolaToka/1.4_Funkcije/1_06.c}
\end{Answer}

\begin{Exercise}[label=v1.4_07] 
Napisati funkciju \kckod{int broj\_ncifara(int x)} koja određuje broj
neparnih cifre u zapisu datog celog broja. Testirati rad ove funkcije
u programu koji učitava cele brojeve dok se ne unese nula i ispisuje
broj neparnih cifara svakog unetog broja.

\linkresenje{v1.4_07}
\end{Exercise}
\begin{Answer}[ref=v1.4_07]
\includecode{resenja/1_KontrolaToka/1.4_Funkcije/1_07.c}
\end{Answer}



\begin{Exercise}[label=p1.4_01] 
Napisati funkciju \kckod{int min(int x, int y, int z)} koja izračunava
minimum tri broja. Napisati program koji učitava tri cela broja i
ispisuje rezultat poziva funkcije.

\begin{miditest}
\begin{upotreba}{1}
#\naslovInt#
#\izlaz{Unesite brojeve:}\ulaz{19 8 14}#
#\izlaz{Minimum je: 8}#
\end{upotreba}
\end{miditest}
\begin{miditest}
\begin{upotreba}{2}
#\naslovInt#
#\izlaz{Unesite brojeve:}\ulaz{-6 11 -12}#
#\izlaz{Minimum je: -12}#
\end{upotreba}
\end{miditest}
\linkresenje{p1.4_01}
\end{Exercise}
\begin{Answer}[ref=p1.4_01]
\includecode{resenja/1_KontrolaToka/1.4_Funkcije/praktikumi8/4_01.c}
\end{Answer}


\begin{Exercise}[label=p1.4_02] 
Napisati funkciju \kckod{unsigned int apsolutna\_vrednost(int x)} koja
izračunava apsolutnu vrednost broja $x$. Napisati program koji sa
učitava jedan ceo broj i ispisuje rezultat poziva funkcije.

\begin{miditest}
\begin{upotreba}{1}
#\naslovInt#
#\izlaz{Unesite broj:}\ulaz{-34}#
#\izlaz{Apsolutna vrednost: 34}#
\end{upotreba}
\end{miditest}
\begin{miditest}
\begin{upotreba}{2}
#\naslovInt#
#\izlaz{Unesite broj:}\ulaz{5}#
#\izlaz{Apsolutna vrednost: 5}#
\end{upotreba}
\end{miditest}

\linkresenje{p1.4_02}
\end{Exercise}
\begin{Answer}[ref=p1.4_02]
\includecode{resenja/1_KontrolaToka/1.4_Funkcije/praktikumi8/4_02.c}
\end{Answer}


\begin{Exercise}[label=p1.4_03] 
Napisati funkciju \kckod{float razlomljeni\_deo(float x)} koja
izračunava razlomljeni deo broja $x$. Napisati program koji sa učitava
jedan realan broj i ispisuje rezultat poziva funkcije.
 
\begin{miditest}
\begin{upotreba}{1}
#\naslovInt#
#\izlaz{Unesite broj:}\ulaz{8.235}#
#\izlaz{Razlomljeni deo: 0.235000}#
\end{upotreba}
\end{miditest}
\begin{miditest}
\begin{upotreba}{2}
#\naslovInt#
#\izlaz{Unesite broj:}\ulaz{-5.11}#
#\izlaz{Razlomljeni deo: 0.110000}#
\end{upotreba}
\end{miditest}
\linkresenje{p1.4_03}
\end{Exercise}
\begin{Answer}[ref=p1.4_03]
\includecode{resenja/1_KontrolaToka/1.4_Funkcije/praktikumi8/4_03.c}
\end{Answer}


\begin{Exercise}[label=p1.4_04] 
Napisati funkciju \kckod{void romb(int n)} koja iscrtava romb čija je
stranica dužine $n$. Napisati program koji učitava ceo pozitivan broj
i ispisuje rezultat poziva funkcije.
 
\begin{miditest}
\begin{upotreba}{1}
#\naslovInt#
#\izlaz{Unesite broj n:}\ulaz{5}#
#\izlaz{\ \ \ \ *****}#
#\izlaz{\ \ \ *****}#
#\izlaz{\ \ *****}#
#\izlaz{\ *****}#
#\izlaz{*****}#
\end{upotreba}
\end{miditest}
\begin{miditest}
\begin{upotreba}{2}
#\naslovInt#
#\izlaz{Unesite broj n:}\ulaz{2}#
#\izlaz{\ **}#
#\izlaz{**}#
\end{upotreba}
\end{miditest}

\begin{miditest}
\begin{upotreba}{3}
#\naslovInt#
#\izlaz{Unesite broj n:}\ulaz{-5}#
#\izlaz{Greska: pogresna dimenzija!}#
\end{upotreba}
\end{miditest}
\linkresenje{p1.4_04}
\end{Exercise}
\begin{Answer}[ref=p1.4_04]
\includecode{resenja/1_KontrolaToka/1.4_Funkcije/praktikumi8/4_04.c}
\end{Answer}


\begin{Exercise}[label=p1.4_05] 
Napisati funkciju \kckod{void grafikon\_h(int a, int b, int c, int d)}
koja iscrtava horizontalni prikaz zadatih vrednosti. Napisati program
koji učitava četiri pozitivna cela broja i prikazuje rezultat poziva
funkcije. U slučaju pogrešnog unosa, ispisati poruku o grešci.
 
\begin{miditest}
\begin{upotreba}{1}
#\naslovInt#
#\izlaz{Unesite vrednosti:}\ulaz{4 1 7 5}#
#\izlaz{****}#
#\izlaz{*}#
#\izlaz{*******}#
#\izlaz{*****}#
\end{upotreba}
\end{miditest}
\begin{miditest}
\begin{upotreba}{2}
#\naslovInt#
#\izlaz{Unesite vrednosti:}\ulaz{8 -2 5 4}#
#\izlaz{Greska: pogresan unos!}#
\end{upotreba}
\end{miditest}

\begin{miditest}
\begin{upotreba}{3}
#\naslovInt#
#\izlaz{Unesite vrednosti:}\ulaz{5 2 2 10}#
#\izlaz{*****}#
#\izlaz{**}#
#\izlaz{**}#
#\izlaz{**********}#
\end{upotreba}
\end{miditest}
\linkresenje{p1.4_05}
\end{Exercise}
\begin{Answer}[ref=p1.4_05]
\includecode{resenja/1_KontrolaToka/1.4_Funkcije/praktikumi8/4_05.c}
\end{Answer}


\begin{Exercise}[label=p1.4_06] 
Napisati funkciju
\kckod{void grafikon\_v(int a, int b, int c, int d)} koja iscrtava
vertikalni prikaz zadatih vrednosti. Napisati program koji učitava
četiri pozitivna cela broja i ispisuje rezultat poziva funkcije. U
slučaju pogrešnog unosa, ispisati poruku o grešci.
 
\begin{miditest}
\begin{upotreba}{1}
#\naslovInt#
#\izlaz{Unesite vrednosti:}\ulaz{4 1 7 5}#
#\izlaz{\ \ *}#
#\izlaz{\ \ *}#
#\izlaz{\ \ **}#
#\izlaz{*\ **}#
#\izlaz{*\ **}#
#\izlaz{*\ **}#
#\izlaz{****}#
\end{upotreba}
\end{miditest}
\begin{miditest}
\begin{upotreba}{2}
#\naslovInt#
#\izlaz{Unesite vrednosti:}\ulaz{8 -2 5 4}#
#\izlaz{Greska: pogresan unos!}#
\end{upotreba}
\end{miditest}

\begin{miditest}
\begin{upotreba}{3}
#\naslovInt#
#\izlaz{Unesite vrednosti:}\ulaz{5 2 2 4}#
#\izlaz{*}#
#\izlaz{*\ \ *}#
#\izlaz{*\ \ *}#
#\izlaz{****}#
#\izlaz{****}#
\end{upotreba}
\end{miditest}
\linkresenje{p1.4_06}
\end{Exercise}
\begin{Answer}[ref=p1.4_06]
\includecode{resenja/1_KontrolaToka/1.4_Funkcije/praktikumi8/4_06.c}
\end{Answer}


\begin{Exercise}[label=p1.4_07] 
Napisati funkciju \kckod{int prestupna(int godina)} koja za zadatu
godinu proverava da li je prestupna. Funkcija treba da vrati $1$ ako
je godina prestupna ili $0$ ako nije. Napisati program koji učitava
dva cela broja $g1$ i $g2$ i ispisuje sve godine iz intervala $[g1,
  g2]$ koje su prestupne.
 
\begin{miditest}
\begin{upotreba}{1}
#\naslovInt#
#\izlaz{Unesite dve godine:}\ulaz{2001 2010}#
#\izlaz{Prestupne godine su: 2004 2008}#
\end{upotreba}
\end{miditest}
\begin{miditest}
\begin{upotreba}{2}
#\naslovInt#
#\izlaz{Unesite dve godine:}\ulaz{2005 2015}#
#\izlaz{Prestupne godine su: 2008 2012}#
\end{upotreba}
\end{miditest}

\begin{miditest}
\begin{upotreba}{3}
#\naslovInt#
#\izlaz{Unesite dve godine:}\ulaz{2010 2001}#
#\izlaz{Greska: pogresan unos!}#
\end{upotreba}
\end{miditest}
\begin{miditest}
\begin{upotreba}{4}
#\naslovInt#
#\izlaz{Unesite dve godine:}\ulaz{2001 2002}#
#\izlaz{Nema prestupnih godina u ovom intervalu!}#
\end{upotreba}
\end{miditest}
\linkresenje{p1.4_07}
\end{Exercise}
\begin{Answer}[ref=p1.4_07]
\includecode{resenja/1_KontrolaToka/1.4_Funkcije/praktikumi8/4_07.c}
\end{Answer}


\begin{Exercise}[label=p1.4_] 
Napisati funkciju \kckod{int broj\_dana(int mesec, int godina)} koja
za dati mesec i godinu vraća broj dana u datom mesecu. Napisati
program koji učitava dva cela broja (mesec i godinu) i ispisuje broj
dana u datom mesecu. U slučaju nekorektnog unosa ispisati odgovarajuću
poruku o grešci.

\linkresenje{p1.4_}
\end{Exercise}
\begin{Answer}[ref=p1.4_]
\includecode{resenja/1_KontrolaToka/1.4_Funkcije/praktikumi8/4_08.c}
\end{Answer}

\begin{Exercise}[label=p1.4_] 
Napisati funkciju \kckod{int ispravan(int dan, int mesec, int godina)}
koja za dati datum proverava da li je ispravan. Napisati program koji
učitava tri cela broja (dan, mesec, godinu) i ispisuje da li je datum
ispravan ili ne.

\linkresenje{p1.4_}
\end{Exercise}
\begin{Answer}[ref=p1.4_]
%\includecode{resenja/1_KontrolaToka/1.4_Funkcije/praktikumi8/4_.c}
\end{Answer}


\begin{Exercise}[label=p1.4_] 
Napisati funkciju \kckod{void sledeci\_dan(int dan, int mesec, int
  godina)} koja za dati datum određuje datum sledećeg dana. Napisati
program koji učitava tri cela broja i ispisuje datum sledećeg dana. 

\linkresenje{p1.4_}
\end{Exercise}
\begin{Answer}[ref=p1.4_]
%\includecode{resenja/1_KontrolaToka/1.4_Funkcije/1_14.c}
\end{Answer}

\begin{Exercise}[label=p1.4_] 
%\komentar{preimenovala sam funkciju, preimenovati i u resenju}
Napisati funkciju \kckod{int od\_nove\_godine(int dan, int mesec, int
  godina)} koja određuje koliko je dana proteklo od Nove godine do
datog datuma. Napisati program koji učitava tri cela broja i ispisuje
koliko dana je proteklo od Nove godine.

\linkresenje{p1.4_}
\end{Exercise}
\begin{Answer}[ref=p1.4_]
%\includecode{resenja/1_KontrolaToka/1.4_Funkcije/1_14.c}
\end{Answer}

\begin{Exercise}[label=p1.4_] 
%\komentar{preimenovala sam funkciju, preimenovati i u resenju}
Napisati funkciju \kckod{int do\_kraja\_godine(int dan, int mesec, int
  godina)} koja određuje broj dana od datog datuma do kraja
godine. Napisati program koji učitava tri cela broja i ispisuje broj
dana do krja godine.
 
\linkresenje{p1.4_}
\end{Exercise}
\begin{Answer}[ref=p1.4_]
%\includecode{resenja/1_KontrolaToka/1.4_Funkcije/1_14.c}
\end{Answer}

\begin{Exercise}[label=p1.4_] 
%\komentar{preimenovala sam funkciju, preimenovati i u resenju}
Napisati funkciju \kckod{int broj\_dana\_između(int dan1, int mesec1,
  int godina1, int dan2, int mesec2, int godina2)} koja određuje broj
dana između dva datuma. Napisati program koji učitava šest celih
brojeva, koji označavaju dva datuma, i na standarni izlaz ispisuje
broj dana između ta dva datuma. U slučaju pogrešnog unosa ispisati
poruku o grešci.

\linkresenje{p1.4_}
\end{Exercise}
\begin{Answer}[ref=p1.4_]
%\includecode{resenja/1_KontrolaToka/1.4_Funkcije/1_14.c}
\end{Answer}


\begin{Exercise}[label=p1.4_08] 
 Napisati funkciju \kckod{int zbir\_delilaca(int n)} koja izračunava
 zbir delilaca broja $n$. Napisati program koji učitava ceo broj $k$ i
 ispisuje zbir delilaca svakog broja od 1 do $k$.
 
\begin{miditest}
\begin{upotreba}{1}
#\naslovInt#
#\izlaz{Unesite broj k:}\ulaz{6}#
#\izlaz{1 3 4 7 6 12}#
\end{upotreba}
\end{miditest}
\begin{miditest}
\begin{upotreba}{2}
#\naslovInt#
#\izlaz{Unesite broj k:}\ulaz{-2}#
#\izlaz{Greska: pogresan unos!}#
\end{upotreba}
\end{miditest}

\linkresenje{p1.4_08}
\end{Exercise}
\begin{Answer}[ref=p1.4_08]
\includecode{resenja/1_KontrolaToka/1.4_Funkcije/praktikumi8/4_08.c}
\end{Answer}


\begin{Exercise}[label=p1.4_09] 
Napisati funkciju \kckod{int ukloni\_stotine(int\ n)} koja modifikuje
zadati broj tako što iz njegovog zapisa uklanja cifru stotina (ako
postoji). Napisati program koji za brojeve koji se unose sve do pojave
broja $0$ ispisuje rezultat poziva funkcije.
 
\begin{miditest}
\begin{upotreba}{1}
#\naslovInt#
#\izlaz{Unesite broj:}\ulaz{1210}#
#\izlaz{110}#
#\izlaz{Unesite broj:}\ulaz{18}#
#\izlaz{18}#
#\izlaz{Unesite broj:}\ulaz{3856}#
#\izlaz{356}#
#\izlaz{Unesite broj:}\ulaz{0}#
\end{upotreba}
\end{miditest}
\begin{miditest}
\begin{upotreba}{2}
#\naslovInt#
#\izlaz{Unesite broj:}\ulaz{-9632}#
#\izlaz{-932}#
#\izlaz{Unesite broj:}\ulaz{246}#
#\izlaz{46}#
#\izlaz{Unesite broj:}\ulaz{-52}#
#\izlaz{-52}#
#\izlaz{Unesite broj:}\ulaz{0}#
\end{upotreba}
\end{miditest}


\linkresenje{p1.4_09}
\end{Exercise}
\begin{Answer}[ref=p1.4_09]
\includecode{resenja/1_KontrolaToka/1.4_Funkcije/praktikumi8/4_09.c}
\end{Answer}


\begin{Exercise}[label=p1.4_10] 
Napisati funkciju \kckod{int rotacija(int n)} koja rotira cifre
zadatog broja za jednu poziciju u levo. Napisati program koji za
brojeve koji se unose sve do pojave broja $0$ ispisuje rezultat poziva
funkcije.
 
\begin{miditest}
\begin{upotreba}{1}
#\naslovInt#
#\izlaz{Unesite broj:}\ulaz{146}#
#\izlaz{461}#
#\izlaz{Unesite broj:}\ulaz{18}#
#\izlaz{81}#
#\izlaz{Unesite broj:}\ulaz{3856}#
#\izlaz{8563}#
#\izlaz{Unesite broj:}\ulaz{7}#
#\izlaz{7}#
#\izlaz{Unesite broj:}\ulaz{0}#
\end{upotreba}
\end{miditest}
\begin{miditest}
\begin{upotreba}{2}
#\naslovInt#
#\izlaz{Unesite broj:}\ulaz{89}#
#\izlaz{98}#
#\izlaz{Unesite broj:}\ulaz{-369}#
#\izlaz{-693}#
#\izlaz{Unesite broj:}\ulaz{-55281}#
#\izlaz{-52815}#
#\izlaz{Unesite broj:}\ulaz{0}#
\end{upotreba}
\end{miditest}


\linkresenje{p1.4_10}
\end{Exercise}
\begin{Answer}[ref=p1.4_10]
\includecode{resenja/1_KontrolaToka/1.4_Funkcije/praktikumi8/4_10.c}
\end{Answer}


\begin{Exercise}[label=v1.4_08] 
Broj je prost ako je deljiv samo sa $1$ i sa samim sobom. Napisati
funkciju \kckod{int prost (int x)} koja ispituje da li je dati ceo
broj prost. Funkcija treba da vrati $1$ ako je broj prost i $0$ u
suprotnom. Testirati rad funkcije u programu koji za uneti ceo broj
$n$ ispisuje prvih $n$ prostih brojeva.

\begin{miditest}
\begin{upotreba}{1}
#\naslovInt#
#\izlaz{Unesite broj:}\ulaz{17}#
#\izlaz{Broj je prost!}#
\end{upotreba}
\end{miditest}

\begin{miditest}
\begin{upotreba}{2}
#\naslovInt#
#\izlaz{Unesite broj:}\ulaz{24}#
#\izlaz{Broj nije prost!}#
\end{upotreba}
\end{miditest}

\begin{miditest}
\begin{upotreba}{3}
#\naslovInt#
#\izlaz{Unesite broj:}\ulaz{-11}#
#\izlaz{Broj nije prost!}#
\end{upotreba}
\end{miditest}

\linkresenje{v1.4_08}
\end{Exercise}
\begin{Answer}[ref=v1.4_08]
\includecode{resenja/1_KontrolaToka/1.4_Funkcije/1_08.c}
\end{Answer}

\begin{Exercise}[label=v1.4_09] 
Napisati funkciju \kckod{int sadrzi(int x, int c)} koja ispituje da li
se cifra $c$ nalazi u zapisu celog broja $x$. Napisati program koji
učitava dva cela broja i ispisuje rezultat poziva funkcije.

\linkresenje{v1.4_09}
\end{Exercise}
\begin{Answer}[ref=v1.4_09]
\includecode{resenja/1_KontrolaToka/1.4_Funkcije/1_09.c}
\end{Answer}

\begin{Exercise}[label=v1.4_10] 
Napisati program za ispitivanje svojstava cifara datog celog broja.
\begin{enumerate}
\item Napisati funkciju \kckod{sve\_parne\_cifre} koja ispituje da li
  se dati ceo broj sastoji isključivo iz parnih cifara. Funkcija treba
  da vrati $1$ ako su sve cifre broja parne i $0$ u suprotnom.
\item Napisati funkciju \kckod{sve\_cifre\_jednake} koja ispituje da
  li su sve cifre datog celog broja jednake. Funkcija treba da vrati $1$
  ako su sve cifre broja jednake i $0$ u suprotnom.
\end{enumerate}
Napisati program koji učitava ceo broj i ispisuje da li su sve cifre
parne i da li su sve cifre jednake.

\linkresenje{v1.4_10}
\end{Exercise}
\begin{Answer}[ref=v1.4_10]
\includecode{resenja/1_KontrolaToka/1.4_Funkcije/1_10.c}
\end{Answer}

\begin{Exercise}[label=v1.4_11] 
Napisati funkciju \kckod{int je\_stepen(unsigned x, unsigned n)} koja
za dva broja $x$ i $n$ utvrđuje da li je $x$ neki stepen broja
$n$. Ukoliko jeste, funkcija vraća izložilac stepena, a u suprotnom
vraća $-1$. Napisati program koji učitava dva cela pozitivna broja i
ispisuje rezultat poziva funkcije.

\linkresenje{v1.4_11}
\end{Exercise}
\begin{Answer}[ref=v1.4_11]
\includecode{resenja/1_KontrolaToka/1.4_Funkcije/1_11.c}
\end{Answer}

\begin{Exercise}[label=v1.4_12] 
Napisati funkciju \kckod{double e\_na\_x(double x, double eps)} koja
računa vrednost $e^x$ kao parcijalnu sumu reda
$\sum_{n=0}^{\infty}\frac{x^n}{n!}$, pri čemu se sumiranje vrši dok je
razlika sabiraka u redu po apsolutnoj vrednosti manja od
$\varepsilon$. Napisati program koji učitava dva realna broja $x$ i
$eps$ i ispisuje izračunatu vrednost $e^x$.

\linkresenje{v1.4_12}
\end{Exercise}
\begin{Answer}[ref=v1.4_12]
\includecode{resenja/1_KontrolaToka/1.4_Funkcije/1_12.c}
\end{Answer}

\begin{Exercise}[label=v1.4_13] 
Za dati broj može se formirati niz tako da je svaki sledeći član niza
dobijen kao suma cifara prethodnog člana niza. Broj je \emph{srećan}
ako se dati niz završava jedinicom. Napisati funkciju \kckod{int
  srecan(int x)} koja vraća $1$ ako je broj srećan, a $0$ u
suprotnom. Napisati program koji za uneti prirodan broj $n$ ispisuje
sve srećne brojeve od 1 do $n$.

\linkresenje{v1.4_13}
\end{Exercise}
\begin{Answer}[ref=v1.4_13]
\includecode{resenja/1_KontrolaToka/1.4_Funkcije/1_13.c}
\end{Answer}

\begin{Exercise}[label=v1.4_14] 
Napisati funkciju \kckod{int konverzija (int c)} koja prebacuje veliko
slovo u ekvivalentno malo i obrnuto. Napisati program koji testira ovu
funkciju na karakterima koji se unose do pojave znaka \kckod{EOF}.

\linkresenje{v1.4_14}
\end{Exercise}
\begin{Answer}[ref=v1.4_14]
\includecode{resenja/1_KontrolaToka/1.4_Funkcije/1_14.c}
\end{Answer}


\begin{Exercise}[label=p1.4_] 
Napisati funkciju \kckod{void prosti\_brojevi(int m)} koja ispisuje
sve proste brojeve manje od broja $m$.  Napisati program koji učitava
ceo broj veći od $1$ i ispisati rezultat poziva funkcije. U slučaju
pogrešnog unosa, ispisati poruku o grešci. \napomena{Napomena:
  koristiti funkciju \kckod{int prost(int x}} napisanu u prethodnom
zadatku.}
 
\begin{miditest}
\begin{upotreba}{1}
#\naslovInt#
#\izlaz{Unesite broj:}\ulaz{15}#
#\izlaz{2 3 5 7 11 13}#
\end{upotreba}
\end{miditest}
\begin{miditest}
\begin{upotreba}{2}
#\naslovInt#
#\izlaz{Unesite broj:}\ulaz{9}#
#\izlaz{2 3 5 7}#
\end{upotreba}
\end{miditest}

\begin{miditest}
\begin{upotreba}{3}
#\naslovInt#
#\izlaz{Unesite broj:}\ulaz{1}#
#\izlaz{Greska: pogresan unos!}#
\end{upotreba}
\end{miditest}


\begin{Exercise}[label=p1.4_11] 
Napisati funkciju \kckod{float aritmeticka\_sredina(int n)} koja
računa aritmetičku sredinu cifara datog broja. Napisati i program koji
učitava ceo broj i ispisuje rezultat na tri decimale.
 
\begin{miditest}
\begin{upotreba}{1}
#\naslovInt#
#\izlaz{Unesite broj:}\ulaz{461}#
#\izlaz{3.667}#
\end{upotreba}
\end{miditest}
\begin{miditest}
\begin{upotreba}{2}
#\naslovInt#
#\izlaz{Unesite broj:}\ulaz{1001}#
#\izlaz{0.500}#
\end{upotreba}
\end{miditest}

\begin{miditest}
\begin{upotreba}{3}
#\naslovInt#
#\izlaz{Unesite broj:}\ulaz{-84723}#
#\izlaz{4.800}#
\end{upotreba}
\end{miditest}

\linkresenje{p1.4_11}
\end{Exercise}
\begin{Answer}[ref=p1.4_11]
\includecode{resenja/1_KontrolaToka/1.4_Funkcije/praktikumi9/4_11.c}
\end{Answer}
\end{comment}


\begin{Exercise}[label=p1.4_12] 
Napisati funkciju \kckod{int zapis(int x, int y)} koja proverava da li
se brojevi $x$ i $y$ zapisuju pomoću istih cifara. Funkcija treba da
vrati vrednost $1$ ako je uslov ispunjen, a $0$ ako nije. Napisati i
program koji učitava dva cela broja i ispisuje rezultat primene
funkcije.
 
\begin{miditest}
\begin{upotreba}{1}
#\naslovInt#
#\izlaz{Unesite dva broja:}\ulaz{251 125}#
#\izlaz{Uslov je ispunjen!}#
\end{upotreba}
\end{miditest}
\begin{miditest}
\begin{upotreba}{2}
#\naslovInt#
#\izlaz{Unesite dva broja:}\ulaz{8898 9988}#
#\izlaz{Uslov nije ispunjen!}#
\end{upotreba}
\end{miditest}

\begin{miditest}
\begin{upotreba}{3}
#\naslovInt#
#\izlaz{Unesite dva broja:}\ulaz{-7391 1397}#
#\izlaz{Uslov je ispunjen!}#
\end{upotreba}
\end{miditest} 

\linkresenje{p1.4_12}
\end{Exercise}
\begin{Answer}[ref=p1.4_12]
\includecode{resenja/1_KontrolaToka/1.4_Funkcije/praktikumi9/4_12.c}
\end{Answer}

\begin{Exercise}[label=p1.4_13] 
Napisati funkciju \kckod{int faktorijel(int n)} koja računa faktorijel
broja $n$. Napisati i program koji učitava dva cela broja $x$ i $y$ iz
intervala $[0,12]$ i ispisuje vrednost zbira $x!+y!$.
 
\begin{miditest}
\begin{upotreba}{1}
#\naslovInt#
#\izlaz{Unesite dva broja:}\ulaz{4 5}#
#\izlaz{144}#
\end{upotreba}
\end{miditest}
\begin{miditest}
\begin{upotreba}{2}
#\naslovInt#
#\izlaz{Unesite dva broja:}\ulaz{18 -5}#
#\izlaz{Greska: pogresan unos!}#
\end{upotreba}
\end{miditest}

\begin{miditest}
\begin{upotreba}{3}
#\naslovInt#
#\izlaz{Unesite dva broja:}\ulaz{6 0}#
#\izlaz{721}#
\end{upotreba}
\end{miditest}

\linkresenje{p1.4_13}
\end{Exercise}
\begin{Answer}[ref=p1.4_13]
\includecode{resenja/1_KontrolaToka/1.4_Funkcije/praktikumi9/4_13.c}
\end{Answer}

\begin{Exercise}[label=p1.4_14] 
Napisati funkciju \kckod{int rastuce(int n)} koja ispituje da li su
cifre datog celog broja u rastućem poretku. Funkcija treba da vrati
vrednost $1$ ako cifre ispunjavaju uslov, odnosno $0$ ako ne
ispunjavaju uslov. Napisati i program koji učitava ceo broj i ispisuje
rezultat poziva funkcije.

\begin{miditest}
\begin{upotreba}{1}
#\naslovInt#
#\izlaz{Unesite broj:}\ulaz{2689}#
#\izlaz{Cifre su u rastucem poretku!}#
\end{upotreba}
\end{miditest}
\begin{miditest}
\begin{upotreba}{2}
#\naslovInt#
#\izlaz{Unesite broj:}\ulaz{559}#
#\izlaz{Cifre su u rastucem poretku!}#
\end{upotreba}
\end{miditest}

\begin{miditest}
\begin{upotreba}{3}
#\naslovInt#
#\izlaz{Unesite broj:}\ulaz{628}#
#\izlaz{Cifre nisu u rastucem poretku!}#
\end{upotreba}
\end{miditest}

\linkresenje{p1.4_14}
\end{Exercise}
\begin{Answer}[ref=p1.4_14]
\includecode{resenja/1_KontrolaToka/1.4_Funkcije/praktikumi9/4_14.c}
\end{Answer}

\begin{Exercise}[label=p1.4_15]
%\komentarD{izbrisati Amstrongov broj u petljama} 
 Broj je Armstrongov ako je jednak sumi nekog stepena svojih cifara.
\begin{enumerate}
\item Napisati funkciju \kckod{int stepen(int x, int n)} koja
  izračunava $n$-ti stepen broja $x$.
\item Napisati funkciju \kckod{int armstrong(int x)} koja vraća $1$
  ako je broj Armstrongov, odnosno $0$ ako nije.
\end{enumerate}
Napisati program koji učitava ceo broj i proverava da li je
Armstrongov.

\begin{miditest}
\begin{upotreba}{1}
#\naslovInt#
#\izlaz{Unesite broj:}\ulaz{153}#
#\izlaz{Broj je Armstrongov!}#
\end{upotreba}
\end{miditest}
\begin{miditest}
\begin{upotreba}{2}
#\naslovInt#
#\izlaz{Unesite broj:}\ulaz{1634}#
#\izlaz{Broj je Armstrongov!}#
\end{upotreba}
\end{miditest}

\begin{miditest}
\begin{upotreba}{3}
#\naslovInt#
#\izlaz{Unesite broj:}\ulaz{118}#
#\izlaz{Broj nije Armstrongov!}#
\end{upotreba}
\end{miditest}

\linkresenje{p1.4_15}
\end{Exercise}
\begin{Answer}[ref=p1.4_15]
\includecode{resenja/1_KontrolaToka/1.4_Funkcije/praktikumi9/4_15.c}
\end{Answer}

\begin{Exercise}[label=p1.4_16] 
Napisati funkciju \kckod{int par\_nepar(int n)} koja ispituje da li su
cifre datog celog broja naizmenično parne i neparne. Funkcija treba da
vrati vrednost $1$ ako cifre ispunjavaju uslov, odnosno $0$ ako ne
ispunjavaju uslov. Napisati i program koji učitava ceo broj i testira
rad funkcije.
  
\begin{miditest}
\begin{upotreba}{1}
#\naslovInt#
#\izlaz{Unesite broj:}\ulaz{2749}#
#\izlaz{Broj ispunjava uslov!}#
\end{upotreba}
\end{miditest}
\begin{miditest}
\begin{upotreba}{2}
#\naslovInt#
#\izlaz{Unesite broj:}\ulaz{-963}#
#\izlaz{Broj ispunjava uslov!}#
\end{upotreba}
\end{miditest}

\begin{miditest}
\begin{upotreba}{3}
#\naslovInt#
#\izlaz{Unesite broj:}\ulaz{27449}#
#\izlaz{Broj ne ispunjava uslov!}#
\end{upotreba}
\end{miditest}

\linkresenje{p1.4_16}
\end{Exercise}
\begin{Answer}[ref=p1.4_16]
\includecode{resenja/1_KontrolaToka/1.4_Funkcije/praktikumi9/4_16.c}
\end{Answer}


\begin{Exercise}[label=p1.4_17] 
Napisati funkciju \kckod{int prebrojavanje(float x)} koja prebrojava
koliko puta se broj $x$ pojavljuje u nizu brojeva koji se unose sve do
pojave nule. Napisati program koji učitava vrednost broja $x$ i
testira rad napisane funkcije.

\begin{miditest}
\begin{upotreba}{1}
#\naslovInt#
#\izlaz{Unesite broj x:}\ulaz{2.84}#
#\izlaz{Unesite brojeve:}\ulaz{8.13 2.84 5 21.6 2.84 11.5 0}#
#\izlaz{Broj pojavljivanja broja 2.84 je: 2}#
\end{upotreba}
\end{miditest}
\begin{miditest}
\begin{upotreba}{2}
#\naslovInt#
#\izlaz{Unesite broj x:}\ulaz{-1.17}#
#\izlaz{Unesite brojeve:}\ulaz{-128.35 8.965 8.968 89.36 0}#
#\izlaz{Broj pojavljivanja broja -1.17 je: 0}#
\end{upotreba}
\end{miditest}

\linkresenje{p1.4_17}
\end{Exercise}
\begin{Answer}[ref=p1.4_17]
\includecode{resenja/1_KontrolaToka/1.4_Funkcije/praktikumi9/4_17.c}
\end{Answer}


\begin{Exercise}[label=p1.4_18]
%komentarD{Uskladiti sa Fibonacijevim u petljama} 
Napisati funkciju \kckod{long int fibonaci(int n)} koja računa $n$-ti
element Fibonačijevog niza. Napisati i program koji učitava ceo broj
$n\ (0\leq n\leq 50)$ i ispisuje traženi Fibonačijev broj.

\begin{miditest}
\begin{upotreba}{1}
#\naslovInt#
#\izlaz{Unesite broj n:}\ulaz{7}#
#\izlaz{21}#
\end{upotreba}
\end{miditest}
\begin{miditest}
\begin{upotreba}{2}
#\naslovInt#
#\izlaz{Unesite broj n:}\ulaz{65}#
#\izlaz{Greska: nedozvoljena vrednost!}#
\end{upotreba}
\end{miditest}

\linkresenje{p1.4_18}
\end{Exercise}
\begin{Answer}[ref=p1.4_18]
\includecode{resenja/1_KontrolaToka/1.4_Funkcije/praktikumi9/4_18.c}
\end{Answer}


\begin{Exercise}[label=p1.4_19] 
Napisati funkciju \kckod{char sifra(char c, int k)} koja za dati
karakter $c$ određuje šifru na sledeći način: ukoliko je $c$ slovo,
šifra je karakter koji se nalazi $k$ pozicija ispred njega u
abecedi. Karakteri koji nisu slova se ne šifruju. Šifrovanje treba da
bude kružno, što znači da je, na primer, šifra za karakter $b$ i
pomeraj $2$ karakter $z$. Napisati program koji učitava karakter po
karakter do kraja ulaza i ispisuje šifrovani tekst.

\begin{miditest}
\begin{upotreba}{1}
#\naslovInt#
#\izlaz{Unesite broj k:}\ulaz{2}#
#\izlaz{Unesite tekst (CTRL+D za prekid):}#
#\ulaz{c}#
#\izlaz{a}#
#\ulaz{8}#
#\izlaz{8}#
#\ulaz{+}#
#\izlaz{+}#
#\ulaz{Z}#
#\izlaz{X}#
\end{upotreba}
\end{miditest}
\linkresenje{p1.4_19}
\end{Exercise}
\begin{Answer}[ref=p1.4_19]
\includecode{resenja/1_KontrolaToka/1.4_Funkcije/praktikumi9/4_19.c}
\end{Answer}


\section{Rešenja}
\shipoutAnswer
