\section{Funkcije}

\begin{Exercise}[label=FUN_01] 
Napisati funkciju \kckod{int min(int x, int y, int z)} koja izračunava
minimum tri broja. Napisati program koji učitava tri cela broja i
ispisuje njihov minimum.

\begin{miditest}
\begin{upotreba}{1}
#\naslovInt#
#\izlaz{Unesite brojeve:}\ulaz{19 8 14}#
#\izlaz{Minimum: 8}#
\end{upotreba}
\end{miditest}
\begin{miditest}
\begin{upotreba}{2}
#\naslovInt#
#\izlaz{Unesite brojeve:}\ulaz{-6 11 -12}#
#\izlaz{Minimum: -12}#
\end{upotreba}
\end{miditest}
\linkresenje{FUN_01}
\end{Exercise}
\ifresenja 
\begin{Answer}[ref=FUN_01]
\includecode{resenja/2_KontrolaToka/1.4_Funkcije/sve/fun01.c}
\end{Answer} 
\fi

\begin{Exercise}[label=FUN_02] 
Napisati funkciju \kckod{float razlomljeni\_deo(float x)} koja
izračunava razlomljeni deo broja $x$. Napisati program koji učitava
jedan realan broj i ispisuje njegov razlomljeni deo na šest decimala.
 
\begin{miditest}
\begin{upotreba}{1}
#\naslovInt#
#\izlaz{Unesite broj:}\ulaz{8.235}#
#\izlaz{Razlomljeni deo: 0.235000}#
\end{upotreba}
\end{miditest}
\begin{miditest}
\begin{upotreba}{2}
#\naslovInt#
#\izlaz{Unesite broj:}\ulaz{-5.11}#
#\izlaz{Razlomljeni deo: 0.110000}#
\end{upotreba}
\end{miditest}
\linkresenje{FUN_02}
\end{Exercise}
\ifresenja 
\begin{Answer}[ref=FUN_02]
\includecode{resenja/2_KontrolaToka/1.4_Funkcije/sve/fun02.c}
\end{Answer} 
\fi

\begin{Exercise}[label=FUN_03] 
 Napisati funkciju \kckod{int zbir\_delilaca(int n)} koja izračunava
 zbir delilaca broja $n$. Napisati program koji učitava ceo pozitivan broj $k$ i
 ispisuje zbir delilaca svakog broja od 1 do $k$.
U slučaju neispravnog unosa, ispisati odgovarajuću poruku o grešci.

\begin{miditest}
\begin{upotreba}{1}
#\naslovInt#
#\izlaz{Unesite broj k:}\ulaz{6}#
#\izlaz{1 3 4 7 6 12}#
\end{upotreba}
\end{miditest}
\begin{miditest}
\begin{upotreba}{2}
#\naslovInt#
#\izlaz{Unesite broj k:}\ulaz{-2}#
#\izlaz{Greska: neispravan unos.}#
\end{upotreba}
\end{miditest}

\linkresenje{FUN_03}
\end{Exercise}
\ifresenja 
\begin{Answer}[ref=FUN_03]
\includecode{resenja/2_KontrolaToka/1.4_Funkcije/sve/fun03.c}
\end{Answer} 
\fi


\begin{Exercise}[label=FUN_04] 
Napisati funkciju \kckod{int je\_stepen(unsigned x, unsigned n)} koja
za dva broja $x$ i $n$ utvrđuje da li je $x$ neki stepen broja
$n$. Ukoliko jeste, funkcija vraća izložilac stepena, a u suprotnom
vraća $-1$. Napisati program koji učitava dva neoznačena broja i ispisuje
da li vrednost prvog broja odgovara vrednosti nekog stepena drugog broja.

\begin{miditest}
\begin{upotreba}{1}
#\naslovInt#
#\izlaz{Unesite dva broja:}\ulaz{81 3}#
#\izlaz{Jeste: 81 = 3\^\ 4}#
\end{upotreba}
\end{miditest}
\begin{miditest}
\begin{upotreba}{2}
#\naslovInt#
#\izlaz{Unesite dva broja:}\ulaz{162 11}#
#\izlaz{Broj 162 nije stepen broja 11.}#
\end{upotreba}
\end{miditest}

\linkresenje{FUN_04}
\end{Exercise}
\ifresenja 
\begin{Answer}[ref=FUN_04]
\includecode{resenja/2_KontrolaToka/1.4_Funkcije/sve/fun04.c}
\end{Answer} 
\fi


\begin{Exercise}[label=FUN_05] 
Napisati funkciju \kckod{int euklid(int x, int y)} koja za dva data
cela broja određuje najvećeg zajedničkog delioca primenom Euklidovog
algoritma. Napisati program koji učitava dva cela broja i ispisuje
vrednost njihovog najvećeg zajedničkog delioca.

\begin{miditest}
\begin{upotreba}{1}
#\naslovInt#
#\izlaz{Unesite dva cela broja:}\ulaz{1024 832}#
#\izlaz{Najveci zajednicki delilac: 64}#
\end{upotreba}
\end{miditest}
\begin{miditest}
\begin{upotreba}{2}
#\naslovInt#
#\izlaz{Unesite dva cela broja:}\ulaz{-900 112}#
#\izlaz{Najveci zajednicki delilac: 4}#
\end{upotreba}
\end{miditest}

\linkresenje{FUN_05}
\end{Exercise}
\ifresenja 
\begin{Answer}[ref=FUN_05]
\includecode{resenja/2_KontrolaToka/1.4_Funkcije/sve/fun05.c}
\end{Answer} 
\fi


\begin{Exercise}[label=FUN_06] 
Napisati funkciju \kckod{float zbir\_reciprocnih(int n)} koja za dato
$n$ vraća zbir recipročnih vrednosti brojeva od $1$ do $n$. Napisati
program koji učitava ceo pozitivan broj $n$ i ispisuje odgovarajući zbir
zaokružen na dve decimale. 
U slučaju neispravnog unosa, ispisati odgovarajuću poruku o grešci.

\begin{minitest}
\begin{upotreba}{1}
#\naslovInt#
#\izlaz{Unesite broj n:}\ulaz{10}#
#\izlaz{Zbir reciprocnih: 2.93}#
\end{upotreba}
\end{minitest}
\begin{minitest}
\begin{upotreba}{2}
#\naslovInt#
#\izlaz{Unesite broj n:}\ulaz{100}#
#\izlaz{Zbir reciprocnih: 5.19}#
\end{upotreba}
\end{minitest}
\begin{minitest}
\begin{upotreba}{3}
#\naslovInt#
#\izlaz{Unesite broj n:}\ulaz{-100}#
#\izlaz{Greska: neispravan unos.}#
\end{upotreba}
\end{minitest}

\linkresenje{FUN_06}
\end{Exercise}
\ifresenja 
\begin{Answer}[ref=FUN_06]
\includecode{resenja/2_KontrolaToka/1.4_Funkcije/sve/fun06.c}
\end{Answer} 
\fi


\begin{Exercise}[label=FUN_07] 
Napisati funkciju \kckod{int prebrojavanje(float x)} koja prebrojava
koliko puta se broj $x$ pojavljuje u nizu brojeva koji se unose sve do
unosa broja nula. Napisati program koji učitava vrednost broja $x$ i
ispisuje koliko puta se njegova vrednost pojavila u unetom nizu.

\begin{miditest}
\begin{upotreba}{1}
#\naslovInt#
#\izlaz{Unesite broj x:}\ulaz{2.84}#
#\izlaz{Unesite brojeve:}#
#\ulaz{8.13 2.84 5 21.6 2.84 11.5 0}#
#\izlaz{Broj pojavljivanja broja 2.84: 2}#
\end{upotreba}
\end{miditest}
\begin{miditest}
\begin{upotreba}{2}
#\naslovInt#
#\izlaz{Unesite broj x:}\ulaz{-1.17}#
#\izlaz{Unesite brojeve:}#
#\ulaz{-128.35 8.965 8.968 89.36 0}#
#\izlaz{Broj pojavljivanja broja -1.17: 0}#
\end{upotreba}
\end{miditest}

\linkresenje{FUN_07}
\end{Exercise}
\ifresenja 
\begin{Answer}[ref=FUN_07]
\includecode{resenja/2_KontrolaToka/1.4_Funkcije/sve/fun07.c}
\end{Answer} 
\fi


\begin{Exercise}[label=FUN_08] 
Broj je prost ako je deljiv samo sa $1$ i sa samim sobom. 
\begin{itemize}
 \item [a)] Napisati funkciju \kckod{int prost(int x)} koja ispituje da li je dati ceo
  broj prost. Funkcija treba da vrati jedinicu ako je broj prost i nulu u suprotnom.
 \item [b)] Napisati funkciju \kckod{void prvih\_n\_prostih(int n)} koja ispisuje prvih $n$ prostih brojeva.
 \item [c)] Napisati funkciju \kckod{void prosti\_brojevi\_manji\_od\_n(int n)} koja ispisuje
sve proste brojeve manje od broja $n$.
\end{itemize}
Napisati program koji učitava pozitivan ceo broj $n$ i ispisuje prvih $n$ prostih brojeva, kao i sve proste brojeve manje od $n$.
U slučaju neispravnog unosa, ispisati odgovarajuću poruku o grešci.

\begin{miditest}
\begin{upotreba}{1}
#\naslovInt#
#\izlaz{Unesite broj n:}\ulaz{5}#
#\izlaz{Prvih n prostih: 2 3 5 7 11}#
#\izlaz{Prosti manji od n: 2 3}#
\end{upotreba}
\end{miditest}
\begin{miditest}
\begin{upotreba}{2}
#\naslovInt#
#\izlaz{Unesite broj n:}\ulaz{2}#
#\izlaz{Prvih n prostih: 2 3}#
#\izlaz{Prosti manji od n: ne postoje}#
\end{upotreba}
\end{miditest}

\begin{miditest}
\begin{upotreba}{3}
#\naslovInt#
#\izlaz{Unesite broj n:}\ulaz{-11}#
#\izlaz{Greska: neispravan unos. }#
\end{upotreba}
\end{miditest}

\linkresenje{FUN_08}
\end{Exercise}
\ifresenja 
\begin{Answer}[ref=FUN_08]
\includecode{resenja/2_KontrolaToka/1.4_Funkcije/sve/fun08.c}
\end{Answer} 
\fi


\begin{Exercise}[label=FUN_OLD_1] 
Rešiti sledeće zadatke korišćenjem funkcija.
\begin{itemize}
 \item [a)] Zadatak \ref{UZ_NI_02} rešiti korišćenjem funkcija \kckod{int kvadrat(int x)} koja računa kvadrat datog broja i \kckod{int kub(int x)} koja računa kub datog broja.
 \item [b)] Zadatak \ref{KT_NG_02} rešiti korišćenjem funkcije \kckod{float apsolutna\_vrednost(float x)} koja izračunava apsolutnu vrednost datog broja.
 \item [c)] Zadatak \ref{PET_07} rešiti korišćenjem funkcije \kckod{float stepen(float x, int n)} koja računa vrednost $n$-tog stepena realnog broja $x$.
 \item [d)] Zadatak \ref{PET_31} rešiti korišćenjem funkcije \kckod{int fibonaci(int n)} koja računa $n$-ti element Fibonačijevog niza.
\end{itemize}
\end{Exercise}

%%%%%%%%%%%%%%%%%%%%%%%%%%%%%%%%
%%%%%% CIFRE
%%%%%%%%%%%%%%%%%%%%%%%%%%%%%%%%


\begin{Exercise}[label=FUN_10] 
Napisati funkciju \kckod{float aritmeticka\_sredina(int n)} koja
računa aritmetičku sredinu cifara datog broja. Napisati i program koji
učitava ceo broj i ispisuje aritmetičku sredinju njegovih cifara
zaokruženu na tri decimale.
 
\begin{minitest}
\begin{upotreba}{1}
#\naslovInt#
#\izlaz{Unesite broj:}\ulaz{461}#
#\izlaz{3.667}#
\end{upotreba}
\end{minitest}
\begin{minitest}
\begin{upotreba}{2}
#\naslovInt#
#\izlaz{Unesite broj:}\ulaz{1001}#
#\izlaz{0.500}#
\end{upotreba}
\end{minitest}
\begin{minitest}
\begin{upotreba}{3}
#\naslovInt#
#\izlaz{Unesite broj:}\ulaz{-84723}#
#\izlaz{4.800}#
\end{upotreba}
\end{minitest}

\linkresenje{FUN_10}
\end{Exercise}
\ifresenja 
\begin{Answer}[ref=FUN_10]
\includecode{resenja/2_KontrolaToka/1.4_Funkcije/sve/fun10.c}
\end{Answer} 
\fi


\begin{Exercise}[label=FUN_11] 
Napisati funkciju \kckod{int sadrzi(int x, int c)} koja ispituje da li
se cifra $c$ nalazi u zapisu celog broja $x$. Funkcija treba da vrati
jedinicu ako se cifra nalazi u broju, a nulu inače. Napisati program koji
učitava jedan ceo broj i jednu cifru i u zavisnosti od toga da li se 
uneta cifra nalazi u zapisu unetog broja, ispisuje odgovarajuću poruku.
U slučaju neispravnog unosa, ispisati odgovarajuću poruku o grešci.

\begin{miditest}
\begin{upotreba}{1}
#\naslovInt#
#\izlaz{Unesite broj i cifru:}\ulaz{17890 7}#
#\izlaz{Cifra 7 se nalazi u zapisu broja 17890.}#
\end{upotreba}
\end{miditest}
\begin{miditest}
\begin{upotreba}{2}
#\naslovInt#
#\izlaz{Unesite broj i cifru:}\ulaz{19 6}#
#\izlaz{Cifra 6 se ne nalazi u zapisu broja 19.}#
\end{upotreba}
\end{miditest}

\begin{miditest}
\begin{upotreba}{3}
#\naslovInt#
#\izlaz{Unesite broj i cifru:}\ulaz{17890 26}#
#\izlaz{Greska: neispravan unos.}#
\end{upotreba}
\end{miditest}
\begin{miditest}
\begin{upotreba}{4}
#\naslovInt#
#\izlaz{Unesite broj i cifru:}\ulaz{-1982 9}#
#\izlaz{Cifra 9 se nalazi u zapisu broja -1982.}#
\end{upotreba}
\end{miditest}

\linkresenje{FUN_11}
\end{Exercise}
\ifresenja 
\begin{Answer}[ref=FUN_11]
\includecode{resenja/2_KontrolaToka/1.4_Funkcije/sve/fun11.c}
\end{Answer} 
\fi


\begin{Exercise}[label=FUN_12] 
Napisati funkciju \kckod{int broj\_neparnih\_cifara(int x)} koja
određuje broj neparnih cifara u zapisu datog celog broja. Napisati
program koji učitava cele brojeve sve do unosa broja nula
i ispisuje broj neparnih cifara svakog unetog broja.

\begin{miditest}
\begin{upotreba}{1}
#\naslovInt#
#\izlaz{Unesite cele brojeve:}#
#\ulaz{2341}# 
#\izlaz{Broj neparnih cifara: 2}#
#\ulaz{78}# 
#\izlaz{Broj neparnih cifara: 1}#
#\ulaz{800}# 
#\izlaz{Broj neparnih cifara: 0}#
#\ulaz{-99761}# 
#\izlaz{Broj neparnih cifara: 4}#
#\ulaz{0}# 
\end{upotreba}
\end{miditest}
\begin{miditest}
\begin{upotreba}{2}
#\naslovInt#
#\izlaz{Unesite cele brojeve:}#
#\ulaz{987611}#
#\izlaz{Broj neparnih cifara: 4}#
#\ulaz{135 }#
#\izlaz{Broj neparnih cifara: 3}#
#\ulaz{-701}#
#\izlaz{Broj neparnih cifara: 2}#
#\ulaz{602}#
#\izlaz{Broj neparnih cifara: 0}#
#\ulaz{-884}#
#\izlaz{Broj neparnih cifara: 0}#
#\ulaz{79901}#
#\izlaz{Broj neparnih cifara: 4}#
#\ulaz{0}#
\end{upotreba}
\end{miditest}
\linkresenje{FUN_12}
\end{Exercise}
\ifresenja 
\begin{Answer}[ref=FUN_12]
\includecode{resenja/2_KontrolaToka/1.4_Funkcije/sve/fun12.c}
\end{Answer} 
\fi


\begin{Exercise}[label=FUN_13] 
Napisati program za ispitivanje svojstava cifara datog celog broja.
\begin{enumerate}
\item Napisati funkciju \kckod{int sve\_parne\_cifre(int x)} koja ispituje da li
  se dati ceo broj sastoji isključivo iz parnih cifara. Funkcija treba
  da vrati jedinicu ako su sve cifre broja parne, a nulu inače.
\item Napisati funkciju \kckod{int sve\_cifre\_jednake(int x)} koja ispituje da
  li su sve cifre datog celog broja jednake. Funkcija treba da vrati jedinicu
  ako su sve cifre broja jednake, a nulu inače.
\end{enumerate}
Program učitava ceo broj i u zavisnosti od toga da li su navedena svojstva ispunjena ili ne,
ispisuje odgovarajuću poruku.

\begin{miditest}
\begin{upotreba}{1}
#\naslovInt#
#\izlaz{Unesite broj:}\ulaz{86422}#
#\izlaz{Sve cifre broja su parne.}#
#\izlaz{Cifre broja nisu jednake.}#
\end{upotreba}
\end{miditest}
\begin{miditest}
\begin{upotreba}{2}
#\naslovInt#
#\izlaz{Unesite broj:}\ulaz{55555}#
#\izlaz{Broj sadrzi bar jednu neparnu cifru.}#
#\izlaz{Cifre broja su jednake.}#
\end{upotreba}
\end{miditest}

\begin{miditest}
\begin{upotreba}{3}
#\naslovInt#
#\izlaz{Unesite broj:}\ulaz{-88}#
#\izlaz{Sve cifre broja su parne.}#
#\izlaz{Cifre broja su jednake.}#
\end{upotreba}
\end{miditest}
\begin{miditest}
\begin{upotreba}{4}
#\naslovInt#
#\izlaz{Unesite broj i cifru:}\ulaz{-342}#
#\izlaz{Broj sadrzi bar jednu neparnu cifru.}#
#\izlaz{Cifre broja nisu jednake.}#
\end{upotreba}
\end{miditest}

\linkresenje{FUN_13}
\end{Exercise}
\ifresenja 
\begin{Answer}[ref=FUN_13]
\includecode{resenja/2_KontrolaToka/1.4_Funkcije/sve/fun13.c}
\end{Answer} 
\fi


\begin{Exercise}[label=FUN_14] 
Napisati funkciju \kckod{int ukloni(int n, int p)} koja menja
broj $n$ tako što iz njegovog zapisa uklanja cifru na poziciji p.
Pozicije se broje sa desna na levo. Cifra jedinica ima poziciju $1$.
Napisati program koji učitava redni broj pozicije i zatim 
za cele brojeve koji se unose sve do unosa broja nula,
ispisuje brojeve kojima je uklonjena cifra na poziciji p.
U slučaju neispravnog unosa, ispisati odgovarajuću poruku o grešci.

\begin{minitest}
\begin{upotreba}{1}
#\naslovInt#
#\izlaz{Unesite poziciju:}\ulaz{3}#
#\izlaz{Unesite broj:}\ulaz{1210}#
#\izlaz{110}#
#\izlaz{Unesite broj:}\ulaz{18}#
#\izlaz{18}#
#\izlaz{Unesite broj:}\ulaz{3856}#
#\izlaz{356}#
#\izlaz{Unesite broj:}\ulaz{0}#
\end{upotreba}
\end{minitest}
\begin{minitest}
\begin{upotreba}{2}
#\naslovInt#
#\izlaz{Unesite poziciju:}\ulaz{1}#
#\izlaz{Unesite broj:}\ulaz{-9632}#
#\izlaz{-963}#
#\izlaz{Unesite broj:}\ulaz{-2}#
#\izlaz{0}#
#\izlaz{Unesite broj:}\ulaz{246}#
#\izlaz{24}#
#\izlaz{Unesite broj:}\ulaz{-52}#
#\izlaz{-5}#
#\izlaz{Unesite broj:}\ulaz{0}#
\end{upotreba}
\end{minitest}
\begin{minitest}
\begin{upotreba}{3}
#\naslovInt#
#\izlaz{Unesite poziciju:}\ulaz{0}#
#\izlaz{Greska: neispravan unos.}#
\end{upotreba}
\end{minitest}

\linkresenje{FUN_14}
\end{Exercise}
\ifresenja 
\begin{Answer}[ref=FUN_14]
\includecode{resenja/2_KontrolaToka/1.4_Funkcije/sve/fun14.c}
\end{Answer} 
\fi


\begin{Exercise}[label=FUN_15] 
Napisati funkciju \kckod{int zapis(int x, int y)} koja proverava da li
se brojevi $x$ i $y$ zapisuju pomoću istih cifara. Funkcija treba da
vrati jedinicu ako je uslov ispunjen, a nulu inače. Napisati
program koji učitava dva cela broja i ispisuje da li je za njih 
pomenuti uslov ispunjen ili ne.

\begin{miditest}
\begin{upotreba}{1}
#\naslovInt#
#\izlaz{Unesite dva broja:}\ulaz{251 125}#
#\izlaz{Uslov je ispunjen.}#
\end{upotreba}
\end{miditest}
\begin{miditest}
\begin{upotreba}{2}
#\naslovInt#
#\izlaz{Unesite dva broja:}\ulaz{8898 9988}#
#\izlaz{Uslov nije ispunjen.}#
\end{upotreba}
\end{miditest}

\begin{miditest}
\begin{upotreba}{3}
#\naslovInt#
#\izlaz{Unesite dva broja:}\ulaz{-7391 1397}#
#\izlaz{Uslov je ispunjen.}#
\end{upotreba}
\end{miditest}
\begin{miditest}
\begin{upotreba}{4}
#\naslovInt#
#\izlaz{Unesite dva broja:}\ulaz{-777 77}#
#\izlaz{Uslov nije ispunjen.}#
\end{upotreba}
\end{miditest} 

\linkresenje{FUN_15}
\end{Exercise}
\ifresenja 
\begin{Answer}[ref=FUN_15]
\includecode{resenja/2_KontrolaToka/1.4_Funkcije/sve/fun15.c}
\end{Answer} 
\fi


\begin{Exercise}[label=FUN_16] 
Napisati funkciju \kckod{int neopadajuce(int n)} koja ispituje da li su
cifre datog celog broja u neopadajućem poretku. Funkcija treba da vrati
jedinicu ako cifre ispunjavaju uslov, a nulu inače. 
Napisati program koji učitava ceo broj i ispisuje
poruku da li su cifre unetog broja u neopadajućem poretku.

\begin{miditest}
\begin{upotreba}{1}
#\naslovInt#
#\izlaz{Unesite broj:}\ulaz{2289}#
#\izlaz{Cifre su u neopadajucem poretku.}#
\end{upotreba}
\end{miditest}
\begin{miditest}
\begin{upotreba}{2}
#\naslovInt#
#\izlaz{Unesite broj:}\ulaz{5}#
#\izlaz{Cifre su u neopadajucem poretku.}#
\end{upotreba}
\end{miditest}

\begin{miditest}
\begin{upotreba}{3}
#\naslovInt#
#\izlaz{Unesite broj:}\ulaz{6628}#
#\izlaz{Cifre nisu u neopadajucem poretku.}#
\end{upotreba}
\end{miditest}
\begin{miditest}
\begin{upotreba}{4}
#\naslovInt#
#\izlaz{Unesite broj:}\ulaz{-23}#
#\izlaz{Cifre su u neopadajucem poretku.}#
\end{upotreba}
\end{miditest}

\linkresenje{FUN_16}
\end{Exercise}
\ifresenja 
\begin{Answer}[ref=FUN_16]
\includecode{resenja/2_KontrolaToka/1.4_Funkcije/sve/fun16.c}
\end{Answer} 
\fi


\begin{Exercise}[label=FUN_17] 
Napisati funkciju \kckod{int par\_nepar(int n)} koja ispituje da li su
cifre datog celog broja naizmenično parne i neparne. Funkcija treba da
vrati jedinicu ako cifre ispunjavaju uslov, a nulu inače. 
Napisati program koji učitava ceo broj i ispisuje da li on ispunjava
pomenuti uslov ili ne.
  
\begin{miditest}
\begin{upotreba}{1}
#\naslovInt#
#\izlaz{Unesite broj n:}\ulaz{2749}#
#\izlaz{Broj ispunjava uslov.}#
\end{upotreba}
\end{miditest}
\begin{miditest}
\begin{upotreba}{2}
#\naslovInt#
#\izlaz{Unesite broj n:}\ulaz{-963}#
#\izlaz{Broj ispunjava uslov.}#
\end{upotreba}
\end{miditest}

\begin{miditest}
\begin{upotreba}{3}
#\naslovInt#
#\izlaz{Unesite broj n:}\ulaz{27449}#
#\izlaz{Broj ne ispunjava uslov.}#
\end{upotreba}
\end{miditest}
\begin{miditest}
\begin{upotreba}{4}
#\naslovInt#
#\izlaz{Unesite broj n:}\ulaz{5}#
#\izlaz{Broj ispunjava uslov.}#
\end{upotreba}
\end{miditest}

\linkresenje{FUN_17}
\end{Exercise}
\ifresenja 
\begin{Answer}[ref=FUN_17]
\includecode{resenja/2_KontrolaToka/1.4_Funkcije/sve/fun17.c}
\end{Answer} 
\fi


\begin{Exercise}[label=FUN_18] 
Napisati funkciju \kckod{int rotacija(int n)} koja rotira cifre
zadatog celog broja za jednu poziciju u levo. Napisati program koji za
brojeve koji se unose sve do unosa broja nula ispisuje odgovarajuće
rotirane brojeve.
 
\begin{miditest}
\begin{upotreba}{1}
#\naslovInt#
#\izlaz{Unesite broj:}\ulaz{146}#
#\izlaz{461}#
#\izlaz{Unesite broj:}\ulaz{18}#
#\izlaz{81}#
#\izlaz{Unesite broj:}\ulaz{3856}#
#\izlaz{8563}#
#\izlaz{Unesite broj:}\ulaz{7}#
#\izlaz{7}#
#\izlaz{Unesite broj:}\ulaz{0}#
\end{upotreba}
\end{miditest}
\begin{miditest}
\begin{upotreba}{2}
#\naslovInt#
#\izlaz{Unesite broj:}\ulaz{89}#
#\izlaz{98}#
#\izlaz{Unesite broj:}\ulaz{-369}#
#\izlaz{-693}#
#\izlaz{Unesite broj:}\ulaz{-55281}#
#\izlaz{-52815}#
#\izlaz{Unesite broj:}\ulaz{0}#
\end{upotreba}
\end{miditest}


\linkresenje{FUN_18}
\end{Exercise}
\ifresenja 
\begin{Answer}[ref=FUN_18]
\includecode{resenja/2_KontrolaToka/1.4_Funkcije/sve/fun18.c}
\end{Answer} 
\fi


\begin{Exercise}[label=FUN_19] 
Za dati broj može se formirati niz tako da je svaki sledeći član niza
dobijen kao suma cifara prethodnog člana niza. Broj je \emph{srećan}
ako se dati niz završava jedinicom. Napisati funkciju \kckod{int
  srecan(int x)} koja vraća jedinicu ako je broj srećan, a nulu inače. 
  Napisati program koji za uneti pozitivan ceo broj $n$ ispisuje
sve srećne brojeve od 1 do $n$.
U slučaju neispravnog unosa, ispisati odgovarajuću poruku o grešci. 

\begin{miditest}
\begin{upotreba}{1}
#\naslovInt#
#\izlaz{Unesite broj n:}\ulaz{100}#
#\izlaz{Srecni brojevi:}#
#\izlaz{1 10 19 28 37 46 55 64 73 82 91 100}#
\end{upotreba}
\end{miditest}
\begin{miditest}
\begin{upotreba}{2}
#\naslovInt#
#\izlaz{Unesite broj n:}\ulaz{0}#
#\izlaz{Greska: neispravan unos.}#
\end{upotreba}
\end{miditest}

\linkresenje{FUN_19}
\end{Exercise}
\ifresenja 
\begin{Answer}[ref=FUN_19]
\includecode{resenja/2_KontrolaToka/1.4_Funkcije/sve/fun19.c}
\end{Answer} 
\fi


\begin{Exercise}[label=FUN_20]
Prirodan broj $a$ je Armstrongov ako je jednak sumi $n$-tih stepena svojih
cifara, pri čemu je $n$ broj cifara broja $a$.  Napisati funkciju
\kckod{int armstrong(int x)} koja vraća jedinicu ako je broj Armstrongov, a
nulu inače. Napisati program koji za učitani pozitivan ceo broj
proverava da li je Armstrongov.
U slučaju neispravnog unosa, ispisati odgovarajuću poruku o grešci. 

\begin{miditest}
\begin{upotreba}{1}
#\naslovInt#
#\izlaz{Unesite broj:}\ulaz{153}#
#\izlaz{Broj je Armstrongov.}#
\end{upotreba}
\end{miditest}
\begin{miditest}
\begin{upotreba}{2}
#\naslovInt#
#\izlaz{Unesite broj:}\ulaz{1634}#
#\izlaz{Broj je Armstrongov.}#
\end{upotreba}
\end{miditest}

\begin{miditest}
\begin{upotreba}{3}
#\naslovInt#
#\izlaz{Unesite broj:}\ulaz{118}#
#\izlaz{Broj nije Armstrongov.}#
\end{upotreba}
\end{miditest}
\begin{miditest}
\begin{upotreba}{4}
#\naslovInt#
#\izlaz{Unesite broj:}\ulaz{0}#
#\izlaz{Greska: neispravan unos.}#
\end{upotreba}
\end{miditest}

\linkresenje{FUN_20}
\end{Exercise}
\ifresenja 
\begin{Answer}[ref=FUN_20]
\includecode{resenja/2_KontrolaToka/1.4_Funkcije/sve/fun20.c}
\end{Answer} 
\fi


%%%%%%%%%%%%%%%%%%
%%% MATH
%%%%%%%%%%%%%%%%%%

\begin{Exercise}[label=FUN_21] 
Napisati funkciju \kckod{double e\_na\_x(double x, double eps)} koja
računa vrednost $e^x$ kao parcijalnu sumu reda
$\sum_{n=0}^{\infty}\frac{x^n}{n!}$, pri čemu se sumiranje vrši dok je
razlika sabiraka u redu po apsolutnoj vrednosti manja od
$\varepsilon$. Napisati program koji učitava dva realna broja $x$ i
$eps$ i ispisuje izračunatu vrednost $e^x$.

\begin{miditest}
\begin{upotreba}{1}
#\naslovInt#
#\izlaz{Unesite broj x:}\ulaz{5}#
#\izlaz{Unesite eps:}\ulaz{0.001}#
#\izlaz{Rezultat: 148.412951}#
\end{upotreba}
\end{miditest}
\begin{miditest}
\begin{upotreba}{2}
#\naslovInt#
#\izlaz{Unesite broj x:}\ulaz{-3}#
#\izlaz{Unesite eps:}\ulaz{0.0001}#
#\izlaz{Rezultat: 0.049796}#
\end{upotreba}
\end{miditest}
\linkresenje{FUN_21}
\end{Exercise}
\ifresenja 
\begin{Answer}[ref=FUN_21]
\includecode{resenja/2_KontrolaToka/1.4_Funkcije/sve/fun21.c}
\end{Answer} 
\fi


\begin{Exercise}[label=FUN_22] 
Napisati funkciju \kckod{void ispis(float x, float y, int n)}
koja za dva realna broja $x$ i $y$ i jedan pozitivan ceo broj $n$
ispisuje vrednosti sinusne funkcije u $n$ ravnomerno raspoređenih
tačaka intervala $[x,y]$.  Napisati program koji učitava granice intervala
i broj tačaka i ispisuje odgovarajuće vrednosti sinusne funkcije, zaokružene na
četiri decimale.
U slučaju neispravnog unosa, ispisati odgovarajuću poruku o grešci. 

\begin{miditest}
\begin{upotreba}{1}
#\naslovInt#
#\izlaz{Unesite dva realna broja:}\ulaz{7 32}#
#\izlaz{Unesite broj n:}\ulaz{10}#
#\izlaz{0.6570 -0.3457 -0.0108 0.3659 -0.6731}#
#\izlaz{0.8922 -0.9945 0.9666 -0.8122}#
\end{upotreba}
\end{miditest}
\begin{miditest}
\begin{upotreba}{2}
#\naslovInt#
#\izlaz{Unesite dva realna broja:}\ulaz{20.5 -8.32}#
#\izlaz{Unesite broj n:}\ulaz{5}#
#\izlaz{-0.8934 -0.8979 -0.1920 0.6658 0.9968}#
\end{upotreba}
\end{miditest}

\begin{miditest}
\begin{upotreba}{3}
#\naslovInt#
#\izlaz{Unesite dva realna broja:}\ulaz{8 8}#
#\izlaz{Greska: neispravan unos.}
\end{upotreba}
\end{miditest}
\begin{miditest}
\begin{upotreba}{1}
#\naslovInt#
#\izlaz{Unesite dva realna broja:}\ulaz{7 32}#
#\izlaz{Unesite broj n:}\ulaz{-10}#
#\izlaz{Greska: neispravan unos.}#
\end{upotreba}
\end{miditest}


\linkresenje{FUN_22}
\end{Exercise}
\ifresenja 
\begin{Answer}[ref=FUN_22]
\includecode{resenja/2_KontrolaToka/1.4_Funkcije/sve/fun22.c}
\end{Answer} 
\fi

%%%%%%%%%%%%%%%%%%
%%% CHAR
%%%%%%%%%%%%%%%%%%


\begin{Exercise}[label=FUN_23] 
Napisati funkciju \kckod{char sifra(char c, int k)} koja za dati
karakter $c$ određuje šifru na sledeći način: ukoliko je $c$ slovo,
šifra je karakter koji se nalazi $k$ pozicija pre njega u
abecedi. Karakteri koji nisu slova se ne šifruju. Šifrovanje treba da
bude kružno, što znači da je, na primer, šifra za karakter $b$ i
pomeraj $2$ karakter $z$. Napisati program koji učitava nenegativan
ceo broj $k$, a zatim i karaktere sve do kraja ulaza i 
nakon svakog učitanog karaktera ispisuje njegovu šifru.
U slučaju neispravnog unosa, ispisati odgovarajuću poruku o grešci. 

\begin{miditest}
\begin{upotreba}{1}
#\naslovInt#
#\izlaz{Unesite broj k:}\ulaz{2}#
#\izlaz{Unesite tekst (CTRL+D za prekid):}#
#\ulaz{c}#
#\izlaz{a}#
#\ulaz{8}#
#\izlaz{8}#
#\ulaz{+}#
#\izlaz{+}#
#\ulaz{Z}#
#\izlaz{X}#
\end{upotreba}
\end{miditest}
\begin{miditest}
\begin{upotreba}{2}
#\naslovInt#
#\izlaz{Unesite broj k:}\ulaz{-2}#
#\izlaz{Greska: neispravan unos.}#
\end{upotreba}
\end{miditest}

\linkresenje{FUN_23}
\end{Exercise}
\ifresenja 
\begin{Answer}[ref=FUN_23]
\includecode{resenja/2_KontrolaToka/1.4_Funkcije/sve/fun23.c}
\end{Answer} 
\fi

\begin{Exercise}[label=FUN_OLD_2] 
Rešiti sledeće zadatke korišćenjem funkcija.
\begin{itemize}
 \item [a)] Zadatak \ref{PET_34} rešiti korišćenjem funkcije \kckod{char konverzija(char c)} koja malo slovo pretvara u odgovarajuće veliko i obrnuto.
 \item [b)] Zadatak \ref{PET_35} rešiti korišćenjem funkcije  \kckod{void prebrojavanje()} koja učitava karaktere sve do kraja ulaza i ispisuje broj malih slova, 
 velikih slova, cifara, belina, kao i sumu svih unetih cifara.
\end{itemize}
\end{Exercise}

%%%%%%%%%%%%%%%%%%
%%% GODINE
%%%%%%%%%%%%%%%%%%

\begin{Exercise}[label=FUN_25] 
Napisati program koji učitava tri cela broja i ispisuje datum sledećeg dana. 
Zadatak rešiti korišćenjem narednih funkcija.
\begin{itemize}
 \item [a)] \kckod{int prestupna(int godina)} koja za zadatu
godinu proverava da li je prestupna. Funkcija treba da vrati jedinicu ako
je godina prestupna ili nulu ako nije.
 \item [b)] \kckod{int broj\_dana(int mesec, int godina)} koja
za dati mesec i godinu vraća broj dana u datom mesecu.
 \item [c)] \kckod{int ispravan(int dan, int mesec, int godina)}
koja za dati datum proverava da li je ispravan.
 \item [d)] \kckod{void sledeci\_dan(int dan, int mesec, int
  godina)} koja za dati datum ispisuje datum sledećeg dana.
\end{itemize}
U slučaju neispravnog unosa, ispisati odgovarajuću poruku o grešci. 

\begin{miditest}
\begin{upotreba}{1}
#\naslovInt#
#\izlaz{Unesite datum:}\ulaz{24.8.1998.}#
#\izlaz{Datum sledeceg dana je: 25.8.1998.}#
\end{upotreba}
\end{miditest}
\begin{miditest}
\begin{upotreba}{2}
#\naslovInt#
#\izlaz{Unesite datum:}\ulaz{31.12.1789.}#
#\izlaz{Datum sledeceg dana je: 1.1.1790.}#
\end{upotreba}
\end{miditest}

\begin{miditest}
\begin{upotreba}{3}
#\naslovInt#
#\izlaz{Unesite datum:}\ulaz{28.2.2003.}#
#\izlaz{Datum sledeceg dana je: 1.3.2004.}#
\end{upotreba}
\end{miditest}
\begin{miditest}
\begin{upotreba}{4}
#\naslovInt#
#\izlaz{Unesite datum:}\ulaz{31.4.2004.}#
#\izlaz{Greska: neispravan unos.}#
\end{upotreba}
\end{miditest}

\linkresenje{FUN_25}
\end{Exercise}
\ifresenja 
\begin{Answer}[ref=FUN_25]
\includecode{resenja/2_KontrolaToka/1.4_Funkcije/sve/fun25.c}
\end{Answer} 
\fi


\begin{Exercise}[label=FUN_26] 
Napisati funkciju \kckod{int od\_nove\_godine(int dan, int mesec, int
  godina)} koja određuje koliko je dana proteklo od Nove godine do
datog datuma. Napisati program koji učitava tri cela broja i ispisuje
koliko dana je proteklo od Nove godine. 
U slučaju neispravnog unosa, ispisati odgovarajuću poruku o grešci. 

\begin{miditest}
\begin{upotreba}{1}
#\naslovInt#
#\izlaz{Unesite datum:}\ulaz{24.8.1998.}#
#\izlaz{Broj dana od Nove godine je: 235}#
\end{upotreba}
\end{miditest}
\begin{miditest}
\begin{upotreba}{2}
#\naslovInt#
#\izlaz{Unesite datum:}\ulaz{31.12.1680.}#
#\izlaz{Broj dana od Nove godine je: 366}#
\end{upotreba}
\end{miditest}

\begin{miditest}
\begin{upotreba}{3}
#\naslovInt#
#\izlaz{Unesite datum:}\ulaz{28.2.2003.}#
#\izlaz{Broj dana od Nove godine je: 58}#
\end{upotreba}
\end{miditest}
\begin{miditest}
\begin{upotreba}{4}
#\naslovInt#
#\izlaz{Unesite datum:}\ulaz{31.4.2004.}#
#\izlaz{Greska: neispravan unos.}#
\end{upotreba}
\end{miditest}

\linkresenje{FUN_26}
\end{Exercise}
\ifresenja 
\begin{Answer}[ref=FUN_26]

Za rešavanje ovog zadatka koristi se funkcija \kckod{od\_nove\_godine} koja je definisana u rešenju zadatka \ref{FUN_28}.
\end{Answer} 
\fi


\begin{Exercise}[label=FUN_27] 
Napisati funkciju \kckod{int do\_kraja\_godine(int dan, int mesec, int
  godina)} koja određuje broj dana od datog datuma do kraja
godine. Napisati program koji učitava tri cela broja i ispisuje broj
dana do krja godine. 
U slučaju neispravnog unosa, ispisati odgovarajuću poruku o grešci.

\begin{miditest}
\begin{upotreba}{1}
#\naslovInt#
#\izlaz{Unesite datum:}\ulaz{24.8.1998.}#
#\izlaz{Broj dana do Nove godine je: 129}#
\end{upotreba}
\end{miditest}
\begin{miditest}
\begin{upotreba}{2}
#\naslovInt#
#\izlaz{Unesite datum:}\ulaz{31.12.1680.}#
#\izlaz{Broj dana do Nove godine je: 0}#
\end{upotreba}
\end{miditest}

\begin{miditest}
\begin{upotreba}{3}
#\naslovInt#
#\izlaz{Unesite datum:}\ulaz{28.2.2004.}#
#\izlaz{Broj dana do Nove godine je: 307}#
\end{upotreba}
\end{miditest}
\begin{miditest}
\begin{upotreba}{4}
#\naslovInt#
#\izlaz{Unesite datum:}\ulaz{31.4.2004.}#
#\izlaz{Greska: neispravan unos.}#
\end{upotreba}
\end{miditest}

\linkresenje{FUN_27}
\end{Exercise}
\ifresenja 
\begin{Answer}[ref=FUN_27]

Za rešavanje ovog zadatka koristi se funkcija \kckod{do\_kraja\_godine} koja je definisana u rešenju zadatka \ref{FUN_28}.
\end{Answer} 
\fi

\begin{Exercise}[label=FUN_28] 
Napisati funkciju \kckod{int broj\_dana\_izmedju(int dan1, int mesec1,
  int godina1, int dan2, int mesec2, int godina2)} koja određuje broj
dana između dva datuma. Napisati program koji učitava dva datuma u formatu
\kckod{dd.mm.gggg} i na standarni izlaz ispisuje
broj dana između ta dva datuma.
U slučaju neispravnog unosa, ispisati odgovarajuću poruku o grešci. 

\begin{miditest}
\begin{upotreba}{1}
#\naslovInt#
#\izlaz{Unesite prvi datum:}\ulaz{12.3.2008.}#
#\izlaz{Unesite drugi datum:}\ulaz{5.12.2008.}#
#\izlaz{Broj dana izmedju dva datuma je: 268}#
\end{upotreba}
\end{miditest}
\begin{miditest}
\begin{upotreba}{2}
#\naslovInt#
#\izlaz{Unesite prvi datum:}\ulaz{26.9.1986.}#
#\izlaz{Unesite drugi datum:}\ulaz{2.2.1701.}#
#\izlaz{Broj dana izmedju dva datuma je: 104301}#
\end{upotreba}
\end{miditest}

\begin{miditest}
\begin{upotreba}{3}
#\naslovInt#
#\izlaz{Unesite prvi datum:}\ulaz{24.8.1998.}#
#\izlaz{Unesite drugi datum:}\ulaz{12.10.2010.}#
#\izlaz{Broj dana izmedju dva datuma je: 4440}#
\end{upotreba}
\end{miditest}
\begin{miditest}
\begin{upotreba}{4}
#\naslovInt#
#\izlaz{Unesite prvi datum:}\ulaz{24.8.1998.}#
#\izlaz{Unesite drugi datum:}\ulaz{31.4.2004.}#
#\izlaz{Greska: neispravan unos.}#
\end{upotreba}
\end{miditest}

\linkresenje{FUN_28}
\end{Exercise}
\ifresenja 
\begin{Answer}[ref=FUN_28]
\includecode{resenja/2_KontrolaToka/1.4_Funkcije/sve/fun28.c}
\end{Answer} 
\fi


%%%%%%%%%%%%%%%%%%
%%%% ZVEZDICE
%%%%%%%%%%%%%%%%%%

\begin{Exercise}[label=FUN_29] 
Napisati funkciju \kckod{void romb(int n)} koja iscrtava romb čija je
stranica dužine $n$. Napisati program koji učitava ceo pozitivan broj
i iscrtava odgovarajuću sliku.
U slučaju neispravnog unosa, ispisati odgovarajuću poruku o grešci. 
 
\begin{minitest}
\begin{upotreba}{1}
#\naslovInt#
#\izlaz{Unesite broj n:}\ulaz{5}#
#\izlaz{\ \ \ \ *****}#
#\izlaz{\ \ \ *****}#
#\izlaz{\ \ *****}#
#\izlaz{\ *****}#
#\izlaz{*****}#
\end{upotreba}
\end{minitest}
\begin{minitest}
\begin{upotreba}{2}
#\naslovInt#
#\izlaz{Unesite broj n:}\ulaz{2}#
#\izlaz{\ **}#
#\izlaz{**}#
\end{upotreba}
\end{minitest}
\begin{minitest}
\begin{upotreba}{3}
#\naslovInt#
#\izlaz{Unesite broj n:}\ulaz{-5}#
#\izlaz{Greska: neispravan unos.}#
\end{upotreba}
\end{minitest}
\linkresenje{FUN_29}
\end{Exercise}
\ifresenja 
\begin{Answer}[ref=FUN_29]
\includecode{resenja/2_KontrolaToka/1.4_Funkcije/sve/fun29.c}
\end{Answer} 
\fi


\begin{Exercise}[label=FUN_30] 
Napisati funkciju \kckod{void grafikon\_h(int a, int b, int c, int d)}
koja iscrtava horizontalni prikaz zadatih vrednosti. Napisati program
koji učitava četiri pozitivna cela broja i iscrtava odgovarajuću sliku.
U slučaju neispravnog unosa, ispisati odgovarajuću poruku o grešci. 
 
\begin{minitest}
\begin{upotreba}{1}
#\naslovInt#
#\izlaz{Unesite brojeve:}\ulaz{4 1 7 5}#
#\izlaz{****}#
#\izlaz{*}#
#\izlaz{*******}#
#\izlaz{*****}#
\end{upotreba}
\end{minitest}
\begin{minitest}
\begin{upotreba}{2}
#\naslovInt#
#\izlaz{Unesite brojeve:}\ulaz{5 2 2 10}#
#\izlaz{*****}#
#\izlaz{**}#
#\izlaz{**}#
#\izlaz{**********}#
\end{upotreba}
\end{minitest}
\begin{minitest}
\begin{upotreba}{3}
#\naslovInt#
#\izlaz{Unesite brojeve:}\ulaz{8 -2 5 4}#
#\izlaz{Greska: neispravan unos.}#
\end{upotreba}
\end{minitest}

\linkresenje{FUN_30}
\end{Exercise}
\ifresenja 
\begin{Answer}[ref=FUN_30]
\includecode{resenja/2_KontrolaToka/1.4_Funkcije/sve/fun30.c}
\end{Answer} 
\fi


\begin{Exercise}[label=FUN_31] 
Napisati funkciju
\kckod{void grafikon\_v(int a, int b, int c, int d)} koja iscrtava
vertikalni prikaz zadatih vrednosti. Napisati program koji učitava
četiri pozitivna cela broja i iscrtava odgovarajuću sliku.
U slučaju neispravnog unosa, ispisati odgovarajuću poruku o grešci. 
 
\begin{minitest}
\begin{upotreba}{1}
#\naslovInt#
#\izlaz{Unesite brojeve:}\ulaz{4 1 7 5}#
#\izlaz{\ \ *}#
#\izlaz{\ \ *}#
#\izlaz{\ \ **}#
#\izlaz{*\ **}#
#\izlaz{*\ **}#
#\izlaz{*\ **}#
#\izlaz{****}#
\end{upotreba}
\end{minitest}
\begin{minitest}
\begin{upotreba}{2}
#\naslovInt#
#\izlaz{Unesite brojeve:}\ulaz{5 2 2 4}#
#\izlaz{*}#
#\izlaz{*\ \ *}#
#\izlaz{*\ \ *}#
#\izlaz{****}#
#\izlaz{****}#
\end{upotreba}
\end{minitest}
\begin{minitest}
\begin{upotreba}{3}
#\naslovInt#
#\izlaz{Unesite brojeve:}\ulaz{8 -2 5 4}#
#\izlaz{Greska: neispravan unos.}#
\end{upotreba}
\end{minitest}
\linkresenje{FUN_31}
\end{Exercise}
\ifresenja 
\begin{Answer}[ref=FUN_31]
\includecode{resenja/2_KontrolaToka/1.4_Funkcije/sve/fun31.c}
\end{Answer} 
\fi


\ifresenja
\section{Rešenja}
\shipoutAnswer
\fi
