\section{Funkcije}



\begin{Exercise}[label=v1.4_01] 
Tekst
\linkresenje{v1.4_01}
\end{Exercise}
\begin{Answer}[ref=v1.4_01]
\includecode{resenja/1_KontrolaToka/1.4_Funkcije/1_01.c}
\end{Answer}

\begin{Exercise}[label=v1.4_02] 
Tekst
\linkresenje{v1.4_02}
\end{Exercise}
\begin{Answer}[ref=v1.4_02]
\includecode{resenja/1_KontrolaToka/1.4_Funkcije/1_02.c}
\end{Answer}

\begin{Exercise}[label=v1.4_03] 
Tekst
\linkresenje{v1.4_03}
\end{Exercise}
\begin{Answer}[ref=v1.4_03]
\includecode{resenja/1_KontrolaToka/1.4_Funkcije/1_03.c}
\end{Answer}

\begin{Exercise}[label=v1.4_04] 
Tekst
\linkresenje{v1.4_04}
\end{Exercise}
\begin{Answer}[ref=v1.4_04]
\includecode{resenja/1_KontrolaToka/1.4_Funkcije/1_04.c}
\end{Answer}

\begin{Exercise}[label=v1.4_05] 
Tekst
\linkresenje{v1.4_05}
\end{Exercise}
\begin{Answer}[ref=v1.4_05]
\includecode{resenja/1_KontrolaToka/1.4_Funkcije/1_05.c}
\end{Answer}

\begin{Exercise}[label=v1.4_06] 
Tekst
\linkresenje{v1.4_06}
\end{Exercise}
\begin{Answer}[ref=v1.4_06]
\includecode{resenja/1_KontrolaToka/1.4_Funkcije/1_06.c}
\end{Answer}

\begin{Exercise}[label=v1.4_07] 
Tekst
\linkresenje{v1.4_07}
\end{Exercise}
\begin{Answer}[ref=v1.4_07]
\includecode{resenja/1_KontrolaToka/1.4_Funkcije/1_07.c}
\end{Answer}

\begin{Exercise}[label=v1.4_08] 
Tekst
\linkresenje{v1.4_08}
\end{Exercise}
\begin{Answer}[ref=v1.4_08]
\includecode{resenja/1_KontrolaToka/1.4_Funkcije/1_08.c}
\end{Answer}

\begin{Exercise}[label=v1.4_09] 
Tekst
\linkresenje{v1.4_09}
\end{Exercise}
\begin{Answer}[ref=v1.4_09]
\includecode{resenja/1_KontrolaToka/1.4_Funkcije/1_09.c}
\end{Answer}

\begin{Exercise}[label=v1.4_10] 
Tekst
\linkresenje{v1.4_10}
\end{Exercise}
\begin{Answer}[ref=v1.4_10]
\includecode{resenja/1_KontrolaToka/1.4_Funkcije/1_10.c}
\end{Answer}

\begin{Exercise}[label=v1.4_11] 
Tekst
\linkresenje{v1.4_11}
\end{Exercise}
\begin{Answer}[ref=v1.4_11]
\includecode{resenja/1_KontrolaToka/1.4_Funkcije/1_11.c}
\end{Answer}

\begin{Exercise}[label=v1.4_12] 
Tekst
\linkresenje{v1.4_12}
\end{Exercise}
\begin{Answer}[ref=v1.4_12]
\includecode{resenja/1_KontrolaToka/1.4_Funkcije/1_12.c}
\end{Answer}

\begin{Exercise}[label=v1.4_13] 
Tekst
\linkresenje{v1.4_13}
\end{Exercise}
\begin{Answer}[ref=v1.4_13]
\includecode{resenja/1_KontrolaToka/1.4_Funkcije/1_13.c}
\end{Answer}

\begin{Exercise}[label=v1.4_14] 
Tekst
\linkresenje{v1.4_14}
\end{Exercise}
\begin{Answer}[ref=v1.4_14]
\includecode{resenja/1_KontrolaToka/1.4_Funkcije/1_14.c}
\end{Answer}



\begin{Exercise}[label=p1.4_] 
Napisati funkciju $int\ min(int\ x, int\ y, int\ z)$ koja izračunava minimun tri broja. Napisati program koji sa standardnog ulaza učitava tri cela broja i ispisuje rezultat poziva funkcije. \\
\begin{miditest}
\begin{upotreba}{1}
#\naslovInt#
#\izlaz{Unesite brojeve:}\ulaz{19 8 14}#
#\izlaz{Minimum je: 8}#
\end{upotreba}
\end{miditest}
\begin{miditest}
\begin{upotreba}{2}
#\naslovInt#
#\izlaz{Unesite brojeve:}\ulaz{-6 11 -12}#
#\izlaz{Minimum je: -12}#
\end{upotreba}
\end{miditest}
\linkresenje{p1.4_}
\end{Exercise}
\begin{Answer}[ref=p1.4_]
%\includecode{resenja/1_KontrolaToka/1.4_Funkcije/1_14.c}
\end{Answer}

\begin{Exercise}[label=p1.4_] 
Napisati funkciju $unsigned\ int\ apsolutna_vrednost(int\ x)$ koja izračunava apsolutnu vrednost broja $x$. Napisati program koji sa standardnog ulaza učitava jedan ceo broj i ispisuje rezultat poziva funkcije.  \\
\begin{miditest}
\begin{upotreba}{1}
#\naslovInt#
#\izlaz{Unesite broj:}\ulaz{-34}#
#\izlaz{Apsolutna vrednost: 34}#
\end{upotreba}
\end{miditest}
\begin{miditest}
\begin{upotreba}{2}
#\naslovInt#
#\izlaz{Unesite broj:}\ulaz{5}#
#\izlaz{Apsolutna vrednost: 5}#
\end{upotreba}
\end{miditest}

\linkresenje{p1.4_}
\end{Exercise}
\begin{Answer}[ref=p1.4_]
%\includecode{resenja/1_KontrolaToka/1.4_Funkcije/1_14.c}
\end{Answer}

\begin{Exercise}[label=p1.4_] 
 Napisati funkciju $float\ razlomljeni\_deo(float\ x)$ koja izračunava razlomljeni deo broja $x$. Napisati program koji sa standardnog ulaza učitava jedan realan broj i ispisuje rezultat poziva funkcije.  \\
\begin{miditest}
\begin{upotreba}{1}
#\naslovInt#
#\izlaz{Unesite broj:}\ulaz{8.235}#
#\izlaz{Razlomljeni deo: 0.235000}#
\end{upotreba}
\end{miditest}
\begin{miditest}
\begin{upotreba}{2}
#\naslovInt#
#\izlaz{Unesite broj:}\ulaz{-5.11}#
#\izlaz{Razlomljeni deo: 0.110000}#
\end{upotreba}
\end{miditest}

\iffalse
%koristi se na vezbama u VIII nedelji
\item Napisati funkciju $int\ jednake_cifre(int\ x)$ koja proverava da li su sve cifre iz zapisa broj $x$ jednake. Funkcija treba da vrati broj $1$ ako je uslov ispunjen ili $0$ ako nije. Zatim sa standardnog ulaza učitati jedan četvorocifren broj i ispisati rezultat poziva funkcije. U slučaju pogrešnog unosa, ispisati poruku o grešci. \\
\begin{miditest}
\begin{upotreba}{1}
#\naslovInt#
#\izlaz{Unesite broj:}\ulaz{2854}#
#\izlaz{Cifre u zapisu nisu jednake!}#
\end{upotreba}
\end{miditest}
\begin{miditest}
\begin{upotreba}{2}
#\naslovInt#
#\izlaz{Unesite broj:}\ulaz{5555}#
#\izlaz{Cifre u zapisu su jednake!}#
\end{upotreba}
\end{miditest}
\begin{miditest}
\begin{upotreba}{3}
#\naslovInt#
#\izlaz{Unesite broj:}\ulaz{214}#
#\izlaz{Greska: pogresan unos!}#
\end{upotreba}
\end{miditest}

\linkresenje{p1.4_}
\end{Exercise}
\begin{Answer}[ref=p1.4_]
%\includecode{resenja/1_KontrolaToka/1.4_Funkcije/1_14.c}
\end{Answer}

\begin{Exercise}[label=p1.4_] 
 Napisati funkciju $int\ prost(int\ x)$ koja proverava da li je zadati broj prost. Funkcija treba da vrati $1$ ako je broj prost ili $0$ ako nije. Broj $x$ je prost ako je deljiv samo brojevima $1$ i $x$. Zatim sa standardnog ulaza učitati jedan ceo broj veći od 1 i ispisati rezultat poziva funkcije. U slučaju pogrešnog unosa, ispisati poruku o grešci.\\
\begin{miditest}
\begin{upotreba}{1}
#\naslovInt#
#\izlaz{Unesite broj:}\ulaz{17}#
#\izlaz{Broj je prost!}#
\end{upotreba}
\end{miditest}
\begin{miditest}
\begin{upotreba}{2}
#\naslovInt#
#\izlaz{Unesite broj:}\ulaz{24}#
#\izlaz{Broj nije prost!}#
\end{upotreba}
\end{miditest}
\begin{miditest}
\begin{upotreba}{3}
#\naslovInt#
#\izlaz{Unesite broj:}\ulaz{-11}#
#\izlaz{Greska: pogresan unos!}#
\end{upotreba}
\end{miditest}

\linkresenje{p1.4_}
\end{Exercise}
\begin{Answer}[ref=p1.4_]
%\includecode{resenja/1_KontrolaToka/1.4_Funkcije/1_14.c}
\end{Answer}

\begin{Exercise}[label=p1.4_] 
 Napisati funkciju $void\ prosti_brojevi(int\ m)$ koja ispisuje sve proste brojeve manje od broja $m$.  Zatim sa standardnog ulaza učitati jedan ceo broj veći od 1 i ispisati rezultat poziva funkcije. U slučaju pogrešnog unosa, ispisati poruku o grešci. \textit{Napomena: koristiti funkciju napisanu u prethodnom zadatku.} \\
\begin{miditest}
\begin{upotreba}{1}
#\naslovInt#
#\izlaz{Unesite broj:}\ulaz{15}#
#\izlaz{2 3 5 7 11 13}#
\end{upotreba}
\end{miditest}
\begin{miditest}
\begin{upotreba}{2}
#\naslovInt#
#\izlaz{Unesite broj:}\ulaz{9}#
#\izlaz{2 3 5 7}#
\end{upotreba}
\end{miditest}
\begin{miditest}
\begin{upotreba}{3}
#\naslovInt#
#\izlaz{Unesite broj:}\ulaz{1}#
#\izlaz{Greska: pogresan unos!}#
\end{upotreba}
\end{miditest}
\fi

\linkresenje{p1.4_}
\end{Exercise}
\begin{Answer}[ref=p1.4_]
%\includecode{resenja/1_KontrolaToka/1.4_Funkcije/1_14.c}
\end{Answer}

\begin{Exercise}[label=p1.4_] 
 Napisati funkciju $void\ romb(int\ n)$ koja iscrtava romb čija je stranica dužine $n$. Napisati program koji učitava ceo pozitivan broj i ispisuje rezultat poziva funkcije. U slučaju pogrešnog unosa, ispisati poruku o grešci.\\
\begin{miditest}
\begin{upotreba}{1}
#\naslovInt#
#\izlaz{Unesite broj n:}\ulaz{5}#
#\izlaz{\ \ \ \ *****}#
#\izlaz{\ \ \ *****}#
#\izlaz{\ \ *****}#
#\izlaz{\ *****}#
#\izlaz{*****}#
\end{upotreba}
\end{miditest}
\begin{miditest}
\begin{upotreba}{2}
#\naslovInt#
#\izlaz{Unesite broj n:}\ulaz{2}#
#\izlaz{\ **}#
#\izlaz{**}#
\end{upotreba}
\end{miditest}
\begin{miditest}
\begin{upotreba}{3}
#\naslovInt#
#\izlaz{Unesite broj n:}\ulaz{-5}#
#\izlaz{Greska: pogresna dimenzija!}#
\end{upotreba}
\end{miditest}


\linkresenje{p1.4_}
\end{Exercise}
\begin{Answer}[ref=p1.4_]
%\includecode{resenja/1_KontrolaToka/1.4_Funkcije/1_14.c}
\end{Answer}

\begin{Exercise}[label=p1.4_] 
 Napisati funkciju $void\ grafikon\_h(int\ a,\ int\ b,\ int\ c,\ int\ d)$ koja vrši horizontalno prikazivanje zadatih vrednosti. Napisati program koji učitava četiri pozitivna cela broja i prikazuje rezultat poziva funkcije. U slučaju pogrešnog unosa, ispisati poruku o grešci. \\
\begin{miditest}
\begin{upotreba}{1}
#\naslovInt#
#\izlaz{Unesite vrednosti:}\ulaz{4 1 7 5}#
#\izlaz{****}#
#\izlaz{*}#
#\izlaz{*******}#
#\izlaz{*****}#
\end{upotreba}
\end{miditest}
\begin{miditest}
\begin{upotreba}{2}
#\naslovInt#
#\izlaz{Unesite vrednosti:}\ulaz{8 -2 5 4}#
#\izlaz{Greska: pogresan unos!}#
\end{upotreba}
\end{miditest}
\begin{miditest}
\begin{upotreba}{3}
#\naslovInt#
#\izlaz{Unesite vrednosti:}\ulaz{5 2 2 10}#
#\izlaz{*****}#
#\izlaz{**}#
#\izlaz{**}#
#\izlaz{**********}#
\end{upotreba}
\end{miditest}


\linkresenje{p1.4_}
\end{Exercise}
\begin{Answer}[ref=p1.4_]
%\includecode{resenja/1_KontrolaToka/1.4_Funkcije/1_14.c}
\end{Answer}

\begin{Exercise}[label=p1.4_] 
 Napisati funkciju $void\ grafikon\_v(int\ a,\ int\ b,\ int\ c,\ int\ d)$ koja vrši vertikalno prikazivanje zadatih vrednosti. Napisati program koji učitava četiri pozitivna cela broja i ispisuje rezultat poziva funkcije. U slučaju pogrešnog unosa, ispisati poruku o grešci. \\
\begin{miditest}
\begin{upotreba}{1}
#\naslovInt#
#\izlaz{Unesite vrednosti:}\ulaz{4 1 7 5}#
#\izlaz{\ \ *}#
#\izlaz{\ \ *}#
#\izlaz{\ \ **}#
#\izlaz{*\ **}#
#\izlaz{*\ **}#
#\izlaz{*\ **}#
#\izlaz{****}#
\end{upotreba}
\end{miditest}
\begin{miditest}
\begin{upotreba}{2}
#\naslovInt#
#\izlaz{Unesite vrednosti:}\ulaz{8 -2 5 4}#
#\izlaz{Greska: pogresan unos!}#
\end{upotreba}
\end{miditest}
\begin{miditest}
\begin{upotreba}{3}
#\naslovInt#
#\izlaz{Unesite vrednosti:}\ulaz{5 2 2 4}#
#\izlaz{*}#
#\izlaz{*\ \ *}#
#\izlaz{*\ \ *}#
#\izlaz{****}#
#\izlaz{****}#
\end{upotreba}
\end{miditest}

\linkresenje{p1.4_}
\end{Exercise}
\begin{Answer}[ref=p1.4_]
%\includecode{resenja/1_KontrolaToka/1.4_Funkcije/1_14.c}
\end{Answer}

\begin{Exercise}[label=p1.4_] 
 Napisati funkciju $int\ prestupna(int\ godina)$ koja za zadatu godinu proverava da li je prestupna. Funkcija treba da vrati $1$ ako je godina prestupna ili $0$ ako nije. Napisati program koji učitava dva cela broja $g1$ i $g2$ i ispisuje sve godine iz intervala $[g1, g2]$ koje su prestupne.\\
\begin{miditest}
\begin{upotreba}{1}
#\naslovInt#
#\izlaz{Unesite dve godine:}\ulaz{2001 2010}#
#\izlaz{Prestupne godine su: 2004 2008}#
\end{upotreba}
\end{miditest}
\begin{miditest}
\begin{upotreba}{2}
#\naslovInt#
#\izlaz{Unesite dve godine:}\ulaz{2005 2015}#
#\izlaz{Prestupne godine su: 2008 2012}#
\end{upotreba}
\end{miditest}
\begin{miditest}
\begin{upotreba}{3}
#\naslovInt#
#\izlaz{Unesite dve godine:}\ulaz{2010 2001}#
#\izlaz{Greska: pogresan unos!}#
\end{upotreba}
\end{miditest}

\begin{miditest}
\begin{upotreba}{4}
#\naslovInt#
#\izlaz{Unesite dve godine:}\ulaz{2001 2002}#
#\izlaz{Nema prestupnih godina u ovom intervalu!}#
\end{upotreba}
\end{miditest}


\linkresenje{p1.4_}
\end{Exercise}
\begin{Answer}[ref=p1.4_]
%\includecode{resenja/1_KontrolaToka/1.4_Funkcije/1_14.c}
\end{Answer}

\begin{Exercise}[label=p1.4_] 
 Napisati funkciju $int\ zbir\_delilaca(int\ n)$ koja izračunava zbir delilaca broja $n$. Napisati program koji sa standardnog ulaza učitava ceo broj $k$ i ispisuje zbir delilaca svakog broja od 1 do $k$. \\ 
\begin{miditest}
\begin{upotreba}{1}
#\naslovInt#
#\izlaz{Unesite broj k:}\ulaz{6}#
#\izlaz{1 3 4 7 6 12}#
\end{upotreba}
\end{miditest}
\begin{miditest}
\begin{upotreba}{2}
#\naslovInt#
#\izlaz{Unesite broj k:}\ulaz{-2}#
#\izlaz{Greska: pogresan unos!}#
\end{upotreba}
\end{miditest}

\linkresenje{p1.4_}
\end{Exercise}
\begin{Answer}[ref=p1.4_]
%\includecode{resenja/1_KontrolaToka/1.4_Funkcije/1_14.c}
\end{Answer}

\begin{Exercise}[label=p1.4_] 
 Napisati funkciju $int\ ukloni\_stotine(int\ n)$ koja modifikuje zadati broj tako što iz njegovog zapisa uklanja cifru stotina (ako postoji). Napisati program koji za brojeve koji se unose sa standardnog ulaza sve do pojave broja 0 ispisuje rezultat primene funkcije. \\
\begin{miditest}
\begin{upotreba}{1}
#\naslovInt#
#\izlaz{Unesite broj:}\ulaz{1210}#
#\izlaz{110}#
#\izlaz{Unesite broj:}\ulaz{18}#
#\izlaz{18}#
#\izlaz{Unesite broj:}\ulaz{3856}#
#\izlaz{356}#
#\izlaz{Unesite broj:}\ulaz{0}#
\end{upotreba}
\end{miditest}
\begin{miditest}
\begin{upotreba}{2}
#\naslovInt#
#\izlaz{Unesite broj:}\ulaz{-9632}#
#\izlaz{-932}#
#\izlaz{Unesite broj:}\ulaz{246}#
#\izlaz{46}#
#\izlaz{Unesite broj:}\ulaz{-52}#
#\izlaz{-52}#
#\izlaz{Unesite broj:}\ulaz{0}#
\end{upotreba}
\end{miditest}


\linkresenje{p1.4_}
\end{Exercise}
\begin{Answer}[ref=p1.4_]
%\includecode{resenja/1_KontrolaToka/1.4_Funkcije/1_14.c}
\end{Answer}

\begin{Exercise}[label=p1.4_] 
 Napisati funkciju $int\ rotacija(int\ n)$ koja rotira cifre zadatog broja za jednu poziciju u levo. Napisati program koji za brojeve koji se unose sa standardnog ulaza sve do pojave broja 0 ispisuje rezultat primene funkcije. \\
\begin{miditest}
\begin{upotreba}{1}
#\naslovInt#
#\izlaz{Unesite broj:}\ulaz{146}#
#\izlaz{461}#
#\izlaz{Unesite broj:}\ulaz{18}#
#\izlaz{81}#
#\izlaz{Unesite broj:}\ulaz{3856}#
#\izlaz{8563}#
#\izlaz{Unesite broj:}\ulaz{7}#
#\izlaz{7}#
#\izlaz{Unesite broj:}\ulaz{0}#
\end{upotreba}
\end{miditest}
\begin{miditest}
\begin{upotreba}{2}
#\naslovInt#
#\izlaz{Unesite broj:}\ulaz{89}#
#\izlaz{98}#
#\izlaz{Unesite broj:}\ulaz{-369}#
#\izlaz{-693}#
#\izlaz{Unesite broj:}\ulaz{-55281}#
#\izlaz{-52815}#
#\izlaz{Unesite broj:}\ulaz{0}#
\end{upotreba}
\end{miditest}


\linkresenje{p1.4_}
\end{Exercise}
\begin{Answer}[ref=p1.4_]
%\includecode{resenja/1_KontrolaToka/1.4_Funkcije/1_14.c}
\end{Answer}

\begin{Exercise}[label=p1.4_] 
 Napisati funkciju $float\ aritmeticka\_sredina(int\ n)$ koja računa aritmetičku sredinu cifara datog broja. Napisati i program koji testira rad napisane funkcije. Rezultat ispisivati na tri decimale.\\
\begin{miditest}
\begin{upotreba}{1}
#\naslovInt#
#\izlaz{Unesite broj:}\ulaz{461}#
#\izlaz{3.667}#
\end{upotreba}
\end{miditest}
\begin{miditest}
\begin{upotreba}{2}
#\naslovInt#
#\izlaz{Unesite broj:}\ulaz{1001}#
#\izlaz{0.500}#
\end{upotreba}
\end{miditest}
\begin{miditest}
\begin{upotreba}{3}
#\naslovInt#
#\izlaz{Unesite broj:}\ulaz{-84723}#
#\izlaz{4.800}#
\end{upotreba}
\end{miditest}

\linkresenje{p1.4_}
\end{Exercise}
\begin{Answer}[ref=p1.4_]
%\includecode{resenja/1_KontrolaToka/1.4_Funkcije/1_14.c}
\end{Answer}

\begin{Exercise}[label=p1.4_] 
 Napisati funkciju $int\ zapis(int\ x, int\ y)$ koja proverava da li se brojevi $x$ i $y$ zapisuju pomoću istih cifara. Funkcija treba da vrati vrednost 1 ako je uslov ispunjen, odnosno 0 ako nije. Napisati i program koji učitava dva cela broja i ispisuje rezultat primene funkcije. \\
\begin{miditest}
\begin{upotreba}{1}
#\naslovInt#
#\izlaz{Unesite dva broja:}\ulaz{251 125}#
#\izlaz{Uslov je ispunjen!}#
\end{upotreba}
\end{miditest}
\begin{miditest}
\begin{upotreba}{2}
#\naslovInt#
#\izlaz{Unesite dva broja:}\ulaz{8898 9988}#
#\izlaz{Uslov nije ispunjen!}#
\end{upotreba}
\end{miditest}
\begin{miditest}
\begin{upotreba}{3}
#\naslovInt#
#\izlaz{Unesite dva broja:}\ulaz{-7391 1397}#
#\izlaz{Uslov je ispunjen!}#
\end{upotreba}
\end{miditest} 

\linkresenje{p1.4_}
\end{Exercise}
\begin{Answer}[ref=p1.4_]
%\includecode{resenja/1_KontrolaToka/1.4_Funkcije/1_14.c}
\end{Answer}

\begin{Exercise}[label=p1.4_] 
 Napisati funkciju $int\ faktorijel(int\ n)$ koja računa faktorijel broja $n$. Napisati i program koji učitava dva cela broja $x$ i $y$ ($0\leq x,y \leq12$) i ispisuje vrednost zbira $x!+y!$. \\
\begin{miditest}
\begin{upotreba}{1}
#\naslovInt#
#\izlaz{Unesite dva broja:}\ulaz{4 5}#
#\izlaz{144}#
\end{upotreba}
\end{miditest}
\begin{miditest}
\begin{upotreba}{2}
#\naslovInt#
#\izlaz{Unesite dva broja:}\ulaz{18 -5}#
#\izlaz{Greska: pogresan unos!}#
\end{upotreba}
\end{miditest}
\begin{miditest}
\begin{upotreba}{3}
#\naslovInt#
#\izlaz{Unesite dva broja:}\ulaz{6 0}#
#\izlaz{721}#
\end{upotreba}
\end{miditest}

\linkresenje{p1.4_}
\end{Exercise}
\begin{Answer}[ref=p1.4_]
%\includecode{resenja/1_KontrolaToka/1.4_Funkcije/1_14.c}
\end{Answer}

\begin{Exercise}[label=p1.4_] 
  Napisati funkciju $int\ rastuce(int\ n)$ koja ispituje da li su cifre datog celog broja u
rastućem poretku. Funkcija treba da vrati vrednost 1 ako cifre ispunjavaju uslov, odnosno 0 ako ne ispunjavaju uslov. Napisati i program koji učitava ceo broj i ispisuje rezultat primene funkcije. \\
\begin{miditest}
\begin{upotreba}{1}
#\naslovInt#
#\izlaz{Unesite broj:}\ulaz{2689}#
#\izlaz{Cifre su u rastucem poretku!}#
\end{upotreba}
\end{miditest}
\begin{miditest}
\begin{upotreba}{2}
#\naslovInt#
#\izlaz{Unesite broj:}\ulaz{559}#
#\izlaz{Cifre su u rastucem poretku!}#
\end{upotreba}
\end{miditest}
\begin{miditest}
\begin{upotreba}{3}
#\naslovInt#
#\izlaz{Unesite broj:}\ulaz{628}#
#\izlaz{Cifre nisu u rastucem poretku!}#
\end{upotreba}
\end{miditest}

\linkresenje{p1.4_}
\end{Exercise}
\begin{Answer}[ref=p1.4_]
%\includecode{resenja/1_KontrolaToka/1.4_Funkcije/1_14.c}
\end{Answer}

\begin{Exercise}[label=p1.4_] 
 Broj je Armstrongov ako je jednak sumi nekog stepena svojih cifara.
\begin{itemize}
\item [a)] Napisati funkciju $int\ stepen(int\ x,\ int\ n)$ koja izračunava $n$-ti stepen broja $x$.
\item [b)] Napisati funkciju $int\ armstrong(int\ x)$ koja vraća 1 ako je broj Armstrongov, odnosno 0 ako nije.
\item [c)] Napisati program koji za ceo broj koji se unosi sa standardnog ulaza proverava da li je Armstrongov (koristeci funkciju $armstrong$).
\end{itemize}
\begin{miditest}
\begin{upotreba}{1}
#\naslovInt#
#\izlaz{Unesite broj:}\ulaz{153}#
#\izlaz{Broj je Armstrongov!}#
\end{upotreba}
\end{miditest}
\begin{miditest}
\begin{upotreba}{2}
#\naslovInt#
#\izlaz{Unesite broj:}\ulaz{1634}#
#\izlaz{Broj je Armstrongov!}#
\end{upotreba}
\end{miditest}
\begin{miditest}
\begin{upotreba}{3}
#\naslovInt#
#\izlaz{Unesite broj:}\ulaz{118}#
#\izlaz{Broj nije Armstrongov!}#
\end{upotreba}
\end{miditest}

\linkresenje{p1.4_}
\end{Exercise}
\begin{Answer}[ref=p1.4_]
%\includecode{resenja/1_KontrolaToka/1.4_Funkcije/1_14.c}
\end{Answer}

\begin{Exercise}[label=p1.4_] 
  Napisati funkciju $int\ par\_nepar(int\ n)$ koja ispituje da li su cifre datog celog broja naizmenično parne i neparne. Funkcija treba da vrati vrednost 1 ako cifre ispunjavaju uslov, odnosno 0 ako ne ispunjavaju uslov. Napisati i program koji učitava ceo broj i testira rad funkcije. \\
\begin{miditest}
\begin{upotreba}{1}
#\naslovInt#
#\izlaz{Unesite broj:}\ulaz{2749}#
#\izlaz{Broj ispunjava uslov!}#
\end{upotreba}
\end{miditest}
\begin{miditest}
\begin{upotreba}{2}
#\naslovInt#
#\izlaz{Unesite broj:}\ulaz{-963}#
#\izlaz{Broj ispunjava uslov!}#
\end{upotreba}
\end{miditest}
\begin{miditest}
\begin{upotreba}{3}
#\naslovInt#
#\izlaz{Unesite broj:}\ulaz{27449}#
#\izlaz{Broj ne ispunjava uslov!}#
\end{upotreba}
\end{miditest}

\linkresenje{p1.4_}
\end{Exercise}
\begin{Answer}[ref=p1.4_]
%\includecode{resenja/1_KontrolaToka/1.4_Funkcije/1_14.c}
\end{Answer}

\begin{Exercise}[label=p1.4_] 
 Napisati funkciju $int\ prebrojavanje(float\ x)$ koja prebrojava koliko puta se broj $x$ pojavljuje u nizu brojeva koji se unose sa standardnog ulaza sve do pojave nule. Napisati program koji učitava vrednost broja $x$ i testira rad napisane funkcije. \\
\begin{miditest}
\begin{upotreba}{1}
#\naslovInt#
#\izlaz{Unesite broj x:}\ulaz{2.84}#
#\izlaz{Unesite brojeve:}\ulaz{8.13 2.84 5 21.6 2.84 11.5 0}#
#\izlaz{Broj pojavljivanja broja 2.84 je: 2}#
\end{upotreba}
\end{miditest}
\begin{miditest}
\begin{upotreba}{2}
#\naslovInt#
#\izlaz{Unesite broj x:}\ulaz{-1.17}#
#\izlaz{Unesite brojeve:}\ulaz{-128.35 8.965 8.968 89.36 0}#
#\izlaz{Broj pojavljivanja broja -1.17 je: 0}#
\end{upotreba}
\end{miditest}

\linkresenje{p1.4_}
\end{Exercise}
\begin{Answer}[ref=p1.4_]
%\includecode{resenja/1_KontrolaToka/1.4_Funkcije/1_14.c}
\end{Answer}

\begin{Exercise}[label=p1.4_] 
 Napisati funkciju $long\ int\ fibonaci(int\ n)$ koja računa n-ti element Fibonačijevog niza. Fibonačijev niz je niz za koji važi: $F_0=1$, $F_1=1$, $F_{n+2}=F_{n+1}+F_{n}$ za $n \geq 0$. Napisati i program koji učitava ceo broj $n\ (0\leq n\leq 50)$ i ispisuje traženi Fibonačijev broj. \\
\begin{miditest}
\begin{upotreba}{1}
#\naslovInt#
#\izlaz{Unesite broj n:}\ulaz{7}#
#\izlaz{21}#
\end{upotreba}
\end{miditest}
\begin{miditest}
\begin{upotreba}{2}
#\naslovInt#
#\izlaz{Unesite broj n:}\ulaz{65}#
#\izlaz{Greska: nedozvoljena vrednost!}#
\end{upotreba}
\end{miditest}

\linkresenje{p1.4_}
\end{Exercise}
\begin{Answer}[ref=p1.4_]
%\includecode{resenja/1_KontrolaToka/1.4_Funkcije/1_14.c}
\end{Answer}

\begin{Exercise}[label=p1.4_] 
 Napisati funkciju $char\ sifra(char\ c,\ int\ k)$ koja za dati karakter $c$ određuje šifru na sledeći način: ukoliko je $c$ slovo, šifra je karakter koji se nalazi $k$ pozicija iza njega u abecedi. U suprotnom karakter ostaje nepromenjen. Šifrovanje treba da bude kružno, što znači da je, na primer, šifra za c='b' i k=2 karakter 'z'. Napisati program koji učitava karakter po karakter do kraja ulaza (do pojave EOF koji se generiše kombinacijom CTRL+D) i ispisuje šifrovani tekst. \\
\begin{miditest}
\begin{upotreba}{1}
#\naslovInt#
#\izlaz{Unesite broj k:}\ulaz{2}#
#\izlaz{Unesite tekst (CTRL+D za prekid):}#
#\ulaz{c}#
#\izlaz{a}#
#\ulaz{8}#
#\izlaz{8}#
#\ulaz{+}#
#\izlaz{+}#
#\ulaz{Z}#
#\izlaz{X}#
\end{upotreba}
\end{miditest}
\linkresenje{p1.4_}
\end{Exercise}
\begin{Answer}[ref=p1.4_]
%\includecode{resenja/1_KontrolaToka/1.4_Funkcije/1_14.c}
\end{Answer}









\section{Rešenja}
\shipoutAnswer

