
\chapter{Ulaz i izlaz programa}

%\section{Standardni tokovi}

%\section{Argumenti komandne linije}

\section{Datoteke}

\begin{Exercise}[label=v3_01] 
Napisati program koji prepisuje sadrzaj datoteke ulaz.txt u
datoteku izlaz.txt karakter po karakter.
\linkresenje{v3_01}
\end{Exercise}
\ifresenja
\begin{Answer}[ref=v3_01]
\includecode{resenja/3_Datoteke/1/1.c}
\end{Answer}
\fi

\begin{Exercise}[label=v3_02] 
Napisati program koji u datoteci cije se ime navodi kao prvi
argument komandne linije odredjuje liniju maksimalne duzine i
ispisuje je na standarni izlaz. Ukoliko ima vise takvih linija ,
ispisati onu koja je leksikografski prva. Mozemo pretpostaviti
da datoteka ne sadrzi linije duze od 80 karaktera.
\linkresenje{v3_02}
\end{Exercise}
\ifresenja
\begin{Answer}[ref=v3_02]
\includecode{resenja/3_Datoteke/2/2.c}
\end{Answer}
\fi

\begin{Exercise}[label=v3_03] 
U datoteci cije se ime zadaje kao prvi argument komandne linije
nalazi se
prirodan broj n a zatim i n celih brojeva. Napisati program koji
prebrojava
koliko k-tocifrenih brojeva postoji u datoteci , pri cemu se
prirodan broj k
zadaje kao drugi argument komandne linije.

\linkresenje{v3_03}
\end{Exercise}
\ifresenja
\begin{Answer}[ref=v3_03]
\includecode{resenja/3_Datoteke/3/3.c}
\end{Answer}
\fi

\begin{Exercise}[label=v3_04] 
U datoteci cije se ime navodi kao prvi argument komandne
linije navedena je rec r i niz linija. Napisati
program koji u datoteku cije se ime navodi kao
drugi argument komandne linije upisuje sve linije
u kojima se rec r pojavljuje bar n puta , gde je
n prirodan broj koji se unosi sa standardnog ulaza. Ispis
treba da bude u formatu broj\_pojavljivanja : linija.
\linkresenje{v3_04}
\end{Exercise}
\ifresenja
\begin{Answer}[ref=v3_04]
\includecode{resenja/3_Datoteke/4/4.c}
\end{Answer}
\fi

\begin{Exercise}[label=v3_05] 
Program se pokrece tako sto se navedu nazivi dve datoteke(ulazna i
izlazna) i opcije.
U datoteci cije se ime navodi kao prvi argument komandne linije
nalaze se podaci o razlomcima:
u prvom redu se nalazi broj razlomaka , a u svakom sledecem redu
brojilac i imenilac jednog razlomka.
Potrebno je kreirati strukturu koja opisuje razlomak i ucitati niz
razlomaka
iz datoteke , a potom:\\
a) ukoliko je navedena opcija x, upisati u datoteku cije je ime
drugi argument komandne linije
reciprocni razlomak za svaki razlomak iz niza (npr. za 2/3
treba upisati 3/2) \\
b) ukoliko je navedena opcija y, upisati u datoteku cije je ime
drugi argument komandne linije
realnu vrednost reciprocnog razlomka svakog razlomka iz niza
(npr. za 2/3 treba upisati 1.5)

Mozemo pretpostaviti da se u datoteci sa podacima o razlomcima
nalazi najvise 100 razlomaka.
\linkresenje{v3_05}
\end{Exercise}
\ifresenja
\begin{Answer}[ref=v3_05]
\includecode{resenja/3_Datoteke/5/5.c}
\end{Answer}
\fi

\begin{Exercise}[label=v3_06] 
Za svaki automobil poznati su marka , model i cena. Iz datoteke cije
se ime zadaje sa standardnog ulaza ucitava se broj automobila a
potom
i podaci za svaki automobil. Program treba da: \\
a) izracuna prosecnu cenu po marki kola \\
b) za maksimalnu cenu koju je kupac spreman da plati , a koja se
zadaje kao argument komandne linije, da ispise automobile u tom cenovnom
rangu zajednu sa prosecnom cenom odgovarajuce marke

Mozemo pretpostaviti da se model i marka sastoje od jedne reci i
da svaka od njih sadrzi najvise 30 karaktera kao i da se u datoteci
nalaze podaci za najvise 100 automobila.
\linkresenje{v3_06}
\end{Exercise}
\ifresenja
\begin{Answer}[ref=v3_06]
\includecode{resenja/3_Datoteke/6/6.c}
\end{Answer}
\fi







\begin{Exercise}[label=p3_01] 
Napisati program koji prebrojava mala slova u datoteci $test.txt$. 

\begin{miditest}
\begin{upotreba}{1}
#\naslovDat{test.txt}#
#\datoteka{Abcd EFGH+ijKLMN}#

#\naslovIzlaz#
#\izlaz{Broj malih slova je: 5}#
\end{upotreba}
\end{miditest}
\begin{miditest}
\begin{upotreba}{2}
#\naslovDat{test.txt}#
#\datoteka{PrograMiranje}#

#\naslovIzlaz#
#\izlaz{Broj malih slova je: 11}#
\end{upotreba}
\end{miditest}
\linkresenje{p3_01}
\end{Exercise}
\ifresenja
\begin{Answer}[ref=p3_01]
\includecode{resenja/3_Datoteke/praktikumi14/1_01.c}
\end{Answer}
\fi

\begin{Exercise}[label=p3_02] 
Napisati program koji prepisuje svaki treći karakter datoteke $ulaz.txt$ u datoteku $izlaz.txt$.\\
\begin{miditest}
\begin{upotreba}{1}
#\naslovDat{ulaz.txt}#
#\datoteka{Volim programiranje.}#
#\naslovDat{izlaz.txt}#
#\datoteka{Vipgmae}#
\end{upotreba}
\end{miditest}

\linkresenje{p3_02}
\end{Exercise}
\ifresenja
\begin{Answer}[ref=p3_02]
\includecode{resenja/3_Datoteke/praktikumi14/1_02.c}
\end{Answer}
\fi

\begin{Exercise}[label=p3_03] 
 Kao argumenti komandne linije se zadaju ime datoteke i ceo broj $k$. Napisati program koji na  standardni izlaz ispisuje sve linije zadate datoteke čija je dužina veća od k. Može se pretpostaviti da dužina linije neće biti veća od 80 karaktera.\\
\begin{miditest}
\begin{upotreba}{1}
#\poziv{./a.out test.txt 7}#
#\naslovDat{test.txt}#
#\datoteka{Teme koje su obradjivane:}#
#\datoteka{Petlje}#
#\datoteka{Funkcije}#
#\datoteka{Nizovi}#
#\datoteka{Strukture}#

#\naslovIzlaz#
#\izlaz{Teme koje su obradjivane:}#
#\izlaz{Funkcije}#
#\izlaz{Strukture}#
\end{upotreba}
\end{miditest}
\begin{miditest}
\begin{upotreba}{2}
#\poziv{./a.out test.txt}#

#\naslovIzlaz#
#\izlaz{Greska: Pogresan broj argumenata!}#
\end{upotreba}
\end{miditest}

\linkresenje{p3_03}
\end{Exercise}
\ifresenja
\begin{Answer}[ref=p3_03]
\includecode{resenja/3_Datoteke/praktikumi14/1_03.c}
\end{Answer}
\fi

\begin{Exercise}[label=p3_04] 
 Napisati program koji prebrojava koliko se linija datoteke $ulaz.txt$ završava niskom $s$ koja se učitava sa standardnog ulaza. Može se pretpostaviti da dužina linije neće biti veća od 80 karaktera, kao i da dužina niske $s$ neće biti veća od 20 karaktera.\\
\begin{miditest}
\begin{upotreba}{1}
#\naslovDat{ulaz.txt}#
#\datoteka{abcde abcde}#
#\datoteka{abcde aab}#
#\datoteka{abcde abcde abcde}#
#\datoteka{abcde abcde Aab}#
#\datoteka{abcde abcde ab}#
#\datoteka{abcde abcde abcde abcde}#

#\naslovInt#
#\izlaz{Unesite nisku s:}\ulaz{ab}#
#\izlaz{Broj linija: 3}#
\end{upotreba}
\end{miditest}
\begin{miditest}
\begin{upotreba}{2}
#\naslovDat{ulaz.txt}#
#\datoteka{abcde abcde}#
#\datoteka{abcde}#
#\datoteka{abcde abcde AB}#

#\naslovInt#
#\izlaz{Unesite nisku s:}\ulaz{ab}#
#\izlaz{Broj linija: 0}#
\end{upotreba}
\end{miditest}

\linkresenje{p3_04}
\end{Exercise}
\ifresenja
\begin{Answer}[ref=p3_04]
\includecode{resenja/3_Datoteke/praktikumi14/1_04.c}
\end{Answer}
\fi

\begin{Exercise}[label=p3_05] 
 Napisati program koji pronalazi maksimum brojeva zapisanih u datoteci $brojevi.txt$. \\
\begin{miditest}
\begin{upotreba}{1}
#\naslovDat{brojevi.txt}#
#\datoteka{2.36 -16.11 5.96 8.88}#
#\datoteka{-265.31 54.96 38.4}#

#\naslovIzlaz#
#\izlaz{Najveci broj je: 54.96}#
\end{upotreba}
\end{miditest}

\linkresenje{p3_05}
\end{Exercise}
\ifresenja
\begin{Answer}[ref=p3_05]
\includecode{resenja/3_Datoteke/praktikumi14/1_05.c}
\end{Answer}
\fi

\begin{Exercise}[label=p3_06] 
 U datoteci $studenti.txt$ se nalaze informacije o studentima: prvo broj studenata, a zatim u pojedinačnim linijama korisničko ime i pet poslednjih ocena koje je student dobio. Napisati program koji pronalazi studenta koji je ostvario najbolji uspeh i ispisuje njegove podatke. Pretpostaviti da broj studenata neće biti veći od 100.\\
\begin{miditest}
\begin{upotreba}{1}
#\naslovDat{studenti.txt}#
#\datoteka{mr15239 10 9 9 8 10}#
#\datoteka{mi14005 8 8 9 8 10}#
#\datoteka{ml15112 9 8 8 7 10}#
#\datoteka{mr15007 10 10 10 10 10}#
#\datoteka{mn13208 7 7 9 6 10}#

#\naslovIzlaz#
#\izlaz{korisnicko ime: mr15007, prosek ocena: 10.00}#
\end{upotreba}
\end{miditest}

\linkresenje{p3_06}
\end{Exercise}
\ifresenja
\begin{Answer}[ref=p3_06]
\includecode{resenja/3_Datoteke/praktikumi14/1_06.c}
\end{Answer}
\fi

\begin{Exercise}[label=p3_07] 
 U datoteci $tacke.txt$ se nalazi prvo broj tačaka, a zatim u pojedinačnim linijama x i y koordinate tačke. Napisati program koji u datoteku $rastojanja.txt$ upisuje rastojanje svake od pročitanih tačaka od koordinatnog početka, a na standardni izlaz koordinate tačke koja je najudaljenija. Koristiti strukturu $Tacka$ sa poljima $x$ i $y$, kao i funkciju kojom se računa rastojanje. Pretpostaviti da broj tačaka u datoteci neće biti veći od 50. \\  
\begin{miditest}
\begin{upotreba}{1}
#\naslovDat{tacke.txt}#
#\datoteka{4}#
#\datoteka{11 -2}#
#\datoteka{3 5}#
#\datoteka{8 -8}#
#\datoteka{0 4}#

#\naslovDat{rastojanja.txt}#
#\datoteka{11.18}#
#\datoteka{5.29}#
#\datoteka{11.31}#
#\datoteka{4.00}#

#\naslovIzlaz#
#\izlaz{Najudaljenija je tačka: 8 -8}#
\end{upotreba}
\end{miditest}
\begin{miditest}
\begin{upotreba}{1}
#\naslovDat{tacke.txt}#
#\datoteka{-2}#
#\datoteka{0 0}#
#\datoteka{9 -8}#

#\naslovIzlaz#
#\izlaz{Greska: Nedozvoljen broj tacaka!}#
\end{upotreba}
\end{miditest}

\linkresenje{p3_07}
\end{Exercise}
\ifresenja
\begin{Answer}[ref=p3_07]
%\includecode{resenja/3_Datoteke/praktikumi14/1_07.c}
\end{Answer}
\fi

\begin{Exercise}[label=p3_08] 
 Napisati program koji za reč $s$ maksimalne dužine 20 karaktera koja se zadaje sa standardnog ulaza u datoteku $rotacije.txt$ upisuje sve rotacije reči $s$. \\
\begin{miditest}
\begin{upotreba}{1}
#\naslovInt#
#\izlaz{Unesite rec: }\ulaz{abcde}#

#\naslovDat{rotacije.txt}#
#\datoteka{abcde}#
#\datoteka{bcdea}#
#\datoteka{cdeab}#
#\datoteka{deabc}#
#\datoteka{eabcd}#
\end{upotreba}
\end{miditest}
\linkresenje{p3_08}
\end{Exercise}
\ifresenja
\begin{Answer}[ref=p3_08]
\includecode{resenja/3_Datoteke/praktikumi14/1_08.c}
\end{Answer}
\fi

\begin{Exercise}[label=p3_09] 
 Napisati program koji linije koji se učitavaju sa standardnog ulaza sve do kraja ulaza prepisuje u datoteku $izlaz.txt$ i to, ako je prilikom pokretanja zadata opcija \textit{-v} ili \textit{-V} samo one linije koje počinju velikim slovom, ako je zadata opcija \textit{-m} ili \textit{-M} samo one linije koje počinju malim slovom, a ako je opcija izostavljena sve linije. Pretpostaviti da linije neće biti duže od 80 karaktera. \\
\begin{miditest}
\begin{upotreba}{1}
#\poziv{./a.out -m}#
#\naslovInt#
#\izlaz{Unesite recenice: }#
#\ulaz{programiranje u C-u je zanimljivo}#
#\ulaz{Volim programiranje!}#
#\ulaz{Kada porastem bicu programer!}#
#\ulaz{u slobodno vreme programiram}#

#\naslovDat{izlaz.txt}#
#\datoteka{programiranje u C-u je zanimljivo}#
#\datoteka{u slobodno vreme programiram}#
\end{upotreba}
\end{miditest}
\begin{miditest}
\begin{upotreba}{2}
#\poziv{./a.out -V}#
#\naslovInt#
#\izlaz{Unesite recenice: }#
#\ulaz{programiranje u C-u je zanimljivo}#
#\ulaz{Volim programiranje!}#
#\ulaz{Kada porastem bicu programer!}#
#\ulaz{u slobodno vreme programiram}#

#\naslovDat{izlaz.txt}#
#\datoteka{Volim programiranje!}#
#\datoteka{Kada porastem bicu programer!}#
\end{upotreba}
\end{miditest}

\begin{miditest}
\begin{upotreba}{3}
#\poziv{./a.out -k}#
#\naslovInt#
#\izlaz{Greska: Pogresno pokretanje programa!}#
\end{upotreba}
\end{miditest}
\linkresenje{p3_09}
\end{Exercise}
\ifresenja
\begin{Answer}[ref=p3_09]
\includecode{resenja/3_Datoteke/praktikumi14/1_09.c}
\end{Answer}
\fi


\begin{Exercise}[label=p3_] 
Sa standarnog ulaza u\v citavaju se imena dve tekstualne datoteke i
jedan karakter.  Napisati program koji prepisuje datoteku \v cije se
ime navodi kao prvo u datoteku \v cije ime se navodi kao
drugo. Ukoliko je ucitan karakter \verb|u| program prilikom
prepisivanja treba da zamenjuje sva mala slova velikim, a ukoliko je
u\v citan karakter \verb|l| sva velika slova se zamenjuju malim. U
slu\v caju greske ispisati -1. Gre\v ska mo\v ze biti neuspe\v sno
otvaranje datoteke ili pogre\v sno zadat karakter. Maksimalna du\v
zina naziva datoteka je 20 karaktera. \\
\begin{miditest}
\begin{upotreba}{1}
#\naslovInt#
#\ulaz{ulaz.txt izlaz.txt u}#
#\naslovDat{ulaz.txt}#
#\datoteka{danas je lep dan}#
#\datoteka{i Ja zelim}#
#\datoteka{da postanem programer}#
#\naslovDat{izlaz.txt}#
#\datoteka{DANAS JE LEP DAN}#
#\datoteka{I JA ZELIM}#
#\datoteka{DA POSTANEM PROGRAMER}#
\end{upotreba}
\end{miditest}
\begin{miditest}
\begin{upotreba}{2}
#\naslovInt#
#\ulaz{prva.dat druga.dat l}#
#\naslovDat{prva.dat}#
#\datoteka{Cena soka je 30}#
#\datoteka{Cena vina je 150}#
#\datoteka{Cena limunade je 200}#
#\datoteka{Cena sendvica je 120}#
#\naslovDat{druga.dat}#
#\datoteka{cena soka je 30}#
#\datoteka{cena vina je 150}#
#\datoteka{cena limunade je 200}#
#\datoteka{cena sendvica je 120}#
\end{upotreba}
\end{miditest}
\begin{miditest}
\begin{upotreba}{3}
#\naslovInt#
#\ulaz{primer.c prazna.txt V}#
#\naslovDat{primer.c}#
#\datoteka{\#include <stdio.h>}#
#\datoteka{int main()}#
#\datoteka{\{}#
#\datoteka{\}}#
#\naslovDat{prazna.txt}#
#\datoteka{}#
#\naslovIzlaz#
#\izlaz{-1}#
\end{upotreba}
\end{miditest}
\linkresenje{p3_}
\end{Exercise}
\ifresenja
\begin{Answer}[ref=p3_]
%\includecode{resenja/3_Datoteke/6/6.c}
\end{Answer}
\fi

           
\begin{Exercise}[label=p3_]         
Sastaviti program koji sa standardnog ulaza prima ime datoteke koju
treba otvoriti. Ispisati (na standardnom izlazu) koja cifra (me\d u
svim ciframa koje se pojavljuju u datoteci) ima najve\' ci broj
pojavljivanja. U slu\v caju gre\v ske pri otvaranju datoteke ispisati
{\tt -1}. Ukoliko nema cifara u datoteci ispisati {\tt -1}.
Maksimalna du\v zina naziva datoteka je 20 karaktera. \\
\begin{miditest}
\begin{upotreba}{1}
#\naslovInt#
#\ulaz{ulaz.txt}#
#\naslovDat{ulaz.txt}#
#\datoteka{danas je lep dan}#
#\datoteka{i Ja zelim}#
#\datoteka{da postanem programer}#
#\naslovIzlaz#
#\izlaz{-1}#
\end{upotreba}
\end{miditest}
\begin{miditest}
\begin{upotreba}{2}
#\naslovInt#
#\ulaz{prva.dat druga.dat l}#
#\naslovDat{prva.dat}#
#\datoteka{Cena soka je 30}#
#\datoteka{Cena vina je 150}#
#\datoteka{Cena limunade je 200}#
#\datoteka{Cena sendvica je 120}#
#\naslovIzlaz#
#\izlaz{0}#
\end{upotreba}
\end{miditest}
\begin{miditest}
\begin{upotreba}{3}
#\naslovInt#
#\ulaz{primer.c}#
#\naslovDat{primer.c}#
#\datoteka{\#include <stdio.h>}#
#\datoteka{int main()}#
#\datoteka{\{}#
#\datoteka{\}}#
#\naslovDat{prazna.txt}#
#\datoteka{}#
#\naslovIzlaz#
#\izlaz{-1}#
\end{upotreba}
\end{miditest}
\linkresenje{p3_}
\end{Exercise}
\ifresenja
\begin{Answer}[ref=p3_]
%\includecode{resenja/3_Datoteke/6/6.c}
\end{Answer}
\fi


\begin{Exercise}[label=p3_]         
Prvi red datoteke \verb|matrice.txt| sadr\v zi 2 cela broja manja od
50 koji predstavljaju redom broj vrsta i broj kolona realne matrice
A. Svaki slede\'ci red sadr\v zi po jednu vrstu matrice. Napisati
program koji pronalazi sve elemente matrice A koji su jednaki zbiru
svih svojih susednih elemenata i \v stampa ih u obliku
\begin{verbatim}
(broj vrste, broj kolone, vrednost elementa).
\end{verbatim}
U slu\v caju gre\v ske prilikom otvaranja datoteke ispisati {\tt -1}.
Pretpostaviti da je sadr\v zaj datoteke ispravan. \\
\begin{miditest}
\begin{upotreba}{1}
#\naslovDat{matrice.txt}#
#\datoteka{1  2  3  4}#
#\datoteka{7  2 15 -3}#
#\datoteka{-1  3  1  3}#
#\naslovIzlaz#
#\izlaz{(1, 0, 7)}#
#\izlaz{(1, 2, 15)}#
\end{upotreba}
\end{miditest}
\linkresenje{p3_}
\end{Exercise}
\ifresenja
\begin{Answer}[ref=p3_]
%\includecode{resenja/3_Datoteke/6/6.c}
\end{Answer}
\fi


\begin{Exercise}[label=p3_]         
Napisati program koji za dve datoteke \v cija su imena data kao prvi i
drugo na standarnom ulazu, radi slede\'ce: za cifru u prvoj datoteci,
u drugu datoteku se upisuje 0, za slovo se upisuje 1, a za sve ostale
karaktere se upisuje 2. Maksimalna du\v zina naziva datoteka je 20
karaktera. \\
\begin{miditest}
\begin{upotreba}{2}
#\naslovInt#
#\ulaz{prva.dat druga.dat}#
#\naslovDat{prva.dat}#
#\datoteka{Cena soka je 30}#
#\datoteka{Cena vina je 150}#
#\datoteka{Cena limunade je 200}#
#\datoteka{Cena sendvica je 120}#
#\naslovDat{druga.dat}#
#\datoteka{11112111121120021111211112112000211112111111112112000211112111111112112000}
\end{upotreba}
\end{miditest}
\linkresenje{p3_}
\end{Exercise}
\ifresenja
\begin{Answer}[ref=p3_]
%\includecode{resenja/3_Datoteke/6/6.c}
\end{Answer}
\fi


\begin{Exercise}[label=p3_]         
Ako je data tekstualna datoteka \verb|plain.txt| napraviti tekstualnu
datoteku \verb|sifra.txt| tako \v sto se svako slovo zamenjuje svojim
prethodnikom (cikli\v cno) suprotne velicine \verb|’b’| sa \verb|’A’|,
\verb|’B’| sa \verb|’a’|, \verb|’a’| sa \verb|’Z’|, \verb|’A’| sa
\verb|’z’|, itd. Podrazumevati da se na sistemu koristi tabela
karaktera ASCII.
\linkresenje{p3_}
\end{Exercise}
\ifresenja
\begin{Answer}[ref=p3_]
%\includecode{resenja/3_Datoteke/6/6.c}
\end{Answer}
\fi


\begin{Exercise}[label=p3_]         
Sa standarnog ulaza se u\v citava ime tekstualne datoteke i prirodan
broj k. Podrazumeva se da zadata datoteka sadr\v zi samo slova i
beline i da je svaka re\v c iz datoteke du\v zine najvi\v se
100. Program treba da u\v citava re\v ci iz datoteke, da svaku re\v c
rotira za k mesta i da tako dobijenu re\v c upi\v se u datoteku \v
cije je ime \verb|rotirano.txt|. Maksimalna du\v zina naziva datoteka
je 20 karaktera. \\
\linkresenje{p3_}
\end{Exercise}
\ifresenja
\begin{Answer}[ref=p3_]
%\includecode{resenja/3_Datoteke/6/6.c}
\end{Answer}
\fi


\begin{Exercise}[label=p3_]         
Napisati program koji u datoteku \verb|izlaz.txt| prepisuje sve
re\v{c}i iz datoteke \verb|ulaz.txt| \v{c}iji je zbir ascii kodova
slova strogo ve\'{c}i od 1000. Re\v ci su odvojene prazninama i nisu
du\v ze od 200 karaktera.  \\
\begin{miditest}
\begin{upotreba}{1}
#\naslovDat{ulaz.txt}#
#\datoteka{Sa standardnog ulaza unosi se neoznacen}#
#\datoteka{ceo broj. Formirati novi broj koji se dobija}#
#\datoteka{izbacivanjem svake druge cifre iz polaznog broja.}#
#\naslovDat{izlaz.txt}#
#\datoteka{standardnog izbacivanjem}
\end{upotreba}
\end{miditest}
\begin{miditest}
\begin{upotreba}{2}
#\naslovDat{ulaz.txt}#
#\datoteka{i sada jedan kratak primer}#
#\datoteka{p1: 1234567890}#
#\datoteka{p2: ABCDEFGHIJ}#
#\datoteka{p3: abcdefghij}#
#\naslovDat{izlaz.txt}#
#\datoteka{abcdefghij}
\end{upotreba}
\end{miditest}
\begin{miditest}
\begin{upotreba}{3}
#\naslovDat{ulaz.txt}#
#\datoteka{konstruisanje test-primera sa}#
#\datoteka{i dugackim recima kao prestolonaslednik}#
#\datoteka{brojevima1234567890}#
#\naslovDat{izlaz.txt}#
#\datoteka{konstruisanje test-primera}#
#\datoteka{prestolonaslednik}#
#\datoteka{brojevima1234567890}#
\end{upotreba}
\end{miditest}
\begin{miditest}
\begin{upotreba}{4}
#\naslovDat{ulaz.txt}#
#\datoteka{ima jos dugackih reci: predskazanje,}#
#\datoteka{potom}#
#\datoteka{nelogicnosti, zanemarivati, odugovlaciti, a ima}#
#\datoteka{i i malih reci koje su kratke}#
#\datoteka{predosecaj}#
#\naslovDat{izlaz.txt}#
#\datoteka{predskazanje, nelogicnosti,}#
#\datoteka{zanemarivati, odugovlaciti,}#
#\datoteka{predosecaj}#
\end{upotreba}
\end{miditest}
\linkresenje{p3_}
\end{Exercise}
\ifresenja
\begin{Answer}[ref=p3_]
%\includecode{resenja/3_Datoteke/6/6.c}
\end{Answer}
\fi


\begin{Exercise}[label=p3_]         
U datoteci \verb|razno.txt| nalazi se tekst. U datoteku
\verb|palindromi.txt| prepisati sve re\v ci iz datoteke
\verb|razno.txt| koje su palindromi. Re\v c je palindrom ako se \v
cita isto sa leve i desne strane. Za re\v c smatramo niz karaktera
koji se nalazi izme\d u belina i koji nije du\v zi od 200
karaktera. Dozvoljeno je kori\v s\'cenje specifikatora za \v citanje
re\v ci. Maksimalan broj re\v ci nije poznat. U slu\v caju gre\v ske
ispisati -1 i prekinuti izvr\v savanje programa. \\
\begin{miditest}
\begin{upotreba}{1}
#\naslovDat{razno.txt}#
#\datoteka{Ana i melem su primeri palindroma.}#
#\naslovDat{palindromi.txt: }#
#\datoteka{Ana i melem}
\end{upotreba}
\end{miditest}
\begin{miditest}
\begin{upotreba}{2}
#\naslovDat{razno.txt}#
#\datoteka{jabuka neven pomorandza kuk}#
#\datoteka{Oko kapAk pero radar caj}#
#\naslovDat{palindromi.txt: }#
#\datoteka{neven kuk}
#\datoteka{Oko kapAk radar}#
\end{upotreba}
\end{miditest}
\linkresenje{p3_}
\end{Exercise}
\ifresenja
\begin{Answer}[ref=p3_]
%\includecode{resenja/3_Datoteke/6/6.c}
\end{Answer}
\fi


\begin{Exercise}[label=p3_]         
U datoteci \v cije se ime navodi na standarnom ulazu programa nalazi
se broj \verb|n|, a zatim i \verb|n| re\v ci (du\v zine najvi\v se 50
karaktera). Napisati program koji u\v citava ovaj niz i
  \begin{enumerate}
  \item ispisuje ga \hfill[3],
  \item iz njega uklanja sve duplikate i u datoteku \verb|rez.txt|
    ispisuje transformisani niz \hfill[4]
  \end{enumerate}
U slu\v caju gre\v ske ispisati -1. Maksimalna du\v zina naziva
datoteka je 20 karaktera. \\
\begin{miditest}
\begin{upotreba}{1}
#\naslovInt#
#\ulaz{dat1.txt}#
#\naslovDat{dat1.txt}#
#\datoteka{12 jha14 hahaha deda mraz deda}#
#\datoteka{mraz deda deda jase konj konj konj}#
#\naslovIzlaz#
#\izlaz{jha14 hahaha deda mraz deda mraz deda}#
#\izlaz{deda jase konj konj konj}#
#\naslovDat{rez.txt: }#
#\datoteka{jha14 hahaha deda mraz jase konj}#
\end{upotreba}
\end{miditest}
\begin{miditest}
\begin{upotreba}{2}
#\naslovInt#
#\ulaz{dat2.txt}#
#\naslovDat{dat2.txt}#
#\datoteka{14}#
#\datoteka{so secer supa so ljuto secer kiselo slatko}#
#\datoteka{ljuto}#
#\datoteka{paprika, ljuta paprika, ljuto dete}#
#\naslovIzlaz#
#\izlaz{so secer supa so ljuto secer kiselo slatko}#
#\izlaz{ljuto paprika, ljuta paprika, ljuto dete}#
#\naslovDat{rez.txt: }#
#\datoteka{so secer supa ljuto kiselo slatko}#
#\datoteka{paprika, ljuta dete}#
\end{upotreba}
\end{miditest}
\linkresenje{p3_}
\end{Exercise}
\ifresenja
\begin{Answer}[ref=p3_]
%\includecode{resenja/3_Datoteke/6/6.c}
\end{Answer}
\fi



\begin{Exercise}[label=p3_]         
U datoteci \v cije se ime navodi na standarnom ulazu programa nalazi
se broj $n$, a zatim i $n$ re\v ci (du\v zine najvi\v se 50
karaktera). Napisati program koji u\v citava ovaj niz i
  \begin{enumerate}
  \item ispisuje ga, \hfill[3]
  \item u datoteku \verb|rez.txt| upisuje sve re\v ci koje sadr\v ze
    prvu re\v c i podvlaku. \hfill[4]
  \end{enumerate}
U slu\v caju gre\v ske ispisati -1. Maksimalna du\v zina naziva
datoteka je 20 karaktera. \\
\begin{miditest}
\begin{upotreba}{1}
#\naslovInt#
#\ulaz{dat1.txt}#
#\naslovDat{dat1.txt}#
#\datoteka{7 rec Opet \_rec Reci rec\_enica}#
#\datoteka{DVa recica\_}#
#\naslovIzlaz#
#\izlaz{rec Opet \_rec Reci rec\_enica }#
#\izlaz{DVa recica\_}#
#\naslovDat{rez.txt: }#
#\datoteka{\_rec rec\_enica recica\_}#
\end{upotreba}
\end{miditest}
\begin{miditest}
\begin{upotreba}{2}
#\naslovInt#
#\ulaz{dat2.txt}#
#\naslovDat{dat2.txt}#
#\datoteka{11 Sunce sija iznad grada}#
#\datoteka{Sunce\_Moje Jedan Dva Su\_nce Sve Sunce123\_123 suncanica.}#
#\naslovIzlaz#
#\izlaz{Sunce sija iznad grada}#
#\izlaz{Sunce\_Moje Jedan Dva Su\_nce Sve Sunce123\_123 suncanica.}#
#\naslovDat{rez.txt: }#
#\datoteka{Sunce\_Moje Sunce123\_123}#
\end{upotreba}
\end{miditest}
\linkresenje{p3_}
\end{Exercise}
\ifresenja
\begin{Answer}[ref=p3_]
%\includecode{resenja/3_Datoteke/6/6.c}
\end{Answer}
\fi


\begin{Exercise}[label=p3_]         
Imena dve datoteke se zadaje na standarnom ulazu.  U prvoj datoteci
navedena je rec {\tt r} i niz linija. Napisati program koji u drugu
datoteku upisuje sve linije u kojima se re\v c {\tt r} pojavljuje bar
{\tt n} puta, gde je n prirodan broj koji se unosi sa standardnog
ulaza. Ispis treba da bude u formatu {\tt broj\_pojavljivanja:
  linija}. Linije brojati po\v cev\v si od {\tt 1}. Maksimalna du\v
zina naziva datoteka je 20 karaktera.
\linkresenje{p3_}
\end{Exercise}
\ifresenja
\begin{Answer}[ref=p3_]
%\includecode{resenja/3_Datoteke/6/6.c}
\end{Answer}
\fi


\begin{Exercise}[label=p3_]         
Napisati program koji poredi dva fajla i ispisuje redni broj linija u
kojima se fajlovi razlikuju.  Imena fajlova se zadaju kao argumenti
komandne linije. U slu\v caju neuspe\v snog otvaranja datoteka
ispisati poruku o gre\v sci. Pretpostaviti da je maksimalna du\v zina
reda u datoteci 200 karaktera. Ukoliko nisu zadati potrebni argumenti
komadne linije ispisati poruku o gre\v sci. Linije brojati pov cev\v
si od {\tt 1}. \\
\begin{miditest}
\begin{upotreba}{1}
#\poziv{./a.out ulaz.txt izlaz.txt}#
#\naslovDat{ulaz.txt}#
#\datoteka{danas vezbamo}#
#\datoteka{programiranje}#
#\datoteka{ovo je primer kad su}#
#\datoteka{datoteke iste}#
#\naslovDat{izlaz.txt: }#
#\datoteka{danas vezbamo}#
#\datoteka{programiranje}#
#\datoteka{ovo je primer kad su}#
#\datoteka{datoteke iste}#
#\naslovIzlaz#
#\izlaz{}#
\end{upotreba}
\end{miditest}
\begin{miditest}
\begin{upotreba}{2}
#\poziv{./a.out primer1.dat primer2.dat}#
#\naslovDat{primer1.dat}#
#\datoteka{danas vezbamo}#
#\datoteka{analizu}#
#\datoteka{ovo je primer kad}#
#\datoteka{su datoteke razlicite}#
#\naslovDat{priemr2.dat}#
#\datoteka{danas vezbamo}#
#\datoteka{programiranje}#
#\datoteka{ovo je primer kad su}#
#\datoteka{datoteke razlicite}#
#\naslovIzlaz#
#\izlaz{2 3 4}#
\end{upotreba}
\end{miditest}
\begin{miditest}
\begin{upotreba}{3}
#\poziv{./a.out prva.dat}#
#\naslovIzlaz#
#\izlaz{greska}#
\end{upotreba}
\end{miditest}
\begin{miditest}
\begin{upotreba}{2}
#\poziv{./a.out prva.dat druga.dat}#
#\naslovDat{prva.dat}#
#\datoteka{ovo je primer}#
#\datoteka{kada su}#
#\datoteka{datoteke}#
#\datoteka{razlicite duzine}#
#\naslovDat{druga.dat}#
#\datoteka{kada su }#
#\datoteka{programiranje}#
#\datoteka{datoteke}#
#\datoteka{razlicite}#
#\datoteka{duzine}#
#\datoteka{i kada treba ispisati broj}#
#\datoteka{tih redova}#
#\naslovIzlaz#
#\izlaz{1 4 5 6 7}#
\end{upotreba}
\end{miditest}
\linkresenje{p3_}
\end{Exercise}
\ifresenja
\begin{Answer}[ref=p3_]
%\includecode{resenja/3_Datoteke/6/6.c}
\end{Answer}
\fi



\begin{Exercise}[label=p3_]         
Definisati strukturu 
\begin{verbatim}
typedef struct{
    unsigned int a, b;
    char ime[5];
}_pravougaonik;
\end{verbatim}
kojom se opisuje pravougaonik du\v zinama svojih stranica i
imenom. Napisati program koji iz datoteke \v cije ime se zadaje kao
argument komandne linije u\v citava pravougaonike (nepoznato koliko),
a zatim ispisuje imena onih pravougaonika koji su kvadrati i vrednost
najve\' ce povr\v sine medju pravougaonicima koji nisu kvadrati.  U
slu\v caju unosa nekorektnih du\v zina stranica pravougaonika ili
nekorektne vrednosti broja \verb|n|, ispisati -1 i odmah prekinuti
izvr\v savanje programa.  Maksimalan broj pravougaonika je 200. \\
\begin{miditest}
\begin{upotreba}{1}
#\poziv{./a.out pravougaonici.dat}#
#\naslovDat{pravougaonici.dat}#
#\datoteka{2 4 p1}#
#\datoteka{3 3 p2}#
#\datoteka{1 6 p3}#
#\naslovIzlaz#
#\izlaz{p2 8}#
\end{upotreba}
\end{miditest}
\begin{miditest}
\begin{upotreba}{2}
#\poziv{./a.out dva.dat}#
#\naslovDat{dva.dat}#
#\datoteka{5 2 pm}#
#\datoteka{4 7 pv}#
#\naslovIzlaz#
#\izlaz{28}#
\end{upotreba}
\end{miditest}
\begin{miditest}
\begin{upotreba}{3}
#\poziv{./a.out tri.dat}#
#\naslovDat{tri.dat}#
#\datoteka{5 5 m}#
#\datoteka{3 3 s}#
#\datoteka{8 8 xl}#
#\naslovIzlaz#
#\izlaz{m s xl}#
\end{upotreba}
\end{miditest}
\begin{miditest}
\begin{upotreba}{4}
#\poziv{./a.out primerx.dat}#
#\naslovDat{primerx.dat}#
#\datoteka{9 7 p}#
#\naslovIzlaz#
#\izlaz{63}#
\end{upotreba}
\end{miditest}
\begin{miditest}
\begin{upotreba}{5}
#\poziv{./a.out prazna.dat}#
#\naslovDat{prazna.dat}#
#\naslovIzlaz#
#\izlaz{}#
\end{upotreba}
\end{miditest}
\linkresenje{p3_}
\end{Exercise}
\ifresenja
\begin{Answer}[ref=p3_]
%\includecode{resenja/3_Datoteke/6/6.c}
\end{Answer}
\fi


\begin{Exercise}[label=p3_]         
Ime datoteke dato je kao argument komandne linije. U datoteci se
nalaze otvorene i zatvorene zagrade i jo\v s nekakav tekst. Proveriti
da li su zagrade pravilno uparene. Npr. \verb|ab( cd) ..| odgovor je
\verb|jesu|, a \verb|..)ba()| odgovor je \verb|nisu|. Ukoliko nisu
zadati svi argumenti komadne linije ispisati poruku o gre\v sci. \\
\begin{minitest}
\begin{upotreba}{1}
#\poziv{./a.out zagrade.txt}#
#\naslovDat{zagrade.txt}#
#\datoteka{ab( cd) ..}#
#\datoteka{((3+4)*5+1)*9}#
#\naslovIzlaz#
#\izlaz{jesu}#
\end{upotreba}
\end{minitest}
\begin{minitest}
\begin{upotreba}{2}
#\poziv{./a.out primer2.dat}#
#\naslovDat{primer2.dat}#
#\datoteka{(7+8 }#
#\datoteka{nisu(}#
#\datoteka{uparene}#
#\naslovIzlaz#
#\izlaz{nisu}#
\end{upotreba}
\end{minitest}
\begin{minitest}
\begin{upotreba}{3}
#\poziv{./a.out primer3.dat}#
#\naslovDat{primer3.dat}#
#\datoteka{)) 7 + 6 ((}#
#\naslovIzlaz#
#\izlaz{nisu}#
\end{upotreba}
\end{minitest}
\begin{minitest}
\begin{upotreba}{4}
#\poziv{./a.out}#
#\naslovIzlaz#
#\izlaz{greska}#
\end{upotreba}
\end{minitest}
\linkresenje{p3_}
\end{Exercise}
\ifresenja
\begin{Answer}[ref=p3_]
%\includecode{resenja/3_Datoteke/6/6.c}
\end{Answer}
\fi


\begin{Exercise}[label=p3_]         
Napraviti strukturu \verb|STUDENT| koja sadr\v zi:
\begin{itemize}
\item \verb|ime| (u polju se \v cuva ime i prezime studenta,
  napr. "Marko Markovic", maksimalna du\v zina polja je 100
  karaktera),
\item \verb|oc| (sadr\v zi najvi\v se 10 ocena studenta)
\item \verb|br_ocena| (ukupan broj ocena za studenata)
\item \verb|pr_oc| (prose\v cna ocena)
\end{itemize}
U datoteci se nalaze podaci o studentima. Za svakog studenta unosi se
ime i prezime razdvojeno razmakom (uputstvo: mo\v ze se korisiti
\verb|strcat| da spoji ime i prezime koji se mogu pro\v citati sa
specifikatorom \verb|%s|), a potom ocene koje se zavr\v savaju sa
0. Prona\'ci studenta koji ima najve\'ci prosek i ispisati sve njegove
podatke (prosek ispisati na 2 decimale).  Maksimalan broj studenta je
100. Ime datoteke se zadaje kao argument komandne linije. \\
\begin{minitest}
\begin{upotreba}{1}
#\poziv{./a.out studenti.txt}#
#\naslovDat{studenti.txt}#
#\datoteka{Marko Markovic 5 6 7 8 9 0}#
#\datoteka{Jelena Jankovic 10 10 10 0}#
#\datoteka{Filip Viskovic 10 9 8 7 6 0}#
#\datoteka{Jana Peric 10 10 9 9 8 8 7 7 0}#
#\naslovIzlaz#
#\izlaz{Jelena Jankovic 10 10 10 0 10.00}#
\end{upotreba}
\end{minitest}
\begin{minitest}
\begin{upotreba}{2}
#\poziv{./a.out}#
#\naslovIzlaz#
#\izlaz{greska}#
\end{upotreba}
\end{minitest}
\linkresenje{p3_}
\end{Exercise}
\ifresenja
\begin{Answer}[ref=p3_]
%\includecode{resenja/3_Datoteke/6/6.c}
\end{Answer}
\fi



\begin{Exercise}[label=p3_]         
\begin{description}
\item{a)} Napisati C funkciju \verb|int unesiSkup(char *s, FILE* f)|
  kojom se unosi skup elemenata iz datoteke F. Skup se predstavlja kao
  niz karaktera, pri \v cemu su dozvoljeni elementi skupa mala i
  velika slova abecede, kao i cifre.  Unos se prekida kada se nai\d e
  na znak za novi red ili nedozvoljeni karakter za skup (maksimalan
  broj elemenata skupa je 1000).  Funkcija vra\' ca broj elemenata
  skupa koji su uspesno u\v citani.
\item{b)} Napisati funkciju
  \verb|void prebroj(char *s, int *br_slova,int *br_cifara)| kojom se
  odre\d uje broj slovnih elemenata skupa (velikih ili malih slova)
  kao i broj cifara u skupu.
\item{c)} Napisati glavni program gde se unose podaci o skupu
  elemenata. Ime datoteke se zadaje kao argument komandne linije.  Na
  stadardni izlaz ispisati informacije o broju slova i cifara
  (koristiti funkcije pod a) i b)).
\end{description}
\begin{minitest}
\begin{upotreba}{1}
#\poziv{./a.out skup.txt}#
#\naslovDat{skup.txt}#
#\datoteka{abc56ighj9012hjFGHH  }#
#\naslovIzlaz#
#\izlaz{broj slova: 13}#
#\izlaz{broj cifara: 6}#
\end{upotreba}
\end{minitest}
\begin{minitest}
\begin{upotreba}{2}
#\poziv{./a.out skup2.txt}#
#\naslovDat{skup2.txt}#
#\datoteka{ovdeimamo\$dolar}#
#\naslovIzlaz#
#\izlaz{broj slova: 9}#
#\izlaz{broj cifara: 0}#
\end{upotreba}
\end{minitest}
\begin{minitest}
\begin{upotreba}{3}
#\poziv{./a.out skup3.txt}#
#\naslovDat{skup3.txt}#
#\datoteka{broJ3}#
#\datoteka{ broj5}#
#\naslovIzlaz#
#\izlaz{broj slova: 4}#
#\izlaz{broj cifara: 1}#
\end{upotreba}
\end{minitest}
\begin{minitest}
\begin{upotreba}{4}
#\poziv{./a.out}#
#\naslovIzlaz#
#\izlaz{greska}#
\end{upotreba}
\end{minitest}
\linkresenje{p3_}
\end{Exercise}
\ifresenja
\begin{Answer}[ref=p3_]
%\includecode{resenja/3_Datoteke/6/6.c}
\end{Answer}
\fi



\begin{Exercise}[label=p3_]         
Definisati strukturu 
\begin{verbatim}
typedef struct{
    int x;
    int y;
    int z;
} vektor;
\end{verbatim}
kojom se opisuje trodimenzioni vektor. U datoteci \verb|vektori.txt|
nalazi se nepoznati broj vektora (maksimalno ih mo\v ze biti
200). U\v citati ih u niz i ispisuje na standardnom izlazu koordinate
vektora sa najve\' com du\v zinom. Du\v zina vektora se izra\v cunava
po formuli:
$$|v|= \sqrt{x^2+y^2+z^2}$$ U slu\v caju gre\v ske ispisati -1 i
prekinuti izvr\v savanje programa. \\
\begin{minitest}
\begin{upotreba}{1}
#\naslovDat{vektori.txt}#
#\datoteka{2}#
#\datoteka{4 -1 7}#
#\datoteka{3 1 2}#
#\naslovIzlaz#
#\izlaz{4 -1 7}#
\end{upotreba}
\end{minitest}
\begin{minitest}
\begin{upotreba}{2}
#\naslovDat{vektori.txt}#
#\datoteka{67}#
#\naslovIzlaz#
#\izlaz{-1}#
\end{upotreba}
\end{minitest}
\begin{minitest}
\begin{upotreba}{3}
#\naslovDat{vektori.txt}#
#\datoteka{3}#
#\datoteka{0 0 0}#
#\datoteka{0 1 0}#
#\datoteka{1 0 0}#
#\naslovIzlaz#
#\izlaz{0 1 0}#
\end{upotreba}
\end{minitest}
\begin{minitest}
\begin{upotreba}{4}
#\naslovDat{vektori.txt}#
#\datoteka{4}#
#\datoteka{3 0 1}#
#\datoteka{4 5 2}#
#\datoteka{1 0 0}#
#\datoteka{2 -1 2}#
#\naslovIzlaz#
#\izlaz{4 5 2}#
\end{upotreba}
\end{minitest}
\linkresenje{p3_}
\end{Exercise}
\ifresenja
\begin{Answer}[ref=p3_]
%\includecode{resenja/3_Datoteke/6/6.c}
\end{Answer}
\fi


\begin{Exercise}[label=p3_]         
Prvi red datoteke \verb|ulaz.txt| sadr\v zi 2 cela broja manja od 50
koji predstavljaju redom broj vrsta i broj kolona realne matrice
A. Svaki slede\'ci red sadr\v zi po jednu vrstu matrice. Napisati
program koji nalazi i \v stampa sve \v cetvorke oblika
\verb|(A(i,j), A(i+1,j), A(i,j+1),A(i+1,j+1))| u kojima su svi
elementi međusobno razli\v citi.
\linkresenje{p3_}
\end{Exercise}
\ifresenja
\begin{Answer}[ref=p3_]
%\includecode{resenja/3_Datoteke/6/6.c}
\end{Answer}
\fi



\ifresenja
\section{Rešenja}
\shipoutAnswer
\fi