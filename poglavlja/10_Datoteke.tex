\chapter{Datoteke}

\begin{Exercise}[label=v3_01] 
Tekst
\linkresenje{v3_01}
\end{Exercise}
\begin{Answer}[ref=v3_01]
\includecode{resenja/3_Datoteke/1/1.c}
\end{Answer}

\begin{Exercise}[label=v3_02] 
Tekst
\linkresenje{v3_02}
\end{Exercise}
\begin{Answer}[ref=v3_02]
\includecode{resenja/3_Datoteke/2/2.c}
\end{Answer}

\begin{Exercise}[label=v3_03] 
Tekst
\linkresenje{v3_03}
\end{Exercise}
\begin{Answer}[ref=v3_03]
\includecode{resenja/3_Datoteke/3/3.c}
\end{Answer}

\begin{Exercise}[label=v3_04] 
Tekst
\linkresenje{v3_04}
\end{Exercise}
\begin{Answer}[ref=v3_04]
\includecode{resenja/3_Datoteke/4/4.c}
\end{Answer}

\begin{Exercise}[label=v3_05] 
Tekst
\linkresenje{v3_05}
\end{Exercise}
\begin{Answer}[ref=v3_05]
\includecode{resenja/3_Datoteke/5/5.c}
\end{Answer}

\begin{Exercise}[label=v3_06] 
Tekst
\linkresenje{v3_06}
\end{Exercise}
\begin{Answer}[ref=v3_06]
\includecode{resenja/3_Datoteke/6/6.c}
\end{Answer}







\begin{enumerate}
\item Napisati program koji prebrojava mala slova u datoteci $test.txt$. \\
\begin{miditest}
\begin{upotreba}{1}
#\naslovDat{test.txt}#
#\datoteka{Abcd EFGH+ijKLMN}#

#\naslovIzlaz#
#\izlaz{Broj malih slova je: 5}#
\end{upotreba}
\end{miditest}
\begin{miditest}
\begin{upotreba}{2}
#\naslovDat{test.txt}#
#\datoteka{PrograMiranje}#

#\naslovIzlaz#
#\izlaz{Broj malih slova je: 11}#
\end{upotreba}
\end{miditest}

\item Napisati program koji prepisuje svaki treći karakter datoteke $ulaz.txt$ u datoteku $izlaz.txt$.\\
\begin{miditest}
\begin{upotreba}{1}
#\naslovDat{ulaz.txt}#
#\datoteka{Volim programiranje.}#
#\naslovDat{izlaz.txt}#
#\datoteka{Vipgmae}#
\end{upotreba}
\end{miditest}

\item Kao argumenti komandne linije se zadaju ime datoteke i ceo broj $k$. Napisati program koji na  standardni izlaz ispisuje sve linije zadate datoteke čija je dužina veća od k. Može se pretpostaviti da dužina linije neće biti veća od 80 karaktera.\\
\begin{miditest}
\begin{upotreba}{1}
#\poziv{./a.out test.txt 7}#
#\naslovDat{test.txt}#
#\datoteka{Teme koje su obradjivane:}#
#\datoteka{Petlje}#
#\datoteka{Funkcije}#
#\datoteka{Nizovi}#
#\datoteka{Strukture}#

#\naslovIzlaz#
#\izlaz{Teme koje su obradjivane:}#
#\izlaz{Funkcije}#
#\izlaz{Strukture}#
\end{upotreba}
\end{miditest}
\begin{miditest}
\begin{upotreba}{2}
#\poziv{./a.out test.txt}#

#\naslovIzlaz#
#\izlaz{Greska: Pogresan broj argumenata!}#
\end{upotreba}
\end{miditest}

\item Napisati program koji prebrojava koliko se linija datoteke $ulaz.txt$ završava niskom $s$ koja se učitava sa standardnog ulaza. Može se pretpostaviti da dužina linije neće biti veća od 80 karaktera, kao i da dužina niske $s$ neće biti veća od 20 karaktera.\\
\begin{miditest}
\begin{upotreba}{1}
#\naslovDat{ulaz.txt}#
#\datoteka{abcde abcde}#
#\datoteka{abcde aab}#
#\datoteka{abcde abcde abcde}#
#\datoteka{abcde abcde Aab}#
#\datoteka{abcde abcde ab}#
#\datoteka{abcde abcde abcde abcde}#

#\naslovInt#
#\izlaz{Unesite nisku s:}\ulaz{ab}#
#\izlaz{Broj linija: 3}#
\end{upotreba}
\end{miditest}
\begin{miditest}
\begin{upotreba}{2}
#\naslovDat{ulaz.txt}#
#\datoteka{abcde abcde}#
#\datoteka{abcde}#
#\datoteka{abcde abcde AB}#

#\naslovInt#
#\izlaz{Unesite nisku s:}\ulaz{ab}#
#\izlaz{Broj linija: 0}#
\end{upotreba}
\end{miditest}

\item Napisati program koji pronalazi maksimum brojeva zapisanih u datoteci $brojevi.txt$. \\
\begin{miditest}
\begin{upotreba}{1}
#\naslovDat{brojevi.txt}#
#\datoteka{2.36 -16.11 5.96 8.88}#
#\datoteka{-265.31 54.96 38.4}#

#\naslovIzlaz#
#\izlaz{Najveci broj je: 54.96}#
\end{upotreba}
\end{miditest}

\item U datoteci $studenti.txt$ se nalaze informacije o studentima: prvo broj studenata, a zatim u pojedinačnim linijama korisničko ime i pet poslednjih ocena koje je student dobio. Napisati program koji pronalazi studenta koji je ostvario najbolji uspeh i ispisuje njegove podatke. Pretpostaviti da broj studenata neće biti veći od 100.\\
\begin{miditest}
\begin{upotreba}{1}
#\naslovDat{studenti.txt}#
#\datoteka{mr15239 10 9 9 8 10}#
#\datoteka{mi14005 8 8 9 8 10}#
#\datoteka{ml15112 9 8 8 7 10}#
#\datoteka{mr15007 10 10 10 10 10}#
#\datoteka{mn13208 7 7 9 6 10}#

#\naslovIzlaz#
#\izlaz{korisnicko ime: mr15007, prosek ocena: 10.00}#
\end{upotreba}
\end{miditest}

\item U datoteci $tacke.txt$ se nalazi prvo broj tačaka, a zatim u pojedinačnim linijama x i y koordinate tačke. Napisati program koji u datoteku $rastojanja.txt$ upisuje rastojanje svake od pročitanih tačaka od koordinatnog početka, a na standardni izlaz koordinate tačke koja je najudaljenija. Koristiti strukturu $Tacka$ sa poljima $x$ i $y$, kao i funkciju kojom se računa rastojanje. Pretpostaviti da broj tačaka u datoteci neće biti veći od 50. \\  
\begin{miditest}
\begin{upotreba}{1}
#\naslovDat{tacke.txt}#
#\datoteka{4}#
#\datoteka{11 -2}#
#\datoteka{3 5}#
#\datoteka{8 -8}#
#\datoteka{0 4}#

#\naslovDat{rastojanja.txt}#
#\datoteka{11.18}#
#\datoteka{5.29}#
#\datoteka{11.31}#
#\datoteka{4.00}#

#\naslovIzlaz#
#\izlaz{Najudaljenija je tačka: 8 -8}#
\end{upotreba}
\end{miditest}
\begin{miditest}
\begin{upotreba}{1}
#\naslovDat{tacke.txt}#
#\datoteka{-2}#
#\datoteka{0 0}#
#\datoteka{9 -8}#

#\naslovIzlaz#
#\izlaz{Greska: Nedozvoljen broj tacaka!}#
\end{upotreba}
\end{miditest}

\item Napisati program koji za reč $s$ maksimalne dužine 20 karaktera koja se zadaje sa standardnog ulaza u datoteku $rotacije.txt$ upisuje sve rotacije reči $s$. \\
\begin{miditest}
\begin{upotreba}{1}
#\naslovInt#
#\izlaz{Unesite rec: }\ulaz{abcde}#

#\naslovDat{rotacije.txt}#
#\datoteka{abcde}#
#\datoteka{bcdea}#
#\datoteka{cdeab}#
#\datoteka{deabc}#
#\datoteka{eabcd}#
\end{upotreba}
\end{miditest}


\item Napisati program koji linije koji se učitavaju sa standardnog ulaza sve do kraja ulaza prepisuje u datoteku $izlaz.txt$ i to, ako je prilikom pokretanja zadata opcija \textit{-v} ili \textit{-V} samo one linije koje počinju velikim slovom, ako je zadata opcija \textit{-m} ili \textit{-M} samo one linije koje počinju malim slovom, a ako je opcija izostavljena sve linije. Pretpostaviti da linije neće biti duže od 80 karaktera. \\
\begin{miditest}
\begin{upotreba}{1}
#\poziv{./a.out -m}#
#\naslovInt#
#\izlaz{Unesite recenice: }#
#\ulaz{programiranje u C-u je zanimljivo}#
#\ulaz{Volim programiranje!}#
#\ulaz{Kada porastem bicu programer!}#
#\ulaz{u slobodno vreme programiram}#

#\naslovDat{izlaz.txt}#
#\datoteka{programiranje u C-u je zanimljivo}#
#\datoteka{u slobodno vreme programiram}#
\end{upotreba}
\end{miditest}
\begin{miditest}
\begin{upotreba}{2}
#\poziv{./a.out -V}#
#\naslovInt#
#\izlaz{Unesite recenice: }#
#\ulaz{programiranje u C-u je zanimljivo}#
#\ulaz{Volim programiranje!}#
#\ulaz{Kada porastem bicu programer!}#
#\ulaz{u slobodno vreme programiram}#

#\naslovDat{izlaz.txt}#
#\datoteka{Volim programiranje!}#
#\datoteka{Kada porastem bicu programer!}#
\end{upotreba}
\end{miditest}

\begin{miditest}
\begin{upotreba}{3}
#\poziv{./a.out -k}#
#\naslovInt#
#\izlaz{Greska: Pogresno pokretanje programa!}#
\end{upotreba}
\end{miditest}



\end{enumerate}




\section{Rešenja}
\shipoutAnswer
