
\chapter{Ulaz i izlaz programa}

%\section{Standardni tokovi}

%\section{Argumenti komandne linije}

\section{Datoteke}
\komentarJ{
poruke o gresci:\\
- ako je dat fiksan naziv datoteke, da li u test primere stavljamo slucaj kada datoteka ne postoji?\\
- da li je poruka o gresci uvek -1 ili da navodimo customized?\\
- da li naglasavati da je tekstualna datoteka (u zadacima sa i smera nekad stoji na pocetku zadatka 'tekstualna datoteka', kod nas ne\\
- kada treba da se ispise minimum ili maksimum nekakvog niza, ako nije precizirano sta se radi ako vise elemenata ima maksimalnu/minimalnu vrednost i nema resenja, da li da preciziramo u zadatku?)
}
%% ----------- KARAKTER PO KARAKTER POCETAK ----------

\subsection{Unos karakter po karakter}

\begin{Exercise}[label=v3_01] 
Napisati program koji prepisuje sadržaj datoteke $ulaz.txt$ u datoteku $izlaz.txt$ karakter po karakter.
\\
\komentarJ{v301, karakter po karakter, prepisivanje, fiksni naziv}
\begin{miditest}
\begin{upotreba}{1}
#\naslovDat{ulaz.txt}#
#\datoteka{Danas je 21. mart.}#
#\datoteka{To je prvi dan proleca.}#
#\naslovDat{izlaz.txt}#
#\datoteka{Danas je 21. mart.}#
#\datoteka{To je prvi dan proleca.}#
\end{upotreba}
\end{miditest}
\begin{miditest}
\begin{upotreba}{2}
#\naslovDat{ulaz.txt}#
#\datoteka{Ispit iz Programiranja 1 je zakazan za 10. jun.}#
#\naslovDat{izlaz.txt}#
#\datoteka{Ispit iz Programiranja 1 je zakazan za 10. jun.}#
\end{upotreba}
\end{miditest}
\linkresenje{v3_01}
\end{Exercise}
\begin{Answer}[ref=v3_01]
\includecode{resenja/3_Datoteke/1/1.c}
\end{Answer}

\begin{Exercise}[label=p3_02] 
Napisati program koji prepisuje svaki treći karakter datoteke $ulaz.txt$ u datoteku $izlaz.txt$.\\
\komentarJ{p302, karakter po karakter, prepisivanje, fiksni naziv}
\begin{minitest}
\begin{upotreba}{1}
#\naslovDat{ulaz.txt}#
#\datoteka{Volim programiranje.}#
#\naslovDat{izlaz.txt}#
#\datoteka{Vipgmae}#
\end{upotreba}
\end{minitest}
\begin{minitest}
\begin{upotreba}{2}
#\naslovDat{ulaz.txt}#
#\datoteka{abcdefghi}#
#\datoteka{123456789}#
#\naslovDat{izlaz.txt}#
#\datoteka{adg}#
#\datoteka{147}#
\end{upotreba}
\end{minitest}
\begin{minitest}
\begin{upotreba}{3}
#\naslovDat{ulaz.txt}#
#\datoteka{U Beogradu ce biti}#
#\datoteka{suncan i lep}#
#\datoteka{dan.}#
#\naslovDat{izlaz.txt}#
#\datoteka{Ueruei}#
#\datoteka{nn pa}#
\end{upotreba}
\end{minitest}
\linkresenje{p3_02}
\end{Exercise}
\begin{Answer}[ref=p3_02]
\includecode{resenja/3_Datoteke/praktikumi14/1_02.c}
\end{Answer}

\begin{Exercise}[label=p3_id17]         
Napisati program koji \v sifrira sadr\v zaj datoteke $plain.txt$ tako što svako slovo cikli\v cno zamenjuje njegovim prethodnikom suprotne veličine i upisuje u datoteku $sifra.txt$. Na primer, \verb|’b’| se zamenjuje sa \verb|’A’|,
\verb|’B’| sa \verb|’a’|, \verb|’a’| sa \verb|’Z’|, \verb|’A’| sa
\verb|’z’|, itd. Ostali karakteri ostaju nepromenjeni. Podrazumevati da se na sistemu koristi tabela
karaktera ASCII. U slu\v caju da datoteka $plain.txt$ ne postoji, napisati na standardni izlaz poruku o gre\v sci. \\
\komentarJ{p3id17, karakter po karakter, prepisivanje, fiksno ime, nema resenje}
\komentarJ{da li moramo da naglasavamo ovo za ASCII?}
\begin{minitest}
\begin{upotreba}{1}
#\naslovDat{plain.txt}#
#\datoteka{ABC.123.xyz}#
#\naslovDat{sifra.txt}#
#\datoteka{zab.123.WXY}#
\end{upotreba}
\end{minitest}
\begin{minitest}
\begin{upotreba}{2}
#\naslovDat{plain.txt}#
#\datoteka{a=x+y;}#
#\datoteka{x=b+5;}#
#\naslovDat{sifra.txt}#
#\datoteka{Z=W+X;}#
#\datoteka{W=A+5;}#
\end{upotreba}
\end{minitest}
\begin{minitest}
\begin{upotreba}{3}
#\naslovDat{plain.txt ne postoji}#
#\naslovIzlaz#
#\izlaz{Ulazna datoteka ne postoji}
\end{upotreba}
\end{minitest}
\linkresenje{p3_id17}
\end{Exercise}
\begin{Answer}[ref=p3_id17]
%\includecode{resenja/3_Datoteke/6/6.c}
\end{Answer}

\begin{Exercise}[label=p3_iv3] 
Sa standarnog ulaza učitavaju se imena dve tekstualne datoteke i
jedan karakter.  Napisati program koji prepisuje sadr\v zaj prve datoteke u drugu na slede\' ci na\v cin:
\begin{itemize}
\item ukoliko je učitan karakter \verb|u|, sva mala slova zamenjuje velikim
\item ukoliko je učitan karakter \verb|l|, sva velika slova zamenjuje malim
\end{itemize}
U slučaju greške ispisati -1. Greška može biti neuspešno
otvaranje datoteke ili pogrešno zadat karakter. Maksimalna du\v
zina naziva datoteka je 20 karaktera. \\
\komentarJ{p3iv3, karakter po karakter, prepisivanje, ima resenje}
\komentarJ{da li da izbrisemo sta moze biti greska?}
\begin{minitest}
\begin{upotreba}{1}
#\naslovInt#
#\ulaz{ulaz.txt izlaz.txt u}#
#\naslovDat{ulaz.txt}#
#\datoteka{danas je lep dan}#
#\datoteka{i Ja zelim}#
#\datoteka{da postanem programer}#
#\naslovDat{izlaz.txt}#
#\datoteka{DANAS JE LEP DAN}#
#\datoteka{I JA ZELIM}#
#\datoteka{DA POSTANEM PROGRAMER}#
\end{upotreba}
\end{minitest}
\begin{minitest}
\begin{upotreba}{2}
#\naslovInt#
#\ulaz{prva.dat druga.dat l}#
#\naslovDat{prva.dat}#
#\datoteka{Cena soka je 30}#
#\datoteka{Cena vina je 150}#
#\datoteka{Cena limunade je 200}#
#\datoteka{Cena sendvica je 120}#
#\naslovDat{druga.dat}#
#\datoteka{cena soka je 30}#
#\datoteka{cena vina je 150}#
#\datoteka{cena limunade je 200}#
#\datoteka{cena sendvica je 120}#
\end{upotreba}
\end{minitest}
\begin{minitest}
\begin{upotreba}{3}
#\naslovInt#
#\ulaz{primer.c prazna.txt V}#
#\naslovDat{primer.c}#
#\datoteka{\#include <stdio.h>}#
#\datoteka{int main()}#
#\datoteka{\{}#
#\datoteka{\}}#
#\naslovDat{prazna.txt}#
#\datoteka{}#
#\naslovIzlaz#
#\izlaz{-1}#
\end{upotreba}
\end{minitest}
\linkresenje{p3_iv3}
\end{Exercise}
\begin{Answer}[ref=p3_i1]
%\includecode{resenja/3_Datoteke/6/6.c}
\end{Answer}


\begin{Exercise}[label=p3_id16]         
Napisati program koji za dve datoteke čija se imena unose sa standarnog ulaza, radi sledeće:
\begin{itemize}
\item{za svaku cifru u prvoj datoteci,
u drugu datoteku upisuje 0}
\item{za svako slovo u prvoj datoteci, u drugu datoteku upisuje 1}
\item{za sve ostale
karaktere u prvoj datoteci, u drugu datoteku upisuje 2}
\end{itemize}
Maksimalna dužina naziva datoteka je 20
karaktera. U slu\v caju da prva datoteka ne postoji, napisati na standardni izlaz poruku o gre\v sci. \\
\komentarJ{p3id16, karakter po karakter, prepisivanje, nema resenje}
\komentarJ{Dodat je deo zadatka za slucaj da datoteka ne postoji.}
\komentarJ{ispraviti ovoliki prored izmedju stavki u okviru itemize (to se menja u zavisnosti od prostora na stranici)}
\begin{minitest}
\begin{upotreba}{1}
#\naslovInt#
#\ulaz{prva.dat druga.dat}#
#\naslovDat{prva.dat}#
#\datoteka{abc.123.[]}#
#\datoteka{567.ABC.{}}#
#\naslovDat{druga.dat}#
#\datoteka{111200022220002111222}
\end{upotreba}
\end{minitest}
\begin{minitest}
\begin{upotreba}{2}
#\naslovInt#
#\ulaz{ulaz.txt izlaz.txt}#
#\naslovDat{ulaz.txt}#
#\datoteka{18. februar 2019.}#
#\naslovDat{izlaz.txt}#
#\datoteka{11220000000211112}
\end{upotreba}
\end{minitest}
\begin{minitest}
\begin{upotreba}{3}
#\naslovInt#
#\ulaz{in.txt out.txt}#
#\naslovDat{in.txt ne postoji}#
#\naslovIzlaz#
#\izlaz{Ulazna datoteka ne postoji}
\end{upotreba}
\end{minitest}
\linkresenje{p3_}
\end{Exercise}
\begin{Answer}[ref=p3_]
%\includecode{resenja/3_Datoteke/6/6.c}
\end{Answer}






\begin{Exercise}[label=p3_01] 
Napisati program koji prebrojava mala slova u datoteci $test.txt$. \\
\komentarJ{p301, karakter po karakter, brojanje, fiksni naziv}
\begin{minitest}
\begin{upotreba}{1}
#\naslovDat{test.txt}#
#\datoteka{Abcd EFGH+ijKLMN}#

#\naslovIzlaz#
#\izlaz{Broj malih slova je: 5}#
\end{upotreba}
\end{minitest}
\begin{minitest}
\begin{upotreba}{2}
#\naslovDat{test.txt}#
#\datoteka{PrograMiranje}#

#\naslovIzlaz#
#\izlaz{Broj malih slova je: 11}#
\end{upotreba}
\end{minitest}
\begin{minitest}
\begin{upotreba}{3}
#\naslovDat{test.txt}#
#\datoteka{123456}#
#\datoteka{ABCDEF}#

#\naslovIzlaz#
#\izlaz{Broj malih slova je: 0}#
\end{upotreba}
\end{minitest}
\linkresenje{p3_01}
\end{Exercise}
\begin{Answer}[ref=p3_01]
\includecode{resenja/3_Datoteke/praktikumi14/1_01.c}
\end{Answer}


\begin{Exercise}[label=p3_id14]         
Napisati program koji u datoteci \v cije se ime unosi sa standardnog ulaza prebrojava koliko se puta svaka cifra pojavljuje i na standardni izlaz ispisuje cifru sa najve\' cim brojem pojavljivanja. Ukoliko ima vi\v se takvih cifara, ispisati sve. U slučaju da nema cifara u datoteci ili u slu\v caju greške pri otvaranju, ispisati
{\tt -1} na standardni izlaz. Maksimalna dužina naziva datoteka je 20 karaktera. \\
\komentarJ{p2id14, karakter po karakter, brojanje, nema resenje (domaci)}
\komentarJ{opet, da obrisemo o kojoj se gresci radi, da stavimo samo u slucaju gresaka?}
\begin{minitest}
\begin{upotreba}{1}
#\naslovInt#
#\ulaz{ulaz.txt}#
#\naslovDat{ulaz.txt}#
#\datoteka{danas je lep dan}#
#\datoteka{i ja zelim}#
#\datoteka{da postanem programer}#
#\naslovIzlaz#
#\izlaz{-1}#
\end{upotreba}
\end{minitest}
\begin{minitest}
\begin{upotreba}{2}
#\naslovInt#
#\ulaz{prva.dat}#
#\naslovDat{prva.dat}#
#\datoteka{Cena soka je 30}#
#\datoteka{Cena vina je 150}#
#\datoteka{Cena limunade je 200}#
#\datoteka{Cena sendvica je 120}#
#\naslovIzlaz#
#\izlaz{0}#
\end{upotreba}
\end{minitest}
\begin{minitest}
\begin{upotreba}{3}
#\naslovInt#
#\ulaz{primer.c}#
#\naslovDat{primer.c}#
#\datoteka{1 22 333.444 }#
#\naslovIzlaz#
#\izlaz{3 4}#
\end{upotreba}
\end{minitest}
\linkresenje{p3_id14}
\end{Exercise}
\begin{Answer}[ref=p3_id14]
%\includecode{resenja/3_Datoteke/6/6.c}
\end{Answer}


\begin{Exercise}[label=p3_x2]         
Napisati program koji u datoteci \v cije je ime dato kao argument komandne linije proverava da li su zagrade pravilno uparene. Ukoliko nisu
zadati svi argumenti komadne linije, ispisati poruku o grešci na standardni izlaz. \\
\komentarJ{p3x2, argumenti, karakter po karakter, brojanje, nema resenja}
\begin{miditest}
\begin{upotreba}{1}
#\poziv{./a.out zagrade.txt}#
#\naslovDat{zagrade.txt}#
#\datoteka{ab( cd) ..}#
#\datoteka{((3+4)*5+1)*9}#
#\naslovIzlaz#
#\izlaz{jesu}#
\end{upotreba}
\end{miditest}
\begin{miditest}
\begin{upotreba}{2}
#\poziv{./a.out primer2.dat}#
#\naslovDat{primer2.dat}#
#\datoteka{(7+8 }#
#\datoteka{nisu(}#
#\datoteka{uparene}#
#\naslovIzlaz#
#\izlaz{nisu}#
\end{upotreba}
\end{miditest}
\begin{miditest}
\begin{upotreba}{3}
#\poziv{./a.out primer3.dat}#
#\naslovDat{primer3.dat}#
#\datoteka{)) 7 + 6 ((}#
#\naslovIzlaz#
#\izlaz{nisu}#
\end{upotreba}
\end{miditest}
\begin{miditest}
\begin{upotreba}{4}
#\poziv{./a.out}#
#\naslovIzlaz#
#\izlaz{greska}#
\end{upotreba}
\end{miditest}
\linkresenje{p3_x2}
\end{Exercise}
\begin{Answer}[ref=p3_x2]
%\includecode{resenja/3_Datoteke/6/6.c}
\end{Answer}

\begin{Exercise}[label=p3_x4]         
Potrebno je napisati program koji prebrojava slova i cifre u datoteci.
\begin{description}
\item{a)} Napisati C funkciju \verb|int unesiSkup(char *s, FILE* f)|
  kojom se unosi skup elemenata iz datoteke \verb|F|. Skup se predstavlja kao
  niz karaktera, pri čemu su dozvoljeni elementi skupa mala i
  velika slova abecede, kao i cifre.  Unos se prekida kada se naiđe
  na znak za novi red ili nedozvoljeni karakter za skup (maksimalan
  broj elemenata skupa je 1000).  Funkcija vraća broj elemenata
  skupa koji su uspešno učitani.
\item{b)} Napisati funkciju
  \verb|void prebroj(char *s, int *br_slova,int *br_cifara)| kojom se
  određuje broj slovnih elemenata skupa (velikih ili malih slova)
  kao i broj cifara u skupu.
\item{c)} Napisati glavni program koji koriste\' ci prethodne funkcije prebrojava cifre i slova u datoteci \v cije se ime unosi kao argument komandne linije i ispisuje dobijene vrednosti na standardni izlaz. U slu\v caju gre\v ske, ispisati -1 na standardni izlaz.
\end{description}
\komentarJ{x4, karakter, brojanje, argumenti, nema resenje}
\begin{minitest}
\begin{upotreba}{1}
#\poziv{./a.out skup.txt}#
#\naslovDat{skup.txt}#
#\datoteka{abc56ighj9012hjFGHH  }#
#\naslovIzlaz#
#\izlaz{broj slova: 13}#
#\izlaz{broj cifara: 6}#
\end{upotreba}
\end{minitest}
\begin{minitest}
\begin{upotreba}{2}
#\poziv{./a.out skup2.txt}#
#\naslovDat{skup2.txt}#
#\datoteka{ovdeimamo\$dolar}#
#\naslovIzlaz#
#\izlaz{broj slova: 9}#
#\izlaz{broj cifara: 0}#
\end{upotreba}
\end{minitest}
\begin{minitest}
\begin{upotreba}{3}
#\poziv{./a.out skup3.txt}#
#\naslovDat{skup3.txt}#
#\datoteka{broJ3}#
#\datoteka{ broj5}#
#\naslovIzlaz#
#\izlaz{broj slova: 4}#
#\izlaz{broj cifara: 1}#
\end{upotreba}
\end{minitest}
\begin{minitest}
\begin{upotreba}{4}
#\poziv{./a.out skup4.txt}#
#\naslovDat{skup4.txt}#
#\datoteka{11.2.2019.}#
#\naslovIzlaz#
#\izlaz{broj slova: 0}#
#\izlaz{broj cifara: 2}#
\end{upotreba}
\end{minitest}
\begin{minitest}
\begin{upotreba}{5}
#\poziv{./a.out skup5.txt}#
#\naslovDat{skup4.txt ne postoji}#
#\naslovIzlaz#
#\izlaz{-1}#
\end{upotreba}
\end{minitest}
\begin{minitest}
\begin{upotreba}{6}
#\poziv{./a.out}#
#\naslovIzlaz#
#\izlaz{-1}#
\end{upotreba}
\end{minitest}
\linkresenje{p3_x4}
\end{Exercise}
\begin{Answer}[ref=p3_x4]
%\includecode{resenja/3_Datoteke/6/6.c}
\end{Answer}


%% ----------- KARAKTER PO KARAKTER KRAJ ----------

%% ----------- REC PO REC POCETAK ----------

\subsection{Rec po rec}

\begin{Exercise}[label=p3_08] 
 Napisati program koji za reč $s$ maksimalne dužine 20 karaktera koja se zadaje sa standardnog ulaza u datoteku $rotacije.txt$ upisuje sve rotacije reči $s$. \\
\komentarJ{p308, rec po rec, fiksno ime}
\begin{minitest}
\begin{upotreba}{1}
#\naslovInt#
#\izlaz{Unesite rec: }\ulaz{abcde}#

#\naslovDat{rotacije.txt}#
#\datoteka{abcde}#
#\datoteka{bcdea}#
#\datoteka{cdeab}#
#\datoteka{deabc}#
#\datoteka{eabcd}#
\end{upotreba}
\end{minitest}
\begin{minitest}
\begin{upotreba}{2}
#\naslovInt#
#\izlaz{Unesite rec: }\ulaz{1234}#

#\naslovDat{rotacije.txt}#
#\datoteka{1234}#
#\datoteka{2341}#
#\datoteka{3412}#
#\datoteka{4123}#
\end{upotreba}
\end{minitest}
\begin{minitest}
\begin{upotreba}{3}
#\naslovInt#
#\izlaz{Unesite rec: }\ulaz{a=3*x+5;}#

#\naslovDat{rotacije.txt}#
#\datoteka{a=3*x+5;}#
#\datoteka{=3*x+5;a}#
#\datoteka{3*x+5;a=}#
#\datoteka{*x+5;a=3}#
#\datoteka{x+5;a=3*}#
#\datoteka{+5;a=3*x}#
#\datoteka{5;a=3*x+}#
#\datoteka{;a=3*x+5}#
\end{upotreba}
\end{minitest}
\linkresenje{p3_08}
\end{Exercise}
\begin{Answer}[ref=p3_08]
\includecode{resenja/3_Datoteke/praktikumi14/1_08.c}
\end{Answer}

\begin{Exercise}[label=p3_id18]         
Sa standarnog ulaza se učitava ime tekstualne datoteke i prirodan
broj $k$. Podrazumeva se da zadata datoteka sadrži samo slova i
beline i da je svaka reč iz datoteke dužine najviše
100. Program treba da učitava reči iz datoteke, da svaku reč
rotira za $k$ mesta i da tako dobijenu reč upiše u datoteku čije je ime $rotirano.txt$. Maksimalna dužina naziva datoteke
je 20 karaktera. \\
\komentarJ{p3id18, rec po rec, ime jedne datoteke sa standardnog ulaza a drugo fiksno ime, nema resenje}
\begin{miditest}
\begin{upotreba}{1}
#\naslovInt#
#\izlaz{Unesite ime datoteke: }\ulaz{ulaz.txt}#
#\izlaz{Unesite prirodan broj: }\ulaz{3}#
#\naslovDat{ulaz.txt}#
#\datoteka{jedan dva}#
#\datoteka{tri cetiri}#
#\naslovDat{rotirano.txt}#
#\datoteka{anjed dva tri iricet}#
\end{upotreba}
\end{miditest}
\begin{miditest}
\begin{upotreba}{2}
#\naslovInt#
#\izlaz{Unesite ime datoteke: }\ulaz{in.dat}#
#\izlaz{Unesite prirodan broj: }\ulaz{5}#
#\naslovDat{in.dat}#
#\datoteka{Popodne ce biti kise}#
#\naslovDat{rotirano.txt}#
#\datoteka{nePopod ec itib isek}#
\end{upotreba}
\end{miditest}
\begin{miditest}
\begin{upotreba}{3}
#\naslovInt#
#\izlaz{Unesite ime datoteke: }\ulaz{input.txt}#
#\izlaz{Unesite prirodan broj: }\ulaz{0}#
#\naslovDat{input.txt}#
#\datoteka{Popodne ce}#
#\datoteka{biti kise}#
#\naslovDat{rotirano.txt}#
#\datoteka{Popodne ce biti kise}#
\end{upotreba}
\end{miditest}
\begin{miditest}
\begin{upotreba}{4}
#\naslovInt#
#\izlaz{Unesite ime datoteke: }\ulaz{tekst.dat}#
#\izlaz{Unesite prirodan broj: }\ulaz{7}#
#\naslovDat{tekst.dat ne postoji}#
#\naslovIzlaz#
#\izlaz{-1}#
\end{upotreba}
\end{miditest}
\linkresenje{p3_id18}
\end{Exercise}
\begin{Answer}[ref=p3_id18]
%\includecode{resenja/3_Datoteke/6/6.c}
\end{Answer}


\begin{Exercise}[label=p3_iv6]         
Napisati program koji iz datoteke čije se ime zadaje sa standardnog ulaza prepisuje re\v ci na standardni izlaz a one re\v ci koje sadr\v ze prvu re\v c iz datoteke i podvlaku upisuje u datoteku $rez.txt$. Maksimalna dužina naziva
datoteke je 20 karaktera a re\v ci u datoteci 50 karaktera.U slučaju greške ispisati -1. \\
\komentarJ{p3iv6, rec po rec, fiksno ime, ima resenje}
\begin{miditest}
\begin{upotreba}{1}
#\naslovInt#
#\ulaz{dat1.txt}#
#\naslovDat{dat1.txt}#
#\datoteka{rec Opet \_rec Reci rec\_enica}#
#\datoteka{DVa recica\_}#
#\naslovIzlaz#
#\izlaz{rec Opet \_rec Reci rec\_enica }#
#\izlaz{DVa recica\_}#
#\naslovDat{rez.txt: }#
#\datoteka{\_rec rec\_enica recica\_}#
\end{upotreba}
\end{miditest}
\begin{miditest}
\begin{upotreba}{2}
#\naslovInt#
#\ulaz{dat2.txt}#
#\naslovDat{dat2.txt}#
#\datoteka{Sunce sija iznad grada}#
#\datoteka{Sunce\_Moje Jedan Dva Su\_nce Sve Sunce123\_123 suncanica.}#
#\naslovIzlaz#
#\izlaz{Sunce sija iznad grada}#
#\izlaz{Sunce\_Moje Jedan Dva Su\_nce Sve Sunce123\_123 suncanica.}#
#\naslovDat{rez.txt: }#
#\datoteka{Sunce\_Moje Sunce123\_123}#
\end{upotreba}
\end{miditest}
\begin{miditest}
\begin{upotreba}{3}
#\naslovInt#
#\ulaz{dat3.txt}#
#\naslovDat{dat3.txt}#
#\datoteka{4 abc 1234 (5+3)*12\_4 11-k}#
#\naslovIzlaz#
#\izlaz{4 abc 1234 (5+3)*12\_4 11-k}#
#\naslovDat{rez.txt: }#
#\datoteka{(5+3)*12\_4}
\end{upotreba}
\end{miditest}
\begin{miditest}
\begin{upotreba}{4}
#\naslovInt#
#\ulaz{dat4.txt}#
#\naslovDat{dat4.txt ne postoji}#
#\naslovIzlaz#
#\izlaz{-1}#
\end{upotreba}
\end{miditest}
\linkresenje{p3_iv6}
\end{Exercise}
\begin{Answer}[ref=p3_iv6]
%\includecode{resenja/3_Datoteke/6/6.c}
\end{Answer}

\begin{Exercise}[label=p3_id20]         
Napisati program koji iz datoteke $razno.txt$ u datoteku $palindromi.txt$ prepisuje sve palindrome. Podrazumevamo da je reč palindrom ako se čita isto sa leve i desne strane bez obzira na veli\v cinu slova. Maksimalna du\v zina re\v ci je 200 karaktera a maksimalan broj reči nije poznat. U slučaju greške
ispisati -1 i prekinuti izvršavanje programa. \\
\komentarJ{p3id20, rec po rec, fiksno ime, nema resenje}
\begin{miditest}
\begin{upotreba}{1}
#\naslovDat{razno.txt}#
#\datoteka{Ana i melem su primeri palindroma.}#
#\naslovDat{palindromi.txt: }#
#\datoteka{Ana i melem}
\end{upotreba}
\end{miditest}
\begin{miditest}
\begin{upotreba}{2}
#\naslovDat{razno.txt}#
#\datoteka{jabuka neven pomorandza kuk}#
#\datoteka{Oko kapAk pero radar caj}#
#\naslovDat{palindromi.txt: }#
#\datoteka{neven kuk}#
#\datoteka{Oko kapAk radar}#
\end{upotreba}
\end{miditest}
\begin{miditest}
\begin{upotreba}{3}
#\naslovDat{razno.txt}#
#\datoteka{ovde nema palindroma}#
#\naslovDat{palindromi.txt: }#
\end{upotreba}
\end{miditest}
\begin{miditest}
\begin{upotreba}{4}
#\naslovDat{razno.txt}#
#\datoteka{Ana voli Milovana.}#
#\naslovDat{palindromi.txt: }#
#\datoteka{Ana}#
\end{upotreba}
\end{miditest}
\linkresenje{p3_id20}
\end{Exercise}
\begin{Answer}[ref=p3_id20]
%\includecode{resenja/3_Datoteke/6/6.c}
\end{Answer}

\begin{Exercise}[label=p3_iv5]         
U datoteci čije se ime zadaje sa standardnog ulaza nalazi
se broj $n$ ($n\leq 256$), a zatim i $n$ reči dužine najviše 50 karaktera. Napisati program koji učitava re\v ci iz datoteke u niz i:
  \begin{enumerate}
  \item ispisuje ga na standardni izlaz 
  \item iz niza uklanja sve duplikate i upisuje transformisani niz u datoteku $rez.txt$ 
  \end{enumerate}
U slučaju greške ispisati -1 na standardni izlaz. Maksimalna dužina naziva
datoteka je 20 karaktera. \\
\komentarJ{p3iv5, rec po rec, fiksno ime, ime se unosi sa standardnog ulaza, ima resenje}
\komentarJ{ima dve varijante resenja, koristi se u obe dvodimenzioni niz}
\begin{miditest}
\begin{upotreba}{1}
#\naslovInt#
#\ulaz{dat1.txt}#
#\naslovDat{dat1.txt}#
#\datoteka{12 jha14 hahaha deda mraz deda}#
#\datoteka{mraz deda deda jase konj konj konj}#
#\naslovIzlaz#
#\izlaz{jha14 hahaha deda mraz deda mraz deda}#
#\izlaz{deda jase konj konj konj}#
#\naslovDat{rez.txt: }#
#\datoteka{jha14 hahaha deda mraz jase konj}#
\end{upotreba}
\end{miditest}
\begin{miditest}
\begin{upotreba}{2}
#\naslovInt#
#\ulaz{dat2.txt}#
#\naslovDat{dat2.txt}#
#\datoteka{14}#
#\datoteka{so secer supa so ljuto secer kiselo slatko}#
#\datoteka{ljuto}#
#\datoteka{paprika, ljuta paprika, ljuto dete}#
#\naslovIzlaz#
#\izlaz{so secer supa so ljuto secer kiselo slatko}#
#\izlaz{ljuto paprika, ljuta paprika, ljuto dete}#
#\naslovDat{rez.txt: }#
#\datoteka{so secer supa ljuto kiselo slatko}#
#\datoteka{paprika, ljuta dete}#
\end{upotreba}
\end{miditest}
\begin{miditest}
\begin{upotreba}{3}
#\naslovInt#
#\ulaz{dat3.txt}#
#\naslovDat{dat3.txt}#
#\datoteka{2}#
#\datoteka{4 abc 1234 (5+3)*12.4 11-k}#
#\naslovIzlaz#
#\izlaz{abc 1234 (5+3)*12.4 11-k}#
#\naslovDat{rez.txt: }#
#\datoteka{abc 1234 (5+3)*12.4 11-k}#
\end{upotreba}
\end{miditest}
\begin{miditest}
\begin{upotreba}{4}
#\naslovInt#
#\ulaz{dat4.txt}#
#\naslovDat{dat4.txt ne postoji}#
#\naslovIzlaz#
#\izlaz{-1}#
\end{upotreba}
\end{miditest}
\linkresenje{p3_iv5}
\end{Exercise}
\begin{Answer}[ref=p3_iv5]
%\includecode{resenja/3_Datoteke/6/6.c}
\end{Answer}




\begin{Exercise}[label=p3_id19]         
Napisati program koji u datoteku $izlaz.txt$ prepisuje sve
reči iz datoteke $ulaz.txt$ čiji je zbir ASCII kodova
karaktera strogo veći od 1000. Reči su odvojene prazninama i nisu
duže od 200 karaktera.  \\
\komentarJ{p3id19, karakter po karakter, prepisivanje, fiksno ime, nema resenje}
\begin{miditest}
\begin{upotreba}{1}
#\naslovDat{ulaz.txt}#
#\datoteka{Sa standardnog ulaza unosi se neoznacen}#
#\datoteka{ceo broj. Formirati novi broj koji se dobija}#
#\datoteka{izbacivanjem svake druge cifre iz polaznog broja.}#
#\naslovDat{izlaz.txt}#
#\datoteka{standardnog izbacivanjem}
\end{upotreba}
\end{miditest}
\begin{miditest}
\begin{upotreba}{2}
#\naslovDat{ulaz.txt}#
#\datoteka{i sada jedan kratak primer}#
#\datoteka{p1: 1234567890}#
#\datoteka{p2: ABCDEFGHIJ}#
#\datoteka{p3: abcdefghij}#
#\naslovDat{izlaz.txt}#
#\datoteka{abcdefghij}
\end{upotreba}
\end{miditest}
\begin{miditest}
\begin{upotreba}{3}
#\naslovDat{ulaz.txt}#
#\datoteka{konstruisanje test-primera sa}#
#\datoteka{i dugackim recima kao prestolonaslednik}#
#\datoteka{brojevima1234567890}#
#\naslovDat{izlaz.txt}#
#\datoteka{konstruisanje test-primera}#
#\datoteka{prestolonaslednik}#
#\datoteka{brojevima1234567890}#
\end{upotreba}
\end{miditest}
\begin{miditest}
\begin{upotreba}{4}
#\naslovDat{ulaz.txt}#
#\datoteka{ima jos dugackih reci: predskazanje,}#
#\datoteka{potom}#
#\datoteka{nelogicnosti, zanemarivati, odugovlaciti, a ima}#
#\datoteka{i i malih reci koje su kratke}#
#\datoteka{predosecaj}#
#\naslovDat{izlaz.txt}#
#\datoteka{predskazanje, nelogicnosti,}#
#\datoteka{zanemarivati, odugovlaciti,}#
#\datoteka{predosecaj}#
\end{upotreba}
\end{miditest}
\linkresenje{p3_id19}
\end{Exercise}
\begin{Answer}[ref=p3_id19]
%\includecode{resenja/3_Datoteke/6/6.c}
\end{Answer}


%% ----------- REC PO REC KRAJ ----------



%% ----------- BROJ PO BROJ POCETAK ----------

\subsection{Broj po broj}

\begin{Exercise}[label=v3_03] 
U datoteci čije se ime zadaje kao prvi argument komandne linije
nalazi se
prirodan broj $n$ a zatim i $n$ celih brojeva. Napisati program koji
na standardni izlaz ispisuje 
koliko $k$-tocifrenih brojeva postoji u datoteci, pri čemu se
prirodan broj $k$
zadaje kao drugi argument komandne linije.
\\
\komentarJ{v303, brojevi, argumenti}
\begin{miditest}
\begin{upotreba}{1}
#\poziv{./a.out ulaz.txt 2}#
#\naslovDat{ulaz.txt}#
#\datoteka{6}#
#\datoteka{15}#
#\datoteka{193}#
#\datoteka{-27}#
#\datoteka{9790}#
#\datoteka{35}#
#\datoteka{1}#
#\naslovIzlaz#
#\izlaz{3}#
\end{upotreba}
\end{miditest}
\begin{miditest}
\begin{upotreba}{2}
#\poziv{./a.out in.dat 5}#
#\naslovDat{ulaz.txt}#
#\datoteka{4}#
#\datoteka{15}#
#\datoteka{193}#
#\datoteka{-27}#
#\datoteka{9790}#
#\naslovIzlaz#
#\izlaz{0}#
\end{upotreba}
\end{miditest}
\begin{miditest}
\begin{upotreba}{3}
#\poziv{./a.out in.txt 3}#
#\naslovDat{ulaz.txt ne postoji}#
#\naslovIzlaz#
#\izlaz{-1}#
\end{upotreba}
\end{miditest}
\begin{miditest}
\begin{upotreba}{4}
#\poziv{./a.out in.txt}#
#\naslovIzlaz#
#\izlaz{-1}#
\end{upotreba}
\end{miditest}
\linkresenje{v3_03}
\end{Exercise}
\begin{Answer}[ref=v3_03]
\includecode{resenja/3_Datoteke/3/3.c}
\end{Answer}

\begin{Exercise}[label=p3_05] 
 Napisati program koji na standardni izlaz ispisuje maksimum brojeva iz datoteke $brojevi.txt$. \\
\komentarJ{p305, brojevi}
\begin{minitest}
\begin{upotreba}{1}
#\naslovDat{brojevi.txt}#
#\datoteka{2.36 -16.11 5.96 8.88}#
#\datoteka{-265.31 54.96 38.4}#

#\naslovIzlaz#
#\izlaz{Najveci broj je: 54.96}#
\end{upotreba}
\end{minitest}
\begin{minitest}
\begin{upotreba}{2}
#\naslovDat{brojevi.txt}#
#\datoteka{10.5 183.111 -90.2 3.167}#

#\naslovIzlaz#
#\izlaz{Najveci broj je: 183.111}#
\end{upotreba}
\end{minitest}
\begin{minitest}
\begin{upotreba}{3}
#\naslovDat{brojevi.txt}#
#\datoteka{-62.7 -190.2 -2.3 -1000}#
#\datoteka{-198.25 -8}#

#\naslovIzlaz#
#\izlaz{Najveci broj je: -2.3}#
\end{upotreba}
\end{minitest}
\linkresenje{p3_05}
\end{Exercise}
\begin{Answer}[ref=p3_05]
\includecode{resenja/3_Datoteke/praktikumi14/1_05.c}
\end{Answer}

\begin{Exercise}[label=p3_iv4]         
Prvi red datoteke $matrice.txt$ sadrži 2 cela broja manja od
50 koji predstavljaju redom broj vrsta i broj kolona realne matrice
$A$. Svaki sledeći red sadrži po jednu vrstu matrice. Napisati
program koji pronalazi sve elemente matrice $A$ koji su jednaki zbiru
svih svojih susednih elemenata i štampa ih u obliku
\verb|(broj vrste, broj kolone, vrednost elementa).| U slučaju greške prilikom otvaranja datoteke ispisati {\tt -1}.
Pretpostaviti da je sadržaj datoteke ispravan. \\
\komentarJ{p3iv4, matrice, ima resenje}
\begin{minitest}
\begin{upotreba}{1}
#\naslovDat{matrice.txt}#
#\datoteka{3 4}#
#\datoteka{1  2  3  4}#
#\datoteka{7  2 15 -3}#
#\datoteka{-1  3  1  3}#
#\naslovIzlaz#
#\izlaz{(1, 0, 7)}#
#\izlaz{(1, 2, 15)}#
\end{upotreba}
\end{minitest}
\begin{minitest}
\begin{upotreba}{2}
#\naslovDat{matrice.txt}#
#\datoteka{2 2}#
#\datoteka{1  1  }#
#\datoteka{-2  2 }#
#\naslovIzlaz#
#\izlaz{(0, 0, 1)}#
#\izlaz{(0, 1, 1)}#
\end{upotreba}
\end{minitest}
\begin{minitest}
\begin{upotreba}{3}
#\naslovDat{matrice.txt}#
#\datoteka{1 4}#
#\datoteka{9 3 5 2}#
#\naslovIzlaz#
#\izlaz{(0, 2, 5)}#
\end{upotreba}
\end{minitest}
\linkresenje{p3_iv4}
\end{Exercise}
\begin{Answer}[ref=p3_iv4]
%\includecode{resenja/3_Datoteke/6/6.c}
\end{Answer}

\begin{Exercise}[label=p3_x6]         
Prvi red datoteke $ulaz.txt$ sadrži 2 cela broja između 2 i 50
koji predstavljaju redom broj vrsta i broj kolona realne matrice
$A$. Svaki sledeći red sadrži po jednu vrstu matrice. Napisati
program koji nalazi i štampa sve četvorke oblika
\verb|(A(i,j), A(i+1,j), A(i,j+1),A(i+1,j+1))| u kojima su svi
elementi međusobno različiti.\\
\komentarJ{p3x6, matrice, fiksno ime, nema resenje}
\komentarJ{preciziran tekst, uskladiti resenje}
\begin{minitest}
\begin{upotreba}{1}
#\naslovDat{ulaz.txt}#
#\datoteka{3 4}#
#\datoteka{1  2  3  4}#
#\datoteka{7  2 15 -3}#
#\datoteka{-1  3  1  3}#
#\naslovIzlaz#
#\izlaz{(3, 15, 4, -3)}#
#\izlaz{(7, -1, 2, 3)}#
#\izlaz{(2, 3, 15, 1)}#
#\izlaz{(15, 1, -3, 3)}#
\end{upotreba}
\end{minitest}
\begin{minitest}
\begin{upotreba}{2}
#\naslovDat{matrice.txt}#
#\datoteka{2 2}#
#\datoteka{1  1  }#
#\datoteka{-2  2 }#
#\naslovIzlaz#
\end{upotreba}
\end{minitest}
\begin{minitest}
\begin{upotreba}{3}
#\naslovDat{matrice.txt}#
#\datoteka{1 4}#
#\datoteka{9 3 5 2}#
#\naslovIzlaz#
#\izlaz{-1}#
\end{upotreba}
\end{minitest}
\linkresenje{p3_x6}
\end{Exercise}
\begin{Answer}[ref=p3_x6]
%\includecode{resenja/3_Datoteke/6/6.c}
\end{Answer}

%% ----------- BROJ PO BROJ KRAJ ----------


%% ----------- STRUKTURE POCETAK ----------

\subsection{Strukture}

\komentarJ{Prva tri zadatka nisu resena, trebalo bi resiti bar prvi}

\begin{Exercise}[label=p3_07] 
 U datoteci $tacke.txt$ se nalazi broj tačaka, a zatim u posebnim linijama za svaku ta\v cku njene $x$ i $y$ koordinate. Napisati program koji u datoteku $rastojanja.txt$ upisuje rastojanje svake od učitanih tačaka od koordinatnog početka, a na standardni izlaz koordinate tačke koja je od njega najudaljenija. Koristiti strukturu $Tacka$ sa poljima $x$ i $y$, kao i funkciju kojom se računa rastojanje. Pretpostaviti da broj tačaka u datoteci neće biti veći od 50. \\  
\\
\komentarJ{p307, strukture, fiksno ime}
\begin{miditest}
\begin{upotreba}{1}
#\naslovDat{tacke.txt}#
#\datoteka{4}#
#\datoteka{11 -2}#
#\datoteka{3 5}#
#\datoteka{8 -8}#
#\datoteka{0 4}#

#\naslovDat{rastojanja.txt}#
#\datoteka{11.18}#
#\datoteka{5.29}#
#\datoteka{11.31}#
#\datoteka{4.00}#

#\naslovIzlaz#
#\izlaz{Najudaljenija je tačka: 8 -8}#
\end{upotreba}
\end{miditest}
\begin{miditest}
\begin{upotreba}{2}
#\naslovDat{tacke.txt}#
#\datoteka{-2}#
#\datoteka{0 0}#
#\datoteka{9 -8}#

#\naslovIzlaz#
#\izlaz{Greska: Nedozvoljen broj tacaka!}#
\end{upotreba}
\end{miditest}
\linkresenje{p3_07}
\end{Exercise}
\begin{Answer}[ref=p3_07]
%\includecode{resenja/3_Datoteke/praktikumi14/1_07.c}
\end{Answer}

\begin{Exercise}[label=p3_x5]         
Data je struktura kojom se opisuje trodimenzioni vektor:
\begin{verbatim}
typedef struct{
    int x;
    int y;
    int z;
} vektor;
\end{verbatim}
U datoteci $vektori.txt$ nalazi se nepoznati broj vektora (najvi\v se 200). Napisati program koji učitava vektore iz ove datoteke u niz i ispisuje na standardni izlaz koordinate vektora sa najvećom dužinom. Dužina vektora se izračunava po formuli:
$$|v|= \sqrt{x^2+y^2+z^2}$$ U slučaju greške ispisati -1. \\
\komentarJ{p3x5, struktura, fiksno ime, nema resenje}
\begin{minitest}
\begin{upotreba}{1}
#\naslovDat{vektori.txt}#
#\datoteka{2}#
#\datoteka{4 -1 7}#
#\datoteka{3 1 2}#
#\naslovIzlaz#
#\izlaz{4 -1 7}#
\end{upotreba}
\end{minitest}
\begin{minitest}
\begin{upotreba}{2}
#\naslovDat{vektori.txt}#
#\datoteka{67}#
#\naslovIzlaz#
#\izlaz{-1}#
\end{upotreba}
\end{minitest}
\begin{minitest}
\begin{upotreba}{3}
#\naslovDat{vektori.txt}#
#\datoteka{3}#
#\datoteka{0 0 0}#
#\datoteka{0 1 0}#
#\datoteka{1 0 0}#
#\naslovIzlaz#
#\izlaz{0 1 0}#
\end{upotreba}
\end{minitest}
\begin{minitest}
\begin{upotreba}{4}
#\naslovDat{vektori.txt}#
#\datoteka{4}#
#\datoteka{3 0 1}#
#\datoteka{4 5 2}#
#\datoteka{1 0 0}#
#\datoteka{2 -1 2}#
#\naslovIzlaz#
#\izlaz{4 5 2}#
\end{upotreba}
\end{minitest}
\begin{minitest}
\begin{upotreba}{5}
#\naslovDat{vektori.txt}#
#\datoteka{4}#
#\datoteka{3 0 1}#
#\datoteka{4 5 2}#
#\datoteka{1}#
#\datoteka{2 -1 2}#
#\naslovIzlaz#
#\izlaz{-1}#
\end{upotreba}
\end{minitest}
\begin{minitest}
\begin{upotreba}{6}
#\naslovDat{vektori.txt}#
#\datoteka{}#
#\naslovIzlaz#
#\izlaz{-1}#
\end{upotreba}
\end{minitest}
\linkresenje{p3_x5}
\end{Exercise}
\begin{Answer}[ref=p3_x5]
%\includecode{resenja/3_Datoteke/6/6.c}
\end{Answer}


\begin{Exercise}[label=p3_x1]         
Data je struktura koja opisuje pravougaonik dužinama svojih stranica i imenom:
\begin{verbatim}
typedef struct{
    unsigned int a, b;
    char ime[5];
}_pravougaonik;
\end{verbatim}
Napisati program koji iz datoteke čije ime se zadaje kao argument komandne linije učitava podatke o pravougaonicima (nepoznato koliko), a zatim ispisuje imena onih pravougaonika koji su kvadrati i vrednost najveće površine među pravougaonicima koji nisu kvadrati. U slučaju unosa nekorektnih dužina stranica pravougaonika ili nekorektne vrednosti broja $n$, ispisati -1. Maksimalan broj pravougaonika je 200. \\
\komentarJ{p3x1, strukture, argumenti, nema resenja}
\begin{minitest}
\begin{upotreba}{1}
#\poziv{./a.out pravougaonici.dat}#
#\naslovDat{pravougaonici.dat}#
#\datoteka{2 4 p1}#
#\datoteka{3 3 p2}#
#\datoteka{1 6 p3}#
#\naslovIzlaz#
#\izlaz{p2 8}#
\end{upotreba}
\end{minitest}
\begin{minitest}
\begin{upotreba}{2}
#\poziv{./a.out dva.dat}#
#\naslovDat{dva.dat}#
#\datoteka{5 2 pm}#
#\datoteka{4 7 pv}#
#\naslovIzlaz#
#\izlaz{28}#
\end{upotreba}
\end{minitest}
\begin{minitest}
\begin{upotreba}{3}
#\poziv{./a.out tri.dat}#
#\naslovDat{tri.dat}#
#\datoteka{5 5 m}#
#\datoteka{3 3 s}#
#\datoteka{8 8 xl}#
#\naslovIzlaz#
#\izlaz{m s xl}#
\end{upotreba}
\end{minitest}
\begin{minitest}
\begin{upotreba}{4}
#\poziv{./a.out primerx.dat}#
#\naslovDat{primerx.dat}#
#\datoteka{9 7 p}#
#\naslovIzlaz#
#\izlaz{63}#
\end{upotreba}
\end{minitest}
\begin{minitest}
\begin{upotreba}{5}
#\poziv{./a.out empty.dat}#
#\naslovDat{empty.dat}#
#\naslovIzlaz#
#\izlaz{}#
\end{upotreba}
\end{minitest}
\begin{minitest}
\begin{upotreba}{6}
#\poziv{./a.out rectangles.txt}#
#\naslovDat{rectangles.txt ne postoji}#
#\naslovIzlaz#
#\izlaz{-1}#
\end{upotreba}
\end{minitest}
\linkresenje{p3_x1}
\end{Exercise}
\begin{Answer}[ref=p3_x1]
%\includecode{resenja/3_Datoteke/6/6.c}
\end{Answer}



\begin{Exercise}[label=p3_06] 
 U prvom redu datoteke $studenti.txt$ se nalazi broj studenata, a zatim u posebnim linijama za svakog studenta korisničko ime na Alasu i poslednjih pet ocena koje je dobio. Napisati program koji pronalazi studenta koji je ostvario najbolji uspeh i ispisuje njegove podatke. Ukoliko vi\v se studenata ima maksimalni prosek, ispisati prvog. Pretpostaviti da broj studenata neće biti veći od 100.\\
\komentarJ{p306, strukture, fiksno ime}
\begin{miditest}
\begin{upotreba}{1}
#\naslovDat{studenti.txt}#
#\datoteka{5}#
#\datoteka{mr15239 10 9 9 8 10}#
#\datoteka{mi14005 8 8 9 8 10}#
#\datoteka{ml15112 9 8 8 7 10}#
#\datoteka{mr15007 10 10 10 10 10}#
#\datoteka{mn13208 7 7 9 6 10}#

#\naslovIzlaz#
#\izlaz{korisnicko ime: mr15007, prosek ocena: 10.00}#
\end{upotreba}
\end{miditest}
\begin{miditest}
\begin{upotreba}{2}
#\naslovDat{studenti.txt}#
#\datoteka{3}#
#\datoteka{mr16156 10 9 9 10 10}#
#\datoteka{mi17234 9 9 10 10 10}#
#\datoteka{ml17084 9 8 8 8 8}#

#\naslovIzlaz#
#\izlaz{korisnicko ime: mr16156, prosek ocena: 9.6}#
\end{upotreba}
\end{miditest}
\linkresenje{p3_06}
\end{Exercise}
\begin{Answer}[ref=p3_06]
\includecode{resenja/3_Datoteke/praktikumi14/1_06.c}
\end{Answer}

\begin{Exercise}[label=p3_x3]         
Kreirati strukturu $Student$ koja sadrži:
\begin{itemize}
\item $ime\_i\_prezime$ (u polju se čuva ime i prezime studenta,
  napr. "Marko Markovic", maksimalna dužina polja je 100
  karaktera),
\item $oc$ (sadrži najviše 10 ocena studenta)
\item $br\_ocena$ (ukupan broj ocena za studenata)
\item $pr\_oc$ (prosečna ocena)
\end{itemize}
U datoteci \v cije se ime zadaje kao argument komandne linije se nalaze podaci o studentima. Za svakog studenta dato je ime, prezime i niz ocena koji se završava nulom. Svi podaci su razdvojeni razmacima. Napisati program koji u\v citava podatke o studentima i na standardni izlaz ispisuje podatke za studenta sa najvećim prosekom (prosek ispisati na 2 decimale).  Maksimalan broj studenta je
100. \uputstvo{Ime i prezime studenta se mogu pročitati pomo\' cu
specifikatora $\%s$ a potom se za kreiranje niske $ime\_i\_prezime$ u tra\v zenom formatu mo\v ze iskoristiti funkcija $strcat$.} \\
\komentarJ{p3x3, struktura, argumenti, nema resenja}
\begin{minitest}
\begin{upotreba}{1}
#\poziv{./a.out studenti.txt}#
#\naslovDat{studenti.txt}#
#\datoteka{Marko Markovic 5 6 7 8 9 0}#
#\datoteka{Jelena Jankovic 10 10 10 0}#
#\datoteka{Filip Viskovic 10 9 8 7 6 0}#
#\datoteka{Jana Peric 10 10 9 9 8 8 7 0}#
#\naslovIzlaz#
#\izlaz{Jelena Jankovic 10 10 10 0 10.00}#
\end{upotreba}
\end{minitest}
\begin{minitest}
\begin{upotreba}{2}
#\poziv{./a.out}#
#\naslovIzlaz#
#\izlaz{-1}#
\end{upotreba}
\end{minitest}
\begin{minitest}
\begin{upotreba}{3}
#\poziv{./a.out}#
#\naslovIzlaz#
#\izlaz{-1}#
\end{upotreba}
\end{minitest}
\linkresenje{p3_x3}
\end{Exercise}
\begin{Answer}[ref=p3_x3]
%\includecode{resenja/3_Datoteke/6/6.c}
\end{Answer}



\begin{Exercise}[label=v3_05] Imena ulazne i izlazne datoteke se redom navode kao argumenti komandne linije.  U ulaznoj datoteci se nalaze podaci o razlomcima:
u prvom redu se nalazi broj razlomaka , a u svakom sledećem redu brojilac i imenilac jednog razlomka. Potrebno je kreirati strukturu koja opisuje razlomak i učitati niz
razlomaka iz datoteke, a potom:
\begin{enumerate}
\item ukoliko je navedena opcija $x$, upisati u izlaznu datoteku recipročni razlomak za svaki razlomak iz niza (npr. za $2/3$
treba upisati $3/2$) 
\item ukoliko je navedena opcija $y$, upisati u izlaznu datoteku realnu vrednost recipročnog razlomka svakog razlomka iz niza
(npr. za $2/3$ treba upisati $1.5$)
\end{enumerate}
Pretpostaviti da se u ulaznoj datoteci nalazi najviše 100 razlomaka. U slu\v caju gre\v ske, na standardni izlaz ispisati -1.
\\
\komentarJ{v305, strukture, argumenti, opcije}
\begin{miditest}
\begin{upotreba}{1}
#\poziv{./a.out ulaz.txt izlaz.txt -x}#
#\naslovDat{ulaz.txt}#
#\datoteka{4}#
#\datoteka{1 5}#
#\datoteka{19 3}#
#\datoteka{-2 7}#
#\datoteka{97 90}#
#\naslovDat{izlaz.txt}#
#\datoteka{5/1}#
#\datoteka{3/19}#
#\datoteka{-7/2}#
#\datoteka{90/97}#
\end{upotreba}
\end{miditest}
\begin{miditest}
\begin{upotreba}{2}
#\poziv{./a.out ulaz.txt izlaz.txt -y}#
#\naslovDat{ulaz.txt}#
#\datoteka{4}#
#\datoteka{1 5}#
#\datoteka{19 3}#
#\datoteka{-2 7}#
#\datoteka{97 90}#
#\naslovDat{izlaz.txt}#
#\datoteka{5.000000}#
#\datoteka{0.157894}#
#\datoteka{-3.500000}#
#\datoteka{0.927835}#
\end{upotreba}
\end{miditest}
\begin{miditest}
\begin{upotreba}{3}
#\poziv{./a.out ulaz.txt izlaz.txt -y}#
#\naslovDat{ulaz.txt ne postoji}#
#\naslovIzlaz#
#\izlaz{-1}#
\end{upotreba}
\end{miditest}
\begin{miditest}
\begin{upotreba}{4}
#\poziv{./a.out ulaz.txt izlaz.txt}#
#\naslovIzlaz#
#\izlaz{-1}#
\end{upotreba}
\end{miditest}
\linkresenje{v3_05}
\end{Exercise}
\begin{Answer}[ref=v3_05]
\includecode{resenja/3_Datoteke/5/5.c}
\end{Answer}

\begin{Exercise}[label=v3_06] 
Za svaki automobil poznati su marka, model i cena. Iz datoteke čije se ime zadaje sa standardnog ulaza učitava se broj automobila a potom i podaci za svaki automobil. Program treba da: 
\begin{enumerate}
\item ispi\v se prosečnu cenu po marki kola 
\item za maksimalnu cenu koju je kupac spreman da plati, a koja se
zadaje kao argument komandne linije, da ispiše automobile u tom cenovnom
rangu zajedno sa prosečnom cenom odgovarajuće marke
\end{enumerate}
Pretpostaviti da se model i marka sastoje od jedne reči i
da svaka od njih sadrži najviše 30 karaktera kao i da se u datoteci
nalaze podaci za najviše 100 automobila.\\
\komentarJ{v306, strukture, argumenti}
\begin{minitest}
\begin{upotreba}{1}
#\poziv{./a.out 4000}#
#\naslovInt#
#\ulaz{dat1.txt}#
#\naslovDat{dat1.txt}#
#\datoteka{7}#
#\datoteka{renault twingo 2900}#
#\datoteka{renault megan 6250}#
#\datoteka{renault clio 3650}#
#\datoteka{dacia logan 5400}#
#\datoteka{dacia sandero 7800}#
#\datoteka{fiat bravo 4900}#
#\datoteka{fiat linea 4290}#
#\naslovIzlaz#
#\izlaz{Informacije o prosecnoj ceni po markama:}#
#\izlaz{renault 4266.67 3}#
#\izlaz{dacia 6600.00 2}#
#\izlaz{fiat 4595.00 2}#
#\izlaz{Kola u vasem cenovnom rangu:}#
#\izlaz{renault twingo 4266.67}#
#\izlaz{renault clio 4266.67}#
\end{upotreba}
\end{minitest}
\begin{minitest}
\begin{upotreba}{2}
#\poziv{./a.out 5000}#
#\naslovInt#
#\ulaz{dat1.txt}#
#\naslovDat{dat1.txt ne postoji}#
#\naslovIzlaz#
#\izlaz{-1}#
\end{upotreba}
\end{minitest}
\begin{minitest}
\begin{upotreba}{3}
#\poziv{./a.out}#
#\naslovIzlaz#
#\izlaz{-1}#
\end{upotreba}
\end{minitest}
\linkresenje{v3_06}
\end{Exercise}
\begin{Answer}[ref=v3_06]
\includecode{resenja/3_Datoteke/6/6.c}
\end{Answer}




%% ----------- STRUKTURE KRAJ ----------

%% ---------- LINIJE POCETAK ----------

\subsection{Linija po linija}

\begin{Exercise}[label=v3_02] 
Napisati program koji u datoteci čije se ime navodi kao argument komandne linije određuje liniju maksimalne dužine i ispisuje je na standarni izlaz. Ukoliko ima više takvih linija, ispisati onu koja je leksikografski prva. Pretpostaviti da su linije maksimalne dužine 80 karaktera.\\
\komentarJ{v302, linije, duzina, argumenti}
\begin{miditest}
\begin{upotreba}{1}
#\poziv{./a.out test.txt}#
#\naslovDat{test.txt}#
#\datoteka{Danas je veoma hladno decembarsko}#
#\datoteka{popodne. Ne pada sneg, kazu mozda}#
#\datoteka{ce sutra.}#

#\naslovIzlaz#
#\izlaz{Danas je veoma hladno decembarsko}#
\end{upotreba}
\end{miditest}
\begin{minitest}
\begin{upotreba}{2}
#\poziv{./a.out in.txt}#
#\naslovDat{in.txt ne postoji}#

#\naslovIzlaz#
#\izlaz{-1}#
\end{upotreba}
\end{minitest}
\begin{minitest}
\begin{upotreba}{3}
#\poziv{./a.out}#
#\naslovIzlaz#
#\izlaz{-1}#
\end{upotreba}
\end{minitest}

\linkresenje{v3_02}
\end{Exercise}
\begin{Answer}[ref=v3_02]
\includecode{resenja/3_Datoteke/2/2.c}
\end{Answer}

\begin{Exercise}[label=p3_03] 
 Kao argumenti komandne linije se zadaju ime datoteke i ceo broj $k$. Napisati program koji na  standardni izlaz ispisuje sve linije zadate datoteke čija je dužina veća od $k$. Možemo pretpostaviti da dužina linije neće biti veća od 80 karaktera.\\
\komentarJ{p303, duzina, argumenti, linije}
\begin{miditest}
\begin{upotreba}{1}
#\poziv{./a.out test.txt 7}#
#\naslovDat{test.txt}#
#\datoteka{Teme koje su obradjivane:}#
#\datoteka{Petlje}#
#\datoteka{Funkcije}#
#\datoteka{Nizovi}#
#\datoteka{Strukture}#

#\naslovIzlaz#
#\izlaz{Teme koje su obradjivane:}#
#\izlaz{Funkcije}#
#\izlaz{Strukture}#
\end{upotreba}
\end{miditest}
\begin{miditest}
\begin{upotreba}{2}
#\poziv{./a.out test.txt}#

#\naslovIzlaz#
#\izlaz{Greska: Pogresan broj argumenata!}#
\end{upotreba}
\end{miditest}
\linkresenje{p3_03}
\end{Exercise}
\begin{Answer}[ref=p3_03]
\includecode{resenja/3_Datoteke/praktikumi14/1_03.c}
\end{Answer}

\begin{Exercise}[label=v3_04] 
U datoteci čije se ime navodi kao prvi argument komandne
linije navedena je reč $r$ i niz linija. Napisati
program koji u datoteku čije se ime navodi kao
drugi argument komandne linije upisuje sve linije
u kojima se reč $r$ pojavljuje bar $n$ puta gde je
$n$ prirodan broj koji se unosi sa standardnog ulaza. Ra\v cunaju se i samostalna pojavljivanja re\v ci $r$ i pojavljivanja u okviru neke druge re\v ci. Ispis treba da bude u formatu \verb|broj_pojavljivanja:linija|.\\
\komentarJ{v304, linije, podniska, argumenti}
\begin{miditest}
\begin{upotreba}{1}
#\poziv{./a.out input.txt output.txt}#
#\naslovInt#
#\izlaz{Unesite prirodan broj:} \ulaz{2}#
#\naslovDat{input.txt}#
#\datoteka{sto}#
#\datoteka{stolica lampa}#
#\datoteka{postotak Stopiranje stopa}# 
#\datoteka{presto Ostoja stotina prostorija}#
#\naslovDat{output.txt}#
#\datoteka{2: postotak Stopiranje stopa}# 
#\datoteka{4: presto Ostoja stotina prostorija}#
\end{upotreba}
\end{miditest}
\begin{miditest}
\begin{upotreba}{2}
#\poziv{./a.out input.txt output.txt}#
#\naslovInt#
#\izlaz{Unesite prirodan broj:} \ulaz{3}#
#\naslovDat{input.txt}#
#\datoteka{red}#
#\datoteka{redar za ovu nedelju}#
#\datoteka{redosled ured}# 
#\datoteka{odrediti raspored}#
#\naslovDat{output.txt}#

\end{upotreba}
\end{miditest}
\begin{miditest}
\begin{upotreba}{3}
#\poziv{./a.out in.txt out.txt}#
#\naslovDat{in.txt ne postoji}#

#\naslovInt#
#\izlaz{-1}#
\end{upotreba}
\end{miditest}
\begin{miditest}
\begin{upotreba}{4}
#\poziv{./a.out in.txt}#

#\naslovInt#
#\izlaz{-1}#
\end{upotreba}
\end{miditest}
\linkresenje{v3_04}
\end{Exercise}
\begin{Answer}[ref=v3_04]
\includecode{resenja/3_Datoteke/4/4.c}
\end{Answer}


\begin{Exercise}[label=p3_04] 
 Napisati program koji prebrojava koliko se linija datoteke $ulaz.txt$ završava niskom $s$ koja se učitava sa standardnog ulaza. Može se pretpostaviti da dužina linije neće biti veća od 80 karaktera, kao i da dužina niske $s$ neće biti veća od 20 karaktera.\\
\komentarJ{p304, linije, podniska}
\begin{miditest}
\begin{upotreba}{1}
#\naslovDat{ulaz.txt}#
#\datoteka{abcde abcde}#
#\datoteka{abcde aab}#
#\datoteka{abcde abcde abcde}#
#\datoteka{abcde abcde Aab}#
#\datoteka{abcde abcde ab}#
#\datoteka{abcde abcde abcde abcde}#

#\naslovInt#
#\izlaz{Unesite nisku s:}\ulaz{ab}#
#\izlaz{Broj linija: 3}#
\end{upotreba}
\end{miditest}
\begin{miditest}
\begin{upotreba}{2}
#\naslovDat{ulaz.txt}#
#\datoteka{abcde abcde}#
#\datoteka{abcde}#
#\datoteka{abcde abcde AB}#

#\naslovInt#
#\izlaz{Unesite nisku s:}\ulaz{ab}#
#\izlaz{Broj linija: 0}#
\end{upotreba}
\end{miditest}
\linkresenje{p3_04}
\end{Exercise}
\begin{Answer}[ref=p3_04]
\includecode{resenja/3_Datoteke/praktikumi14/1_04.c}
\end{Answer}

\begin{Exercise}[label=p3_09] 
 Napisati program koji linije koje se učitavaju sa standardnog ulaza sve do kraja ulaza prepisuje u datoteku $izlaz.txt$ i to, ako je prilikom pokretanja zadata opcija \textit{-v} ili \textit{-V} samo one linije koje počinju velikim slovom, ako je zadata opcija \textit{-m} ili \textit{-M} samo one linije koje počinju malim slovom, a ako je opcija izostavljena sve linije. Pretpostaviti da linije neće biti duže od 80 karaktera. \\
\komentarJ{p309, linije, prepisivanje, opcije, fiksno ime}
\begin{minitest}
\begin{upotreba}{1}
#\poziv{./a.out -m}#
#\naslovInt#
#\izlaz{Unesite recenice: }#
#\ulaz{programiranje u C-u je zanimljivo}#
#\ulaz{Volim programiranje!}#
#\ulaz{Kada porastem bicu programer!}#
#\ulaz{u slobodno vreme programiram}#

#\naslovDat{izlaz.txt}#
#\datoteka{programiranje u C-u je zanimljivo}#
#\datoteka{u slobodno vreme programiram}#
\end{upotreba}
\end{minitest}
\begin{minitest}
\begin{upotreba}{2}
#\poziv{./a.out -V}#
#\naslovInt#
#\izlaz{Unesite recenice: }#
#\ulaz{programiranje u C-u je zanimljivo}#
#\ulaz{Volim programiranje!}#
#\ulaz{Kada porastem bicu programer!}#
#\ulaz{u slobodno vreme programiram}#

#\naslovDat{izlaz.txt}#
#\datoteka{Volim programiranje!}#
#\datoteka{Kada porastem bicu programer!}#
\end{upotreba}
\end{minitest}
\begin{minitest}
\begin{upotreba}{3}
#\poziv{./a.out -k}#
#\naslovInt#
#\izlaz{Greska: Pogresno pokretanje programa!}#
\end{upotreba}
\end{minitest}
\linkresenje{p3_09}
\end{Exercise}
\begin{Answer}[ref=p3_09]
\includecode{resenja/3_Datoteke/praktikumi14/1_09.c}
\end{Answer}



           






\begin{Exercise}[label=p3_iv10]         
Napisati program koji poredi dve datoteke i ispisuje redni broj linija u
kojima se datoteke razlikuju.  Imena datoteka se zadaju kao argumenti
komandne linije. U slučaju neuspešnog otvaranja datoteka
ispisati poruku o grešci. Pretpostaviti da je maksimalna dužina
reda u datoteci 200 karaktera. Ukoliko nisu zadati potrebni argumenti
komadne linije ispisati poruku o grešci. Linije brojati počevši od {\tt 1}. \\
\komentarJ{iv10, linije, poredjenje, argumenti, ima resenje}
\begin{miditest}
\begin{upotreba}{1}
#\poziv{./a.out ulaz.txt izlaz.txt}#
#\naslovDat{ulaz.txt}#
#\datoteka{danas vezbamo}#
#\datoteka{programiranje}#
#\datoteka{ovo je primer kad su}#
#\datoteka{datoteke iste}#
#\naslovDat{izlaz.txt: }#
#\datoteka{danas vezbamo}#
#\datoteka{programiranje}#
#\datoteka{ovo je primer kad su}#
#\datoteka{datoteke iste}#
#\naslovIzlaz#
#\izlaz{}#
\end{upotreba}
\end{miditest}
\begin{miditest}
\begin{upotreba}{2}
#\poziv{./a.out primer1.dat primer2.dat}#
#\naslovDat{primer1.dat}#
#\datoteka{danas vezbamo}#
#\datoteka{analizu}#
#\datoteka{ovo je primer kad}#
#\datoteka{su datoteke razlicite}#
#\naslovDat{priemr2.dat}#
#\datoteka{danas vezbamo}#
#\datoteka{programiranje}#
#\datoteka{ovo je primer kad su}#
#\datoteka{datoteke razlicite}#
#\naslovIzlaz#
#\izlaz{2 3 4}#
\end{upotreba}
\end{miditest}
\begin{miditest}
\begin{upotreba}{3}
#\poziv{./a.out prva.dat druga.dat}#
#\naslovDat{prva.dat}#
#\datoteka{ovo je primer}#
#\datoteka{kada su}#
#\datoteka{datoteke}#
#\datoteka{razlicite duzine}#
#\naslovDat{druga.dat}#
#\datoteka{kada su }#
#\datoteka{programiranje}#
#\datoteka{datoteke}#
#\datoteka{razlicite}#
#\datoteka{duzine}#
#\datoteka{i kada treba ispisati broj}#
#\datoteka{tih redova}#
#\naslovIzlaz#
#\izlaz{1 4 5 6 7}#
\end{upotreba}
\end{miditest}
\begin{miditest}
\begin{upotreba}{4}
#\poziv{./a.out prva.dat}#
#\naslovIzlaz#
#\izlaz{greska}#
\end{upotreba}
\end{miditest}
\linkresenje{p3_iv10}
\end{Exercise}
\begin{Answer}[ref=p3_iv10]
%\includecode{resenja/3_Datoteke/6/6.c}
\end{Answer}


% ---------- LINIJE KRAJ ----------













\section{Rešenja}


\shipoutAnswer


