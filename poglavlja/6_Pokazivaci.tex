
\section{Pokazivači}


\begin{Exercise}[label=v2.2_01] 
Napisati funkciju koja uređuje svoja dva celobrojna argumenta tako da se u prvom nalazi manji, a u drugom veći. Napisati program koji učitava dva cela broja i ispisuje rezultat poziva funkcije. 

\begin{miditest}
\begin{upotreba}{1}
#\naslovInt#
#\izlaz{Unesite vrednosti promenljivih x i y: }\ulaz{2 5}#
#\izlaz{Uredjene promenljive: x=2, y=5}#
\end{upotreba}
\end{miditest}
\begin{miditest}
\begin{upotreba}{2}
#\naslovInt#
#\izlaz{Unesite vrednosti promenljivih x i y: }\ulaz{11 -4}#
#\izlaz{Uredjene promenljive: x=-4, y=11}#
\end{upotreba}
\end{miditest}

\linkresenje{v2.2_01}
\end{Exercise}
\ifresenja
\begin{Answer}[ref=v2.2_01]
\includecode{resenja/2_PredstavljanjePodataka/2.2_Pokazivaci/1_01.c}
\end{Answer}
 \fi

\begin{Exercise}[label=v2.2_02] 
  Napisati funkciju koja za boju datu u \textit{rgb} formatu računa \textit{cmy} format po formulama:\\
  $c = 1 - ( r / 255 )$\\
  $m = 1 - ( g / 255 )$\\
  $y = 1 - ( b / 255 )$\\
  Napisati program koji učitava tri cela broja broja (\textit{rgb} format) i ispisuje rezultat poziva funkcije (\textit{cmy} format). \napomena{Vrednosti boja u \textit{rgb} formatu su u opsegu $[0,255]$.}\\
\begin{miditest}
\begin{upotreba}{1}
#\naslovInt#
#\izlaz{Unesite boju u rgb formatu: }\ulaz{56 111 24}#
#\izlaz{c=0.78, m=0.56, y=0.91}#
\end{upotreba}
\end{miditest}
\begin{miditest}
\begin{upotreba}{2}
#\naslovInt#
#\izlaz{Unesite boju u rgb formatu: }\ulaz{156 -90 5}#
#\izlaz{Nekorektan unos.}#
\end{upotreba}
\end{miditest}
\begin{miditest}
\begin{upotreba}{3}
#\naslovInt#
#\izlaz{Unesite boju u rgb formatu: }\ulaz{9 0 237}#
#\izlaz{c=0.96, m=1.00, y=0.07}#
\end{upotreba}
\end{miditest}
\begin{miditest}
\begin{upotreba}{4}
#\naslovInt#
#\izlaz{Unesite boju u rgb formatu: }\ulaz{300 11 27}#
#\izlaz{Nekorektan unos.}#
\end{upotreba}
\end{miditest}

\linkresenje{v2.2_02}
\end{Exercise}
\ifresenja
\begin{Answer}[ref=v2.2_02]
\includecode{resenja/2_PredstavljanjePodataka/2.2_Pokazivaci/1_02.c}
\end{Answer}
 \fi

\begin{Exercise}[label=v2.2_03] 
   Napisati funkciju koja za dve prave date svojim koeficijentima
   pravca i slobodnim članovima određuje njihovu tačku preseka. 
   Funkcija treba da vrati 1 ako se prave seku i 0 ako nemaju
   tačku preseka (ako su paralelne). Napisati program
   koji učitava podatke o pravama, poziva napisanu funkciju i 
   ispisuje odgovarajuću poruku.
   
\begin{miditest}
\begin{upotreba}{1}
#\naslovInt#
#\izlaz{Unesite k i n za prvu pravu: }\ulaz{4 5}#
#\izlaz{Unesite k i n za drugu pravu: }\ulaz{11 -4}#
#\izlaz{Prave se seku u tacki (1.29,10.14).}#
\end{upotreba}
\end{miditest}
\begin{miditest}
\begin{upotreba}{2}
#\naslovInt#
#\izlaz{Unesite k i n za prvu pravu: }\ulaz{0.5 -4.7}#
#\izlaz{Unesite k i n za drugu pravu: }\ulaz{0.5 9.1}#
#\izlaz{Prave su paralelne.}#
\end{upotreba}
\end{miditest}

\linkresenje{v2.2_03}
\end{Exercise}
\ifresenja
\begin{Answer}[ref=v2.2_03]
\includecode{resenja/2_PredstavljanjePodataka/2.2_Pokazivaci/1_03.c}
\end{Answer}
 \fi

\begin{Exercise}[label=v2.xy] 
Napisati funkciju koja za dva cela broja izračunava njihov količnik i ostatak pri deljenju. Funkcija treba da vrati vrednost 1 ukoliko je uspešno izračunala vrednosti i 0 ukoliko deljenje nije moguće. Napisati program koji testira rad ove funkcije. 
\end{Exercise}

\begin{Exercise}[label=v2.xyz] 
Napisati funkciju koja za dinarsku vrednost cene artikla izračunava odgovarajuću cenu u evrima i u dolarima. Napisati program koji testira rad ove funkcije. 
%TODO dodati kurs evra i dolara, ovo bi trebalo da budu dve funkcije, treba nekako opravdati stavljanje u istu funkciju
\end{Exercise}

\begin{Exercise}[label=v2.xyz] 
Napisati funkciju koja za dužinu trajanja filma kaja je data u sekundama, određuje ukupno trajanje filma u satima, minutama i sekundama. Napisati program koji testira rad ove funkcije. 
\end{Exercise}



\begin{Exercise}[label=p2.6_06] 
 Napisati funkciju \kckod{void\ modifikacija(char*\ s,\ char*\ t, int*\ br\_modifikacija)} koja na osnovu niske $s$ formira nisku $t$ tako što svako malo slovo zamanjuje velikim. Broj izvršenih modifikacija se čuva u okviru argumenta $br\_modifikacija$. Pretpostaviti da niska $s$ neće biti duža od 20 karaktera. Napisati program koji testira rad napisane funkcije. 
%TODO ovaj zadatak je potpuno neprirodan jer se broj modifikacija moze vratiti kao povratna vrednost funkcije 
%ovo bi trebalo modifikovati tako da se malo zamenjuje velikim, veliko malim, a da onda imamo dva argumenta koja prate modifikacije

\begin{miditest}
\begin{upotreba}{1}
#\naslovInt#
#\izlaz{Unesite nisku:}\ulaz{123abc789XY}#
#\izlaz{Modifikovana niska je: 123ABC789XY}#
#\izlaz{Broj modifikacija je: 3}#
\end{upotreba}
\end{miditest}
\begin{miditest}
\begin{upotreba}{2}
#\naslovInt#
#\izlaz{Unesite nisku:}\ulaz{zimA}#
#\izlaz{Modifikovana niska je: ZIMA}#
#\izlaz{Broj modifikacija je: 3}#
\end{upotreba}
\end{miditest}

\begin{miditest}
\begin{upotreba}{3}
#\naslovInt#
#\izlaz{Unesite nisku:}\ulaz{SNEG}#
#\izlaz{Modifikovana niska je: SNEG}#
#\izlaz{Broj modifikacija je: 0}#
\end{upotreba}
\end{miditest}
\begin{miditest}
\begin{upotreba}{4}
#\naslovInt#
#\izlaz{Unesite nisku:}\ulaz{1234}#
#\izlaz{Modifikovana niska je: 1234}#
#\izlaz{Broj modifikacija je: 0}#
\end{upotreba}
\end{miditest}

\linkresenje{p2.6_06}
\end{Exercise}
\ifresenja
\begin{Answer}[ref=p2.6_06]
\includecode{resenja/2_PredstavljanjePodataka/2.2_Pokazivaci/praktikumi12/9.c}
\end{Answer}
 \fi


\begin{Exercise}[label=p2.6_07] 
 Napisati funkciju \kckod{void\ interpunkcija(int*\ br\_tacaka,\ int*\ br\_zareza)} koja prebrojava tačke i zareze u tekstu koji se unosi sa standardnog ulaza. Napisati program koji testira napisanu funkciju.\\
\begin{minitest}
\begin{upotreba}{1}
#\naslovInt#
#\izlaz{Unesite tekst:}#
#\ulaz{a.b.c.d}#
#\ulaz{a,b,,c,d,e}#
#\izlaz{Broj tacaka: 3}#
#\izlaz{Broj zareza: 5}#
\end{upotreba}
\end{minitest}
\begin{minitest}
\begin{upotreba}{2}
#\naslovInt#
#\izlaz{Unesite tekst:}#
#\ulaz{.....789.....}#
#\izlaz{Broj tacaka: 10}#
#\izlaz{Broj zareza: 0}#
\end{upotreba}
\end{minitest}
\begin{minitest}
\begin{upotreba}{3}
#\naslovInt#
#\izlaz{Unesite tekst:}#
#\ulaz{sunce}#
#\izlaz{Broj tacaka: 0}#
#\izlaz{Broj zareza: 0}#
\end{upotreba}
\end{minitest}

\linkresenje{p2.6_07}
\end{Exercise}
\ifresenja
\begin{Answer}[ref=p2.6_07]
\includecode{resenja/2_PredstavljanjePodataka/2.2_Pokazivaci/praktikumi12/10.c}
\end{Answer}
 \fi


\begin{Exercise}[label=p2.6_08] 
 Napisati funkciju \kckod{void\ par\_nepar(int\ a[],\ int\ n,\ int\ parni[],\ int*\ pn,\ int\ neparni[],\ int*\ nn)} koja razbija niz $a$ na niz parnih i niz neparnih brojeva. Pokazivači $pn$ i $nn$ redom treba da sadrže broj elemenata niza parnih tj. niza neparnih elemenata. Pretpostaviti da dužina niza $a$ neće biti veća od 50. Napisati program koji testira napisanu funkciju.\\
\begin{miditest}
\begin{upotreba}{1}
#\naslovInt#
#\izlaz{Unesite broj elemenata niza: }\ulaz{8}#
#\izlaz{Unesite elemente niza: }#
#\ulaz{1 8 9 -7 -16 24 77 4}#
#\izlaz{Niz parnih brojeva: 8 -16 24 4}#
#\izlaz{Niz neparnih brojeva: 1 9 -7 77}#
\end{upotreba}
\end{miditest}
\begin{miditest}
\begin{upotreba}{2}
#\naslovInt#
#\izlaz{Unesite broj elemenata niza: }\ulaz{5}#
#\izlaz{Unesite elemente niza: }#
#\ulaz{2 4 6 8 -11}#
#\izlaz{Niz parnih brojeva: 2 4 6 8}#
#\izlaz{Niz neparnih brojeva: -11}#
\end{upotreba}
\end{miditest}
\begin{miditest}
\begin{upotreba}{3}
#\naslovInt#
#\izlaz{Unesite broj elemenata niza: }\ulaz{2}#
#\izlaz{Unesite elemente niza: }#
#\ulaz{-15 15}#
#\izlaz{Niz parnih brojeva: }#
#\izlaz{Niz neparnih brojeva: -15 15}#
\end{upotreba}
\end{miditest}
\begin{miditest}
\begin{upotreba}{4}
#\naslovInt#
#\izlaz{Unesite broj elemenata niza: }\ulaz{1}#
#\izlaz{Unesite elemente niza: }#
#\ulaz{0}#
#\izlaz{Niz parnih brojeva: 0}#
#\izlaz{Niz neparnih brojeva:}#
\end{upotreba}
\end{miditest}

\linkresenje{p2.6_08}
\end{Exercise}
\ifresenja
\begin{Answer}[ref=p2.6_08]
\includecode{resenja/2_PredstavljanjePodataka/2.2_Pokazivaci/praktikumi12/11.c}
\end{Answer}
 \fi


\begin{Exercise}[label=p2.6_09] 
 Napisati funckiju \kckod{void\ min\_max(float\ a[],\ int\ n,\ float*\ min,\ float*\ max)} koja izračunava minimalni i maksimalni element niza $a$ dužine $n$. Napisati program koji učitava niz realnih brojeva maksimalne dužine 50 i ispisuje vrednosti minimuma i maksimuma zaokruženu na tri decimale. \\
\begin{miditest}
\begin{upotreba}{1}
#\naslovInt#
#\izlaz{Unesite broj elemenata niza: }\ulaz{5}#
#\izlaz{Unesite elemente niza: }#
#\ulaz{24.16 -32.11 999.25 14.25 11}#
#\izlaz{Minimum: -32.110}#
#\izlaz{Maksimum: 999.250}#
\end{upotreba}
\end{miditest}
\begin{miditest}
\begin{upotreba}{2}
#\naslovInt#
#\izlaz{Unesite broj elemenata niza: }\ulaz{4}#
#\izlaz{Unesite elemente niza: }#
#\ulaz{-5.126 -18.29 44 29.268}#
#\izlaz{Minimum: -18.290}#
#\izlaz{Maksimum: 44.000}#
\end{upotreba}
\end{miditest}
\begin{miditest}
\begin{upotreba}{3}
#\naslovInt#
#\izlaz{Unesite broj elemenata niza: }\ulaz{1}#
#\izlaz{Unesite elemente niza: }#
#\ulaz{4.16}#
#\izlaz{Minimum: 4.160}#
#\izlaz{Maksimum: 4.160}#
\end{upotreba}
\end{miditest}
\begin{miditest}
\begin{upotreba}{4}
#\naslovInt#
#\izlaz{Unesite broj elemenata niza: }\ulaz{3}#
#\izlaz{Unesite elemente niza: }#
#\ulaz{7.82 18.989 7.82}#
#\izlaz{Minimum: 7.820}#
#\izlaz{Maksimum: 18.989}#
\end{upotreba}
\end{miditest}

\linkresenje{p2.6_09}
\end{Exercise}
\ifresenja
\begin{Answer}[ref=p2.6_09]
\includecode{resenja/2_PredstavljanjePodataka/2.2_Pokazivaci/praktikumi12/12.c}
\end{Answer}
 \fi


\begin{Exercise}[label=v2.2_04] 
    Napisati program koji ispisuje broj navedenih argumenata komandne linije,
    a zatim i same argumenate i njihove redne brojeve. \\
\begin{miditest}
\begin{upotreba}{1}
#\poziv{./a.out abcde 123 -5 3.7}#
#\naslovInt#
#\izlaz{Broj argumenata je 5:}#
#\izlaz{0: ./a.out}#
#\izlaz{1: abcde}#
#\izlaz{2: 123}#
#\izlaz{3: -5}#
#\izlaz{4: 3.7}#
\end{upotreba}
\end{miditest}
\begin{miditest}
\begin{upotreba}{2}
#\poziv{./a.out}#
#\naslovInt#
#\izlaz{Broj argumenata je 1:}#
#\izlaz{0: ./a.out}#
\end{upotreba}
\end{miditest}

\linkresenje{v2.2_04}
\end{Exercise}
\ifresenja
\begin{Answer}[ref=v2.2_04]
\includecode{resenja/2_PredstavljanjePodataka/2.2_Pokazivaci/1_04.c}
\end{Answer}
 \fi


\begin{Exercise}[label=p2.6_01] 
Napisati program koji ispisuje zbir numeričkih argumenata komandne linije. \uputstvo{Koristiti funkciju \kckod{atoi}}.\\
\begin{miditest}
\begin{upotreba}{1}
#\poziv{./a.out 5 mkp 9 -2 11 a 4 2}#
#\naslovInt#
#\izlaz{Zbir numerickih argumenata: 29}#
\end{upotreba}
\end{miditest}
\begin{miditest}
\begin{upotreba}{2}
#\poziv{./a.out ab u f hj}#
#\naslovInt#
#\izlaz{Zbir numerickih argumenata: 0}#
\end{upotreba}
\end{miditest}
\begin{miditest}
\begin{upotreba}{3}
#\poziv{./a.out 33 1 p 44}#
#\naslovInt#
#\izlaz{Zbir numerickih argumenata: 78}#
\end{upotreba}
\end{miditest}
\begin{miditest}
\begin{upotreba}{4}
#\poziv{./a.out}#
#\naslovInt#
#\izlaz{Zbir numerickih argumenata: 0}#
\end{upotreba}
\end{miditest}
\linkresenje{p2.6_01}
\end{Exercise}
\ifresenja
\begin{Answer}[ref=p2.6_01]
\includecode{resenja/2_PredstavljanjePodataka/2.2_Pokazivaci/praktikumi12/4.c}
\end{Answer}
 \fi


\begin{Exercise}[label=p2.6_02] 
Napisati program koji ispisuje argumente komandne linije koji počinju slovom \textit{z}.\\
\begin{miditest}
\begin{upotreba}{1}
#\poziv{./a.out zima jabuka zvezda Zrak}#
#\naslovInt#
#\izlaz{zima zvezda}#
\end{upotreba}
\end{miditest}
\begin{miditest}
\begin{upotreba}{2}
#\poziv{./a.out bundeva pomorandza}#
#\naslovInt#
#\izlaz{}#
\end{upotreba}
\end{miditest}
\begin{miditest}
\begin{upotreba}{3}
#\poziv{./a.out sanke zapad zujanje}#
#\naslovInt#
#\izlaz{zapad zujanje}#
\end{upotreba}
\end{miditest}
\begin{miditest}
\begin{upotreba}{4}
#\poziv{./a.out}#
#\naslovInt#
#\izlaz{}#
\end{upotreba}
\end{miditest}
\linkresenje{p2.6_02}
\end{Exercise}
\ifresenja
\begin{Answer}[ref=p2.6_02]
\includecode{resenja/2_PredstavljanjePodataka/2.2_Pokazivaci/praktikumi12/5.c}
\end{Answer}
 \fi


\begin{Exercise}[label=p2.6_03] 
Napisati program koji ispisuje broj argumenata komandne linije koji sadrže slovo \textit{z}.\\
\begin{miditest}
\begin{upotreba}{1}
#\poziv{./a.out zvezda grozd jesen kisa}#
#\naslovInt#
#\izlaz{2}#
\end{upotreba}
\end{miditest}
\begin{miditest}
\begin{upotreba}{2}
#\poziv{./a.out AZBUKA deda mraz }#
#\naslovInt#
#\izlaz{2}#
\end{upotreba}
\end{miditest}
\begin{miditest}
\begin{upotreba}{3}
#\poziv{./a.out japan caj}#
#\naslovInt#
#\izlaz{0}#
\end{upotreba}
\end{miditest}
\begin{miditest}
\begin{upotreba}{4}
#\poziv{./a.out}#
#\naslovInt#
#\izlaz{0}#
\end{upotreba}
\end{miditest}


\linkresenje{p2.6_03}
\end{Exercise}
\ifresenja
\begin{Answer}[ref=p2.6_03]
\includecode{resenja/2_PredstavljanjePodataka/2.2_Pokazivaci/praktikumi12/6.c}
\end{Answer}
 \fi


\begin{Exercise}[label=p2.6_04] 
 Napisati program koji na osnovu broja $n$ koji se zadaje kao argument komandne linije ispisuje cele brojeve iz intervala $[-n,\ n]$. \\
\begin{miditest}
\begin{upotreba}{1}
#\poziv{./a.out 2}#
#\naslovInt#
#\izlaz{-2 -1 0 1 2}#
\end{upotreba}
\end{miditest}
\begin{miditest}
\begin{upotreba}{2}
#\poziv{./a.out 4}#
#\naslovInt#
#\izlaz{-4 -3 -2 -1 0 1 2 3 4}#
\end{upotreba}
\end{miditest}
\begin{miditest}
\begin{upotreba}{3}
#\poziv{./a.out 0}#
#\naslovInt#
#\izlaz{0}#
\end{upotreba}
\end{miditest}
\begin{maxitest}
\begin{upotreba}{4}
#\poziv{./a.out}#
#\naslovInt#
#\izlaz{Greska: nedostaje argument komandne linije!}#
\end{upotreba}
\end{maxitest}

\linkresenje{p2.6_04}
\end{Exercise}
\ifresenja
\begin{Answer}[ref=p2.6_04]
\includecode{resenja/2_PredstavljanjePodataka/2.2_Pokazivaci/praktikumi12/7.c}
\end{Answer}
 \fi


\begin{Exercise}[label=p2.6_05] 
 Napisati program koji proverava da li se među zadatim argumentima komandne linije nalaze barem dva ista.  \\
\begin{miditest}
\begin{upotreba}{1}
#\poziv{./a.out pec zima deda mraz pec}#
#\naslovInt#
#\izlaz{Medju argumentima ima istih.}#
\end{upotreba}
\end{miditest}
\begin{miditest}
\begin{upotreba}{2}
#\poziv{./a.out xyz abc abc abc efgh}#
#\naslovInt#
#\izlaz{Medju argumentima ima istih.}#
\end{upotreba}
\end{miditest}
\begin{miditest}
\begin{upotreba}{3}
#\poziv{./a.out 11 15 abc 888}#
#\naslovInt#
#\izlaz{Medju argumentima nema istih.}#
\end{upotreba}
\end{miditest}
\begin{miditest}
\begin{upotreba}{4}
#\poziv{./a.out}#
#\naslovInt#
#\izlaz{Medju argumentima nema istih.}#
\end{upotreba}
\end{miditest}

\linkresenje{p2.6_05}
\end{Exercise}
\ifresenja
\begin{Answer}[ref=p2.6_05]
\includecode{resenja/2_PredstavljanjePodataka/2.2_Pokazivaci/praktikumi12/8.c}
\end{Answer}
 \fi


\begin{Exercise}[label=v2.2_05] 
Napisati funkciju koja za dva data stringa određuje koliko se uzastopnih karaktera prvog stringa nalazi u drugom stringu počev od početka. Napisati 
   program koji testira napisanu funkciju  za dva stringa 
   koji se unose kao argumenti komandne linije.   \\
\begin{miditest}
\begin{upotreba}{1}
#\poziv{./a.out aladin bal}#
#\naslovInt#
#\izlaz{3}#
\end{upotreba}
\end{miditest}
\begin{miditest}
\begin{upotreba}{2}
#\poziv{./a.out aladin lad}#
#\naslovInt#
#\izlaz{4}#
\end{upotreba}
\end{miditest}
\begin{miditest}
\begin{upotreba}{3}
#\poziv{./a.out Aladin ala}#
#\naslovInt#
#\izlaz{0}#
\end{upotreba}
\end{miditest}
\begin{miditest}
\begin{upotreba}{4}
#\poziv{./a.out aladin}#
#\naslovInt#
#\izlaz{Nekorektan poziv}#
#\izlaz{Program treba pozvati sa ./a.out arg1 arg2}#
\end{upotreba}
\end{miditest}

\linkresenje{v2.2_05}
\end{Exercise}
\ifresenja
\begin{Answer}[ref=v2.2_05]
\includecode{resenja/2_PredstavljanjePodataka/2.2_Pokazivaci/1_05.c}
\end{Answer}
 \fi



\begin{Exercise}[label=v2.2_01] 
 Napisati program koji
ispisuje sve opcije koje su navedene u komandnoj liniji. \\
\begin{miditest}
\begin{upotreba}{1}
#\poziv{./a.out -abc input.txt -d -Fg output}#
#\naslovInt#
#\izlaz{a b c d F g}#
\end{upotreba}
\end{miditest}
\begin{miditest}
\begin{upotreba}{2}
#\poziv{./a.out}#
#\naslovInt#
#\izlaz{}#
\end{upotreba}
\end{miditest}
\begin{miditest}
\begin{upotreba}{3}
#\poziv{./a.out ulaz.txt  }#
#\naslovInt#
#\izlaz{}#
\end{upotreba}
\end{miditest}
\begin{miditest}
\begin{upotreba}{4}
#\poziv{./a.out file.txt -x -yZ -g output}#
#\naslovInt#
#\izlaz{x yZ g}#
\end{upotreba}
\end{miditest}

\linkresenje{v2.2_01}
\end{Exercise}
\ifresenja
\begin{Answer}[ref=v2.2_01]
\includecode{resenja/2_PredstavljanjePodataka/2.2_Pokazivaci/1_07.c}
\end{Answer}
 \fi


\begin{Exercise}[label=v2.2_06] 
Napisati funkciju \kckod{void sifruj(char s[], char c, int k)} koja šifruje
   string s na sledeći način: svako malo i veliko slovo stringa s konvertuje u
   slovo koje je u abecedi od njega udaljeno k pozicija, i to 
   k pozicija ulevo, ako je karakter c jednak karakteru 'L' ili udesno
   ako je karakter c jednak karakteru 'D'. Šifrovanje treba da bude kružno. Ako string
   s sadrži karakter koji nije alfanumerički, ostaviti ga nešifriranog. Napisati program koji testira napisanu funkciju za string i prirodan
   broj koji se unose kao argumenti komandne linije dok se pravac šifrovanja unosi
   kao opcija -p koja može imati vrednosti 'L' ili 'D'. Ukoliko opcija -p nije 
   navedena, podrazumevani pravac je udesno. \napomena{Možemo podrazumevati da string sadrži najviše 30 karaktera}.\\
\begin{miditest}
\begin{upotreba}{1}
#\poziv{./a.out abcd 2}#
#\naslovInt#
#\izlaz{cdef}#
\end{upotreba}
\end{miditest}
\begin{miditest}
\begin{upotreba}{2}
#\poziv{./a.out abcd 2 -p D}#
#\naslovInt#
#\izlaz{cdef}#
\end{upotreba}
\end{miditest}
\begin{miditest}
\begin{upotreba}{3}
#\poziv{./a.out abcd 2 -p L}#
#\naslovInt#
#\izlaz{yzab}#
\end{upotreba}
\end{miditest}
\begin{miditest}
\begin{upotreba}{4}
#\poziv{./a.out abcd -3 -p L}#
#\naslovInt#
#\izlaz{Nekorektan unos}#
\end{upotreba}
\end{miditest}
\begin{miditest}
\begin{upotreba}{5}
#\poziv{./a.out abcd 3 -p X}#
#\naslovInt#
#\izlaz{Nekorektan unos}#
\end{upotreba}
\end{miditest}
\begin{miditest}
\begin{upotreba}{6}
#\poziv{./a.out ab12cd 2 -p D}#
#\naslovInt#
#\izlaz{Nekorektan unos}#
\end{upotreba}
\end{miditest}

\linkresenje{v2.2_06}
\end{Exercise}
\ifresenja
\begin{Answer}[ref=v2.2_06]
\includecode{resenja/2_PredstavljanjePodataka/2.2_Pokazivaci/1_06.c}
\end{Answer}
 \fi

\iffalse
\begin{Exercise}[label=v2.2_01] 
Tekst
%\komentarJ{Ovaj zadatak nema mnogo smisla, predlazem da ga izbacimo.}
\linkresenje{v2.2_01}
\end{Exercise}
\ifresenja
\begin{Answer}[ref=v2.2_01]
\includecode{resenja/2_PredstavljanjePodataka/2.2_Pokazivaci/1_07.c}
\end{Answer}
 \fi
\fi

\iffalse
\begin{Exercise}[label=v2.2_01] 
%\komentarJ{Da li je ovo zadatak iz pokazivaca? Link ka resenju je pogresan, da nije doslo do greske?}
Ako su celi brojevi \verb|a| i \verb|b| argumenti komandne linije
napraviti niz \verb|A[0] = a, A[1] = a+1,|
\verb|A[2] = a+2, ..., A[b-a] = b| i ispisati ga. Pretpostaviti da je
maksimalna du\v zina niza 200 elemenata. Proveriti da li $a < b$ i
$b-a < 200$ i ako ovi uslovi nisu ispunjeni ispisati poruku da je do\v
slo do gre\v ske. U slu\v caju da je dato manje ili vi\v se argumenata
komandne linije ispisati poruku o gre\v sci. \\ 
\begin{miditest}
\begin{upotreba}{1}
#\poziv{./a.out 34}#
#\naslovInt#
#\izlaz{greska}#
\end{upotreba}
\end{miditest}
\begin{miditest}
\begin{upotreba}{2}
#\poziv{./a.out 12 20}#
#\naslovInt#
#\izlaz{12 13 14 15 16 17 18 19 20}#
\end{upotreba}
\end{miditest}
\begin{miditest}
\begin{upotreba}{3}
#\poziv{./a.out 30 8}#
#\naslovInt#
#\izlaz{greska}#
\end{upotreba}
\end{miditest}
\begin{miditest}
\begin{upotreba}{4}
#\poziv{./a.out -4 -1}#
#\naslovInt#
#\izlaz{-4 -3 -2 -1}#
\end{upotreba}
\end{miditest}
\linkresenje{v2.2_01}
\end{Exercise}
\ifresenja
\begin{Answer}[ref=v2.2_01]
\includecode{resenja/2_PredstavljanjePodataka/2.2_Pokazivaci/1_07.c}
\end{Answer}
 \fi

\fi

\begin{Exercise}[label=v2.2_01] 
%\komentarJ{Da li je ovo zadatak iz pokazivaca? Link ka resenju je pogresan, da nije doslo do greske?}
Parametri komandne linije su $n, a$ i $b$ ($a < b$). Treba popuniti
prvih {\tt n} elemenata niza {\tt A} celim slu\v cajnim brojevima koji
su između {\tt a} i {\tt b}. I\v stampati niz {\tt A} na standarni
izlaz. Maksimalan broj elemenata niza {\tt A} je 200. Ukoliko nisu
zadati svi argumenti komandne linije ili ne zadovoljavaju potrebna
svojstva ispisati poruku o gre\v sci. 

%\linkresenje{v2.2_01}
\end{Exercise}
%\ifresenja
%\begin{Answer}[ref=v2.2_01]
%\includecode{resenja/2_PredstavljanjePodataka/2.2_Pokazivaci/1_07.c}
%\end{Answer}
% \fi



\ifresenja
\section{Rešenja}
\shipoutAnswer
\fi