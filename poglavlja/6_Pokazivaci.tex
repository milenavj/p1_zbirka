
\section{Pokazivači}


\begin{Exercise}[label=v2.2_01] 
Tekst
\linkresenje{v2.2_01}
\end{Exercise}
\begin{Answer}[ref=v2.2_01]
\includecode{resenja/2_PredstavljanjePodataka/2.2_Pokazivaci/1_01.c}
\end{Answer}

\begin{Exercise}[label=v2.2_02] 
Tekst
\linkresenje{v2.2_02}
\end{Exercise}
\begin{Answer}[ref=v2.2_02]
\includecode{resenja/2_PredstavljanjePodataka/2.2_Pokazivaci/1_02.c}
\end{Answer}

\begin{Exercise}[label=v2.2_03] 
Tekst
\linkresenje{v2.2_03}
\end{Exercise}
\begin{Answer}[ref=v2.2_03]
\includecode{resenja/2_PredstavljanjePodataka/2.2_Pokazivaci/1_03.c}
\end{Answer}

\begin{Exercise}[label=v2.2_04] 
Tekst
\linkresenje{v2.2_04}
\end{Exercise}
\begin{Answer}[ref=v2.2_04]
\includecode{resenja/2_PredstavljanjePodataka/2.2_Pokazivaci/1_04.c}
\end{Answer}

\begin{Exercise}[label=v2.2_05] 
Tekst
\linkresenje{v2.2_05}
\end{Exercise}
\begin{Answer}[ref=v2.2_05]
\includecode{resenja/2_PredstavljanjePodataka/2.2_Pokazivaci/1_05.c}
\end{Answer}

\begin{Exercise}[label=v2.2_06] 
Tekst
\linkresenje{v2.2_06}
\end{Exercise}
\begin{Answer}[ref=v2.2_06]
\includecode{resenja/2_PredstavljanjePodataka/2.2_Pokazivaci/1_06.c}
\end{Answer}



\begin{Exercise}[label=p2.6_01] 
Napisati program koji ispisuje zbir numeričkih argumenata komandne linije. Napomena: može se koristi funkcija \textit{atoi}.\\
\begin{miditest}
\begin{upotreba}{1}
#\poziv{./a.out 5 mkp 9 -2 11 a 4 2}#
#\naslovInt#
#\izlaz{Zbir numerickih argumenata: 29}#
\end{upotreba}
\end{miditest}
\begin{miditest}
\begin{upotreba}{2}
#\poziv{./a.out ab u f hj}#
#\naslovInt#
#\izlaz{Zbir numerickih argumenata: 0}#
\end{upotreba}
\end{miditest}
\begin{miditest}
\begin{upotreba}{3}
#\poziv{./a.out 33 1 p 44}#
#\naslovInt#
#\izlaz{Zbir numerickih argumenata: 78}#
\end{upotreba}
\end{miditest}

\begin{miditest}
\begin{upotreba}{4}
#\poziv{./a.out}#
#\naslovInt#
#\izlaz{Zbir numerickih argumenata: 0}#
\end{upotreba}
\end{miditest}
\linkresenje{p2.6_01}
\end{Exercise}
\begin{Answer}[ref=p2.6_01]
\includecode{resenja/2_PredstavljanjePodataka/2.2_Pokazivaci/praktikumi12/4.c}
\end{Answer}


\begin{Exercise}[label=p2.6_02] 
Napisati program koji ispisuje argumente komandne linije koji počinju slovom \textit{z}.\\
\begin{miditest}
\begin{upotreba}{1}
#\poziv{./a.out zima jabuka zvezda Zrak}#
#\naslovInt#
#\izlaz{zima zvezda}#
\end{upotreba}
\end{miditest}
\begin{miditest}
\begin{upotreba}{2}
#\poziv{./a.out bundeva pomorandza}#
#\naslovInt#
#\izlaz{}#
\end{upotreba}
\end{miditest}
\begin{miditest}
\begin{upotreba}{3}
#\poziv{./a.out sanke zapad zujanje}#
#\naslovInt#
#\izlaz{zapad zujanje}#
\end{upotreba}
\end{miditest}

\begin{miditest}
\begin{upotreba}{4}
#\poziv{./a.out}#
#\naslovInt#
#\izlaz{}#
\end{upotreba}
\end{miditest}
\linkresenje{p2.6_02}
\end{Exercise}
\begin{Answer}[ref=p2.6_02]
\includecode{resenja/2_PredstavljanjePodataka/2.2_Pokazivaci/praktikumi12/5.c}
\end{Answer}


\begin{Exercise}[label=p2.6_03] 
Napisati program koji ispisuje broj argumenata komandne linije koji sadrže slovo \textit{z}.\\
\begin{miditest}
\begin{upotreba}{1}
#\poziv{./a.out zvezda grozd jesen kisa}#
#\naslovInt#
#\izlaz{2}#
\end{upotreba}
\end{miditest}
\begin{miditest}
\begin{upotreba}{2}
#\poziv{./a.out AZBUKA deda mraz }#
#\naslovInt#
#\izlaz{2}#
\end{upotreba}
\end{miditest}
\begin{miditest}
\begin{upotreba}{3}
#\poziv{./a.out japan caj}#
#\naslovInt#
#\izlaz{0}#
\end{upotreba}
\end{miditest}

\begin{miditest}
\begin{upotreba}{4}
#\poziv{./a.out}#
#\naslovInt#
#\izlaz{0}#
\end{upotreba}
\end{miditest}


\linkresenje{p2.6_03}
\end{Exercise}
\begin{Answer}[ref=p2.6_03]
\includecode{resenja/2_PredstavljanjePodataka/2.2_Pokazivaci/praktikumi12/6.c}
\end{Answer}


\begin{Exercise}[label=p2.6_04] 
 Napisati program koji na osnovu broja \textit{n} koji se zadaje kao argument komandne linije ispisuje cele brojeve iz intervala $[-n,\ n]$. \\
\begin{miditest}
\begin{upotreba}{1}
#\poziv{./a.out 2}#
#\naslovInt#
#\izlaz{-2 -1 0 1 2}#
\end{upotreba}
\end{miditest}
\begin{miditest}
\begin{upotreba}{2}
#\poziv{./a.out 4}#
#\naslovInt#
#\izlaz{-4 -3 -2 -1 0 1 2 3 4}#
\end{upotreba}
\end{miditest}
\begin{miditest}
\begin{upotreba}{3}
#\poziv{./a.out 0}#
#\naslovInt#
#\izlaz{0}#
\end{upotreba}
\end{miditest}

\begin{maxitest}
\begin{upotreba}{4}
#\poziv{./a.out}#
#\naslovInt#
#\izlaz{Greska: nedostaje argument komandne linije!}#
\end{upotreba}
\end{maxitest}

\linkresenje{p2.6_04}
\end{Exercise}
\begin{Answer}[ref=p2.6_04]
\includecode{resenja/2_PredstavljanjePodataka/2.2_Pokazivaci/praktikumi12/7.c}
\end{Answer}


\begin{Exercise}[label=p2.6_05] 
 Napisati program koji proverava da li se među zadatim argumentima komandne linije nalaze barem dva ista.  \\
\begin{miditest}
\begin{upotreba}{1}
#\poziv{./a.out pec zima deda mraz pec}#
#\naslovInt#
#\izlaz{Medju argumentima ima istih.}#
\end{upotreba}
\end{miditest}
\begin{miditest}
\begin{upotreba}{2}
#\poziv{./a.out xyz abc abc abc efgh}#
#\naslovInt#
#\izlaz{Medju argumentima ima istih.}#
\end{upotreba}
\end{miditest}
\begin{miditest}
\begin{upotreba}{3}
#\poziv{./a.out 11 15 abc 888}#
#\naslovInt#
#\izlaz{Medju argumentima nema istih.}#
\end{upotreba}
\end{miditest}

\begin{miditest}
\begin{upotreba}{4}
#\poziv{./a.out}#
#\naslovInt#
#\izlaz{Medju argumentima nema istih.}#
\end{upotreba}
\end{miditest}

\linkresenje{p2.6_05}
\end{Exercise}
\begin{Answer}[ref=p2.6_05]
\includecode{resenja/2_PredstavljanjePodataka/2.2_Pokazivaci/praktikumi12/8.c}
\end{Answer}


\begin{Exercise}[label=p2.6_06] 
 Napisati funkciju $void\ modifikacija(char*\ s,\ char*\ t, int*\ br\_modifikacija)$ koja na osnovu niske $s$ formira nisku $t$ tako što svako malo slovo zamanjuje velikim. Broj izvršenih modifikacija se čuva u okviru argumenta $br\_modifikacija$. Pretpostaviti da niska $s$ neće biti duža od 20 karaktera. Napisati i program koji testira rad napisane funkcije. \\
\begin{miditest}
\begin{upotreba}{1}
#\naslovInt#
#\izlaz{Unesite nisku:}\ulaz{123abc789XY}#
#\izlaz{Modifikovana niska je: 123ABC789XY}#
#\izlaz{Broj modifikacija je: 3}#
\end{upotreba}
\end{miditest}
\begin{miditest}
\begin{upotreba}{2}
#\naslovInt#
#\izlaz{Unesite nisku:}\ulaz{zimA}#
#\izlaz{Modifikovana niska je: ZIMA}#
#\izlaz{Broj modifikacija je: 3}#
\end{upotreba}
\end{miditest}
\begin{miditest}
\begin{upotreba}{3}
#\naslovInt#
#\izlaz{Unesite nisku:}\ulaz{SNEG}#
#\izlaz{Modifikovana niska je: SNEG}#
#\izlaz{Broj modifikacija je: 0}#
\end{upotreba}
\end{miditest}

\linkresenje{p2.6_06}
\end{Exercise}
\begin{Answer}[ref=p2.6_06]
\includecode{resenja/2_PredstavljanjePodataka/2.2_Pokazivaci/praktikumi12/9.c}
\end{Answer}


\begin{Exercise}[label=p2.6_07] 
 Napisati funkciju $void\ interpunkcija(int*\ br\_tacaka,\ int*\ br\_zareza)$ koja za tekst koji se unosi sa standardnog ulaza sve do kraja ulaza prebrojava broj tačaka i zareza. Napisati zatim program koji testira napisanu funkciju.\\
\begin{miditest}
\begin{upotreba}{1}
#\naslovInt#
#\izlaz{Unesite tekst:}#
#\ulaz{a.b.c.d}#
#\ulaz{a,b,,c,d,e}#
#\izlaz{Broj tacaka: 3}#
#\izlaz{Broj zareza: 5}#
\end{upotreba}
\end{miditest}
\begin{miditest}
\begin{upotreba}{2}
#\naslovInt#
#\izlaz{Unesite tekst:}#
#\ulaz{.....789.....}#
#\izlaz{Broj tacaka: 10}#
#\izlaz{Broj zareza: 0}#
\end{upotreba}
\end{miditest}
\begin{miditest}
\begin{upotreba}{2}
#\naslovInt#
#\izlaz{Unesite tekst:}#
#\ulaz{sunce}#
#\izlaz{Broj tacaka: 0}#
#\izlaz{Broj zareza: 0}#
\end{upotreba}
\end{miditest}

\linkresenje{p2.6_07}
\end{Exercise}
\begin{Answer}[ref=p2.6_07]
\includecode{resenja/2_PredstavljanjePodataka/2.2_Pokazivaci/praktikumi12/10.c}
\end{Answer}


\begin{Exercise}[label=p2.6_08] 
 Napisati funkciju $void\ par\_nepar(int\ a[],\ int\ n,\ int\ parni[],\ int*\ pn,\ int\ neparni[],\ int*\ nn)$ koja razbija niz $a$ na niz parnih i niz neparnih brojeva. Pokazivači $pn$ i $nn$ redom treba da sadrže broj elemenata niza parnih tj. niza neparnih elemenata. Pretpostaviti da dužina niza $a$ neće biti veća od 50. Napisati program koji testira napisanu funkciju.\\
\begin{miditest}
\begin{upotreba}{1}
#\naslovInt#
#\izlaz{Unesite broj elemenata niza: }\ulaz{8}#
#\izlaz{Unesite elemente niza: }#
#\ulaz{1 8 9 -7 -16 24 77 4}#
#\izlaz{Niz parnih brojeva: 8 -16 24 4}#
#\izlaz{Niz neparnih brojeva: 1 9 -7 77}#
\end{upotreba}
\end{miditest}
\begin{miditest}
\begin{upotreba}{2}
#\naslovInt#
#\izlaz{Unesite broj elemenata niza: }\ulaz{5}#
#\izlaz{Unesite elemente niza: }#
#\ulaz{2 4 6 8 -11}#
#\izlaz{Niz parnih brojeva: 2 4 6 8}#
#\izlaz{Niz neparnih brojeva: -11}#
\end{upotreba}
\end{miditest}
\begin{miditest}
\begin{upotreba}{3}
#\naslovInt#
#\izlaz{Unesite broj elemenata niza: }\ulaz{2}#
#\izlaz{Unesite elemente niza: }#
#\ulaz{-15 15}#
#\izlaz{Niz parnih brojeva: }#
#\izlaz{Niz neparnih brojeva: -15 15}#
\end{upotreba}
\end{miditest}

\linkresenje{p2.6_08}
\end{Exercise}
\begin{Answer}[ref=p2.6_08]
\includecode{resenja/2_PredstavljanjePodataka/2.2_Pokazivaci/praktikumi12/11.c}
\end{Answer}


\begin{Exercise}[label=p2.6_09] 
 Napisati funckiju $void\ min\_max(float\ a[],\ int\ n,\ float*\ min,\ float*\ max)$ koja izračunava minimalni i maksimalni element niza $a$ dužine $n$. Napisati zatim i program koji učitava niz realnih brojeva maksimalne dužine 50 i ispisuje vrednosti minimuma i maksimuma na tri decimale. \\
\begin{miditest}
\begin{upotreba}{1}
#\naslovInt#
#\izlaz{Unesite broj elemenata niza: }\ulaz{5}#
#\izlaz{Unesite elemente niza: }#
#\ulaz{24.16 -32.11 999.25 14.25 11}#
#\izlaz{Minimum: -32.110}#
#\izlaz{Maksimum: 999.250}#
\end{upotreba}
\end{miditest}
\begin{miditest}
\begin{upotreba}{2}
#\naslovInt#
#\izlaz{Unesite broj elemenata niza: }\ulaz{4}#
#\izlaz{Unesite elemente niza: }#
#\ulaz{-5.126 -18.29 44 29.268}#
#\izlaz{Minimum: -18.290}#
#\izlaz{Maksimum: 44.000}#
\end{upotreba}
\end{miditest}
\begin{miditest}
\begin{upotreba}{3}
#\naslovInt#
#\izlaz{Unesite broj elemenata niza: }\ulaz{1}#
#\izlaz{Unesite elemente niza: }#
#\ulaz{4.16}#
#\izlaz{Minimum: 4.160}#
#\izlaz{Maksimum: 4.160}#
\end{upotreba}
\end{miditest}

\linkresenje{p2.6_09}
\end{Exercise}
\begin{Answer}[ref=p2.6_09]
\includecode{resenja/2_PredstavljanjePodataka/2.2_Pokazivaci/praktikumi12/12.c}
\end{Answer}

\begin{Exercise}[label=v2.2_01] 
Tekst
\linkresenje{v2.2_01}
\end{Exercise}
\begin{Answer}[ref=v2.2_01]
\includecode{resenja/2_PredstavljanjePodataka/2.2_Pokazivaci/1_07.c}
\end{Answer}


\begin{Exercise}[label=v2.2_01] 
Ako su celi brojevi \verb|a| i \verb|b| argumenti komandne linije
napraviti niz \verb|A[0] = a, A[1] = a+1,|
\verb|A[2] = a+2, ..., A[b-a] = b| i ispisati ga. Pretpostaviti da je
maksimalna du\v zina niza 200 elemenata. Proveriti da li $a < b$ i
$b-a < 200$ i ako ovi uslovi nisu ispunjeni ispisati poruku da je do\v
slo do gre\v ske. U slu\v caju da je dato manje ili vi\v se argumenata
komandne linije ispisati poruku o gre\v sci. \\ 
\begin{miditest}
\begin{upotreba}{1}
#\poziv{./a.out 34}#
#\naslovInt#
#\izlaz{greska}#
\end{upotreba}
\end{miditest}
\begin{miditest}
\begin{upotreba}{2}
#\poziv{./a.out 12 20}#
#\naslovInt#
#\izlaz{12 13 14 15 16 17 18 19 20}#
\end{upotreba}
\end{miditest}
\begin{miditest}
\begin{upotreba}{3}
#\poziv{./a.out 30 8}#
#\naslovInt#
#\izlaz{greska}#
\end{upotreba}
\end{miditest}
\begin{miditest}
\begin{upotreba}{4}
#\poziv{./a.out -4 -1}#
#\naslovInt#
#\izlaz{-4 -3 -2 -1}#
\end{upotreba}
\end{miditest}
\linkresenje{v2.2_01}
\end{Exercise}
\begin{Answer}[ref=v2.2_01]
\includecode{resenja/2_PredstavljanjePodataka/2.2_Pokazivaci/1_07.c}
\end{Answer}


\begin{Exercise}[label=v2.2_01] 
Uobi\v cajena praksa na UNIX sistemima je da se argumenti komandne
linije dele na opcije i argumente u u\v zem smislu. Opcije po\v cinju
znakom ’-’ nakon \v cega obi\v cno sledi jedan ili vi\v se karaktera
koji ozna\v cavaju koja je opcija u pitanju. Ovim se naj\v ce\v s\' ce
upravlja funkcionisanjem programa i neke mogu\' cnosti se uklju\v cuju
ili isklju\v cuju. Argumenti na\v c\v s\' ce predstavljaju opisne
informacije poput na primer imena datoteka. Napisati program koji
ispisuje sve opcije koje su navedene u komandnoj liniji. \\
\begin{miditest}
\begin{upotreba}{1}
#\poziv{./a.out -abc input.txt -d -Fg output}#
#\naslovInt#
#\izlaz{a b c d F g}#
\end{upotreba}
\end{miditest}
\begin{miditest}
\begin{upotreba}{2}
#\poziv{./a.out}#
#\naslovInt#
#\izlaz{}#
\end{upotreba}
\end{miditest}
\begin{miditest}
\begin{upotreba}{3}
#\poziv{./a.out ulaz.txt  }#
#\naslovInt#
#\izlaz{}#
\end{upotreba}
\end{miditest}
\linkresenje{v2.2_01}
\end{Exercise}
\begin{Answer}[ref=v2.2_01]
\includecode{resenja/2_PredstavljanjePodataka/2.2_Pokazivaci/1_07.c}
\end{Answer}


\begin{Exercise}[label=v2.2_01] 
Parametri komandne linije su {\tt n, a, b} ($a < b$). Treba popuniti
prvih {\tt n} elemenata niza {\tt A} celim slu\v cajnim brojevima koji
su izme\d u {\tt a} i {\tt b}. I\v stampati niz {\tt A} na standarni
izlaz. Maksimalan broj elemenata niza {\tt A} je 200. Ukoliko nisu
zadati svi argumenti komandne linije ili ne zadovoljavaju potrebna
svojstva ispisati poruku o gre\v sci. \\
\linkresenje{v2.2_01}
\end{Exercise}
\begin{Answer}[ref=v2.2_01]
\includecode{resenja/2_PredstavljanjePodataka/2.2_Pokazivaci/1_07.c}
\end{Answer}



\section{Rešenja}
\shipoutAnswer
