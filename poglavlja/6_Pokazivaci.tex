
\section{Pokazivači}


\begin{Exercise}[label=POK_01] 
Napisati funkciju \kckod{void uredi(int *pa, int *pb)} koja uređuje svoja dva celobrojna argumenta tako da se
u prvom nalazi manja vrednost, a u drugom veća. 
Napisati program koji učitava dva cela broja i ispisuje uređene brojeve. 

\begin{miditest}
\begin{upotreba}{1}
#\naslovInt#
#\izlaz{Unesite dva broja: }\ulaz{2 5}#
#\izlaz{Uredjene promenljive: 2, 5}#
\end{upotreba}
\end{miditest}
\begin{miditest}
\begin{upotreba}{2}
#\naslovInt#
#\izlaz{Unesite dva broja: }\ulaz{11 -4}#
#\izlaz{Uredjene promenljive: -4, 11}#
\end{upotreba}
\end{miditest}

\linkresenje{POK_01}
\end{Exercise}
\ifresenja
\begin{Answer}[ref=POK_01]
\includecode{resenja/3_PredstavljanjePodataka/2.2_Pokazivaci/sve/pok01.c}
\end{Answer}
 \fi

 
\begin{Exercise}[label=POK_02] 
  Napisati funkciju \kckod{void rgb\_u\_cmy(int r, int g, int b, float* c, float* m, float* y)} koja datu boju u \textit{rgb} 
  formatu konvertuje u boju u \textit{cmy} formatu po sledećim formulama:\\
  $$c = 1 - r / 255$$
  $$m = 1 - g / 255$$
  $$y = 1 - b / 255$$
  Napisati program koji učitava boju u \textit{rgb} formatu i ispisuje vrednosti unete boje u \textit{cmy} formatu.
  U slučaju neispravnog unosa, ispisati odgovarajuću poruku o grešci. 
  \napomena{Vrednosti boja u \textit{rgb} formatu su u opsegu $[0,255]$.}\\

\begin{miditest}
\begin{upotreba}{1}
#\naslovInt#
#\izlaz{Unesite boju u rgb formatu:}\ulaz{56 111 24}#
#\izlaz{cmy: (0.78, 0.56, 0.91)}#
\end{upotreba}
\end{miditest}
\begin{miditest}
\begin{upotreba}{2}
#\naslovInt#
#\izlaz{Unesite boju u rgb formatu:}\ulaz{156 -90 5}#
#\izlaz{Greska: neispravan unos.}#
\end{upotreba}
\end{miditest}

\begin{miditest}
\begin{upotreba}{3}
#\naslovInt#
#\izlaz{Unesite boju u rgb formatu:}\ulaz{9 0 237}#
#\izlaz{cmy: (0.96, 1.00, 0.07)}#
\end{upotreba}
\end{miditest}
\begin{miditest}
\begin{upotreba}{4}
#\naslovInt#
#\izlaz{Unesite boju u rgb formatu:}\ulaz{300 11 27}#
#\izlaz{Greska: neispravan unos.}#
\end{upotreba}
\end{miditest}

\linkresenje{POK_02}
\end{Exercise}
\ifresenja
\begin{Answer}[ref=POK_02]
\includecode{resenja/3_PredstavljanjePodataka/2.2_Pokazivaci/sve/pok02.c}
\end{Answer}
 \fi

 
\begin{Exercise}[label=POK_03] 
Napisati funkciju \kckod{int presek(float k1, float n1, float k2, float n2, float *px, float *py)} 
koja za dve prave date svojim koeficijentima
pravca i slobodnim članovima određuje njihovu tačku preseka. 
Funkcija treba da vrati jedinicu ako se prave seku i nulu ako nemaju
tačku preseka (ako su paralelne). Napisati program
koji učitava podatke o pravama i ukoliko prave imaju presek, ispisuje
koordinate tačke preseka, a ako nemaju, ispisuje odgovarajuću poruku.
   
\begin{miditest}
\begin{upotreba}{1}
#\naslovInt#
#\izlaz{Unesite k i n za prvu pravu:}\ulaz{4 5}#
#\izlaz{Unesite k i n za drugu pravu:}\ulaz{11 -4}#
#\izlaz{Prave se seku u tacki (1.29,10.14).}#
\end{upotreba}
\end{miditest}
\begin{miditest}
\begin{upotreba}{2}
#\naslovInt#
#\izlaz{Unesite k i n za prvu pravu:}\ulaz{0.5 -4.7}#
#\izlaz{Unesite k i n za drugu pravu:}\ulaz{0.5 9.1}#
#\izlaz{Prave su paralelne.}#
\end{upotreba}
\end{miditest}

\linkresenje{POK_03}
\end{Exercise}
\ifresenja
\begin{Answer}[ref=POK_03]
\includecode{resenja/3_PredstavljanjePodataka/2.2_Pokazivaci/sve/pok03.c}
\end{Answer}
 \fi

 
\begin{Exercise}[label=POK_04] 
Napisati funkciju koja za dva cela broja izračunava njihov količnik i ostatak pri deljenju. 
Funkcija treba da vrati vrednost jedinicu ukoliko je uspešno izračunala vrednosti, 
a nulu ukoliko deljenje nije moguće. Napisati program koji učitava dva cela broja i 
ispisuje njihov količnik i ostatak pri deljenju. 
U slučaju neispravnog unosa, ispisati odgovarajuću poruku o grešci. 

\begin{minitest}
\begin{upotreba}{1}
#\naslovInt#
#\izlaz{Unesite brojeve:}\ulaz{4 5}#
#\izlaz{Kolicnik: 0}#
#\izlaz{Ostatak: 4}#
\end{upotreba}
\end{minitest}
\begin{minitest}
\begin{upotreba}{2}
#\naslovInt#
#\izlaz{Unesite brojeve:}\ulaz{4 0}#
#\izlaz{Greska: neispravan unos.}#
\end{upotreba}
\end{minitest}
\begin{minitest}
\begin{upotreba}{3}
#\naslovInt#
#\izlaz{Unesite brojeve:}\ulaz{-123 11}#
#\izlaz{Kolicnik: -11}#
#\izlaz{Ostatak: -2}#
\end{upotreba}
\end{minitest}

\linkresenje{POK_04}
\end{Exercise}
\ifresenja
\begin{Answer}[ref=POK_04]
\includecode{resenja/3_PredstavljanjePodataka/2.2_Pokazivaci/sve/pok04.c}
\end{Answer}
\fi


\begin{Exercise}[label=POK_05] 
Napisati funkciju koja za dužinu trajanja filma koje je dato u sekundama, 
određuje ukupno trajanje filma u satima, minutama i sekundama. 
Napisati program koji učitava trajanje filma u sekundama i ispisuje odgovarajuće
vreme trajanja u formatu \textit{broj\_sati:broj\_minuta:broj\_sekundi}.
U slučaju neispravnog unosa, ispisati odgovarajuću poruku o grešci. 

\begin{miditest}
\begin{upotreba}{1}
#\naslovInt#
#\izlaz{Trajanje fima u sekundama:}\ulaz{5000}#
#\izlaz{1h:23m:20s}#
\end{upotreba}
\end{miditest}
\begin{miditest}
\begin{upotreba}{2}
#\naslovInt#
#\izlaz{Trajanje fima u sekundama:}\ulaz{-300}#
#\izlaz{Greska: neispravan unos.}#
\end{upotreba}
\end{miditest}

\begin{miditest}
\begin{upotreba}{3}
#\naslovInt#
#\izlaz{Trajanje fima u sekundama:}\ulaz{2500}#
#\izlaz{0h:41m:40s}#
\end{upotreba}
\end{miditest}
\begin{miditest}
\begin{upotreba}{4}
#\naslovInt#
#\izlaz{Trajanje fima u sekundama:}\ulaz{7824}#
#\izlaz{2h:10m:24s}#
\end{upotreba}
\end{miditest}

\linkresenje{POK_05}
\end{Exercise}
\ifresenja
\begin{Answer}[ref=POK_05]
\includecode{resenja/3_PredstavljanjePodataka/2.2_Pokazivaci/sve/pok05.c}
\end{Answer}
\fi


\begin{Exercise}[label=POK_06] 
 Napisati funkciju koja sa ulaza učitava karakter po karakter sve do kraja ulaza i prebrojava sva pojavljivanja
 karaktera tačka i sva pojavljivanja karaktera zarez. 
 Napisati program koji za uneti tekst ispisuje koliko puta se pojavila tačka, a koliko
 puta se pojavio zarez.\\
 
\begin{minitest}
\begin{upotreba}{1}
#\naslovInt#
#\izlaz{Unesite tekst:}#
#\ulaz{a.b.c.d}#
#\ulaz{a,b,,c,d,e}#
#\izlaz{Broj tacaka: 3}#
#\izlaz{Broj zareza: 5}#
\end{upotreba}
\end{minitest}
\begin{minitest}
\begin{upotreba}{2}
#\naslovInt#
#\izlaz{Unesite tekst:}#
#\ulaz{.....789.....}#
#\izlaz{Broj tacaka: 10}#
#\izlaz{Broj zareza: 0}#
\end{upotreba}
\end{minitest}
\begin{minitest}
\begin{upotreba}{3}
#\naslovInt#
#\izlaz{Unesite tekst:}#
#\ulaz{sunce}#
#\izlaz{Broj tacaka: 0}#
#\izlaz{Broj zareza: 0}#
\end{upotreba}
\end{minitest}

\linkresenje{POK_06}
\end{Exercise}
\ifresenja
\begin{Answer}[ref=POK_06]
\includecode{resenja/3_PredstavljanjePodataka/2.2_Pokazivaci/sve/pok06.c}
\end{Answer}
 \fi


\begin{Exercise}[label=POK_07] 
 Napisati funkciju \kckod{void par\_nepar(int a[], int n, int parni[], int* np, int neparni[], int* nn)} 
 koja razbija niz $a$ na niz parnih i niz neparnih brojeva. 
 Pokazivači $np$ i $nn$ redom treba da sadrže broj elemenata niza parnih tj. niza neparnih elemenata. 
 Maksimalan broj elemenata niza je $50$. 
 Napisati program koji učitava dimenziju niza, a zatim i elemente niza i 
 ispisuje odgovarajuće nizove parnih, odnosno neparnih elemenata unetog niza. 
 U slučaju neispravnog unosa, ispisati odgovarajuću poruku o grešci. 

\begin{miditest}
\begin{upotreba}{1}
#\naslovInt#
#\izlaz{Unesite broj elemenata niza:}\ulaz{8}#
#\izlaz{Unesite elemente niza:}#
#\ulaz{1 8 9 -7 -16 24 77 4}#
#\izlaz{Niz parnih brojeva: 8 -16 24 4}#
#\izlaz{Niz neparnih brojeva: 1 9 -7 77}#
\end{upotreba}
\end{miditest}
\begin{miditest}
\begin{upotreba}{2}
#\naslovInt#
#\izlaz{Unesite broj elemenata niza:}\ulaz{5}#
#\izlaz{Unesite elemente niza:}#
#\ulaz{2 4 6 8 -11}#
#\izlaz{Niz parnih brojeva: 2 4 6 8}#
#\izlaz{Niz neparnih brojeva: -11}#
\end{upotreba}
\end{miditest}

\begin{miditest}
\begin{upotreba}{3}
#\naslovInt#
#\izlaz{Unesite broj elemenata niza:}\ulaz{2}#
#\izlaz{Unesite elemente niza: }#
#\ulaz{-15 15}#
#\izlaz{Niz parnih brojeva:}#
#\izlaz{Niz neparnih brojeva: -15 15}#
\end{upotreba}
\end{miditest}
\begin{miditest}
\begin{upotreba}{4}
#\naslovInt#
#\izlaz{Unesite broj elemenata niza:}\ulaz{0}#
#\izlaz{Greska: neispravan unos.}#
\end{upotreba}
\end{miditest}

\linkresenje{POK_07}
\end{Exercise}
\ifresenja
\begin{Answer}[ref=POK_07]
\includecode{resenja/3_PredstavljanjePodataka/2.2_Pokazivaci/sve/pok07.c}
\end{Answer}
 \fi


\begin{Exercise}[label=POK_08] 
 Napisati funckiju koja izračunava najmanji i najveći element niza realnih brojeva. 
 Napisati program koji učitava niz realnih brojeva maksimalne dužine $50$ i 
 ispisuje vrednosti namanjeg i najvećeg elementa niza, zaokružene na tri decimale.
U slučaju neispravnog unosa, ispisati odgovarajuću poruku o grešci. 

\begin{miditest}
\begin{upotreba}{1}
#\naslovInt#
#\izlaz{Unesite broj elemenata niza: }\ulaz{5}#
#\izlaz{Unesite elemente niza: }#
#\ulaz{24.16 -32.11 999.25 14.25 11}#
#\izlaz{Najmanji: -32.110}#
#\izlaz{Najveci: 999.250}#
\end{upotreba}
\end{miditest}
\begin{miditest}
\begin{upotreba}{2}
#\naslovInt#
#\izlaz{Unesite broj elemenata niza: }\ulaz{4}#
#\izlaz{Unesite elemente niza: }#
#\ulaz{-5.126 -18.29 44 29.268}#
#\izlaz{Najmanji: -18.290}#
#\izlaz{Najveci: 44.000}#
\end{upotreba}
\end{miditest}

\begin{miditest}
\begin{upotreba}{3}
#\naslovInt#
#\izlaz{Unesite broj elemenata niza: }\ulaz{1}#
#\izlaz{Unesite elemente niza: }#
#\ulaz{4.16}#
#\izlaz{Najmanji: 4.160}#
#\izlaz{Najveci: 4.160}#
\end{upotreba}
\end{miditest}
\begin{miditest}
\begin{upotreba}{4}
#\naslovInt#
#\izlaz{Unesite broj elemenata niza: }\ulaz{-3}#
#\izlaz{Greska: neispravan unos.}#
\end{upotreba}
\end{miditest}

\linkresenje{POK_08}
\end{Exercise}
\ifresenja
\begin{Answer}[ref=POK_08]
\includecode{resenja/3_PredstavljanjePodataka/2.2_Pokazivaci/sve/pok08.c}
\end{Answer}
 \fi

\ifresenja
\section{Rešenja}
\shipoutAnswer
\fi