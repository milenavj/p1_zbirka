
\section{Niske}

%\subsection{Osnovna konverzija}

\begin{Exercise}[label=NIS_01] 
   Napisati funkciju \kckod{void konvertuj(char s[])} koja menja nisku $s$ tako što
   mala slova zamenjuje odgovarajućim velikim slovima, a velika slova zamenjuje 
   odgovarajućim malim slovima. Napisati program koji učitava nisku maksimalne dužine 
   10 karaktera i ispisuje konvertovanu nisku.
   
\begin{minitest}
\begin{upotreba}{1}
#\naslovInt#
#\izlaz{Unesite nisku:}\ulaz{BeoGrad}#
#\izlaz{Konvertovana niska: bEOgRAD}#
\end{upotreba}
\end{minitest}
\begin{minitest}
\begin{upotreba}{2}
#\naslovInt#
#\izlaz{Unesite nisku:}\ulaz{A+B+C}#
#\izlaz{Konvertovana niska: a+b+c}#
\end{upotreba}
\end{minitest}
\begin{minitest}
\begin{upotreba}{3}
#\naslovInt#
#\izlaz{Unesite nisku:}\ulaz{12345}#
#\izlaz{Konvertovana niska: 12345}#
\end{upotreba}
\end{minitest}

\linkresenje{NIS_01}
\end{Exercise}
\ifresenja
\begin{Answer}[ref=NIS_01]
\includecode{resenja/3_PredstavljanjePodataka/2.3_Niske/sve/niske01.c}
\end{Answer}
\fi


\begin{Exercise}[label=NIS_02] 
 Napisati funkciju \kckod{void ubaci\_zvezdice(char s[])} koja menja nisku $s$ tako što u njoj svaki drugi karakter zamenjuje zvezdicom. 
Napisati program koji učitava nisku maksimalne dužine 20 karaktera i ispisuje izmenjenu nisku. 
 
\begin{minitest}
\begin{upotreba}{1}
#\naslovInt#
#\izlaz{Unesite nisku:}\ulaz{*a*b*c*}#
#\izlaz{Izmenjena niska: *******}#
\end{upotreba}
\end{minitest}
\begin{minitest}
\begin{upotreba}{2}
#\naslovInt#
#\izlaz{Unesite nisku:}\ulaz{zimA}#
#\izlaz{Izmenjena niska: z*m*}#
\end{upotreba}
\end{minitest}
\begin{minitest}
\begin{upotreba}{3}
#\naslovInt#
#\izlaz{Unesite nisku:}\ulaz{123abc789}#
#\izlaz{Izmenjena niska: 1*3*b*7*9}#
\end{upotreba}
\end{minitest}

\linkresenje{NIS_02}
\end{Exercise}
\ifresenja
\begin{Answer}[ref=NIS_02]
\includecode{resenja/3_PredstavljanjePodataka/2.3_Niske/sve/niske02.c}
\end{Answer}
\fi


\begin{Exercise}[label=NIS_03] 
Napisati program koji vrši poređenje niski. Napisati funkcije:
\begin{enumerate}
\setlength\itemsep{0em}
\item \kckod{int jednake(char s1[], char s2[])} koja vraća jedinicu ako su $s_1$ i $s_2$ jednake niske, a nulu inače.
\item \kckod{void u\_velika\_slova(char s[])} koja pretvara sva slova niske $s$ u velika slova, a ostale karaktere ne menja.
\end{enumerate}
Program učitava dve reči maksimalne dužine 20 karaktera i ispituje da li su unete reči jednake. Pri poređenju treba zanemariti
razliku između malih i velikih slova.

\begin{minitest}
\begin{upotreba}{1}
#\naslovInt#
#\izlaz{Unesite niske:}#
#\ulaz{isPit2010}#
#\ulaz{IsPiT2010}#
#\izlaz{Niske su jednake.}#
\end{upotreba}
\end{minitest}
\begin{minitest}
\begin{upotreba}{2}
#\naslovInt#
#\izlaz{Unesite niske:}#
#\ulaz{Prog1}#
#\ulaz{prog2}#
#\izlaz{Niske nisu jednake.}#
\end{upotreba}
\end{minitest}
\begin{minitest}
\begin{upotreba}{3}
#\naslovInt#
#\izlaz{Unesite niske:}#
#\ulaz{jun}#
#\ulaz{JUNSKI}#
#\izlaz{Niske nisu jednake.}#
\end{upotreba}
\end{minitest}

\linkresenje{NIS_03}
\end{Exercise}
\ifresenja
\begin{Answer}[ref=NIS_03]
\includecode{resenja/3_PredstavljanjePodataka/2.3_Niske/sve/niske03.c}
\end{Answer}
\fi

%\subsection{Karakteristike (da/ne)}

\begin{Exercise}[label=NIS_04] 
Napisati program koji proverava da li se uneta niska završava samoglasnikom. Napisati funkcije:
\begin{enumerate}
\setlength\itemsep{0em}
\item  \kckod{int samoglasnik(char c)} koja ispituje da li je karakter $c$ samoglasnik i vraća 1 ako jeste ili 0 ako nije.
\item  \kckod{int samoglasnik\_na\_kraju(char s[])}  koja ispituje da li se niska $s$ završava samoglasnikom. 
\end{enumerate}
Program učitava reč maksimalne dužine 20 karaktera i ispisuje da li se reč završava samoglasnikom ili ne.

\begin{miditest}
\begin{upotreba}{1}
#\naslovInt#
#\izlaz{Unesite nisku:}\ulaz{kestenje}#
#\izlaz{Niska se zavrsava samoglasnikom.}#
\end{upotreba}
\end{miditest}
\begin{miditest}
\begin{upotreba}{2}
#\naslovInt#
#\izlaz{Unesite nisku:}\ulaz{vetar}#
#\izlaz{Niska se ne zavrsava samoglasnikom.}#
\end{upotreba}
\end{miditest}

\begin{miditest}
\begin{upotreba}{3}
#\naslovInt#
#\izlaz{Unesite nisku:}\ulaz{OLUJA}#
#\izlaz{Niska se zavrsava samoglasnikom.}#
\end{upotreba}
\end{miditest}
\begin{miditest}
\begin{upotreba}{4}
#\naslovInt#
#\izlaz{Unesite nisku:}\ulaz{Programiranje1}#
#\izlaz{Niska se ne zavrsava samoglasnikom.}#
\end{upotreba}
\end{miditest}
\linkresenje{NIS_04}
\end{Exercise}
\ifresenja
\begin{Answer}[ref=NIS_04]
\includecode{resenja/3_PredstavljanjePodataka/2.3_Niske/sve/niske04.c}
\end{Answer}
\fi


\begin{Exercise}[label=NIS_05] 
Napisati funkciju \kckod{int sadrzi\_veliko(char s[])} koja proverava da li niska $s$ sadrži veliko slovo. 
Napisati program koji za učitanu nisku maksimalne dužine 20 karaktera proverava da li sadrži veliko slovo i ispisuje odgovarajuću poruku.

\begin{minitest}
\begin{upotreba}{1}
#\naslovInt#
#\izlaz{Unesite nisku:}#
#\ulaz{naocare}#
#\izlaz{Ne sadrzi veliko slovo.}#
\end{upotreba}
\end{minitest}
\begin{minitest}
\begin{upotreba}{2}
#\naslovInt#
#\izlaz{Unesite nisku:}#
#\ulaz{DiopTrija0.75}#
#\izlaz{Sadrzi veliko slovo.}#
\end{upotreba}
\end{minitest}
\begin{minitest}
\begin{upotreba}{3}
#\naslovInt#
#\izlaz{Unesite nisku:}#
#\ulaz{21.06.2017.}#
#\izlaz{Ne sadrzi veliko slovo.}#
\end{upotreba}
\end{minitest}

\linkresenje{NIS_05}
\end{Exercise}
\ifresenja
\begin{Answer}[ref=NIS_05]
%\includecode{resenja/3_PredstavljanjePodataka/2.3_Niske/sve/niske05.c}
Pogledajte zadatak \ref{NIS_01}.
\end{Answer}
\fi


\begin{Exercise}[label=NIS_06] 
Napisati program koji za učitanu nisku $s$ i karakter $c$ ispituje da li se karakter $c$ pojavljuje u niski $s$. 
Ako je to slučaj, program treba da ispiše indeks prvog pojavljivanja karaktera $c$ u niski $s$, a u suprotnom -1. 
Pretpostaviti da niska može da ima najviše 20 karaktera.

\begin{minitest}
\begin{upotreba}{1}
#\naslovInt#
#\izlaz{Unesite nisku:}\ulaz{bazen}#
#\izlaz{Unesite karakter:}\ulaz{z}#
#\izlaz{Pozicija: 2}#
\end{upotreba}
\end{minitest}
\begin{minitest}
\begin{upotreba}{2}
#\naslovInt#
#\izlaz{Unesite nisku:}\ulaz{lezaljka}#
#\izlaz{Unesite karakter:}\ulaz{a}#
#\izlaz{Pozicija: 3}#
\end{upotreba}
\end{minitest}
\begin{minitest}
\begin{upotreba}{3}
#\naslovInt#
#\izlaz{Unesite nisku:}\ulaz{limunada}#
#\izlaz{Unesite karakter:}\ulaz{b}#
#\izlaz{Pozicija: -1}#
\end{upotreba}
\end{minitest}

\linkresenje{NIS_06}
\end{Exercise}
\ifresenja
\begin{Answer}[ref=NIS_06]
\includecode{resenja/3_PredstavljanjePodataka/2.3_Niske/sve/niske06.c}
\end{Answer}
\fi


\begin{Exercise}[label=NIS_07] 
 Napisati funkciju \kckod{int podniska(char s[], char t[])} koja proverava da li je niska $t$ uzastopna podniska niske $s$. 
 Napisati program koji učitava dve niske maksimalne dužine 10 karaktera i ispisuje da li je druga niska podniska prve.
 
\begin{minitest}
\begin{upotreba}{1}
#\naslovInt#
#\izlaz{Unesite nisku s:}\ulaz{abcde}#
#\izlaz{Unesite nisku t:}\ulaz{bcd}#
#\izlaz{t je podniska niske s.}#
\end{upotreba}
\end{minitest}
\begin{minitest}
\begin{upotreba}{2}
#\naslovInt#
#\izlaz{Unesite nisku s:}\ulaz{abcde}#
#\izlaz{Unesite nisku t:}\ulaz{bCd}#
#\izlaz{t nije podniska niske s.}#
\end{upotreba}
\end{minitest}
\begin{minitest}
\begin{upotreba}{3}
#\naslovInt#
#\izlaz{Unesite nisku s:}\ulaz{abcde}#
#\izlaz{Unesite nisku t:}\ulaz{def}#
#\izlaz{t nije podniska niske s.}#
\end{upotreba}
\end{minitest}

\linkresenje{NIS_07}
\end{Exercise}
\ifresenja
\begin{Answer}[ref=NIS_07]
\includecode{resenja/3_PredstavljanjePodataka/2.3_Niske/sve/niske07.c}
\end{Answer}
\fi


%\subsection{Izbacivanje karaktera}

\begin{Exercise}[label=NIS_08] 
   Napisati funkciju \kckod{void skrati(char s[])} koja uklanja beline sa
   kraja date niske. Napisati program koji učitava liniju maksimalne dužine 100 karaktera i ispisuje
   učitanu i izmenjenu nisku između zvezdica.

\begin{miditest}
\begin{upotreba}{1}
#\naslovInt#
#\izlaz{Unesite nisku:}#
#\ulaz{rep belina }#
#\izlaz{Ucitana niska:}#
#\izlaz{*rep belina }#             #\izlaz{ *}#
#\izlaz{Izmenjena niska:}#
#\izlaz{*rep belina*}#
\end{upotreba}
\end{miditest}
\begin{miditest}
\begin{upotreba}{2}
#\naslovInt#
#\izlaz{Unesite nisku:}#
#\ulaz{tri tabulatora na kraju            }#
#\izlaz{Ucitana niska:}#
#\izlaz{*tri tabulatora na kraju}#			#\izlaz{ *}#
#\izlaz{Izmenjena niska:}#
#\izlaz{*tri tabulatora na kraju*}#
\end{upotreba}
\end{miditest}
\linkresenje{NIS_08}
\end{Exercise}
\ifresenja
\begin{Answer}[ref=NIS_08]
\includecode{resenja/3_PredstavljanjePodataka/2.3_Niske/sve/niske08.c}
\end{Answer}
\fi


\begin{Exercise}[label=NIS_09] 
Napisati funkciju \kckod{void ukloni\_slova(char s[])} koja iz niske $s$ uklanja sva mala i sva velika slova. 
Napisati program koji za učitanu nisku maksimalne dužine 20 karaktera ispisuje odgovarajuću izmenjenu nisku.

\begin{minitest}
\begin{upotreba}{1}
#\naslovInt#
#\izlaz{Unesite nisku:}\ulaz{a1b2c3def}#
#\izlaz{Rezultat: 123}#
\end{upotreba}
\end{minitest}
\begin{minitest}
\begin{upotreba}{2}
#\naslovInt#
#\izlaz{Unesite nisku:}\ulaz{1+2=3}#
#\izlaz{Rezultat: 1+2=3}#
\end{upotreba}
\end{minitest}
\begin{minitest}
\begin{upotreba}{3}
#\naslovInt#
#\izlaz{Unesite nisku:}\ulaz{malaVELIKA}#
#\izlaz{Rezultat: }#
\end{upotreba}
\end{minitest}

\linkresenje{NIS_09}
\end{Exercise}
\ifresenja
\begin{Answer}[ref=NIS_09]
\includecode{resenja/3_PredstavljanjePodataka/2.3_Niske/sve/niske09.c}
\end{Answer}
\fi


\begin{Exercise}[label=NIS_10] 
 Napisati funkciju \kckod{void ukloni(char *s)} koja iz niske uklanja
 sva slova iza kojih neposredno sledi slovo koje je u engelskoj abecedi nakon
 njih, pri čemu se veličina slova zanemaruje. Pravilo se ne primenjuje na nisku dobijenu uklanjanjem.
 Napisati program koji učitava liniju teksta koja ima najviše 100 karaktera
 i ispisuje liniju koja se dobije nakon uklanjanja pomenutih karaktera.
 
\begin{minitest}
\begin{upotreba}{1}
#\naslovInt#
#\izlaz{Unesite nisku:}#
#\ulaz{Zdravo svima!}#
#\izlaz{Izmenjena niska:}#
#\izlaz{Zrvo vma!}#
\end{upotreba}
\end{minitest}
\begin{minitest}
\begin{upotreba}{2}
#\naslovInt#
#\izlaz{Unesite nisku:}#
#\ulaz{Danas je 10 stepeni.}#
#\izlaz{Izmenjena niska:}#
#\izlaz{Dns j 10 tpni.}#
\end{upotreba}
\end{minitest}
\begin{minitest}
\begin{upotreba}{3}
#\naslovInt#
#\izlaz{Unesite nisku:}#
#\ulaz{Ima vetra, kise i hladnoce.}#
#\izlaz{Izmenjena niska:}#
#\izlaz{ma vtra, se i loe.}#
\end{upotreba}
\end{minitest}

 \linkresenje{NIS_10}
\end{Exercise}
\ifresenja
\begin{Answer}[ref=NIS_10]
\includecode{resenja/3_PredstavljanjePodataka/2.3_Niske/sve/niske10.c}
\end{Answer}
\fi

%\subsection{Kopiranje jedne niske u drugu}


\begin{Exercise}[label=NIS_11] 
   Napisati program koji učitava nisku $s$ maksimalne dužine $30$ karaktera i formira nisku $t$
   trostrukim nadovezivanjem niske $s$. 
   
\begin{minitest}
\begin{upotreba}{1}
#\naslovInt#
#\izlaz{Unesite nisku:}\ulaz{dan}#
#\izlaz{Rezultujuca niska:}#
#\izlaz{dandandan}#
\end{upotreba}
\end{minitest}
\begin{minitest}
\begin{upotreba}{2}
#\naslovInt#
#\izlaz{Unesite nisku:}\ulaz{3sesira}#
#\izlaz{Rezultujuca niska:}#
#\izlaz{3sesira3sesira3sesira}#
\end{upotreba}
\end{minitest}
\begin{minitest}
\begin{upotreba}{3}
#\naslovInt#
#\izlaz{Unesite nisku:}\ulaz{a-b=5}#
#\izlaz{Rezultujuca niska:}#
#\izlaz{a-b=5a-b=5a-b=5}#
\end{upotreba}
\end{minitest}

\linkresenje{NIS_11}
\end{Exercise}
\ifresenja
\begin{Answer}[ref=NIS_11]
\includecode{resenja/3_PredstavljanjePodataka/2.3_Niske/sve/niske11.c}
\end{Answer}
\fi


\begin{Exercise}[label=NIS_12]  Napisati program koji za unetu reč maksimalne dužine 20 karaktera i pozitivan broj $n$ 
manji od $10$, formira rezultujuću reč tako što unetu reč kopira $n$ puta pri čemu se između svaka dva kopiranja umeće crtica. 
U slučaju neispravnog unosa, ispisati odgovarajuću poruku o grešci. 

\begin{minitest}
\begin{upotreba}{1}
#\naslovInt#
#\izlaz{Unesite nisku:}\ulaz{ana}#
#\izlaz{Unesite broj n:}\ulaz{4}#
#\izlaz{Rezultujuca niska:}#
#\izlaz{ana-ana-ana-ana}#
\end{upotreba}
\end{minitest}
\begin{minitest}
\begin{upotreba}{2}
#\naslovInt#
#\izlaz{Unesite nisku:}\ulaz{123}#
#\izlaz{Unesite broj n:}\ulaz{1}#
#\izlaz{Rezultujuca niska:}#
#\izlaz{123}#
\end{upotreba}
\end{minitest}
\begin{minitest}
\begin{upotreba}{3}
#\naslovInt#
#\izlaz{Unesite nisku:}\ulaz{x*y}#
#\izlaz{Unesite broj n:}\ulaz{3}#
#\izlaz{Rezultujuca niska:}#
#\izlaz{x*y-x*y-x*y}#
\end{upotreba}
\end{minitest}

\linkresenje{NIS_12}
\end{Exercise}
\ifresenja
\begin{Answer}[ref=NIS_12]
\includecode{resenja/3_PredstavljanjePodataka/2.3_Niske/sve/niske12.c}
\end{Answer}
\fi


\begin{Exercise}[label=NIS_13] 
Napisati funkciju \kckod{void kopiraj\_n(char t[], char s[], int n)} koja kopira najviše $n$ karaktera niske $s$ u nisku $t$. 
Napisati program koji testira rad napisane funkcije. Pretpostaviti da je maksimalna dužina niske $s$ $20$ karaktera.
U slučaju neispravnog unosa, ispisati odgovarajuću poruku o grešci.

\begin{minitest}
\begin{upotreba}{1}
#\naslovInt#
#\izlaz{Unesite nisku:}\ulaz{petar}#
#\izlaz{Unesite broj n:}\ulaz{3}#
#\izlaz{Rezultujuca niska: pet}#
\end{upotreba}
\end{minitest}
\begin{minitest}
\begin{upotreba}{2}
#\naslovInt#
#\izlaz{Unesite nisku:}\ulaz{gromobran}#
#\izlaz{Unesite broj n:}\ulaz{4}#
#\izlaz{Rezultujuca niska: grom}#
\end{upotreba}
\end{minitest}
\begin{minitest}
\begin{upotreba}{3}
#\naslovInt#
#\izlaz{Unesite nisku:}\ulaz{abc}#
#\izlaz{Unesite broj n:}\ulaz{15}#
#\izlaz{Rezultujuca niska: abc}#
\end{upotreba}
\end{minitest}

\linkresenje{NIS_13}
\end{Exercise}
\ifresenja
\begin{Answer}[ref=NIS_13]
\includecode{resenja/3_PredstavljanjePodataka/2.3_Niske/sve/niske13.c}
\end{Answer}
\fi


\begin{Exercise}[label=NIS_14] 
 Napisati funkciju \kckod{void dupliranje(char t[], char s[])} koja na osnovu niske $s$ formira 
 nisku $t$ tako što duplira svaki karakter niske $s$. 
 Napisati program koji učitava nisku maksimalne dužine $20$ karaktera i ispisuje nisku koja se
 dobije nakon dupliranja karaktera.

\begin{minitest}
\begin{upotreba}{1}
#\naslovInt#
#\izlaz{Unesite nisku:}\ulaz{zima}#
#\izlaz{Rezultujuca niska: zziimmaa}#
\end{upotreba}
\end{minitest}
\begin{minitest}
\begin{upotreba}{2}
#\naslovInt#
#\izlaz{Unesite nisku:}\ulaz{C++}#
#\izlaz{Rezultujuca niska: CC++++}#
\end{upotreba}
\end{minitest}
\begin{minitest}
\begin{upotreba}{3}
#\naslovInt#
#\izlaz{Unesite nisku:}\ulaz{C}#
#\izlaz{Rezultujuca niska: CC}#
\end{upotreba}
\end{minitest}
 
\linkresenje{NIS_14}
\end{Exercise}
\ifresenja
\begin{Answer}[ref=NIS_14]
\includecode{resenja/3_PredstavljanjePodataka/2.3_Niske/sve/niske14.c}
\end{Answer}
\fi

%\subsection{Numericki}

\begin{Exercise}[label=NIS_15] 
Napisati program koji učitava nisku cifara sa eventualnim vodećim znakom i pretvara je u ceo broj.
\napomena{Pretpostaviti da je unos ispravan.}

\begin{minitest}
\begin{upotreba}{1}
#\naslovInt#
#\izlaz{Unesite nisku:}\ulaz{-1238}#
#\izlaz{Rezultat: -1238}#
\end{upotreba}
\end{minitest}
\begin{minitest}
\begin{upotreba}{2}
#\naslovInt#
#\izlaz{Unesite nisku:}\ulaz{73}#
#\izlaz{Rezultat: 73}#
\end{upotreba}
\end{minitest}
\begin{minitest}
\begin{upotreba}{3}
#\naslovInt#
#\izlaz{Unesite nisku:}\ulaz{+1}#
#\izlaz{Rezultat: 1}#
\end{upotreba}
\end{minitest}

\linkresenje{NIS_15}
\end{Exercise}
\ifresenja
\begin{Answer}[ref=NIS_15]
\includecode{resenja/3_PredstavljanjePodataka/2.3_Niske/sve/niske15.c}
\end{Answer}
\fi


\begin{Exercise}[label=NIS_16] 
Napisati program koji učitava ceo broj, pretvara ga u nisku i ispisuje dobijenu nisku. 
	
\begin{minitest}
\begin{upotreba}{1}
#\naslovInt#
#\izlaz{Unesite ceo broj:}\ulaz{-6543}#
#\izlaz{Rezultat: -6543}#
\end{upotreba}
\end{minitest}
\begin{minitest}
\begin{upotreba}{2}
#\naslovInt#
#\izlaz{Unesite ceo broj:}\ulaz{84}#
#\izlaz{Rezultat: 84}#
\end{upotreba}
\end{minitest}
\begin{minitest}
\begin{upotreba}{3}
#\naslovInt#
#\izlaz{Unesite ceo broj:}\ulaz{5}#
#\izlaz{Rezultat: 5}#
\end{upotreba}
\end{minitest}

\linkresenje{NIS_16}
\end{Exercise}
\ifresenja
\begin{Answer}[ref=NIS_16]
\includecode{resenja/3_PredstavljanjePodataka/2.3_Niske/sve/niske16.c}
\end{Answer}
\fi


\begin{Exercise}[label=NIS_17] 
Napisati funkciju \kckod{int heksadekadni\_broj(char s[])} koja proverava da li je niskom $s$ zadat 
korektan heksadekadni broj. Funkcija treba da vrati vrednost 1 ukoliko je uslov ispunjen, odnosno 0 ako nije. 
Napisati program koji za učitanu nisku maksimalne dužine 7 karaktera ispisuje da li je korektan heksadekadni broj. 
\uputstvo{Heksadekadni broj je korektno zadat ako počinje prefiksom $0x$ ili $0X$ i ako sadrži samo cifre 
i mala ili velika slova $A$, $B$, $C$, $D$, $E$ i $F$.} 
  
\begin{minitest}
\begin{upotreba}{1}
#\naslovInt#
#\izlaz{Unesite nisku:}\ulaz{0x12EF}#
#\izlaz{Korektan heksadekadni broj.}#
\end{upotreba}
\end{minitest}
\begin{minitest}
\begin{upotreba}{2}
#\naslovInt#
#\izlaz{Unesite nisku:}\ulaz{0X22af}#
#\izlaz{Korektan heksadekadni broj.}#
\end{upotreba}
\end{minitest}
\begin{minitest}
\begin{upotreba}{3}
#\naslovInt#
#\izlaz{Unesite nisku:}\ulaz{0xErA9}#
#\izlaz{Nekorektan heksadekadni broj.}#
\end{upotreba}
\end{minitest}

\linkresenje{NIS_17}
\end{Exercise}
\ifresenja
\begin{Answer}[ref=NIS_17]
\includecode{resenja/3_PredstavljanjePodataka/2.3_Niske/sve/niske17.c}
\end{Answer}
\fi


\begin{Exercise}[label=NIS_18] 
 Napisati funkciju \kckod{int dekadna\_vrednost(char s[])} koja izračunava dekadnu 
 vrednost heksadekadnog broja zadatog niskom $s$. Napisati program koji za učitanu 
 nisku maksimalne dužine 7 karaktera ispisuje odgovarajuću dekadnu vrednost. 
 Pretpostaviti da je uneta niska korektan heksadekadni broj.

\begin{minitest}
\begin{upotreba}{1}
#\naslovInt#
#\izlaz{Unesite nisku:}\ulaz{0x2A34}#
#\izlaz{Rezultat: 10804}#
\end{upotreba}
\end{minitest}
\begin{minitest}
\begin{upotreba}{2}
#\naslovInt#
#\izlaz{Unesite nisku:}\ulaz{0Xff2}#
#\izlaz{Rezultat: 4082}#
\end{upotreba}
\end{minitest}
\begin{minitest}
\begin{upotreba}{3}
#\naslovInt#
#\izlaz{Unesite nisku:}\ulaz{0xE1A9}#
#\izlaz{Rezultat: 57769}#
\end{upotreba}
\end{minitest}

\linkresenje{NIS_18}
\end{Exercise}
\ifresenja
\begin{Answer}[ref=NIS_18]
\includecode{resenja/3_PredstavljanjePodataka/2.3_Niske/sve/niske18.c}
\end{Answer}
\fi


%\subsection{Linije sa min/max}

\begin{Exercise}[label=NIS_19] 
Napisati funkciju \kckod{int ucitaj\_liniju(char s[], int n)}
koja učitava liniju maksimalne dužine $n$ u nisku $s$
i vraća dužinu učitane linije.  
Napisati program koji učitava linije
do kraja ulaza i ispisuje najdužu liniju i njenu dužinu. Ukoliko
ima više linija maksimalne dužine, ispisati prvu. 
Pretpostaviti da svaka linija sadrži najviše 80 karaktera.
\napomena{Linija može da sadrži blanko znakove, ali ne može sadržati znak za novi red ili \kckod{EOF}.}

\begin{minitest}
\begin{upotreba}{1}
#\naslovInt#
#\izlaz{Unesite tekst:}#
#\ulaz{Dobar dan!}#
#\ulaz{Kako ste, sta ima novo?}#
#\ulaz{Ja sam dobro.}#
#\izlaz{Najduza linija:}#
#\izlaz{Kako ste, sta ima novo?}#
#\izlaz{Duzina: 23}#
\end{upotreba}
\end{minitest}
\begin{minitest}
\begin{upotreba}{2}
#\naslovInt#
#\izlaz{Unesite tekst:}#
#\ulaz{Prva linija}#
#\ulaz{Druga linija}#
#\ulaz{Treca linija}#
#\izlaz{Najduza linija:}#
#\izlaz{Druga linija}#
#\izlaz{Duzina: 12}#
\end{upotreba}
\end{minitest}
\begin{minitest}
\begin{upotreba}{3}
#\naslovInt#
#\izlaz{Unesite tekst:}#
#\ulaz{Danas je lep dan.}#
#\izlaz{Najduza linija:}#
#\izlaz{Danas je lep dan.}#
#\izlaz{Duzina: 17}#
\end{upotreba}
\end{minitest}

\linkresenje{NIS_19}
\end{Exercise}
\ifresenja
\begin{Answer}[ref=NIS_19]
\includecode{resenja/3_PredstavljanjePodataka/2.3_Niske/sve/niske19.c}
\end{Answer}
\fi


\begin{Exercise}[difficulty=1, label=NIS_20] 
Napisati funkcije za rad sa rečenicama:
\begin{enumerate}
\setlength\itemsep{0em}
\item \kckod{int procitaj\_recenicu(char s[], int n)} koja učitava rečenicu sa ulaza i smešta je u nisku $s$. 
Funkcija vraća dužinu učitane rečenice. Učitavanje se završava nakon učitanog karaktera \kckod{.}, nakon $n$ učitanih karaktera ili ako
se dođe do kraja ulaza.
\item \kckod{void prebroj(char s[], int *broj\_malih, int *broj\_velikih)} koja prebrojava mala i velika slova u niski $s$.
\end{enumerate}
 Napisati program koji učitava rečenice do kraja ulaza i ispisuje onu rečenicu kod koje je apsolutna razlika broja malih i velikih slova najveća.
 Pri učitavanju rečenica zanemariti sve beline koje se nalaze između dve rečenice. 
Pretpostaviti da jedna rečenica sadrži najviše 80 karaktera.

\begin{maxitest}
\begin{upotreba}{1}
#\naslovInt#
#\izlaz{Unesite tekst:}#
#\ulaz{U ovom poglavlju se govori o niskama.    Niske su nizovi karaktera ciji je }#
#\ulaz{poslednji element terminalna nula.}#
#\ulaz{U ovom zadatku je potrebno ucitati recenice. Svaka recenica pocinje sa bilo }#
#\ulaz{kojim karakterom koji nije belina. Na kraju recenice se nalazi tacka. }#
#\izlaz{Rezultujuca recenica:}#
#\izlaz{Niske su nizovi karaktera ciji je poslednji element terminalna nula.}#
\end{upotreba}
\end{maxitest}

 \linkresenje{NIS_20}
\end{Exercise}
\ifresenja
\begin{Answer}[ref=NIS_20]
\includecode{resenja/3_PredstavljanjePodataka/2.3_Niske/sve/niske20.c}
\end{Answer}
\fi

%\subsection{Klonovi}


\begin{Exercise}[label=NIS_21] 
Napisati funkciju \kckod{char* strchr\_klon(char s[], char c)} koja vraća pokazivač na prvo pojavljivanje 
karaktera $c$ u niski $s$ ili $NULL$ ukoliko se karakter $c$ ne pojavljuje u niski $s$.
\footnote{Funkcija $strchr\_klon$ odgovara funkciji $strchr$ čija se deklaracija nalazi u zaglavlju string.h.
Slično važi i za ostale \textit{klon} funkcije iz narednih zadataka.} 
Napisati program koji za učitanu nisku maksimalne dužine 20 karaktera i karakter $c$ ispisuje
indeks prvog pojavljivanja karaktera c u okviru učitane niske ili $-1$ ukoliko učitana niska ne
sadrži uneti karakter.

\begin{miditest}
\begin{upotreba}{1}
#\naslovInt#
#\izlaz{Unesite nisku s:}\ulaz{programiranje}#
#\izlaz{Unesite karakter c:}\ulaz{a}#
#\izlaz{Pozicija: 5}#
\end{upotreba}
\end{miditest}
\begin{miditest}
\begin{upotreba}{2}
#\naslovInt#
#\izlaz{Unesite nisku s:}\ulaz{123456789}#
#\izlaz{Unesite karakter c:}\ulaz{y}#
#\izlaz{Pozicija: -1}#
\end{upotreba}
\end{miditest}

\begin{miditest}
\begin{upotreba}{3}
#\naslovInt#
#\izlaz{Unesite nisku s:}\ulaz{leto2017}#
#\izlaz{Unesite karakter c:}\ulaz{0}#
#\izlaz{Pozicija: 5}#
\end{upotreba}
\end{miditest}
\begin{miditest}
\begin{upotreba}{4}
#\naslovInt#
#\izlaz{Unesite nisku s:}\ulaz{jedrilica}#
#\izlaz{Unesite karakter c:}\ulaz{I}#
#\izlaz{Pozicija: -1}#
\end{upotreba}
\end{miditest}

\linkresenje{NIS_21}
\end{Exercise}
\ifresenja
\begin{Answer}[ref=NIS_21]
\includecode{resenja/3_PredstavljanjePodataka/2.3_Niske/sve/niske21.c}
\end{Answer}
\fi


\begin{Exercise}[label=NIS_22] 
 Napisati funkciju \kckod{int strspn\_klon(char t[], char s[])} koja izračunava dužinu prefiksa niske $t$ sastavljenog od karaktera niske $s$. 
 Napisati program koji za učitane dve niske maksimalne dužine 20 karaktera ispisuje rezultat poziva napisane funkcije.  

\begin{miditest}
\begin{upotreba}{1}
#\naslovInt#
#\izlaz{Unesite nisku t:}\ulaz{program}#
#\izlaz{Unesite nisku s:}\ulaz{pero}#
#\izlaz{Rezultat: 3}#
\end{upotreba}
\end{miditest}
\begin{miditest}
\begin{upotreba}{2}
#\naslovInt#
#\izlaz{Unesite nisku t:}\ulaz{Barselona}#
#\izlaz{Unesite nisku s:}\ulaz{Brazil}#
#\izlaz{Rezultat: 3}#
\end{upotreba}
\end{miditest}

\begin{miditest}
\begin{upotreba}{3}
#\naslovInt#
#\izlaz{Unesite nisku t:}\ulaz{24.10.2017.}#
#\izlaz{Unesite nisku s:}\ulaz{0123456789}#
#\izlaz{Rezultat: 2}#
\end{upotreba}
\end{miditest}
\begin{miditest}
\begin{upotreba}{4}
#\naslovInt#
#\izlaz{Unesite nisku t:}\ulaz{12345}#
#\izlaz{Unesite nisku s:}\ulaz{9876543210}#
#\izlaz{Rezultat: 5}#
\end{upotreba}
\end{miditest}

\linkresenje{NIS_22}
\end{Exercise}
\ifresenja
\begin{Answer}[ref=NIS_22]
\includecode{resenja/3_PredstavljanjePodataka/2.3_Niske/sve/niske22.c}
\end{Answer}
\fi


\begin{Exercise}[label=NIS_23] 
Napisati funkciju \kckod{int strcspn\_klon(char t[], char s[])}
koja izračunava dužinu prefiksa niske $t$ sastavljenog
isključivo od karaktera koji se ne nalaze u niski $s$. 
Napisati program koji testira ovu funkciju za dve unete niske maksimalne dužine 100 karaktera. 

\begin{minitest}
\begin{upotreba}{1}
#\naslovInt#
#\izlaz{Unesite nisku t:}#
#\ulaz{programiranje}#
#\izlaz{Unesite nisku s:}#
#\ulaz{pero}#
#\izlaz{Rezultat: 0}#
\end{upotreba}
\end{minitest}
\begin{minitest}
\begin{upotreba}{2}
#\naslovInt#
#\izlaz{Unesite nisku t:}#
#\ulaz{programiranje}#
#\izlaz{Unesite nisku s:}#
#\ulaz{analiza}#
#\izlaz{Rezultat: 5}#
\end{upotreba}
\end{minitest}
\begin{minitest}
\begin{upotreba}{3}
#\naslovInt#
#\izlaz{Unesite nisku t:}#
#\ulaz{programiranje}#
#\izlaz{Unesite nisku s:}#
#\ulaz{1.10.}#
#\izlaz{Rezultat: 13}#
\end{upotreba}
\end{minitest}
 \linkresenje{NIS_23}
\end{Exercise}
\ifresenja
\begin{Answer}[ref=NIS_23]
Rešenje ovog zadatka se svodi na rešenje zadatka \ref{NIS_22}, uz razliku da se
ovde prebrojavaju karakteri koji se ne nalaze u zapisu niske \kckod{s}. 
\end{Answer}
\fi


\begin{Exercise}[label=NIS_24] 
Napisati funkciju \kckod{char* strstr\_klon(char s[], char t[])} koja vraća pokazivač na prvo
pojavljivanje niske $t$ u niski $s$ ili $NULL$ ukoliko se niska $t$ ne pojavljuje u niski $s$. 
Napisati program koji testira napisanu funkciju tako što učitava pet linija i ispisuje redne brojeve svih 
linija koje sadr\v ze nisku $program$. Ukoliko ne postoji linija sa niskom $program$, 
ispisati odgovarajuću poruku. 
Pretpostaviti da je svaka linija maksimalne dužine 100 karaktera kao i da se linije numerišu od broja 1. 

\begin{minitest}
\begin{upotreba}{1}
#\naslovInt#
#\izlaz{Unesite pet linija:}#
#\ulaz{tv program}#
#\ulaz{c prog. jezik}#
#\ulaz{c++ programskih jezik}#
#\ulaz{Programski odbor}#
#\ulaz{<b>program</b>}#
#\izlaz{Rezultat: 1 3 5}#
\end{upotreba}
\end{minitest}
\begin{minitest}
\begin{upotreba}{2}
#\naslovInt#
#\izlaz{Unesite pet linija:}#
#\ulaz{Programske paradigme}#
#\ulaz{su predmet na}#
#\ulaz{trecoj godini}#
#\ulaz{programerskih}#
#\ulaz{smerova.}#
#\izlaz{Rezultat: 4}#
\end{upotreba}
\end{minitest}
\begin{minitest}
\begin{upotreba}{3}
#\naslovInt#
#\izlaz{Unesite pet linija:}#
#\ulaz{U narednim}#
#\ulaz{linijama}#
#\ulaz{necemo navoditi}#
#\ulaz{nisku koja se}#
#\ulaz{trazi.}#
#\izlaz{Nijedna linija ne sadrzi}#
#\izlaz{nisku program.}#
\end{upotreba}
\end{minitest}

\linkresenje{NIS_24}
\end{Exercise}
\ifresenja
\begin{Answer}[ref=NIS_24]
\includecode{resenja/3_PredstavljanjePodataka/2.3_Niske/sve/niske24.c}
\end{Answer}
\fi


\begin{Exercise}[label=NIS_25] 
Napisati funkciju \kckod{int strcmp\_klon(char s[], char t[])} koja vraća 0
ako su niske s i t jednake, neku pozitivnu vrednost ako je $s$ leksikografski iza $t$,
a neku negativnu vrednost inače. 
Napisati program koji učitava dve niske maksimalne dužine 20 karaktera i ako su različite, ispisuje
učitane niske u rastućem leksikografskom poretku, a ako su jednake, ispisuje samo jednu nisku.

\begin{minitest}
\begin{upotreba}{1}
#\naslovInt#
#\izlaz{Unesite nisku s:}\ulaz{Beograd}#
#\izlaz{Unesite nisku t:}\ulaz{Amsterdam}#
#\izlaz{Rezultat:}#
#\izlaz{Amsterdam}#
#\izlaz{Beograd}#
\end{upotreba}
\end{minitest}
\begin{minitest}
\begin{upotreba}{2}
#\naslovInt#
#\izlaz{Unesite nisku s:}\ulaz{Beograd}#
#\izlaz{Unesite nisku t:}\ulaz{Beograd}#
#\izlaz{Rezultat:}#
#\izlaz{Beograd}#
\end{upotreba}
\end{minitest}
\begin{minitest}
\begin{upotreba}{3}
#\naslovInt#
#\izlaz{Unesite nisku s:}\ulaz{radnik}#
#\izlaz{Unesite nisku t:}\ulaz{radnica}#
#\izlaz{Rezultat:}#
#\izlaz{radnica}#
#\izlaz{radnik}#
\end{upotreba}
\end{minitest}

\linkresenje{NIS_25}
\end{Exercise}
\ifresenja
\begin{Answer}[ref=NIS_25]
\includecode{resenja/3_PredstavljanjePodataka/2.3_Niske/sve/niske25.c}
\end{Answer}
\fi

%\subsection{Rotacije}

\begin{Exercise}[label=NIS_26] 
Napisati funkciju \kckod{void obrni(char s[])} koja obrće
nisku $s$. Napisati program koji obrće učitanu nisku maksimalne 
dužine 20 karaktera i ispisuje obrnutu nisku.

\begin{minitest}
\begin{upotreba}{1}
#\naslovInt#
#\izlaz{Unesite nisku:}\ulaz{kisobran}#
#\izlaz{Rezultat: narbosik}#
\end{upotreba}
\end{minitest}
\begin{minitest}
\begin{upotreba}{2}
#\naslovInt#
#\izlaz{Unesite nisku:}\ulaz{Aleksandar}#
#\izlaz{Rezultat: radnaskelA}#
\end{upotreba}
\end{minitest}
\begin{minitest}
\begin{upotreba}{3}
#\naslovInt#
#\izlaz{Unesite nisku:}\ulaz{kajak}#
#\izlaz{Rezultat: kajak}#
\end{upotreba}
\end{minitest}

\linkresenje{NIS_26}
\end{Exercise}
\ifresenja
\begin{Answer}[ref=NIS_26]
\includecode{resenja/3_PredstavljanjePodataka/2.3_Niske/sve/niske26.c}
\end{Answer}
\fi


\begin{Exercise}[label=NIS_27] 
Napisati funkciju \kckod{void rotiraj(char s[], int k)} koja rotira
nisku $s$ za $k$ mesta ulevo. Napisati program koji učitava nisku maksimalne dužine 20 karaktera
i nenegativan ceo broj k i ispisuje rotiranu nisku. 
U slučaju neispravnog unosa, ispisati odgovarajuću poruku o grešci. 

\begin{minitest}
\begin{upotreba}{1}
#\naslovInt#
#\izlaz{Unesite nisku i broj k:}#
#\ulaz{sveska 2}#
#\izlaz{Rezultat: eskasv}#
\end{upotreba}
\end{minitest}
\begin{minitest}
\begin{upotreba}{2}
#\naslovInt#
#\izlaz{Unesite nisku i broj k:}#
#\ulaz{olovka 6}#
#\izlaz{Rezultat: olovka}#
\end{upotreba}
\end{minitest}
\begin{minitest}
\begin{upotreba}{3}
#\naslovInt#
#\izlaz{Unesite nisku i broj k:}#
#\ulaz{rezac 8}#
#\izlaz{Rezultat: acrez}#
\end{upotreba}
\end{minitest}

\linkresenje{NIS_27}
\end{Exercise}
\ifresenja
\begin{Answer}[ref=NIS_27]
\includecode{resenja/3_PredstavljanjePodataka/2.3_Niske/sve/niske27.c}
\end{Answer}
\fi

%\subsection{Sifriranja}

\begin{Exercise}[label=NIS_28] 
Napisati program koji šifuje unetu nisku tako što svako slovo zamenjuje 
sledećim slovom engelske abecede (slova ’z' i 'Z' zamenjuje, redom, sa 'a' i ’A’),
a ostale karaktere ostavlja nepromenjene. Ispisati nisku dobijenu na ovaj nacin. 
Pretpostaviti da uneta niska nije duža od 20 karaktera.

\begin{minitest}
\begin{upotreba}{1}
#\naslovInt#
#\izlaz{Unesite nisku:}\ulaz{bundeva}#
#\izlaz{Rezultat: cvoefwb}#
\end{upotreba}
\end{minitest}
\begin{minitest}
\begin{upotreba}{2}
#\naslovInt#
#\izlaz{Unesite nisku:}\ulaz{zimzelen}#
#\izlaz{Rezultat: ajnafmfo}#
\end{upotreba}
\end{minitest}
\begin{minitest}
\begin{upotreba}{3}
#\naslovInt#
#\izlaz{Unesite nisku:}\ulaz{Oktobar17}#
#\izlaz{Rezultat: Plupcbs17}#
\end{upotreba}
\end{minitest}

\linkresenje{NIS_28}
\end{Exercise}
\ifresenja
\begin{Answer}[ref=NIS_28]
\includecode{resenja/3_PredstavljanjePodataka/2.3_Niske/sve/niske28.c}
\end{Answer}
\fi


\begin{Exercise}[label=NIS_29] 
Napisati funkciju \kckod{void sifruj(char rec[], char sifra[])} koja na osnovu date 
reči formira šifru tako što se svako slovo u reči zameni sa naredna tri slova engleske abecede. 
Napisati program koji testira napisanu funkciju za reč maksimalne dužine 20 karaktera. 

\begin{minitest}
\begin{upotreba}{1}
#\naslovInt#
#\izlaz{Unesite nisku:}\ulaz{tamo}#
#\izlaz{Rezultat: uvwbcdnoppqr}#
\end{upotreba}
\end{minitest}
\begin{minitest}
\begin{upotreba}{2}
#\naslovInt#
#\izlaz{Unesite nisku:}\ulaz{Zec}#
#\izlaz{Rezultat: ABCfghdef}#
\end{upotreba}
\end{minitest}
\begin{minitest}
\begin{upotreba}{3}
#\naslovInt#
#\izlaz{Unesite nisku:}\ulaz{a+b=c}#
#\izlaz{Rezultat: bcd+cde=def}#
\end{upotreba}
\end{minitest}

\linkresenje{NIS_29}
\end{Exercise}
\ifresenja
\begin{Answer}[ref=NIS_29]
\includecode{resenja/3_PredstavljanjePodataka/2.3_Niske/sve/niske29.c}
\end{Answer}
\fi


\begin{Exercise}[label=NIS_30] 
Napisati funkciju \kckod{void formiraj(char s1[], char s2[], char c1, char c2)} koja na osnovu 
niske $s_1$ formira nisku $s_2$ udvajanjem svih karaktera $c_1$ u niski $s_1$ i  
izbacivanjem svih karaktera $c_2$ iz niske $s_1$, dok ostali karakteri ostaju nepromenjeni. 
Napisati program koji testira ovu funkciju za unetu nisku i dva uneta karaktera. 
Pretpostaviti da uneta niska nije duža od 20 karaktera.

\begin{minitest}
\begin{upotreba}{1}
#\naslovInt#
#\izlaz{Unesite nisku:}\ulaz{flomaster}#
#\izlaz{Unesite prvi karakter:}\ulaz{s}#
#\izlaz{Unesite drugi karakter:}\ulaz{m}#
#\ulaz{Rezultat: floasster}#
\end{upotreba}
\end{minitest}
\begin{minitest}
\begin{upotreba}{2}
#\naslovInt#
#\izlaz{Unesite nisku:}\ulaz{bojica}#
#\izlaz{Unesite prvi karakter:}\ulaz{b}#
#\izlaz{Unesite drugi karakter:}\ulaz{a}#
#\ulaz{Rezultat: bbojic}#
\end{upotreba}
\end{minitest}
\begin{minitest}
\begin{upotreba}{3}
#\naslovInt#
#\izlaz{Unesite nisku:}\ulaz{patentara}#
#\izlaz{Unesite prvi karakter:}\ulaz{t}#
#\izlaz{Unesite drugi karakter:}\ulaz{a}#
#\ulaz{Rezultat: pttenttr}#
\end{upotreba}
\end{minitest}


\linkresenje{NIS_30}
\end{Exercise}
\ifresenja
\begin{Answer}[ref=NIS_30]
\includecode{resenja/3_PredstavljanjePodataka/2.3_Niske/sve/niske30.c}
\end{Answer}
\fi

%\subsection{Teski}

\begin{Exercise}[difficulty=1, label=NIS_31] 
Napisati program za rad sa brojevima zapisanim u različitim brojevnim sistemima.
\begin{enumerate}
\setlength\itemsep{0em}
\item Napisati funkciju \kckod{unsigned int u\_dekadni\_sistem(char broj[], unsigned int osnova)} koja
određuje dekadnu vrednost zapisa datog neoznačenog broja $broj$ u datoj osnovi. 
\item Napisati funkciju \kckod{void iz\_dekadnog\_sistema(unsigned int broj, unsigned int osnova, char rezultat[])} 
koja datu dekadnu vrednost $broj$ zapisuje u datoj osnovi $osnovi$ i smešta rezultat u nisku $rezultat$. 
Pretpostaviti da je $0 < osnova \leq 16$.  
\end{enumerate}

Napisati program koji učitava broj $n$ i osnove $o_1$ i $o_2$ i ispisuje dekadnu vrednost broja $n$ u osnovi $o_1$, kao i
vrednost koja se dobije kada se ta dekadna vrednost zapiše u osnovi $o_2$.
Pretpostaviti da je ulaz ispravan i da će svi brojevi biti u opsegu tipa \kckod{unsigned}. 

\begin{miditest}
\begin{upotreba}{1}
#\naslovInt#
#\izlaz{Unesite n, o1 i o2:}\ulaz{10101011 2 16}#
#\izlaz{Dekadna vrednost broja 10101011: 171}#
#\izlaz{Vrednost broja 171 u osnovi 16: AB}#
\end{upotreba}
\end{miditest}
\begin{miditest}
\begin{upotreba}{2}
#\naslovInt#
#\izlaz{Unesite n, o1 i o2:}\ulaz{1067 8 3}#
#\izlaz{Dekadna vrednost broja 1067: 567}#
#\izlaz{Vrednost broja 567 u osnovi 3: 210000}#
\end{upotreba}
\end{miditest}

\begin{miditest}
\begin{upotreba}{3}
#\naslovInt#
#\izlaz{Unesite n, o1 i o2:}\ulaz{1010111001010 2 3}#
#\izlaz{Dekadna vrednost broja 1010111001010: 5578}#
#\izlaz{Vrednost broja 5578 u osnovi 3: 21122121}#
\end{upotreba}
\end{miditest}
\begin{miditest}
\begin{upotreba}{4}
#\naslovInt#
#\izlaz{Unesite n, o1 i o2:}\ulaz{111 3 5}#
#\izlaz{Dekadna vrednost broja 111: 13}#
#\izlaz{Vrednost broja 13 u osnovi 5: 23}#
\end{upotreba}
\end{miditest}
\linkresenje{NIS_31}
\end{Exercise}
\ifresenja
\begin{Answer}[ref=NIS_31]
\includecode{resenja/3_PredstavljanjePodataka/2.3_Niske/sve/niske31.c}
\end{Answer}
\fi

\ifresenja
\section{Rešenja}
\shipoutAnswer
\fi

