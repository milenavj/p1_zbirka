
\section{Niske}



\begin{Exercise}[label=v2.3_01] 
Tekst
\linkresenje{v2.3_01}
\end{Exercise}
\begin{Answer}[ref=v2.3_01]
\includecode{resenja/2_PredstavljanjePodataka/2.3_Niske/1_01.c}
\end{Answer}

\begin{Exercise}[label=v2.3_02] 
Tekst
\linkresenje{v2.3_02}
\end{Exercise}
\begin{Answer}[ref=v2.3_02]
\includecode{resenja/2_PredstavljanjePodataka/2.3_Niske/1_02.c}
\end{Answer}

\begin{Exercise}[label=v2.3_03] 
Tekst
\linkresenje{v2.3_03}
\end{Exercise}
\begin{Answer}[ref=v2.3_03]
\includecode{resenja/2_PredstavljanjePodataka/2.3_Niske/1_03.c}
\end{Answer}

\begin{Exercise}[label=v2.3_04] 
Tekst
\linkresenje{v2.3_04}
\end{Exercise}
\begin{Answer}[ref=v2.3_04]
\includecode{resenja/2_PredstavljanjePodataka/2.3_Niske/1_04.c}
\end{Answer}

\begin{Exercise}[label=v2.3_05] 
Tekst
\linkresenje{v2.3_05}
\end{Exercise}
\begin{Answer}[ref=v2.3_05]
\includecode{resenja/2_PredstavljanjePodataka/2.3_Niske/1_05.c}
\end{Answer}

\begin{Exercise}[label=v2.3_06] 
Tekst
\linkresenje{v2.3_06}
\end{Exercise}
\begin{Answer}[ref=v2.3_06]
\includecode{resenja/2_PredstavljanjePodataka/2.3_Niske/1_06.c}
\end{Answer}

\begin{Exercise}[label=v2.3_07] 
Tekst
\linkresenje{v2.3_07}
\end{Exercise}
\begin{Answer}[ref=v2.3_07]
\includecode{resenja/2_PredstavljanjePodataka/2.3_Niske/1_07.c}
\end{Answer}

\begin{Exercise}[label=v2.3_08] 
Tekst
\linkresenje{v2.3_08}
\end{Exercise}
\begin{Answer}[ref=v2.3_08]
\includecode{resenja/2_PredstavljanjePodataka/2.3_Niske/1_08.c}
\end{Answer}

\begin{Exercise}[label=p2.3_] 
\begin{itemize}
\item [a)] Napisati funkciju $int\ samoglasnik(char\ c)$ koja proverava da li je zadati karakter samoglasnik. Funkcija treba da vrati vrednost 1 ako karakter $c$ jeste samoglasnik, odnosno 0 ako nije. 
\item [b)] Napisati funkciju $int\ samoglasnik\_na\_kraju(char\ s[])$ koja proverava da li se niska $s$ završava samoglasnikom (koristiti funkciju iz tačke a)). 
\item [c)] Napisati program koji učitava nisku maksimalne dužine 20 karaktera i ispisuje da li završava samoglasnikom ili ne. 
\end{itemize}
\begin{miditest}
\begin{upotreba}{1}
#\naslovInt#
#\izlaz{Unesite nisku:}\ulaz{abcde}#
#\izlaz{Niska se zavrsava samoglasnikom!}#
\end{upotreba}
\end{miditest}
\begin{miditest}
\begin{upotreba}{2}
#\naslovInt#
#\izlaz{Unesite nisku:}\ulaz{AaBb+cCdD}#
#\izlaz{Niska se ne zavrsava samoglasnikom!}#
\end{upotreba}
\end{miditest}
\begin{miditest}
\begin{upotreba}{3}
#\naslovInt#
#\izlaz{Unesite nisku:}\ulaz{pRograMiranjE}#
#\izlaz{Niska se zavrsava samoglasnikom!}#
\end{upotreba}
\end{miditest}
\linkresenje{p2.3_}
\end{Exercise}
\begin{Answer}[ref=p2.3_]
%\includecode{resenja/2_PredstavljanjePodataka/2.3_Niske/1_08.c}
\end{Answer}

\begin{Exercise}[label=p2.3_] 
Napisati funkciju $void\ kopiraj\_n(char\ t[],\ char\ s[],\ int\ n)$ koja kopira najviše $n$ karaktera niske $s$ u nisku $t$. Napisati i program koji učitava nisku maksimalne dužine 20 karaktera i jedan ceo broj i testira rad napisane funkcije.\\
\begin{miditest}
\begin{upotreba}{1}
#\naslovInt#
#\izlaz{Unesite nisku:}\ulaz{abcdef}#
#\izlaz{Unesite broj n:}\ulaz{3}#
#\izlaz{Rezultujuca niska: abc}#
\end{upotreba}
\end{miditest}
\begin{miditest}
\begin{upotreba}{2}
#\naslovInt#
#\izlaz{Unesite nisku:}\ulaz{programiranje}#
#\izlaz{Unesite broj n:}\ulaz{5}#
#\izlaz{Rezultujuca niska: progr}#
\end{upotreba}
\end{miditest}
\begin{miditest}
\begin{upotreba}{3}
#\naslovInt#
#\izlaz{Unesite nisku:}\ulaz{abc}#
#\izlaz{Unesite broj n:}\ulaz{15}#
#\izlaz{Rezultujuca niska: abc}#
\end{upotreba}
\end{miditest}


\linkresenje{p2.3_}
\end{Exercise}
\begin{Answer}[ref=p2.3_]
%\includecode{resenja/2_PredstavljanjePodataka/2.3_Niske/1_08.c}
\end{Answer}

\begin{Exercise}[label=p2.3_] 
 Napisati funkciju $void\ dupliranje(char\ t[],\ char\ s[])$ koja na osnovu niske $s$ formira nisku $t$ tako što duplira svaki karakter niske $s$. Napisati i program koji učitava nisku maksimalne dužine 20 karaktera i testira rad napisane funkcije.\\
\begin{miditest}
\begin{upotreba}{1}
#\naslovInt#
#\izlaz{Unesite nisku:}\ulaz{zima}#
#\izlaz{zziimmaa}#
\end{upotreba}
\end{miditest}
\begin{miditest}
\begin{upotreba}{2}
#\naslovInt#
#\izlaz{Unesite nisku:}\ulaz{A+B+C}#
#\izlaz{AA++BB++CC}#
\end{upotreba}
\end{miditest}
\begin{miditest}
\begin{upotreba}{3}
#\naslovInt#
#\izlaz{Unesite nisku:}\ulaz{C}#
#\izlaz{CC}#
\end{upotreba}
\end{miditest}
 

\linkresenje{p2.3_}
\end{Exercise}
\begin{Answer}[ref=p2.3_]
%\includecode{resenja/2_PredstavljanjePodataka/2.3_Niske/1_08.c}
\end{Answer}

\begin{Exercise}[label=p2.3_] 
 Napisati funkciju $int\ heksa\_broj(char\ s[])$ koja proverava da li je niskom $s$ zadat korektan heksadekadni broj. Heksadekadni broj je korektno zadat ako počinje prefiksom $0x$ ili $0X$ i ako sadrži samo cifre i mala ili velika slova $A$, $B$, $C$, $D$, $E$ i $F$. Funkcija treba da vrati vrednost 1 ako je niska korektan heksadekadni broj, odnosno 0 ako nije. Napisati i program koji učitava nisku maksimalne dužine 7 karaktera i ispisuje rezultat rada funkcije. \\
\begin{miditest}
\begin{upotreba}{1}
#\naslovInt#
#\izlaz{Unesite nisku:}\ulaz{0x12EF}#
#\izlaz{Korektan heksadekadni broj!}#
\end{upotreba}
\end{miditest}
\begin{miditest}
\begin{upotreba}{2}
#\naslovInt#
#\izlaz{Unesite nisku:}\ulaz{0X22af}#
#\izlaz{Korektan heksadekadni broj!}#
\end{upotreba}
\end{miditest}
\begin{miditest}
\begin{upotreba}{3}
#\naslovInt#
#\izlaz{Unesite nisku:}\ulaz{0xErA9}#
#\izlaz{Nekorektan heksadekadni broj!}#
\end{upotreba}
\end{miditest}

\linkresenje{p2.3_}
\end{Exercise}
\begin{Answer}[ref=p2.3_]
%\includecode{resenja/2_PredstavljanjePodataka/2.3_Niske/1_08.c}
\end{Answer}

\begin{Exercise}[label=p2.3_] 
 Napisati funkciju $int\ heksa\_broj(char\ s[])$ koja izračunava dekadnu vrednost heksadekadnog broja zadatog niskom $s$. Napisati i program koji učitava nisku maksimalne dužine 7 karaktera i ispisuje rezultat rada funkcije. Pretpostaviti da je uneta niska korektan heksadekadni broj. \\
\begin{miditest}
\begin{upotreba}{1}
#\naslovInt#
#\izlaz{Unesite nisku:}\ulaz{0x2A34}#
#\izlaz{10804}#
\end{upotreba}
\end{miditest}
\begin{miditest}
\begin{upotreba}{2}
#\naslovInt#
#\izlaz{Unesite nisku:}\ulaz{0Xff2}#
#\izlaz{4082}#
\end{upotreba}
\end{miditest}
\begin{miditest}
\begin{upotreba}{3}
#\naslovInt#
#\izlaz{Unesite nisku:}\ulaz{0xE1A9}#
#\izlaz{57769}#
\end{upotreba}
\end{miditest}

\linkresenje{p2.3_}
\end{Exercise}
\begin{Answer}[ref=p2.3_]
%\includecode{resenja/2_PredstavljanjePodataka/2.3_Niske/1_08.c}
\end{Answer}

\begin{Exercise}[label=p2.3_] 
 Napisati funkciju $int\ podniska(char\ s[], char\ t[])$ koja proverava da li je niska $t$ podniska niske $s$. Napisati i program koji učitava dve niske maksimalne dužine 10 karaktera i testira rad napisane funkcije.\\
\begin{miditest}
\begin{upotreba}{1}
#\naslovInt#
#\izlaz{Unesite nisku s:}\ulaz{abcde}#
#\izlaz{Unesite nisku t:}\ulaz{bcd}#
#\izlaz{t je podniska niske s!}#
\end{upotreba}
\end{miditest}
\begin{miditest}
\begin{upotreba}{2}
#\naslovInt#
#\izlaz{Unesite nisku s:}\ulaz{abcde}#
#\izlaz{Unesite nisku t:}\ulaz{bCd}#
#\izlaz{t nije podniska niske s!}#
\end{upotreba}
\end{miditest}
\begin{miditest}
\begin{upotreba}{3}
#\naslovInt#
#\izlaz{Unesite nisku s:}\ulaz{abcde}#
#\izlaz{Unesite nisku t:}\ulaz{def}#
#\izlaz{t nije podniska niske s!}#
\end{upotreba}
\end{miditest}

\linkresenje{p2.3_}
\end{Exercise}
\begin{Answer}[ref=p2.3_]
%\includecode{resenja/2_PredstavljanjePodataka/2.3_Niske/1_08.c}
\end{Answer}

\begin{Exercise}[label=p2.3_] 
 Napisati funkciju $void\ modifikacija(char*\ s)$ koja modifikuje nisku $s$ tako što svaki drugi karakter zameni zvezdicom. Pretpostaviti da niska $s$ neće biti duža od 20 karaktera. Napisati i program koji testira rad napisane funkcije. \\
\begin{miditest}
\begin{upotreba}{1}
#\naslovInt#
#\izlaz{Unesite nisku:}\ulaz{123abc789XY}#
#\izlaz{Modifikovana niska je: 1*3*b*7*9*Y}#
\end{upotreba}
\end{miditest}
\begin{miditest}
\begin{upotreba}{2}
#\naslovInt#
#\izlaz{Unesite nisku:}\ulaz{zimA}#
#\izlaz{Modifikovana niska je: z*m*}#
\end{upotreba}
\end{miditest}
\begin{miditest}
\begin{upotreba}{3}
#\naslovInt#
#\izlaz{Unesite nisku:}\ulaz{SNEG}#
#\izlaz{Modifikovana niska je: S*E*}#
\end{upotreba}
\end{miditest}

\linkresenje{p2.3_}
\end{Exercise}
\begin{Answer}[ref=p2.3_]
%\includecode{resenja/2_PredstavljanjePodataka/2.3_Niske/1_08.c}
\end{Answer}

\begin{Exercise}[label=p2.3_] 
 Napisati funkciju $int\ strspn\_klon(char*\ t,\ char*\ s)$ koja izračunava dužinu prefiksa niske $t$ sastavljenog od karaktera niske $s$. Napisati zatim i program koji učitava dve niske maksimalne dužine 20 karaktera i ispisuje rezultat poziva napisane funkcije. \\
\begin{miditest}
\begin{upotreba}{1}
#\naslovInt#
#\izlaz{Unesite nisku t:}\ulaz{programiranje}#
#\izlaz{Unesite nisku s:}\ulaz{opqr}#
#\izlaz{3}#
\end{upotreba}
\end{miditest}
\begin{miditest}
\begin{upotreba}{2}
#\naslovInt#
#\izlaz{Unesite nisku t:}\ulaz{aaiioo124}#
#\izlaz{Unesite nisku s:}\ulaz{aeiou}#
#\izlaz{6}#
\end{upotreba}
\end{miditest}
\begin{miditest}
\begin{upotreba}{3}
#\naslovInt#
#\izlaz{Unesite nisku t:}\ulaz{5296abc}#
#\izlaz{Unesite nisku s:}\ulaz{0123456789}#
#\izlaz{4}#
\end{upotreba}
\end{miditest}

\linkresenje{p2.3_}
\end{Exercise}
\begin{Answer}[ref=p2.3_]
%\includecode{resenja/2_PredstavljanjePodataka/2.3_Niske/1_08.c}
\end{Answer}

\begin{Exercise}[label=p2.3_] 
 Napisati implementaciju funkcije $char*\ strchr_klon(char*\ s,\ char\ c)$ koja vraća pokazivač na prvo pojavljivanje karaktera $c$ u niski $s$ ili NULL ukoliko se karakter $c$ ne pojavljuje u niski $s$. Učitati potom jednu nisku maksimalne dužine 20 karaktera i jedan dodatni karakter i testirati rad napisane funkcije. \\
\begin{miditest}
\begin{upotreba}{1}
#\naslovInt#
#\izlaz{Unesite nisku s:}\ulaz{programiranje}#
#\izlaz{Unesite karakter c:}\ulaz{a}#
#\izlaz{Karakter se nalazi u niski!}#
\end{upotreba}
\end{miditest}
\begin{miditest}
\begin{upotreba}{2}
#\naslovInt#
#\izlaz{Unesite nisku s:}\ulaz{123456789}#
#\izlaz{Unesite karakter c:}\ulaz{y}#
#\izlaz{Karakter se ne nalazi u niski!}#
\end{upotreba}
\end{miditest}
\linkresenje{p2.3_}
\end{Exercise}
\begin{Answer}[ref=p2.3_]
%\includecode{resenja/2_PredstavljanjePodataka/2.3_Niske/1_08.c}
\end{Answer}


