
\section{Niske}

%\subsection{Osnovna konverzija}

\begin{Exercise}[label=v2.3_01] 
   Napisati funkciju \kckod{void konvertuj(char s[])} koja menja datu nisku $s$ tako što
   u njoj mala slova zamenjuje odgovarajućim velikim slovima, a velika slova zamenjuje odgovarajućim malim slovima. Napisati 
   program koji testira ovu funkciju za učitanu nisku maksimalne dužine 10 karaktera.

\komentarJ{Proveriti naziv funkcije u resenju}
   
\begin{minitest}
\begin{upotreba}{1}
#\naslovInt#
#\izlaz{Unesite nisku:}\ulaz{BeoGrad}#
#\izlaz{bEOgRAD}#
\end{upotreba}
\end{minitest}
\begin{minitest}
\begin{upotreba}{2}
#\naslovInt#
#\izlaz{Unesite nisku:}\ulaz{A+B+C}#
#\izlaz{a+b+c}#
\end{upotreba}
\end{minitest}
\begin{minitest}
\begin{upotreba}{3}
#\naslovInt#
#\izlaz{Unesite nisku:}\ulaz{12345}#
#\izlaz{12345}#
\end{upotreba}
\end{minitest}

\linkresenje{v2.3_01}
\end{Exercise}
\begin{Answer}[ref=v2.3_01]
\includecode{resenja/2_PredstavljanjePodataka/2.3_Niske/1_01.c}
\end{Answer}


\begin{Exercise}[label=p2.3_07] 
 Napisati funkciju \kckod{void modifikacija(char s[])} koja modifikuje nisku $s$ tako što u njoj svaki drugi karakter zameni zvezdicom. 
Napisati program koji testira rad napisane funkcije za učitanu nisku maksimalne dužine 20 karaktera. 
 
\begin{miditest}
\begin{upotreba}{1}
#\naslovInt#
#\izlaz{Unesite nisku:}\ulaz{123abc789XY}#
#\izlaz{Modifikovana niska je: 1*3*b*7*9*Y}#
\end{upotreba}
\end{miditest}
\begin{miditest}
\begin{upotreba}{2}
#\naslovInt#
#\izlaz{Unesite nisku:}\ulaz{zimA}#
#\izlaz{Modifikovana niska je: z*m*}#
\end{upotreba}
\end{miditest}

\begin{miditest}
\begin{upotreba}{3}
#\naslovInt#
#\izlaz{Unesite nisku:}\ulaz{SNEG}#
#\izlaz{Modifikovana niska je: S*E*}#
\end{upotreba}
\end{miditest}
\begin{miditest}
\begin{upotreba}{4}
#\naslovInt#
#\izlaz{Unesite nisku:}\ulaz{*a*b*c*}#
#\izlaz{Modifikovana niska je: *******}#
\end{upotreba}
\end{miditest}

\linkresenje{p2.3_07}
\end{Exercise}
\begin{Answer}[ref=p2.3_07]
%\includecode{resenja/2_PredstavljanjePodataka/2.3_Niske/1_08.c}
\end{Answer}

\begin{Exercise}[label=p2.3_] 
Napisati program koji vrši poređenje niski. Napisati funkcije:
\begin{enumerate}
\item \kckod{int poredjenje(char s1[], char s2[])}
--- vraća 1 ako su $s_1$ i $s_2$ jednake niske, a 0 u suprotnom;

\item \kckod{void u\_velika\_slova(char s[])}
--- pretvara sva slova niske $s$ u velika, a ostale znakove ne menja.
\end{enumerate}
Napisati program koji za učitane dve reči dužine najviše 20 znakova ispituje da li su jednake, pri čemu se zanemaruje razlika između velikih i malih slova. 

\begin{minitest}
\begin{upotreba}{1}
#\naslovInt#
#\ulaz{isPit2010}#
#\ulaz{IsPiT2010}#
#\izlaz{jesu jednake}#
\end{upotreba}
\end{minitest}
\begin{minitest}
\begin{upotreba}{2}
#\naslovInt#
#\ulaz{Prog1}#
#\ulaz{prog2}#
#\izlaz{nisu jednake}#
\end{upotreba}
\end{minitest}
\begin{minitest}
\begin{upotreba}{3}
#\naslovInt#
#\ulaz{junski}#
#\ulaz{septembarski}#
#\izlaz{nisu jednake}#
\end{upotreba}
\end{minitest}

\linkresenje{p2.3_}
\end{Exercise}
\begin{Answer}[ref=p2.3_]
%\includecode{resenja/2_PredstavljanjePodataka/2.3_Niske/1_08.c}
\end{Answer}

%\subsection{Karakteristike (da/ne)}

\begin{Exercise}[label=p2.3_01] 
Napisati program koji proverava da li se uneta niska završava samoglasnikom. Napisati funkcije:
\begin{enumerate}
\item  \kckod{int samoglasnik(char c)} --- ispituje da li je karakter $c$ samoglasnik;
\item  \kckod{int samoglasnik\_na\_kraju(char s[])} --- ispituje da li se niska $s$ završava samoglasnikom.
\end{enumerate}
Pretpostaviti da je uneta niska maksimalne dužine 20 karaktera. 


\komentarM{Smisliti neke lepse test primere sa smislenijim recima, npr u Andjelkinom stilu :-) }

\begin{miditest}
\begin{upotreba}{1}
#\naslovInt#
#\izlaz{Unesite nisku:}\ulaz{abcde}#
#\izlaz{Niska se zavrsava samoglasnikom!}#
\end{upotreba}
\end{miditest}
\begin{miditest}
\begin{upotreba}{2}
#\naslovInt#
#\izlaz{Unesite nisku:}\ulaz{AaBb+cCdD}#
#\izlaz{Niska se ne zavrsava samoglasnikom!}#
\end{upotreba}
\end{miditest}

\begin{miditest}
\begin{upotreba}{3}
#\naslovInt#
#\izlaz{Unesite nisku:}\ulaz{PrograMiranjE}#
#\izlaz{Niska se zavrsava samoglasnikom!}#
\end{upotreba}
\end{miditest}
\begin{miditest}
\begin{upotreba}{4}
#\naslovInt#
#\izlaz{Unesite nisku:}\ulaz{pRograMiranjE1}#
#\izlaz{Niska se ne zavrsava samoglasnikom!}#
\end{upotreba}
\end{miditest}
\linkresenje{p2.3_01}
\end{Exercise}
\begin{Answer}[ref=p2.3_01]
\includecode{resenja/2_PredstavljanjePodataka/2.3_Niske/praktikumi11/7.c}
\end{Answer}


\begin{Exercise}[label=v2.3_08] 
   Napisati program koji za učitanu nisku $s$ i karakter $c$ ispituje da li se $c$ pojavljuje u niski $s$. Ako se pojavljuje, program treba da ispiše indeks prvog pojavljivanja karaktera $c$ u niski $s$, a u suprotnom -1. Pretpostaviti da niska može da ima najviše 20 karaktera.
   
\begin{minitest}
\begin{upotreba}{1}
#\naslovInt#
#\izlaz{Unesite nisku:}#
#\ulaz{bazen}#
#\izlaz{Unesite karakter:}#
#\ulaz{z}#
#\izlaz{2}#
\end{upotreba}
\end{minitest}
\begin{minitest}
\begin{upotreba}{2}
#\naslovInt#
#\izlaz{Unesite nisku:}#
#\ulaz{lezaljka}#
#\izlaz{Unesite karakter:}#
#\ulaz{a}#
#\izlaz{3}#
\end{upotreba}
\end{minitest}
\begin{minitest}
\begin{upotreba}{3}
#\naslovInt#
#\izlaz{Unesite nisku:}#
#\ulaz{limunada}#
#\izlaz{Unesite karakter:}#
#\ulaz{b}#
#\izlaz{-1}#
\end{upotreba}
\end{minitest}

\linkresenje{v2.3_08}
\end{Exercise}
\begin{Answer}[ref=v2.3_08]
\includecode{resenja/2_PredstavljanjePodataka/2.3_Niske/1_08.c}
\end{Answer}


\begin{Exercise}[label=p2.3_] 
Napisati funkciju \kckod{int sadrzi\_veliko(char s[])} koja proverava da li niska $s$ sadrži veliko slovo. Napisati program koji za učitanu nisku maksimalne dužine 20 karaktera proverava da li sadrži veliko slovo i ispisuje odgovarajuću poruku.

\begin{minitest}
\begin{upotreba}{1}
#\naslovInt#
#\izlaz{Unesite nisku:}#
#\ulaz{naocare}#
#\izlaz{Niska ne sadrzi veliko slovo}#
\end{upotreba}
\end{minitest}
\begin{minitest}
\begin{upotreba}{2}
#\naslovInt#
#\izlaz{Unesite nisku:}#
#\ulaz{DiopTrija0.75}#
#\izlaz{Niska sadrzi veliko slovo}#
\end{upotreba}
\end{minitest}
\begin{minitest}
\begin{upotreba}{3}
#\naslovInt#
#\izlaz{Unesite nisku:}#
#\ulaz{21.06.2017.}#
#\izlaz{Niska ne sadrzi veliko slovo}#
\end{upotreba}
\end{minitest}


\komentarJ{Proveriti naziv funkcije u resenju}
\linkresenje{p2.3_}
\end{Exercise}
\begin{Answer}[ref=p2.3_]
%\includecode{resenja/2_PredstavljanjePodataka/2.3_Niske/1_08.c}
\end{Answer}

\begin{Exercise}[label=p2.3_06] 
 Napisati funkciju \kckod{int podniska(char s[], char t[])} koja proverava da li je niska $t$ podniska niske $s$. Napisati  program koji učitava dve niske maksimalne dužine 10 karaktera i testira rad napisane funkcije.\\
\begin{minitest}
\begin{upotreba}{1}
#\naslovInt#
#\izlaz{Unesite nisku s:}\ulaz{abcde}#
#\izlaz{Unesite nisku t:}\ulaz{bcd}#
#\izlaz{t je podniska niske s!}#
\end{upotreba}
\end{minitest}
\begin{minitest}
\begin{upotreba}{2}
#\naslovInt#
#\izlaz{Unesite nisku s:}\ulaz{abcde}#
#\izlaz{Unesite nisku t:}\ulaz{bCd}#
#\izlaz{t nije podniska niske s!}#
\end{upotreba}
\end{minitest}
\begin{minitest}
\begin{upotreba}{3}
#\naslovInt#
#\izlaz{Unesite nisku s:}\ulaz{abcde}#
#\izlaz{Unesite nisku t:}\ulaz{def}#
#\izlaz{t nije podniska niske s!}#
\end{upotreba}
\end{minitest}

\linkresenje{p2.3_06}
\end{Exercise}
\begin{Answer}[ref=p2.3_06]
\includecode{resenja/2_PredstavljanjePodataka/2.3_Niske/praktikumi11/12.c}
\end{Answer}

%\subsection{Izbacivanje karaktera}

\begin{Exercise}[label=v2.3_02] 
   Napisati funkciju \kckod{void skrati(char s[])} koja uklanja beline sa
   kraja date niske. Napisati 
   program koji testira ovu funkciju za učitanu liniju maksimalne dužine 100 karaktera. Prikazati učitanu i izmenjenu nisku između zvezdica.

\begin{miditest}
\begin{upotreba}{1}
#\naslovInt#
#\izlaz{Unesite nisku:}\ulaz{rep belina }#
#\izlaz{ucitana niska: *rep belina }#             #\izlaz{ *}#
#\izlaz{izmenjena niska: *rep belina*}#
\end{upotreba}
\end{miditest}

\begin{maxitest}
\begin{upotreba}{2}
#\naslovInt#
#\izlaz{Unesite nisku:}\ulaz{tri tabulatora na kraju            }#
#\izlaz{ucitana niska: *tri tabulatora na kraju}#			#\izlaz{ *}#
#\izlaz{izmenjena niska: *tri tabulatora na kraju*}#
\end{upotreba}
\end{maxitest}
\linkresenje{v2.3_02}
\end{Exercise}
\begin{Answer}[ref=v2.3_02]
\includecode{resenja/2_PredstavljanjePodataka/2.3_Niske/1_02.c}
\end{Answer}



\begin{Exercise}[label=p2.3_] 
Napisati funkciju \kckod{void ukloni\_slova(char s[])} koja iz niske $s$ uklanja sva mala i sva velika slova. Napisati program koji za učitanu nisku maksimalne dužine 20 karaktera ispisuje odgovarajuću izmenjenu nisku.

\begin{minitest}
\begin{upotreba}{1}
#\naslovInt#
#\izlaz{Unesite nisku:}#
#\ulaz{abcd123ABCD}#
#\izlaz{123}#
\end{upotreba}
\end{minitest}
\begin{minitest}
\begin{upotreba}{2}
#\naslovInt#
#\izlaz{Unesite nisku:}#
#\ulaz{1+2=3}#
#\izlaz{1+2=3}#
\end{upotreba}
\end{minitest}
\begin{minitest}
\begin{upotreba}{2}
#\naslovInt#
#\izlaz{Unesite nisku:}#
#\ulaz{malaVELIKA}#
#\izlaz{}#
\end{upotreba}
\end{minitest}


\komentarJ{Proveriti naziv funkcije u resenju}
\linkresenje{p2.3_}
\end{Exercise}
\begin{Answer}[ref=p2.3_]
%\includecode{resenja/2_PredstavljanjePodataka/2.3_Niske/1_08.c}
\end{Answer}


\begin{Exercise}[label=p2.3_] 
 Napisati funkciju \kckod{void ukloni(char *s)} koja iz niske uklanja
 sva slova iza kojih neposredno sledi slovo koje je u abecedi nakon
 njih, pri čemu se veličina slova zanemaruje. Testirati funkciju u programu
 za učitanu liniju od najviše 100 karaktera. 
 
\komentarM{Ovi su test primeri stvarno ruzni - naci neke koji znace nesto} 
 
\begin{minitest}
\begin{upotreba}{1}
#\naslovInt#
#\ulaz{zdRaVo svIma}#
#\izlaz{zRVo vma}#
\end{upotreba}
\end{minitest}
\begin{minitest}
\begin{upotreba}{2}
#\naslovInt#
#\ulaz{12345AbcD}#
#\izlaz{12345D}#
\end{upotreba}
\end{minitest}
\begin{minitest}
\begin{upotreba}{3}
#\naslovInt#
#\ulaz{JeD1aN D52Va.}#
#\izlaz{JeD1N D52Va.}#
\end{upotreba}
\end{minitest}
%\begin{minitest}
%\begin{upotreba}{4}
%#\naslovInt#
%#\ulaz{abcd efg}#
%#\izlaz{d g}#
%\end{upotreba}
%\end{minitest}
 \linkresenje{p2.3_}
\end{Exercise}
\begin{Answer}[ref=p2.3_]
%\includecode{resenja/2_PredstavljanjePodataka/2.3_Niske/1_08.c}
\end{Answer}

%\subsection{Kopiranje jedne niske u drugu}


\begin{Exercise}[label=v2.3_03] 
   Napisati program koji učitava nisku $src$ i formira nisku $dst$
   trostrukim nadovezivanjem niske $src$. Možemo pretpostaviti da niska $src$ sadrži najviše 30 karaktera. 
   
\begin{minitest}
\begin{upotreba}{1}
#\naslovInt#
#\izlaz{Unesite nisku:}\ulaz{dan}#
#\izlaz{dandandan}#
\end{upotreba}
\end{minitest}
\begin{minitest}
\begin{upotreba}{2}
#\naslovInt#
#\izlaz{Unesite nisku:}\ulaz{3sesira}#
#\izlaz{3sesira3sesira3sesira}#
\end{upotreba}
\end{minitest}
\begin{minitest}
\begin{upotreba}{3}
#\naslovInt#
#\izlaz{Unesite nisku:}\ulaz{a-b=5}#
#\izlaz{a-b=5a-b=5a-b=5}#
\end{upotreba}
\end{minitest}

\linkresenje{v2.3_03}
\end{Exercise}
\begin{Answer}[ref=v2.3_03]
\includecode{resenja/2_PredstavljanjePodataka/2.3_Niske/1_03.c}
\end{Answer}


\begin{Exercise}[label=p2.3_]  Napisati program
koji za unetu reč maksimalne dužine 20 karaktera formira rezultujuću reč tako što   unetu reč kopira 4 puta,  pri čemu se između svaka dva kopiranja umeće  crtica. 

Zadatak uraditi:
\begin{enumerate}
\item pisanjem odgovaraju\'ce funkcije koja vr\v si nadovezivanje re\v ci,
\item koriste\'ci postoje\'cu funkciju \kckod{strcat} iz biblioteke \kckod{string.h}.
\end{enumerate}

\begin{minitest}
\begin{upotreba}{1}
#\naslovInt#
#\izlaz{Unesite nisku:}\ulaz{ana}#
#\izlaz{ana-ana-ana-ana}#
\end{upotreba}
\end{minitest}
\begin{minitest}
\begin{upotreba}{2}
#\naslovInt#
#\izlaz{Unesite nisku:}\ulaz{123}#
#\izlaz{123-123-123-123}#
\end{upotreba}
\end{minitest}
\begin{minitest}
\begin{upotreba}{3}
#\naslovInt#
#\izlaz{Unesite nisku:}\ulaz{x*y}#
#\izlaz{x*y-x*y-x*y-x*y}#
\end{upotreba}
\end{minitest}

\linkresenje{p2.3_}
\end{Exercise}
\begin{Answer}[ref=p2.3_]
%\includecode{resenja/2_PredstavljanjePodataka/2.3_Niske/1_08.c}
\end{Answer}


\begin{Exercise}[label=p2.3_02] 
Napisati funkciju \kckod{void kopiraj\_n(char t[], char s[], int n)} koja kopira najviše $n$ karaktera niske $s$ u nisku $t$. Napisati program koji testira rad napisane funkcije. Pretpostaviti da je maksimalna dužina niske $s$ 20 karaktera.

\begin{minitest}
\begin{upotreba}{1}
#\naslovInt#
#\izlaz{Unesite nisku:}\ulaz{petar}#
#\izlaz{Unesite broj n:}\ulaz{3}#
#\izlaz{Rezultujuca niska: pet}#
\end{upotreba}
\end{minitest}
\begin{minitest}
\begin{upotreba}{2}
#\naslovInt#
#\izlaz{Unesite nisku:}\ulaz{gromobran}#
#\izlaz{Unesite broj n:}\ulaz{4}#
#\izlaz{Rezultujuca niska: grom}#
\end{upotreba}
\end{minitest}
\begin{minitest}
\begin{upotreba}{3}
#\naslovInt#
#\izlaz{Unesite nisku:}\ulaz{abc}#
#\izlaz{Unesite broj n:}\ulaz{15}#
#\izlaz{Rezultujuca niska: abc}#
\end{upotreba}
\end{minitest}


\linkresenje{p2.3_02}
\end{Exercise}
\begin{Answer}[ref=p2.3_02]
\includecode{resenja/2_PredstavljanjePodataka/2.3_Niske/praktikumi11/8.c}
\end{Answer}


\begin{Exercise}[label=p2.3_03] 
 Napisati funkciju \kckod{void dupliranje(char t[], char s[])} koja na osnovu niske $s$ formira nisku $t$ tako što duplira svaki karakter niske $s$. Napisati program koji učitava nisku maksimalne dužine 20 karaktera i testira rad napisane funkcije.

\begin{minitest}
\begin{upotreba}{1}
#\naslovInt#
#\izlaz{Unesite nisku:}\ulaz{zima}#
#\izlaz{zziimmaa}#
\end{upotreba}
\end{minitest}
\begin{minitest}
\begin{upotreba}{2}
#\naslovInt#
#\izlaz{Unesite nisku:}\ulaz{A+B+C}#
#\izlaz{AA++BB++CC}#
\end{upotreba}
\end{minitest}
\begin{minitest}
\begin{upotreba}{3}
#\naslovInt#
#\izlaz{Unesite nisku:}\ulaz{C}#
#\izlaz{CC}#
\end{upotreba}
\end{minitest}
 

\linkresenje{p2.3_03}
\end{Exercise}
\begin{Answer}[ref=p2.3_03]
\includecode{resenja/2_PredstavljanjePodataka/2.3_Niske/praktikumi11/9.c}
\end{Answer}

%\subsection{Numericki}

\begin{Exercise}[label=v2.3_05] 
Napisati program koji učitava nisku cifara sa eventualnim vodećim znakom i pretvara je u ceo broj. \napomena{Zadatak realizovati bez korišćenja ugrađene funkcije atoi iz biblioteke stdlib.h}  
  
\begin{minitest}
\begin{upotreba}{1}
#\naslovInt#
#\izlaz{Unesite nisku:}#
#\ulaz{-1238}#
#\izlaz{-1238}#
\end{upotreba}
\end{minitest}
\begin{minitest}
\begin{upotreba}{2}
#\naslovInt#
#\izlaz{Unesite nisku:}#
#\ulaz{73}#
#\izlaz{73}#
\end{upotreba}
\end{minitest}
\begin{minitest}
\begin{upotreba}{3}
#\naslovInt#
#\izlaz{Unesite nisku:}#
#\ulaz{+1}#
#\izlaz{1}#
\end{upotreba}
\end{minitest}

\linkresenje{v2.3_05}
\end{Exercise}
\begin{Answer}[ref=v2.3_05]
\includecode{resenja/2_PredstavljanjePodataka/2.3_Niske/1_05.c}
\end{Answer}


\begin{Exercise}[label=v2.3_06] 
Napisati program koji učitava ceo broj i pretvara ga u nisku. 
	
\begin{minitest}
\begin{upotreba}{1}
#\naslovInt#
#\izlaz{Unesite ceo broj:}#
#\ulaz{-6543}#
#\izlaz{-6543}#
\end{upotreba}
\end{minitest}
\begin{minitest}
\begin{upotreba}{2}
#\naslovInt#
#\izlaz{Unesite ceo broj:}#
#\ulaz{84}#
#\izlaz{84}#
\end{upotreba}
\end{minitest}
\begin{minitest}
\begin{upotreba}{3}
#\naslovInt#
#\izlaz{Unesite ceo broj:}#
#\ulaz{5}#
#\izlaz{5}#
\end{upotreba}
\end{minitest}

\linkresenje{v2.3_06}
\end{Exercise}
\begin{Answer}[ref=v2.3_06]
\includecode{resenja/2_PredstavljanjePodataka/2.3_Niske/1_06.c}
\end{Answer}


\begin{Exercise}[label=p2.3_04] 
Napisati funkciju \kckod{int heksadekadni\_broj(char s[])} koja proverava da li je niskom $s$ zadat korektan heksadekadni broj. Funkcija treba da vrati vrednost 1 ukoliko je uslov ispunjen, odnosno 0 ako nije. Napisati program koji za učitanu nisku maksimalne dužine 7 karaktera ispisuje da li je korektan heksadekadni broj. 
\uputstvo{Heksadekadni broj je korektno zadat ako počinje prefiksom $0x$ ili $0X$ i ako sadrži samo cifre i mala ili velika slova $A$, $B$, $C$, $D$, $E$ i $F$.} 
 
\komentarJ{Proveriti naziv funkcije u resenju} 
 
\begin{minitest}
\begin{upotreba}{1}
#\naslovInt#
#\izlaz{Unesite nisku:}\ulaz{0x12EF}#
#\izlaz{Korektan heksadekadni broj!}#
\end{upotreba}
\end{minitest}
\begin{minitest}
\begin{upotreba}{2}
#\naslovInt#
#\izlaz{Unesite nisku:}\ulaz{0X22af}#
#\izlaz{Korektan heksadekadni broj!}#
\end{upotreba}
\end{minitest}
\begin{minitest}
\begin{upotreba}{3}
#\naslovInt#
#\izlaz{Unesite nisku:}\ulaz{0xErA9}#
#\izlaz{Nekorektan heksadekadni broj!}#
\end{upotreba}
\end{minitest}

\linkresenje{p2.3_04}
\end{Exercise}
\begin{Answer}[ref=p2.3_04]
\includecode{resenja/2_PredstavljanjePodataka/2.3_Niske/praktikumi11/10.c}
\end{Answer}


\begin{Exercise}[label=p2.3_05] 
 Napisati funkciju \kckod{int dekadna\_vrednost(char s[])} koja izračunava dekadnu vrednost heksadekadnog broja zadatog niskom $s$. Napisati program koji za učitanu nisku maksimalne dužine 7 karaktera ispisuje odgovarajuću dekadnu vrednost. Pretpostaviti da je uneta niska korektan heksadekadni broj.

\komentarJ{Proveriti naziv funkcije u resenju} 

\begin{minitest}
\begin{upotreba}{1}
#\naslovInt#
#\izlaz{Unesite nisku:}\ulaz{0x2A34}#
#\izlaz{10804}#
\end{upotreba}
\end{minitest}
\begin{minitest}
\begin{upotreba}{2}
#\naslovInt#
#\izlaz{Unesite nisku:}\ulaz{0Xff2}#
#\izlaz{4082}#
\end{upotreba}
\end{minitest}
\begin{minitest}
\begin{upotreba}{3}
#\naslovInt#
#\izlaz{Unesite nisku:}\ulaz{0xE1A9}#
#\izlaz{57769}#
\end{upotreba}
\end{minitest}

\linkresenje{p2.3_05}
\end{Exercise}
\begin{Answer}[ref=p2.3_05]
\includecode{resenja/2_PredstavljanjePodataka/2.3_Niske/praktikumi11/11.c}
\end{Answer}




%\subsection{Linije sa min/max}

\begin{Exercise}[label=v2.3_04] 
Napisati funkciju \kckod{int ucitaj\_liniju(char s[], int n)}
koja učitava liniju maksimalne dužine $n$ u nisku $s$
i vraća dužinu učitane linije.  
Napisati program koji učitava linije
do \kckod{EOF} i ispisuje najdužu liniju i njenu dužinu. Ukoliko
ima više linija maksimalne dužine, ispisati prvu. 
Pretpostviti da svaka linija sadrži najviše 80 karaktera.
\napomena{Linija može da sadrži
blanko znakove, ali ne sadrži znak za novi red ili \kckod{EOF}.}

\begin{minitest}
\begin{upotreba}{1}
#\naslovInt#
#\izlaz{Unesite linije:}#
#\ulaz{Dobar dan!}#
#\ulaz{Kako ste, sta ima novo?}#
#\ulaz{Ja sam dobro.}#
#\izlaz{Kako ste, sta ima novo?}#
#\izlaz{23}#
\end{upotreba}
\end{minitest}
\begin{minitest}
\begin{upotreba}{2}
#\naslovInt#
#\izlaz{Unesite linije:}#
#\ulaz{Prva linija}#
#\ulaz{Druga linija}#
#\ulaz{Treca linija}#
#\izlaz{Druga linija}#
#\izlaz{12}#
\end{upotreba}
\end{minitest}
\begin{minitest}
\begin{upotreba}{3}
#\naslovInt#
#\izlaz{Unesite linije:}#
#\ulaz{Danas je lep dan.}#
#\izlaz{Danas je lep dan.}#
#\izlaz{17}#
\end{upotreba}
\end{minitest}

\linkresenje{v2.3_04}
\end{Exercise}
\begin{Answer}[ref=v2.3_04]
\includecode{resenja/2_PredstavljanjePodataka/2.3_Niske/1_04.c}
\end{Answer}

\begin{Exercise}[label=p2.3_] 
Napisati funkcije za rad sa rečenicama:
\begin{enumerate}
\item \kckod{int procitaj\_recenicu(char s[], int max\_len)} koja učitava rečenicu i smešta je u nisku $s$. Funkcija vraća dužinu učitane rečenice. Učitavanje se završava nakon učitanog karaktera \kckod{.} ili nakon $max\_len-1$ učitanih karaktera.
\item \kckod{void prebroj(char s[], int *broj\_malih, int *broj\_velikih)} koja prebrojava mala i velika slova u niski $s$.
\end{enumerate}
 Napisati program koji učitava rečenice do kraja ulaza i ispisuje onu rečenicu kod koje je razlika broja malih i velikih slova najveća.

  \komentarJ{Ovo nema resenje? Nedostaju test primeri}

 \linkresenje{p2.3_}
\end{Exercise}
\begin{Answer}[ref=p2.3_]
%\includecode{resenja/2_PredstavljanjePodataka/2.3_Niske/1_08.c}
\end{Answer}

%\subsection{Klonovi}


\begin{Exercise}[label=p2.3_09] 
Napisati funkciju \kckod{char* strchr\_klon(char s[], char c)} koja vraća pokazivač na prvo pojavljivanje karaktera $c$ u niski $s$ ili $NULL$ ukoliko se karakter $c$ ne pojavljuje u niski $s$. Napisati program koji za učitanu nisku maksimalne dužine 20 karaktera i dodatni karakter testira rad napisane funkcije. 
 
\begin{miditest}
\begin{upotreba}{1}
#\naslovInt#
#\izlaz{Unesite nisku s:}\ulaz{programiranje}#
#\izlaz{Unesite karakter c:}\ulaz{a}#
#\izlaz{Karakter se nalazi u niski!}#
\end{upotreba}
\end{miditest}
\begin{minitest}
\begin{upotreba}{2}
#\naslovInt#
#\izlaz{Unesite nisku s:}\ulaz{123456789}#
#\izlaz{Unesite karakter c:}\ulaz{y}#
#\izlaz{Karakter se ne nalazi u niski!}#
\end{upotreba}
\end{minitest}

\begin{minitest}
\begin{upotreba}{3}
#\naslovInt#
#\izlaz{Unesite nisku s:}\ulaz{leto2017}#
#\izlaz{Unesite karakter c:}\ulaz{0}#
#\izlaz{Karakter se nalazi u niski!}#
\end{upotreba}
\end{minitest}
\linkresenje{p2.3_09}
\end{Exercise}
\begin{Answer}[ref=p2.3_09]
%\includecode{resenja/2_PredstavljanjePodataka/2.3_Niske/1_08.c}
\end{Answer}

\begin{Exercise}[label=p2.3_08] 
 Napisati funkciju \kckod{int strspn\_klon(char t[], char s[])} koja izračunava dužinu prefiksa niske $t$ sastavljenog od karaktera niske $s$. Napisati program koji za učitane dve niske maksimalne dužine 20 karaktera ispisuje rezultat poziva napisane funkcije. 
 
\komentarM{Smisliti neke smislenije test primere, ovi su mnogo ruzni} 

\begin{minitest}
\begin{upotreba}{1}
#\naslovInt#
#\izlaz{Unesite nisku t:}\ulaz{program}#
#\izlaz{Unesite nisku s:}\ulaz{opqr}#
#\izlaz{3}#
\end{upotreba}
\end{minitest}
\begin{minitest}
\begin{upotreba}{2}
#\naslovInt#
#\izlaz{Unesite nisku t:}\ulaz{aaiioo124}#
#\izlaz{Unesite nisku s:}\ulaz{aeiou}#
#\izlaz{6}#
\end{upotreba}
\end{minitest}
\begin{minitest}
\begin{upotreba}{3}
#\naslovInt#
#\izlaz{Unesite nisku t:}\ulaz{5296abc}#
#\izlaz{Unesite nisku s:}\ulaz{0123456789}#
#\izlaz{4}#
\end{upotreba}
\end{minitest}

\linkresenje{p2.3_08}
\end{Exercise}
\begin{Answer}[ref=p2.3_08]
%\includecode{resenja/2_PredstavljanjePodataka/2.3_Niske/1_08.c}
\end{Answer}

% Ovaj zadatak treba da ide iza prethodnog da bi se videla razlika
\begin{Exercise}[label=p2.3_] 
Napisati funkciju \kckod{int duzina(char s[], char t[])}
koja izračunava dužinu početnog dela niske $s$ sastavljenog
isključivo od karaktera sadržanih u niski $t$. Napisati  program koji testira ovu funkciju za dve unete niske maksimalne dužine 100 karaktera. 

\komentarM{Da li je ovo neciji klon? Cini mi se da nije, ako jeste, dodajte odgovarajuce ime, ako nije, izmenila sam ime jer je ono f bilo bezveze.}

\begin{minitest}
\begin{upotreba}{1}
#\naslovInt#
#\izlaz{Unesite prvu nisku:}#
#\ulaz{734a.bf62}#
#\izlaz{Unesite drugu nisku:}#
#\ulaz{0123456789}#
#\izlaz{3}#
\end{upotreba}
\end{minitest}
\begin{minitest}
\begin{upotreba}{2}
#\naslovInt#
#\izlaz{Unesite prvu nisku:}#
#\ulaz{abrakadabra}#
#\izlaz{Unesite drugu nisku:}#
#\ulaz{brada}#
#\izlaz{4}#
\end{upotreba}
\end{minitest}
\begin{minitest}
\begin{upotreba}{3}
#\naslovInt#
#\izlaz{Unesite prvu nisku:}#
#\ulaz{popokatepetl}#
#\izlaz{Unesite drugu nisku:}#
#\ulaz{opna}#
#\izlaz{2}#
\end{upotreba}
\end{minitest}
 \linkresenje{p2.3_}
\end{Exercise}
\begin{Answer}[ref=p2.3_]
%\includecode{resenja/2_PredstavljanjePodataka/2.3_Niske/1_08.c}
\end{Answer}



\begin{Exercise}[label=p2.3_] 
Napisati funkciju \kckod{char* strstr\_klon(char s[], char t[])} koja vraća pokazivač na prvo pojavljivanje niske $t$ u niski $s$ ili $NULL$ ukoliko se niska $t$ ne pojavljuje u niski $s$. Napisati program koji testira napisanu funkciju tako što učitava pet linija i ispisuje sve redne brojeve linija koje sadr\v ze nisku $program$. Ukoliko ne postoji linija sa niskom $program$, ispisati odgovarajuću poruku. Pretpostaviti da je svaka linija maksimalne dužine 100 karaktera kao i da se linije numerišu od broja 1. 

\komentarJ{Tekst je malo izmenjen, prilagoditi resenje}

\begin{minitest}
\begin{upotreba}{1}
#\naslovInt#
#\izlaz{Unesite pet linija:}#
#\ulaz{tv program}#
#\ulaz{c prog. jezik}#
#\ulaz{c++ programskih jezik}#
#\ulaz{Programski odbor}#
#\ulaz{<b>program</b>}#
#\izlaz{1 3 5}#
\end{upotreba}
\end{minitest}
\begin{minitest}
\begin{upotreba}{2}
#\naslovInt#
#\izlaz{Unesite pet linija:}#
#\ulaz{Programske paradigme}#
#\ulaz{su predmet na}#
#\ulaz{trecoj godini}#
#\ulaz{programerskih}#
#\ulaz{smerova.}#
#\izlaz{4}#
\end{upotreba}
\end{minitest}
\begin{minitest}
\begin{upotreba}{3}
#\naslovInt#
#\izlaz{Unesite pet linija:}#
#\ulaz{U narednim}#
#\ulaz{linijama}#
#\ulaz{necemo navoditi}#
#\ulaz{nisku koja se}#
#\ulaz{trazi.}#
#\izlaz{Nijedna linija ne sadrzi}#
#\izlaz{nisku program.}#
\end{upotreba}
\end{minitest}

\linkresenje{p2.3_}
\end{Exercise}
\begin{Answer}[ref=p2.3_]
%\includecode{resenja/2_PredstavljanjePodataka/2.3_Niske/1_08.c}
\end{Answer}


%\subsection{Rotacije}

\begin{Exercise}[label=p2.3_] 
Napisati funkciju \kckod{void rotiraj(char s[], int k)} koja rotira
nisku $s$ za $k$ mesta ulevo. Napisati program koji rotira učitanu nisku maksimalne dužine 20 karaktera i ispisuje rotiranu nisku.

\komentarM{Ovde je bio zadatak "napsiati funkciju obrni koja rotira nisku". Mozda treba dodati zadatak koji obce nisku?}

\begin{minitest}
\begin{upotreba}{1}
#\naslovInt#
#\ulaz{sveska}#
#\ulaz{2}#
#\izlaz{eskasv}#
\end{upotreba}
\end{minitest}
\begin{minitest}
\begin{upotreba}{2}
#\naslovInt#
#\ulaz{olovka}#
#\ulaz{6}#
#\izlaz{olovka}#
\end{upotreba}
\end{minitest}
\begin{minitest}
\begin{upotreba}{3}
#\naslovInt#
#\ulaz{rezac}#
#\ulaz{8}#
#\izlaz{acrez}#
\end{upotreba}
\end{minitest}

\linkresenje{p2.3_}
\end{Exercise}
\begin{Answer}[ref=p2.3_]
%\includecode{resenja/2_PredstavljanjePodataka/2.3_Niske/1_08.c}
\end{Answer}

%\subsection{Sifriranja}

\begin{Exercise}[label=p2.3_] 
Napisati program koji šifrira unetu nisku tako sto svako slovo zamenjuje sledećim slovom abecede, slova ’z' i 'Z' zamenjuje redom sa 'a' i ’A’, a ostale karaktere ostavlja nepromenjene.  pretpostaviti da uneta niska nije duža od 20 karaktera.

\komentarM{Izmeniti test primere tako da budu smisleni}

\begin{minitest}
\begin{upotreba}{1}
#\naslovInt#
#\izlaz{Unesite nisku:}#
#\ulaz{AbcXyz}#
#\izlaz{BcdYza}#
\end{upotreba}
\end{minitest}
\begin{minitest}
\begin{upotreba}{2}
#\naslovInt#
#\izlaz{Unesite nisku:}#
#\ulaz{lmnopqr123}#
#\izlaz{mnopqrs123}#
\end{upotreba}
\end{minitest}
\begin{minitest}
\begin{upotreba}{3}
#\naslovInt#
#\izlaz{Unesite nisku:}#
#\ulaz{1,2,3,4,5}#
#\izlaz{1,2,3,4,5}#
\end{upotreba}
\end{minitest}

\linkresenje{p2.3_}
\end{Exercise}
\begin{Answer}[ref=p2.3_]
%\includecode{resenja/2_PredstavljanjePodataka/2.3_Niske/1_08.c}
\end{Answer}


\begin{Exercise}[label=p2.3_] 
Napisati funkciju \kckod{void sifruj(char rec[], char sifra[])} koja na osnovu date reči formira šifru tako što se svako slovo u reči zameni sa naredna tri slova u abecedi. Napisati program koji testira napisanu funkciju za reč maksimalne dužine 20 karaktera. 

\begin{minitest}
\begin{upotreba}{1}
#\naslovInt#
#\izlaz{Unesite nisku:}#
#\ulaz{tamo}#
#\izlaz{uvwbcdnoppqr}#
\end{upotreba}
\end{minitest}
\begin{minitest}
\begin{upotreba}{2}
#\naslovInt#
#\izlaz{Unesite nisku:}#
#\ulaz{Zec}#
#\izlaz{ABCfghdef}#
\end{upotreba}
\end{minitest}
\begin{minitest}
\begin{upotreba}{3}
#\naslovInt#
#\izlaz{Unesite nisku:}#
#\ulaz{a+b=c}#
#\izlaz{bcd+cde=def}#
\end{upotreba}
\end{minitest}

\linkresenje{p2.3_}
\end{Exercise}
\begin{Answer}[ref=p2.3_]
%\includecode{resenja/2_PredstavljanjePodataka/2.3_Niske/1_08.c}
\end{Answer}





\begin{Exercise}[label=p2.3_] 
Napisati funkciju \kckod{void indel(char s1[], char s2[], char c1, char c2)} koja na osnovu niske $s_1$ formira nisku $s_2$ udvajanjem svih karaktera $c_1$ u niski $s_1$ i  izbacivanjem svih karaktera $c_2$ iz niske $s_1$, dok ostali karakteri ostaju nepromenjeni. Napisati program koji testira ovu funkciju za unetu nisku i dva uneta karaktera. Pretpostaviti da uneta niska nije duža od 20 karaktera.

\begin{minitest}
\begin{upotreba}{1}
#\naslovInt#
#\izlaz{Unesite nisku:}#
#\ulaz{flomaster}#
#\izlaz{Unesite prvi karakter:}#
#\ulaz{m}#
#\izlaz{Unesite drugi karakter:}#
#\ulaz{s}#
#\ulaz{floasster}#
\end{upotreba}
\end{minitest}
\begin{minitest}
\begin{upotreba}{2}
#\naslovInt#
#\izlaz{Unesite nisku:}#
#\ulaz{bojica}#
#\izlaz{Unesite prvi karakter:}#
#\ulaz{b}#
#\izlaz{Unesite drugi karakter:}#
#\ulaz{a}#
#\ulaz{bbojic}#
\end{upotreba}
\end{minitest}
\begin{minitest}
\begin{upotreba}{3}
#\naslovInt#
#\izlaz{Unesite nisku:}#
#\ulaz{patentara}#
#\izlaz{Unesite prvi karakter:}#
#\ulaz{t}#
#\izlaz{Unesite drugi karakter:}#
#\ulaz{a}#
#\ulaz{pttenttr}#
\end{upotreba}
\end{minitest}


\linkresenje{p2.3_}
\end{Exercise}
\begin{Answer}[ref=p2.3_]
%\includecode{resenja/2_PredstavljanjePodataka/2.3_Niske/1_08.c}
\end{Answer}




%\subsection{Teski}

\begin{Exercise}[label=p2.3_] 
Napisati funkciju
 \kckod{int prepis(char a[][21], int na, char b[][21])}
koja iz niza reči $a$ dužine $na$ prepisuje u niz $b$ one reči koje su sastavljene
samo od malih ili samo od velikih slova i vraća dužinu niza $b$.
Napisati program koji za učitani broj $n$
 ($0<n\leq50$) i $n$ reči razdvojenih blanko znakom ispisuje sve unete reči sastavljene samo od malih ili samo od velikih slova. Pretpostaviti da su unete reči maksimalne dužine 20 karaktera. U slučaju da je $n$ van dozvoljenog opsega, ispisati odgovarajuću poruku. 



\begin{minitest}
\begin{upotreba}{1}
#\naslovInt#
#\ulaz{3 abc ABC aBc}#
#\izlaz{abc ABC}#
\end{upotreba}
\end{minitest}
\begin{minitest}
\begin{upotreba}{2}
#\naslovInt#
#\ulaz{2 mnB RGa}#
#\izlaz{}#
\end{upotreba}
\end{minitest}
\begin{minitest}
\begin{upotreba}{3}
#\naslovInt#
#\ulaz{-3}#
#\izlaz{Nekorektan broj reci!}#
\end{upotreba}
\end{minitest}
%\begin{minitest}
%\begin{upotreba}{4}
%#\naslovInt#
%#\ulaz{4 2abc AVF\$ abc AV4}#
%#\izlaz{abc}#
%\end{upotreba}
%\end{minitest}

\linkresenje{p2.3_}
\end{Exercise}
\begin{Answer}[ref=p2.3_]
%\includecode{resenja/2_PredstavljanjePodataka/2.3_Niske/1_08.c}
\end{Answer}


\begin{Exercise}[label=p2.3_] 
Napisati program za rad sa brojevima zapisanim u različitim brojevnim sistemima.
\begin{enumerate}
\item Napisati funkciju \kckod{unsigned btoi(char s[], unsigned char b)} koja
određuje dekadnu vrednost zapisa datog neoznačenog broja $s$ u datoj
osnovi $b$. 
\item Napisati funkciju
\kckod{void itob(unsigned n, unsigned char b, char s[])} koja datu
dekadnu vrednost $n$ zapisuje u datoj osnovi $b$ i smešta
rezultat u nisku $s$. Pretpostaviti da je $0 < b \leq 16$.  
 \end{enumerate}
Napisati program koji za svaku učitanu liniju koja sadrže po jedan dekadni, oktalni ili
heksadekadni broj (zapisan kao što se zapisuju konstante u programskom
jeziku C) ispisuje odgovarajući binarni zapis. 
Linije se unose sve do kraja ulaza.
Pretpostaviti da će sve linije sadržati ispravne brojeve i da će ti brojevi biti u opsegu tipa
\kckod{unsigned}. 

\komentarJ{Pogledati resenje i prilagoditi tekst.}

\begin{miditest}
\begin{upotreba}{1}
#\naslovInt#
#\ulaz{0x49 0x1ABC}#
#\izlaz{1001001 1101010111100}#
\end{upotreba}
\end{miditest}
\begin{miditest}
\begin{upotreba}{2}
#\naslovInt#
#\ulaz{012 435 0x64FE}#
#\izlaz{1010 110110011 110010011111110}#
\end{upotreba}
\end{miditest}

\begin{miditest}
\begin{upotreba}{3}
#\naslovInt#
#\ulaz{123 0777}#
#\izlaz{1111011 111111111}#
\end{upotreba}
\end{miditest}
\begin{miditest}
\begin{upotreba}{4}
#\naslovInt#
#\ulaz{981}#
#\izlaz{1111010101}#
\end{upotreba}
\end{miditest}
\linkresenje{p2.3_}
\end{Exercise}
\begin{Answer}[ref=p2.3_]
%\includecode{resenja/2_PredstavljanjePodataka/2.3_Niske/1_08.c}
\end{Answer}




\subsection{Treba u drugu sekciju}

\begin{Exercise}[label=p2.3_] Napisati program za šifrovanje reči na različite načine.
\begin{enumerate}
 \item  Uvesti tip podataka \kckod{Sifra} kojim se opisuje način
   šifrovanja alfanumeričkih karaktera.  Svaka šifra se
   opisuje celobrojnom vrednošću $b$ koja određuje broj
   pozicija pomeranja, kao i karakterom L ili D koji
   određuje smer pomeranja (levo ili desno).
  \item Napisati funkciju \kckod{void sifruj(char rec[],Sifra s)}
    koja transformiše reč $rec$ po šifri
    $s$. Reč se šifruje tako što se svako slovo
    zamenjuje slovom za $b$ mesta levo ili desno od njega u
    abecedi, i to ciklično. Cifre se šifriraju na isti način.
 \item Napisati program koji učitava način šifrovanja u
   obliku $b_1 c_1 \ldots b_m c_m$ ($1\leq m \leq 20$), broj $n$ i $n$ re\v ci maksimalne dužine 20 karaktera i ispisuje šifrovane reči.

\komentarJ{Pogledati da li je duzina sifre ogranicena na 20 i da li se unosi nova sifra za svaku rec (bilo je neprecizno napisano). Dodati naredni primer u test primere. Zar ovo ne ide u strukture?}
Npr: za b=2,
i smer='D' : a se menja sa c, b sa d,..., x sa z,y sa a, z sa b, 1
sa 3, .. 8 sa 0, 9 sa 1
\end{enumerate}
\linkresenje{p2.3_}
\end{Exercise}
\begin{Answer}[ref=p2.3_]
%\includecode{resenja/2_PredstavljanjePodataka/2.3_Niske/1_08.c}
\end{Answer}



\subsection{Haoticni - za izbacivanje}


\begin{Exercise}[label=v2.3_07] 
   Napisati program koji učitava dve niske $s$ i $t$ i ako su jednake, izdaje odgovarajuću poruku a u suprotnom ispituje da li je niska $t$ podniska niske $t$ i ukoliko jeste, ispisuje počev od kog indeksa niske $s$ počinje prvo pojavljivanje niske $t$. Ako niska $t$ nije podniska niske $s$, ispisati odgovarajuću poruku. Možemo
   pretpostaviti da niske ne sadrže više od 20 karaktera.

   \komentarJ{konfuzija. Predlazem da izbacimo.}
\linkresenje{v2.3_07}
\end{Exercise}
\begin{Answer}[ref=v2.3_07]
\includecode{resenja/2_PredstavljanjePodataka/2.3_Niske/1_07.c}
\end{Answer}


\begin{Exercise}[label=p2.3_] 
Napisati funkciju
\kckod{void min\_razlika(char s[], char s1[], char s2[])} koja u niski $s$ pronalazi dve reči koje imaju minimalnu razliku između
svojih samoglasnika.  ( Reč je niz karaktera između dve praznine;
razmak između samoglasnika reči \kckod{danas} i \kckod{jutro} je 2,
a razmak izmedju \kckod{sutrk} i \kckod{mnozenje} je 5). Napisati program koji testira napisanu funkciju za unete niske maksimalne dužine 20 karaktera.

\komentarJ{Konfuzno. Sta je razlika/razmak? Predlazem da izbacimo.}
\linkresenje{p2.3_}
\end{Exercise}
\begin{Answer}[ref=p2.3_]
%\includecode{resenja/2_PredstavljanjePodataka/2.3_Niske/1_08.c}
\end{Answer}


\begin{Exercise}[label=p2.3_] 
Napisati funkciju \kckod{int pp(char s[], char t[])} koja vraća poziciju pojavljivanja poslednjeg karaktera niske $s$ u niski $t$, zanemarujući pritom razliku između velikih i
malih slova, ili -1 ako takvog karaktera nema. Napisati program koji učitava dve niske maksimalne dužine 20 karaktera i testira napisanu funkciju.

\komentarJ{Konfuzno. Predlazem da izbacimo.}

\begin{miditest}
\begin{upotreba}{1}
#\naslovInt#
#\ulaz{a4BA3Bc A3b}#
#\izlaz{5}#
\end{upotreba}
\end{miditest}
\linkresenje{p2.3_}
\end{Exercise}
\begin{Answer}[ref=p2.3_]
%\includecode{resenja/2_PredstavljanjePodataka/2.3_Niske/1_08.c}
\end{Answer}





\begin{Exercise}[label=p2.3_] 
Napisati funkciju \kckod{void sifrat(char rec[], char kljuc[])} koja šifruje $rec$ na sledeći način: za svako slovo reči
$rec$ i odgovarajuće slovo reči $kljuc$ određuje koliki je
(alfabetski) razmak između njih ($k$) i potom  $k$-to slovo reči $rec$ zamenjuje $k$-tim slovom alfabeta. Podrazumeva se da je $kljuc$ duži od reci. 

\komentarJ{konfuzija. Predlazem da izbacimo}

\begin{miditest}
\begin{upotreba}{1}
#\naslovInt#
#\ulaz{bac}#
#\ulaz{dfge}#
#\izlaz{bed}#
\end{upotreba}
\end{miditest}
\linkresenje{p2.3_}
\end{Exercise}
\begin{Answer}[ref=p2.3_]
%\includecode{resenja/2_PredstavljanjePodataka/2.3_Niske/1_08.c}
\end{Answer}

\begin{comment}

\end{comment}

%\section{Rešenja}
%\shipoutAnswer

