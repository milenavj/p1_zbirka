
\section{Niske}



\begin{Exercise}[label=v2.3_01] 
   Napisati funkciju koja konvertuje dati string tako sto 
   mala slova menja u velika a velika u mala. Napisati 
   potom glavni program koji ucitava string, poziva napisanu 
   funkciju i ispisuje konvertovani string. Mozemo pretpostaviti
   da string ne sadrzi vise od 10 karaktera.
\linkresenje{v2.3_01}
\end{Exercise}
\begin{Answer}[ref=v2.3_01]
\includecode{resenja/2_PredstavljanjePodataka/2.3_Niske/1_01.c}
\end{Answer}

\begin{Exercise}[label=v2.3_02] 
   Napisati funkciju skrati koja uklanja beline sa
   kraja datog stringa. Napisati program koji testira napisanu
   funkciju na stringu\\ \verb|"rep belina          "|.
\linkresenje{v2.3_02}
\end{Exercise}
\begin{Answer}[ref=v2.3_02]
\includecode{resenja/2_PredstavljanjePodataka/2.3_Niske/1_02.c}
\end{Answer}

\begin{Exercise}[label=v2.3_03] 
   Napisati program koji ucitava string src i formira string dst
   trostrukim nadovezivanjem stringa src. Program treba da ispise
   string dst. Na primer, za uneti string ''dan'', string dst treba
   da bude ''dandandan''. Pretpostaviti da string src nije duzi od
   30 karaktera.
\linkresenje{v2.3_03}
\end{Exercise}
\begin{Answer}[ref=v2.3_03]
\includecode{resenja/2_PredstavljanjePodataka/2.3_Niske/1_03.c}
\end{Answer}

\begin{Exercise}[label=v2.3_04] 
   Napisati funkciju int ucitaj\_liniju(char s[], int n)
   koja ucitava liniju maksimalne duzine n u string s
   i vraca duzinu ucitane linije. Linija moze da sadrzi
   blanko znakove ali ne moze da sadrzi \verb"\n" ili EOF. 
   
   Napisati potom glavni program koji ucitava linije
   do EOF i ispisuje najduzu liniju i njenu duzinu. Ukoliko
   ima vise linija maksimalne duzine, ispisati prvu. Mozemo
   pretpostviti da svaka linija sadrzi najvise 80 karaktera,
   zajedno sa \verb"\n".
\linkresenje{v2.3_04}
\end{Exercise}
\begin{Answer}[ref=v2.3_04]
\includecode{resenja/2_PredstavljanjePodataka/2.3_Niske/1_04.c}
\end{Answer}

\begin{Exercise}[label=v2.3_05] 
  Napisati program koji pretvara nisku u ceo broj.
  Npr. za ulaz ''-1238'' se generise rezultat -1238
  Pogledati funkcije atoi i atof koje postoje u biblioteci stdlib.h
\linkresenje{v2.3_05}
\end{Exercise}
\begin{Answer}[ref=v2.3_05]
\includecode{resenja/2_PredstavljanjePodataka/2.3_Niske/1_05.c}
\end{Answer}

\begin{Exercise}[label=v2.3_06] 
	Napisati program koji pretvara zadatu broj u nisku. 
	Npr. za broj -453 treba generisati nisku ''-453''
\linkresenje{v2.3_06}
\end{Exercise}
\begin{Answer}[ref=v2.3_06]
\includecode{resenja/2_PredstavljanjePodataka/2.3_Niske/1_06.c}
\end{Answer}

\begin{Exercise}[label=v2.3_07] 
   Napisati program koji ucitava dva stringa i ispituje najpre da li su jednaki. Ako jesu, program
   treba da izda odgovarajucu poruku, a ako nisu, treba da ispita da li je drugi podstring 
   prvog. Ukoliko jeste, program treba da ispise pocev od kog indeksa prvog
   stringa pocinje drugi string. U suprotnom, ispisati odgovarajucu poruku. Mozemo
   pretpostaviti da stringovi ne sadrze vise od 20 karaktera.
\linkresenje{v2.3_07}
\end{Exercise}
\begin{Answer}[ref=v2.3_07]
\includecode{resenja/2_PredstavljanjePodataka/2.3_Niske/1_07.c}
\end{Answer}

\begin{Exercise}[label=v2.3_08] 
   Napisati program koji za uneti string s i karakter c utvrdjuje
   da li se c pojavljuje u stringu s i ukoliko se pojavljuje,
   ispisuje indeks prvog pojavljivanja a u suprotnom ispisuje
   odgovarajucu poruku. Mozemo pretpostaviti da string ima najvise
   20 karaktera.
\linkresenje{v2.3_08}
\end{Exercise}
\begin{Answer}[ref=v2.3_08]
\includecode{resenja/2_PredstavljanjePodataka/2.3_Niske/1_08.c}
\end{Answer}

\begin{Exercise}[label=p2.3_01] 
\begin{itemize}
\item [a)] Napisati funkciju $int\ samoglasnik(char\ c)$ koja proverava da li je zadati karakter samoglasnik. Funkcija treba da vrati vrednost 1 ako karakter $c$ jeste samoglasnik, odnosno 0 ako nije. 
\item [b)] Napisati funkciju $int\ samoglasnik\_na\_kraju(char\ s[])$ koja proverava da li se niska $s$ završava samoglasnikom (koristiti funkciju iz tačke a)). 
\item [c)] Napisati program koji učitava nisku maksimalne dužine 20 karaktera i ispisuje da li završava samoglasnikom ili ne. 
\end{itemize}
\begin{miditest}
\begin{upotreba}{1}
#\naslovInt#
#\izlaz{Unesite nisku:}\ulaz{abcde}#
#\izlaz{Niska se zavrsava samoglasnikom!}#
\end{upotreba}
\end{miditest}
\begin{miditest}
\begin{upotreba}{2}
#\naslovInt#
#\izlaz{Unesite nisku:}\ulaz{AaBb+cCdD}#
#\izlaz{Niska se ne zavrsava samoglasnikom!}#
\end{upotreba}
\end{miditest}
\begin{miditest}
\begin{upotreba}{3}
#\naslovInt#
#\izlaz{Unesite nisku:}\ulaz{pRograMiranjE}#
#\izlaz{Niska se zavrsava samoglasnikom!}#
\end{upotreba}
\end{miditest}
\linkresenje{p2.3_01}
\end{Exercise}
\begin{Answer}[ref=p2.3_01]
\includecode{resenja/2_PredstavljanjePodataka/2.3_Niske/praktikumi11/7.c}
\end{Answer}

\begin{Exercise}[label=p2.3_02] 
Napisati funkciju $void\ kopiraj\_n(char\ t[],\ char\ s[],\ int\ n)$ koja kopira najviše $n$ karaktera niske $s$ u nisku $t$. Napisati i program koji učitava nisku maksimalne dužine 20 karaktera i jedan ceo broj i testira rad napisane funkcije.\\
\begin{miditest}
\begin{upotreba}{1}
#\naslovInt#
#\izlaz{Unesite nisku:}\ulaz{abcdef}#
#\izlaz{Unesite broj n:}\ulaz{3}#
#\izlaz{Rezultujuca niska: abc}#
\end{upotreba}
\end{miditest}
\begin{miditest}
\begin{upotreba}{2}
#\naslovInt#
#\izlaz{Unesite nisku:}\ulaz{programiranje}#
#\izlaz{Unesite broj n:}\ulaz{5}#
#\izlaz{Rezultujuca niska: progr}#
\end{upotreba}
\end{miditest}
\begin{miditest}
\begin{upotreba}{3}
#\naslovInt#
#\izlaz{Unesite nisku:}\ulaz{abc}#
#\izlaz{Unesite broj n:}\ulaz{15}#
#\izlaz{Rezultujuca niska: abc}#
\end{upotreba}
\end{miditest}


\linkresenje{p2.3_02}
\end{Exercise}
\begin{Answer}[ref=p2.3_02]
\includecode{resenja/2_PredstavljanjePodataka/2.3_Niske/praktikumi11/8.c}
\end{Answer}

\begin{Exercise}[label=p2.3_03] 
 Napisati funkciju $void\ dupliranje(char\ t[],\ char\ s[])$ koja na osnovu niske $s$ formira nisku $t$ tako što duplira svaki karakter niske $s$. Napisati i program koji učitava nisku maksimalne dužine 20 karaktera i testira rad napisane funkcije.\\
\begin{miditest}
\begin{upotreba}{1}
#\naslovInt#
#\izlaz{Unesite nisku:}\ulaz{zima}#
#\izlaz{zziimmaa}#
\end{upotreba}
\end{miditest}
\begin{miditest}
\begin{upotreba}{2}
#\naslovInt#
#\izlaz{Unesite nisku:}\ulaz{A+B+C}#
#\izlaz{AA++BB++CC}#
\end{upotreba}
\end{miditest}
\begin{miditest}
\begin{upotreba}{3}
#\naslovInt#
#\izlaz{Unesite nisku:}\ulaz{C}#
#\izlaz{CC}#
\end{upotreba}
\end{miditest}
 

\linkresenje{p2.3_03}
\end{Exercise}
\begin{Answer}[ref=p2.3_03]
\includecode{resenja/2_PredstavljanjePodataka/2.3_Niske/praktikumi11/9.c}
\end{Answer}

\begin{Exercise}[label=p2.3_04] 
 Napisati funkciju $int\ heksa\_broj(char\ s[])$ koja proverava da li je niskom $s$ zadat korektan heksadekadni broj. Heksadekadni broj je korektno zadat ako počinje prefiksom $0x$ ili $0X$ i ako sadrži samo cifre i mala ili velika slova $A$, $B$, $C$, $D$, $E$ i $F$. Funkcija treba da vrati vrednost 1 ako je niska korektan heksadekadni broj, odnosno 0 ako nije. Napisati i program koji učitava nisku maksimalne dužine 7 karaktera i ispisuje rezultat rada funkcije. \\
\begin{miditest}
\begin{upotreba}{1}
#\naslovInt#
#\izlaz{Unesite nisku:}\ulaz{0x12EF}#
#\izlaz{Korektan heksadekadni broj!}#
\end{upotreba}
\end{miditest}
\begin{miditest}
\begin{upotreba}{2}
#\naslovInt#
#\izlaz{Unesite nisku:}\ulaz{0X22af}#
#\izlaz{Korektan heksadekadni broj!}#
\end{upotreba}
\end{miditest}
\begin{miditest}
\begin{upotreba}{3}
#\naslovInt#
#\izlaz{Unesite nisku:}\ulaz{0xErA9}#
#\izlaz{Nekorektan heksadekadni broj!}#
\end{upotreba}
\end{miditest}

\linkresenje{p2.3_04}
\end{Exercise}
\begin{Answer}[ref=p2.3_04]
\includecode{resenja/2_PredstavljanjePodataka/2.3_Niske/praktikumi11/10.c}
\end{Answer}

\begin{Exercise}[label=p2.3_05] 
 Napisati funkciju $int\ heksa\_broj(char\ s[])$ koja izračunava dekadnu vrednost heksadekadnog broja zadatog niskom $s$. Napisati i program koji učitava nisku maksimalne dužine 7 karaktera i ispisuje rezultat rada funkcije. Pretpostaviti da je uneta niska korektan heksadekadni broj. \\
\begin{miditest}
\begin{upotreba}{1}
#\naslovInt#
#\izlaz{Unesite nisku:}\ulaz{0x2A34}#
#\izlaz{10804}#
\end{upotreba}
\end{miditest}
\begin{miditest}
\begin{upotreba}{2}
#\naslovInt#
#\izlaz{Unesite nisku:}\ulaz{0Xff2}#
#\izlaz{4082}#
\end{upotreba}
\end{miditest}
\begin{miditest}
\begin{upotreba}{3}
#\naslovInt#
#\izlaz{Unesite nisku:}\ulaz{0xE1A9}#
#\izlaz{57769}#
\end{upotreba}
\end{miditest}

\linkresenje{p2.3_05}
\end{Exercise}
\begin{Answer}[ref=p2.3_05]
\includecode{resenja/2_PredstavljanjePodataka/2.3_Niske/praktikumi11/11.c}
\end{Answer}

\begin{Exercise}[label=p2.3_06] 
 Napisati funkciju $int\ podniska(char\ s[], char\ t[])$ koja proverava da li je niska $t$ podniska niske $s$. Napisati i program koji učitava dve niske maksimalne dužine 10 karaktera i testira rad napisane funkcije.\\
\begin{miditest}
\begin{upotreba}{1}
#\naslovInt#
#\izlaz{Unesite nisku s:}\ulaz{abcde}#
#\izlaz{Unesite nisku t:}\ulaz{bcd}#
#\izlaz{t je podniska niske s!}#
\end{upotreba}
\end{miditest}
\begin{miditest}
\begin{upotreba}{2}
#\naslovInt#
#\izlaz{Unesite nisku s:}\ulaz{abcde}#
#\izlaz{Unesite nisku t:}\ulaz{bCd}#
#\izlaz{t nije podniska niske s!}#
\end{upotreba}
\end{miditest}
\begin{miditest}
\begin{upotreba}{3}
#\naslovInt#
#\izlaz{Unesite nisku s:}\ulaz{abcde}#
#\izlaz{Unesite nisku t:}\ulaz{def}#
#\izlaz{t nije podniska niske s!}#
\end{upotreba}
\end{miditest}

\linkresenje{p2.3_06}
\end{Exercise}
\begin{Answer}[ref=p2.3_06]
\includecode{resenja/2_PredstavljanjePodataka/2.3_Niske/praktikumi11/12.c}
\end{Answer}

\begin{Exercise}[label=p2.3_07] 
 Napisati funkciju $void\ modifikacija(char*\ s)$ koja modifikuje nisku $s$ tako što svaki drugi karakter zameni zvezdicom. Pretpostaviti da niska $s$ neće biti duža od 20 karaktera. Napisati i program koji testira rad napisane funkcije. \\
\begin{miditest}
\begin{upotreba}{1}
#\naslovInt#
#\izlaz{Unesite nisku:}\ulaz{123abc789XY}#
#\izlaz{Modifikovana niska je: 1*3*b*7*9*Y}#
\end{upotreba}
\end{miditest}
\begin{miditest}
\begin{upotreba}{2}
#\naslovInt#
#\izlaz{Unesite nisku:}\ulaz{zimA}#
#\izlaz{Modifikovana niska je: z*m*}#
\end{upotreba}
\end{miditest}
\begin{miditest}
\begin{upotreba}{3}
#\naslovInt#
#\izlaz{Unesite nisku:}\ulaz{SNEG}#
#\izlaz{Modifikovana niska je: S*E*}#
\end{upotreba}
\end{miditest}

\linkresenje{p2.3_07}
\end{Exercise}
\begin{Answer}[ref=p2.3_07]
%\includecode{resenja/2_PredstavljanjePodataka/2.3_Niske/1_08.c}
\end{Answer}

\begin{Exercise}[label=p2.3_08] 
 Napisati funkciju $int\ strspn\_klon(char*\ t,\ char*\ s)$ koja izračunava dužinu prefiksa niske $t$ sastavljenog od karaktera niske $s$. Napisati zatim i program koji učitava dve niske maksimalne dužine 20 karaktera i ispisuje rezultat poziva napisane funkcije. \\
\begin{miditest}
\begin{upotreba}{1}
#\naslovInt#
#\izlaz{Unesite nisku t:}\ulaz{programiranje}#
#\izlaz{Unesite nisku s:}\ulaz{opqr}#
#\izlaz{3}#
\end{upotreba}
\end{miditest}
\begin{miditest}
\begin{upotreba}{2}
#\naslovInt#
#\izlaz{Unesite nisku t:}\ulaz{aaiioo124}#
#\izlaz{Unesite nisku s:}\ulaz{aeiou}#
#\izlaz{6}#
\end{upotreba}
\end{miditest}
\begin{miditest}
\begin{upotreba}{3}
#\naslovInt#
#\izlaz{Unesite nisku t:}\ulaz{5296abc}#
#\izlaz{Unesite nisku s:}\ulaz{0123456789}#
#\izlaz{4}#
\end{upotreba}
\end{miditest}

\linkresenje{p2.3_08}
\end{Exercise}
\begin{Answer}[ref=p2.3_08]
%\includecode{resenja/2_PredstavljanjePodataka/2.3_Niske/1_08.c}
\end{Answer}

\begin{Exercise}[label=p2.3_09] 
 Napisati implementaciju funkcije $char*\ strchr_klon(char*\ s,\ char\ c)$ koja vraća pokazivač na prvo pojavljivanje karaktera $c$ u niski $s$ ili NULL ukoliko se karakter $c$ ne pojavljuje u niski $s$. Učitati potom jednu nisku maksimalne dužine 20 karaktera i jedan dodatni karakter i testirati rad napisane funkcije. \\
\begin{miditest}
\begin{upotreba}{1}
#\naslovInt#
#\izlaz{Unesite nisku s:}\ulaz{programiranje}#
#\izlaz{Unesite karakter c:}\ulaz{a}#
#\izlaz{Karakter se nalazi u niski!}#
\end{upotreba}
\end{miditest}
\begin{miditest}
\begin{upotreba}{2}
#\naslovInt#
#\izlaz{Unesite nisku s:}\ulaz{123456789}#
#\izlaz{Unesite karakter c:}\ulaz{y}#
#\izlaz{Karakter se ne nalazi u niski!}#
\end{upotreba}
\end{miditest}
\linkresenje{p2.3_09}
\end{Exercise}
\begin{Answer}[ref=p2.3_09]
%\includecode{resenja/2_PredstavljanjePodataka/2.3_Niske/1_08.c}
\end{Answer}

\begin{Exercise}[label=p2.3_] 
\begin{itemize}
\item Napisati funkciju
\begin{center}
 \verb|int prepis(char a[][21], int na, char b[][21])|
\end{center}
\noindent koja iz niza re\v{c}i \verb|a| du\v zine \verb|na| prepisuje u niz \verb|b| re\v ci koje su zapisane
samo malim ili samo velikim slovima. Informaciju o du\v zini niza \verb|b| (broj re\v{c}i koje zadovoljavaju prethodni uslov)
vratiti kao povratnu vrednost funkcije.\\

\item 
Napisati program koji sa standardnog ulaza u\v citava prvo broj
re\v{c}i (strogo ve\'ci od nule, manji od 50), a zatim i same
re\v{c}i razdvojene blanko znakom (smatrati da re\v{c}i koje se
unose sa ulaza ne\'ce biti du\v{z}e od 20 karaktera - ovaj uslov
ne proveravati). Za slu\v{c}aj kada je broj re\v{c}i izvan
tra\v{z}enog opsega ispisati -1 i prekinuti izvr\v{s}avanje
programa. Kori\v{s}\'cenjem prethodno definisane funkcije
\verb|prepis|, odrediti sve re\v{c}i koje su zapisane samo malim
ili samo velikim slovima. Rezultat ispisati na standardni izlaz.
Napomena: Ukoliko se pri re\v{s}avanju zadatka ne bude koristila
funkcija \verb|prepis|, zadatak ne\'ce biti pregledan i nosi\'ce
nula poena.
\end{itemize}
\begin{miditest}
\begin{upotreba}{1}
#\naslovInt#
#\ulaz{3 abc ABC aBc}#
#\izlaz{abc ABC}#
\end{upotreba}
\end{miditest}
\begin{miditest}
\begin{upotreba}{2}
#\naslovInt#
#\ulaz{2 mnB RGa}#
#\izlaz{}#
\end{upotreba}
\end{miditest}
\begin{miditest}
\begin{upotreba}{3}
#\naslovInt#
#\ulaz{-3}#
#\izlaz{-1}#
\end{upotreba}
\end{miditest}
\begin{miditest}
\begin{upotreba}{4}
#\naslovInt#
#\ulaz{4 2abc AVF\$ abc AV4}#
#\izlaz{abc}#
\end{upotreba}
\end{miditest}
\linkresenje{p2.3_}
\end{Exercise}
\begin{Answer}[ref=p2.3_]
%\includecode{resenja/2_PredstavljanjePodataka/2.3_Niske/1_08.c}
\end{Answer}

\begin{Exercise}[label=p2.3_] 
Napisati funkciju
\verb|void min_razlika(char s[], char s1[], char s2[])| koja u datotoj
nisci s pronalazi dve re\v ci koje imaju minimalnu razliku izme\d u
svojih samoglasnika.  ( Re\v c je niz karaktera izme\d u dve praznine;
razmak izme\d u samoglasnika re\v ci \verb|danas| i \verb|jutro| je 2,
a razmak izmedju \verb|sutrk| i \verb|mnozenje| je 5).  Testirati
pozivom u main-u.  Maksimalna du\v zina niske je 20 karaktera.
\linkresenje{p2.3_}
\end{Exercise}
\begin{Answer}[ref=p2.3_]
%\includecode{resenja/2_PredstavljanjePodataka/2.3_Niske/1_08.c}
\end{Answer}

\begin{Exercise}[label=p2.3_] 
Napisati funkciju \verb|int pp(char s[], char t[])| koja odre\d uje
poziciju poslednjeg karaktera niske \verb|s| sadr\v zanog u okviru
niske \verb|t|, zanemaruju\' ci pri tom razliku izme\d u velikih i
malih slova, ili -1 ako takvog karaktera nema.  Testirati pozivom u
main-u.  Maksimalna du\v zina niske je 20 karaktera.
\begin{miditest}
\begin{upotreba}{1}
#\naslovInt#
#\ulaz{a4BA3Bc A3b}#
#\izlaz{5}#
\end{upotreba}
\end{miditest}
\linkresenje{p2.3_}
\end{Exercise}
\begin{Answer}[ref=p2.3_]
%\includecode{resenja/2_PredstavljanjePodataka/2.3_Niske/1_08.c}
\end{Answer}

\begin{Exercise}[label=p2.3_] 
Napisati funkciju \verb|int f1(char s[])| koja prihvata tu nisku i
proverava da li niska sadr\v{z}i veliko slovo. Funkcija vra\'ca 1 ako
sadr\v zi veliko slovo, ina\v ce 0.  Testirati pozivom u main-u.
Maksimalna du\v zina niske je 20 karaktera.
\linkresenje{p2.3_}
\end{Exercise}
\begin{Answer}[ref=p2.3_]
%\includecode{resenja/2_PredstavljanjePodataka/2.3_Niske/1_08.c}
\end{Answer}

\begin{Exercise}[label=p2.3_] 
Napisati funkciju \verb|void ukloniSlova(char s[])| koja iz niske
\verb|s| uklanja sva mala i velika slova.  Testirati pozivom u main-u.
Maksimalna du\v zina niske je 20 karaktera.  \linkresenje{p2.3_}
\end{Exercise}
\begin{Answer}[ref=p2.3_]
%\includecode{resenja/2_PredstavljanjePodataka/2.3_Niske/1_08.c}
\end{Answer}


\begin{Exercise}[label=p2.3_] 
Napisati funkciju \verb|unsigned btoi(char* s, unsigned char b)| koja
odre\d uje vrednost zapisa datog neozna\v cenog broja \verb|s| u datoj
osnovi \verb|b|. Napisati funkciju
\verb|void itob(unsigned n, unsigned char b, char* s)| koja datu
vrednost \verb|n| zapisuje u datoj osnovi \verb|b| i sme\v sta
rezultat u nisku \verb|s|. Napisati zatim program koji \v cita liniju
po liniju sa standardnog ulaza i obra\d uje ih sve dok ne nai\d e na
praznu liniju. Svaka linija sadr\v zi jedan dekadni, oktalni ili
heksadekadni broj (zapisan kako se zapisuju konstante u programskom
jeziku C). Program za svaki uneti broj ispisuje njegov binarni
zapis. Pretpostaviti da \' ce svi uneti brojevi biti u opsegu tipa
\verb|unsigned|. \\
\begin{miditest}
\begin{upotreba}{1}
#\naslovInt#
#\ulaz{0x49 0x1ABC}#
#\izlaz{1001001 1101010111100}#
\end{upotreba}
\end{miditest}
\begin{miditest}
\begin{upotreba}{2}
#\naslovInt#
#\ulaz{012 435 0x64FE}#
#\izlaz{1010 110110011 110010011111110}#
\end{upotreba}
\end{miditest}
\begin{miditest}
\begin{upotreba}{3}
#\naslovInt#
#\ulaz{123 0777}#
#\izlaz{1111011 111111111}#
\end{upotreba}
\end{miditest}
\begin{miditest}
\begin{upotreba}{4}
#\naslovInt#
#\ulaz{981}#
#\izlaz{1111010101}#
\end{upotreba}
\end{miditest}
\linkresenje{p2.3_}
\end{Exercise}
\begin{Answer}[ref=p2.3_]
%\includecode{resenja/2_PredstavljanjePodataka/2.3_Niske/1_08.c}
\end{Answer}

\begin{Exercise}[label=p2.3_] 
Implementirati funkciju int \verb|str_str(char s[], char t[])| koja
proverava da li niska \verb|s| sadrzi nisku \verb|t|. Zatim napisati
program koji sa standardnog ulaza u\v{c}itava pet redova (svaki red
ima najvise 100 karaktera) i koji ispisuje sve redne brojeve linija
koje sadr\v ze nisku \verb|program| (linije se numeri\v{s}u od broja
1). Ukoliko ne postoji red sa niskom \verb|program| ispisati -1. \\
\begin{miditest}
\begin{upotreba}{1}
#\naslovInt#
#\ulaz{novi red*nprogram}#
#\ulaz{c prog. jezik}#
#\ulaz{c? programskih jezik}#
#\ulaz{Programski odbor}#
#\ulaz{<b>program</b>}#
#\izlaz{1 3 5}#
\end{upotreba}
\end{miditest}
\linkresenje{p2.3_}
\end{Exercise}
\begin{Answer}[ref=p2.3_]
%\includecode{resenja/2_PredstavljanjePodataka/2.3_Niske/1_08.c}
\end{Answer}

\begin{Exercise}[label=p2.3_] 
Napisati funkciju \verb|void sifrat(char* rec, char* kljuc)| koja \v
sifruje \verb|rec| na slede\' ci na\v cin: za svako slovo re\v ci
\verb|rec| i odgovaraju\' ce slovo \verb|kljuca| određuje koliki je
(alfabetski) razmak između njih i ozna\v cimo taj broj sa \verb|k|.
Potom to slovo \verb|reci| zamenjuje \verb|k-tim| slovom alfabeta.
Podrazumeva se da je \verb|kljuc| du\v zi od reci. \\ 
\begin{miditest}
\begin{upotreba}{1}
#\naslovInt#
#\ulaz{bac}#
#\ulaz{dfge}#
#\izlaz{bed}#
\end{upotreba}
\end{miditest}
\linkresenje{p2.3_}
\end{Exercise}
\begin{Answer}[ref=p2.3_]
%\includecode{resenja/2_PredstavljanjePodataka/2.3_Niske/1_08.c}
\end{Answer}


\begin{Exercise}[label=p2.3_] 
Napisati funkciju \verb|void obrni(char rec[], int k)| koja rotira
re\v c za k mesta ulevo.
\begin{miditest}
\begin{upotreba}{1}
#\naslovInt#
#\ulaz{sveska}#
#\ulaz{2}#
#\izlaz{eskasv}#
\end{upotreba}
\end{miditest}
\linkresenje{p2.3_}
\end{Exercise}
\begin{Answer}[ref=p2.3_]
%\includecode{resenja/2_PredstavljanjePodataka/2.3_Niske/1_08.c}
\end{Answer}


\begin{Exercise}[label=p2.3_] 
 Napisati sledece funkcije:

\verb|int poredjenje(char* s1, char* s2);|

\verb|// vraca 1 ako su s1 i s2 iste niske, 0 u suprotnom|
\vspace*{2mm}

\verb|void uVelikaSlova(char* s);|

\verb|// pretvara sva slova niske s u velika, ostale znakove ne menja|

Napisati program koji sa standardnog ulaza u\v{c}itava dve re\v{c}i (du\v{z}ine
najvi\v{s}se 20 znakova) i, koriste\'ci ove dve funkcije, ispisuje da li su
one jednake ako se sva slova pretvore u velika slova. \\
\begin{miditest}
\begin{upotreba}{1}
#\naslovInt#
#\ulaz{isPit2010}#
#\ulaz{IsPiT2010}#
#\izlaz{jesu jednake}#
\end{upotreba}
\end{miditest}
\linkresenje{p2.3_}
\end{Exercise}
\begin{Answer}[ref=p2.3_]
%\includecode{resenja/2_PredstavljanjePodataka/2.3_Niske/1_08.c}
\end{Answer}


\begin{Exercise}[label=p2.3_] 
Napisati program kojim se sadr\v zaj unetog stirnga \v sifrira tako \v
sto se svako slovo zamenjuje slede\'cim ASCII slovom, a znakovi ’z' i
'Z' zamenjuju redom sa 'a' i ’A’.  Uneta re\v c nije du\v za od 20
karaktera.
\linkresenje{p2.3_}
\end{Exercise}
\begin{Answer}[ref=p2.3_]
%\includecode{resenja/2_PredstavljanjePodataka/2.3_Niske/1_08.c}
\end{Answer}


\begin{Exercise}[label=p2.3_] 
Napisati funkciju \verb|void sifruj(char rec[], char sifra[])|
koja na osnovu date re\v ci formira \v sifru koja se dobija tako \v sto se
svako slovo u re\v ci zameni sa naredna tri slova koja su mu susedna u abecedi. 
Na primer, re\v c ”tamo” treba da bude zamenjena sa ”uvwbcdnoppqr” a re\v c ”zec”
sa ”abcfghdef”. \\
Napisati program koji \v sifruje unetu re\v c sa standardnog ulaza i \v stampa
dobijeni rezultat na standardni izlaz. Za re\v c pretpostaviti da nije 
du\v za od 20 karaktera. Unos re\v ci ostvariti koristeci specifikator ”%s”.
\linkresenje{p2.3_}
\end{Exercise}
\begin{Answer}[ref=p2.3_]
%\includecode{resenja/2_PredstavljanjePodataka/2.3_Niske/1_08.c}
\end{Answer}


\begin{Exercise}[label=p2.3_] 
Sa ulaza se unosi re\v c koja nije du\v za od 20 znakova. Napisati program
koji formira i \v stampa rezultuju\'cu re\v c koja se dobija tako \v sto se 
uneta re\v c kopira 4 puta pri \v cemu se izme\d u svakog kopiranja ume\'ce 
crtica. Na primer ako je uneta re\v c \verb|ana|,formirana re\v c treba da 
bude \verb|ana-ana-ana-ana|.\\
Zadatak uraditi:
\begin{description}
\item[(a)] pisanjem odgovaraju\'ce funkcije koja vr\v si nadovezivanje re\v ci,
\item[(b)] koriste\'ci postoje\'cu funkciju iz biblioteke string.h (\verb|strcat|).
\end{description}
Napomena: voditi ra\v cuna da se za rezultuju\'cu re\v c odvoji odgovaraju\'ca 
koli\v cina memorije.
\linkresenje{p2.3_}
\end{Exercise}
\begin{Answer}[ref=p2.3_]
%\includecode{resenja/2_PredstavljanjePodataka/2.3_Niske/1_08.c}
\end{Answer}

\begin{Exercise}[label=p2.3_] 
Sa ulaza se unosi re\v c koja nije du\v za od 20 znakova. Napisati
funkciju koja svako pojavljivanje znaka koji se zadaje kao prvi
argument funkcije udvaja a svako po- javljivanje znaka koji se zadaje
kao drugi argument funkcije izbacuje. Napisati program koji poziva ovu
funkciju za re\v c unetu sa standardnog ulaza i za znakove koji se
tako\d e zadaju sa standardnog ulaza. Na primer, ako se unese re\v c
\verb|ana| i znakovi \verb|a| i \verb|n|, tada funkcija treba da
izmeni re\v c tako da ona postane \verb|aaaa|, ako se unese re\v c
\verb|abrakadabra| i znakovi \verb|a| i \verb|b|, tada funkcija treba
re\v c da izmeni tako da ona postane
\verb|aaraakaadkkraa|.\\ Napomena: voditi ra\v cuna da novonastala
izmenjena re\v c mo\v ze imati ve\'ci broj karaktera i u skladu sa tim
rezervisati odgovaraju\'cu koli\v cinu memorije. Dopu\v steno je
koristiti pomo\' can niz. \\
\linkresenje{p2.3_}
\end{Exercise}
\begin{Answer}[ref=p2.3_]
%\includecode{resenja/2_PredstavljanjePodataka/2.3_Niske/1_08.c}
\end{Answer}


\begin{Exercise}[label=p2.3_] 
 Napisati funkciju \verb|void ukloni(char *s);| koja iz niske uklanja
 sva slova iza kojih neposredno sledi slovo koje je u abecedi nakon
 njih (veli\v cina slova se zanemaruje). Testirati funkciju u programu
 koji u\v citava liniju teksta (najvi\v se 100 karaktera). \\
\begin{miditest}
\begin{upotreba}{1}
#\naslovInt#
#\ulaz{zdRaVo svIma}#
#\izlaz{zRVo vma}#
\end{upotreba}
\end{miditest}
\begin{miditest}
\begin{upotreba}{2}
#\naslovInt#
#\ulaz{12345AbcD}#
#\izlaz{12345D}#
\end{upotreba}
\end{miditest}
\begin{miditest}
\begin{upotreba}{3}
#\naslovInt#
#\ulaz{JeD1aN D52Va.}#
#\izlaz{JeD1N D52Va.}#
\end{upotreba}
\end{miditest}
\begin{miditest}
\begin{upotreba}{4}
#\naslovInt#
#\ulaz{abcd efg}#
#\izlaz{d g}#
\end{upotreba}
\end{miditest}
 \linkresenje{p2.3_}
\end{Exercise}
\begin{Answer}[ref=p2.3_]
%\includecode{resenja/2_PredstavljanjePodataka/2.3_Niske/1_08.c}
\end{Answer}


\begin{Exercise}[label=p2.3_] 
\begin{description}
\item{a)} Napisati C funkciju
  \verb|int procitaj_recenicu(char *s, int max_len)|, koja sa
  standarnog ulaza \v{c}ita re\v{c}enicu i sme\v{s}ta je u nisku
  \verb|s|. \v{C}itanje re\v{c}enice se zaustavlja ako se pro\v{c}ita
  simbol \verb|.| ili je ve\' c u\v{c}itano \verb|max_len-1|
  karaktera. Funkcija treba da vrati broj pro\v{c}itanih karaktera.

\item{b)} Napisati C funkciju
  \verb|void prebroj(char *s, int *broj_malih, int *broj_velikih)|,
  koja za zadatu nisku \verb|s| ra\v{c}una broj malih i velikih slova
  koji se u njoj pojavljuju.

\item{c)} Napisati glavni program koji sa standardnog ulaza \v{c}ita
  re\v{c}enice i na standardni izlaz ispisuje onu kod koje je razlika
  broja malih i velikih slova najve\'ca.
\end{description}
 \linkresenje{p2.3_}
\end{Exercise}
\begin{Answer}[ref=p2.3_]
%\includecode{resenja/2_PredstavljanjePodataka/2.3_Niske/1_08.c}
\end{Answer}


\begin{Exercise}[label=p2.3_] 
\begin{description}
 \item {a)} Uvesti tip podataka \verb|Sifra| kojim se opisuje na\v cin
   \v sifrovanja alfanumeri\v ckih karaktera.  Svaka \v sifra se
   opisuje celobrojnom vrednoscu \verb|b| koja odre\d uje broj
   pozicija pomeranja, kao i karakterom \verb|'L'| ili \verb|'D'| koji
   odre\d uje smer pomeranja (levo ili desno).
  \item{b)} Napisati funkciju \verb|void sifruj(char rec[],Sifra s)|
    koja transformi\v se zadatu re\v c \verb|rec| po \v sifri
    \verb|s|. Re\v c se \v sifruje tako \v sto se svako slovo
    zamenjuje slovom za \verb|b| mesta levo ili desno od njega u
    abecedi, i to cikli\v cno, a isto tako i za cifre.\\ Npr: za b=2,
    i smer='D' : a se menja sa c, b sa d,..., x sa z,y sa a, z sa b, 1
    sa 3, .. 8 sa 0, 9 sa 1
 \item{c)} Sa standarnog ulaza se zadaje na\v cin \v sifrovanja i to u
   obliku \verb|2 D 5 L|(\v sifra mo\v ze biti i du\v za). Potom se
   u\v citava $n$ i $n$ re\v ci sa standarnog ulaza (maksimalna du\v
   zina re\v ci je 20 karaktera).  Ispisati re\v ci na standarni izlaz
   nakon primenjenih svih zadatih na\v cina \v sifrovanja.
\end{description}
 \linkresenje{p2.3_}
\end{Exercise}
\begin{Answer}[ref=p2.3_]
%\includecode{resenja/2_PredstavljanjePodataka/2.3_Niske/1_08.c}
\end{Answer}


\begin{Exercise}[label=p2.3_] 
Implementirati funkciju \verb|int strspn(char* s, char* t)|
koja izra\v cunava du\v zinu po\v cetnog dela niske \verb|s| sastavljenog
isklju\v civo od karaktera sadr\v zanih u niski \verb|t|.

Napisati i program koji sa standardnog ulaza u\v citava dve niske (du\v zine
najvi\v se 100 karaktera, svaku u zasebnom redu) i ispisuje rezultat poziva
funkcije \verb|strspn| na standardni izlaz.

Na primer, za u\v{c}itane podatke \verb|"734a.bf62", "0123456789")| program
ispisuje vrednost 3.
 \linkresenje{p2.3_}
\end{Exercise}
\begin{Answer}[ref=p2.3_]
%\includecode{resenja/2_PredstavljanjePodataka/2.3_Niske/1_08.c}
\end{Answer}

\ifresenja
\section{Rešenja}
\shipoutAnswer
\fi
