\sstrana
\section{Višedimenzioni nizovi}


\begin{Exercise}[label=mat.01]
Napisati program koji učitava i zatim ispisuje elemente učitane matrice. Sa ulaza se najpre učitavaju 
dva cela broja $m$ i $n$, a potom i elementi matrice celih brojeva dimenzije $m \times n$. Pretpostaviti
da je maksimalna dimenzija matrice $50 \times 50$. 
U slučaju neispravnog unosa, ispisati odgovarajuću poruku o grešci. 

\skrati{1}
\begin{minitest}
\begin{upotreba}{1}
#\naslovInt#
#\izlaz{Unesite broj vrsta i}#
#\izlaz{broj kolona matrice:}#
#\ulaz{3 4}#
#\izlaz{Unesite elemente matrice:}#
#\ulaz{1 2 3 4}#
#\ulaz{5 6 7 8}#
#\ulaz{9 10 11 12}#
#\izlaz{Matrica je:}#
#\izlaz{1 2 3 4}#
#\izlaz{5 6 7 8}#
#\izlaz{9 10 11 12}#
\end{upotreba}
\end{minitest}
\begin{minitest}
\begin{upotreba}{2}
#\naslovInt#
#\izlaz{Unesite broj vrsta i}#
#\izlaz{broj kolona matrice:}#
#\ulaz{5 3}#
#\izlaz{Unesite elemente matrice:}#
#\ulaz{1 1 2}#
#\ulaz{5 0 2}#
#\ulaz{7 8 9}#
#\ulaz{1 2 4}#
#\ulaz{0 1 1}#
#\izlaz{Matrica je:}#
#\izlaz{1 1 2}#
#\izlaz{5 0 2}#
#\izlaz{7 8 9}#
#\izlaz{1 2 4}#
#\izlaz{0 1 1}#
\end{upotreba}
\end{minitest}
\begin{minitest}
\begin{upotreba}{3}
#\naslovInt#
#\izlaz{Unesite broj vrsta i}#
#\izlaz{broj kolona matrice:}#
#\ulaz{500 3}#
#\izlaz{Greska: neispravan unos.}#
\end{upotreba}
\end{minitest}

\linkresenje{mat.01}
\end{Exercise}
\ifresenja
\begin{Answer}[ref=mat.01]
\includecode{resenja/2_NapredniTipoviPodataka/2.7_VisedimenzioniNizovi/matrice_01.c}
\end{Answer}
\fi


\skrati{1}
\begin{Exercise}[label=mat.001]
Napisati program koji za učitanu celobrojnu matricu\footnote{Pod pojmom \emph{učitati matricu} ili \emph{za
    datu matricu} uvek se podrazumeva da se prvo unose dimenzije
  matrice, a potom i sama matrica.} dimenzije $m \times n$
izračunava i štampa na tri decimale njenu euklidsku normu. Pretpostaviti
da je maksimalna dimenzija matrice $50 \times 50$.
U slučaju neispravnog unosa, ispisati odgovarajuću poruku o grešci. 
\uputstvo{Euklidska norma matrice je kvadratni koren sume kvadrata svih elemenata matrice.}

\skrati{1}
\begin{minitest}
\begin{upotreba}{1}
#\naslovInt#
#\izlaz{Unesite broj vrsta i}#
#\izlaz{broj kolona matrice:}#
#\ulaz{3 4}#
#\izlaz{Unesite elemente matrice:}#
#\ulaz{1 2 3 4}#
#\ulaz{5 6 7 8}#
#\ulaz{9 10 11 12}#
#\izlaz{Euklidska norma: 25.495}#
\end{upotreba}
\end{minitest}
\begin{minitest}
\begin{upotreba}{2}
#\naslovInt#
#\izlaz{Unesite broj vrsta i}#
#\izlaz{broj kolona matrice:}#
#\ulaz{5 3}#
#\izlaz{Unesite elemente matrice:}#
#\ulaz{1 1 2}#
#\ulaz{5 0 2}#
#\ulaz{7 8 9}#
#\ulaz{1 2 4}#
#\ulaz{0 1 1}#
#\izlaz{Euklidska norma: 15.875}#
\end{upotreba}
\end{minitest}
\begin{minitest}
\begin{upotreba}{3}
#\naslovInt#
#\izlaz{Unesite broj vrsta i}#
#\izlaz{broj kolona matrice:}#
#\ulaz{500 3}#
#\izlaz{Greska: neispravan unos.}#
\end{upotreba}
\end{minitest}

\linkresenje{mat.001}
\end{Exercise}
\ifresenja
\begin{Answer}[ref=mat.001]
\includecode{resenja/2_NapredniTipoviPodataka/2.7_VisedimenzioniNizovi/matrice_02.c}
\end{Answer}
\fi


\skrati{1}
\begin{Exercise}[label=mat.1] 
Napisati funkcije za rad sa celobrojnim matricama:
\skrati{1}
\begin{enumerate}
\setlength\itemsep{1em}
  \item \kckod{void ucitaj(int a[][MAKS], int n, int m)} 
      kojom se učitavaju elementi matrice celih brojeva $a$ dimenzije $m \times n$,  
  \item \kckod{void ispisi(int a[][MAKS], int n, int m)} 
      kojom se ispisuju elementi matrice $a$ dimenzije $m \times n$.
\end{enumerate}
Napisati program koji najpre učitava, a zatim i ispisuje elemente 
učitane matrice.
Pretpostaviti da je maksimalna dimenzija matrice $50 \times 50$.
U slučaju neispravnog unosa, ispisati odgovarajuću poruku o grešci. 
\napomena{U ovom i u narednim zadacima, konstanta \kckod{MAKS} u prototipu funkcije označava maksimalni broj kolona date matrice i potrebno ju je definisati u rešenju direkivom \kckod{\#define}.}

\skrati{1}
\begin{minitest}
\begin{upotreba}{1}
#\naslovInt#
#\izlaz{Unesite broj vrsta i}#
#\izlaz{broj kolona matrice:}#
#\ulaz{3 4}#
#\izlaz{Unesite elemente matrice:}#
#\ulaz{1 2 3 4}#
#\ulaz{5 6 7 8}#
#\ulaz{9 10 11 12}#
#\izlaz{Matrica je:}#
#\izlaz{1 2 3 4}#
#\izlaz{5 6 7 8}#
#\izlaz{9 10 11 12}#
\end{upotreba}
\end{minitest}
\begin{minitest}
\begin{upotreba}{2}
#\naslovInt#
#\izlaz{Unesite broj vrsta i}#
#\izlaz{broj kolona matrice:}#
#\ulaz{2 5}#
#\izlaz{Unesite elemente matrice:}#
#\ulaz{1 1 2 3 4}#
#\ulaz{5 0 2 5 7}#
#\izlaz{Matrica je:}#
#\ulaz{1 1 2 3 4}#
#\ulaz{5 0 2 5 7}#
\end{upotreba}
\end{minitest}
\begin{minitest}
\begin{upotreba}{3}
#\naslovInt#
#\izlaz{Unesite broj vrsta i}#
#\izlaz{broj kolona matrice:}#
#\ulaz{500 3}#
#\izlaz{Greska: neispravan unos.}#
\end{upotreba}
\end{minitest}

\linkresenje{mat.1}
\end{Exercise}
\ifresenja
\begin{Answer}[ref=mat.1]
\includecode{resenja/2_NapredniTipoviPodataka/2.7_VisedimenzioniNizovi/matrice_03.c}
\end{Answer}
\fi


\eeskrati{3}
\skrati{1}
\begin{Exercise}[label=mat.2] 
Napisati funkciju \kckod{void transponovana(int a[][MAKS], int m, int
  n, int b[][MAKS])} koja određuje matricu $b$ koja je dobijena
transponovanjem matrice $a$. Napisati program koji za učitanu matricu
celih brojeva ispisuje odgovarajuću
transponovanu matricu.  Pretpostaviti da je maksimalna dimenzija
matrice $50 \times 50$.
U slučaju neispravnog unosa, ispisati odgovarajuću poruku o grešci. 

\eeskrati{3}
\begin{minitest}
\begin{upotreba}{1}
#\naslovInt#
#\izlaz{Unesite broj vrsta i}#
#\izlaz{broj kolona matrice:}#
#\ulaz{3 4}#
#\izlaz{Unesite elemente matrice:}#
#\ulaz{1 2 3 4}#
#\ulaz{5 6 7 8}#
#\ulaz{9 10 11 12}#
#\izlaz{Transponovana matrica je:}#
#\izlaz{1 5 9}#
#\izlaz{2 6 10}#
#\izlaz{3 7 11}#
#\izlaz{4 8 12}#
\end{upotreba}
\end{minitest}
\begin{minitest}
\begin{upotreba}{2}
#\naslovInt#
#\izlaz{Unesite broj vrsta i}#
#\izlaz{broj kolona matrice:}#
#\ulaz{5 3}#
#\izlaz{Unesite elemente matrice:}#
#\ulaz{1 1 2}#
#\ulaz{5 0 2}#
#\ulaz{7 8 9}#
#\ulaz{1 2 4}#
#\ulaz{0 1 1}#
#\izlaz{Transponovana matrica je:}#
#\izlaz{1 5 7 1 0}#
#\izlaz{1 0 8 2 1}#
#\izlaz{2 2 9 4 1}#
\end{upotreba}
\end{minitest}
\begin{minitest}
\begin{upotreba}{3}
#\naslovInt#
#\izlaz{Unesite broj vrsta i}#
#\izlaz{broj kolona matrice:}#
#\ulaz{500 3}#
#\izlaz{Greska: neispravan unos.}#
\end{upotreba}
\end{minitest}
\eeskrati{3}
\linkresenje{mat.2}
\end{Exercise}
\ifresenja
\begin{Answer}[ref=mat.2]
\includecode{resenja/2_NapredniTipoviPodataka/2.7_VisedimenzioniNizovi/matrice_04.c}
\end{Answer}
\fi


\eeskrati{3}
\skrati{3}
\begin{Exercise}[label=mat.3] 
Napisati funkciju \kckod{void razmeni(int a[][MAKS], int m, int n,
  int k, int t)} u kojoj se razmenjuju elementi $k$-te i $t$-te vrste
matrice $a$ dimezije $m \times n$. Napisati program koji za učitanu matricu celih brojeva 
i dva cela broja $k$ i $t$ ispisuje matricu dobijenu razmenjivanjem
$k$--te i $t$--te vrste ulazne matrice.  
Pretpostaviti da je maksimalna dimenzija matrice $50 \times 50$.  
U slučaju neispravnog unosa, ispisati odgovarajuću poruku o grešci.

\eeskrati{3}
\skrati{3}
\begin{minitest}
\begin{upotreba}{1}
#\naslovInt#
#\izlaz{Unesite broj vrsta i}#
#\izlaz{broj kolona matrice:}#
#\ulaz{3 4}#
#\izlaz{Unesite elemente matrice:}#
#\ulaz{1 2 3 4}#
#\ulaz{5 6 7 8}#
#\ulaz{9 10 11 12}#
#\izlaz{Unesite indekse vrsta:}#
#\ulaz{0 2}#
#\izlaz{Rezultujuca matrica:}#
#\izlaz{9 10 11 12}#
#\izlaz{5 6 7 8}#
#\izlaz{1 2 3 4}#
\end{upotreba}
\end{minitest}
\begin{minitest}
\begin{upotreba}{2}
#\naslovInt#
#\izlaz{Unesite broj vrsta i}#
#\izlaz{broj kolona matrice:}#
#\ulaz{5 3}#
#\izlaz{Unesite elemente matrice:}#
#\ulaz{1 1 2}#
#\ulaz{5 0 2}#
#\ulaz{7 8 9}#
#\ulaz{1 2 4}#
#\ulaz{0 1 1}#
#\izlaz{Unesite indekse vrsta:}#
#\ulaz{1 3}#
#\izlaz{Rezultujuca matrica:}#
#\izlaz{1 1 2}#
#\izlaz{1 2 4}#
#\izlaz{7 8 9}#
#\izlaz{5 0 2}#
#\izlaz{0 1 1}#
\end{upotreba}
\end{minitest}
\begin{minitest}
\begin{upotreba}{3}
#\naslovInt#
#\izlaz{Unesite broj vrsta i}#
#\izlaz{broj kolona matrice:}#
#\ulaz{5 3}#
#\izlaz{Unesite elemente matrice:}#
#\ulaz{1 1 2}#
#\ulaz{5 0 2}#
#\ulaz{7 8 9}#
#\ulaz{1 2 4}#
#\ulaz{0 1 1}#
#\izlaz{Unesite indekse vrsta:}#
#\ulaz{-1 50}#
#\izlaz{Greska: neispravan unos.}#
\end{upotreba}
\end{minitest}
\eeskrati{3}
\linkresenje{mat.3}
\end{Exercise}
\ifresenja
\begin{Answer}[ref=mat.3]
\includecode{resenja/2_NapredniTipoviPodataka/2.7_VisedimenzioniNizovi/matrice_05.c}
\end{Answer}
\fi

\skrati{3}
\begin{Exercise}[label=mat.4] 
Napisati program koji za učitanu matricu celih brojeva ispisuje
indekse onih elemenata matrice koji su jednaki zbiru svih svojih
susednih elemenata. Pretpostaviti da je maksimalna dimenzija
matrice $50 \times 50$.  
U slučaju neispravnog unosa, ispisati odgovarajuću poruku o grešci.
\begin{figure}[h!]
\begin{center}
\begin{minipage}{30mm}
\begin{verbatim}
- - - - - - s b s -
- s s s - - s s s -
- s a s - - - - - -
- s s s - - - - - -
- - - - - - - - s s
- - - - - - - - s c
\end{verbatim}
\end{minipage}
\end{center}
\skrati{2}
\caption{Susedni elementi u matrici.}
\label{fig:susedi}
\end{figure}

\uputstvo{Elementi matrice \kckod{m} susedni elementu \kckod{m[i][j]} su svi elementi 
matrice čiji se indeksi, po apsolutnoj vrednosti, razlikuju najviše za jedan. 
Element matrice može imati najviše osam suseda: \kckod{m[i-1][j-1]}, \kckod{m[i-1][j]}, \kckod{m[i-1][j+1]}, 
\kckod{m[i][j-1]}, \kckod{m[i][j+1]}, \kckod{m[i+1][j-1]}, \kckod{m[i+1][j]} i \kckod{m[i+1][j+1]}. 
U zavisnosti od položaja u matrici, element matrice može imati i tri ili pet suseda. 
Na slici \ref{fig:susedi} su slovom \kckod{s} obeleženi 
susedni elementi matrice za elemente \kckod{m[2][2]} (element je na slici obele\v{z}en sa \kckod{a}), \kckod{m[0][7]} (element je na slici obele\v{z}en sa \kckod{b}) i \kckod{m[5][9]}(element je na slici obele\v{z}en sa \kckod{c}).}

\skrati{2}
\begin{minitest}
\begin{upotreba}{1}
#\naslovInt#
#\izlaz{Unesite broj vrsta i}#
#\izlaz{broj kolona matrice:}#
#\ulaz{4 5}#
#\izlaz{Unesite elemente matrice:}#
#\ulaz{1 1 2 1 3}#
#\ulaz{0 8 1 9 0}#
#\ulaz{1 1 1 0 0}#
#\ulaz{0 3 0 2 2}#
#\izlaz{Indeksi elemenata koji su}#
#\izlaz{jednaki zbiru suseda su:}#
#\izlaz{1 1}#
#\izlaz{3 1}#
#\izlaz{3 4}#
\end{upotreba}
\end{minitest}
\begin{minitest}
\begin{upotreba}{2}
#\naslovInt#
#\izlaz{Unesite broj vrsta i}#
#\izlaz{broj kolona matrice:}#
#\ulaz{3 4}#
#\izlaz{Unesite elemente matrice:}#
#\ulaz{7  10 12 20}#
#\ulaz{-1 -3 1   7}#
#\ulaz{0  -47 2  0}#
#\izlaz{Indeksi elemenata koji su}#
#\izlaz{jednaki zbiru suseda su:}#
#\izlaz{0 3}#
#\izlaz{1 2}#
\end{upotreba}
\end{minitest}
\begin{minitest}
\begin{upotreba}{3}
#\naslovInt#
#\izlaz{Unesite broj vrsta i}#
#\izlaz{broj kolona matrice:}#
#\ulaz{5 -3}#
#\izlaz{Greska: neispravan unos.}#
\end{upotreba}
\end{minitest}
\linkresenje{mat.4}
\end{Exercise}
\ifresenja
\begin{Answer}[ref=mat.4]
\includecode{resenja/2_NapredniTipoviPodataka/2.7_VisedimenzioniNizovi/matrice_06.c}
\end{Answer}
\fi

\skrati{4}
\begin{Exercise}[label=mat.20] 
Napisati funkciju koja formira niz $b_0, b_1, \ldots, b_{n-1}$ od matrice $n \times m$
tako što element niza $b_i$ izračunava kao srednju vrednost elemenata
$i$-te vrste matrice.  Napisati program koji za učitanu matricu celih brojeva
ispisuje dobijeni niz.  
Pretpostaviti da je maksimalna dimenzija matrice $50 \times 50$.
U slučaju neispravnog unosa, ispisati odgovarajuću poruku o grešci.

\skrati{2}
\begin{minitest}
\begin{upotreba}{1}
#\naslovInt#
#\izlaz{Unesite broj vrsta i}#
#\izlaz{broj kolona matrice:}#
#\ulaz{4 5}#
#\izlaz{Unesite elemente matrice:}#
#\ulaz{1 1 2 1 3}#
#\ulaz{0 8 1 9 0}#
#\ulaz{1 1 1 0 0}#
#\ulaz{0 3 0 2 2}#
#\izlaz{Dobijeni niz je:}#
#\izlaz{1.6 3.6 0.6 1.4}#
\end{upotreba}
\end{minitest}
\begin{minitest}
\begin{upotreba}{2}
#\naslovInt#
#\izlaz{Unesite broj vrsta i}#
#\izlaz{broj kolona matrice:}#
#\ulaz{3 4}#
#\izlaz{Unesite elemente matrice:}#
#\ulaz{7  10 12 20}#
#\ulaz{-1 -3 1   7}#
#\ulaz{0  -47 2  0}#
#\izlaz{Dobijeni niz je:}#
#\izlaz{12.25 1 -11.25}#
\end{upotreba}
\end{minitest}
\begin{minitest}
\begin{upotreba}{3}
#\naslovInt#
#\izlaz{Unesite broj vrsta i}#
#\izlaz{broj kolona matrice:}#
#\ulaz{51 13}#
#\izlaz{Greska: neispravan unos.}#
\end{upotreba}
\end{minitest}

\linkresenje{mat.20}
\end{Exercise}
\ifresenja
\begin{Answer}[ref=mat.20]
\includecode{resenja/2_NapredniTipoviPodataka/2.7_VisedimenzioniNizovi/matrice_07.c}
\end{Answer}
\fi

% ---------------------------------------------------------------jednostavni zadaci sa kvadratnim matricama

\skrati{2}
\begin{Exercise}[label=mat.5] 
Relacija se može predstaviti kvadratnom matricom nula i jedinica na
sledeći način: element $i$ je u relaciji sa elementom $j$ ukoliko se u
preseku $i$--te vrste i $j$--te kolone nalazi jedinica, a nije u
relaciji ukoliko se tu nalazi nula. Napisati funkcije:
\skrati{2}
\begin{enumerate}
\setlength\itemsep{0em}
  \item \kckod{int refleksivna(int a[][MAKS], int
    n)} kojom se za relaciju zadatu matricom $a$ dimenzije
    $n \times n$ ispituje da li je refleksivna;
  \item \kckod{int simetricna(int a[][MAKS], int n)}
    kojom se za relaciju zadatu matricom $a$ dimenzije
    $n \times n$ ispituje da li je simetrična;
  \item \kckod{int tranzitivna(int a[][MAKS], int n)}
    kojom se za relaciju zadatu matricom $a$ ispituje dimenzije
    $n \times n$ da li je tranzitivna;
  \item \kckod{int ekvivalencija(int a[][MAKS], int
    n)} kojom se za relaciju zadatu matricom $a$ dimenzije
    $n \times n$ ispituje da li je relacija ekvivalencije. 
\end{enumerate}
Napisati program koji za učitanu dimenziju $n$ i kvadratnu matricu dimenzije $n\times
n$ ispisuje osobine odgovarajuće relacije.  
Pretpostaviti da je maksimalna dimenzija matrice $50 \times 50$ i 
da matrica za vrednosti elemenata može imati samo nule i jedinice.
U slučaju neispravnog unosa, ispisati odgovarajuću poruku o grešci. 

\skrati{2}
\begin{minitest}
\begin{upotreba}{1}
#\naslovInt#
#\izlaz{Unesite broj vrsta matrice:}#
#\ulaz{4}#
#\izlaz{Unesite elemente matrice:}#
#\ulaz{1 0 0 0}#
#\ulaz{0 1 1 0}#
#\ulaz{0 0 1 0}#
#\ulaz{0 0 0 0}#
#\izlaz{Relacija nije refleksivna.}#
#\izlaz{Relacija nije simatricna.}#
#\izlaz{Relacija jeste tranzitivna.}#
#\izlaz{Relacija nije ekvivalencija.}#
\end{upotreba}
\end{minitest}
\begin{minitest}
\begin{upotreba}{2}
#\naslovInt#
#\izlaz{Unesite broj vrsta matrice:}#
#\ulaz{4}#
#\izlaz{Unesite elemente matrice:}#
#\ulaz{1 1 0 0}#
#\ulaz{1 1 1 0}#
#\ulaz{0 1 1 0}#
#\ulaz{0 0 0 1}#
#\izlaz{Relacija jeste refleksivna.}#
#\izlaz{Relacija jeste simatricna.}#
#\izlaz{Relacija nije tranzitivna.}#
#\izlaz{Relacija nije ekvivalencija.}#
\end{upotreba}
\end{minitest}
\begin{minitest}
\begin{upotreba}{3}
#\naslovInt#
#\izlaz{Unesite broj vrsta matrice:}#
#\ulaz{54}#
#\izlaz{Greska: neispravan unos.}#
\end{upotreba}
\end{minitest}

\linkresenje{mat.5}
\end{Exercise}
\ifresenja
\begin{Answer}[ref=mat.5]
\includecode{resenja/2_NapredniTipoviPodataka/2.7_VisedimenzioniNizovi/matrice_08.c}
\end{Answer}
\fi


\skrati{2}
\begin{Exercise}[label=mat.6]
Data je kvadratna matrica dimenzije $n \times n$.
\skrati{3}
\begin{enumerate}[itemsep=0pt]
  \item Napisati funkciju \kckod{float trag(float a[][MAKS], int n)}
    koja računa trag matrice, odnosno zbir elemenata na glavnoj
    dijagonali matrice.
  \item Napisati funkciju \kckod{float suma\_sporedna(float a[][MAKS],
    int n)} koja računa zbir elemenata na sporednoj dijagonali
    matrice.
  \item Napisati funkciju \kckod{float suma\_iznad(float a[][MAKS], int n)}
    koja određuje sumu elemenata iznad glavne dijagonale.
  \item Napisati funkciju \kckod{float suma\_ispod(float a[][MAKS], int n)}
    koja određuje sumu elemenata ispod sporedne dijagonale matrice.
\end{enumerate}
\skrati{3}
Napisati program koji
za učitanu matricu realnih brojeva ispisuje na tri decimale trag
matrice, sumu na sporednoj dijagonali, sumu iznad glavne dijagonale i
sumu elemenata ispod sporedne dijagonale. Pretpostaviti da je
maksimalna dimenzija matrice $50 \times 50$.
U slučaju neispravnog unosa, ispisati odgovarajuću poruku o grešci.

\skrati{3}
\begin{miditest}
\begin{upotreba}{1}
#\naslovInt#
#\izlaz{Unesite broj vrsta matrice:}\ulaz{4}#
#\izlaz{Unesite elemente matrice:}#
#\ulaz{6 12.08 -1 20.5}#
#\ulaz{8 90 -33.4 19.02}#
#\ulaz{7.02 5 -20 14.5}#
#\ulaz{8.8 -1 3 -22.8}#
\end{upotreba}
\end{miditest}
\begin{miditest}
\begin{upotreba}{1 (nastavak)}
#\izlaz{Trag: 53.20}#
#\izlaz{Suma na sporednoj dijagonali: 0.90}#
#\izlaz{Suma iznad glavne dijagonale: 31.70}#
#\izlaz{Suma ispod sporedne dijagonale: -7.28}#
\end{upotreba}
\end{miditest}

\begin{miditest}
\begin{upotreba}{2}
#\naslovInt#
#\izlaz{Unesite broj vrsta matrice:}\ulaz{5}#
#\izlaz{Unesite elemente matrice:}#
#\ulaz{1 2 3 5 5}#
#\ulaz{7 8 9 0 1}#
#\ulaz{6 4 3 2 2}#
#\ulaz{8 9 1 3 4}#
#\ulaz{0 3 1 8 6}#
\end{upotreba}
\end{miditest}
\begin{miditest}
\begin{upotreba}{2 (nastavak)}
#\izlaz{Trag: 21.00}#
#\izlaz{Suma na sporednoj dijagonali: 17.00}#
#\izlaz{Suma iznad glavne dijagonale: 33.00}#
#\izlaz{Suma ispod sporedne dijagonale: 31.00}#
\end{upotreba}
\end{miditest}

\linkresenje{mat.6}
\end{Exercise}
\ifresenja
\begin{Answer}[ref=mat.6]
\includecode{resenja/2_NapredniTipoviPodataka/2.7_VisedimenzioniNizovi/matrice_09.c}
\end{Answer}
\fi


\skrati{3}
\begin{Exercise}[label=mat.7] 
Kvadratna matrica je donje trougaona ako se u gornjem trouglu (iznad
glavne dijagonale, ne uključujući dijagonalu) nalaze sve nule.
Napisati program koji za učitanu kvadratnu matricu proverava da li je
ona donje trougaona i ispisuje odgovarajuću poruku.  Pretpostaviti da
je maksimalna dimenzija matrice $100 \times 100$.
U slučaju neispravnog unosa, ispisati odgovarajuću poruku o grešci.

\begin{minitest}
\begin{upotreba}{1}
#\naslovInt#
#\izlaz{Unesite broj vrsta matrice:}#
#\ulaz{5}#
#\izlaz{Unesite elemente matrice:}#
#\ulaz{-1 0 0 0 0}#
#\ulaz{2 10 0 0 0}#
#\ulaz{0 1 5 0 0}#
#\ulaz{7 8 20 14 0}#
#\ulaz{-23 8 5 1 11}#
#\izlaz{Matrica jeste donje}#
#\izlaz{trougaona.}#
\end{upotreba}
\end{minitest}
\begin{minitest}
\begin{upotreba}{2}
#\naslovInt#
#\izlaz{Unesite broj vrsta matrice:}#
#\ulaz{3}#
#\izlaz{Unesite elemente matrice:}#
#\ulaz{2 -2 1}#
#\ulaz{1 2 2}#
#\ulaz{2 1 -2}#
#\izlaz{Matrica nije donje}#
#\izlaz{trougaona.}#
\end{upotreba}
\end{minitest}
\begin{minitest}
\begin{upotreba}{3}
#\naslovInt#
#\izlaz{Unesite broj vrsta matrice:}#
#\ulaz{200}#
#\izlaz{Greska: neispravan unos.}#
\end{upotreba}
\end{minitest}

\linkresenje{mat.7}
\end{Exercise}
\ifresenja
\begin{Answer}[ref=mat.7]
\includecode{resenja/2_NapredniTipoviPodataka/2.7_VisedimenzioniNizovi/matrice_10.c}
\end{Answer}
\fi


\skrati{3}
\begin{Exercise}[label=mat.8] 
Napisati program koji za učitanu celobrojnu kvadratnu matricu ispisuje redni broj
kolone koja ima najveći zbir elemenata.  Pretpostaviti da je
maksimalna dimenzija matrice $50 \times 50$.
U slučaju neispravnog unosa, ispisati odgovarajuću poruku o grešci.

\begin{minitest}
\begin{upotreba}{1}
#\naslovInt#
#\izlaz{Unesite broj vrsta matrice:}#
#\ulaz{3}#
#\izlaz{Unesite elemente matrice:}#
#\ulaz{1 2 3}#
#\ulaz{7 3 4}#
#\ulaz{5 3 1}#
#\izlaz{Indeks kolone je: 0}#
\end{upotreba}
\end{minitest}
\begin{minitest}
\begin{upotreba}{2}
#\naslovInt#
#\izlaz{Unesite broj vrsta matrice:}#
#\ulaz{4}#
#\izlaz{Unesite elemente matrice:}#
#\ulaz{7 8 9 10}#
#\ulaz{7 6 11 4}#
#\ulaz{3 1 2 -2}#
#\ulaz{8 3 9 9}#
#\izlaz{Indeks kolone je: 2}#
\end{upotreba}
\end{minitest}
\begin{minitest}
\begin{upotreba}{3}
#\naslovInt#
#\izlaz{Unesite broj vrsta matrice:}#
#\ulaz{104}#
#\izlaz{Greska: neispravan unos.}#
\end{upotreba}
\end{minitest}

\linkresenje{mat.8}
\end{Exercise}
\ifresenja
\begin{Answer}[ref=mat.8]
\includecode{resenja/2_NapredniTipoviPodataka/2.7_VisedimenzioniNizovi/matrice_11.c}
\end{Answer}
\fi


\skrati{3}
\begin{Exercise}[label=mat.9] 
Napisati program koji za učitanu kvadratnu matricu realnih brojeva izračunava i
ispisuje na dve decimale razliku između zbira elemenata gornjeg
trougla i zbira elemenata donjeg trougla matrice. Gornji trougao čine
svi elementi matrice koji su iznad glavne i sporedne dijagonale (ne
računajući dijagonale), a donji trougao čine svi elementi ispod glavne
i sporedne dijagonale (ne računajući dijagonale). Pretpostaviti da je
maksimalna dimenzija matrice $50 \times 50$.
U slučaju neispravnog unosa, ispisati odgovarajuću poruku o grešci. 

\skrati{3}
\begin{minitest}
\begin{upotreba}{1}
#\naslovInt#
#\izlaz{Unesite broj vrsta matrice:}#
#\ulaz{3}#
#\izlaz{Unesite elemente matrice:}#
#\ulaz{2 3.2 4}#
#\ulaz{7 8.8 1}#
#\ulaz{2.3 1 1}#
#\izlaz{Razlika je: 2.20}#
\end{upotreba}
\end{minitest}
\begin{minitest}
\begin{upotreba}{2}
#\naslovInt#
#\izlaz{Unesite broj vrsta matrice:}#
#\ulaz{5}#
#\izlaz{Unesite elemente matrice:}#
#\ulaz{2.3 1 12 8 -20}#
#\ulaz{4 -8.2 7 14.5 19}#
#\ulaz{1 -2.5 9 11 33}#
#\ulaz{3 4.3 -5.7 2 8}#
#\ulaz{9 56 1.08 7 5.5}#
#\izlaz{Razlika je: -30.38}#
\end{upotreba}
\end{minitest}
\begin{minitest}
\begin{upotreba}{3}
#\naslovInt#
#\izlaz{Unesite broj vrsta matrice:}#
#\ulaz{52}#
#\izlaz{Greska: neispravan unos.}#
\end{upotreba}
\end{minitest}

\linkresenje{mat.9}
\end{Exercise}
\ifresenja
\begin{Answer}[ref=mat.9]
\includecode{resenja/2_NapredniTipoviPodataka/2.7_VisedimenzioniNizovi/matrice_12.c}
\end{Answer}
\fi


% ---------------------------------------------------------------tezi zadaci (kvadrtne + nekvadratne mattrice)

\skrati{3}
\begin{Exercise}[label=mat.10] 
Napisati program koji za učitanu celobrojnu matricu dimenzije $m \times n$ i
uneta dva broja $p$ i $k$ ($p \le m$, $k \le n$) ispisuje sume svih
podmatrica dimenzije $p \times k$ unete matrice.  Pretpostaviti da je
maksimalna dimenzija matrice $50 \times 50$.
U slučaju neispravnog unosa, ispisati odgovarajuću poruku o grešci.
%\napomena{Nije bitan redosled kojim se ispisuju sume.}

\skrati{3}
\begin{miditest}
\begin{upotreba}{1}
#\naslovInt#
#\izlaz{Unesite broj vrsta i broj kolona matrice:}#
#\ulaz{3 4}#
#\izlaz{Unesite elemente matrice:}#
#\ulaz{1 2 3 4}#
#\ulaz{5 6 7 8}#
#\ulaz{9 10 11 12}#
#\izlaz{Unesite dva cela broja:}\ulaz{3 3}#
#\izlaz{Sume podmatrica su: 54 63}#
\end{upotreba}
\end{miditest}
\begin{miditest}
\begin{upotreba}{2}
#\naslovInt#
#\izlaz{Unesite broj vrsta i broj kolona matrice:}#
#\ulaz{3 4}#
#\izlaz{Unesite elemente matrice:}#
#\ulaz{1 2 3 4}#
#\ulaz{5 6 7 8}#
#\ulaz{9 10 11 12}#
#\izlaz{Unesite dva cela broja:}\ulaz{2 3}#
#\izlaz{Sume podmatrica su: 24 30 48 54}#
\end{upotreba}
\end{miditest}

\skrati{3}
\begin{miditest}
\begin{upotreba}{3}
#\naslovInt#
#\izlaz{Unesite broj vrsta i broj kolona matrice:}#
#\ulaz{5 3}#
#\izlaz{Unesite elemente matrice:}#
#\ulaz{1 1 2}#
#\ulaz{5 0 2}#
#\ulaz{7 8 9}#
#\ulaz{1 2 4}#
#\ulaz{0 1 1}#
#\izlaz{Unesite dva cela broja:}\ulaz{2 2}#
#\izlaz{Sume podmatrica su: 7 5 20 19 18 23 4 8}#
\end{upotreba}
\end{miditest}
\begin{miditest}
\begin{upotreba}{4}
#\naslovInt#
#\izlaz{Unesite broj vrsta i broj kolona matrice:}#
#\ulaz{-3 200}#
#\izlaz{Greska: neispravan unos.}#
\end{upotreba}
\end{miditest}


\linkresenje{mat.10}
\end{Exercise}
\ifresenja
\begin{Answer}[ref=mat.10]
\includecode{resenja/2_NapredniTipoviPodataka/2.7_VisedimenzioniNizovi/matrice_13.c}
\sstrana
\end{Answer}
\fi


\skrati{3}
\begin{Exercise}[label=mat.11] 
Napisati program koji za učitanu celobrojnu kvadratnu matricu ispituje da li su
njeni elementi po kolonama, vrstama i dijagonalama (glavnoj i
sporednoj) sortirani strogo rastuće.
Pretpostaviti da je maksimalna dimenzija matrice $50 \times 50$.
U slučaju neispravnog unosa, ispisati odgovarajuću poruku o grešci. 

\skrati{3}
\begin{miditest}
\begin{upotreba}{1}
#\naslovInt#
#\izlaz{Unesite broj vrsta matrice:}\ulaz{2}#
#\izlaz{Unesite elemente matrice:}#
#\ulaz{6 9}#
#\ulaz{4 10}#
#\izlaz{Elementi nisu sortirani po kolonama.}#
#\izlaz{Elementi su sortirani po vrstama.}#
#\izlaz{Elementi nisu sortirani po dijagonalama.}#
\end{upotreba}
\end{miditest}
\begin{miditest}
\begin{upotreba}{2}
#\naslovInt#
#\izlaz{Unesite broj vrsta matrice:}\ulaz{3}#
#\izlaz{Unesite elemente matrice:}#
#\ulaz{1 2 3}#
#\ulaz{4 5 6}#
#\ulaz{7 8 9}#
#\izlaz{Elementi su sortirani po kolonama.}#
#\izlaz{Elementi su sortirani po vrstama.}#
#\izlaz{Elementi su sortirani po dijagonalama.}#
\end{upotreba}
\end{miditest}

\skrati{3}
\begin{miditest}
\begin{upotreba}{3}
#\naslovInt#
#\izlaz{Unesite broj vrsta matrice:}\ulaz{4}#
#\izlaz{Unesite elemente matrice:}#
#\ulaz{5 5 7 9}#
#\ulaz{6 10 11 13}#
#\ulaz{8 12 14 15}#
#\ulaz{13 15 16 20}#
#\izlaz{Elementi su sortirani po kolonama.}#
#\izlaz{Elementi nisu sortirani po vrstama.}#
#\izlaz{Elementi su sortirani po dijagonalama.}#
\end{upotreba}
\end{miditest}
\begin{miditest}
\begin{upotreba}{4}
#\naslovInt#
#\izlaz{Unesite broj vrsta matrice:}\ulaz{1}#
#\izlaz{Unesite elemente matrice:}#
#\ulaz{5}#
#\izlaz{Elementi su sortirani po kolonama.}#
#\izlaz{Elementi su sortirani po vrstama.}#
#\izlaz{Elementi su sortirani po dijagonalama.}#
\end{upotreba}
\end{miditest}

\linkresenje{mat.11}
\end{Exercise}
\ifresenja
\begin{Answer}[ref=mat.11]
\includecode{resenja/2_NapredniTipoviPodataka/2.7_VisedimenzioniNizovi/matrice_14.c}
\end{Answer}
\fi


\skrati{2}
\begin{Exercise}[label=mat.12] 
Napisati program koji za učitanu celobrojnu kvadratnu matricu ispituje da li su
zbirovi elemenata njenih kolona uređeni u strogo rastućem poretku.  
Pretpostaviti da je maksimalna dimenzija matrice $10 \times 10$.
U slučaju neispravnog unosa, ispisati odgovarajuću poruku o grešci.

\skrati{2}
\begin{miditest}
\begin{upotreba}{1}
#\naslovInt#
#\izlaz{Unesite broj vrsta matrice:}\ulaz{4}#
#\izlaz{Unesite elemente matrice:}#
#\ulaz{1 0 0 0}#
#\ulaz{0 0 1 0}#
#\ulaz{0 0 0 1}#
#\ulaz{0 1 0 0}#
#\izlaz{Sume nisu uredjene strogo rastuce.}#
\end{upotreba}
\end{miditest}
\begin{miditest}
\begin{upotreba}{2}
#\naslovInt#
#\izlaz{Unesite broj vrsta matrice:}\ulaz{3}#
#\izlaz{Unesite elemente matrice:}#
#\ulaz{1 2 3}#
#\ulaz{4 5 6}#
#\ulaz{7 8 9}#
#\izlaz{Sume jesu uredjene strogo rastuce.}#
\end{upotreba}
\end{miditest}

\skrati{2}
\begin{miditest}
\begin{upotreba}{3}
#\naslovInt#
#\izlaz{Unesite broj vrsta matrice:}\ulaz{3}#
#\izlaz{Unesite elemente matrice:}#
#\ulaz{2 -2 1}#
#\ulaz{1 2 2}#
#\ulaz{2 1 -2}#
#\izlaz{Sume nisu uredjene strogo rastuce.}#
\end{upotreba}
\end{miditest}
\begin{miditest}
\begin{upotreba}{4}
#\naslovInt#
#\izlaz{Unesite broj vrsta matrice:}\ulaz{5}#
#\izlaz{Unesite elemente matrice:}#
#\ulaz{-1 0 3 0 20}#
#\ulaz{0 0 0 10 0}#
#\ulaz{0 0 -1 0 0}#
#\ulaz{0 1 0 0 0}#
#\ulaz{0 0 0 0 -1}#
#\izlaz{Sume jesu uredjene strogo rastuce.}#
\end{upotreba}
\end{miditest}

\linkresenje{mat.12}
\end{Exercise}
\ifresenja
\begin{Answer}[ref=mat.12]
\includecode{resenja/2_NapredniTipoviPodataka/2.7_VisedimenzioniNizovi/matrice_15.c}
\sstrana
\end{Answer}
\fi

\skrati{2}
\begin{Exercise}[label=mat.13] 
Matrica je \emph{ortonormirana} ako je vrednost skalarnog proizvoda svakog para
različitih vrsta jednak nuli, a vrednost skalarnog proizvoda vrste sa samom sobom
jednak jedinici. Napisati program koji za unetu celobrojnu kvadratnu matricu proverava da 
li je ortonormirana.
Pretpostaviti da je maksimalna dimenzija matrice $50 \times 50$.
U slučaju neispravnog unosa, ispisati odgovarajuću poruku o grešci. 
\napomena{ Skalarni proizvod vektora $a = (a_1, a_2, \ldots,
  a_n)$ i $b = (b_1, b_2, \ldots, b_n)$ je $a_1\cdot b_1 + a_2\cdot
  b_2 + \ldots + a_n\cdot b_n$.}

\skrati{2}
\begin{miditest}
\begin{upotreba}{1}
#\naslovInt#
#\izlaz{Unesite broj vrsta matrice:}\ulaz{4}#
#\izlaz{Unesite elemente matrice:}#
#\ulaz{1 0 0 0}#
#\ulaz{0 0 1 0}#
#\ulaz{0 0 0 1}#
#\ulaz{0 1 0 0}#
#\izlaz{Matrica jeste ortonormirana.}#
\end{upotreba}
\end{miditest}
\begin{miditest}
\begin{upotreba}{2}
#\naslovInt#
#\izlaz{Unesite broj vrsta matrice:}\ulaz{3}#
#\izlaz{Unesite elemente matrice:}#
#\ulaz{1 2 3}#
#\ulaz{4 5 6}#
#\ulaz{7 8 9}#
#\izlaz{Matrica nije ortonormirana.}#
\end{upotreba}
\end{miditest}

\skrati{2}
\begin{miditest}
\begin{upotreba}{3}
#\naslovInt#
#\izlaz{Unesite broj vrsta matrice:}\ulaz{3}#
#\izlaz{Unesite elemente matrice:}#
#\ulaz{2 -2 1}#
#\ulaz{1 2 2}#
#\ulaz{2 1 -2}#
#\izlaz{Matrica nije ortonormirana.}#
\end{upotreba}
\end{miditest}
\begin{miditest}
\begin{upotreba}{4}
#\naslovInt#
#\izlaz{Unesite broj vrsta matrice:}\ulaz{5}#
#\izlaz{Unesite elemente matrice:}#
#\ulaz{-1 0 0 0 0}#
#\ulaz{0 0 0 1 0}#
#\ulaz{0 0 -1 0 0}#
#\ulaz{0 1 0 0 0}#
#\ulaz{0 0 0 0 -1}#
#\izlaz{Matrica jeste ortonormirana.}#
\end{upotreba}
\end{miditest}

\linkresenje{mat.13}
\end{Exercise}
\ifresenja
\begin{Answer}[ref=mat.13]
\includecode{resenja/2_NapredniTipoviPodataka/2.7_VisedimenzioniNizovi/matrice_16.c}
\end{Answer}
\fi


\skrati{2}
\begin{Exercise}[label=mat.14] 
Kvadratna matrica je \emph{magični kvadrat} ako su sume elemenata
u svim vrstama i kolonama jednake. Napisati program koji
proverava da li je data celobrojna kvadratna matrica magični kvadrat i
ispisuje odgovarajuću poruku na standardni izlaz. Pretpostaviti 
da je maksimalna dimenzija matrice $50 \times 50$.
U slučaju neispravnog unosa, ispisati odgovarajuću poruku o grešci.

\skrati{3}
\begin{miditest}
\begin{upotreba}{1}
#\naslovInt#
#\izlaz{Unesite broj vrsta matrice:}\ulaz{4}#
#\izlaz{Unesite elemente matrice:}#
#\ulaz{1 5 3 1}#
#\ulaz{2 1 2 5}#
#\ulaz{3 2 2 3}#
#\ulaz{4 2 3 1}#
#\izlaz{Matrica jeste magicni kvadrat.}#
\end{upotreba}
\end{miditest}
\begin{miditest}
\begin{upotreba}{2}
#\naslovInt#
#\izlaz{Unesite broj vrsta matrice:}\ulaz{3}#
#\izlaz{Unesite elemente matrice:}#
#\ulaz{1 2 3}#
#\ulaz{4 5 6}#
#\ulaz{-1 3 3}#
#\izlaz{Matrica nije magicni kvadrat.}#
\end{upotreba}
\end{miditest}

\linkresenje{mat.14}
\end{Exercise}
\ifresenja
\begin{Answer}[ref=mat.14]
\includecode{resenja/2_NapredniTipoviPodataka/2.7_VisedimenzioniNizovi/matrice_17.c}
\end{Answer}
\fi

% ---------------------------------------------------------------teski zadaci (kvadrtne + nekvadratne mattrice)

\skrati{3}
\begin{Exercise}[difficulty=1, label=mat.15] 
Napisati program koji učitava celobrojnu kvadratnu matricu i ispisuje elemente matrice
u grupama koje su paralelne sa njenom sporednom dijagonalom, počevši od gornjeg levog
ugla. Pretpostaviti da je maksimalna dimenzija matrice $100 \times 100$.
U slučaju neispravnog unosa, ispisati odgovarajuću poruku o grešci.

\skrati{3}
\begin{minitest}
\begin{upotreba}{1}
#\naslovInt#
#\izlaz{Unesite broj vrsta matrice:}#
#\ulaz{3}#
#\izlaz{Unesite elemente matrice:}#
#\ulaz{1 2 3}#
#\ulaz{4 5 6}#
#\ulaz{7 8 9}#
#\izlaz{Ispis je:}#
#\izlaz{1}#
#\izlaz{2 4}#
#\izlaz{3 5 7}#
#\izlaz{6 8}#
#\izlaz{9}#
\end{upotreba}
\end{minitest}
\begin{minitest}
\begin{upotreba}{2}
#\naslovInt#
#\izlaz{Unesite broj vrsta matrice:}#
#\ulaz{5}#
#\izlaz{Unesite elemente matrice:}#
#\ulaz{7  -8  1 2 3}#
#\ulaz{90 11  0 5 4}#
#\ulaz{12 -9  14  23 8}#
#\ulaz{80  6  88  17 62}#
#\ulaz{-22 10 44  57 -200}#
#\izlaz{Ispis je:}#
#\izlaz{7}#
#\izlaz{-8 90}#
#\izlaz{1 11 12}#
#\izlaz{2 0 -9 80}#
#\izlaz{3 5 14 6 -22}#
#\izlaz{4 23 88 10}#
#\izlaz{8 17 44}#
#\izlaz{62 57}#
#\izlaz{-200}#
\end{upotreba}
\end{minitest}
\begin{minitest}
\begin{upotreba}{3}
#\naslovInt#
#\izlaz{Unesite broj vrsta matrice:}#
#\ulaz{-5}#
#\izlaz{Greska: neispravan unos.}#
\end{upotreba}
\end{minitest}
\linkresenje{mat.15}
\end{Exercise}
\ifresenja
\begin{Answer}[ref=mat.15]
\includecode{resenja/2_NapredniTipoviPodataka/2.7_VisedimenzioniNizovi/matrice_18.c}
\end{Answer}
\fi


\skrati{3}
\begin{Exercise}[difficulty=1, label=mat.16] 
Napisati funkciju \kckod{void mnozenje(int a[][MAKS], int m, int n, int
  b[][MAKS], int k, int t, int c[][MAKS])} koja računa matricu $c$ kao
proizvod matrica $a$ i $b$.  Dimenzija matrice $a$ je $n \times m$, a
dimenzija matrice $b$ je $k \times t$. Napisati program koji ispisuje
proizvod učitanih matrica. Pretpostaviti da je maksimalna dimenzija
matrica $50 \times 50$. Ukoliko množenje matrica nije moguće ili je
došlo do greške prilikom unosa podataka, ispisati odgovarajuću
poruku o grešci.

\skrati{3}
\begin{miditest}
\begin{upotreba}{1}
#\naslovInt#
#\izlaz{Unesite broj vrsta i broj kolona matrice a:}#
#\ulaz{3 4}#
#\izlaz{Unesite elemente matrice:}#
#\ulaz{1 2 8 9}#
#\ulaz{-4 5 2 3}#
#\ulaz{7 6 4 10}#
#\izlaz{Unesite broj vrsta i broj kolona matrice b:}#
#\ulaz{4 2}#
#\izlaz{Unesite elemente matrice:}#
#\ulaz{11 5}#
#\ulaz{6 7}#
#\ulaz{8 9}#
#\ulaz{0 -3}#
#\izlaz{Rezultat mnozenja je:}#
#\izlaz{87 64}#
#\izlaz{2 24}#
#\izlaz{145 83}#
\end{upotreba}
\end{miditest}
\begin{miditest}
\begin{upotreba}{2}
#\naslovInt#
#\izlaz{Unesite broj vrsta i broj kolona matrice a:}#
#\ulaz{5 2}#
#\izlaz{Unesite elemente matrice:}#
#\ulaz{1 7}#
#\ulaz{9 0}#
#\ulaz{-10 2}#
#\ulaz{92 3}#
#\ulaz{14 -8}#
#\izlaz{Unesite broj vrsta i broj kolona matrice b:}#
#\ulaz{2 4}#
#\izlaz{Unesite elemente matrice:}#
#\ulaz{7 8 9 10}#
#\ulaz{-11 2 34 78}#
#\izlaz{Rezultat mnozenja je:}#
#\izlaz{-70 22 247 556}#
#\izlaz{63 72 81 90}#
#\izlaz{-92 -76 -22 56}# 
#\izlaz{611 742 930 1154}# 
#\izlaz{186 96 -146 -484}#
\end{upotreba}
\end{miditest}

\ssstrana
\estrana
\skrati{3}
\eeskrati{3}
\begin{miditest}
\begin{upotreba}{3}
#\naslovInt#
#\izlaz{Unesite broj vrsta i broj kolona matrice a:}#
#\ulaz{3 4}#
#\izlaz{Unesite elemente matrice:}#
#\ulaz{1 2 8 9}#
#\ulaz{-4 5 2 3}#
#\ulaz{7 6 4 10}#
#\izlaz{Unesite broj vrsta i broj kolona matrice b:}#
#\ulaz{5 2}#
#\izlaz{Mnozenje matrica nije moguce.}#
\end{upotreba}
\end{miditest}
\begin{miditest}
\begin{upotreba}{4}
#\naslovInt#
#\izlaz{Unesite broj vrsta i broj kolona matrice a:}#
#\ulaz{-3 4}#
#\izlaz{Greska: neispravan unos.}#
\end{upotreba}
\end{miditest}
\eeskrati{3}
\linkresenje{mat.16}
\end{Exercise}
\ifresenja
\skrati{3}
\begin{Answer}[ref=mat.16]
\includecode{resenja/2_NapredniTipoviPodataka/2.7_VisedimenzioniNizovi/matrice_19.c}
\skrati{3}
\end{Answer}
\fi


%\skrati{3}
\eeskrati{3}
\begin{Exercise}[difficulty=1, label=mat.17] 
Element matrice naziva se \emph{sedlo} ako je istovremeno najmanji u
svojoj vrsti, a najveći u svojoj koloni. Napisati program koji
ispisuje indekse i vrednosti onih elemenata matrice realnih brojeva
koji su sedlo. Maksimalna dimenzija matrice je $50\times 50$.
U slučaju neispravnog unosa, ispisati odgovarajuću poruku o grešci. 

\eeskrati{3}
\skrati{3}
\begin{minitest}
\begin{upotreba}{1}
#\naslovInt#
#\izlaz{Unesite broj vrsta i}#
#\izlaz{broj kolona matrice:}#
#\ulaz{2 3}#
#\izlaz{Unesite elemente matrice:}#
#\ulaz{1 2 3}#
#\ulaz{0 5 6}#
#\izlaz{Sedlo: 0 0 1}#
\end{upotreba}
\end{minitest}
\begin{minitest}
\begin{upotreba}{2}
#\naslovInt#
#\izlaz{Unesite broj vrsta i}#
#\izlaz{broj kolona matrice:}#
#\ulaz{3 3}#
#\izlaz{Unesite elemente matrice:}#
#\ulaz{10 3 20}#
#\ulaz{15 5 100}#
#\ulaz{30 -1 200}#
#\izlaz{Sedlo: 1 1 5}#
\end{upotreba}
\end{minitest}
\begin{minitest}
\begin{upotreba}{2}
#\naslovInt#
#\izlaz{Unesite broj vrsta i}#
#\izlaz{broj kolona matrice:}#
#\ulaz{3 -3}#
#\izlaz{Greska: neispravan unos.}#
\end{upotreba}
\end{minitest}
\eeskrati{3}
\linkresenje{mat.17}
\end{Exercise}
\ifresenja
\skrati{3}
\begin{Answer}[ref=mat.17]
\includecode{resenja/2_NapredniTipoviPodataka/2.7_VisedimenzioniNizovi/matrice_20.c}
\skrati{3}
\end{Answer}
\fi


\skrati{2}
\begin{Exercise}[difficulty=1, label=mat.18] 
Napisati program koji ispisuje elemente matrice celih brojeva u
spiralnom redosledu počevši od gornjeg levog ugla krećući se u smeru
kazaljke na satu. Maksimalna dimenzija matrice je
$50\times 50$.
U slučaju neispravnog unosa, ispisati odgovarajuću poruku o grešci. 

\skrati{2}
\begin{miditest}
\begin{upotreba}{1}
#\naslovInt#
#\izlaz{Unesite broj vrsta i}#
#\izlaz{broj kolona matrice:}#
#\ulaz{3 3}#
#\izlaz{Unesite elemente matrice:}#
#\ulaz{1 2 3}#
#\ulaz{4 5 6}#
#\ulaz{7 8 9}#
#\izlaz{Ispis je:}#
#\izlaz{1 2 3 6 9 8 7 4 5}#
\end{upotreba}
\end{miditest}
\begin{miditest}
\begin{upotreba}{2}
#\naslovInt#
#\izlaz{Unesite broj vrsta i}#
#\izlaz{broj kolona matrice:}#
#\ulaz{5 7}#
#\izlaz{Unesite elemente matrice:}#
#\ulaz{7  -8  1 2 3 -54 87}#
#\ulaz{90 11  0 5 4 9 18}#
#\ulaz{12 -9  14  23 8 -22 74}#
#\ulaz{80  6  88  17 62 38 41}#
#\ulaz{-22 10 44  57 -200 39 55}#
#\izlaz{Ispis je:}#
#\izlaz{7 -8 1 2 3 -54 87 18 74 41 55}#
#\izlaz{39 -200 57 44 10 -22 80 12 90}#
#\izlaz{11 0 5 4 9 -22 38 62 17 88 6}#
#\izlaz{-9 14 23 8}#
\end{upotreba}
\end{miditest}

\linkresenje{mat.18}
\end{Exercise}
\ifresenja
\skrati{3}
\begin{Answer}[ref=mat.18]
\includecode{resenja/2_NapredniTipoviPodataka/2.7_VisedimenzioniNizovi/matrice_21.c}
\skrati{3}
\end{Answer}
\fi


\ssstrana
\estrana
\begin{Exercise}[difficulty=1, label=mat.19] 
Matrica $a$ se sadrži u matrici $b$ ukoliko postoji podmatrica matrice
$b$ identična matrici $a$.  Napisati program koji za dve učitane
matrice celih brojeva proverava da li se druga matrica sadrži u prvoj
učitanoj matrici. Maksimalna dimenzija matrica je $50\times 50$.
U slučaju neispravnog unosa, ispisati odgovarajuću poruku o grešci. 

\skrati{3}
\begin{miditest}
\begin{upotreba}{1}
#\naslovInt#
#\izlaz{Unesite broj vrsta i broj kolona matrice A:}#
#\ulaz{3 4}#
#\izlaz{Unesite elemente matrice:}#
#\ulaz{1 2 8 9}#
#\ulaz{-4 5 2 3}#
#\ulaz{7 6 4 10}#
#\izlaz{Unesite broj vrsta i broj kolona matrice B:}#
#\ulaz{2 2}#
#\izlaz{Unesite elemente matrice:}#
#\ulaz{2 3}#
#\ulaz{4 10}#
#\izlaz{Druga matrica je sadrzana u prvoj matrici.}#
\end{upotreba}
\end{miditest}
\begin{miditest}
\begin{upotreba}{2}
#\naslovInt#
#\izlaz{Unesite broj vrsta i broj kolona matrice A:}#
#\ulaz{3 4}#
#\izlaz{Unesite elemente matrice:}#
#\ulaz{1 2 8 9}#
#\ulaz{-4 5 2 3}#
#\ulaz{7 6 4 10}#
#\izlaz{Unesite broj vrsta i broj kolona matrice B:}#
#\ulaz{2 2}#
#\izlaz{Unesite elemente matrice:}#
#\ulaz{2 8}#
#\ulaz{6 4}#
#\izlaz{Druga matrica nije sadrzana }#
#\izlaz{u prvoj matrici.}#
\end{upotreba}
\end{miditest}

\skrati{3}
\begin{miditest}
\begin{upotreba}{3}
#\naslovInt#
#\izlaz{Unesite broj vrsta i broj kolona matrice A:}#
#\ulaz{5 5}#
#\izlaz{Unesite elemente matrice:}#
#\ulaz{7  -8  1 2 3}#
#\ulaz{90 11  0 5 4}#
#\ulaz{12 -9  14  23 8}#
#\ulaz{80  6  88  17 62}#
#\ulaz{-22 10 44  57 -200}#
#\izlaz{Unesite broj vrsta i broj kolona matrice B:}#
#\ulaz{3 4}#
#\izlaz{Unesite elemente matrice:}#
#\ulaz{90 11 0 5}#
#\ulaz{12 -9 14 23}#
#\ulaz{80 6 88 17}#
#\izlaz{Druga matrica je sadrzana u prvoj matrici.}#
\end{upotreba}
\end{miditest}
\begin{miditest}
\begin{upotreba}{4}
#\naslovInt#
#\izlaz{Unesite broj vrsta i broj kolona matrice A:}#
#\ulaz{5 5}#
#\izlaz{Unesite elemente matrice:}#
#\ulaz{7  -8  1 2 3}#
#\ulaz{90 11  0 5 4}#
#\ulaz{12 -9  14  23 8}#
#\ulaz{80  6  88  17 62}#
#\ulaz{-22 10 44  57 -200}#
#\izlaz{Unesite broj vrsta i broj kolona matrice B:}#
#\ulaz{53 4}#
#\izlaz{Greska: neispravan unos.}#
\end{upotreba}
\end{miditest}

\linkresenje{mat.19}
\end{Exercise}
\ifresenja
\skrati{3}
\begin{Answer}[ref=mat.19]
\includecode{resenja/2_NapredniTipoviPodataka/2.7_VisedimenzioniNizovi/matrice_22.c}
\end{Answer}
\fi


%------------------------------------------------------------------------------------------------------------------------------

\ifresenja
\sstrana
\section{Rešenja}
\shipoutAnswer
\fi

