\section{Višedimenzioni nizovi}

\begin{Exercise}[label=p1.2_] 
\begin{description}
  \item[a)] Napisati funkciju
    \\ \verb|int refleksivna(int a[][MAX], int n)|\\ kojom se za
    relaciju zadatom matricom \verb|a| (matruca je kvadratna) ispitije
    da li je refleksivna.
  \item[b)] Napisati funkciju
    \\ \verb|int simetricna(int a[][MAX], int n)|\\ kojom se za
    relaciju zadatom matricom \verb|a| (matruca je kvadratna) ispitije
    da li je simetricna.
  \item[c)] Napisati funkciju
    \\ \verb|int tranzitivna(int a[][MAX], int n)|\\ kojom se za
    relaciju zadatom matricom \verb|a| (matruca je kvadratna) ispitije
    da li je tranzitivna. \\
 
    Dva elementa \verb|i| i \verb|j| (\verb|i@j|) su u relaciji akko
    a[i][j] = 1\\ Relacija je refleksivana ako sa svako \verb|i| va\v
    zi: \verb|i@i=1|\\ Relacija je simetricna ako za svako \verb|i| i
    \verb|j| va\v zi: \verb|i@j=1| $=>$ \verb|j@i=1|\\ Relacija je
    tranzitivna ako za svako \verb|i|, \verb|j| i \verb|k| va\v zi:
    \verb|i@j=1| i \verb|j@k=1| $=>$ \verb|i@k=1|\\ Funkcija postavlja
    na 1 odgovarajuci indikator.

  \item[b)] Sa standardnog ulaza prvo se unose dimenzija kvadratne
    matrice \verb|n|, a nakon toga elementi matrice.  U\v citati
    matricu, i ispitati da li je relacija koju predstavlja relacija
    ekvivalencije (refleksivna, simetri\v cna i tranzitivna).
\end{description}
\linkresenje{p1.2_}
\end{Exercise}
\begin{Answer}[ref=p1.2_]
%\includecode{resenja/1_KontrolaToka/1.2_NaredbeGrananja/1_14.c}
\end{Answer}

\begin{Exercise}[label=p1.2_] 
Napisati funkciju \verb|float sumD(float a[][max], int n)| koja odre\d
uje sumu elememenata iznad glavne dijagonale. Potom napisati funkciju
\verb|float sumd(float a[][max], int n)| koja odre\d uje sumu
elememenata ispod glavne dijagonale.  Funkciju testirati pozivom u
main-u. Matrica je maksimalne dimenzije 50x50. Matrica je kvadratna.
\linkresenje{p1.2_}
\end{Exercise}
\begin{Answer}[ref=p1.2_]
%\includecode{resenja/1_KontrolaToka/1.2_NaredbeGrananja/1_14.c}
\end{Answer}


\begin{Exercise}[label=p1.2_] 
Napisati
funkciju\\ \verb|void transponovana(float a[][max], int m, int n, float b[][max])|
koja odre\d uje transponovanu matricu matricu.  Pozivom u main-u
testirati funkciju. Matrica je maksimalne dimenzije 50x50.
\linkresenje{p1.2_}
\end{Exercise}
\begin{Answer}[ref=p1.2_]
%\includecode{resenja/1_KontrolaToka/1.2_NaredbeGrananja/1_14.c}
\end{Answer}

\begin{Exercise}[label=p1.2_] 
Napisati funkciju\\
\verb|void mnozenje(int a[][max], int n, int m, int b[][max], int k,|\\
\verb|int t, int c[][max])|\\
koja ra\v cuna 
proizvod dve matrice. 
Pozivom u main-u testirati funkciju. Matrica je maksimalne dimenzije 50x50. Testirati da li su
podaci korektno uneti i testirati da li je mogu\'ce matrice mno\v ziti.
\linkresenje{p1.2_}
\end{Exercise}
\begin{Answer}[ref=p1.2_]
%\includecode{resenja/1_KontrolaToka/1.2_NaredbeGrananja/1_14.c}
\end{Answer}


\begin{Exercise}[label=p1.2_] 
Napisati funkciju u kojoj se razmenjuju elemeti k-te i t-te vrste
matrice(k i t su argumenti funkcije). Funkciju testirati pozivom u
main-u i ispisom novodobijene matrice na standarni izlaz. Sa
standarnog ulaza u\v citavaju se dimenzije matrice, a potom i elementi
matrice i brojevi k i t. Maksimalna dimenzija matrice je
50x50. Funkciju testirati pozivom u main-u.
\linkresenje{p1.2_}
\end{Exercise}
\begin{Answer}[ref=p1.2_]
%\includecode{resenja/1_KontrolaToka/1.2_NaredbeGrananja/1_14.c}
\end{Answer}


\begin{Exercise}[label=p1.2_] 
Sa standarnog ulaza unose se celi pozitivni brojevi {\tt m} i {\tt n}
koji ozna\v cavaju broj vrsta i broj kolona matrice.  Potom se unose
elementi matrice. Nakon unosa elemenata matrice, unose se jo\v s dva
broja {\tt p} i {\tt k} ($p \le m$, $k \le n$).  Na standari izlaz
ispisati sume svih podmatrica (dimenzije $p \times k$) unete
matrice. U slu\v caju gre\v ske ispisati {\tt -1}. \\ {\bf Napomena
  1:} Ne razmatrati slu\v{c}aj negativnih brojeva. \\ {\bf Napomena
  2:} Nije bitan redosled kojim se ispisuju sume. \\
\begin{miditest}
\begin{upotreba}{1}
#\naslovInt#
#\ulaz{3 4}#
#\ulaz{1 2 3 4}#
#\ulaz{5 6 7 8}#
#\ulaz{9 10 11 12}#
#\ulaz{3 3}#
#\izlaz{54 63}#
\end{upotreba}
\end{miditest}
\begin{miditest}
\begin{upotreba}{2}
#\naslovInt#
#\ulaz{3 4}#
#\ulaz{1 2 3 4}#
#\ulaz{5 6 7 8}#
#\ulaz{9 10 11 12}#
#\ulaz{2 3}#
#\izlaz{24 30 48 54}#
\end{upotreba}
\end{miditest}
\begin{miditest}
\begin{upotreba}{3}
#\naslovInt#
#\ulaz{3 2}#
#\ulaz{1 2}#
#\ulaz{3 4}#
#\ulaz{5 6}#
#\ulaz{7 8}#
#\izlaz{-1}#
\end{upotreba}
\end{miditest}
\begin{miditest}
\begin{upotreba}{4}
#\naslovInt#
#\ulaz{5 3}#
#\ulaz{1 1 2}#
#\ulaz{5 0 2}#
#\ulaz{7 8 9}#
#\ulaz{1 2 4}#
#\ulaz{0 1 1}#
#\ulaz{2 2}#
#\izlaz{7 5 20 19 18 23 4 8}#
\end{upotreba}
\end{miditest}
\linkresenje{p1.2_}
\end{Exercise}
\begin{Answer}[ref=p1.2_]
%\includecode{resenja/1_KontrolaToka/1.2_NaredbeGrananja/1_14.c}
\end{Answer}


\begin{Exercise}[label=p1.2_] 
Sa standarnog ulaza zadata je dimenzija kvadratne matrice $n$ ($0 < n
\leq 50$), a zatim i vrednosti pojedina\v cnih elemenata. Ukoliko je
$n$ izvan ovog opsega ispisati $-1$ i prekinuti izvr\v{s}avanje
programa.  Napisati program koji:
\begin{itemize}
\item[(a)] U\v citava matricu i ispisuje je na izlaz. U slu\v{c}aju
  gre\v{s}ke ispisati -1 i prekinuti izvr\v{s}avanje programa.
\item[(b)] Ispituje da li su elementi matrice po kolonama, vrstama i
  dijagonalama (glavnoj i sporednoj) sortirani strogo rastu\'ce. Za
  svaki od ovih slu\v{c}ajeva redom ispisati $1$ ako jesu i $0$ ako
  nisu sortirani - videti primere.
\end{itemize}
\begin{miditest}
\begin{upotreba}{1}
#\naslovInt#
#\ulaz{3}#
#\ulaz{1 2 3}#
#\ulaz{4 5 6}#
#\ulaz{7 8 9}#
#\izlaz{1 2 3}#
#\izlaz{4 5 6}#
#\izlaz{7 8 9}#
#\izlaz{1 1 1}#
\end{upotreba}
\end{miditest}
\begin{miditest}
\begin{upotreba}{2}
#\naslovInt#
#\ulaz{2}#
#\ulaz{6 9}#
#\ulaz{4 10}#
#\izlaz{6 9}#
#\izlaz{4 10}#
#\izlaz{0 1 0}#
\end{upotreba}
\end{miditest}
\begin{miditest}
\begin{upotreba}{3}
#\naslovInt#
#\ulaz{4}#
#\ulaz{5 5 7 9}#
#\ulaz{6 10 11 13}#
#\ulaz{8 12 14 15}#
#\ulaz{13 15 16 20}#
#\izlaz{5 5 7 9}#
#\izlaz{6 10 11 13}#
#\izlaz{8 12 14 15}#
#\izlaz{13 15 16 20}#
#\izlaz{1 0 1}#
\end{upotreba}
\end{miditest}
\begin{miditest}
\begin{upotreba}{4}
#\naslovInt#
#\ulaz{1}#
#\ulaz{5}#
#\izlaz{5}#
#\izlaz{1 1 1}#
\end{upotreba}
\end{miditest}
\linkresenje{p1.2_}
\end{Exercise}
\begin{Answer}[ref=p1.2_]
%\includecode{resenja/1_KontrolaToka/1.2_NaredbeGrananja/1_14.c}
\end{Answer}


\begin{Exercise}[label=p1.2_] 
Sa standarnog ulaza se unosi broj $n$ ($0 < n \le 10$), a potom i
elementi kvadratne matrice dimenzije $n\times n$.  Elementi matrice su
celi brojevi. Proveriti da li va\v{z}i da su zbirovi elemenata kolona
matrice uredjeni u strogo rastu\'{c}em poretku.  {\bf Napomena 1:}
Ukoliko program uvek ispisuje \verb|da| ili uvek ispisuje \verb|ne|
smatra\'ce se neta\v cnim i poeni se ne mogu osvojiti. \\
\begin{miditest}
\begin{upotreba}{1}
#\naslovInt#
#\ulaz{4}#
#\ulaz{1 0 0 0}#
#\ulaz{0 0 1 0}#
#\ulaz{0 0 0 1}#
#\ulaz{0 1 0 0}#
#\izlaz{ne}#
\end{upotreba}
\end{miditest}
\begin{miditest}
\begin{upotreba}{2}
#\naslovInt#
#\ulaz{3}#
#\ulaz{1 2 3}#
#\ulaz{4 5 6}#
#\ulaz{7 8 9}#
#\izlaz{da}#
\end{upotreba}
\end{miditest}
\begin{miditest}
\begin{upotreba}{3}
#\naslovInt#
#\ulaz{3}#
#\ulaz{2 -2 1}#
#\ulaz{1 2 2}#
#\ulaz{2 1 -2}#
#\izlaz{ne}#
\end{upotreba}
\end{miditest}
\begin{miditest}
\begin{upotreba}{4}
#\naslovInt#
#\ulaz{5}#
#\ulaz{-1 0 2 0 20}#
#\ulaz{0 0 0 10 0}#
#\ulaz{0 0 -1 0 0}#
#\ulaz{0 1 0 0 0}#
#\ulaz{0 0 0 0 -1}#
#\izlaz{da}#
\end{upotreba}
\end{miditest}
\linkresenje{p1.2_}
\end{Exercise}
\begin{Answer}[ref=p1.2_]
%\includecode{resenja/1_KontrolaToka/1.2_NaredbeGrananja/1_14.c}
\end{Answer}


\begin{Exercise}[label=p1.2_] 
Sa standarnog ulaza unosi se broj $n$ ($0 < n \le 200$), a potom i
elementi kvadratne matrice dimenzije $n\times n$.  Elementi matrice su
celi brojevi.  Proveriti da li je uneta matrica ortonormirana i na
standarni izlaz ispisati \verb|da| ako jeste ili \verb|ne| ako nije
ortonormirana. Matrica je ortonormirana ako je skalarni proizvod
svakog para razli\v citih vrsta jednak 0, a skalarni proizvod vrste sa
samom sobom 1.  U slu\v caju gre\v ske ispisati -1.\\ {\bf Napomena
  1:} Skalarni proizvod vektora $a = (a_1, a_2, \ldots, a_n)$ i $b =
(b_1, b_2, \ldots, b_n)$ je $a_1\cdot b_1 + a_2\cdot b_2 + \ldots +
a_n\cdot b_n$.\\ {\bf Napomena 2:} Ukoliko program uvek ispisuje
\verb|da| ili uvek ispisuje \verb|ne| smatra\'ce se neta\v cnim i
poeni se ne mogu osvojiti. \\
\begin{miditest}
\begin{upotreba}{1}
#\naslovInt#
#\ulaz{4}#
#\ulaz{1 0 0 0}#
#\ulaz{0 0 1 0}#
#\ulaz{0 0 0 1}#
#\ulaz{0 1 0 0}#
#\izlaz{da}#
\end{upotreba}
\end{miditest}
\begin{miditest}
\begin{upotreba}{2}
#\naslovInt#
#\ulaz{3}#
#\ulaz{1 2 3}#
#\ulaz{4 5 6}#
#\ulaz{7 8 9}#
#\izlaz{ne}#
\end{upotreba}
\end{miditest}
\begin{miditest}
\begin{upotreba}{3}
#\naslovInt#
#\ulaz{3}#
#\ulaz{2 -2 1}#
#\ulaz{1 2 2}#
#\ulaz{2 1 -2}#
#\izlaz{ne}#
\end{upotreba}
\end{miditest}
\begin{miditest}
\begin{upotreba}{4}
#\naslovInt#
#\ulaz{5}#
#\ulaz{-1 0 2 0 20}#
#\ulaz{0 0 0 10 0}#
#\ulaz{0 0 -1 0 0}#
#\ulaz{0 1 0 0 0}#
#\ulaz{0 0 0 0 -1}#
#\izlaz{da}#
\end{upotreba}
\end{miditest}
\linkresenje{p1.2_}
\end{Exercise}
\begin{Answer}[ref=p1.2_]
%\includecode{resenja/1_KontrolaToka/1.2_NaredbeGrananja/1_14.c}
\end{Answer}


\begin{Exercise}[label=p1.2_] 
Napisati funkciju koja kao argumente prima kvadratnu matricu celih
brojeva i njenu dimenziju, a vra\'ca 1 ako je matrica donja trougaona,
odnosno 0 ako nije. Pretpostavka je da je maksimalna dimenzija matrice
100. Matrica je donja trougaona ako se u gornjem trouglu (iznad glavne
dijagonale, ne uklju\v cuju\' ci je) nalaze sve nule.
\linkresenje{p1.2_}
\end{Exercise}
\begin{Answer}[ref=p1.2_]
%\includecode{resenja/1_KontrolaToka/1.2_NaredbeGrananja/1_14.c}
\end{Answer}


\begin{Exercise}[label=p1.2_] 
Napisati program koji sa standardnog ulaza unosi prvo dimenziju
matrice ($n<10$) pa zatim elemente matrice i izra\v cunava sumu
elemenata iznad sporedne dijagonale matrice.
\linkresenje{p1.2_}
\end{Exercise}
\begin{Answer}[ref=p1.2_]
%\includecode{resenja/1_KontrolaToka/1.2_NaredbeGrananja/1_14.c}
\end{Answer}


\begin{Exercise}[label=p1.2_] 
Za datu kvadratnu matricu ka\v zemo da je \emph{magi\v cni
kvadrat} ako je suma elemenata u svakoj koloni i svakoj vrsti
jednaka. Napisati program koji sa standardnog ulaza u\v citava
prirodni broj $n$ ($n<10$) i zatim elemente kvadratne matrice,
proverava da li je ona \emph{magi\v cni kvadrat} i ispisuje
odgovaraju\' cu poruku na standardni izlaz. \\
\begin{miditest}
\begin{upotreba}{1}
#\naslovInt#
#\ulaz{4}#
#\ulaz{1 5 3 1}#
#\ulaz{2 1 2 5}#
#\ulaz{3 2 2 3}#
#\ulaz{4 2 3 1}#
#\izlaz{da}#
\end{upotreba}
\end{miditest}
\linkresenje{p1.2_}
\end{Exercise}
\begin{Answer}[ref=p1.2_]
%\includecode{resenja/1_KontrolaToka/1.2_NaredbeGrananja/1_14.c}
\end{Answer}


\begin{Exercise}[label=p1.2_] 
Napisati program koji sa standardnog ulaza u\v citava prvo dimenzije
matrice ($n$ i $m$) a zatim redom i elemente matrice (ne postoje
pretpostavke o dimenziji matrice). Nakon toga na standardni izlaz,
zapisati indekse ($i$ i $j$) onih elemenata matrice koji su jednaki
zbiru svih svojih susednih elemenata (pod susednim elementima
podrazumevamo okolnih $8$ polja matrice ako postoje). \\
\begin{miditest}
\begin{upotreba}{1}
#\naslovInt#
#\ulaz{4 5}#
#\ulaz{1 1 2 1 3}#
#\ulaz{0 8 1 9 0}#
#\ulaz{1 1 1 0 0}#
#\ulaz{0 3 0 2 2}#
#\izlaz{1 1}#
#\izlaz{1 3}#
#\izlaz{3 2}#
#\izlaz{3 4}#
\end{upotreba}
\end{miditest}
\linkresenje{p1.2_}
\end{Exercise}
\begin{Answer}[ref=p1.2_]
%\includecode{resenja/1_KontrolaToka/1.2_NaredbeGrananja/1_14.c}
\end{Answer}


\begin{Exercise}[label=p1.2_] 
Sa standarnog ulaza se zadaje prvo dimenziju kvadratne matrice $n$ ($n
< 100$), a zatim elemente matrice.  Nakon toga, na standardni izlaz
ispisati redni broj kolone koja ima najve\' ci zbir elemenata. \\
\begin{miditest}
\begin{upotreba}{1}
#\naslovInt#
#\ulaz{3}#
#\ulaz{1 2 3}#
#\ulaz{7 3 4}#
#\ulaz{5 3 1}#
#\izlaz{0}#
\end{upotreba}
\end{miditest}
\linkresenje{p1.2_}
\end{Exercise}
\begin{Answer}[ref=p1.2_]
%\includecode{resenja/1_KontrolaToka/1.2_NaredbeGrananja/1_14.c}
\end{Answer}


\begin{Exercise}[label=p1.2_] 
Napisati funkciju koja treba da ispi\v{s}e elemente matrice u grupama
koje su paralelne sa sporednom dijagonalom matrice. Mo\v{z}e se
pretpostaviti da matrica nije dimenzije ve\'ce od $100 \times 100$. \\
\begin{miditest}
\begin{upotreba}{1}
#\naslovInt#
#\ulaz{3}#
#\ulaz{1 2 3}#
#\ulaz{4 5 6}#
#\ulaz{7 8 9}#
#\izlaz{1}#
#\izlaz{2 4}#
#\izlaz{3 5 7}#
#\izlaz{6 8}#
#\izlaz{9}#
\end{upotreba}
\end{miditest}
\linkresenje{p1.2_}
\end{Exercise}
\begin{Answer}[ref=p1.2_]
%\includecode{resenja/1_KontrolaToka/1.2_NaredbeGrananja/1_14.c}
\end{Answer}


\begin{Exercise}[label=p1.2_] 
Sa standarnog ulaza u\v citava se broj $n$, a zatim i kvadratna
matrica koja sadr\v zi brojeve tipa \verb|double| dimenzije $n\times
n$. Napisati program koji izra\v cunava i ispisuje razliku (na dve
decimale) izmedju zbira elemenata gornjeg trougla i zbira elemenata
donjeg trougla matrice -- gornji trougao \v cine svi elementi iznad
sporedne dijagonale (ne ra\v cunaju\' ci dijagonalu), a donji trougao
\v cine svi elementi ispod sporedne dijagonale (ra\v cunaju\' ci
dijagonalu). U slu\v caju gre\v ske u datoteku upisati \verb|GRESKA|.
\begin{miditest}
\begin{upotreba}{1}
#\naslovInt#
#\ulaz{3}#
#\ulaz{2 3.2 4}#
#\ulaz{7 8.8 1}#
#\ulaz{2.3 1 1}#
#\izlaz{-2.10}#
\end{upotreba}
\end{miditest}
\begin{miditest}
\begin{upotreba}{2}
#\naslovInt#
#\ulaz{4}#
#\ulaz{2.3 1 12 8}#
#\ulaz{4 -8.2 7 14.5}#
#\ulaz{1 -2.5 9 11}#
#\ulaz{3 4.3 -5.7 2}#
#\izlaz{49.4}#
\end{upotreba}
\end{miditest}
\begin{miditest}
\begin{upotreba}{3}
#\naslovInt#
#\ulaz{-4}#
#\izlaz{GRESKA}#
\end{upotreba}
\end{miditest}
\linkresenje{p1.2_}
\end{Exercise}
\begin{Answer}[ref=p1.2_]
%\includecode{resenja/1_KontrolaToka/1.2_NaredbeGrananja/1_14.c}
\end{Answer}



\begin{Exercise}[label=p1.2_] 
Kao argumenti komandne linje zadate su dimenzije matrice \verb|A|
(\verb|m| i \verb|n|).  Element matrice se naziva sedlo ako je
istovremeno najmanji u svojoj vrsti, a najve\'ci u svojoj
koloni. Ispisati indekse i vrednosti onih elemenata matrice koji su
sedlo. Pretpostaviti da je maksimalna dimenzija matrice $50\times 50$.
Ukoliko nisu zadati svi potrebni argumenti komadne linije ispisati
poruku da je do\v slo do gre\v ske. Ukoliko su dimenzije van opsega
ispisati poruku o gre\v sci. \\
\begin{miditest}
\begin{upotreba}{1}
#\poziv{./a.out 2 3}#
#\naslovInt#
#\ulaz{1 2 3}#
#\ulaz{0 5 6}#
#\izlaz{0 0 1}#
\end{upotreba}
\end{miditest}
\begin{miditest}
\begin{upotreba}{2}
#\poziv{./a.out 3 3}#
#\naslovInt#
#\ulaz{10 3 20}#
#\ulaz{15 5 100}#
#\ulaz{30 -1 200}#
#\izlaz{1 1 5}#
\end{upotreba}
\end{miditest}
\begin{miditest}
\begin{upotreba}{3}
#\poziv{./a.out 3}#
#\naslovInt#
#\izlaz{greska}#
\end{upotreba}
\end{miditest}
\begin{miditest}
\begin{upotreba}{4}
#\poziv{./a.out 200 3}#
#\naslovInt#
#\izlaz{greska}#
\end{upotreba}
\end{miditest}
\linkresenje{p1.2_}
\end{Exercise}
\begin{Answer}[ref=p1.2_]
%\includecode{resenja/1_KontrolaToka/1.2_NaredbeGrananja/1_14.c}
\end{Answer}







\section{Rešenja}
\shipoutAnswer

