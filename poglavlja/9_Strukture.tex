
\section{Strukture}

\begin{Exercise}[label=v2.5_01] 
Data je struktura koja opisuje koordinate
tacke u ravni:

\begin{ckod}
typedef struct point
{
float x;
float y;
} POINT;
\end{ckod}

U glavnom programu date su dve tacke: tacka
A sa fiksiranim koordinatama (1,2) i tacka B
cije koordinate zadaje korisnik. Napisati
funkcije:\\
a) za racunanje rastojanja izmedju dve date tacke\\
b) za odredjivanje tacke koja se nalazi na
sredini duzi odredjene dvema datim tackama.

Testirati napisane funkcije u glavnom programu.

\linkresenje{v2.5_01}
\end{Exercise}
\begin{Answer}[ref=v2.5_01]
\includecode{resenja/2_PredstavljanjePodataka/2.5_Strukture/1/1.c}
\end{Answer}

\begin{Exercise}[label=v2.5_02] 
  Data je struktura
\begin{ckod}
    typedef struct Student
    {
      char ime[MAX];
      char prezime[MAX];
      char smer;
      float prosek;
    } STUDENT;
  \end{ckod}
  \begin{enumerate}
  
\item  Napisati funkciju koja ucitava sa standardnog ulaza podatke o studentu. Mozemo pretpostaviti da 
    ime i prezime studenta ne sadrze vise od 30 karaktera.
\item Napisati funkciju koja ispisuje podatke o studentu na standardni izlaz.
\item Ucitati niz od n studenata i :
\begin{enumerate}
\item ispisati imena i prezimena onih koji su na smeru R
      \item ispisati podatke za studenta sa najvecim prosekom; ako ima vise takvih studenata, ispisati 
\begin{enumerate}
         \item sve njih 
	 \item prvog 
	 \item poslednjeg 
\end{enumerate}
\end{enumerate}
  \end{enumerate}

\linkresenje{v2.5_02}
\end{Exercise}
\begin{Answer}[ref=v2.5_02]
\includecode{resenja/2_PredstavljanjePodataka/2.5_Strukture/2/2.c}
\end{Answer}

\begin{Exercise}[label=v2.5_03] 

Napisati program koji ucitava reci sa standardnog ulaza dok korisnik ne zada EOF i ispisuje
ih na standardni izlaz svaku u posebnom redu, poravnatu udesno u odnosu 
na poslednji karakter najduze reci. Koristiti
\begin{ckod}
strukturu typedef struct rec
	  {
	     char s[21];
	     int duzina;
          }REC;
\end{ckod}
Na primer, ako su unesene sledece reci:
\begin{ckod}
Danas imamo ispit iz programiranja1.
Nadam se da nece biti tesko!
\end{ckod}
onda ispis izgleda ovako:
\begin{ckod}
          Danas 
          imamo 
          ispit 
             iz 
programiranja1.
          Nadam 
             se 
             da 
           nece 
           biti 
         tesko!
\end{ckod}

Program realizovati kroz sledece funkcije:
\begin{enumerate}
\item Funkciju za ucitavanje jedne reci u strukturu REC.
\item Funkciju za ucitavanje niza struktura koja vraca dimenziju niza
\item Funkciju koja odredjuje maksimalnu duzinu reci u datom nizu
\item Funkciju koja ispisuje reci u trazenom formatu
\end{enumerate}

Mozemo pretpostaviti da nijedna rec ne sadrzi vise od 30 karaktera i da nece biti
uneto vise od 1000 reci.

\linkresenje{v2.5_03}
\end{Exercise}
\begin{Answer}[ref=v2.5_03]
\includecode{resenja/2_PredstavljanjePodataka/2.5_Strukture/3/3.c}
\end{Answer}

\begin{Exercise}[label=v2.5_04] 
   Napisati program koji izracunava prosecnu cenu jedne potrosacke
   korpe. Potrosacka korpa se sastoji od broja kupljenih artikala i 
   niza kupljenih artikala. Svaki artikal odredjen je svojim nazivom, 
   kolicinom i cenom. Program treba da ucita broj potrosaca n (najvise 100), 
   zatim podatke za n potrosackih korpi i da na osnovu ucitanih podataka 
   izracuna prosecnu cenu potrosacke korpe. Ucitavanje se vrsi sa standarnog
   ulaza pri cemu se prvo zada broj artikala, a zatim za svaki artikal naziv, 
   kolicina i cena.  Mozemo pretpostaviti da nijedan 
   potrosac nece kupiti vise od 20 artikala, kao i da naziv svakog artikla
   sadrzi maksimalno 30 karaktera.
\linkresenje{v2.5_04}
\end{Exercise}
\begin{Answer}[ref=v2.5_04]
\includecode{resenja/2_PredstavljanjePodataka/2.5_Strukture/4/4.c}
\end{Answer}

\begin{Exercise}[label=v2.5_05] 
   Struktura IZRAZ opisuje numericki izraz nad celim brojevima koji se sastoji
   od dva celobrojna operanda, numericke operacije nad celim brojevima i
   vrednosti izraza:
   
   typedef struct izraz
   {
	char o;
	int x;
	int y;
   } IZRAZ;

\begin{enumerate}

   \item Napisati funkciju koja ispituje da li je dati izraz korektno 
   zadat i vraca 1 ako jeste a 0 u suprotnom. Podrazumevamo da je 
   izraz korektno zadat ako operacija odgovara +,-,* ili / i u slucaju
   deljenja drugi operand je razlicit od 0.
   
   \item Napisati funkciju koja za dati izraz odredjuje vrednost izraza. 
   
   \item Napisati funkciju koja ucitava dati izraz. Funkcija
   treba da ucita sa standardnog ulaza operaciju i dva
   operanda u polja o, x i y strukture IZRAZ. Funkcija vraca
   1 ako je ucitavanje bilo uspesno, tj. ako je izraz bio
   korektno zadat ili 0 u suprotnom. 
   
   \item Napisati funkciju koja stampa dati izraz infiksno, u obliku
   x o y = vr. Na primer, za izraz + 4 17 ispis treba 
   da bude 4+17=21
   
  
\end{enumerate}
Napisati glavni program koji ucitava prirodan broj $n<1000$ a zatim n izraza
   u notaciji \\
   + 4 17 \\
   - 8 -16 \\
   Program treba da ispise maksimalnu vrednost medju unetim izrazima i da ispise one
   izraze cija je vrednost manja od polovine maksimalne vrednosti.
   	
\linkresenje{v2.5_05}
\end{Exercise}
\begin{Answer}[ref=v2.5_05]
\includecode{resenja/2_PredstavljanjePodataka/2.5_Strukture/5/5.c}
\end{Answer}



\begin{Exercise}[label=p2.5_01] 
Definisati strukturu kojom se predstavlja kompleksan broj. Napisati funkcije koje izračunavaju zbir, razliku, proizvod i količnik dva kompleksna broja, a zati i program koji učitava dva kompleksna broja i ispisuje vrednost zbira, razlike, proizvoda i količnika. \\
\begin{maxitest}
\begin{upotreba}{1}
#\naslovInt#
#\izlaz{Unesite realni i imaginarni deo prvog broja: }\ulaz{1 2}#
#\izlaz{Unesite realni i imaginarni deo drugog broja: }\ulaz{-2 3}#
#\izlaz{Zbir: -1.00+5.00*i}#
#\izlaz{Razlika: 3.00-1.00*i}#
#\izlaz{Proizvod: -8.00-1.00*i}#
#\izlaz{Kolicnik: 0.31-0.54*i}#
\end{upotreba}
\end{maxitest}
\linkresenje{p2.5_01}
\end{Exercise}
\begin{Answer}[ref=p2.5_01]
\includecode{resenja/2_PredstavljanjePodataka/2.5_Strukture/praktikumi13/1.c}
\end{Answer}

\begin{Exercise}[label=p2.5_02] 
Definisati strukturu $Lopta$ sa poljima $poluprecnik$ (ceo broj u centimetrima) i $boja$ (enumeracioni tip koji uključuje plavu, žutu, crvenu i zelenu boju). Zatim učitati informacije o $n$ lopti ($0<n<50$) i ispisati ukupnu zapreminu, kao i broj crvenih lopti. \textit{Napomena: probati sa testiranjem zadataka pomoću preusmeravanja.}\\
\begin{maxitest}
\begin{upotreba}{1}
#\naslovInt#
#\izlaz{Unesite broj lopti: }\ulaz{4}#
#\izlaz{Unesite dalje poluprecnike i boje lopti (1-plava, 2-zuta, 3-crvena, 4-zelena): }#
#\izlaz{1.lopta: }\ulaz{4 1}#
#\izlaz{2.lopta: }\ulaz{1 3}#
#\izlaz{3.lopta: }\ulaz{2 3}#
#\izlaz{4.lopta: }\ulaz{10 4}#
#\izlaz{Ukupna zapremina: 4494.57}#
#\izlaz{Broj crvenih lopti: 2}#
\end{upotreba}
\end{maxitest}

\linkresenje{p2.5_02}
\end{Exercise}
\begin{Answer}[ref=p2.5_02]
\includecode{resenja/2_PredstavljanjePodataka/2.5_Strukture/praktikumi13/2.c}
\end{Answer}

\begin{Exercise}[label=p2.5_03] 
 Zimi su prehlade česte i treba unositi više vitamina C. Struktura $Vocka$ sadrži ime voćke (nisku maksimalne dužine 20 karaktera) i količinu vitamina C u miligramima (realan broj). Napisati program koji sa standardnog ulaza učitava podatke o voćkama sve do unosa reči KRAJ i ispisuje ime voćke sa najviše vitamina C. Pretpostaviti da broj voćki neće biti veći od 50. \textit{Napomena: probati sa testiranjem zadataka pomoću preusmeravanja.}\\
\begin{maxitest}
\begin{upotreba}{1}
#\naslovInt#
#\izlaz{Unesite ime voćke i njenu količinu vitamina C: }\ulaz{jabuka 4.6}#
#\izlaz{Unesite ime voćke i njenu količinu vitamina C: }\ulaz{limun 51}#
#\izlaz{Unesite ime voćke i njenu količinu vitamina C: }\ulaz{kivi 92.7}#
#\izlaz{Unesite ime voćke i njenu količinu vitamina C: }\ulaz{banana 8.7}#
#\izlaz{Unesite ime voćke i njenu količinu vitamina C: }\ulaz{pomorandza 53.2}#
#\izlaz{Unesite ime voćke i njenu količinu vitamina C: }\ulaz{KRAJ}#
#\izlaz{Voce sa najvise C vitamina je: kivi}#
\end{upotreba}
\end{maxitest}

\linkresenje{p2.5_03}
\end{Exercise}
\begin{Answer}[ref=p2.5_03]
\includecode{resenja/2_PredstavljanjePodataka/2.5_Strukture/praktikumi13/3.c}
\end{Answer}

\begin{Exercise}[label=p2.5_04] 
 Deda Mraz planira kupovinu poklona za studente koji su vredno učili C u toku godine. Na njegovoj listi se nalazi ime i prezime studenta (niske dužina do 50 karaktera) i njegova želja (niska maksimalne dužine 100 karaktera). Napisati program koji će služiti Deda Mrazu kao podsetnik: na osnovu liste koju je napravio, Deda Mraz može da unese ime i prezime studenta i da proveri njegovu želju. Ako ima više studenata sa istim imenom i prezimenom ispisati sve želje. \textit{Napomena: probati sa testiranjem zadataka pomoću preusmeravanja.}\\
\begin{maxitest}
\begin{upotreba}{1}
#\naslovInt#
#\izlaz{Ime i prezime studenta:}#
#\ulaz{Pera Peric}#
#\izlaz{Njegova zelja:}#
#\ulaz{privezak za kljuceve}#
#\izlaz{Jos vrednih studenata (da/ne)?}#
#\ulaz{da}#
#\izlaz{Ime i prezime studenta:}#
#\ulaz{Zika Zikic}#
#\izlaz{Njegova zelja:}#
#\ulaz{stap za pecanje}#
#\izlaz{Jos vrednih studenata (da/ne)?}#
#\ulaz{da}#
#\izlaz{Ime i prezime studenta:}#
#\ulaz{Mara Maric}#
#\izlaz{Njegova zelja:}#
#\ulaz{komplet Knutovih knjiga}#
#\izlaz{Jos vrednih studenata (da/ne)?}#
#\ulaz{ne}#
#\izlaz{Za podsecanje uneti ime i prezime:}#
#\ulaz{Pera Peric}#
#\izlaz{Novogodisnja zelja: privezak za kljuceve}#
\end{upotreba}
\end{maxitest}

\linkresenje{p2.5_04}
\end{Exercise}
\begin{Answer}[ref=p2.5_04]
\includecode{resenja/2_PredstavljanjePodataka/2.5_Strukture/praktikumi13/4.c}
\end{Answer}

\begin{Exercise}[label=p2.5_05] 
  Definisati strukturu $Grad$ u kojoj se nalazi ime grada (niska dužine 20 karaktera) i prosečna temperatura u toku decembra (realan broj). Napisati program koji učitava imena $n$ ($0<n<50$) gradova i njihove prosečne temperature, a zatim ispisuje one gradove koji imaju idealnu temperaturu za klizanje: od 3 do 8 stepeni. \textit{Napomena: probati sa testiranjem zadataka pomoću preusmeravanja.}\\
\begin{maxitest}
\begin{upotreba}{1}
#\naslovInt#
#\izlaz{Unesite broj n:}\ulaz{4}#
#\izlaz{Unesite grad i temperaturu: }\ulaz{Beograd 7}#
#\izlaz{Unesite grad i temperaturu: }\ulaz{Uzice 1.5}#
#\izlaz{Unesite grad i temperaturu: }\ulaz{Subotica 4}#
#\izlaz{Unesite grad i temperaturu: }\ulaz{Zrenjanin 9}#
#\izlaz{Gradovi sa idealnom temperaturom za klizanje u decembru:}#
#\izlaz{Beograd}#
#\izlaz{Subotica}#
\end{upotreba}
\end{maxitest}

\begin{maxitest}
\begin{upotreba}{2}
#\naslovInt#
#\izlaz{Unesite broj n:}\ulaz{2}#
#\izlaz{Unesite grad i temperaturu: }\ulaz{Varsava 11}#
#\izlaz{Unesite grad i temperaturu: }\ulaz{Prag 2}#
#\izlaz{Gradovi sa idealnom temperaturom za klizanje u decembru:}#
\end{upotreba}
\end{maxitest}
 

\linkresenje{p2.5_05}
\end{Exercise}
\begin{Answer}[ref=p2.5_05]
\includecode{resenja/2_PredstavljanjePodataka/2.5_Strukture/praktikumi13/5.c}
\end{Answer}

\begin{Exercise}[label=p2.5_06] 
 Definisati strukturu $ParReci$ koja sadrži reč na srpskom jeziku i odgovarajući prevod na engleski jezik. Zatim sa standardnog ulaza sve do kraja ulaza učitavati parove reči i, posebno, za rečenicu koja se zadaje sa ulaza ispisati prevod - ako je reč u rečenici nepoznata umesto nje ispisati odgovarajući broj zvezdica. Reči neće biti duže od 50 karaktera, ukupan broj parova reči neće biti veći od 100, a ukupna dužina rečenice neće biti veća od 100 karaktera. \textit{Napomena: probati sa testiranjem zadataka pomoću preusmeravanja.}\\
\begin{miditest}
\begin{upotreba}{1}
#\naslovInt#
#\ulaz{zima winter}#
#\ulaz{godina year}#
#\ulaz{sreca happiness}#
#\ulaz{programiranje programming}#
#\ulaz{caj tea}#
#\izlaz{Unesite recenicu za prevod: }#
#\ulaz{piti caj zimi je sreca}#
#\izlaz{**** tea **** ** happiness}#
\end{upotreba}
\end{miditest}
\linkresenje{p2.5_06}
\end{Exercise}
\begin{Answer}[ref=p2.5_06]
\includecode{resenja/2_PredstavljanjePodataka/2.5_Strukture/praktikumi13/6.c}
\end{Answer}

 
\begin{Exercise}[label=p2.5_] 
Napisati funkcije koje izra\v cunavaju zbir, razliku i proizvod dva
razlomka, \verb|razlomak zbir(razlomak a, razlomak b)| itd. Unosi se
broj n a potom i n razlomaka sa standarnog ulaza (najvi\v se
100). Ispisati njihov zbir, raliku i proizvod na standardni izlaz.
\linkresenje{p2.5_}
\end{Exercise}
\begin{Answer}[ref=p2.5_]
%\includecode{resenja/2_PredstavljanjePodataka/2.5_Strukture/5/5.c}
\end{Answer}


\begin{Exercise}[label=p2.5_] 
Napraviti strukturu \verb|VOCE|
koja sadr\v zi ime (ne du\v ze od 20 karaktera)
i cenu (tipa \verb|float|). Sa standardnog ulaza unosi se broj vo\' cki (ne vi\' ci od 200), a
potom uneti niz vo\' ca i pozvati funkciju koja izra\v cunava prose\v cnu cenu vo\'ca.
Potom ispisati imena onih vo\'cki \v cija je cena ve\' ca od prose\v cne.
\linkresenje{p2.5_}
\end{Exercise}
\begin{Answer}[ref=p2.5_]
%\includecode{resenja/2_PredstavljanjePodataka/2.5_Strukture/5/5.c}
\end{Answer}


\begin{Exercise}[label=p2.5_] 
Sa standarnog ulaza u\v citava se $n$ ($0 < n \le 200$), a potom i 
spisak (du\v zine $n$) engleskih re\v{c}i i njihov prevod
na srpski jezik. Potom se u\v citava jedna re\v c sa standardnog ulaza. Na standardni izlaz ispisati
odgovaraju\' ci prevod date re\v ci ili podatak o tome da se re\v c ne 
nalazi na spisku.

\begin{verbatim}
apple jabuka
pineapple ananas
orange narandza
pear kruska
grape grozdje
\end{verbatim}

\noindent
i re\v c \emph{orange} program treba da ispi\v se \emph{narandza} a za re\v c
\emph{cherry} program treba da ispi\v se poruku \emph{Rec se ne nalazi u recniku}.
U programu se mogu koristiti funkcije iz zaglavlja \emph{string.h}.
\linkresenje{p2.5_}
\end{Exercise}
\begin{Answer}[ref=p2.5_]
%\includecode{resenja/2_PredstavljanjePodataka/2.5_Strukture/5/5.c}
\end{Answer}


\begin{Exercise}[label=p2.5_] 
Napisati program koji sa standardnog ulaza \v{c}itava najpre broj artikala
(ceo broj manji od $20$) a zatim podatke o artiklima. Artikli su vo\' cke
koje imaju po dva podatka: naziv vo\' cke i cenu (naziv vo\' cke je
karakterska niska du\v zine do $20$ karaktera).
Program potom tra\v zi od korisnika da unese neku cenu i \v stampa na
standardni izlaz sve vo\' cke koje imaju zadatu cenu.\\

Primer rada programa:

\begin{verbatim}
4
jabuka 30
kruska 40
ananas 60
limun 40

Unesite cenu: 40
Voce te cene je: kruska limun
\end{verbatim}
\linkresenje{p2.5_}
\end{Exercise}
\begin{Answer}[ref=p2.5_]
%\includecode{resenja/2_PredstavljanjePodataka/2.5_Strukture/5/5.c}
\end{Answer}


\begin{Exercise}[label=p2.5_] 
Definisati strukturu koja opisuje dete atributima 
ime deteta (ne vece od 20 karaktera) , pol deteta (m ili z)
i ocena. Ocenu je svako dete dalo radu obdani\v sta. Maksimalan broj dece je 100. Napisati program koji:

   \begin{description}
   \item{a)} Sa standarnog ulaza se unosi $n$, a potom podaci o $n$ dece. Koristiti strukturu:
   \begin{verbatim}
        typedef struct
        {
            char ime[20];
            char pol;
            int ocena;
        } DETE;
   \end{verbatim}

   \item{b)} ispisati na standarni izlaz statistiku: koliko ima de\v caka,
   a koliko devoj\v cica i prose\v cnu ocenu. Potom ispisuje imena dece
   brojnijeg pola.
   \end{description}
\linkresenje{p2.5_}
\end{Exercise}
\begin{Answer}[ref=p2.5_]
%\includecode{resenja/2_PredstavljanjePodataka/2.5_Strukture/5/5.c}
\end{Answer}


\begin{Exercise}[label=p2.5_] 
\begin{itemize}
\item Definisati tip podataka \verb|TACKA| pogodan za predstavljanje ta\v cke
Dekartovske ravni (\v{c}ije su $x$ i $y$ koordinate podaci tipa \verb|double|).

\item Definisati funkciju \verb|double rastojanje(TACKA a, TACKA b)| koja
izra\v cunava rastojanje izmedju dve ta\v{c}ke.

\item Definisati funkciju \verb|unsigned ucitaj_poligon(TACKA* tacke, unsigned n)|
koja u\v citava \verb|n| puta po dve vrednosti tipa \verb|double|
 (koje predstavljaju koordinate temena poligona) i upisuje ih u zadati niz
 ta\v{c}aka. Funkcija vra\' ca broj uspe\v sno u\v citanih ta\v{c}aka.

\item Definisati funkciju \verb|double obim(TACKA* poligon, unsigned n)|
koja izra\v cunava obim poligona sa \verb|n| ta\v{c}aka u zadatom nizu
(napomena: ne zaboraviti stranicu koja spaja poslednje i prvo teme).

\item Definisati funkciju \verb|double maksimalna_stranica(TACKA* poligon, unsigned n)|
koja izra\v cunava du\v{z}inu najdu\v{z}e stranice poligona sa \verb|n| ta\v{c}aka
u zadatom nizu (napomena: ne zaboraviti stranicu koja spaja poslednje i prvo teme).

\item Definisati funkciju \verb|main| u kojoj se sa standardnog ulaza
u\v{c}itava celobrojna nenegativna vrednost \verb|N| ($0 < N \le 100$).

Ina\v{c}e, poziva se funkcija \verb|ucitaj_poligon|. Ukoliko je
uspe\v{s}no u\v{c}itano \verb|m| ta\v{c}ka (\verb|N| ne mora da
bude jednako \verb|m|), onda se poziva funkcija \verb|obim|
za \verb|m| u\v{c}itanih ta\v{c}aka i ispisuje njen rezultat
na standardni izlaz (ukoliko ova funkcija nije implementirana
--- ispisati na standardni izlaz simbol \verb|?|).
Posle toga se poziva funkcija \verb|maksimalna_stranica|
za \verb|m| u\v{c}itanih ta\v{c}aka i ispisuje njen rezultat
na standardni izlaz (ukoliko ova funkcija nije implementirana
--- ispisati na standardni izlaz simbol \verb|?|).
\end{itemize}
\linkresenje{p2.5_}
\end{Exercise}
\begin{Answer}[ref=p2.5_]
%\includecode{resenja/2_PredstavljanjePodataka/2.5_Strukture/5/5.c}
\end{Answer}

\iffalse
\section{Rešenja}
\shipoutAnswer
\fi

