\sstrana
\section{Strukture}

%---------------------------------------------------------dva lagana zadatka za pocetak

\begin{Exercise}[label=struc.1] 
Definisati strukturu kojom se opisuje kompleksan broj. Napisati
funkcije koje izračunavaju zbir, razliku, proizvod i količnik dva
kompleksna broja. Napisati program koji za učitana dva kompleksna broja 
ispisuje vrednost zbira, razlike, proizvoda i količnika. 
U slučaju neispravnog unosa, ispisati odgovarajuću poruku o grešci.

\begin{maxitest}
\begin{upotreba}{1}
#\naslovInt#
#\izlaz{Unesite realni i imaginarni deo prvog broja:}\ulaz{1 2}#
#\izlaz{Unesite realni i imaginarni deo drugog broja:}\ulaz{-2 3}#
#\izlaz{Zbir: -1.00+5.00*i}#
#\izlaz{Razlika: 3.00-1.00*i}#
#\izlaz{Proizvod: -8.00-1.00*i}#
#\izlaz{Kolicnik: 0.31-0.54*i}#
\end{upotreba}
\end{maxitest}
\linkresenje{struc.1}
\end{Exercise}
\ifresenja
\begin{Answer}[ref=struc.1]
\includecode{resenja/2_NapredniTipoviPodataka/2.9_Strukture/strukture_01.c}
\end{Answer}
\fi


\begin{Exercise}[label=struc.5] 
Definisati strukturu kojom se opisuje razlomak. Napisati funkcije
koje izračunavaju zbir i proizvod dva razlomka. 
Napisati program koji za uneti ceo broj $n$ i
unetih $n$ razlomaka ispisuje njihov ukupan zbir i proizvod.
U slučaju neispravnog unosa, ispisati odgovarajuću poruku o grešci.

\begin{miditest}
\begin{upotreba}{1}
#\naslovInt#
#\izlaz{Unesite broj razlomaka:}\ulaz{5}#
#\izlaz{Unesite razlomke:}#
#\ulaz{1 2}#
#\ulaz{7 8}#
#\ulaz{3 4}#
#\ulaz{5 6}#
#\ulaz{2 9}#
#\izlaz{Suma svih razlomaka: 229/72}#
#\izlaz{Proizvod svih razlomaka: 35/576}#
\end{upotreba}
\end{miditest}
\begin{miditest}
\begin{upotreba}{2}
#\naslovInt#
#\izlaz{Unesite broj razlomaka:}\ulaz{10}#
#\izlaz{Unesite razlomke:}#
#\ulaz{4 3}#
#\ulaz{12 25}#
#\ulaz{3 8}#
#\ulaz{1 3}#
#\ulaz{8 9}#
#\ulaz{2 3}#
#\ulaz{5 6}#
#\ulaz{-24 50}#
#\ulaz{7 18}#
#\ulaz{-7 19}#
#\izlaz{Suma svih razlomaka: 6089/1368}#
#\izlaz{Proizvod svih razlomaka: 1568/577125}#
\end{upotreba}
\end{miditest}

\linkresenje{struc.5}
\end{Exercise}
\ifresenja
\begin{Answer}[ref=struc.5]
\includecode{resenja/2_NapredniTipoviPodataka/2.9_Strukture/strukture_02.c}
\end{Answer}
\fi

%-------------------------------------------------------------------razni tekstualni zadaci poredjani po tezini

\begin{Exercise}[label=struc.2] 
 Zimi su prehlade česte i treba unositi više vitamina C. Struktura
 $Vocka$ sadrži ime voćke i
 količinu vitamina C u miligramima (realan broj). Napisati funkcije:
\begin{enumerate}
  \item \kckod{int ucitaj(Vocka niz[])} koja učitava voćke sa standardnog ulaza
     sve do kraja ulaza i kao povratnu vrednost vraća broj učitanih voćki;
  \item \kckod{Vocka vocka\_sa\_najvise\_vitamina(Vocka niz[], int n)}
     koja pronalazi voćku koja ima najviše vitamina C.
\end{enumerate}
 Napisati program koji učitava podatke o voćkama i ispisuje ime voćke sa najviše vitamina C. 
 Pretpostaviti da broj voćki neće biti veći od $50$, kao i da je ime voćke niska od najviše $20$ karaktera. 
 U slučaju neispravnog unosa, ispisati odgovarajuću poruku o grešci.
 
\begin{maxitest}
\begin{upotreba}{1}
#\naslovInt#
#\izlaz{Unesite ime vocke i njenu kolicinu vitamina C:}\ulaz{jabuka 4.6}#
#\izlaz{Unesite ime vocke i njenu kolicinu vitamina C:}\ulaz{limun 83.5}#
#\izlaz{Unesite ime vocke i njenu kolicinu vitamina C:}\ulaz{kivi 71}#
#\izlaz{Unesite ime vocke i njenu kolicinu vitamina C:}\ulaz{banana 8.7}#
#\izlaz{Unesite ime vocke i njenu kolicinu vitamina C:}\ulaz{pomorandza 70.8}#
#\izlaz{Unesite ime vocke i njenu kolicinu vitamina C:}\ulaz{}#
#\izlaz{Voce sa najvise vitamina C je: limun}#
\end{upotreba}
\end{maxitest}

\linkresenje{struc.2}
\end{Exercise}
\ifresenja
\begin{Answer}[ref=struc.2]
\includecode{resenja/2_NapredniTipoviPodataka/2.9_Strukture/strukture_03.c}
\end{Answer}
\fi


\begin{Exercise}[label=struc.3] 
Definisati strukturu \kckod{Grad} koja sadrži ime grada i njegovu prosečnu temperaturu u toku decembra.
Napisati funkcije:
\begin{enumerate}
  \item \kckod{void ucitaj(Grad gradovi[], int n)} koja učitava sa standardnog ulaza podatke
  o $n$ gradova.  
  \item \kckod{void ispisi(Grad gradovi[], int n)} koja ispisuje podatke
  o gradovima koji imaju idealnu temperaturu za klizanje: od $3$ do $8$ stepeni celzijusa. 
\end{enumerate}
Napisati program koji učitava imena $n$ gradova i njihove prosečne temperature, a zatim ispisuje imena gradova sa idealnom temperaturom za klizanje.
Pretpostaviti da je maksimalan broj gradova $50$ i da je maksimalna dužina imena grada $20$ karaktera.
U slučaju neispravnog unosa, ispisati odgovarajuću poruku o grešci.
 
\begin{miditest}
\begin{upotreba}{1}
#\naslovInt#
#\izlaz{Unesite broj gradova:}\ulaz{4}#
#\izlaz{Unesite grad i temperaturu: }#
#\ulaz{Beograd 7}#
#\izlaz{Unesite grad i temperaturu: }#
#\ulaz{Uzice 1.5}#
#\izlaz{Unesite grad i temperaturu: }#
#\ulaz{Subotica 4}#
#\izlaz{Unesite grad i temperaturu: }#
#\ulaz{Zrenjanin 9}#
#\izlaz{Gradovi sa idealnom temperaturom}#
#\izlaz{za klizanje u decembru:}#
#\izlaz{Beograd}#
#\izlaz{Subotica}#
\end{upotreba}
\end{miditest}
\begin{miditest}
\begin{upotreba}{2}
#\naslovInt#
#\izlaz{Unesite broj gradova:}\ulaz{2}#
#\izlaz{Unesite grad i temperaturu: }#
#\ulaz{Varsava 11}#
#\izlaz{Unesite grad i temperaturu: }#
#\ulaz{Prag 2}#
#\izlaz{Gradovi sa idealnom temperaturom}#
#\izlaz{za klizanje u decembru:}#
\end{upotreba}
\end{miditest}
 
\linkresenje{struc.3}
\end{Exercise}
\ifresenja
\begin{Answer}[ref=struc.3]
%\includecode{resenja/2_NapredniTipoviPodataka/2.9_Strukture/strukture_04.c}
Pogledajte zadatak \ref{struc.2}.
\end{Answer}
\fi


\begin{Exercise}[label=struc.4] 
 Definisati strukturu \kckod{ParReci} koja sadrži reč na srpskom
 jeziku i odgovarajući prevod na engleski jezik. Napisati program koji
 do kraja ulaza učitava sve parove reči, a potom za rečenicu koja se
 zadaje u jednoj liniji ispisuje prevod. Ako je reč u rečenici
 nepoznata umesto nje ispisati nisku zvezdica čija dužina odgovara dužini nepoznate reči. 
 Pretpostaviti da je maksimalna
 dužina reči $50$ karaktera, maksimalan broj parova reči
 $100$, a maksimalna dužina rečenice $100$ karaktera. 

\begin{miditest}
\begin{upotreba}{1}
#\naslovInt#
#\izlaz{Unesite reci i njihove prevode:}#
#\ulaz{zima winter}#
#\ulaz{godina year}#
#\ulaz{sreca happiness}#
#\ulaz{programiranje programming}#
#\ulaz{caj tea}#
#\izlaz{Unesite recenicu za prevod: }#
#\ulaz{piti caj zimi je sreca}#
#\izlaz{**** tea **** ** happiness}#
\end{upotreba}
\end{miditest}
\begin{miditest}
\begin{upotreba}{2}
#\naslovInt#
#\izlaz{Unesite reci i njihove prevode:}#
#\ulaz{je is}#
#\ulaz{zima winter}#
#\ulaz{pas dog}#
#\ulaz{sreca happiness}#
#\ulaz{prijatelj friend}#
#\ulaz{solja cup}#
#\ulaz{covek man}#
#\izlaz{Unesite recenicu za prevod: }#
#\ulaz{pas je covekov najbolji prijatelj}#
#\izlaz{dog is ******* ******** friend}#
\end{upotreba}
\end{miditest}

\linkresenje{struc.4}
\end{Exercise}
\ifresenja
\begin{Answer}[ref=struc.4]
\includecode{resenja/2_NapredniTipoviPodataka/2.9_Strukture/strukture_05.c}
\end{Answer}
\fi


\begin{Exercise}[label=struc.7] 
Statistički zavod Srbije istražuje kako rade obdaništa u Srbiji. 
Za svako obdanište poznat je spisak koji sadrži broj dece u grupi, 
a zatim i ocene koje je svako dete dalo o radu obdaništa. 
Definisati strukturu \kckod{Dete} koja sadrži polja pol (\kckod{m} ili
\kckod{z}), broj godina (od $3$ do $6$) i ocenu koju je dete dalo radu obdaništa (od $1$ do $5$). 
Napisati program koji učitava broj dece u grupi, a zatim i informacije o svakom detetu. 
Ispisati, na tri decimale, prosečnu ocenu
koje je obdanište dobilo od dece sa unetim polom i brojem godina.
Pretpostaviti da je maksimalan broj dece u obdaništu $200$.
U slučaju neispravnog unosa, ispisati odgovarajuću poruku o grešci.

\begin{miditest}
\begin{upotreba}{1}
#\naslovInt#
#\izlaz{Unesite broj dece u grupi:}\ulaz{5}#
#\izlaz{Unesite podatke za svako dete (pol,}#
#\izlaz{broj godina i ocenu):}#
#\ulaz{m 3 5}#
#\ulaz{z 3 4}#
#\ulaz{m 4 2}#
#\ulaz{m 5 4}#
#\ulaz{m 3 4}#
#\izlaz{Unesite pol i broj godina za}#
#\izlaz{statistiku:}\ulaz{m 3}#
#\izlaz{Prosecna ocena je: 4.500.}#
\end{upotreba}
\end{miditest}
\begin{miditest}
\begin{upotreba}{2}
#\naslovInt#
#\izlaz{Unesite broj dece u grupi:}\ulaz{10}#
#\izlaz{Unesite podatke za svako dete (pol,}#
#\izlaz{broj godina i ocenu):}#
#\ulaz{m 3 5}#
#\ulaz{z 4 4}#
#\ulaz{m 5 4}#
#\ulaz{z 4 3}#
#\ulaz{z 3 2}#
#\ulaz{z 4 5}#
#\ulaz{m 6 5}#
#\ulaz{z 4 4}#
#\ulaz{z 4 5}#
#\ulaz{m 6 3}#
#\izlaz{Unesite pol i broj godina za }#
#\izlaz{statistiku:}\ulaz{z 4}#
#\izlaz{Prosecna ocena je: 4.200.}#
\end{upotreba}
\end{miditest}

\begin{miditest}
\begin{upotreba}{3}
#\naslovInt#
#\izlaz{Unesite broj dece u grupi:}\ulaz{15}#
#\izlaz{Unesite podatke za svako dete (pol,}# 
#\izlaz{broj godina i ocenu):}#
#\ulaz{m 3 2}#
#\ulaz{z 7 5}#
#\izlaz{Greska: neispravan broj godina.}#
\end{upotreba}
\end{miditest}
\begin{miditest}
\begin{upotreba}{4}
#\naslovInt#
#\izlaz{Unesite broj dece u grupi:}\ulaz{2}#
#\izlaz{Unesite podatke za svako dete (pol,}# 
#\izlaz{broj godina i ocenu):}#
#\ulaz{m 3 2}#
#\ulaz{z 3 5}#
#\izlaz{Unesite pol i broj godina za}#
#\izlaz{statistiku:}\ulaz{h 5}#
#\izlaz{Greska: neispravan pol.}#
\end{upotreba}
\end{miditest}

\linkresenje{struc.7}
\end{Exercise}
\ifresenja
\begin{Answer}[ref=struc.7]
\includecode{resenja/2_NapredniTipoviPodataka/2.9_Strukture/strukture_06.c}
\end{Answer}
\fi


\begin{Exercise}[label=struc.11] 
Definisati strukturu kojom se opisuje student. Student se opisuje svojim
imenom i prezimenom, smerom
(R, I, V, N, T, M) i prosečnom ocenom. Napisati program koji učitava
podatke o $n$ studenata, a zatim i informaciju o smeru i ispisuje imena i
prezimena onih studenta koji su sa datog smera, kao i podatke studenta koji ima najveći prosek. 
Ako ima više takvih studenata ispisati podatke o svima. 
Pretpostaviti da je maksimalan broj studenata $2000$, a maksimalna dužina imena i prezimena
po $30$ karaktera.
U slučaju neispravnog unosa, ispisati odgovarajuću poruku o grešci.
 
\begin{miditest}
\begin{upotreba}{1}
#\naslovInt#
#\izlaz{Unesite broj studenata:}\ulaz{5}#
#\izlaz{Unesite podatke o studentima:}#
#\izlaz{0. student:}\ulaz{Kocic Marija R 9.14}#
#\izlaz{1. student:}\ulaz{Tanja Mratinkovic R 7.88}#
#\izlaz{2. student:}\ulaz{Mihailo Simic N 8.44}#
#\izlaz{3. student:}\ulaz{Milena Medar I 9.14}#
#\izlaz{4. student:}\ulaz{Ljubica Mihic N 9.00}#
#\izlaz{Unesite smer:}\ulaz{R}#
#\izlaz{Studenti sa R smera:}#
#\izlaz{Kocic Marija}#
#\izlaz{Tanja Mratinkovic}#
#\izlaz{---------------------}#
#\izlaz{Svi studenti koji imaju maksimalni prosek:}#
#\izlaz{Kocic Marija, R, 9.14}#
#\izlaz{Milena Medar, I, 9.14}#
\end{upotreba}
\end{miditest}
\begin{miditest}
\begin{upotreba}{2}
#\naslovInt#
#\izlaz{Unesite broj studenata:}\ulaz{4}#
#\izlaz{Unesite podatke o studentima:}#
#\izlaz{0. student:}\ulaz{Djordje Lazarevic N 9.05}#
#\izlaz{1. student:}\ulaz{Minja Peric W 7.70}#
#\izlaz{Greska: neispravan unos smera.}#
\end{upotreba}
\end{miditest}

\linkresenje{struc.11}
\end{Exercise}
\ifresenja
\begin{Answer}[ref=struc.11]
\includecode{resenja/2_NapredniTipoviPodataka/2.9_Strukture/strukture_07.c}
\end{Answer}
\fi


\begin{Exercise}[label=struc.14]
Definisati strukturu \kckod{Djak} koja sadrži ime đaka i $9$ ocena (ocene su celi brojevi od $1$ do
$5$). Napisati program koji učitava podatke o đacima sve do kraja ulaza
i na standardni izlaz ispisuje prvo imena nedovoljnih đaka, a zatim imena
odličnih đaka. Đak je nedovoljan ako ima barem jednu jedinicu, a 
odličan ako ima prosek ocena veći ili jednak $4.5$.
Pretpostaviti da je maksimalna dužina imena đaka $20$ karaktera, kao i da
je maksimalan broj đaka $30$.
U slučaju neispravnog unosa, ispisati odgovarajuću poruku o grešci.

\begin{minitest}
\begin{upotreba}{1}
#\naslovInt#
#\izlaz{Unesite podatke o djaku: }#
#\ulaz{Maja 4 5 2 3 4 4 3 3 4}#
#\izlaz{Unesite podatke o djaku: }#
#\ulaz{Nikola 5 4 5 5 5 4 4 5 5}#
#\izlaz{Unesite podatke o djaku: }#
#\ulaz{Jasmina 2 2 1 1 2 3 3 1 3}#
#\izlaz{Unesite podatke o djaku: }#
#\ulaz{Pera 5 4 5 3 5 5 1 5 5}#
#\izlaz{Unesite podatke o djaku: }#
#\ulaz{Pavle 4 3 2 4 3 2 4 3 2}#
#\izlaz{Unesite podatke o djaku: }#
#\izlaz{\ }#
#\izlaz{NEDOVOLJNI: Jasmina Pera }#
#\izlaz{ODLICNI: Nikola}#
\end{upotreba}
\end{minitest}
\begin{minitest}
\begin{upotreba}{2}
#\naslovInt#
#\izlaz{Unesite podatke o djaku: }#
#\ulaz{Uros 3 4 2 3 4 2 3 4 4}#
#\izlaz{Unesite podatke o djaku: }#
#\ulaz{Nebojsa 4 5 5 5 4 5 5 5 5}#
#\izlaz{Unesite podatke o djaku: }#
#\ulaz{Sreten 2 3 2 4 5 4 4 4 2}#
#\izlaz{Unesite podatke o djaku: }#
#\izlaz{\ }#
#\izlaz{NEDOVOLJNI:}#
#\izlaz{ODLICNI: Nebojsa}#
\end{upotreba}
\end{minitest}
\begin{minitest}
\begin{upotreba}{3}
#\naslovInt#
#\izlaz{Unesite podatke o djaku: }#
#\ulaz{Mirko 2 3 4 4 4 3 3 3 4}#
#\izlaz{Unesite podatke o djaku: }#
#\ulaz{Mihailo 2 3 10 5 5 2 3 4 2}#
#\izlaz{Greska: neispravna ocena.}#
\end{upotreba}
\end{minitest}

\linkresenje{struc.14}
\end{Exercise}
\ifresenja
\begin{Answer}[ref=struc.14]
\includecode{resenja/2_NapredniTipoviPodataka/2.9_Strukture/strukture_08.c}
\end{Answer}
\fi


\begin{Exercise}[label=struc.13] 
Definisati strukturu \kckod{Osoba} kojom se opisuje jedan unos u
imenik. Za svaku osobu su dati podaci: ime, prezime i imejl adresa.
Napisati program koji učitava ceo
broj $n$, a zatim podatke o $n$ osoba. Ispisati imena
i prezimena svih osoba koje imaju imejl adresu koja se završava sa \kckod{@gmail.com}.
Pretpostaviti da je maksimalan broj osoba $50$, kao i da je 
maksimalna dužina imena osobe $20$ karaktera, 
prezimena $30$ karaktera, a imejl adrese $50$ karaktera.
U slučaju neispravnog unosa, ispisati odgovarajuću poruku o grešci.
\napomena{Može se smatrati da je svaka imejl adresa dobro
zadata i sadrži samo jedno pojavljivanje znaka \kckod{@}.}

\begin{miditest}
\begin{upotreba}{1}
#\naslovInt#
#\izlaz{Unesite broj osoba:}\ulaz{3}#
#\izlaz{Unesite podatke o osobama }#
#\izlaz{(ime, prezime i imejl adresu):}#
#\ulaz{Dusko Dugousko dusko@yahoo.com}#
#\ulaz{Pink Panter panter@gmail.com}#
#\ulaz{Pera Detlic pd@gmail.com}#
#\izlaz{Vlasnici gmail naloga su:}#
#\izlaz{Pink Panter}#
#\izlaz{Pera Detlic}#
\end{upotreba}
\end{miditest}
\begin{miditest}
\begin{upotreba}{2}
#\naslovInt#
#\izlaz{Unesite broj osoba:}\ulaz{3}#
#\izlaz{Unesite podatke o osobama }#
#\izlaz{(ime, prezime i imejl adresu):}#
#\ulaz{Homer Simpson homer@yahoo.com}#
#\ulaz{Mardz Simpson mardz@matf.bg.ac.rs}#
#\izlaz{Nema vlasnika gmail naloga.}#
\end{upotreba}
\end{miditest}

\linkresenje{struc.13}
\end{Exercise}
\ifresenja
\begin{Answer}[ref=struc.13]
\includecode{resenja/2_NapredniTipoviPodataka/2.9_Strukture/strukture_09.c}
\end{Answer}
\fi


\begin{Exercise}[difficulty=1, label=struc.12] 
Napisati program koji izračunava prosečnu cenu jedne potrošačke
korpe. Potrošačka korpa se sastoji od broja kupljenih artikala i niza
kupljenih artikala. Svaki artikal određen je svojim nazivom, količinom
i cenom. Program treba da učita broj potrošača $n$,
zatim podatke za $n$ potrošačkih korpi i da na osnovu učitanih
podataka izračuna prosečnu cenu potrošačke korpe. Program ispisuje na
dve decimale račune svake potrošačke korpe i na kraju ispisuje
prosečnu cenu potrošačke korpe. 
Pretpostaviti da je maksimalan broj potrošačkih korpi $100$, maksimalan
broj artikala u korpi $20$ i da naziv svakog
artikla sadrži maksimalno $30$ karaktera.
U slučaju neispravnog unosa, ispisati odgovarajuću poruku o grešci.

\begin{maxitest}
\begin{upotreba}{1}
#\naslovInt#
#\izlaz{Unesite broj potrosackih korpi:}\ulaz{3}#
#\izlaz{Unesite podatke o korpi: }#
#\izlaz{Broj artikala:}\ulaz{4}#
#\izlaz{Unesite artikal (naziv, kolicinu i cenu):}\ulaz{jabuke 10 22.4}#
#\izlaz{Unesite artikal (naziv, kolicinu i cenu):}\ulaz{dezodorans 1 120.99}#
#\izlaz{Unesite artikal (naziv, kolicinu i cenu):}\ulaz{C\_supa 3 36.56}#
#\izlaz{Unesite artikal (naziv, kolicinu i cenu):}\ulaz{sunka 1 230.99}#
#\izlaz{Unesite podatke o korpi: }#
#\izlaz{Broj artikala:}\ulaz{2}#
#\izlaz{Unesite artikal (naziv, kolicinu i cenu):}\ulaz{Jafa\_keks 1 55.78}#
#\izlaz{Unesite artikal (naziv, kolicinu i cenu):}\ulaz{Najlepse\_zelje 1 62.99}#
#\izlaz{Unesite podatke o korpi: }#
#\izlaz{Broj artikala:}\ulaz{3}#
#\izlaz{Unesite artikal (naziv, kolicinu i cenu):}\ulaz{prasak\_za\_ves 1 1199.99}#
#\izlaz{Unesite artikal (naziv, kolicinu i cenu):}\ulaz{omeksivac 1 279.99}#
#\izlaz{Unesite artikal (naziv, kolicinu i cenu):}\ulaz{protiv\_kamenca 1 699.99}#
#\izlaz{\ }#
#\izlaz{Korpa 0:}#
#\izlaz{\ \ \ \ \ \ \ \ jabuke 10 22.40}#
#\izlaz{\ \ \ \ \ \ \ \ dezodorans 1 120.99}#
#\izlaz{\ \ \ \ \ \ \ \ C\_supa 3 36.56}#
#\izlaz{\ \ \ \ \ \ \ \ sunka 1 230.99}#
#\izlaz{------------------------}#
#\izlaz{\ \ \ \ \ \ \ \ ukupno: 685.66}#
#\izlaz{\ }#
#\izlaz{Korpa 1:}#
#\izlaz{\ \ \ \ \ \ \ \ Jafa\_keks 1 55.78}#
#\izlaz{\ \ \ \ \ \ \ \ Najlepse\_zelje 1 62.99}#
#\izlaz{------------------------}#
#\izlaz{\ \ \ \ \ \ \ \ ukupno: 118.77}#
#\izlaz{\ }#
#\izlaz{Korpa 2:}#
#\izlaz{\ \ \ \ \ \ \ \ prasak\_za\_ves 1 1199.99}#
#\izlaz{\ \ \ \ \ \ \ \ omeksivac 1 279.99}#
#\izlaz{\ \ \ \ \ \ \ \ protiv\_kamenca 1 699.99}#
#\izlaz{------------------------}#
#\izlaz{\ \ \ \ \ \ \ \ ukupno: 2179.97}#
#\izlaz{\ }#
#\izlaz{Prosecna cena potrosacke korpe: 994.80}#
\end{upotreba}
\end{maxitest}

\linkresenje{struc.12}
\end{Exercise}
\ifresenja
\begin{Answer}[ref=struc.12]
\includecode{resenja/2_NapredniTipoviPodataka/2.9_Strukture/strukture_10.c}
\end{Answer}
\fi


\begin{Exercise}[label=struc.9] 
Definisati strukturu \kckod{Lopta} sa poljima \kckod{poluprecnik} (ceo
broj u centimetrima) i \kckod{boja} (enumeracioni tip koji uključuje
plavu, žutu, crvenu i zelenu boju). 
Napisati funkcije: 
\begin{enumerate}
 \item \kckod{void ucitaj(Lopta niz[], int n)} koja učitava podatke o $n$ lopti u niz.
 \item \kckod{double ukupna\_zapremina(Lopta niz[], int n)} koja računa ukupnu zapreminu svih lopti.
 \item \kckod{int broj\_crvenih(Lopta niz[], int n)} koja prebrojava koliko ima crvenih lopti u nizu.
\end{enumerate}
Napisati program koji učitava informacije o $n$
lopti i ispisuje ukupnu zapreminu i broj crvenih lopti.
Pretpostaviti da je maksimalan broj lopti $50$.
U slučaju neispravnog unosa, ispisati odgovarajuću poruku o grešci.

\begin{miditest}
\begin{upotreba}{1}
#\naslovInt#
#\izlaz{Unesite broj lopti:}\ulaz{4}#
#\izlaz{Unesite poluprecnike i boje lopti}# 
#\izlaz{(1-plava, 2-zuta, 3-crvena, 4-zelena):}#
#\izlaz{1.lopta:}\ulaz{4 1}#
#\izlaz{2.lopta:}\ulaz{1 3}#
#\izlaz{3.lopta:}\ulaz{2 3}#
#\izlaz{4.lopta:}\ulaz{10 4}#
#\izlaz{Ukupna zapremina: 4494.57}#
#\izlaz{Broj crvenih lopti: 2}#
\end{upotreba}
\end{miditest}
\begin{miditest}
\begin{upotreba}{2}
#\naslovInt#
#\izlaz{Unesite broj lopti:}\ulaz{8}#
#\izlaz{Unesite poluprecnike i boje lopti}# 
#\izlaz{(1-plava, 2-zuta, 3-crvena, 4-zelena):}#
#\izlaz{1. lopta:}\ulaz{1 2}#
#\izlaz{2. lopta:}\ulaz{2 10}#
#\izlaz{Greska: neispravan unos.}#
\end{upotreba}
\end{miditest}

\begin{miditest}
\begin{upotreba}{3}
#\naslovInt#
#\izlaz{Unesite broj lopti:}\ulaz{8}#
#\izlaz{Unesite poluprecnike i boje lopti}# 
#\izlaz{(1-plava, 2-zuta, 3-crvena, 4-zelena):}#
#\izlaz{1. lopta:}\ulaz{2 1}#
#\izlaz{2. lopta:}\ulaz{30 3}#
#\izlaz{3. lopta:}\ulaz{7 3}#
#\izlaz{4. lopta:}\ulaz{4 1}#  
#\izlaz{5. lopta:}\ulaz{5 2}#
#\izlaz{6. lopta:}\ulaz{6 2}#
#\izlaz{7. lopta:}\ulaz{12 3}#
#\izlaz{8. lopta:}\ulaz{14 2}#
#\izlaz{Ukupna zapremina: 134996.34}#
#\izlaz{Ukupno crvenih lopti: 3}#
\end{upotreba}
\end{miditest}

\linkresenje{struc.9}
\end{Exercise}
\ifresenja
\begin{Answer}[ref=struc.9]
\includecode{resenja/2_NapredniTipoviPodataka/2.9_Strukture/strukture_11.c}
\end{Answer}
\fi


\begin{Exercise}[label=struc.10] 
Napisati program za predstavljanje poligona i izračunavanje
dužine njegovih stranica i obima.
\begin{enumerate}
\item Definisati strukturu \kckod{Tacka} kojom se opisuje
  tačka dekartovske ravni čije su $x$ i $y$ koordinate podaci tipa
  \kckod{double}.

\item Definisati funkciju \kckod{double rastojanje(const Tacka *A, const Tacka *B)}
  koja izračunava rastojanje između dve tačke.

\item Definisati funkciju \kckod{int ucitaj\_poligon(Tacka poligon[], int n)} koja učitava maksimalno $n$ puta po dve
  vrednosti tipa \kckod{double} (koje predstavljaju koordinate temena
  poligona) i upisuje ih u zadati niz tačaka. Funkcija vraća broj
  uspešno učitanih tačaka.

\item Definisati funkciju \kckod{double obim\_poligona(Tacka poligon[], int n)} koja izračunava obim poligona sa $n$ temena u zadatom nizu.
  \uputstvo{Prilikom računanja obima ne zaboraviti stranicu koja spaja
    poslednje i prvo teme.}

\item Definisati funkciju \kckod{double maksimalna\_stranica(Tacka poligon[], int n)} koja izračunava dužinu najduže stranice
  poligona sa $n$ temena u zadatom nizu.

\item Napisati funkciju \kckod{double povrsina\_trougla(const Tacka *A, const Tacka *B, const Tacka *C)} 
koja izračunava površinu trougla čija su temena \kckod{A}, \kckod{B} i \kckod{C}.

\item Napisati funkciju \kckod{double povrsina\_poligona(Tacka poligon[], int n)} 
koja izračunava površinu konveksnog poligona. \uputstvo{Zadatak se može rešiti podelom poligona na trouglove i korišćenjem funkcije \kckod{povrsina\_trougla}}.
\end{enumerate}
Napisati program koji učitava poligon sa maksimalno $n$ temena
  i za učitani poligon ispisuje na tri decimale
  obim, dužinu najduže stranice i površinu. Pretpostaviti da je
  uneti poligon konveksan. Poligon mora imati barem tri temena.
  Pretpostaviti da je maksimalan broj temena $1000$.
  U slučaju neispravnog unosa, ispisati odgovarajuću poruku o grešci.

\begin{miditest}
\begin{upotreba}{1}
#\naslovInt#
#\izlaz{Unesite maksimalan broj temena poligona:}\ulaz{10}#
#\izlaz{Unesite temena poligona:}#
#\ulaz{0 0}#
#\ulaz{0 6}#
#\ulaz{3 3}#
#\izlaz{Obim poligona je 14.485.}#
#\izlaz{Duzina maksimalne stranice je 6.000.}#
#\izlaz{Povrsina poligona je 9.000.}#
\end{upotreba}

\begin{upotreba}{3}
#\naslovInt#
#\izlaz{Unesite maksimalan broj temena poligona:}\ulaz{4}#
#\ulaz{0 0}#
#\izlaz{Greska: poligon mora imati bar tri tacke.}#
\end{upotreba}
\end{miditest}
\begin{miditest}
\begin{upotreba}{2}
#\naslovInt#
#\izlaz{Unesite maksimalan broj temena poligona:}\ulaz{10}#
#\izlaz{Unesite temena poligona:}#
#\ulaz{0 0}#
#\ulaz{12 0}#
#\ulaz{13 2}#
#\ulaz{16 5}#
#\ulaz{20 10}#
#\ulaz{18 15}#
#\ulaz{15 20}#
#\ulaz{10 20}#
#\ulaz{8 15}#
#\ulaz{3 4}#
#\izlaz{Obim poligona je 63.566.}#
#\izlaz{Duzina maksimalne stranice je 12.083.}#
#\izlaz{Povrsina poligona je 247.500.}#
\end{upotreba}
\end{miditest}


\linkresenje{struc.10}
\end{Exercise}
\ifresenja
\begin{Answer}[ref=struc.10]
\includecode{resenja/2_NapredniTipoviPodataka/2.9_Strukture/strukture_12.c}
\end{Answer}
\fi


\begin{Exercise}[difficulty=1, label=struc.8] 
Definisati strukturu \kckod{Izraz} kojom se opisuje numerički izraz
nad celim brojevima koji se sastoji od dva celobrojna operanda i
numeričke operacije (sabiranje, oduzimanje, množenje ili celobrojno
deljenje). 
\begin{enumerate}
\item Napisati funkciju \kckod{int korektan\_izraz(const Izraz *izraz)} koja ispituje da li je dati izraz korektno
  zadat i vraća jedinicu ako jeste, a nulu inače. Podrazumeva se da je
  izraz korektno zadat ako je operacija $+$, $-$, $*$ ili $/$ i
  u slučaju deljenja drugi operand je različit od $0$.
\item Napisati funkciju \kckod{int vrednost(const Izraz *izraz)} koja za dati izraz određuje vrednost izraza.
\item Napisati funkciju \kckod{void ucitaj(Izraz izrazi[], int n)} koja učitava izraze. Funkcija treba da
  učita sa standardnog ulaza $n$ izraza koji su zadati prefiksno --- prvo
  operacija, a potom dva operanda.
\end{enumerate}

Napisati program koji učitava prirodan broj $n$, a
zatim $n$ izraza u prefiksnoj notaciji. Program treba da ispiše
maksimalnu vrednost unetih izraza i sve izraze čija vrednost je manja
od polovine maksimalne vrednosti.
Pretpostaviti da je maksimalan broj izraza $100$.
U slučaju neispravnog unosa, ispisati odgovarajuću poruku o grešci.

\begin{miditest}
\begin{upotreba}{1}
#\naslovInt#
#\izlaz{Unesite broj izraza:}\ulaz{4}#
#\izlaz{Unesite izraze u prefiksnoj notaciji:}#
#\ulaz{+ 10 4}#
#\ulaz{- 9 2}#
#\ulaz{* 11 2}#
#\ulaz{/ 7 3}#
#\izlaz{Maksimalna vrednost izraza: 22}#
#\izlaz{Izrazi cija je vrednost manja}# 
#\izlaz{od polovine maksimalne vrednosti:}#
#\izlaz{9 - 2 = 7}#
#\izlaz{7 / 3 = 2}#
\end{upotreba}

\begin{upotreba}{3}
#\naslovInt#
#\izlaz{Unesite broj izraza:}\ulaz{3}#
#\izlaz{Unesite izraze u prefiksnoj notaciji:}#
#\ulaz{* 1 2}#
#\ulaz{/ 3 0}#
#\izlaz{Greska: deljenje nulom.}#
\end{upotreba}
\end{miditest}
\begin{miditest}
\begin{upotreba}{2}
#\naslovInt#
#\izlaz{Unesite broj izraza:}\ulaz{10}#
#\izlaz{Unesite izraze u prefiksnoj notaciji:}#
#\ulaz{+ 10 2}#
#\ulaz{- -678 34}#
#\ulaz{* 77 2}#
#\ulaz{+ 1000 -23}#
#\ulaz{+ 102 4}#
#\ulaz{- 200 23}#
#\ulaz{/ 67 12}#
#\ulaz{/ 1000 2}#
#\ulaz{* 44 6}#
#\ulaz{/ 13 1}#
#\izlaz{Maksimalna vrednost izraza: 977}#
#\izlaz{Izrazi cija je vrednost manja}#
#\izlaz{od polovine maksimalne vrednosti:}#
#\izlaz{10 + 2 = 12}#
#\izlaz{-678 - 34 = -712}#
#\izlaz{77 * 2 = 154}#
#\izlaz{102 + 4 = 106}#
#\izlaz{200 - 23 = 177}#
#\izlaz{67 / 12 = 5}#
#\izlaz{44 * 6 = 264}#
#\izlaz{13 / 1 = 13}#
\end{upotreba}
\end{miditest}


\linkresenje{struc.8}
\end{Exercise}
\ifresenja
\begin{Answer}[ref=struc.8]
\includecode{resenja/2_NapredniTipoviPodataka/2.9_Strukture/strukture_13.c}
\end{Answer}
\fi


\begin{Exercise}[difficulty=1, label=struc.15] 
Definisati strukturu kojom se opisuje polinom. Polinom je dat svojim
stepenom i realnim koeficijentima. 
\begin{enumerate}
\item Napisati funkciju \kckod{int ucitaj(Polinom niz[])} koja sa standardnog ulaza učitava polinome sve do 
      kraja ulaza. Polinomi su zadati stepenom i koeficijentima počevši od slobodnog člana. Funkcija kao 
      povratnu vrednost vraća broj učitanih polinoma.
\item Napisati funkciju \kckod{void ispis(const Polinom *p)} koja ispisuje polinom stepena $n$ sa koeficijentima $k_0$, $k_1$, ..., $k_n$ u obliku \kckod{$k_0 \pm
  k_1*x \pm k_2*x$\^{}$2 \pm k_3*x$\^{}$3 \pm \ldots \pm
  k_n*x$\^{}$n$}.  Na mesto znaka $\pm$ zapisati odgovarajući znak, $+$ ili $-$,
  u zavisnosti od znaka odgovarajućeg koeficijenta. Koeficijente ispisivati na dve decimale. Koeficijente koji su jednaki
  $0$ ne ispisivati.  
\item Napisati funkciju \kckod{void integral(const Polinom *p, Polinom *tekuci\_integral)} koja za dati polinom \kckod{p} određuje njegov integral \kckod{tekuci\_integral}. Za vrednost slobodnog člana integrala uzeti vrednost 0. 
\end{enumerate}
Napisati program koji učitava polinome do kraja ulaza i za svaki učitani polinom
  određuje i ispisuje njegov integral.  
  Pretpostaviti da je maksimalan broj polinoma $100$, a maksimalan stepen polinoma $10$.
U slučaju neispravnog unosa, ispisati odgovarajuću poruku o grešci.

\begin{miditest}
\begin{upotreba}{1}
#\naslovInt#
#\izlaz{Unesite stepen:}\ulaz{3}#
#\izlaz{Unesite koeficijente polinoma:}#
#\ulaz{1 0 3 1}#
#\izlaz{Unesite stepen:}\ulaz{4}#
#\izlaz{Unesite koeficijente polinoma:}#
#\ulaz{7 9 4 0 4}#
#\izlaz{Unesite stepen:}#
#\izlaz{\ }#
#\izlaz{Integrali su: }#
#\izlaz{1.00*x + 1.00*x\^{}3 + 0.25*x\^{}4}#
#\izlaz{7.00*x + 4.50*x\^{}2 + 1.33*x\^{}3 + 0.80*x\^{}5}#
\end{upotreba}
\end{miditest}
\begin{miditest}
\begin{upotreba}{2}
#\naslovInt#
#\izlaz{Unesite stepen:}\ulaz{3}#
#\izlaz{Unesite koeficijente polinoma:}#
#\ulaz{1 0 -4 1}#
#\izlaz{Unesite stepen:}\ulaz{2}#
#\izlaz{Unesite koeficijente polinoma:}#
#\ulaz{1 2 -3}#
#\izlaz{Unesite stepen:}\ulaz{1}#
#\izlaz{Unesite koeficijente polinoma:}#
#\ulaz{0 -1}#
#\izlaz{Unesite stepen: }#
#\izlaz{\ }#
#\izlaz{Integrali su:}#
#\izlaz{1.00*x - 1.33*x\^{}3 + 0.25*x\^{}4 }#
#\izlaz{1.00*x + 1.00*x\^{}2 - 1.00*x\^{}3 }#
#\izlaz{-0.50*x\^{}2 }#
\end{upotreba}
\end{miditest}

\linkresenje{struc.15}
\end{Exercise}
\ifresenja
\begin{Answer}[ref=struc.15]
\includecode{resenja/2_NapredniTipoviPodataka/2.9_Strukture/strukture_14.c}
\end{Answer}
\fi


\ifresenja
\sstrana
\section{Rešenja}
\shipoutAnswer
\fi


