
\section{Strukture}

\begin{Exercise}[label=struc.1] 
Definisati strukturu kojom se predstavlja kompleksan broj. Napisati
funkcije koje izračunavaju zbir, razliku, proizvod i količnik dva
kompleksna broja, a zati i program koji učitava dva kompleksna broja i
ispisuje vrednost zbira, razlike, proizvoda i količnika. 

\begin{maxitest}
\begin{upotreba}{1}
#\naslovInt#
#\izlaz{Unesite realni i imaginarni deo prvog broja: }\ulaz{1 2}#
#\izlaz{Unesite realni i imaginarni deo drugog broja: }\ulaz{-2 3}#
#\izlaz{Zbir: -1.00+5.00*i}#
#\izlaz{Razlika: 3.00-1.00*i}#
#\izlaz{Proizvod: -8.00-1.00*i}#
#\izlaz{Kolicnik: 0.31-0.54*i}#
\end{upotreba}
\end{maxitest}
\linkresenje{struc.1}
\end{Exercise}
\begin{Answer}[ref=struc.1]
\includecode{resenja/2_PredstavljanjePodataka/2.5_Strukture/1.c}
\end{Answer}

\begin{Exercise}[label=struc.5] 
Definisati strukturu kojom se predstavlja razlomak. Napisati funkcije
koje izračunavaju zbir i proizvod dva razlomka. Unosi se broj $n$ a
potom i $n$ razlomaka sa standarnog ulaza. Ispisati njihov zbir i
proizvod na standardni izlaz.

\begin{miditest}
\begin{upotreba}{1}
#\naslovInt#
#\izlaz{Unesi broj razlomaka: }\ulaz{5}#
#\izlaz{Uneti razlomke:}#
#\ulaz{1 2}#
#\ulaz{7 8}#
#\ulaz{3 4}#
#\ulaz{5 6}#
#\ulaz{2 9}#
#\izlaz{Suma svih razlomaka je 229/72.}#
#\izlaz{Proizvod svih razlomaka je 35/576.}#
\end{upotreba}
\end{miditest}
\begin{miditest}
\begin{upotreba}{2}
#\naslovInt#
#\izlaz{Unesi broj razlomaka: }\ulaz{10}#
#\izlaz{Uneti razlomke:}#
#\ulaz{4 3}#
#\ulaz{12 25}#
#\ulaz{3 8}#
#\ulaz{1 3}#
#\ulaz{8 9}#
#\ulaz{2 3}#
#\ulaz{5 6}#
#\ulaz{-24 50}#
#\ulaz{7 18}#
#\ulaz{-7 19}#
#\izlaz{Suma svih razlomaka je 6089/1368.}#
#\izlaz{Proizvod svih razlomaka je 1568/577125.}#
\end{upotreba}
\end{miditest}

\linkresenje{struc.5}
\end{Exercise}
\begin{Answer}[ref=struc.5]
\includecode{resenja/2_PredstavljanjePodataka/2.5_Strukture/5.c}
\end{Answer}


\begin{Exercise}[label=struc.2] 
 Zimi su prehlade česte i treba unositi više vitamina C. Struktura
 $Vocka$ sadrži ime voćke (nisku maksimalne dužine $20$ karaktera) i
 količinu vitamina C u miligramima (realan broj). Napisati program
 koji učitava podatke o voćkama sve do unosa reči \kckod{KRAJ} i
 ispisuje ime voćke sa najviše vitamina C. Pretpostaviti da broj voćki
 neće biti veći od $50$. \napomena{Probati sa testiranjem zadataka
   pomoću preusmeravanja.}

\begin{maxitest}
\begin{upotreba}{1}
#\naslovInt#
#\izlaz{Unesite ime voćke i njenu količinu vitamina C: }\ulaz{jabuka 4.6}#
#\izlaz{Unesite ime voćke i njenu količinu vitamina C: }\ulaz{limun 51}#
#\izlaz{Unesite ime voćke i njenu količinu vitamina C: }\ulaz{kivi 92.7}#
#\izlaz{Unesite ime voćke i njenu količinu vitamina C: }\ulaz{banana 8.7}#
#\izlaz{Unesite ime voćke i njenu količinu vitamina C: }\ulaz{pomorandza 53.2}#
#\izlaz{Unesite ime voćke i njenu količinu vitamina C: }\ulaz{KRAJ}#
#\izlaz{Voce sa najvise C vitamina je: kivi}#
\end{upotreba}
\end{maxitest}

\linkresenje{struc.2}
\end{Exercise}
\begin{Answer}[ref=struc.2]
\includecode{resenja/2_PredstavljanjePodataka/2.5_Strukture/2.c}
\end{Answer}



\begin{Exercise}[label=struc.3] 
Definisati strukturu $Grad$ u kojoj se nalazi ime grada (niska dužine
$20$ karaktera) i prosečna temperatura u toku decembra (realan
broj). Napisati program koji učitava imena $n$ ($0<n<50$) gradova i
njihove prosečne temperature, a zatim ispisuje one gradove koji imaju
idealnu temperaturu za klizanje: od $3$ do $8$
stepeni. \textit{Napomena: probati sa testiranjem zadataka pomoću
  preusmeravanja.}

\begin{maxitest}
\begin{upotreba}{1}
#\naslovInt#
#\izlaz{Unesite broj n:}\ulaz{4}#
#\izlaz{Unesite grad i temperaturu: }\ulaz{Beograd 7}#
#\izlaz{Unesite grad i temperaturu: }\ulaz{Uzice 1.5}#
#\izlaz{Unesite grad i temperaturu: }\ulaz{Subotica 4}#
#\izlaz{Unesite grad i temperaturu: }\ulaz{Zrenjanin 9}#
#\izlaz{Gradovi sa idealnom temperaturom za klizanje u decembru:}#
#\izlaz{Beograd}#
#\izlaz{Subotica}#
\end{upotreba}
\end{maxitest}

\begin{maxitest}
\begin{upotreba}{2}
#\naslovInt#
#\izlaz{Unesite broj n:}\ulaz{2}#
#\izlaz{Unesite grad i temperaturu: }\ulaz{Varsava 11}#
#\izlaz{Unesite grad i temperaturu: }\ulaz{Prag 2}#
#\izlaz{Gradovi sa idealnom temperaturom za klizanje u decembru:}#
\end{upotreba}
\end{maxitest}
 

\linkresenje{struc.3}
\end{Exercise}
\begin{Answer}[ref=struc.3]
\includecode{resenja/2_PredstavljanjePodataka/2.5_Strukture/3.c}
\end{Answer}


\begin{Exercise}[label=struc.4] 
 Definisati strukturu \kckod{ParReci} koja sadrži reč na srpskom
 jeziku i odgovarajući prevod na engleski jezik. Napisati program koji
 do kraja ulaza učitava sve parove reči, a potom za rečenicu koja se
 zadaje u jednoj liniji ispisati prevod. Ako je reč u rečenici
 nepoznata umesto nje ispisati odgovarajući broj zvezdica. Maksimalna
 dužina reči je 50 karaktera, ukupan broj parova reči je maksimalno
 $100$, a maksimalna dužina rečenice je $100$
 karaktera. \napomena{Probati sa testiranjem zadataka pomoću
   preusmeravanja.}

\begin{miditest}
\begin{upotreba}{1}
#\naslovInt#
#\ulaz{zima winter}#
#\ulaz{godina year}#
#\ulaz{sreca happiness}#
#\ulaz{programiranje programming}#
#\ulaz{caj tea}#
#\izlaz{Unesite recenicu za prevod: }#
#\ulaz{piti caj zimi je sreca}#
#\izlaz{**** tea **** ** happiness}#
\end{upotreba}
\end{miditest}
\begin{miditest}
\begin{upotreba}{2}
#\naslovInt#
#\ulaz{zima winter}#
#\ulaz{pas dog}#
#\ulaz{sreca happiness}#
#\ulaz{prijatelj friend}#
#\ulaz{solja cup}#
#\ulaz{covek man}#
#\izlaz{Unesite recenicu za prevod: }#
#\ulaz{pas je covekov najbolji prijatelj}#
#\izlaz{dog is ******* best friend}#
\end{upotreba}
\end{miditest}

\linkresenje{struc.4}
\end{Exercise}
\begin{Answer}[ref=struc.4]
\includecode{resenja/2_PredstavljanjePodataka/2.5_Strukture/4.c}
\end{Answer}


\begin{Exercise}[label=struc.6] 
Cenoteka pomaže kupcima da pronađu najpovoljniju cenu za proizvod koji
žele da kupe. Napisati program koji učitava najpre broj različitih
prodavnica (ceo broj manji od $50$) a zatim i podatke o ceni traženog
artikla -- zadaje se naziv prodavnice (niske maksimalne dužine $20$
karaktera) i cena u toj prodavnici (realan broj). Korisnik zadaje
željenu cenu proizvoda, a program ispisuje imena svih onih prodavnica
u kojima je cena proizvoda jednaka ili manja od željene. U slučaju
greške ispisati odgovarajuću poruku.

\begin{miditest}
\begin{upotreba}{1}
#\naslovInt#
#\izlaz{Uneti broj prodavnica: }\ulaz{5}#
#\ulaz{idea 58.9}#
#\ulaz{maxi 58.2}#
#\ulaz{roda 55.1}#
#\ulaz{tempo 54.5}#
#\ulaz{interex 57.99}#
#\izlaz{Uneti zeljenu cenu: }\ulaz{57.0}#
#\izlaz{Povoljne prodavnice su:}#
#\izlaz{roda}#
#\izlaz{tempo}#
\end{upotreba}
\end{miditest}
\begin{miditest}
\begin{upotreba}{2}
#\naslovInt#
#\izlaz{Uneti broj prodavnica: }\ulaz{4}#
#\ulaz{dm 43.2}#
#\ulaz{lily 45.99}#
#\ulaz{benu\_apoteke 43.99}#
#\ulaz{sephora 50.99}#
#\izlaz{Uneti zeljenu cenu: }\ulaz{47.00}#
#\izlaz{Povoljne prodavnice su:}#
#\izlaz{dm}#
#\izlaz{lily}#
#\izlaz{benu\_apoteke}#
\end{upotreba}
\end{miditest}

\linkresenje{struc.6}
\end{Exercise}
\begin{Answer}[ref=struc.6]
\includecode{resenja/2_PredstavljanjePodataka/2.5_Strukture/6.c}
\end{Answer}


\begin{Exercise}[label=struc.7] 
Statistički zavod Srbije istražuje kako rade obdaništa u Srbiji. Za
dato obdanište dobija spisak $n$ dece sa kolonama: pol (\kckod{m} ili
\kckod{z}), broj godina (od $3$ do $6$) i ocena koju je dete dalo radu
obdaništa (od $1$ do $5$). Maksimalan broj dece u obdaništu je
$200$. Napisati program koji za decu datog pola i broja godina
ispisuje na tri decimale prosečnu ocenu obdaništa. U slučaju
neispravnog unosa ispisati odgovarajuću poruku.

\begin{miditest}
\begin{upotreba}{1}
#\naslovInt#
#\izlaz{Uneti broj dece: }\ulaz{5}#
#\izlaz{Uneti podatke za svako dete, pol,}#
#\izlaz{broj godina i ocenu:}#
#\ulaz{m 3 5}#
#\ulaz{z 3 4}#
#\ulaz{m 4 2}#
#\ulaz{m 5 4}#
#\ulaz{m 3 4}#
#\izlaz{Uneti pol i broj godina: }\ulaz{m 3}#
#\izlaz{Prosecna ocena je: 4.500.}#
\end{upotreba}
\end{miditest}
\begin{miditest}
\begin{upotreba}{2}
#\naslovInt#
#\izlaz{Uneti broj dece: }\ulaz{10}#
#\izlaz{Uneti podatke za svako dete, pol,}#
#\izlaz{broj godina i ocenu:}#
#\ulaz{m 3 5}#
#\ulaz{z 4 4}#
#\ulaz{m 5 4}#
#\ulaz{z 4 3}#
#\ulaz{z 3 2}#
#\ulaz{z 4 5}#
#\ulaz{m 6 5}#
#\ulaz{z 4 4}#
#\ulaz{z 4 5}#
#\ulaz{m 6 3}#
#\izlaz{Uneti pol i broj godina: }\ulaz{z 4}#
#\izlaz{Prosecna ocena je: 4.200.}#
\end{upotreba}
\end{miditest}

\begin{miditest}
\begin{upotreba}{3}
#\naslovInt#
#\izlaz{Uneti broj dece: }\ulaz{15}#
#\izlaz{Uneti podatke za svako dete, pol,}# 
#\izlaz{broj godina i ocenu:}#
#\ulaz{m 3 2}#
#\ulaz{z 7 5}#
#\izlaz{Neispravan broj godina.}#
\end{upotreba}
\end{miditest}
\begin{miditest}
\begin{upotreba}{4}
#\naslovInt#
#\izlaz{Uneti broj dece: }\ulaz{2}#
#\izlaz{Uneti podatke za svako dete, pol,}# 
#\izlaz{broj godina i ocenu:}#
#\ulaz{m 3 2}#
#\ulaz{z 3 5}#
#\izlaz{Uneti pol i broj godina: }\ulaz{h 5}#
#\izlaz{Neispravan pol.}#
\end{upotreba}
\end{miditest}

\linkresenje{struc.7}
\end{Exercise}
\begin{Answer}[ref=struc.7]
\includecode{resenja/2_PredstavljanjePodataka/2.5_Strukture/7.c}
\end{Answer}

%---------------------------------------------------------------------------------------------------------------------------------------------------------------------------------
\begin{comment}
\begin{Exercise}[label=v2.5_02] 
  Data je struktura
\begin{ckod}
    typedef struct Student
    {
      char ime[MAX];
      char prezime[MAX];
      char smer;
      float prosek;
    } STUDENT;
  \end{ckod}
  \begin{enumerate}
  
\item  Napisati funkciju koja ucitava sa standardnog ulaza podatke o studentu. Mozemo pretpostaviti da 
    ime i prezime studenta ne sadrze vise od 30 karaktera.
\item Napisati funkciju koja ispisuje podatke o studentu na standardni izlaz.
\item Ucitati niz od n studenata i :
\begin{enumerate}
\item ispisati imena i prezimena onih koji su na smeru R
      \item ispisati podatke za studenta sa najvecim prosekom; ako ima vise takvih studenata, ispisati 
\begin{enumerate}
         \item sve njih 
	 \item prvog 
	 \item poslednjeg 
\end{enumerate}
\end{enumerate}
  \end{enumerate}

\linkresenje{v2.5_02}
\end{Exercise}
\begin{Answer}[ref=v2.5_02]
\includecode{resenja/2_PredstavljanjePodataka/2.5_Strukture/2/2.c}
\end{Answer}

\begin{Exercise}[label=v2.5_05] 
   Struktura IZRAZ opisuje numericki izraz nad celim brojevima koji se sastoji
   od dva celobrojna operanda, numericke operacije nad celim brojevima i
   vrednosti izraza:
   
   typedef struct izraz
   {
	char o;
	int x;
	int y;
   } IZRAZ;

\begin{enumerate}

   \item Napisati funkciju koja ispituje da li je dati izraz korektno 
   zadat i vraca 1 ako jeste a 0 u suprotnom. Podrazumevamo da je 
   izraz korektno zadat ako operacija odgovara +,-,* ili / i u slucaju
   deljenja drugi operand je razlicit od 0.
   
   \item Napisati funkciju koja za dati izraz odredjuje vrednost izraza. 
   
   \item Napisati funkciju koja ucitava dati izraz. Funkcija
   treba da ucita sa standardnog ulaza operaciju i dva
   operanda u polja o, x i y strukture IZRAZ. Funkcija vraca
   1 ako je ucitavanje bilo uspesno, tj. ako je izraz bio
   korektno zadat ili 0 u suprotnom. 
   
   \item Napisati funkciju koja stampa dati izraz infiksno, u obliku
   x o y = vr. Na primer, za izraz + 4 17 ispis treba 
   da bude 4+17=21
   
  
\end{enumerate}
Napisati glavni program koji ucitava prirodan broj $n<1000$ a zatim n izraza
   u notaciji \\
   + 4 17 \\
   - 8 -16 \\
   Program treba da ispise maksimalnu vrednost medju unetim izrazima i da ispise one
   izraze cija je vrednost manja od polovine maksimalne vrednosti.
   	
\linkresenje{v2.5_05}
\end{Exercise}
\begin{Answer}[ref=v2.5_05]
\includecode{resenja/2_PredstavljanjePodataka/2.5_Strukture/5/5.c}
\end{Answer}


\begin{Exercise}[label=v2.5_04] 
   Napisati program koji izracunava prosecnu cenu jedne potrosacke
   korpe. Potrosacka korpa se sastoji od broja kupljenih artikala i 
   niza kupljenih artikala. Svaki artikal odredjen je svojim nazivom, 
   kolicinom i cenom. Program treba da ucita broj potrosaca n (najvise 100), 
   zatim podatke za n potrosackih korpi i da na osnovu ucitanih podataka 
   izracuna prosecnu cenu potrosacke korpe. Ucitavanje se vrsi sa standarnog
   ulaza pri cemu se prvo zada broj artikala, a zatim za svaki artikal naziv, 
   kolicina i cena.  Mozemo pretpostaviti da nijedan 
   potrosac nece kupiti vise od 20 artikala, kao i da naziv svakog artikla
   sadrzi maksimalno 30 karaktera.
\linkresenje{v2.5_04}
\end{Exercise}
\begin{Answer}[ref=v2.5_04]
\includecode{resenja/2_PredstavljanjePodataka/2.5_Strukture/4/4.c}
\end{Answer}





\begin{Exercise}[label=p2.5_02] 
Definisati strukturu $Lopta$ sa poljima $poluprecnik$ (ceo broj u centimetrima) i $boja$ (enumeracioni tip koji uključuje plavu, žutu, crvenu i zelenu boju). Zatim učitati informacije o $n$ lopti ($0<n<50$) i ispisati ukupnu zapreminu, kao i broj crvenih lopti. \textit{Napomena: probati sa testiranjem zadataka pomoću preusmeravanja.}\\
\begin{maxitest}
\begin{upotreba}{1}
#\naslovInt#
#\izlaz{Unesite broj lopti: }\ulaz{4}#
#\izlaz{Unesite dalje poluprecnike i boje lopti (1-plava, 2-zuta, 3-crvena, 4-zelena): }#
#\izlaz{1.lopta: }\ulaz{4 1}#
#\izlaz{2.lopta: }\ulaz{1 3}#
#\izlaz{3.lopta: }\ulaz{2 3}#
#\izlaz{4.lopta: }\ulaz{10 4}#
#\izlaz{Ukupna zapremina: 4494.57}#
#\izlaz{Broj crvenih lopti: 2}#
\end{upotreba}
\end{maxitest}

\linkresenje{p2.5_02}
\end{Exercise}
\begin{Answer}[ref=p2.5_02]
\includecode{resenja/2_PredstavljanjePodataka/2.5_Strukture/praktikumi13/2.c}
\end{Answer}


\begin{Exercise}[label=p2.5_04] 
 Deda Mraz planira kupovinu poklona za studente koji su vredno učili C u toku godine. Na njegovoj listi se nalazi ime i prezime studenta (niske dužina do 50 karaktera) i njegova želja (niska maksimalne dužine 100 karaktera). Napisati program koji će služiti Deda Mrazu kao podsetnik: na osnovu liste koju je napravio, Deda Mraz može da unese ime i prezime studenta i da proveri njegovu želju. Ako ima više studenata sa istim imenom i prezimenom ispisati sve želje. \textit{Napomena: probati sa testiranjem zadataka pomoću preusmeravanja.}\\
\begin{maxitest}
\begin{upotreba}{1}
#\naslovInt#
#\izlaz{Ime i prezime studenta:}#
#\ulaz{Pera Peric}#
#\izlaz{Njegova zelja:}#
#\ulaz{privezak za kljuceve}#
#\izlaz{Jos vrednih studenata (da/ne)?}#
#\ulaz{da}#
#\izlaz{Ime i prezime studenta:}#
#\ulaz{Zika Zikic}#
#\izlaz{Njegova zelja:}#
#\ulaz{stap za pecanje}#
#\izlaz{Jos vrednih studenata (da/ne)?}#
#\ulaz{da}#
#\izlaz{Ime i prezime studenta:}#
#\ulaz{Mara Maric}#
#\izlaz{Njegova zelja:}#
#\ulaz{komplet Knutovih knjiga}#
#\izlaz{Jos vrednih studenata (da/ne)?}#
#\ulaz{ne}#
#\izlaz{Za podsecanje uneti ime i prezime:}#
#\ulaz{Pera Peric}#
#\izlaz{Novogodisnja zelja: privezak za kljuceve}#
\end{upotreba}
\end{maxitest}

\linkresenje{p2.5_04}
\end{Exercise}
\begin{Answer}[ref=p2.5_04]
\includecode{resenja/2_PredstavljanjePodataka/2.5_Strukture/praktikumi13/4.c}
\end{Answer}




 













\begin{Exercise}[label=p2.5_] 
\begin{itemize}
\item Definisati tip podataka \verb|TACKA| pogodan za predstavljanje ta\v cke
Dekartovske ravni (\v{c}ije su $x$ i $y$ koordinate podaci tipa \verb|double|).

\item Definisati funkciju \verb|double rastojanje(TACKA a, TACKA b)| koja
izra\v cunava rastojanje izmedju dve ta\v{c}ke.

\item Definisati funkciju \verb|unsigned ucitaj_poligon(TACKA* tacke, unsigned n)|
koja u\v citava \verb|n| puta po dve vrednosti tipa \verb|double|
 (koje predstavljaju koordinate temena poligona) i upisuje ih u zadati niz
 ta\v{c}aka. Funkcija vra\' ca broj uspe\v sno u\v citanih ta\v{c}aka.

\item Definisati funkciju \verb|double obim(TACKA* poligon, unsigned n)|
koja izra\v cunava obim poligona sa \verb|n| ta\v{c}aka u zadatom nizu
(napomena: ne zaboraviti stranicu koja spaja poslednje i prvo teme).

\item Definisati funkciju \verb|double maksimalna_stranica(TACKA* poligon, unsigned n)|
koja izra\v cunava du\v{z}inu najdu\v{z}e stranice poligona sa \verb|n| ta\v{c}aka
u zadatom nizu (napomena: ne zaboraviti stranicu koja spaja poslednje i prvo teme).

\item Definisati funkciju \verb|main| u kojoj se sa standardnog ulaza
u\v{c}itava celobrojna nenegativna vrednost \verb|N| ($0 < N \le 100$).

Ina\v{c}e, poziva se funkcija \verb|ucitaj_poligon|. Ukoliko je
uspe\v{s}no u\v{c}itano \verb|m| ta\v{c}ka (\verb|N| ne mora da
bude jednako \verb|m|), onda se poziva funkcija \verb|obim|
za \verb|m| u\v{c}itanih ta\v{c}aka i ispisuje njen rezultat
na standardni izlaz (ukoliko ova funkcija nije implementirana
--- ispisati na standardni izlaz simbol \verb|?|).
Posle toga se poziva funkcija \verb|maksimalna_stranica|
za \verb|m| u\v{c}itanih ta\v{c}aka i ispisuje njen rezultat
na standardni izlaz (ukoliko ova funkcija nije implementirana
--- ispisati na standardni izlaz simbol \verb|?|).
\end{itemize}
\linkresenje{p2.5_}
\end{Exercise}
\begin{Answer}[ref=p2.5_]
%\includecode{resenja/2_PredstavljanjePodataka/2.5_Strukture/5/5.c}
\end{Answer}

\end{comment}

\section{Rešenja}
\shipoutAnswer


