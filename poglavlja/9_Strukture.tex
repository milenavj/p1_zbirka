
\section{Strukture}

%---------------------------------------------------------dva lagana zadatka za pocetak

\begin{Exercise}[label=struc.1] 
Definisati strukturu kojom se predstavlja kompleksan broj. Napisati
funkcije koje izračunavaju zbir, razliku, proizvod i količnik dva
kompleksna broja, a zati i program koji učitava dva kompleksna broja i
ispisuje vrednost zbira, razlike, proizvoda i količnika. 

\begin{maxitest}
\begin{upotreba}{1}
#\naslovInt#
#\izlaz{Unesite realni i imaginarni deo prvog broja: }\ulaz{1 2}#
#\izlaz{Unesite realni i imaginarni deo drugog broja: }\ulaz{-2 3}#
#\izlaz{Zbir: -1.00+5.00*i}#
#\izlaz{Razlika: 3.00-1.00*i}#
#\izlaz{Proizvod: -8.00-1.00*i}#
#\izlaz{Kolicnik: 0.31-0.54*i}#
\end{upotreba}
\end{maxitest}
\linkresenje{struc.1}
\end{Exercise}
\begin{Answer}[ref=struc.1]
\includecode{resenja/2_PredstavljanjePodataka/2.5_Strukture/1.c}
\end{Answer}

\begin{Exercise}[label=struc.5] 
Definisati strukturu kojom se predstavlja razlomak. Napisati funkcije
koje izračunavaju zbir i proizvod dva razlomka. Unosi se broj $n$ a
potom i $n$ razlomaka sa standarnog ulaza. Ispisati njihov zbir i
proizvod na standardni izlaz.

\begin{miditest}
\begin{upotreba}{1}
#\naslovInt#
#\izlaz{Unesi broj razlomaka: }\ulaz{5}#
#\izlaz{Uneti razlomke:}#
#\ulaz{1 2}#
#\ulaz{7 8}#
#\ulaz{3 4}#
#\ulaz{5 6}#
#\ulaz{2 9}#
#\izlaz{Suma svih razlomaka je 229/72.}#
#\izlaz{Proizvod svih razlomaka je 35/576.}#
\end{upotreba}
\end{miditest}
\begin{miditest}
\begin{upotreba}{2}
#\naslovInt#
#\izlaz{Unesi broj razlomaka: }\ulaz{10}#
#\izlaz{Uneti razlomke:}#
#\ulaz{4 3}#
#\ulaz{12 25}#
#\ulaz{3 8}#
#\ulaz{1 3}#
#\ulaz{8 9}#
#\ulaz{2 3}#
#\ulaz{5 6}#
#\ulaz{-24 50}#
#\ulaz{7 18}#
#\ulaz{-7 19}#
#\izlaz{Suma svih razlomaka je 6089/1368.}#
#\izlaz{Proizvod svih razlomaka je 1568/577125.}#
\end{upotreba}
\end{miditest}

\linkresenje{struc.5}
\end{Exercise}
\begin{Answer}[ref=struc.5]
\includecode{resenja/2_PredstavljanjePodataka/2.5_Strukture/5.c}
\end{Answer}

%-------------------------------------------------------------------razni tekstualni zadaci poredjani po tezini

\begin{Exercise}[label=struc.2] 
 Zimi su prehlade česte i treba unositi više vitamina C. Struktura
 $Vocka$ sadrži ime voćke (nisku maksimalne dužine $20$ karaktera) i
 količinu vitamina C u miligramima (realan broj). Napisati program
 koji učitava podatke o voćkama sve do unosa reči \kckod{KRAJ} i
 ispisuje ime voćke sa najviše vitamina C. Pretpostaviti da broj voćki
 neće biti veći od $50$. \napomena{Probati sa testiranjem zadataka
   pomoću preusmeravanja.}

\begin{maxitest}
\begin{upotreba}{1}
#\naslovInt#
#\izlaz{Unesite ime voćke i njenu količinu vitamina C: }\ulaz{jabuka 4.6}#
#\izlaz{Unesite ime voćke i njenu količinu vitamina C: }\ulaz{limun 51}#
#\izlaz{Unesite ime voćke i njenu količinu vitamina C: }\ulaz{kivi 92.7}#
#\izlaz{Unesite ime voćke i njenu količinu vitamina C: }\ulaz{banana 8.7}#
#\izlaz{Unesite ime voćke i njenu količinu vitamina C: }\ulaz{pomorandza 53.2}#
#\izlaz{Unesite ime voćke i njenu količinu vitamina C: }\ulaz{KRAJ}#
#\izlaz{Voce sa najvise C vitamina je: kivi}#
\end{upotreba}
\end{maxitest}

\linkresenje{struc.2}
\end{Exercise}
\begin{Answer}[ref=struc.2]
\includecode{resenja/2_PredstavljanjePodataka/2.5_Strukture/2.c}
\end{Answer}



\begin{Exercise}[label=struc.3] 
Definisati strukturu \kckod{Grad} u kojoj se nalazi ime grada (niska
dužine $20$ karaktera) i prosečna temperatura u toku decembra (realan
broj). Napisati program koji učitava imena $n$ ($0<n<50$) gradova i
njihove prosečne temperature, a zatim ispisuje one gradove koji imaju
idealnu temperaturu za klizanje: od $3$ do $8$
stepeni. \textit{Napomena: probati sa testiranjem zadataka pomoću
  preusmeravanja.}

\begin{maxitest}
\begin{upotreba}{1}
#\naslovInt#
#\izlaz{Unesite broj n:}\ulaz{4}#
#\izlaz{Unesite grad i temperaturu: }\ulaz{Beograd 7}#
#\izlaz{Unesite grad i temperaturu: }\ulaz{Uzice 1.5}#
#\izlaz{Unesite grad i temperaturu: }\ulaz{Subotica 4}#
#\izlaz{Unesite grad i temperaturu: }\ulaz{Zrenjanin 9}#
#\izlaz{Gradovi sa idealnom temperaturom za klizanje u decembru:}#
#\izlaz{Beograd}#
#\izlaz{Subotica}#
\end{upotreba}
\end{maxitest}

\begin{maxitest}
\begin{upotreba}{2}
#\naslovInt#
#\izlaz{Unesite broj n:}\ulaz{2}#
#\izlaz{Unesite grad i temperaturu: }\ulaz{Varsava 11}#
#\izlaz{Unesite grad i temperaturu: }\ulaz{Prag 2}#
#\izlaz{Gradovi sa idealnom temperaturom za klizanje u decembru:}#
\end{upotreba}
\end{maxitest}
 
\linkresenje{struc.3}
\end{Exercise}
\begin{Answer}[ref=struc.3]
\includecode{resenja/2_PredstavljanjePodataka/2.5_Strukture/3.c}
\end{Answer}


\begin{Exercise}[label=struc.4] 
 Definisati strukturu \kckod{ParReci} koja sadrži reč na srpskom
 jeziku i odgovarajući prevod na engleski jezik. Napisati program koji
 do kraja ulaza učitava sve parove reči, a potom za rečenicu koja se
 zadaje u jednoj liniji ispisati prevod. Ako je reč u rečenici
 nepoznata umesto nje ispisati odgovarajući broj zvezdica. Maksimalna
 dužina reči je 50 karaktera, ukupan broj parova reči je maksimalno
 $100$, a maksimalna dužina rečenice je $100$
 karaktera. \napomena{Probati sa testiranjem zadataka pomoću
   preusmeravanja.}

\begin{miditest}
\begin{upotreba}{1}
#\naslovInt#
#\ulaz{zima winter}#
#\ulaz{godina year}#
#\ulaz{sreca happiness}#
#\ulaz{programiranje programming}#
#\ulaz{caj tea}#
#\izlaz{Unesite recenicu za prevod: }#
#\ulaz{piti caj zimi je sreca}#
#\izlaz{**** tea **** ** happiness}#
\end{upotreba}
\end{miditest}
\begin{miditest}
\begin{upotreba}{2}
#\naslovInt#
#\ulaz{zima winter}#
#\ulaz{pas dog}#
#\ulaz{sreca happiness}#
#\ulaz{prijatelj friend}#
#\ulaz{solja cup}#
#\ulaz{covek man}#
#\izlaz{Unesite recenicu za prevod: }#
#\ulaz{pas je covekov najbolji prijatelj}#
#\izlaz{dog is ******* best friend}#
\end{upotreba}
\end{miditest}

\linkresenje{struc.4}
\end{Exercise}
\begin{Answer}[ref=struc.4]
\includecode{resenja/2_PredstavljanjePodataka/2.5_Strukture/4.c}
\end{Answer}


\begin{Exercise}[label=struc.6] 
Cenoteka pomaže kupcima da pronađu najpovoljniju cenu za proizvod koji
žele da kupe. Napisati program koji učitava najpre broj različitih
prodavnica (ceo broj manji od $50$) a zatim i podatke o ceni traženog
artikla -- zadaje se naziv prodavnice (niske maksimalne dužine $20$
karaktera) i cena u toj prodavnici (realan broj). Korisnik zadaje
željenu cenu proizvoda, a program ispisuje imena svih onih prodavnica
u kojima je cena proizvoda jednaka ili manja od željene. U slučaju
greške ispisati odgovarajuću poruku.

\begin{miditest}
\begin{upotreba}{1}
#\naslovInt#
#\izlaz{Uneti broj prodavnica: }\ulaz{5}#
#\ulaz{idea 58.9}#
#\ulaz{maxi 58.2}#
#\ulaz{roda 55.1}#
#\ulaz{tempo 54.5}#
#\ulaz{interex 57.99}#
#\izlaz{Uneti zeljenu cenu: }\ulaz{57.0}#
#\izlaz{Povoljne prodavnice su:}#
#\izlaz{roda}#
#\izlaz{tempo}#
\end{upotreba}
\end{miditest}
\begin{miditest}
\begin{upotreba}{2}
#\naslovInt#
#\izlaz{Uneti broj prodavnica: }\ulaz{4}#
#\ulaz{dm 43.2}#
#\ulaz{lily 45.99}#
#\ulaz{benu\_apoteke 43.99}#
#\ulaz{sephora 50.99}#
#\izlaz{Uneti zeljenu cenu: }\ulaz{47.00}#
#\izlaz{Povoljne prodavnice su:}#
#\izlaz{dm}#
#\izlaz{lily}#
#\izlaz{benu\_apoteke}#
\end{upotreba}
\end{miditest}

\linkresenje{struc.6}
\end{Exercise}
\begin{Answer}[ref=struc.6]
\includecode{resenja/2_PredstavljanjePodataka/2.5_Strukture/6.c}
\end{Answer}


\begin{Exercise}[label=struc.7] 
Statistički zavod Srbije istražuje kako rade obdaništa u Srbiji. Za
dato obdanište dobija spisak $n$ dece sa kolonama: pol (\kckod{m} ili
\kckod{z}), broj godina (od $3$ do $6$) i ocena koju je dete dalo radu
obdaništa (od $1$ do $5$). Maksimalan broj dece u obdaništu je
$200$. Napisati program koji za decu datog pola i broja godina
ispisuje na tri decimale prosečnu ocenu obdaništa. U slučaju
neispravnog unosa ispisati odgovarajuću poruku.

\begin{miditest}
\begin{upotreba}{1}
#\naslovInt#
#\izlaz{Uneti broj dece: }\ulaz{5}#
#\izlaz{Uneti podatke za svako dete, pol,}#
#\izlaz{broj godina i ocenu:}#
#\ulaz{m 3 5}#
#\ulaz{z 3 4}#
#\ulaz{m 4 2}#
#\ulaz{m 5 4}#
#\ulaz{m 3 4}#
#\izlaz{Uneti pol i broj godina: }\ulaz{m 3}#
#\izlaz{Prosecna ocena je: 4.500.}#
\end{upotreba}
\end{miditest}
\begin{miditest}
\begin{upotreba}{2}
#\naslovInt#
#\izlaz{Uneti broj dece: }\ulaz{10}#
#\izlaz{Uneti podatke za svako dete, pol,}#
#\izlaz{broj godina i ocenu:}#
#\ulaz{m 3 5}#
#\ulaz{z 4 4}#
#\ulaz{m 5 4}#
#\ulaz{z 4 3}#
#\ulaz{z 3 2}#
#\ulaz{z 4 5}#
#\ulaz{m 6 5}#
#\ulaz{z 4 4}#
#\ulaz{z 4 5}#
#\ulaz{m 6 3}#
#\izlaz{Uneti pol i broj godina: }\ulaz{z 4}#
#\izlaz{Prosecna ocena je: 4.200.}#
\end{upotreba}
\end{miditest}

\begin{miditest}
\begin{upotreba}{3}
#\naslovInt#
#\izlaz{Uneti broj dece: }\ulaz{15}#
#\izlaz{Uneti podatke za svako dete, pol,}# 
#\izlaz{broj godina i ocenu:}#
#\ulaz{m 3 2}#
#\ulaz{z 7 5}#
#\izlaz{Neispravan broj godina.}#
\end{upotreba}
\end{miditest}
\begin{miditest}
\begin{upotreba}{4}
#\naslovInt#
#\izlaz{Uneti broj dece: }\ulaz{2}#
#\izlaz{Uneti podatke za svako dete, pol,}# 
#\izlaz{broj godina i ocenu:}#
#\ulaz{m 3 2}#
#\ulaz{z 3 5}#
#\izlaz{Uneti pol i broj godina: }\ulaz{h 5}#
#\izlaz{Neispravan pol.}#
\end{upotreba}
\end{miditest}

\linkresenje{struc.7}
\end{Exercise}
\begin{Answer}[ref=struc.7]
\includecode{resenja/2_PredstavljanjePodataka/2.5_Strukture/7.c}
\end{Answer}


\begin{Exercise}[label=struc.11] 
Definisati strukturu kojom se opisuje student. Student je zadat svojim
imenom i prezimenom (oba su maksimalne dužine $30$ karaktera), smerom
(R, I, V, N, T, O) i prosečnom ocenom. Napisati program koji učitava
podatke o $n$ studenata, zatim učitava smer i ispisuje imena i
prezimena onih studenta koji su sa datog smera. Potom ispisati podatke
za studenta koji ima najveći prosek. Ako ima više takvih studenata
ispisati sve njih. Maksimalan broj studenata je $2000$. U slučaju
greške ispisati odgovarajuću poruku.
 
\begin{miditest}
\begin{upotreba}{1}
#\naslovInt#
#\izlaz{Uneti broj studenata: }\ulaz{5}#
#\izlaz{Uneti podatke o studentima:}#
#\izlaz{0. student: }\ulaz{Kocic Marija R 9.14}#
#\izlaz{1. student: }\ulaz{Tanja Mratinkovic R 7.88}#
#\izlaz{2. student: }\ulaz{Mihailo Simic N 8.44}#
#\izlaz{3. student: }\ulaz{Milena Medar I 9.14}#
#\izlaz{4. student: }\ulaz{Ljubica Mihic N 9.00}#
#\izlaz{Uneti smer: }\ulaz{R}#
#\izlaz{Studenti sa R smera:}#
#\izlaz{Kocic Marija}#
#\izlaz{Tanja Mratinkovic}#
#\izlaz{---------------------}#
#\izlaz{Svi studenti koji imaju maksimalni prosek:}#
#\izlaz{Kocic Marija, R, 9.14}#
#\izlaz{Milena Medar, I, 9.14}#
\end{upotreba}
\end{miditest}
\begin{miditest}
\begin{upotreba}{2}
#\naslovInt#
#\izlaz{Uneti broj studenata: }\ulaz{4}#
#\izlaz{Uneti podatke o studentima:}#
#\izlaz{0. student: }\ulaz{Djordje Lazarevic N 9.05}#
#\izlaz{1. student: }\ulaz{Minja Peric W 7.70}#
#\izlaz{Nekorektan smer.}#
\end{upotreba}
\end{miditest}

\linkresenje{struc.11}
\end{Exercise}
\begin{Answer}[ref=struc.11]
\includecode{resenja/2_PredstavljanjePodataka/2.5_Strukture/11.c}
\end{Answer}

\begin{Exercise}[label=struc.13] 
Defnisati strukturu \kckod{Osoba} kojom se opisuje jedan unos u
imenik. Za svaku osobu su dati podaci: ime (maksimalne dužine $20$
karaktera), prezime (maksimalne dužine $30$ karaktera) i email adresa
(maksimalne dužine $50$ karaktera).  Napisati program koji učitava ceo
broj $n$ ($0 < n \le 50$) a zatim podatke o $n$ osoba. Ispisati imena
i prezimena svih osoba koje imaju gmail adresu (čija se email adresa
završava sa \kckod{@gmail.com}).  U slučaju greške ispisati
odgovarajuću poruku. Može se smatrati da je svaka email adresa dobro
zadata i sadrži samo jedno pojavljivanje znaka \kckod{@}.

\begin{miditest}
\begin{upotreba}{1}
#\naslovInt#
#\izlaz{Uneti broj osoba: }\ulaz{3}#
#\izlaz{Uneti podatke o osobama:}#
#\izlaz{ime, prezime i email.}#
#\ulaz{Dusko Dugousko dusko@yahoo.com}#
#\ulaz{Pink Panter panter@gmail.com}#
#\ulaz{Pera Detlic pd@gmail.com}#
#\izlaz{Vlasnici gmail naloga su:}#
#\izlaz{Pink Panter}#
#\izlaz{Pera Detlic}#
\end{upotreba}
\end{miditest}
\begin{miditest}
\begin{upotreba}{2}
#\naslovInt#
#\izlaz{Uneti broj osoba: }\ulaz{3}#
#\izlaz{Uneti podatke o osobama:}#
#\izlaz{ime, prezime i email.}#
#\ulaz{Homer Simpson homer@yahoo.com}#
#\ulaz{Mardz Simpson mardz@matf.bg.ac.rs}#
#\izlaz{Vlasnici gmail naloga su:}#
\end{upotreba}
\end{miditest}

\linkresenje{struc.13}
\end{Exercise}
\begin{Answer}[ref=struc.13]
\includecode{resenja/2_PredstavljanjePodataka/2.5_Strukture/13.c}
\end{Answer}

\begin{Exercise}[difficulty=1, label=struc.12] 
Napisati program koji izračunava prosečnu cenu jedne potrošačke
korpe. Potrošačka korpa se sastoji od broja kupljenih artikala i niza
kupljenih artikala. Svaki artikal određen je svojim nazivom, količinom
i cenom. Program treba da učita broj potrošača $n$ (najviše $100$),
zatim podatke za $n$ potrošačkih korpi i da na osnovu učitanih
podataka izračuna prosečnu cenu potrošačke korpe. Program ispisuje na
dve decimale račune svake potrošačke korpe i na kraju ispisuje
prosečnu cenu potrošačke korpe. Možemo pretpostaviti da nijedan
potrošač neće kupiti više od $20$ artikala, kao i da naziv svakog
artikla sadrži maksimalno $30$ karaktera. U slučaju greške ispisati
odgovarajuću poruku.

\begin{maxitest}
\begin{upotreba}{1}
#\naslovInt#
#\izlaz{Uneti broj potrosackih korpi: }\ulaz{3}#
#\izlaz{Uneti podatke o korpi: }#
#\izlaz{Broj artikala: }\ulaz{4}#
#\izlaz{Unesi artikal, naziv, kolicinu i cenu: }\ulaz{jabuke 10 22.4}#
#\izlaz{Unesi artikal, naziv, kolicinu i cenu: }\ulaz{dezodorans 1 120.99}#
#\izlaz{Unesi artikal, naziv, kolicinu i cenu: }\ulaz{C\_supa 3 36.56}#
#\izlaz{Unesi artikal, naziv, kolicinu i cenu: }\ulaz{sunka 1 230.99}#
#\izlaz{Uneti podatke o korpi: }#
#\izlaz{Broj artikala: }\ulaz{2}#
#\izlaz{Unesi artikal, naziv, kolicinu i cenu: }\ulaz{Jafa\_keks 55.78}#
#\izlaz{Unesi artikal, naziv, kolicinu i cenu: }\ulaz{Najlepse\_zelje 62.99}#
#\izlaz{Uneti podatke o korpi: }#
#\izlaz{Broj artikala: }\ulaz{3}#
#\izlaz{Unesi artikal, naziv, kolicinu i cenu: }\ulaz{prasak\_za\_ves 1 1199.99}#
#\izlaz{Unesi artikal, naziv, kolicinu i cenu: }\ulaz{omeksivac 1 279.99}#
#\izlaz{Unesi artikal, naziv, kolicinu i cenu: }\ulaz{protiv\_kamenca 1 699.99}#
#\izlaz{\ }#
#\izlaz{Korpa 0:}#
#\izlaz{\ \ \ \ \ \ \ \ jabuke 10 22.40}#
#\izlaz{\ \ \ \ \ \ \ \ dezodorans 1 120.99}#
#\izlaz{\ \ \ \ \ \ \ \ C\_supa 3 36.56}#
#\izlaz{\ \ \ \ \ \ \ \ sunka 1 230.99}#
#\izlaz{------------------------}#
#\izlaz{\ \ \ \ \ \ \ \ ukupno: 685.66}#
#\izlaz{\ }#
#\izlaz{Korpa 1:}#
#\izlaz{\ \ \ \ \ \ \ \ Jafa\_keks 55 0.78}#
#\izlaz{\ \ \ \ \ \ \ \ Najlepse\_zelje 62 0.99}#
#\izlaz{------------------------}#
#\izlaz{\ \ \ \ \ \ \ \ ukupno: 104.28}#
#\izlaz{\ }#
#\izlaz{Korpa 2:}#
#\izlaz{\ \ \ \ \ \ \ \ prasak\_za\_ves 1 1199.99}#
#\izlaz{\ \ \ \ \ \ \ \ omeksivac 1 279.99}#
#\izlaz{\ \ \ \ \ \ \ \ protiv\_kamenca 1 699.99}#
#\izlaz{------------------------}#
#\izlaz{\ \ \ \ \ \ \ \ ukupno: 2179.97}#
#\izlaz{\ }#
#\izlaz{Prosecna cena potrosacke korpe: 989.97}#

\end{upotreba}
\end{maxitest}

\linkresenje{struc.12}
\end{Exercise}
\begin{Answer}[ref=struc.12]
\includecode{resenja/2_PredstavljanjePodataka/2.5_Strukture/12.c}
\end{Answer}


%------------------matematicki zadaci


\begin{Exercise}[label=struc.9] 
Definisati strukturu \kckod{Lopta} sa poljima \kckod{poluprecnik} (ceo
broj u centimetrima) i \kckod{boja} (enumeracioni tip koji uključuje
plavu, žutu, crvenu i zelenu boju). Zatim učitati informacije o $n$
lopti ($0<n<50$) i ispisati ukupnu zapreminu, kao i broj crvenih
lopti. U slučaju greške ispisati odgovarajuću poruku.

\begin{miditest}
\begin{upotreba}{1}
#\naslovInt#
#\izlaz{Unesite broj lopti: }\ulaz{4}#
#\izlaz{Unesite dalje poluprecnike i boje lopti}# 
#\izlaz{(1-plava, 2-zuta, 3-crvena, 4-zelena):}#
#\izlaz{1.lopta: }\ulaz{4 1}#
#\izlaz{2.lopta: }\ulaz{1 3}#
#\izlaz{3.lopta: }\ulaz{2 3}#
#\izlaz{4.lopta: }\ulaz{10 4}#
#\izlaz{Ukupna zapremina: 4494.57}#
#\izlaz{Broj crvenih lopti: 2}#
\end{upotreba}
\end{miditest}
\begin{miditest}
\begin{upotreba}{2}
#\naslovInt#
#\izlaz{Unesite broj lopti: }\ulaz{8}#
#\izlaz{Unesite dalje poluprecnike i boje lopti}# 
#\izlaz{(1-plava, 2-zuta, 3-crvena, 4-zelena):}#
#\izlaz{1. lopta: }\izlaz{2 1}#
#\izlaz{2. lopta: }\izlaz{30 3}#
#\izlaz{3. lopta: }\izlaz{7 3}#
#\izlaz{4. lopta: }\izlaz{4 1}#  
#\izlaz{5. lopta: }\izlaz{5 2}#
#\izlaz{6. lopta: }\izlaz{6 2}#
#\izlaz{7. lopta: }\izlaz{12 3}#
#\izlaz{8. lopta: }\izlaz{14 2}#
#\izlaz{Ukupna zapremina: 134996.34}#
#\izlaz{Ukupno crvenih lopti: 3}#
\end{upotreba}
\end{miditest}

\begin{miditest}
\begin{upotreba}{3}
#\naslovInt#
#\izlaz{Unesite broj lopti: }\ulaz{8}#
#\izlaz{Unesite dalje poluprecnike i boje lopti}# 
#\izlaz{(1-plava, 2-zuta, 3-crvena, 4-zelena):}#
#\izlaz{1. lopta: }\izlaz{1 2}#
#\izlaz{2. lopta: }\izlaz{2 10}#
#\izlaz{Nekorektan unos.}#
\end{upotreba}
\end{miditest}

\linkresenje{struc.9}
\end{Exercise}
\begin{Answer}[ref=struc.9]
\includecode{resenja/2_PredstavljanjePodataka/2.5_Strukture/9.c}
\end{Answer}


\begin{Exercise}[label=struc.10] 
Napisati program za predstavljanje poligona i izračunavanje njegovog
obima i dužine stranica.
\begin{enumerate}
\item Definisati tip podataka \kckod{TACKA} pogodan za predstavljanje
  tačke Dekartovske ravni (čije su $x$ i $y$ koordinate podaci tipa
  \kckod{double}).

\item Definisati funkciju \kckod{double rastojanje(TACKA a, TACKA b)}
  koja izračunava rastojanje između dve tačke.

\item Definisati funkciju \kckod{unsigned ucitaj\_poligon(TACKA*
  tacke, unsigned n)} koja učitava maksimalno $n$ puta po dve
  vrednosti tipa \kckod{double} (koje predstavljaju koordinate temena
  poligona) i upisuje ih u zadati niz tačaka. Funkcija vraća broj
  uspešno učitanih tačaka.

\item Definisati funkciju \kckod{double obim(TACKA* poligon, unsigned
  n)} koja izračunava obim poligona sa $n$ tačaka u zadatom nizu
  \napomena{Prilikom računanja obima ne zaboraviti stranicu koja spaja
    poslednje i prvo teme}.

\item Definisati funkciju \kckod{double maksimalna\_stranica(TACKA*
  poligon, unsigned n)} koja izračunava dužinu najduže stranice
  poligona sa $n$ tačaka u zadatom nizu.

\item Napisati funkciju \kckod{double povrsina\_trougla(TACKA A, TACKA
  B, TACKA C)} za računanje površine trougla.

\item Napisati funkciju \kckod{double povrsina(TACKA* poligon,
  unsigned n)} za računanje površine konveksnog
  poligona. \napomena{Zadatak se može rešiti korišćenjem funkcije
    \kckod{povrsina\_trougla}}.

\item Napisati program koji učitava poligon sa maksimalno $N$ temena
  ($0 < N \le 1000$) i za učitani poligon ispisuje na tri decimale
  obim, dužinu maksimalnu stranice i površinu. Pretpostaviti da je
  uneseni poligon konveksan. Poligon mora imati barem $3$ temena. U
  slučaju greške ispisati odgovarajuću poruku.
\end{enumerate}

\begin{miditest}
\begin{upotreba}{1}
#\naslovInt#
#\izlaz{Uneti maksimalan broj tacaka poligona: }\ulaz{10}#
#\ulaz{0 0}#
#\ulaz{0 6}#
#\ulaz{3 3}#
#\izlaz{Obim poligona je 14.485.}#
#\izlaz{Duzina maksimalne stranice je 6.000.}#
#\izlaz{Povrsina poligona je 9.000.}#
\end{upotreba}
\end{miditest}
\begin{miditest}
\begin{upotreba}{2}
#\naslovInt#
#\izlaz{Uneti maksimalan broj tacaka poligona: }\ulaz{10}#
#\ulaz{0 0}#
#\ulaz{12 0}#
#\ulaz{13 2}#
#\ulaz{16 5}#
#\ulaz{20 10}#
#\ulaz{18 15}#
#\ulaz{15 20}#
#\ulaz{10 20}#
#\ulaz{8 15}#
#\ulaz{3 4}#
#\izlaz{Obim poligona je 63.566.}#
#\izlaz{Duzina maksimalne stranice je 12.083.}#
#\izlaz{Povrsina poligona je 247.500.}#
\end{upotreba}
\end{miditest}

\begin{miditest}
\begin{upotreba}{3}
#\naslovInt#
#\izlaz{Uneti maksimalan broj tacaka poligona: }\ulaz{4}#
#\ulaz{0 0}#
#\izlaz{Neispravan broj tacaka poligona.}#
\end{upotreba}
\end{miditest}

\linkresenje{struc.10}
\end{Exercise}
\begin{Answer}[ref=struc.10]
\includecode{resenja/2_PredstavljanjePodataka/2.5_Strukture/10.c}
\end{Answer}


\begin{Exercise}[difficulty=1, label=struc.8] 
Sefinisati strukturu \kckod{IZRAZ} kojom se opisuje numerički izraz
nad celim brojevima koji se sastoji od dva celobrojna operanda,
numeričke operacije (sabiranje, oduzimanje, množenje ili celobrojno
deljenje) nad celim brojevima.
\begin{enumerate}
\item Napisati funkciju koja ispituje da li je dati izraz korektno
  zadat i vraća $1$ ako jeste a $0$ u suprotnom. Podrazumevamo da je
  izraz korektno zadat ako operacija odgovara $+$, $-$, $*$ ili $/$ i
  u slučaju deljenja drugi operand je različit od $0$.
\item Napisati funkciju koja za dati izraz odredjuje vrednost izraza.
\item Napisati funkciju koja učitava dati izraz. Funkcija treba da
  učita sa standardnog ulaza izlaz koji je zadat prefiksno --- prvo
  operacija, a potom dva operanda. Funkcija vraća $1$ ako je
  učitavanje bilo uspešno, tj. ako je izraz bio korektno zadat ili $0$
  u suprotnom.
\item Napisati funkciju koja štampa dati izraz infiksno, u obliku
  "$operand_1$ $operacija$ $operand_2 = vrednost$".
\end{enumerate}

Napisati glavni program koji učitava prirodan broj $n$, ($n<1000$) a
zatim $n$ izraza u prefiksnoj notaciji. Program treba da ispiše
maksimalnu vrednost unetih izraza i sve izraze čija vrednost je manja
od polovine maksimalne vrednosti.

\begin{miditest}
\begin{upotreba}{1}
#\naslovInt#
#\izlaz{Uneti broj izraza: }\ulaz{4}#
#\izlaz{Uneti izraze u prefiksnoj notaciji:}#
#\ulaz{+ 10 4}#
#\ulaz{- 9 2}#
#\ulaz{* 11 2}#
#\ulaz{/ 7 3}#
#\izlaz{Maksimalna vrednost izraza: 22}#
#\izlaz{Izrazi cija je vrednost manja}# 
#\izlaz{od polovine maksimalne vrednosti:}#
#\izlaz{9 - 2 = 7}#
#\izlaz{7 / 3 = 2}#
\end{upotreba}
\end{miditest}
\begin{miditest}
\begin{upotreba}{2}
#\naslovInt#
#\izlaz{Uneti broj izraza: }\ulaz{10}#
#\izlaz{Uneti izraze u prefiksnoj notaciji:}#
#\ulaz{+ 10 2}#
#\ulaz{- -678 34}#
#\ulaz{* 77 2}#
#\ulaz{+ 1000 -23}#
#\ulaz{+ 102 4}#
#\ulaz{- 200 23}#
#\ulaz{/ 67 12}#
#\ulaz{/ 1000 2}#
#\ulaz{* 44 6}#
#\ulaz{/ 13 1}#
#\izlaz{Maksimalna vrednost izraza: 977}#
#\izlaz{Izrazi cija je vrednost manja}#
#\izlaz{od polovine maksimalne vrednosti:}#
#\izlaz{10 + 2 = 12}#
#\izlaz{-678 - 34 = -712}#
#\izlaz{77 * 2 = 154}#
#\izlaz{102 + 4 = 106}#
#\izlaz{200 - 23 = 177}#
#\izlaz{67 / 12 = 5}#
#\izlaz{44 * 6 = 264}#
#\izlaz{13 / 1 = 13}#
\end{upotreba}
\end{miditest}

\begin{miditest}
\begin{upotreba}{3}
#\naslovInt#
#\izlaz{Uneti broj izraza: }\ulaz{3}#
#\izlaz{Uneti izraze u prefiksnoj notaciji:}#
#\ulaz{* 1 2}#
#\ulaz{/ 3 0}#
#\izlaz{Deljenje nulom!}#
#\izlaz{Nekorektan unos}#
\end{upotreba}
\end{miditest}

\linkresenje{struc.8}
\end{Exercise}
\begin{Answer}[ref=struc.8]
\includecode{resenja/2_PredstavljanjePodataka/2.5_Strukture/8.c}
\end{Answer}


%---------------------------------------------------------------------------------------------------------------------------------------------------------------------------------
\begin{comment}



\begin{Exercise}[label=p2.5_04] 
 Deda Mraz planira kupovinu poklona za studente koji su vredno učili C u toku godine. Na njegovoj listi se nalazi ime i prezime studenta (niske dužina do 50 karaktera) i njegova želja (niska maksimalne dužine 100 karaktera). Napisati program koji će služiti Deda Mrazu kao podsetnik: na osnovu liste koju je napravio, Deda Mraz može da unese ime i prezime studenta i da proveri njegovu želju. Ako ima više studenata sa istim imenom i prezimenom ispisati sve želje. \textit{Napomena: probati sa testiranjem zadataka pomoću preusmeravanja.}\\
\begin{maxitest}
\begin{upotreba}{1}
#\naslovInt#
#\izlaz{Ime i prezime studenta:}#
#\ulaz{Pera Peric}#
#\izlaz{Njegova zelja:}#
#\ulaz{privezak za kljuceve}#
#\izlaz{Jos vrednih studenata (da/ne)?}#
#\ulaz{da}#
#\izlaz{Ime i prezime studenta:}#
#\ulaz{Zika Zikic}#
#\izlaz{Njegova zelja:}#
#\ulaz{stap za pecanje}#
#\izlaz{Jos vrednih studenata (da/ne)?}#
#\ulaz{da}#
#\izlaz{Ime i prezime studenta:}#
#\ulaz{Mara Maric}#
#\izlaz{Njegova zelja:}#
#\ulaz{komplet Knutovih knjiga}#
#\izlaz{Jos vrednih studenata (da/ne)?}#
#\ulaz{ne}#
#\izlaz{Za podsecanje uneti ime i prezime:}#
#\ulaz{Pera Peric}#
#\izlaz{Novogodisnja zelja: privezak za kljuceve}#
\end{upotreba}
\end{maxitest}

\linkresenje{p2.5_04}
\end{Exercise}
\begin{Answer}[ref=p2.5_04]
\includecode{resenja/2_PredstavljanjePodataka/2.5_Strukture/praktikumi13/4.c}
\end{Answer}



\end{comment}

\section{Rešenja}
\shipoutAnswer


