\chapter{Predstavljanje podataka}

\section{Nizovi}

\begin{Exercise}[label=v.skalarni_proizvod] 
Ako su $a = (a_1, \ldots, a_n)$ i $b = (b_1,\ldots, b_n)$ vektori
dimenzije $n$, njihov skalarni proizvod se definiše kao $a \cdot b = a_1\cdot b_1 +
\ldots + a_n\cdot b_n$. Napisati program koji računa skalarni proizvod
dva vektora. Vektori se zadaju kao celobrojni nizovi sa najviše $100$
elemenata.  Program učitava dimenziju i elemente nizova, a na izlaz
ispisuje vrednost skalarnog proizvoda.
U slučaju neispravnog unosa, ispisati odgovarajuću poruku o grešci. 

\begin{miditest}
\begin{upotreba}{1}
#\naslovInt#
#\izlaz{Unesite dimenziju vektora: }#
#\ulaz{5}#
#\izlaz{Unesite koordinate vektora a:}#
#\ulaz{8 -2 0 2 4}#
#\izlaz{Unesite koordinate vektora b:}#
#\ulaz{35 12 5 -6 -1}#
#\izlaz{Skalarni proizvod vektora a i b:}# 
#\izlaz{240}#
\end{upotreba}
\end{miditest}
\begin{miditest}
\begin{upotreba}{2}
#\naslovInt#
#\izlaz{Unesite dimenziju vektora: }#
#\ulaz{3}#
#\izlaz{Unesite koordinate vektora a:}#
#\ulaz{-1 0 1}#
#\izlaz{Unesite koordinate vektora b:}#
#\ulaz{5 5 5}#
#\izlaz{Skalarni proizvod vektora a i b:}# 
#\izlaz{0}#
\end{upotreba}
\end{miditest}

\begin{miditest}
\begin{upotreba}{3}
#\naslovInt#
#\izlaz{Unesite dimenziju vektora: }#
#\ulaz{120}#
#\izlaz{Greska: neispravan unos.}#
\end{upotreba}
\end{miditest}
\linkresenje{v.skalarni_proizvod}
\end{Exercise}

\ifresenja
\begin{Answer}[ref=v.skalarni_proizvod]
\includecode{resenja/3_PredstavljanjePodataka/2.1_Nizovi/skalarni_proizvod.c}
\end{Answer}
\fi


\begin{Exercise}[label=v.parni_elementi] 
Napisati program koji za učitani niz ispisuje:
\begin{enumerate}
\item elemente niza koji se nalaze na parnim pozicijama.
\item parne elemente niza.
\end{enumerate}
Maksimalan broj elemenata niza je $100$.
U slučaju neispravnog unosa, ispisati odgovarajuću poruku o grešci. 

\begin{miditest}
\begin{upotreba}{1}
#\naslovInt#
#\izlaz{Unesite dimenziju niza:}#
#\ulaz{6}#
#\izlaz{Unesite elemente niza:}#
#\ulaz{1 8 2 -5 -13 75}#
#\izlaz{Elementi niza na parnim pozicijama:}#
#\izlaz{1 2 -13}#
#\izlaz{Parni elementi niza:}#
#\izlaz{8 2}#
\end{upotreba}
\end{miditest}
\begin{miditest}
\begin{upotreba}{2}
#\naslovInt#
#\izlaz{Unesite dimenziju niza:}#
#\ulaz{3}#
#\izlaz{Unesite elemente niza:}#
#\ulaz{11 81 -63}#
#\izlaz{Elementi niza na parnim pozicijama:}#
#\izlaz{11 -63}#
#\izlaz{Parni elementi niza:}#
#\izlaz{}#
\end{upotreba}
\end{miditest}

\begin{miditest}
\begin{upotreba}{3}
#\naslovInt#
#\izlaz{Unesite dimenziju niza: }#
#\ulaz{-4}#
#\izlaz{Greska: neispravan unos.}#
\end{upotreba}
\end{miditest}
\linkresenje{v.parni_elementi}
\end{Exercise}
\ifresenja
\begin{Answer}[ref=v.parni_elementi]
\includecode{resenja/3_PredstavljanjePodataka/2.1_Nizovi/parni_elementi.c}
\end{Answer}
\fi


\begin{Exercise}[label=v.nizovi_funkcije_intro] 
Takmičari na Beogradskom maratonu su označeni rednim brojevima počevši
od $0$, a vreme za koje su istrčali maraton je dato minutima. Ovi
podaci zadati su nizom celih brojeva, pri čemu indeks niza označava
redni broj takmičara, a vrednosti u nizu označavaju vreme
trčanja. Napisati funkcije za rad sa ovim nizom.
\begin{enumerate}
\item Napisati funkciju \kckod{void ucitaj(int a[], int n)} koja
  učitava elemente niza $a$ dimenzije $n$.
\item Napisati funkciju \kckod{void stampaj(int a[], int n)} koja
  štampa elemente niza $a$ dimenzije $n$.
\item Napisati funkciju \kckod{int suma(int a[], int n)} koja računa i
  vraća ukupno vreme trčanja svih takmičara.
\item Napisati funkciju \kckod{float prosek(int a[], int n)} koja
  računa i vraća prosečno vreme (aritmetičku sredinu) trčanja
  takmičara.
\item Napisati funkciju \kckod{int maksimum(int a[], int n)} koja
  izračunava i vraća najduže vreme trčanja takmičara.
\item Napisati funkciju \kckod{int pozicija\_minimum(int a[], int n)}
  koja vraća redni broj pobednika Beogradskog maratona, tj. onog
  takmičara koji je najkraće trčao. U slučaju da ima više takvih
  takmičara, vratiti onog sa najmanjim indeksom.
\end{enumerate}
Napisati program koji testira rad zadatih funkcija. Maksimalan broj
takmičara je $1000$.
U slučaju neispravnog unosa, ispisati odgovarajuću poruku o grešci. 

\begin{miditest}
\begin{upotreba}{1}
#\naslovInt#
#\izlaz{Unesite dimenziju niza:}#
#\ulaz{5}#
#\ulaz{19 47 27 34 16}#
#\izlaz{Vreme trcanja takmicara: 19 47 27 34 16}#
#\izlaz{Ukupno vreme: 143}#
#\izlaz{Prosecno vreme trcanja: 28.6}#
#\izlaz{Maksimalno vreme trcanja: 47}#
#\izlaz{Indeks pobednika: 4}#
\end{upotreba}
\end{miditest}
\begin{miditest}
\begin{upotreba}{2}
#\naslovInt#
#\izlaz{Unesite dimenziju niza:}#
#\ulaz{-5}#
#\izlaz{Greska: neispravan unos.}#
\end{upotreba}
\end{miditest}
\linkresenje{v.nizovi_funkcije_intro}
\end{Exercise}

\ifresenja
\begin{Answer}[ref=v.nizovi_funkcije_intro]
\includecode{resenja/3_PredstavljanjePodataka/2.1_Nizovi/nizovi_funkcije_intro.c}
\end{Answer}
\fi


%ovaj ne treba resavati, ima za nijansu tezi zadatak dole, pronalaze se min-max i razmenjuju
\begin{Exercise}[label=vp.bez_resenja_11] 
Napisati funkciju \kckod{int min\_max(int a[], int n)} koja pronalazi
indekse najmanjeg i najvećeg elementa u nizu $a$ dimenzije $n$
koristeći samo jedan prolaz kroz niz. Funkcija kao povratnu vrednost
vraća manji od ta dva indeksa. Napisati program koji testira ovu
funkciju za učitane nizove celih brojeva maksimalne dužine $100$
elemenata.
U slučaju neispravnog unosa, ispisati odgovarajuću poruku o grešci. 

\begin{miditest}
\begin{upotreba}{1}
#\naslovInt#
#\izlaz{Unesite broj elemenata niza:}#
#\ulaz{7}#
#\izlaz{Unesite elemente niza:}#
#\ulaz{5 8 -4 11 17 89 1}#
#\izlaz{2}#
\end{upotreba}
\end{miditest}
\begin{miditest}
\begin{upotreba}{2}
#\naslovInt#
#\izlaz{Unesite broj elemenata niza:}#
#\ulaz{3}#
#\izlaz{Unesite elemente niza:}#
#\ulaz{9 11 6}#
#\izlaz{1}#
\end{upotreba}
\end{miditest}

\begin{miditest}
\begin{upotreba}{3}
#\naslovInt#
#\izlaz{Unesite broj elemenata niza:}#
#\ulaz{-45}#
#\izlaz{Greska: neispravan unos.}#
\end{upotreba}
\end{miditest}
%\linkresenje{vp.bez_resenja_11}
\end{Exercise}

%\ifresenja
%\begin{Answer}[ref=vp.bez_resenja_11]
%\includecode{resenja/3_PredstavljanjePodataka/2.1_Nizovi/bez_resenja_11.c}
%\end{Answer}
%\fi


\begin{Exercise}[label=p.broj_manjih_od_poslednjeg] 
 Napisati funkciju \kckod{int prebrojavanje(int a[], int n)} koja
 izračunava broj elemenata celobrojnog niza $a$ dužine $n$ koji su
 manji od poslednjeg elementa niza. Napisati i program koji testira
 rad funkcije. Maksimalan broj elemenata niza je $100$.
 U slučaju neispravnog unosa, ispisati odgovarajuću poruku o grešci. 

\begin{miditest}
\begin{upotreba}{1}
#\naslovInt#
#\izlaz{Unesite broj elemenata niza:}\ulaz{4}#
#\izlaz{Unesite elemente niza:}\ulaz{11 2 4 9}#
#\izlaz{2}#
\end{upotreba}
\end{miditest}
\begin{miditest}
\begin{upotreba}{2}
#\naslovInt#
#\izlaz{Unesite broj elemenata niza:}\ulaz{7}#
#\izlaz{Unesite elemente niza:}\ulaz{7 2 1 14 65 2 8}#
#\izlaz{4}#
\end{upotreba}
\end{miditest}

\begin{miditest}
\begin{upotreba}{3}
#\naslovInt#
#\izlaz{Unesite broj elemenata niza:}\ulaz{5}#
#\izlaz{Unesite elemente niza:}\ulaz{25 18 29 30 14}#
#\izlaz{0}#
\end{upotreba}
\end{miditest}
\begin{miditest}
\begin{upotreba}{4}
#\naslovInt#
#\izlaz{Unesite broj elemenata niza:}#
#\ulaz{-45}#
#\izlaz{Greska: neispravan unos.}#
\end{upotreba}
\end{miditest}

\linkresenje{p.broj_manjih_od_poslednjeg}
\end{Exercise}

\ifresenja
\begin{Answer}[ref=p.broj_manjih_od_poslednjeg]
\includecode{resenja/3_PredstavljanjePodataka/2.1_Nizovi/broj_manjih_od_poslednjeg.c}
\end{Answer}
\fi


\begin{Exercise}[label=p.broj_manjih_od_maksimuma] 
 Napisati funkciju \kckod{int prebrojavanje(int a[], int n)} koja
 izračunava broj parnih elemenata niza celih brojeva $a$ dužine $n$
 koji prethode maksimalnom elementu niza. Napisati i program koji
 testira rad funkcije. Maksimalan broj elemenata niza je $100$.
U slučaju neispravnog unosa, ispisati odgovarajuću poruku o grešci. 

\begin{miditest}
\begin{upotreba}{1}
#\naslovInt#
#\izlaz{Unesite broj elemenata niza:}\ulaz{4}#
#\izlaz{Unesite elemente niza:}\ulaz{11 2 4 9}#
#\izlaz{0}#
\end{upotreba}
\end{miditest}
\begin{miditest}
\begin{upotreba}{2}
#\naslovInt#
#\izlaz{Unesite broj elemenata niza:}\ulaz{7}#
#\izlaz{Unesite elemente niza:}\ulaz{7 2 1 14 65 2 8}#
#\izlaz{2}#
\end{upotreba}
\end{miditest}

\begin{miditest}
\begin{upotreba}{3}
#\naslovInt#
#\izlaz{Unesite broj elemenata niza:}\ulaz{5}#
#\izlaz{Unesite elemente niza:}\ulaz{25 18 29 30 14}#
#\izlaz{1}#
\end{upotreba}
\end{miditest}

\linkresenje{p.broj_manjih_od_maksimuma}
\end{Exercise}

\ifresenja
\begin{Answer}[ref=p.broj_manjih_od_maksimuma]
\includecode{resenja/3_PredstavljanjePodataka/2.1_Nizovi/broj_manjih_od_maksimuma.c}
\end{Answer}
\fi


\begin{Exercise}[label=v.nizovi_funkcije_razno] 
Elementi niza celih brojeva su podaci o nadmorskim visinama u nekom
području sveta.  Na kartografskoj mapi su indeksima označene različite
tačke, a u nizu je svakom indeksu pridružen neki ceo broj (odnosno
nadmorska visina). Napisati funkcije za rad sa ovim nizom.
\begin{enumerate}
\item Napisati funkciju koja proverava da li niz sadrži zadatu
  nardmorsku visinu $m$. Povratna vrednost funkcije je $1$ ako je
  vrednost sadržana u nizu ili $0$ ako nije.
\item Napisati funkciju koja vraća vrednost prve pozicije na kojoj se
  nalazi element koji ima nadmorsku visinu $m$ ili $-1$ ukoliko
  element nije u nizu.
\item Napisati funkciju koja vraća vrednost poslednje pozicije na
  kojoj se nalazi element koji ima nadmorsku visinu $m$ ili $-1$
  ukoliko element nije u nizu.
\end{enumerate}
Napisati i program koji testira rad napisanih funkcija za uneti broj
$m$. Maksimalan broj elemenata niza je $100$.
U slučaju neispravnog unosa, ispisati odgovarajuću poruku o grešci. 

\begin{maxitest}
\begin{upotreba}{1}
#\naslovInt#
#\izlaz{Unesite dimenziju niza:}#
#\ulaz{7}#
#\ulaz{800 1100 -200 1400 -200 1100 800}#
#\izlaz{Ucitani niz: 800 1100 -200 1400 -200 1100 800}#
#\izlaz{Unesite jedan ceo broj: }#
#\ulaz{1100}#
#\izlaz{Niz sadrzi element cija je vrednost 1100.}#
#\izlaz{Indeks njegovog prvog pojavljivanja u nizu je 1.}#
#\izlaz{Indeks njegovog poslednjeg pojavljivanja u nizu je 5.}#
\end{upotreba}
\end{maxitest}

\begin{miditest}
\begin{upotreba}{2}
#\naslovInt#
#\izlaz{Unesite dimenziju niza:}#
#\ulaz{-5}#
#\izlaz{Greska: neispravan unos.}#
\end{upotreba}
\end{miditest}
\linkresenje{v.nizovi_funkcije_razno}
\end{Exercise}

\ifresenja
\begin{Answer}[ref=v.nizovi_funkcije_razno]
\includecode{resenja/3_PredstavljanjePodataka/2.1_Nizovi/nizovi_funkcije_razno.c}
\end{Answer}
\fi


\begin{Exercise}[label=p.elementi_3_pojavljivanja] 
Marko skuplja sličice za Svetsko prvenstvo u fudbalu. U celobrojnom
nizu $a$ se nalaze brojevi onih sličica koje je već sakupio. Marko je
primetio da mu se neke sličice ponavljaju i rešio je da ih razmeni sa
drugarima. Napisati program koji od datog niza $a$ formira niz $b$
sličica koje se u nizu $a$ pojavljuju više od dva puta. Maksimalan
broj elemenata niza je $100$.
U slučaju neispravnog unosa, ispisati odgovarajuću poruku o grešci. 
 
\begin{miditest}
\begin{upotreba}{1}
#\naslovInt#
#\izlaz{Unesite broj elemenata niza:}\ulaz{8}#
#\izlaz{Unesite elemente niza a:}#
#\ulaz{4 11 4 6 8 4 6 6}#
#\izlaz{Elementi niza b: 4 6}#
\end{upotreba}
\end{miditest}
\begin{miditest}
\begin{upotreba}{2}
#\naslovInt#
#\izlaz{Unesite broj elemenata niza:}\ulaz{13}#
#\izlaz{Unesite elemente niza a:}#
#\ulaz{8 26 7 2 1 1 7 2 2 2 7 5 1}#
#\izlaz{Elementi niza b: 7 2 1}#
\end{upotreba}
\end{miditest}

\begin{miditest}
\begin{upotreba}{3}
#\naslovInt#
#\izlaz{Unesite broj elemenata niza:}\ulaz{2}#
#\izlaz{Unesite elemente niza a:}#
#\ulaz{9 5}#
#\izlaz{Elementi niza b: }#
\end{upotreba}
\end{miditest}

\linkresenje{p.elementi_3_pojavljivanja}
\end{Exercise}

\ifresenja
\begin{Answer}[ref=p.elementi_3_pojavljivanja]
\includecode{resenja/3_PredstavljanjePodataka/2.1_Nizovi/elementi_3_pojavljivanja.c}
\end{Answer}
\fi


\begin{Exercise}[label=p.pretraga_deljivih_sa_k] 
Sa standardnog ulaza se učitava dimenzija niza, elementi niza i jedan
ceo broj $k$. Napisati program koji štampa indekse elemenata koji su
deljivi sa $k$. Maksimalan broj elemenata niza je $100$.
U slučaju neispravnog unosa, ispisati odgovarajuću poruku o grešci. 

\begin{miditest}
\begin{upotreba}{1}
#\naslovInt#
#\izlaz{Unesite dimenziju niza:}\ulaz{4}#
#\izlaz{Unesite elemente niza:}\ulaz{10 14 86 20}#
#\izlaz{Unesite broj k:}\ulaz{5}#
#\izlaz{0 3}#
\end{upotreba}
\end{miditest}
\begin{miditest}
\begin{upotreba}{2}
#\naslovInt#
#\izlaz{Unesite dimenziju niza:}\ulaz{4}#
#\izlaz{Unesite elemente niza:}\ulaz{6 14 8 9}#
#\izlaz{Unesite broj k:}\ulaz{5}#
#\izlaz{U nizu nema elemenata koji su}#
#\izlaz{deljivi brojem 5.}#
\end{upotreba}
\end{miditest}

\begin{miditest}
\begin{upotreba}{3}
#\naslovInt#
#\izlaz{Unesite dimenziju niza:}\ulaz{6}#
#\izlaz{Unesite elemente niza:}\ulaz{8 9 11 -4 8 11}#
#\izlaz{Unesite broj k:}\ulaz{2}#
#\izlaz{0 3 4}#
\end{upotreba}
\end{miditest}

\linkresenje{p.pretraga_deljivih_sa_k}
\end{Exercise}

\ifresenja
\begin{Answer}[ref=p.pretraga_deljivih_sa_k]
\includecode{resenja/3_PredstavljanjePodataka/2.1_Nizovi/pretraga_deljivih_sa_k.c}
\end{Answer}
\fi


\begin{Exercise}[label=autobusi]
  Autobusi su označeni rednim brojevima (počevši od $1$) i u nizu se
  čuva vreme putovanja svakog autobusa u minutima. Međutim, zbog
  radova na putu između Požege i Užica, svi autobusi koji saobraćaju
  na tom potezu (autobusi označeni rednim brojevima od $k$ do $t$)
  saobračaju $m$ minuta duže. Uneti potrebne izmene u niz i ispisati
  elemente niza. Maksimalan broj autobusa je $200$.
U slučaju neispravnog unosa, ispisati odgovarajuću poruku o grešci. 

\begin{miditest}
\begin{upotreba}{1}
#\naslovInt#
#\izlaz{Unesite dimenziju niza:}#
#\ulaz{8}#
#\izlaz{Unesite vreme putovanja:}#
#\ulaz{24 78 13 124 56 90 205 45}#
#\izlaz{Unesite redne brojeve autobusa koji}#
#\izlaz{putuju na potezu Pozega, Uzice:}#
#\ulaz{3 6}#
#\izlaz{Unesite novo vreme:}#
#\ulaz{23}#
#\izlaz{Vreme putovanja nakon izmena:}#
#\izlaz{24 78 36 147 79 113 205 45}#
\end{upotreba}
\end{miditest}
\begin{miditest}
\begin{upotreba}{2}
#\naslovInt#
#\izlaz{Unesite dimenziju niza:}#
#\ulaz{8}#
#\izlaz{Unesite vreme putovanja:}#
#\ulaz{24 78 13 124 56 90 205 45}#
#\izlaz{Unesite redne brojeve autobusa koji}#
#\izlaz{putuju na potezu Pozega, Uzice:}#
#\ulaz{3 15}#
#\izlaz{Redni brojevi autobusa nisu}#
#\izlaz{u dozvoljenom opsegu.}#
\end{upotreba}
\end{miditest}
\linkresenje{autobusi}
\end{Exercise}

\ifresenja
\begin{Answer}[ref=autobusi]
\includecode{resenja/3_PredstavljanjePodataka/2.1_Nizovi/autobusi.c}
\end{Answer}
\fi


\begin{Exercise}[label=p.zbir_opsega_niza] 
 Napisati funkciju \kckod{int zbir(int a[], int n, int i, int j)} koja
 računa zbir elemenata niza celih brojeva $a$ dužine $n$ od pozicije
 $i$ do pozicije $j$. Napisati i program koji testira rad
 funkcije. Maksimalan broj elemenata niza je $100$.
U slučaju neispravnog unosa, ispisati odgovarajuću poruku o grešci. 
 
\begin{miditest}
\begin{upotreba}{1}
#\naslovInt#
#\izlaz{Unesite broj elemenata niza:}\ulaz{5}#
#\izlaz{Unesite elemente niza:}\ulaz{11 5 6 48 8}#
#\izlaz{Unesite vrednosti za i i j:}\ulaz{0 2}#
#\izlaz{Zbir je: 22}#
\end{upotreba}
\end{miditest}
\begin{miditest}
\begin{upotreba}{2}
#\naslovInt#
#\izlaz{Unesite broj elemenata niza:}\ulaz{3}#
#\izlaz{Unesite elemente niza:}\ulaz{-2 8 1}#
#\izlaz{Unesite vrednosti za i i j:}\ulaz{8 12}#
#\izlaz{Greska: neispravan unos.}#
\end{upotreba}
\end{miditest}

\begin{miditest}
\begin{upotreba}{3}
#\naslovInt#
#\izlaz{Unesite broj elemenata niza:}\ulaz{7}#
#\izlaz{Unesite elemente niza:}\ulaz{-2 5 9 11 6 -3 -4}#
#\izlaz{Unesite vrednosti za i i j:}\ulaz{2 5}#
#\izlaz{Zbir: 23}#
\end{upotreba}
\end{miditest}
  
\linkresenje{p.zbir_opsega_niza}
\end{Exercise}

\ifresenja
\begin{Answer}[ref=p.zbir_opsega_niza]
\includecode{resenja/3_PredstavljanjePodataka/2.1_Nizovi/zbir_opsega_niza.c}
\end{Answer}
\fi


\begin{Exercise}[label=p.kvadriranje_elemenata] 
Napisati program koji transformiše uneti niz tako što kvadrira sve
negativne elemente niza. Maksimalan broj elemenata niza je $100$.
U slučaju neispravnog unosa, ispisati odgovarajuću poruku o grešci. 

\begin{miditest}
\begin{upotreba}{1}
#\naslovInt#
#\izlaz{Unesite broj elemenata niza:}\ulaz{6}#
#\izlaz{Unesite elemente niza:}#
#\ulaz{12.34 -6 1 8 32.4 -16}#
#\izlaz{12.34 36 1 8 32.4 256}#
\end{upotreba}
\end{miditest}
\begin{miditest}
\begin{upotreba}{2}
#\naslovInt#
#\izlaz{Unesite broj elemenata niza:}\ulaz{9}#
#\izlaz{Unesite elemente niza:}#
#\ulaz{-8.25 6 17 2 -1.5 1 -7 2.65 -125.2}#
#\izlaz{68.0625 6 17 2 2.25 1 49 2.65 15675.04}#
\end{upotreba}
\end{miditest}

\begin{miditest}
\begin{upotreba}{3}
#\naslovInt#
#\izlaz{Unesite broj elemenata niza:}\ulaz{4}#
#\izlaz{Unesite elemente niza:}#
#\ulaz{9.53 5 1 4.89}#
#\izlaz{9.53 5 1 4.89}#
\end{upotreba}
\end{miditest}
\linkresenje{p.kvadriranje_elemenata}
\end{Exercise}

\ifresenja
\begin{Answer}[ref=p.kvadriranje_elemenata]
\includecode{resenja/3_PredstavljanjePodataka/2.1_Nizovi/kvadriranje_elemenata.c}
\end{Answer}
\fi


\begin{Exercise}[label=p.kvadriranje_parnih] 
 Napisati funkciju \kckod{void kvadriranje(float a[], int n)} koja
 kvadrira elemente realnog niza $a$ dužine $n$ koji se nalaze na
 parnim pozicijama. Napisati program koji tranformiše na ovaj način
 uneti niz. Maksimalan broj elemenata niza je $100$.
 U slučaju neispravnog unosa, ispisati odgovarajuću poruku o grešci. 

\begin{miditest}
\begin{upotreba}{1}
#\naslovInt#
#\izlaz{Unesite broj elemenata niza:}\ulaz{8}#
#\izlaz{Unesite elemente niza:}#
#\ulaz{2.34 1 -12.7 5.2 -8 -6.2 7 14.2}#
#\izlaz{5.4756 1 161.29 5.2 64 -6.2 49 14.2}#
\end{upotreba}
\end{miditest}
\begin{miditest}
\begin{upotreba}{2}
#\naslovInt#
#\izlaz{Unesite broj elemenata niza:}\ulaz{3}#
#\izlaz{Unesite elemente niza:}#
#\ulaz{-6 -8.14 -15}#
#\izlaz{36 -8.14 225}#
\end{upotreba}
\end{miditest}

\begin{miditest}
\begin{upotreba}{3}
#\naslovInt#
#\izlaz{Unesite broj elemenata niza:}\ulaz{1}#
#\izlaz{Unesite elemente niza:}#
#\ulaz{-35.11}#
#\izlaz{1232.71}#
\end{upotreba}
\end{miditest}
\linkresenje{p.kvadriranje_parnih}
\end{Exercise}

\ifresenja
\begin{Answer}[ref=p.kvadriranje_parnih]
\includecode{resenja/3_PredstavljanjePodataka/2.1_Nizovi/kvadriranje_parnih.c}
\end{Answer}
\fi


\begin{Exercise}[label=p.zbir_k_pozitivnih] 
 Napisati funkciju \kckod{float zbir\_pozitivnih(float a[], int n, int
   k)} koja izračunava zbir prvih $k$ pozitivnih elemenata realnog
 niza $a$ dužine $n$. Napisati i program koji testira rad
 funkcije. Maksimalan broj elemenata niza je $100$.
 U slučaju neispravnog unosa, ispisati odgovarajuću poruku o grešci. 

\begin{miditest}
\begin{upotreba}{1}
#\naslovInt#
#\izlaz{Unesite broj elemenata niza:}\ulaz{8}#
#\izlaz{Unesite elemente niza:}#
#\ulaz{2.34 1 -12.7 5.2 -8 -6.2 7 14.2}#
#\izlaz{Unesite vrednost za k:}\ulaz{3}#
#\izlaz{Zbir je: 8.54}#
\end{upotreba}
\end{miditest}
\begin{miditest}
\begin{upotreba}{2}
#\naslovInt#
#\izlaz{Unesite broj elemenata niza:}\ulaz{3}#
#\izlaz{Unesite elemente niza:}#
#\ulaz{-6.598 -8.14 -15}#
#\izlaz{Unesite vrednost za k:}\ulaz{4}#
#\izlaz{Zbir je: 0.00}#
\end{upotreba}
\end{miditest}

\begin{miditest}
\begin{upotreba}{3}
#\naslovInt#
#\izlaz{Unesite broj elemenata niza:}\ulaz{7}#
#\izlaz{Unesite elemente niza:}#
#\ulaz{-35.11 5.29 -1.98 12.1 12.2 -3.33 -4.17}#
#\izlaz{Unesite vrednost za k:}\ulaz{15}# 
#\izlaz{Zbir: 29.59}#
\end{upotreba}
\end{miditest}

\linkresenje{p.zbir_k_pozitivnih}
\end{Exercise}

\ifresenja
\begin{Answer}[ref=p.zbir_k_pozitivnih]
\includecode{resenja/3_PredstavljanjePodataka/2.1_Nizovi/zbir_k_pozitivnih.c}
\end{Answer}
\fi


\begin{Exercise}[label=aritm_sredina_deljivih_sa_tri] 
Napisati funkciju \kckod{int blizu\_3(int a[],int n)} koja pronalazi i
vraća indeks elementa niza koji je po vrednosti najbliži aritmetičkoj
sredini onih elemenata niza koji su deljivi brojem tri. Napisati
program koji testira rad funkcije. Maksimalan broj elemenata niza je
$200$.
U slučaju neispravnog unosa, ispisati odgovarajuću poruku o grešci. 

\begin{miditest}
\begin{upotreba}{1}
#\naslovInt#
#\izlaz{Unesite broj elemenata niza:}\ulaz{5}#
#\izlaz{Unesite elemente niza:}#
#\ulaz{1 2 3 4 5}#
#\izlaz{3}#
\end{upotreba}
\end{miditest}
\begin{miditest}
\begin{upotreba}{2}
#\naslovInt#
#\izlaz{Unesite broj elemenata niza:}\ulaz{5}#
#\izlaz{Unesite elemente niza:}#
#\ulaz{3 6 2 4 7}#
#\izlaz{4}#
\end{upotreba}
\end{miditest}
\linkresenje{aritm_sredina_deljivih_sa_tri}
\end{Exercise}

\ifresenja
\begin{Answer}[ref=aritm_sredina_deljivih_sa_tri]
\includecode{resenja/3_PredstavljanjePodataka/2.1_Nizovi/aritm_sredina_deljivih_sa_tri.c}
\end{Answer}
\fi


%-----------------------------------------------------------------------------------
% Prebrojavanja, karakterski nizovi
%-----------------------------------------------------------------------------------

\begin{Exercise}[label=v.prebrojavanje_cifara] 
Napisati program koji za učitani ceo broj, ispisuje broj pojavljivanja
svake od cifara u zapisu tog broja. \uputstvo{Za evidenciju broja
  pojavljivanja svake cifre pojedinačno, koristiti niz.}

\begin{maxitest}
\begin{upotreba}{1}
#\naslovInt#
#\izlaz{Unesite ceo broj:}#
#\ulaz{2355623}#
#\izlaz{U zapisu broja 2355623, cifra 2 se pojaviljuje 2 puta}#
#\izlaz{U zapisu broja 2355623, cifra 3 se pojaviljuje 2 puta}#
#\izlaz{U zapisu broja 2355623, cifra 5 se pojaviljuje 2 puta}#
#\izlaz{U zapisu broja 2355623, cifra 6 se pojaviljuje 1 puta}#
\end{upotreba}
\end{maxitest}

\begin{maxitest}
\begin{upotreba}{2}
#\naslovInt#
#\izlaz{Unesite ceo broj:}#
#\ulaz{-39902}#
#\izlaz{U zapisu broja -39902, cifra 0 se pojaviljuje 1 puta}#
#\izlaz{U zapisu broja -39902, cifra 2 se pojaviljuje 1 puta}#
#\izlaz{U zapisu broja -39902, cifra 3 se pojaviljuje 1 puta}#
#\izlaz{U zapisu broja -39902, cifra 9 se pojaviljuje 2 puta}#
\end{upotreba}
\end{maxitest}
\linkresenje{v.prebrojavanje_cifara}
\end{Exercise}

\ifresenja
\begin{Answer}[ref=v.prebrojavanje_cifara]
\includecode{resenja/3_PredstavljanjePodataka/2.1_Nizovi/prebrojavanje_cifara.c}
\end{Answer}
\fi


\begin{Exercise}[label=p.broj_cifara] 
 Napisati funkciju \kckod{int cifre(char s[], int n)} koja izračunava
 broj cifara u nizu karaktera $a$ dužine $n$. Napisati program koji za
 karaktere koji se unose u zasebnim redovima ispisuje broj unetih
 cifara. Maksimalan broj elemenata niza je $100$. 
 U slučaju neispravnog unosa, ispisati odgovarajuću poruku o grešci. 

\begin{miditest}
\begin{upotreba}{1}
#\naslovInt#
#\izlaz{Unesite broj elemenata niza:}\ulaz{5}#
#\izlaz{Unesite elemente niza:}#
#\ulaz{4}#
#\ulaz{+}#
#\ulaz{A}#
#\ulaz{u}#
#\ulaz{8}#
#\izlaz{Broj cifara je: 2}#
\end{upotreba}
\end{miditest}
\begin{miditest}
\begin{upotreba}{2}
#\naslovInt#
#\izlaz{Unesite broj elemenata niza:}\ulaz{7}#
#\izlaz{Unesite elemente niza:}#
#\ulaz{J}#
#\ulaz{M}#
#\ulaz{a}#
#\ulaz{5}#
#\ulaz{5}#
#\ulaz{-}#
#\ulaz{2}#
#\izlaz{Broj cifara je: 3}#
\end{upotreba}
\end{miditest}

\begin{miditest}
\begin{upotreba}{3}
#\naslovInt#
#\izlaz{Unesite broj elemenata niza:}\ulaz{3}#
#\izlaz{Unesite elemente niza:}#
#\ulaz{e}#
#\ulaz{k}#
#\ulaz{F}#
#\izlaz{Broj cifara je: 0}#
\end{upotreba}
\end{miditest}

\linkresenje{p.broj_cifara}
\end{Exercise}

\ifresenja
\begin{Answer}[ref=p.broj_cifara]
\includecode{resenja/3_PredstavljanjePodataka/2.1_Nizovi/broj_cifara.c}
\end{Answer}
\fi


\begin{Exercise}[label=v.prebrojavanje] 
Napisati program koji učitava karaktere sa standardnog ulaza sve do
kraja ulaza i izračunava koliko se puta u unetom tekstu pojavila svaka
od cifara, svako malo slovo i svako veliko slovo. Ispisati broj
pojavljivanja samo za karaktere koji su se u unetom tekstu pojavili
barem jednom. \uputstvo{Za evidenciju broja pojavljivanja cifara,
  malih i velih slova korisiti pojedinačne nizove.}

\begin{miditest}
\begin{upotreba}{1}
#\naslovInt#
#\ulaz{123 abcabcabc 123}#
#\izlaz{Karakter 1 se pojavljuje 2 puta}#
#\izlaz{Karakter 2 se pojavljuje 2 puta}#
#\izlaz{Karakter 3 se pojavljuje 2 puta}#
#\izlaz{Karakter a se pojavljuje 3 puta}#
#\izlaz{Karakter b se pojavljuje 3 puta}#
#\izlaz{Karakter c se pojavljuje 3 puta}#
\end{upotreba}
\end{miditest}
\begin{miditest}
\begin{upotreba}{2}
#\naslovInt#
#\ulaz{Programiranje 1 je zanimljivo!!}#
#\izlaz{Karakter 1 se pojavljuje 1 puta}#
#\izlaz{Karakter a se pojavljuje 3 puta}#
#\izlaz{Karakter e se pojavljuje 2 puta}#
#\izlaz{Karakter g se pojavljuje 1 puta}#
#\izlaz{Karakter i se pojavljuje 3 puta}#
#\izlaz{Karakter j se pojavljuje 3 puta}#
#\izlaz{Karakter l se pojavljuje 1 puta}#
#\izlaz{Karakter m se pojavljuje 2 puta}#
#\izlaz{Karakter n se pojavljuje 2 puta}#
#\izlaz{Karakter o se pojavljuje 2 puta}#
#\izlaz{Karakter r se pojavljuje 3 puta}#
#\izlaz{Karakter v se pojavljuje 1 puta}#
#\izlaz{Karakter z se pojavljuje 1 puta}#
#\izlaz{Karakter P se pojavljuje 1 puta}#
\end{upotreba}
\end{miditest}

\linkresenje{v.prebrojavanje}
\end{Exercise}

\ifresenja
\begin{Answer}[ref=v.prebrojavanje]
\includecode{resenja/3_PredstavljanjePodataka/2.1_Nizovi/prebrojavanje.c}
\end{Answer}
\fi


\begin{Exercise}[label=brojanje_slova] 
Sa standardnog ulaza se unosi jedna linija teksta. Napisati program
koji izračunava i ispisuje koliko puta se pojavilo svako od slova
engleskog alfabeta u unetom tekstu. Ne praviti razliku između malih i
velikih slova.

\begin{maxitest}
\begin{upotreba}{1}
#\naslovInt#
#\ulaz{Tasi, tasi, TaNaNa i SVILENA marama.....}#
#\izlaz{ a:9  b:0  c:0  d:0  e:1  f:0  g:0  h:0  i:4  j:0  k:0  l:1  m:2}#
#\izlaz{ n:3  o:0  p:0  q:0  r:1  s:3  t:3  u:0  v:1  w:0  x:0  y:0  z:0}#
\end{upotreba}
\end{maxitest}

\begin{maxitest}
\begin{upotreba}{2}
#\naslovInt#
#\ulaz{Mihailo Petrovic Alas (6 Maj 1868 - 8 Jun 1943)}#
#\izlaz{ a:4  b:0  c:1  d:0  e:1  f:0  g:0  h:1  i:3  j:2  k:0  l:2  m:2}#
#\izlaz{ n:1  o:2  p:1  q:0  r:1  s:1  t:1  u:1  v:1  w:0  x:0  y:0  z:0}#
\end{upotreba}
\end{maxitest}

\begin{maxitest}
\begin{upotreba}{3}
#\naslovInt#
#\ulaz{Alan Matison Tjuring (London, 23. jun 1912 - Cesir, 7. jun 1954) }#
#\izlaz{ a:3  b:0  c:1  d:1  e:1  f:0  g:1  h:0  i:3  j:3  k:0  l:2  m:1}#
#\izlaz{ n:7  o:3  p:0  q:0  r:2  s:2  t:2  u:3  v:0  w:0  x:0  y:0  z:0}#
\end{upotreba}
\end{maxitest}
\linkresenje{brojanje_slova}
\end{Exercise}

\ifresenja
\begin{Answer}[ref=brojanje_slova]
\includecode{resenja/3_PredstavljanjePodataka/2.1_Nizovi/brojanje_slova.c}
\end{Answer}
\fi


\begin{Exercise}[label=p.niz_karaktera_obrnuto] 
 Napisati program koji učitane karaktere (najviše njih $100$,
 učitavaju se sve do pojave karaktera \textit{*}) ispisuje u redosledu
 suprotnom od redosleda čitanja.
U slučaju neispravnog unosa, ispisati odgovarajuću poruku o grešci. 
 
\begin{miditest}
\begin{upotreba}{1}
#\naslovInt#
#\izlaz{Unesite karakter:}\ulaz{a}#
#\izlaz{Unesite karakter:}\ulaz{8}#
#\izlaz{Unesite karakter:}\ulaz{5}#
#\izlaz{Unesite karakter:}\ulaz{Y}#
#\izlaz{Unesite karakter:}\ulaz{I}#
#\izlaz{Unesite karakter:}\ulaz{o}#
#\izlaz{Unesite karakter:}\ulaz{?}#
#\izlaz{Unesite karakter:}\ulaz{*}#
#\izlaz{? o I Y 5 8 a}#
\end{upotreba}
\end{miditest}
\begin{miditest}
\begin{upotreba}{2}
#\naslovInt#
#\izlaz{Unesite karakter:}\ulaz{g}#
#\izlaz{Unesite karakter:}\ulaz{g}#
#\izlaz{Unesite karakter:}\ulaz{2}#
#\izlaz{Unesite karakter:}\ulaz{2}#
#\izlaz{Unesite karakter:}\ulaz{)}#
#\izlaz{Unesite karakter:}\ulaz{)}#
#\izlaz{Unesite karakter:}\ulaz{*}#
#\izlaz{) ) 2 2 g g}#
\end{upotreba}
\end{miditest}

\begin{miditest}
\begin{upotreba}{3}
#\naslovInt#
#\izlaz{Unesite karakter:}\ulaz{U}#
#\izlaz{Unesite karakter:}\ulaz{4}#
#\izlaz{Unesite karakter:}\ulaz{a}#
#\izlaz{Unesite karakter:}\ulaz{u}#
#\izlaz{Unesite karakter:}\ulaz{*}#
#\izlaz{u a 4 U}#
\end{upotreba}
\end{miditest}

\linkresenje{p.niz_karaktera_obrnuto}
\end{Exercise}

\ifresenja
\begin{Answer}[ref=p.niz_karaktera_obrnuto]
\includecode{resenja/3_PredstavljanjePodataka/2.1_Nizovi/niz_karaktera_obrnuto.c}
\end{Answer}
\fi


\begin{Exercise}[label=palindrom]
Palindrom je tekst koji se isto čita i sa leve i sa desne
strane. Napisati funkciju koja proverava da li elementi niza karaktera
čine palindrom (zanemariti velika/mala slova). Maksimalan broj
elemenata niza je $200$.
U slučaju neispravnog unosa, ispisati odgovarajuću poruku o grešci. 

\begin{miditest}
\begin{upotreba}{1}
#\naslovInt#
#\izlaz{Unesite dimenziju niza:}#
#\ulaz{15}#
#\izlaz{Unesite elemente niza:}#
#\ulaz{AnaVoliMilovana}#
#\izlaz{Jeste palindrom.}#  
\end{upotreba}
\end{miditest}
\begin{miditest}
\begin{upotreba}{2}
#\naslovInt#
#\izlaz{Unesite dimenziju niza:}#
#\ulaz{26}#
#\izlaz{Unesite elemente niza:}#
#\ulaz{Zanimljivo je programirati!}#
#\izlaz{Nije palindrom.}#  
\end{upotreba}
\end{miditest}
\linkresenje{palindrom}
\end{Exercise}

\ifresenja
\begin{Answer}[ref=palindrom]
\includecode{resenja/3_PredstavljanjePodataka/2.1_Nizovi/palindrom.c}
\end{Answer}
\fi

%-----------------------------------------------------------------------------------
% Pravljenje novih nizova, transformacija nizova
%-----------------------------------------------------------------------------------

\begin{Exercise}[label=p.razmena_min_max] 
  Napisati program koji učitava dimenziju i elemente niza i štampa niz
  u kojem su najveći i najmanji element niza razmenili mesta. Ukoliko
  se najmanji ili najveći element više puta pojavljuju u nizu, uzeti u
  obzir njihova prva pojavljivanja. Maksimalan broj elemenata niza je
  $100$. 
  U slučaju neispravnog unosa, ispisati odgovarajuću poruku o grešci. 

\begin{miditest}
\begin{upotreba}{1}
#\naslovInt#
#\izlaz{Unesite dimenziju niza:}\ulaz{5}#
#\izlaz{Unesite elemente niza:}\ulaz{8 -2 11 19 4}#
#\izlaz{8 19 11 -2 4}#
\end{upotreba}
\end{miditest}
\begin{miditest}
\begin{upotreba}{2}
#\naslovInt#
#\izlaz{Unesite dimenziju niza:}\ulaz{10}#
#\izlaz{Unesite elemente niza:}#
#\ulaz{46 -2 51 8 -5 66 2 8 3 14}#
#\izlaz{46 -2 51 8 66 -5 2 8 3 14}#
\end{upotreba}
\end{miditest}

\begin{miditest}
\begin{upotreba}{3}
#\naslovInt#
#\izlaz{Unesite dimenziju niza:}\ulaz{145}#
#\izlaz{Greska: neispravan unos.}#
\end{upotreba}
\end{miditest}
\linkresenje{p.razmena_min_max}
\end{Exercise}

\ifresenja
\begin{Answer}[ref=p.razmena_min_max]
\includecode{resenja/3_PredstavljanjePodataka/2.1_Nizovi/razmena_min_max.c}
\end{Answer}
\fi


\begin{Exercise}[label=p.unija_presek_razlika] 
Korišćenjem nizova moguće je predstaviti skupove podataka. Napisati
program koji demonstrira osnovne operacije nad skupovima --- unija,
presek i razlika. Pomoću dva niza predstaviti dva skupa celih
brojeva. Maksimalan broj elemenata niza je $500$.
U slučaju neispravnog unosa, ispisati odgovarajuću poruku o grešci. 
  
\begin{miditest}
\begin{upotreba}{1}
#\naslovInt#
#\izlaz{Unesite broj elemenata niza a:}\ulaz{5}#
#\izlaz{Unesite elemente niza a:}\ulaz{2 8 1 5 2}#
#\izlaz{Unesite broj elemenata niza b:}\ulaz{3}#
#\izlaz{Unesite elemente niza b:}\ulaz{5 7 8}#
#\izlaz{Unija: 2 8 1 5 2 5 7 8}#
#\izlaz{Presek: 5}#
#\izlaz{Razlika: 2 1 2}#
\end{upotreba}
\end{miditest}
\begin{miditest}
\begin{upotreba}{2}
#\naslovInt#
#\izlaz{Unesite broj elemenata niza a:}\ulaz{3}#
#\izlaz{Unesite elemente niza a:}\ulaz{11 4 4}#
#\izlaz{Unesite broj elemenata niza b:}\ulaz{2}#
#\izlaz{Unesite elemente niza b:}\ulaz{18 9}#
#\izlaz{Unija: 11 4 4 18 9}#
#\izlaz{Presek: }#
#\izlaz{Razlika: 11 4 4}#
\end{upotreba}
\end{miditest}

\begin{miditest}
\begin{upotreba}{3}
#\naslovInt#
#\izlaz{Unesite broj elemenata niza a:}\ulaz{6}#
#\izlaz{Unesite elemente niza a:}\ulaz{12 7 9 12 5 1}#
#\izlaz{Unesite broj elemenata niza b:}\ulaz{4}#
#\izlaz{Unesite elemente niza b:}\ulaz{1 12 22 12}#
#\izlaz{Unija: 12 7 9 12 5 1 1 12 22 12}#
#\izlaz{Presek: 12 12 1}#
#\izlaz{Razlika: 7 9 5}#
\end{upotreba}
\end{miditest}

\linkresenje{p.unija_presek_razlika}
\end{Exercise}

\ifresenja
\begin{Answer}[ref=p.unija_presek_razlika]
\includecode{resenja/3_PredstavljanjePodataka/2.1_Nizovi/unija_presek_razlika.c}
\end{Answer}
\fi


\begin{Exercise}[label=v.ukrstanje_nizova] 
Napisati program koji za dva učitana niza $a$ i $b$ dimenzije $n$
formira i na izlaz ispisuje niz $c$ koji se dobija naizmeničnim
raspoređivanjem elemenata nizova $a$ i $b$, tj.~$c = [a_0, b_0, a_1,
  b_1, \ldots, a_{n-1}, b_{n-1}]$. Maksimalan broj elemenata niza je
$100$.
U slučaju neispravnog unosa, ispisati odgovarajuću poruku o grešci. 

\begin{miditest}
\begin{upotreba}{1}
#\naslovInt#
#\izlaz{Unesite dimenziju nizova:}#
#\ulaz{5}#
#\izlaz{Unesite elemente niza a:}#
#\ulaz{2 -5 11 4 8}#
#\izlaz{Unesite elemente niza b:}#
#\ulaz{3 3 9 -1 17}#
#\izlaz{Rezultujuci niz:}#
#\izlaz{2 3 -5 3 11 9 4 -1 8 17}#
\end{upotreba}
\end{miditest}
\begin{miditest}
\begin{upotreba}{2}
#\naslovInt#
#\izlaz{Unesite dimenziju nizova:}#
#\ulaz{105}#
#\izlaz{Greska: neispravan unos.}#
\end{upotreba}
\end{miditest}
\linkresenje{v.ukrstanje_nizova}
\end{Exercise}

\ifresenja
\begin{Answer}[ref=v.ukrstanje_nizova]
\includecode{resenja/3_PredstavljanjePodataka/2.1_Nizovi/ukrstanje_nizova.c}
\end{Answer}
\fi


\begin{Exercise}[label=p.nizovi_spajanje] 
Napisati program koji za dva učitana niza $a$ i $b$ dimenzije $n$
formira i na izlaz ispisuje niz $c$ čija prva polovina odgovara
elemetima niza $b$, a druga polovina elementima niza $a$, tj.~$c =
[b_0, b_1, \ldots, b_{n-1}, a_0, a_1, \ldots, a_{n-1}]$.  Maksimalan
broj elemenata niza je $100$.
U slučaju neispravnog unosa, ispisati odgovarajuću poruku o grešci. 

\begin{miditest}
\begin{upotreba}{1}
#\naslovInt#
#\izlaz{Unesite broj n:}\ulaz{3}#
#\izlaz{Unesite elemente niza a:}\ulaz{4 -8 32}#
#\izlaz{Unesite elemente niza b:}\ulaz{5 2 11}#
#\izlaz{5 2 11 4 -8 32}#
\end{upotreba}
\end{miditest}
\begin{miditest}
\begin{upotreba}{2}
#\naslovInt#
#\izlaz{Unesite broj n:}\ulaz{4}#
#\izlaz{Unesite elemente niza a:}\ulaz{1 0 -1 0}#
#\izlaz{Unesite elemente niza b:}\ulaz{5 5 5 3}#
#\izlaz{5 5 5 3 1 0 -1 0}#
\end{upotreba}
\end{miditest}

\begin{miditest}
\begin{upotreba}{3}
#\naslovInt#
#\izlaz{Unesite dimenziju niza:}\ulaz{145}#
#\izlaz{Greska: neispravan unos.}#
\end{upotreba}
\end{miditest}
\linkresenje{p.nizovi_spajanje}
\end{Exercise}

\ifresenja
\begin{Answer}[ref=p.nizovi_spajanje]
\includecode{resenja/3_PredstavljanjePodataka/2.1_Nizovi/nizovi_spajanje.c}
\end{Answer}
\fi


% mislim da treba zadrzati zadatak. jeste ovo p2, ali sta fali, malo
% vezbaju napredne koncepte, ovo je zadatak za bolje studente, ne mora
% da se da na ispitu. moze da se stavi zvezdica.
\begin{Exercise}[difficulty=1, label=p.nizovi_spajanje_sortiranih] 
 Sa standardnog ulaza se učitava ceo broj $n$ manji od $100$ i
 elementi dvaju celobrojnih, sortiranih neopadajuće nizova $a$ i $b$
 dimenzije $n$. Napisati program koji formira i ispisuje niz $c$ koji
 se dobija spajanjem nizova $a$ i $b$ u treći, takođe sortiran
 neopadajuće, niz.
U slučaju neispravnog unosa, ispisati odgovarajuću poruku o grešci. 

\begin{miditest}
\begin{upotreba}{1}
#\naslovInt#
#\izlaz{Uneti broj elemenata niza:}\ulaz{5}#
#\izlaz{Uneti elemente sortiranog niza:}#
#\ulaz{2 11 28 40 63}#
#\izlaz{Uneti elemente sortiranog niza:}#
#\ulaz{-19 -5 5 11 52}#
#\izlaz{Niz c:}#
#\izlaz{-19 -5 2 5 11 11 28 40 52 63}#
\end{upotreba}
\end{miditest}
\begin{miditest}
\begin{upotreba}{2}
#\naslovInt#
#\izlaz{Uneti broj elemenata niza:}\ulaz{3}#
#\izlaz{Uneti elemente sortiranog niza:}#
#\ulaz{-2 4 8}#
#\izlaz{Uneti elemente sortiranog niza:}#
#\ulaz{6 15 19}#
#\izlaz{Niz c:}#
#\izlaz{-2 4 6 8 15 19}#
\end{upotreba}
\end{miditest}

\begin{miditest}
\begin{upotreba}{3}
#\naslovInt#
#\izlaz{Uneti broj elemenata niza:}\ulaz{145}#
#\izlaz{Greska: neispravan unos.}#
\end{upotreba}
\end{miditest}
\linkresenje{p.nizovi_spajanje_sortiranih}
\end{Exercise}

\ifresenja
\begin{Answer}[ref=p.nizovi_spajanje_sortiranih]
\includecode{resenja/3_PredstavljanjePodataka/2.1_Nizovi/nizovi_spajanje_sortiranih.c}
\end{Answer}
\fi


\begin{Exercise}[label=vp.bez_resenja_7] 
Napisati program koji sa standardnog ulaza učitava $10$ celih brojeva
i razdvaja ih na parne i neparne tako što parne brojeve upisuje na
početak niza, a neparne brojeve na kraj niza. Ispisati niz dobijen na
ovaj način. \napomena{Nije dozvoljeno koristiti pomoćne nizove.}

\begin{miditest}
\begin{upotreba}{1}
#\naslovInt#
#\izlaz{Unesite 10 brojeva:}#
#\ulaz{-2 8 11 53 59 20 17 -8 3 14}#
#\izlaz{Rezultujuci niz:}#
#\izlaz{14 142 -6 -278 28 34 33 -69 -9 9}#
\end{upotreba}
\end{miditest}
\begin{miditest}
\begin{upotreba}{2}
#\naslovInt#
#\izlaz{Unesite 10 brojeva:}#
#\ulaz{9 142 -9 -278 -69 33 34 28 -6 14}#
#\izlaz{Rezultujuci niz:}#
#\izlaz{-2 8 14 -8 20 59 17 53 3 11}#
\end{upotreba}
\end{miditest}
\linkresenje{vp.bez_resenja_7}
\end{Exercise}

\ifresenja
\begin{Answer}[ref=vp.bez_resenja_7]
\includecode{resenja/3_PredstavljanjePodataka/2.1_Nizovi/vp.bez_resenja_7.c}
\end{Answer}
\fi


\begin{Exercise}[label=v.nizovi_funkcije_pomeranja]
Napisati funkcije za rad sa nizovima celih brojeva. 
\begin{enumerate}
\item Napisati funkciju koja obrće elemente niza.     
\item Napisati funkciju koja rotira niz ciklično za jedno mesto u levo.
\item Napisati funkciju koja rotira niz ciklično za $k$ mesta u levo.
\end{enumerate}
Napisati i program koji testira rad napisanih funkcija za uneti broj
$m$. Maksimalan broj elemenata niza je $100$.
U slučaju neispravnog unosa, ispisati odgovarajuću poruku o grešci. 

\begin{miditest}
\begin{upotreba}{1}
#\naslovInt#
#\izlaz{Unesite dimenziju niza:}#
#\ulaz{6}#
#\ulaz{7 -3 11 783 26 -19}#
#\izlaz{Elementi niza nakon obrtanja:}#
#\izlaz{-17	28 785	13	-1	9}#
#\izlaz{Elementi niza nakon rotiranja za 1 mesto ulevo:}#
#\izlaz{28 785 13 -1 9 -17}#
#\izlaz{Unesite jedan pozitivan ceo broj:}#
#\ulaz{3}#
#\izlaz{Elementi niza nakon rotiranja za 3 mesto ulevo:}#
#\izlaz{-1 9 -17 28	785	13}#
\end{upotreba}
\end{miditest}
\begin{miditest}
\begin{upotreba}{2}
#\naslovInt#
#\izlaz{Unesite dimenziju niza:}#
#\ulaz{252}#
#\izlaz{Greska: neispravan unos.}#
\end{upotreba}
\end{miditest}
\linkresenje{v.nizovi_funkcije_pomeranja}
\end{Exercise}

\ifresenja
\begin{Answer}[ref=v.nizovi_funkcije_pomeranja]
\includecode{resenja/3_PredstavljanjePodataka/2.1_Nizovi/nizovi_funkcije_pomeranja.c}
\end{Answer}
\fi

%-----------------------------------------------------------------------------------
% Brisanja u nizu
%-----------------------------------------------------------------------------------

\begin{Exercise}[label=izbacivanje_ubacivanje_u_niz] 
Prilikom ulaska u banku klijent dobija redni broj, a u nizu se čuva
redosled opsluživanja klijenata. Tako, prvi klijent u nizu će biti
prvi uslužen, a klijent koji je poslednji dosao se nalazi na kraju
niza. Redni brojevi se izdaju počevši od $1$ svakog radnog dana, ali
se niz za redosled stalno menja. Dodatno, postoje specijalni klijenti
(npr. oni koji plaćaju platni promet ili oni koji podižu stambeni
kredit) koji mogu dobiti i negativan redni broj da bi se razlikovali
od uobičajenih usluga koje banka omogućava. Pomozite radniku
obezbeđenja da lako prati redosled opsluživanja klijenata.
\begin{enumerate}
\item Napisati funkciju koja ubacuje datog klijenta sa rednim brojem
  $x$ na kraj niza.
\item Napisati funkciju koja ubacuje datog klijenta sa rednim brojem
  $x$ na početak niza (lica sa posebnim potrebama, trudnice, stara
  lica i ostale ugrožene kategorije).
\item Napisati funkciju koja ubacuje datog klijenta sa rednim brojem
  $x$ na datu poziciju $k$ (manje prioritetna lica, recimo službena
  lica ili roditelji sa decom, poziciju $k$ bira radnik obezbeđenja).
\item Napisati funkciju koja izbacuje prvi element niza (usluženi
  klijent).
\item Napisati funkciju koja izbacuje poslednji element niza (klijent
  je odustao jer je shvatio da ima mnogo klijenata ispred njega).
\item Napisati funkciju koja izbacuje element sa date pozicije $k$
  (klijent je odustao jer je dugo čekao).
\end{enumerate}
Napisati program koji testira rad funkcija. Maksimalan broj klijenata
u jednom danu je $2000$.
U slučaju neispravnog unosa, ispisati odgovarajuću poruku o grešci. 

\begin{maxitest}
\begin{upotreba}{1}
#\naslovInt#
#\izlaz{Unesite trenutni broj klijenata:}#
#\ulaz{8}#
#\izlaz{Unesite niz sa rednim brojevima klijenata:}#  
#\ulaz{2 5 -2 16 33 19 8 11}#
#\izlaz{Unesite klijenta kojeg treba ubaciti u niz: }#
#\ulaz{35}#
#\izlaz{Niz nakon ubacivanja klijenta: 2 5 -2 16 33 19 8 11 35}#
#\izlaz{Unesite prioritetnog klijenta kojeg treba ubaciti u niz: }#
#\ulaz{36}#
#\izlaz{Niz nakon ubacivanja klijenta: 36 2 5 -2 16 33 19 8 11 35}#
#\izlaz{Unesite prioritetnog klijenta kojeg treba ubaciti u niz i njegovu poziciju: }#
#\ulaz{-6 2}#
#\izlaz{Niz nakon ubacivanja klijenta: 36 2 -6 5 -2 16 33 19 8 11 35}#  
#\izlaz{Niz nakon odlaska klijenta: 2 -6 5 -2 16 33 19 8 11 35}#  
#\izlaz{Niz nakon odlaska poslednjeg klijenta: 2 -6 5 -2 16 33 19 8 11}#
#\izlaz{Unesite redni broj klijenta koji je napustio red: }#
#\ulaz{-2}#
#\izlaz{Niz nakon odlaska klijenta: 2 -6 5 16 33 19 8 11}#
\end{upotreba}
\end{maxitest}
\linkresenje{izbacivanje_ubacivanje_u_niz}
\end{Exercise}

\ifresenja
\begin{Answer}[ref=izbacivanje_ubacivanje_u_niz]
\includecode{resenja/3_PredstavljanjePodataka/2.1_Nizovi/izbacivanje_ubacivanje_u_niz.c}
\end{Answer}
\fi


\begin{Exercise}[label=p.izbacivanje_elemenata] 
Napisati program koji za učitani niz formira i ispisuje niz koji se
dobija izbacivanjem svih neparnih elemenata niza. Zadatak rešiti na
dva načina: korišćenjem pomoćnog niza i transformacijom polaznog
niza. Maksimalan broj elemenata niza je $100$.
U slučaju neispravnog unosa, ispisati odgovarajuću poruku o grešci. 
 
\begin{miditest}
\begin{upotreba}{1}
#\naslovInt#
#\izlaz{Unesite broj elemenata niza:}\ulaz{4}#
#\izlaz{Unesite elemente niza:}\ulaz{8 9 15 12}#
#\izlaz{8 12}#
\end{upotreba}
\end{miditest}
\begin{miditest}
\begin{upotreba}{2}
#\naslovInt#
#\izlaz{Unesite broj elemenata niza:}\ulaz{6}#
#\izlaz{Unesite elemente niza:}\ulaz{21 5 3 22 19 188}#
#\izlaz{22 188}#
\end{upotreba}
\end{miditest}

\begin{miditest}
\begin{upotreba}{3}
#\naslovInt#
#\izlaz{Unesite broj elemenata niza:}\ulaz{4}#
#\izlaz{Unesite elemente niza:}\ulaz{133 129 121 101}#
#\izlaz{}#
\end{upotreba}
\end{miditest}

\begin{maxitest}
\begin{upotreba}{4}
#\naslovInt#
#\izlaz{Unesite broj elemenata niza:}\ulaz{8}#
#\izlaz{Unesite elemente niza:}\ulaz{15 -22 -23 13 18 46 14 -31}#
#\izlaz{-22 18 46 14}#
\end{upotreba}
\end{maxitest}
\linkresenje{p.izbacivanje_elemenata}
\end{Exercise}

\ifresenja
\begin{Answer}[ref=p.izbacivanje_elemenata]
\includecode{resenja/3_PredstavljanjePodataka/2.1_Nizovi/izbacivanje_elemenata.c}
\end{Answer}
\fi


\begin{Exercise}[label=v.brisanje_elemenata] 
Napisati program koji učitava dimenziju $n$ celobrojnog niza $a$ i
njegove elemente, i iz niza $a$ izbacuje sve elemente koji nisu
deljivi svojom poslednjom cifrom. Izuzetak su elementi čija je poslednja
cifra nula i koje zbog toga treba zadržati. Program treba da ispiše izmenjeni niz
na standardni izlaz. Maksimalan broj elemenata niza je $100$.
U slučaju neispravnog unosa, ispisati odgovarajuću poruku o grešci. 

\begin{miditest}
\begin{upotreba}{1}
#\naslovInt#
#\izlaz{Unesite dimenziju niza:}#
#\ulaz{9}#
#\izlaz{Unesite elemente niza a:}#
#\ulaz{173 -25 23 7 17 25 34 61 -4612}#
#\izlaz{Niz a nakon izmene:}#
#\izlaz{-25 7 25 61 -4612}#
\end{upotreba}
\end{miditest}
\begin{miditest}
\begin{upotreba}{2}
#\naslovInt#
#\izlaz{Unesite dimenziju niza:}#
#\ulaz{0}#
#\izlaz{Greska: neispravan unos.}#
\end{upotreba}
\end{miditest}
\linkresenje{v.brisanje_elemenata}
\end{Exercise}

\ifresenja
\begin{Answer}[ref=v.brisanje_elemenata]
\includecode{resenja/3_PredstavljanjePodataka/2.1_Nizovi/brisanje_elemenata.c}
\end{Answer}
\fi


\begin{Exercise}[label=deljivi_indeksom] 
Napisati program koji u nizu dužine $n$ čiji se elementi učitavaju sa
ulaza eliminiše sve brojeve koji nisu deljivi svojim indeksom. Niz
reorganizovati tako da nema \emph{rupa} koje su nastale eliminacijom
elemenata i ispisati na standardni izlaz. Maksimalan broj elemenata
niza je $700$. Ne razmatrati da li je u novom nizu, nakom brisanja i
pomeranja, element deljiv svojim indeksom.
U slučaju neispravnog unosa, ispisati odgovarajuću poruku o grešci. 
\napomena{Nulti element niza treba zadržati jer nije dozvoljeno
  deljenje nulom.}

\begin{miditest}
\begin{upotreba}{1}
#\naslovInt#
#\izlaz{Unesite broj elemenata niza:}\ulaz{10}#
#\izlaz{Unesite elemente niza:}#
#\ulaz{4 2 1 6 7 8 10 2 16 3}#
#\izlaz{Novi niz:}#  
#\izlaz{4 2 6 16}#
\end{upotreba}
\end{miditest}
\begin{miditest}
\begin{upotreba}{2}
#\naslovInt#
#\izlaz{Unesite broj elemenata niza:}\ulaz{10}#
#\izlaz{Unesite elemente niza:}#
#\ulaz{-8 5 10 6 7 10 8 2 16 27}#
#\izlaz{Novi niz:}#  
#\izlaz{-8 5 10 6 10 16 27}#
\end{upotreba}
\end{miditest}
\linkresenje{deljivi_indeksom}
\end{Exercise}

\ifresenja
\begin{Answer}[ref=deljivi_indeksom]
\includecode{resenja/3_PredstavljanjePodataka/2.1_Nizovi/deljivi_indeksom.c}
\end{Answer}
\fi


\begin{Exercise}[label=p.izbacivanje_prostih_elemenata] 
Napisati program koji za učitani niz ispisuje niz koji se dobija
izbacivanjem svih elemenata koji su prosti brojevi. Zadatak rešiti na
dva načina: korišćenjem pomoćnog niza i transformacijom polaznog niza.
Maksimalan broj elemenata niza je $100$.  \napomena{Broj $1$ nije
  prost}. U slučaju neispravnog unosa, ispisati odgovarajuću poruku o grešci. 

\begin{miditest}
\begin{upotreba}{1}
#\naslovInt#
#\izlaz{Unesite broj elemenata niza:}\ulaz{5}#
#\izlaz{Unesite elemente niza:}\ulaz{11 5 6 48 8}#
#\izlaz{6 48 8}#
\end{upotreba}
\end{miditest}
\begin{miditest}
\begin{upotreba}{2}
#\naslovInt#
#\izlaz{Unesite broj elemenata niza:}\ulaz{4}#
#\izlaz{Unesite elemente niza:}\ulaz{11 5 19 21}#
#\izlaz{21}#
\end{upotreba}
\end{miditest}

\begin{miditest}
\begin{upotreba}{3}
#\naslovInt#
#\izlaz{Unesite broj elemenata niza:}\ulaz{5}#
#\izlaz{Unesite elemente niza:}\ulaz{12 18 9 31 7}#
#\izlaz{12 18 9}#
\end{upotreba}
\end{miditest}
\begin{miditest}
\begin{upotreba}{4}
#\naslovInt#
#\izlaz{Unesite broj elemenata niza:}\ulaz{3}#
#\izlaz{Unesite elemente niza:}\ulaz{-31 11 -19}#
#\izlaz{}#
\end{upotreba}
\end{miditest}

\begin{miditest}
\begin{upotreba}{5}
#\naslovInt#
#\izlaz{Unesite broj elemenata niza:}\ulaz{5}#
#\izlaz{Unesite elemente niza:}\ulaz{-2 15 -11 8 7}#
#\izlaz{15 8}#
\end{upotreba}
\end{miditest}
\linkresenje{p.izbacivanje_prostih_elemenata}
\end{Exercise}

\ifresenja
\begin{Answer}[ref=p.izbacivanje_prostih_elemenata]
\includecode{resenja/3_PredstavljanjePodataka/2.1_Nizovi/izbacivanje_prostih_elemenata.c}
\end{Answer}
\fi


%-----------------------------------------------------------------------------------
% Serije u nizu
%-----------------------------------------------------------------------------------

\begin{Exercise}[label=neopadajuce]
  Napisati funkciju koja proverava da li su elementi celebrojnog niza
  uređeni neopadajuće. Maksimalan broj elemenata niza je $300$.
U slučaju neispravnog unosa, ispisati odgovarajuću poruku o grešci. 

\begin{miditest}
\begin{upotreba}{1}
#\naslovInt#
#\izlaz{Unesite dimenziju niza:}#
#\ulaz{8}#
#\izlaz{Unesite elemente niza:}#
#\ulaz{-40 -8 -8 2 30 30 46 200}#
#\izlaz{Jeste uredjen neopadajuce.}#  
\end{upotreba}
\end{miditest}
\begin{miditest}
\begin{upotreba}{2}
#\naslovInt#
#\izlaz{Unesite dimenziju niza:}#
#\ulaz{4}#
#\izlaz{Unesite elemente niza:}#
#\ulaz{4 23 15 30}#
#\izlaz{Nije uredjen neopadajuce.}#  
\end{upotreba}
\end{miditest}
\linkresenje{neopadajuce}
\end{Exercise}

\ifresenja
\begin{Answer}[ref=neopadajuce]
\includecode{resenja/3_PredstavljanjePodataka/2.1_Nizovi/neopadajuce.c}
\end{Answer}
\fi


\begin{Exercise}[label=najduzi_neopadajuci]
Svaki indeks niza označava jedan dan u mesecu, a elementi celobrojnog
niza predstavljaju broj artikala koji se prodao tog dana. Naći koliko
najduže je iz dana u dan broj prodatih artikala rastao.
U slučaju neispravnog unosa, ispisati odgovarajuću poruku o grešci. 

\begin{miditest}
\begin{upotreba}{1}
#\naslovInt#
#\izlaz{Unesite dimenziju niza:}\ulaz{30}#
#\izlaz{Unesite broj prodatih artikala:}#
#\ulaz{89 171 112 67 119 36 181 157}#
#\ulaz{49 96 73 116 21 172}#
#\ulaz{140 0 23 71 157 135 11 166 21}#
#\ulaz{56 56 87 103 183 148 174}#
#\izlaz{Duzina najduzeg neopadajuceg}#
#\izlaz{prodavanja je 6.}#
\end{upotreba}
\end{miditest}
\begin{miditest}
\begin{upotreba}{2}
#\naslovInt#
#\izlaz{Unesite dimenziju niza:}\ulaz{31}#
#\izlaz{Unesite broj prodatih artikala:}#
#\ulaz{215 223 262 95 18 116 334 97}#
#\ulaz{146 146 19 314 270 115 21 40}#
#\ulaz{253 27 210 68 96 175 41 242}#
#\ulaz{98 163 8 218 107 102}#
#\izlaz{Duzina najduzeg neopadajuceg}#
#\izlaz{prodavanja je 3.}#
\end{upotreba}
\end{miditest}
\linkresenje{najduzi_neopadajuci}
\end{Exercise}

\ifresenja
\begin{Answer}[ref=najduzi_neopadajuci]
\includecode{resenja/3_PredstavljanjePodataka/2.1_Nizovi/najduzi_neopadajuci.c}
\end{Answer}
\fi


\begin{Exercise}[label=uzastopni_jednaki] 
Napisati funkciju koja određuje dužinu najduže serije jednakih
uzastopnih elemenata u datom nizu brojeva. Maksimalan broj elemenata
niza je $100$.
U slučaju neispravnog unosa, ispisati odgovarajuću poruku o grešci. 

\begin{miditest}
\begin{upotreba}{1}
#\naslovInt#
#\izlaz{Unesite dimenziju niza: }\ulaz{8}#
#\izlaz{Unesite elemente niza: }#
#\ulaz{9 -1 2 2 2 2 80 -200}#
#\izlaz{Duzina najduze serije je 4.}#
\end{upotreba}
\end{miditest}
\begin{miditest}
\begin{upotreba}{2}
#\naslovInt#
#\izlaz{Unesite dimenziju niza: }\ulaz{8}#
#\izlaz{Unesite elemente niza: }#
#\ulaz{9 9 0 -3 -3 -3 -3 72}#
#\izlaz{Duzina najduze serije je 4.}#
\end{upotreba}
\end{miditest}
\linkresenje{uzastopni_jednaki}
\end{Exercise}

\ifresenja
\begin{Answer}[ref=uzastopni_jednaki]
\includecode{resenja/3_PredstavljanjePodataka/2.1_Nizovi/uzastopni_jednaki.c}
\end{Answer}
\fi


\begin{Exercise}[label=podniz] 
Napisati funkciju koja određuje da li se jedan niz javlja kao podniz
uzastopnih elemenata drugog niza. Zadatak rešiti na dva načina:
\begin{enumerate}
\item Tako da elementi jednog niza moraju da budu uzastopni u drugom nizu.
\item Tako da elementi jednog niza ne moraju da budu uzastopni u drugom nizu, ali je redosled
  pojavljivanja isti.
\end{enumerate}
Maksimalan broj elemenata niza je $100$.
U slučaju neispravnog unosa, ispisati odgovarajuću poruku o grešci. 

\begin{miditest}
\begin{upotreba}{1}
#\naslovInt#
#\izlaz{Unesite dimenziju niza: }\ulaz{8}#
#\izlaz{Unesite elemente niza: }#
#\ulaz{-4 2 7 90 -22 15  14 7}#
#\izlaz{Unesite dimenziju niza: }\ulaz{4}#
#\izlaz{Unesite elemente niza: }\ulaz{90 -22 15 14}#
#\izlaz{Elementi drugog niza cine}#
#\izlaz{uzastopni podniz prvog niza.}#
#\izlaz{Elementi drugog niza cine}#
#\izlaz{podniz prvog niza.}#
\end{upotreba}
\end{miditest}
\begin{miditest}
\begin{upotreba}{2}
#\naslovInt#
#\izlaz{Unesite dimenziju niza: }\ulaz{8}#
#\izlaz{Unesite elemente niza: }#
#\ulaz{-4 2 7 90 -22 15  14 7}#
#\izlaz{Unesite dimenziju niza: }\ulaz{4}#
#\izlaz{Unesite elemente niza: }\ulaz{2 7 15 7}#
#\izlaz{Elementi drugog niza ne cine}#
#\izlaz{uzastopni podniz prvog niza.}#
#\izlaz{Elementi drugog niza cine}#
#\izlaz{podniz prvog niza.}#
\end{upotreba}
\end{miditest}

\begin{miditest}  
\begin{upotreba}{3}
#\naslovInt#
#\izlaz{Unesite dimenziju niza: }\ulaz{8}#
#\izlaz{Unesite elemente niza: }#
#\ulaz{-4 2 7 90 -22 15 14 7}#
#\izlaz{Unesite dimenziju niza: }\ulaz{4}#
#\izlaz{Unesite elemente niza: }\ulaz{90 -22 200 1}#
#\izlaz{Elementi drugog niza ne cine}#
#\izlaz{uzastopni podniz prvog niza.}#
#\izlaz{Elementi drugog niza ne}#
#\izlaz{cine podniz prvog niza.}#  
\end{upotreba}
\end{miditest}
\begin{miditest}
\begin{upotreba}{4}
#\naslovInt#
#\izlaz{Unesite dimenziju niza: }\ulaz{8}#
#\izlaz{Unesite elemente niza: }#
#\ulaz{-4 2 7 90 -22 15 14 7}#
#\izlaz{Unesite dimenziju niza: }\ulaz{1}#
#\izlaz{Unesite elemente niza: }\ulaz{90}#
#\izlaz{Elementi drugog niza cine}#
#\izlaz{uzastopni podniz prvog niza.}#
#\izlaz{Elementi drugog niza cine}#
#\izlaz{podniz prvog niza.}#
\end{upotreba}
\end{miditest}
\linkresenje{podniz}
\end{Exercise}

\ifresenja
\begin{Answer}[ref=podniz]
\includecode{resenja/3_PredstavljanjePodataka/2.1_Nizovi/podniz.c}
\end{Answer}
\fi

%-----------------------------------------------------------------------------------
% Permutacija
%-----------------------------------------------------------------------------------

\begin{Exercise}[label=permutacija] 
Za celobrojni niz $a$ dimenzije $n$ kažemo da je \textit{permutacija}
ako sadrži sve brojeve od $1$ do $n$.
\begin{enumerate}
\item Napisati funkciju \kckod{void brojanje(int a[], int b[], int n)}
  koja na osnovu celobrojnog niza $a$ dimenzije $n$ formira niz $b$
  tako što $i$-ti element niza $b$ odgovara broju pojavljivanja
  vrednosti $i$ u nizu $a$.
\item Napisati funkciju \kckod{int permutacija(int a[], int n)} koja
  proverava da li je zadati niz permutacija. Funkcija vraća vrednost
  $1$ ako je svojstvo ispunjeno, odnosno $0$ ako
  nije. \uputstvo{Koristiti funkciju $brojanje$ iz tačke (a).}
\end{enumerate}
Napisati program koji sa standardnog ulaza učitava dimenziju niza i
elemente niza i ispisuje da li je uneti niz permutacija ili
ne. Maksimalan broj elemenata niza je $100$.
U slučaju neispravnog unosa, ispisati odgovarajuću poruku o grešci. 

\begin{miditest}
\begin{upotreba}{1}
#\naslovInt#
#\izlaz{Unesite broj elemenata niza:}\ulaz{5}#
#\izlaz{Unesite elemente niza:}#
#\ulaz{1 5 4 3 2}#
#\izlaz{Uneti niz je permutacija.}#
\end{upotreba}
\end{miditest}
\begin{miditest}
\begin{upotreba}{2}
#\naslovInt#
#\izlaz{Unesite broj elemenata niza:}\ulaz{6}#
#\izlaz{Unesite elemente niza:}#
#\ulaz{2 3 3 1 1 5}#
#\izlaz{Uneti niz nije permutacija.}#
\end{upotreba}
\end{miditest}
\linkresenje{permutacija}
\end{Exercise}

\ifresenja
\begin{Answer}[ref=permutacija]
\includecode{resenja/3_PredstavljanjePodataka/2.1_Nizovi/permutacija.c}
\end{Answer}
\fi


\begin{Exercise}[label=p.brojevi_sa_istim_zapisima] 
 Napisati program koji za dva cela broja $x$ i $y$ koja se učitavaju
 sa standardnog ulaza proverava da li se zapisuju pomoću istih
 cifara. \uputstvo{Rešiti korišćenjem nizova. Pogledati zadatak
   \ref{FUN_15}}.
 
\begin{miditest}
\begin{upotreba}{1}
#\naslovInt#
#\izlaz{Unesite dva broja:}\ulaz{251 125}#
#\izlaz{Brojevi se zapisuju istim ciframa.}#
\end{upotreba}
\end{miditest}
\begin{miditest}
\begin{upotreba}{2}
#\naslovInt#
#\izlaz{Unesite dva broja:}\ulaz{8898 9988}#
#\izlaz{Brojevi se ne zapisuju istim ciframa.}#
\end{upotreba}
\end{miditest}

\begin{miditest}
\begin{upotreba}{3}
#\naslovInt#
#\izlaz{Unesite dva broja:}\ulaz{-7391 1397}#
#\izlaz{Brojevi se zapisuju istim ciframa.}#
\end{upotreba}
\end{miditest} 
\linkresenje{p.brojevi_sa_istim_zapisima}
\end{Exercise}

\ifresenja
\begin{Answer}[ref=p.brojevi_sa_istim_zapisima]
\includecode{resenja/3_PredstavljanjePodataka/2.1_Nizovi/brojevi_sa_istim_zapisima.c}
\end{Answer}
\fi


\ifresenja
\section{Rešenja}
\shipoutAnswer
\fi
