\chapter{Predstavljanje podataka}

\section{Nizovi}

\begin{Exercise}[label=v.skalarni_proizvod] 
Ako su $a = (a_1, \ldots, a_n)$ i $b = (b_1,\ldots, b_n)$ vektori dimenzije $n$, njihov skalarni proizvod je suma $a_1\cdot b_1 + \ldots + a_n\cdot b_n$. Napisati program koji računa skalarni proizvod dva vektora. Vektori se zadaju kao celobrojni nizovi sa najviše $100$ elemenata. Program prvo učitava dimenziju nizova, zatim elemente nizova i na izlaz ispisuje vrednost skalarnog proizvoda ili odgovarajuću poruku u slučaju greške. \\ 
\begin{miditest}
\begin{upotreba}{1}
#\naslovInt#
#\izlaz{Unesite dimenziju vektora: }#
#\ulaz{5}#
#\izlaz{Unesite koordinate vektora a:}#
#\ulaz{8 -2 0 2 4}#
#\izlaz{Unesite koordinate vektora b:}#
#\ulaz{35 12 5 -6 -1}#
#\izlaz{Skalarni proizvod vektora a i b:}# 
#\izlaz{240}#
\end{upotreba}
\end{miditest}
\begin{miditest}
\begin{upotreba}{2}
#\naslovInt#
#\izlaz{Unesite dimenziju vektora: }#
#\ulaz{3}#
#\izlaz{Unesite koordinate vektora a:}#
#\ulaz{-1 0 1}#
#\izlaz{Unesite koordinate vektora b:}#
#\ulaz{5 5 5}#
#\izlaz{Skalarni proizvod vektora a i b:}# 
#\izlaz{0}#
\end{upotreba}
\end{miditest}
\begin{miditest}
\begin{upotreba}{3}
#\naslovInt#
#\izlaz{Unesite dimenziju vektora: }#
#\ulaz{120}#
#\izlaz{Greska: Nedozvoljena vrednost!}#
\end{upotreba}
\end{miditest}
\linkresenje{v.skalarni_proizvod}
\end{Exercise}
\begin{Answer}[ref=v.skalarni_proizvod]
\includecode{resenja/2_PredstavljanjePodataka/2.1_Nizovi/skalarni_proizvod.c}
\end{Answer}

\begin{Exercise}[label=v.parni_elementi] 
Napisati program koji učitava broj elemenata niza (ne veći od $100$), zatim elemente niza, i ispisuje:
\begin{description}
\item{a)} elemente niza koji se nalaze na parnim pozicijama
\item{b)} parne elemente niza
\end{description}
U slučaju greške ispisati odgovarajuću poruku.  \\
\begin{miditest}
\begin{upotreba}{1}
#\naslovInt#
#\izlaz{Unesite dimenziju niza:}#
#\ulaz{6}#
#\izlaz{Unesite elemente niza:}#
#\ulaz{1 8 2 -5 -13 75}#
#\izlaz{Elementi niza na parnim pozicijama:}#
#\izlaz{1 2 -13}#
#\izlaz{Parni elementi niza:}#
#\izlaz{8 2}#
\end{upotreba}
\end{miditest}
\begin{miditest}
\begin{upotreba}{2}
#\naslovInt#
#\izlaz{Unesite dimenziju niza:}#
#\ulaz{3}#
#\izlaz{Unesite elemente niza:}#
#\ulaz{11 81 -63}#
#\izlaz{Elementi niza na parnim pozicijama:}#
#\izlaz{11 -63}#
#\izlaz{Parni elementi niza:}#
#\izlaz{}#
\end{upotreba}
\end{miditest}
\begin{miditest}
\begin{upotreba}{3}
#\naslovInt#
#\izlaz{Unesite dimenziju niza: }#
#\ulaz{-4}#
#\izlaz{Greska: Nedozvoljena vrednost!}#
\end{upotreba}
\end{miditest}
\linkresenje{v.parni_elementi}
\end{Exercise}
\begin{Answer}[ref=v.parni_elementi]
\includecode{resenja/2_PredstavljanjePodataka/2.1_Nizovi/parni_elementi.c}
\end{Answer}

\begin{Exercise}[label=v.prebrojavanje_cifara] 
Napisati program koji učitava jedan ceo broj, a zatim ispisuje broj pojavljivanja svake od cifara u zapisu tog broja. \uputstvo{Za evidenciju broja pojavljivanja svake cifre pojedinačno, koristiti niz.} \\
\begin{miditest}
\begin{upotreba}{1}
#\naslovInt#
#\izlaz{Unesite ceo broj:}#
#\ulaz{2355623}#
#\izlaz{U zapisu broja 2355623, cifra 2 se pojaviljuje 2 puta}#
#\izlaz{U zapisu broja 2355623, cifra 3 se pojaviljuje 2 puta}#
#\izlaz{U zapisu broja 2355623, cifra 5 se pojaviljuje 2 puta}#
#\izlaz{U zapisu broja 2355623, cifra 6 se pojaviljuje 1 puta}#
\end{upotreba}
\end{miditest}
\begin{miditest}
\begin{upotreba}{2}
#\naslovInt#
#\izlaz{Unesite ceo broj:}#
#\ulaz{-39902}#
#\izlaz{U zapisu broja -39902, cifra 0 se pojaviljuje 1 puta}#
#\izlaz{U zapisu broja -39902, cifra 2 se pojaviljuje 1 puta}#
#\izlaz{U zapisu broja -39902, cifra 3 se pojaviljuje 1 puta}#
#\izlaz{U zapisu broja -39902, cifra 9 se pojaviljuje 2 puta}#
\end{upotreba}
\end{miditest}
\linkresenje{v.prebrojavanje_cifara}
\end{Exercise}
\begin{Answer}[ref=v.prebrojavanje_cifara]
\includecode{resenja/2_PredstavljanjePodataka/2.1_Nizovi/prebrojavanje_cifara.c}
\end{Answer}

\begin{Exercise}[label=p.brojevi_sa_istim_zapisima] 
 Napisati program koji za dva cela broja $x$ i $y$ koja se učitavaju sa standardnog ulaza proverava da li se zapisuju pomoću istih cifara. \\
\begin{miditest}
\begin{upotreba}{1}
#\naslovInt#
#\izlaz{Unesite dva broja:}\ulaz{251 125}#
#\izlaz{Brojevi se zapisuju istim ciframa!}#
\end{upotreba}
\end{miditest}
\begin{miditest}
\begin{upotreba}{2}
#\naslovInt#
#\izlaz{Unesite dva broja:}\ulaz{8898 9988}#
#\izlaz{Brojevi se ne zapisuju istim ciframa!}#
\end{upotreba}
\end{miditest}
\begin{miditest}
\begin{upotreba}{3}
#\naslovInt#
#\izlaz{Unesite dva broja:}\ulaz{-7391 1397}#
#\izlaz{Brojevi se zapisuju istim ciframa!}#
\end{upotreba}
\end{miditest} 
\linkresenje{p.brojevi_sa_istim_zapisima}
\end{Exercise}
\begin{Answer}[ref=p.brojevi_sa_istim_zapisima]
\includecode{resenja/2_PredstavljanjePodataka/2.1_Nizovi/brojevi_sa_istim_zapisima.c}
\end{Answer}

\begin{Exercise}[label=v.prebrojavanje] 
Napisati program koji učitava karakter po karakter sa standardnog ulaza sve do kraja ulaza, a zatim izračunava i ispisuje koliko se puta u unetom tekstu pojavila svaka od cifara, svako malo slovo i svako veliko slovo. Ispisati broj pojavljivanja samo za karaktere koji su se u unetom tekstu pojavili barem jednom. \uputstvo{Za evidenciju broja pojavljivanja cifara, malih i velih slova korisiti pojedinačne nizove.} \\
\begin{maxitest}
\begin{upotreba}{1}
#\naslovInt#
#\ulaz{123 abcabcabc 123}#
#\izlaz{U zapisu broja 2355623, cifra 2 se pojaviljuje 2 puta}#
#\izlaz{Karakter 1 se pojavljuje 2 puta}#
#\izlaz{Karakter 2 se pojavljuje 2 puta}#
#\izlaz{Karakter 3 se pojavljuje 2 puta}#
#\izlaz{Karakter a se pojavljuje 3 puta}#
#\izlaz{Karakter b se pojavljuje 3 puta}#
#\izlaz{Karakter c se pojavljuje 3 puta}#
\end{upotreba}
\end{maxitest}
\linkresenje{v.prebrojavanje}
\end{Exercise}
\begin{Answer}[ref=v.prebrojavanje]
\includecode{resenja/2_PredstavljanjePodataka/2.1_Nizovi/prebrojavanje.c}
\end{Answer}

%TODO resiti
\begin{Exercise}[label=vp.bez_resenja_3] 
Sa standardnog ulaza se unosi jedna linija teksta. Napisati program koji izračunava i ispisuje koliko puta se pojavilo svako od slova engleskog alfabeta u unetom tekstu. Ne praviti razliku između malih i velikih slova. \\
\begin{maxitest}
\begin{upotreba}{1}
#\naslovInt#
#\ulaz{haHJjkL}#
#\izlaz{ a:1  b:0  c:0  d:0  e:0  f:0  g:0  h:2  i:0  j:2  k:1  l:1  m:0  n:0  o:0  p:0  q:0  r:0  s:0
   t:0  u:0  v:0  w:0  x:0  y:0  z:0}#
\end{upotreba}
\end{maxitest}
\begin{maxitest}
\begin{upotreba}{2}
#\naslovInt#
#\ulaz{DanaS j3 \_j\_utRo laBU78d}#
#\izlaz{a:3  b:1  c:0  d:2  e:0  f:0  g:0  h:2  i:0  j:2  k:0  l:1  m:0  n:1  o:1  p:0  q:0  r:1  s:1
  t:1  u:2  v:0  w:0  x:0  y:0  z:0}#
\end{upotreba}
\end{maxitest}
\begin{maxitest}
\begin{upotreba}{3}
#\naslovInt#
#\ulaz{Sao PaoLo 1998 \_JuZna Amerika90}#
#\izlaz{a:5  b:0  c:0  d:2  e:1  f:0  g:0  h:0  i:1  j:1  k:1  l:1  m:1  n:1  o:3  p:1  q:0  r:1  s:1
  t:0  u:1  v:0  w:0  x:0  y:0  z:0}#
\end{upotreba}
\end{maxitest}
%\linkresenje{vp.bez_resenja_3}
\end{Exercise}
\begin{Answer}[ref=vp.bez_resenja_3]
%\includecode{resenja/2_PredstavljanjePodataka/2.1_Nizovi/....c}
\end{Answer}

\begin{Exercise}[label=v.ukrstanje_nizova] 
Napisati program koji formira i na izlaz ispisuje niz $a_0, b_0, a_1, b_1, \ldots, a_{n-1},
b_{n-1}$ koji se dobija naizmeničnim raspoređivanjem elemenata nizova $a$ i $b$ dužine $n$. Sa standardnog ulaza se prvo učitava dimenzija nizova $n$ (broj ne veći od $100$), potom elementi niza $a$ i na kraju, elementi niza $b$. U slučaju neispravnog unosa ispisati odgovarajuću poruku o grešci. \\
\begin{miditest}
\begin{upotreba}{1}
#\naslovInt#
#\izlaz{Unesite dimenziju nizova:}#
#\ulaz{5}#
#\izlaz{Unesite elemente niza a:}#
#\ulaz{2 -5 11 4 8}#
#\izlaz{Unesite elemente niza b:}#
#\ulaz{3 3 9 -1 17}#
#\izlaz{Rezultujuci niz:}#
#\izlaz{2 3 -5 3 11 9 4 -1 8 17}#
\end{upotreba}
\end{miditest}
\begin{miditest}
\begin{upotreba}{2}
#\naslovInt#
#\izlaz{Unesite dimenziju nizova:}#
#\ulaz{105}#
#\izlaz{Greska: Nedozvoljena vrednost!}#
\end{upotreba}
\end{miditest}
\linkresenje{v.ukrstanje_nizova}
\end{Exercise}
\begin{Answer}[ref=v.ukrstanje_nizova]
\includecode{resenja/2_PredstavljanjePodataka/2.1_Nizovi/ukrstanje_nizova.c}
\end{Answer}

\begin{Exercise}[label=p.nizovi_spajanje] 
 Sa standardnog ulaza se učitava ceo broj $n$ manji od $100$, zatim i elementi dvaju nizova $a$ i $b$ dimenzije $n$. Napisati program koji formira i ispisuje niz $c$ čiju prvu polovinu čine elementi niza $b$, a drugu polovinu elementi niza $a$. U slučaju greške ispisati odgovarajuću poruku.\\
\begin{miditest}
\begin{upotreba}{1}
#\naslovInt#
#\izlaz{Unesite broj n:}\ulaz{3}#
#\izlaz{Unesite elemente niza a:}\ulaz{4 -8 32}#
#\izlaz{Unesite elemente niza b:}\ulaz{5 2 11}#
#\izlaz{5 2 11 4 -8 32}#
\end{upotreba}
\end{miditest}
\begin{miditest}
\begin{upotreba}{2}
#\naslovInt#
#\izlaz{Unesite broj n:}\ulaz{4}#
#\izlaz{Unesite elemente niza a:}\ulaz{1 0 -1 0}#
#\izlaz{Unesite elemente niza b:}\ulaz{5 5 5 3}#
#\izlaz{5 5 5 3 1 0 -1 0}#
\end{upotreba}
\end{miditest}
\begin{miditest}
\begin{upotreba}{3}
#\naslovInt#
#\izlaz{Unesite dimenziju niza:}\ulaz{145}#
#\izlaz{Greska: Nedozvoljena vrednost!}#
\end{upotreba}
\end{miditest}
\linkresenje{p.nizovi_spajanje}
\end{Exercise}
\begin{Answer}[ref=p.nizovi_spajanje]
\includecode{resenja/2_PredstavljanjePodataka/2.1_Nizovi/nizovi_spajanje.c}
\end{Answer}

%TODO: rešiti, izdvojeno iz jednog većeg zadatka... obavezno
\begin{Exercise}[label=p.nizovi_spajanje_sortiranih] 
 Sa standardnog ulaza se učitava ceo broj $n$ manji od $100$, zatim i elementi dvaju sortiranih neopadajuće nizova $a$ i $b$ dimenzije $n$. Napisati program koji formira i ispisuje niz $c$ koji se dobija spajanjem nizova $a$ i $b$ u treći, takođe sortiran neopadajuće, niz. U slučaju greške ispisati odgovarajuću poruku.\\
\begin{miditest}
\begin{upotreba}{1}
#\naslovInt#
#\izlaz{Unesite broj n:}\ulaz{5}#
#\izlaz{Unesite elemente niza a:}#
#\ulaz{2 11 28 40 63}#
#\izlaz{Unesite elemente niza b:}#
#\ulaz{-19 -5 5 11 52}#
#\izlaz{Niz c:}#
#\izlaz{-19 -5 2 5 11 11 28 40 52 63}#
\end{upotreba}
\end{miditest}
\begin{miditest}
\begin{upotreba}{2}
#\naslovInt#
#\izlaz{Unesite broj n:}\ulaz{3}#
#\izlaz{Unesite elemente niza a:}#
#\ulaz{-2 4 8}#
#\izlaz{Unesite elemente niza b:}#
#\ulaz{6 15 19}#
#\izlaz{Niz c:}#
#\izlaz{-2 4 6 8 15 19}#
\end{upotreba}
\end{miditest}
\begin{miditest}
\begin{upotreba}{3}
#\naslovInt#
#\izlaz{Unesite dimenziju niza:}\ulaz{145}#
#\izlaz{Greska: Nedozvoljena vrednost!}#
\end{upotreba}
\end{miditest}
\linkresenje{p.nizovi_spajanje_sortiranih}
\end{Exercise}
\begin{Answer}[ref=p.nizovi_spajanje_sortiranih]
%\includecode{resenja/2_PredstavljanjePodataka/2.1_Nizovi/nizovi_spajanje_sortiranih.c}
\end{Answer}

%TODO: resiti
\begin{Exercise}[label=vp.bez_resenja_7] 
Napisati program koji sa standardnog ulaza učitava $10$ celih brojeva i razdvaja ih na parne i neparne tako što parne brojeve upisuje na početak niza, a neparne brojeve na kraj niza. Ispisati niz dobijen na ovaj način. \napomena{Nije dozvoljeno koristiti pomoćne nizove.}\\
\begin{miditest}
\begin{upotreba}{1}
#\naslovInt#
#\izlaz{Unesite 10 brojeva:}#
#\ulaz{-2 8 11 53 59 20 17 -8 3 14}#
#\izlaz{Rezultujuci niz:}#
#\izlaz{-2 8 20 -8 14 3 17 59 53 11}#
\end{upotreba}
\end{miditest}
\begin{miditest}
\begin{upotreba}{2}
#\naslovInt#
#\izlaz{Unesite 10 brojeva:}#
#\ulaz{9 142 -9 -278 -69 33 34 28 -6 14}#
#\izlaz{Rezultujuci niz:}#
#\izlaz{142 -278 34 28 -6 14 33 -69 -9 9}#
\end{upotreba}
\end{miditest}
%\linkresenje{vp.bez_resenja_7}
\end{Exercise}
\begin{Answer}[ref=vp.bez_resenja_7]
%\includecode{resenja/2_PredstavljanjePodataka/2.1_Nizovi/vp.bez_resenja_7.c}
\end{Answer}

\begin{Exercise}[label=v.brisanje_elemenata] 
Napisati program koji učitava dimenziju $n$ celobrojnog niza $a$ i
njegove elemente, a zatim iz niza $a$ izbacuje sve elemente koji nisu
deljivi svojom poslednjom cifrom. Izuzetak su elementi čija je poslednja
cifra $0$ koje treba zadržati. Program treba da ispiše izmenjeni niz
na standardni izlaz. Niz $a$ sadrži najviše $100$ elemenata. \\
\begin{miditest}
\begin{upotreba}{1}
#\naslovInt#
#\izlaz{Unesite dimenziju niza:}#
#\ulaz{9}#
#\izlaz{Unesite elemente niza a:}#
#\ulaz{173 -25 23 7 17 25 34 61 -4612}#
#\izlaz{Niz a nakon izmene:}#
#\izlaz{-25 7 25 61 -4612}#
\end{upotreba}
\end{miditest}
\begin{miditest}
\begin{upotreba}{2}
#\naslovInt#
#\izlaz{Unesite dimenziju niza:}#
#\ulaz{0}#
#\izlaz{Greska: Nedozvoljena vrednost!}#
\end{upotreba}
\end{miditest}
\linkresenje{v.brisanje_elemenata}
\end{Exercise}
\begin{Answer}[ref=v.brisanje_elemenata]
\includecode{resenja/2_PredstavljanjePodataka/2.1_Nizovi/brisanje_elemenata.c}
\end{Answer}

%TODO ubaciti zadatak sa brisanjem 
\begin{Exercise}[label=vp.zadatak_bez_resenja_12] 
Napisati program koji u nizu dužine $n$ (broj manji od $100$) čiji se elementi učitavaju sa ulaza eliminiše sve brojeve koji nisu deljivi svojim indeksom. Niz reorganizovati tako da nema
\emph{rupa} koje su nastale eliminacijom elemenata i ispisati na standardni izlaz. U slučaju greške, ispisati odgovarajuću poruku. \napomena{Nulti element niza treba zadržati jer nije dozvoljeno deljenje nulom.}
\begin{miditest}
\begin{upotreba}{1}
#\naslovInt#
#\izlaz{Unesite broj elemenata niza:}\ulaz{10}#
#\izlaz{Unesite elemente niza:}#
#\ulaz{4 2 1 6 7 8 10 2 16 3}#
#\izlaz{4 2 6 16}#
\end{upotreba}
\end{miditest}
%\linkresenje{vp.zadatak_bez_resenja_12}
\end{Exercise}
\begin{Answer}[ref=vp.zadatak_bez_resenja_12]
%\includecode{resenja/2_PredstavljanjePodataka/2.1_Nizovi/zadatak_bez_resenja_12.c}
\end{Answer}


\begin{Exercise}[label=v.nizovi_funkcije_intro] 
Napisati funkcije za rad sa nizovima celih brojeva. 
\begin{description}
\item{a)} Napisati funkciju \kckod{void ucitaj(int a[], int n)} koja učitava elemente niza $a$ dimenzije $n$. 
\item{b)} Napisati funkciju \kckod{void stampaj(int a[], int n)} koja štampa elemente niza $a$ dimenzije $n$.
\item{c)} Napisati funkciju \kckod{int suma(int a[], int n)} koja računa i vraća sumu elemenata niza $a$ dimenzije $n$.   
\item{d)} Napisati funkciju \kckod{int prosek(int a[], int n)} koja računa i vraća prosečnu vrednost (aritmetičku sredinu) elemenata niza $a$ dimenzije $n$.
\item{e)} Napisati funkciju \kckod{int minimum(int a[], int n)} koja izračunava i vraća minimum elemenata niza $a$ dimenzije $n$.
\item{f)} Napisati funkciju \kckod{int pozicija\_maksimuma(int a[], int n)} koja izračunava i vraća poziciju maksimalnog elementa u nizu $a$ dimenzije $n$. U slučaju više pojavljivanja maksimalnog elementa, vratiti najmanju poziciju.  
\end{description}
Napisati program koji testira rad zadatih funkcija. Sa standardnog ulaza učitati dimenziju niza (broj ne veći od $100$). U slučaju greške ispisati odgovarajuću poruku. \\
\begin{miditest}
\begin{upotreba}{1}
#\naslovInt#
#\izlaz{Unesite dimenziju niza:}#
#\ulaz{5}#
#\ulaz{2 5 -2 8 11}#
#\izlaz{Ucitani niz: 2 5 -2 8 11}#
#\izlaz{Suma elemenata niza: 24}#
#\izlaz{Prosecna vrednost elemenata niza: 4.80}#
#\izlaz{Minimumalni element niza: -2}#
#\izlaz{Indeks maksimalnog elementa niza: 4}#
\end{upotreba}
\end{miditest}
\begin{miditest}
\begin{upotreba}{2}
#\naslovInt#
#\izlaz{Unesite dimenziju niza:}#
#\ulaz{-5}#
#\izlaz{Greska: Nedozvoljena vrednost!}#
\end{upotreba}
\end{miditest}
\linkresenje{v.nizovi_funkcije_intro}
\end{Exercise}
\begin{Answer}[ref=v.nizovi_funkcije_intro]
\includecode{resenja/2_PredstavljanjePodataka/2.1_Nizovi/nizovi_funkcije_intro.c}
\end{Answer}

\begin{Exercise}[label=v.nizovi_funkcije_razno] 
Napisati funkcije za rad sa nizovima celih brojeva. 
\begin{description}
\item{a)} Napisati funkciju koja proverava da li niz sadrži zadatu vrednost $m$. Povratna vrednost funkcije je $1$ ako je vrednost sadržana u nizu ili $0$ ako nije.
\item{b)} Napisati funkciju koja vraća vrednost prve pozicije na kojoj se nalazi element koji ima vrednost $m$ ili $-1$ ukoliko element nije u nizu.
\item{c)} Napisati funkciju koja vraća vrednost poslednje pozicije na kojoj se 
nalazi element koji ima vrednost $m$ ili $-1$ ukoliko element nije u nizu.
\item{d)} Napisati funkciju koja proverava da li elementi niza čine palindrom.
\item{e)} Napisati funkciju koja proverava da li su elementi niza uređeni neopadajuće.
%ovaj zadatak se nalazi negde pri kraju u okviru slicnog zadatka... bez_resenja_6
%\item{f)} Napisati funkciju koja pronalazi najdužu seriju jednakih elemenata u nizu.
\end{description}
Napisati i program koji testira rad napisanih funkcija. Na standardnom ulazu se zadaju dimenzija niza (broj ne veći od $100$), elementi niza i ceo broj $m$. U slučaju greške ispisati odgovarajuću poruku. \\
\begin{miditest}
\begin{upotreba}{1}
#\naslovInt#
#\izlaz{Unesite dimenziju niza:}#
#\ulaz{7}#
#\ulaz{8 11 -2 14 -2 11 8}#
#\izlaz{Ucitani niz: 8 11 -2 14 -2 11 8}#
#\izlaz{Unesite jedan ceo broj: }#
#\ulaz{11}#
#\izlaz{Niz sadrzi element cija je vrednost 11.}#
#\izlaz{Niz sadrzi element cija je vrednost 11.}#
#\izlaz{Indeks njegovog prvog pojavljivanja u nizu je 1.}#
#\izlaz{Niz sadrzi element cija je vrednost 11.}#
#\izlaz{Indeks njegovog poslednjeg pojavljivanja u nizu je 5.}#
#\izlaz{Elementi niza cine palindrom.}#
#\izlaz{Niz nije sortiran neopadajuce.}#
\end{upotreba}
\end{miditest}
\begin{miditest}
\begin{upotreba}{2}
#\naslovInt#
#\izlaz{Unesite dimenziju niza:}#
#\ulaz{-5}#
#\izlaz{Greska: Nedozvoljena vrednost!}#
\end{upotreba}
\end{miditest}
\linkresenje{v2.nizovi_funkcije_razno}
\end{Exercise}
\begin{Answer}[ref=v2.nizovi_funkcije_razno]
\includecode{resenja/2_PredstavljanjePodataka/2.1_Nizovi/nizovi_funkcije_razno.c}
\end{Answer}

\begin{Exercise}[label=v.nizovi_funkcije_pomeranja]
Napisati funkcije za rad sa nizovima celih brojeva. 
\begin{description}
\item{a)} Napisati funkciju koja sve vrednosti niza uvećava za zadatu vrednost $m$.
\item{b)} Napisati funkciju koja obrće elemente niza.     
\item{c)} Napisati funkciju koja rotira niz ciklično za jedno mesto u levo.
\item{d)} Napisati funkciju koja rotira niz ciklično za $k$ mesta u levo.
\end{description}
Napisati i program koji testira rad napisanih funkcija. Na standardnom ulazu se zadaju dimenzija niza (broj ne veći od $100$), elementi niza i celi brojevi $m$ i $k$. U slučaju greške ispisati odgovarajuću poruku. \\
\begin{miditest}
\begin{upotreba}{1}
#\naslovInt#
#\izlaz{Unesite dimenziju niza:}#
#\ulaz{6}#
#\ulaz{7 -3 11 783 26 -19}#
#\izlaz{Unesite jedan ceo broj:}#
#\ulaz{2}#
#\izlaz{Elementi niza nakon uvecanja za 2:}#
#\izlaz{9 -1 13	785	28 -17}#
#\izlaz{Elementi niza nakon obrtanja:}#
#\izlaz{-17	28 785	13	-1	9}#
#\izlaz{Elementi niza nakon rotiranja za 1 mesto ulevo:}#
#\izlaz{28 785 13 -1 9 -17}#
#\izlaz{Unesite jedan pozitivan ceo broj:}#
#\ulaz{3}#
#\izlaz{Elementi niza nakon rotiranja za 3 mesto ulevo:}#
#\izlaz{-1 9 -17 28	785	13}#
\end{upotreba}
\end{miditest}
\begin{miditest}
\begin{upotreba}{2}
#\naslovInt#
#\izlaz{Unesite dimenziju niza:}#
#\ulaz{252}#
#\izlaz{Greska: Nedozvoljena vrednost!}#
\end{upotreba}
\end{miditest}
\linkresenje{v.nizovi_funkcije_pomeranja}
\end{Exercise}
\begin{Answer}[ref=v.nizovi_funkcije_pomeranja]
\includecode{resenja/2_PredstavljanjePodataka/2.1_Nizovi/nizovi_funkcije_pomeranja.c}
\end{Answer}

\begin{Exercise}[label=p.kvadriranje_elemenata] 
Sa standardnog ulaza se učitava broj elemenata niza (broj manji od $100$), a zatim i njegovi elementi. Napisati program koji kvadrira sve negativne elemente niza i ispisuje rezultujući niz. U slučaju greške ispisati odgovarajuću poruku.\\
\begin{miditest}
\begin{upotreba}{1}
#\naslovInt#
#\izlaz{Unesite broj elemenata niza:}\ulaz{6}#
#\izlaz{Unesite elemente niza:}#
#\ulaz{12.34 -6 1 8 32.4 -16}#
#\izlaz{12.34 36 1 8 32.4 256}#
\end{upotreba}
\end{miditest}
\begin{miditest}
\begin{upotreba}{2}
#\naslovInt#
#\izlaz{Unesite broj elemenata niza:}\ulaz{9}#
#\izlaz{Unesite elemente niza:}#
#\ulaz{-8.25 6 17 2 -1.5 1 -7 2.65 -125.2}#
#\izlaz{68.0625 6 17 2 2.25 1 49 2.65 15675.04}#
\end{upotreba}
\end{miditest}
\begin{miditest}
\begin{upotreba}{3}
#\naslovInt#
#\izlaz{Unesite broj elemenata niza:}\ulaz{4}#
#\izlaz{Unesite elemente niza:}#
#\ulaz{9.53 5 1 4.89}#
#\izlaz{9.53 5 1 4.89}#
\end{upotreba}
\end{miditest}
\linkresenje{p.kvadriranje_elemenata}
\end{Exercise}
\begin{Answer}[ref=p.kvadriranje_elemenata]
\includecode{resenja/2_PredstavljanjePodataka/2.1_Nizovi/kvadriranje_elemenata.c}
\end{Answer}

\begin{Exercise}[label=p.pretraga_deljivih_sa_k] 
Sa standardnog ulaza se učitava dimenzija niza (broj manji od $100$), elementi niza i jedan ceo broj $k$. Napisati program koji štampa indekse elemenata koji su deljivi sa $k$. U slučaju greške ispisati odgovarajuću poruku. \\
\begin{miditest}
\begin{upotreba}{1}
#\naslovInt#
#\izlaz{Unesite dimenziju niza:}\ulaz{4}#
#\izlaz{Unesite elemente niza:}\ulaz{10 14 86 20}#
#\izlaz{Unesite broj k:}\ulaz{5}#
#\izlaz{0 3}#
\end{upotreba}
\end{miditest}
\begin{miditest}
\begin{upotreba}{2}
#\naslovInt#
#\izlaz{Unesite dimenziju niza:}\ulaz{4}#
#\izlaz{Unesite elemente niza:}\ulaz{6 14 8 9}#
#\izlaz{Unesite broj k:}\ulaz{5}#
#\izlaz{U nizu nema elemenata koji su deljivi brojem 5!}#
\end{upotreba}
\end{miditest}
\begin{miditest}
\begin{upotreba}{3}
#\naslovInt#
#\izlaz{Unesite dimenziju niza:}\ulaz{6}#
#\izlaz{Unesite elemente niza:}\ulaz{8 9 11 -4 8 11}#
#\izlaz{Unesite broj k:}\ulaz{2}#
#\izlaz{0 3 4}#
\end{upotreba}
\end{miditest}

\linkresenje{p.pretraga_deljivih_sa_k}
\end{Exercise}
\begin{Answer}[ref=p.pretraga_deljivih_sa_k]
\includecode{resenja/2_PredstavljanjePodataka/2.1_Nizovi/pretraga_deljivih_sa_k.c}
\end{Answer}

\begin{Exercise}[label=p.razmena_min_max] 
 Napisati program koji sa standardnog ulaza učitava dimenziju niza (broj manji od $100$) i elemente niza, a zatim štampa niz u kojem su najveći i najmanji element niza razmenili mesta. Ukoliko se najmanji ili najveći element više puta pojavljuju u nizu, uzeti u obzir njihova prva pojavljivanja. U slučaju greške ispisati odgovarajuću poruku.
\begin{miditest}
\begin{upotreba}{1}
#\naslovInt#
#\izlaz{Unesite dimenziju niza:}\ulaz{5}#
#\izlaz{Unesite elemente niza:}\ulaz{8 -2 11 19 4}#
#\izlaz{8 19 11 -2 4}#
\end{upotreba}
\end{miditest}
\begin{miditest}
\begin{upotreba}{2}
#\naslovInt#
#\izlaz{Unesite dimenziju niza:}\ulaz{10}#
#\izlaz{Unesite elemente niza:}#
#\ulaz{46 -2 51 8 -5 66 2 8 3 14}#
#\izlaz{46 -2 51 8 66 -5 2 8 3 14}#
\end{upotreba}
\end{miditest}
\begin{miditest}
\begin{upotreba}{3}
#\naslovInt#
#\izlaz{Unesite dimenziju niza:}\ulaz{145}#
#\izlaz{Greska: Nedozvoljena vrednost!}#
\end{upotreba}
\end{miditest}
\linkresenje{p.razmena_min_max}
\end{Exercise}
\begin{Answer}[ref=p.razmena_min_max]
\includecode{resenja/2_PredstavljanjePodataka/2.1_Nizovi/razmena_min_max.c}
\end{Answer}

%TODO resiti
\begin{Exercise}[label=vp.bez_resenja_11] 
Napisati funkciju \kckod{int min\_max(int a[], int n)} koja pronalazi indekse najmanjeg i najvećeg
elementa u nizu $a$ dimenzije $n$ koristeći samo jedan prolaz kroz niz, a zatim kao povratnu vrednost vraća
manji od ta dva indeksa. Napisati i program koji testira ovu funkciju učitavanjem niza celih brojeva maksimalne dužine $100$ elemenata. U slučaju greške ispisati odgovarajuću poruku. \\
\begin{miditest}
\begin{upotreba}{1}
#\naslovInt#
#\izlaz{Unesite broj elemenata niza:}#
#\ulaz{7}#
#\izlaz{Unesite elemente niza:}#
#\ulaz{5 8 -4 11 17 89 1}#
#\izlaz{2}#
\end{upotreba}
\end{miditest}
\begin{miditest}
\begin{upotreba}{2}
#\naslovInt#
#\izlaz{Unesite broj elemenata niza:}#
#\ulaz{3}#
#\izlaz{Unesite elemente niza:}#
#\ulaz{9 11 6}#
#\izlaz{1}#
\end{upotreba}
\end{miditest}
\begin{miditest}
\begin{upotreba}{3}
#\naslovInt#
#\izlaz{Unesite broj elemenata niza:}#
#\ulaz{-45}#
#\izlaz{Greska: Nedozvoljena vrednost!}#
\end{upotreba}
\end{miditest}
%\linkresenje{vp.bez_resenja_11}
\end{Exercise}
\begin{Answer}[ref=vp.bez_resenja_11]
%\includecode{resenja/2_PredstavljanjePodataka/2.1_Nizovi/bez_resenja_11.c}
\end{Answer}

\begin{Exercise}[label=p.niz_karaktera_obrnuto] 
 Napisati program koji učitava karaktere sa standardnog ulaza (najviše njih $100$) sve do pojave karaktera \textit{*}, a zatim ih ispisuje u redosledu suprotnom od redosleda čitanja. \\
\begin{miditest}
\begin{upotreba}{1}
#\naslovInt#
#\izlaz{Unesite karakter:}\ulaz{a}#
#\izlaz{Unesite karakter:}\ulaz{8}#
#\izlaz{Unesite karakter:}\ulaz{5}#
#\izlaz{Unesite karakter:}\ulaz{Y}#
#\izlaz{Unesite karakter:}\ulaz{I}#
#\izlaz{Unesite karakter:}\ulaz{o}#
#\izlaz{Unesite karakter:}\ulaz{?}#
#\izlaz{Unesite karakter:}\ulaz{*}#
#\izlaz{? o I Y 5 8 a}#
\end{upotreba}
\end{miditest}
\begin{miditest}
\begin{upotreba}{2}
#\naslovInt#
#\izlaz{Unesite karakter:}\ulaz{g}#
#\izlaz{Unesite karakter:}\ulaz{g}#
#\izlaz{Unesite karakter:}\ulaz{2}#
#\izlaz{Unesite karakter:}\ulaz{2}#
#\izlaz{Unesite karakter:}\ulaz{)}#
#\izlaz{Unesite karakter:}\ulaz{)}#
#\izlaz{Unesite karakter:}\ulaz{*}#
#\izlaz{) ) 2 2 g g}#
\end{upotreba}
\end{miditest}
\begin{miditest}
\begin{upotreba}{3}
#\naslovInt#
#\izlaz{Unesite karakter:}\ulaz{U}#
#\izlaz{Unesite karakter:}\ulaz{4}#
#\izlaz{Unesite karakter:}\ulaz{a}#
#\izlaz{Unesite karakter:}\ulaz{u}#
#\izlaz{Unesite karakter:}\ulaz{*}#
#\izlaz{u a 4 U}#
\end{upotreba}
\end{miditest}

\linkresenje{p.niz_karaktera_obrnuto}
\end{Exercise}
\begin{Answer}[ref=p.niz_karaktera_obrnuto]
\includecode{resenja/2_PredstavljanjePodataka/2.1_Nizovi/niz_karaktera_obrnuto.c}
\end{Answer}

\begin{Exercise}[label=p.elementi_3_pojavljivanja] 
 Sa standardnog ulaza se unosi broj elemenata niza $a$ (broj manji od $100$), a zatim i njegovi elementi. Napisati program koji od datog niza formira niz $b$ u koji ulaze elementi niza $a$ koji se pojavljuju tačno $3$ puta. U slučaju greške, ispisati odgovarajuću poruku. \\
\begin{miditest}
\begin{upotreba}{1}
#\naslovInt#
#\izlaz{Unesite broj elemenata niza:}\ulaz{8}#
#\izlaz{Unesite elemente niza a:}#
#\ulaz{4 11 4 6 8 4 6 6}#
#\izlaz{Elementi niza b: 4 6}#
\end{upotreba}
\end{miditest}
\begin{miditest}
\begin{upotreba}{2}
#\naslovInt#
#\izlaz{Unesite broj elemenata niza:}\ulaz{13}#
#\izlaz{Unesite elemente niza a:}#
#\ulaz{-8 26 7 2 1 1 7 2 2 2 7 5 1}#
#\izlaz{Elementi niza b: 7 1}#
\end{upotreba}
\end{miditest}
\begin{miditest}
\begin{upotreba}{3}
#\naslovInt#
#\izlaz{Unesite broj elemenata niza:}\ulaz{2}#
#\izlaz{Unesite elemente niza a:}#
#\ulaz{9 5}#
#\izlaz{Elementi niza b: }#
\end{upotreba}
\end{miditest}

\linkresenje{p.elementi_3_pojavljivanja}
\end{Exercise}
\begin{Answer}[ref=p.elementi_3_pojavljivanja]
\includecode{resenja/2_PredstavljanjePodataka/2.1_Nizovi/elementi_3_pojavljivanja.c}
\end{Answer}

%TODO: resiti
\begin{Exercise}[label=vp.bez_resenja_1] 
Napisati funkciju \kckod{int svojstvo(int a[], int na, int b[], int nb)} koja proverava da li niz $a$ dužine $na$ sadrži barem dva broja koja se pojavljuju u nizu $b$ dužine $nb$. Funkcija vraća vrednost $1$ ako je svojstvo ispunjeno ili $0$ ako nije.  Napisati i program koji sa standardnog ulaza, redom, učitava dimenzije i elemente nizova $a$ i $b$ i ispisuje rezultat rada funkcije. Pretpostaviti da su dimenzije nizova manje od $100$. U slučaju greške ispisati odgovarajuću poruku. \\
\begin{miditest}
\begin{upotreba}{1}
#\naslovInt#
#\izlaz{Unesite broj elemenata niza a:}#
#\ulaz{5}#
#\izlaz{Unesite elemente niza a:}#
#\ulaz{5 8 7 -2 6}#
#\izlaz{Unesite broj elemenata niza b:}#
#\ulaz{6}#
#\izlaz{Unesite elemente niza b:}#
#\ulaz{11 -11 7 -7 6}#
#\izlaz{Svojstvo je ispunjeno.}#
\end{upotreba}
\end{miditest}
%\linkresenje{vp.bez_resenja_1}
\end{Exercise}
\begin{Answer}[ref=vp.bez_resenja_1]
%\includecode{resenja/2_PredstavljanjePodataka/2.1_Nizovi/...c}
\end{Answer}

%TODO: doterati resenje
\begin{Exercise}[label=p.unija_presek_razlika] 
 Sa standardnog ulaza se, redom, učitavaju dimenzije i elementi dvaju nizova $a$ i $b$. Napisati program koji određuje i ispisuje njihovu uniju, presek i razliku (redosled prikaza elemenata nije bitan). Pretpostaviti da će nizovi imati manje od $100$ elemenata. \\
\begin{miditest}
\begin{upotreba}{1}
#\naslovInt#
#\izlaz{Unesite broj elemenata niza a:}\ulaz{5}#
#\izlaz{Unesite elemente niza a:}\ulaz{2 8 1 5 2}#
#\izlaz{Unesite broj elemenata niza b:}\ulaz{3}#
#\izlaz{Unesite elemente niza b:}\ulaz{5 7 8}#
#\izlaz{Unija: 2 8 1 5 2 5 7 8}#
#\izlaz{Presek: 5}#
#\izlaz{Razlika: 2 1 2}#
\end{upotreba}
\end{miditest}
\begin{miditest}
\begin{upotreba}{2}
#\naslovInt#
#\izlaz{Unesite broj elemenata niza a:}\ulaz{3}#
#\izlaz{Unesite elemente niza a:}\ulaz{11 4 4}#
#\izlaz{Unesite broj elemenata niza b:}\ulaz{2}#
#\izlaz{Unesite elemente niza b:}\ulaz{18 9}#
#\izlaz{Unija: 11 4 4 18 9}#
#\izlaz{Presek: }#
#\izlaz{Razlika: 11 4 4}#
\end{upotreba}
\end{miditest}
\begin{miditest}
\begin{upotreba}{3}
#\naslovInt#
#\izlaz{Unesite broj elemenata niza a:}\ulaz{6}#
#\izlaz{Unesite elemente niza a:}\ulaz{12 7 9 12 5 1}#
#\izlaz{Unesite broj elemenata niza b:}\ulaz{4}#
#\izlaz{Unesite elemente niza b:}\ulaz{1 12 22 12}#
#\izlaz{Unija: 12 7 9 12 5 1 1 12 22 12}#
#\izlaz{Presek: 12 12 1}#
#\izlaz{Razlika: 7 9 5}#
\end{upotreba}
\end{miditest}

\linkresenje{p.unija_presek_razlika}
\end{Exercise}
\begin{Answer}[ref=p.unija_presek_razlika]
\includecode{resenja/2_PredstavljanjePodataka/2.1_Nizovi/unija_presek_razlika.c}
\end{Answer}

\begin{Exercise}[label=p.izbacivanje_elemenata] 
 Napisati program koji učitava broj elemenata niza (broj manji od $100$) i elemente niza, a zatim formira i ispisuje niz koji se dobija izbacivanjem svih neparnih elemenata niza. Zadatak rešiti na dva načina: korišćenjem pomoćnog niza i transformacijom polaznog niza. U slučaju greške, ispisati odgovarajuću poruku.\\
\begin{miditest}
\begin{upotreba}{1}
#\naslovInt#
#\izlaz{Unesite broj elemenata niza:}\ulaz{4}#
#\izlaz{Unesite elemente niza:}\ulaz{8 9 15 12}#
#\izlaz{8 12}#
\end{upotreba}
\end{miditest}
\begin{miditest}
\begin{upotreba}{2}
#\naslovInt#
#\izlaz{Unesite broj elemenata niza:}\ulaz{6}#
#\izlaz{Unesite elemente niza:}\ulaz{21 5 3 22 19 188}#
#\izlaz{22 188}#
\end{upotreba}
\end{miditest}
\begin{miditest}
\begin{upotreba}{3}
#\naslovInt#
#\izlaz{Unesite broj elemenata niza:}\ulaz{4}#
#\izlaz{Unesite elemente niza:}\ulaz{133 129 121 101}#
#\izlaz{}#
\end{upotreba}
\end{miditest}

\begin{maxitest}
\begin{upotreba}{4}
#\naslovInt#
#\izlaz{Unesite broj elemenata niza:}\ulaz{8}#
#\izlaz{Unesite elemente niza:}\ulaz{15 -22 -23 13 18 46 14 -31}#
#\izlaz{-22 18 46 14}#
\end{upotreba}
\end{maxitest}
\linkresenje{p.izbacivanje_elemenata}
\end{Exercise}
\begin{Answer}[ref=p.izbacivanje_elemenata]
\includecode{resenja/2_PredstavljanjePodataka/2.1_Nizovi/izbacivanje_elemenata.c}
\end{Answer}

\begin{Exercise}[label=p.izbacivanje_prostih_elemenata] 
 Napisati program koji učitava broj elemenata niza (broj manji od $100$) i elemente niza, a zatim formira i ispisuje niz koji se dobija izbacivanjem svih elemenata koji su prosti brojevi. Zadatak rešiti na dva načina: korišćenjem pomoćnog niza i transformacijom polaznog niza. U slučaju greške, ispisati odgovarajuću poruku. \napomena{Brojeve $-1$ i $1$ smatrati prostim}. \\
\begin{miditest}
\begin{upotreba}{1}
#\naslovInt#
#\izlaz{Unesite broj elemenata niza:}\ulaz{5}#
#\izlaz{Unesite elemente niza:}\ulaz{11 5 6 48 8}#
#\izlaz{6 48 8}#
\end{upotreba}
\end{miditest}
\begin{miditest}
\begin{upotreba}{2}
#\naslovInt#
#\izlaz{Unesite broj elemenata niza:}\ulaz{4}#
#\izlaz{Unesite elemente niza:}\ulaz{11 5 19 21}#
#\izlaz{21}#
\end{upotreba}
\end{miditest}
\begin{miditest}
\begin{upotreba}{3}
#\naslovInt#
#\izlaz{Unesite broj elemenata niza:}\ulaz{5}#
#\izlaz{Unesite elemente niza:}\ulaz{12 18 9 31 7}#
#\izlaz{12 18 9}#
\end{upotreba}
\end{miditest}

\begin{miditest}
\begin{upotreba}{4}
#\naslovInt#
#\izlaz{Unesite broj elemenata niza:}\ulaz{3}#
#\izlaz{Unesite elemente niza:}\ulaz{-31 11 -19}#
#\izlaz{}#
\end{upotreba}
\end{miditest}
\begin{miditest}
\begin{upotreba}{5}
#\naslovInt#
#\izlaz{Unesite broj elemenata niza:}\ulaz{5}#
#\izlaz{Unesite elemente niza:}\ulaz{-2 15 -11 8 7}#
#\izlaz{15 8}#
\end{upotreba}
\end{miditest}
\linkresenje{p.izbacivanje_prostih_elemenata}
\end{Exercise}
\begin{Answer}[ref=p.izbacivanje_prostih_elemenata]
\includecode{resenja/2_PredstavljanjePodataka/2.1_Nizovi/izbacivanje_prostih_elemenata.c}
\end{Answer}


\begin{Exercise}[label=p.broj_manjih_od_poslednjeg] 
 Napisati funkciju \kckod{int prebrojavanje(int a[], int n)} koja izračunava broj elemenata celobrojnog niza $a$ dužine $n$ koji su manji od poslednjeg elementa niza. Napisati i program koji testira rad funkcije. Pretpostaviti da dužina niza neće biti veća od $100$. \\
\begin{miditest}
\begin{upotreba}{1}
#\naslovInt#
#\izlaz{Unesite broj elemenata niza:}\ulaz{4}#
#\izlaz{Unesite elemente niza:}\ulaz{11 2 4 9}#
#\izlaz{2}#
\end{upotreba}
\end{miditest}
\begin{miditest}
\begin{upotreba}{2}
#\naslovInt#
#\izlaz{Unesite broj elemenata niza:}\ulaz{7}#
#\izlaz{Unesite elemente niza:}\ulaz{7 2 1 14 65 2 8}#
#\izlaz{4}#
\end{upotreba}
\end{miditest}
\begin{miditest}
\begin{upotreba}{3}
#\naslovInt#
#\izlaz{Unesite broj elemenata niza:}\ulaz{5}#
#\izlaz{Unesite elemente niza:}\ulaz{25 18 29 30 14}#
#\izlaz{0}#
\end{upotreba}
\end{miditest}

\linkresenje{p.broj_manjih_od_poslednjeg}
\end{Exercise}
\begin{Answer}[ref=p.broj_manjih_od_poslednjeg]
\includecode{resenja/2_PredstavljanjePodataka/2.1_Nizovi/broj_manjih_od_poslednjeg.c}
\end{Answer}

\begin{Exercise}[label=p.broj_manjih_od_maksimuma] 
 Napisati funkciju \kckod{int prebrojavanje(int a[], int n)} koja izračunava broj parnih elemenata niza celih brojeva $a$ dužine $n$ koji prethode maksimalnom elementu niza. Napisati i program koji testira rad funkcije. Pretpostaviti da dužina niza neće biti veća od $100$. \\
\begin{miditest}
\begin{upotreba}{1}
#\naslovInt#
#\izlaz{Unesite broj elemenata niza:}\ulaz{4}#
#\izlaz{Unesite elemente niza:}\ulaz{11 2 4 9}#
#\izlaz{0}#
\end{upotreba}
\end{miditest}
\begin{miditest}
\begin{upotreba}{2}
#\naslovInt#
#\izlaz{Unesite broj elemenata niza:}\ulaz{7}#
#\izlaz{Unesite elemente niza:}\ulaz{7 2 1 14 65 2 8}#
#\izlaz{2}#
\end{upotreba}
\end{miditest}
\begin{miditest}
\begin{upotreba}{3}
#\naslovInt#
#\izlaz{Unesite broj elemenata niza:}\ulaz{5}#
#\izlaz{Unesite elemente niza:}\ulaz{25 18 29 30 14}#
#\izlaz{1}#
\end{upotreba}
\end{miditest}

\linkresenje{p.broj_manjih_od_maksimuma}
\end{Exercise}
\begin{Answer}[ref=p.broj_manjih_od_maksimuma]
\includecode{resenja/2_PredstavljanjePodataka/2.1_Nizovi/broj_manjih_od_maksimuma.c}
\end{Answer}

\begin{Exercise}[label=p.broj_cifara] 
 Napisati funkciju \kckod{int cifre(char s[], int n)} koja izračunava broj cifara u nizu karaktera $a$ dužine $n$. Napisati i program koji testira rad funkcije. Pretpostaviti da dužina niza neće biti veća od $100$. \\
\begin{miditest}
\begin{upotreba}{1}
#\naslovInt#
#\izlaz{Unesite broj elemenata niza:}\ulaz{5}#
#\izlaz{Unesite elemente niza:}#
#\ulaz{4}#
#\ulaz{+}#
#\ulaz{A}#
#\ulaz{u}#
#\ulaz{8}#
#\izlaz{Broj cifara je: 2}#
\end{upotreba}
\end{miditest}
\begin{miditest}
\begin{upotreba}{2}
#\naslovInt#
#\izlaz{Unesite broj elemenata niza:}\ulaz{7}#
#\izlaz{Unesite elemente niza:}#
#\ulaz{J}#
#\ulaz{M}#
#\ulaz{a}#
#\ulaz{5}#
#\ulaz{5}#
#\ulaz{-}#
#\ulaz{2}#
#\izlaz{Broj cifara je: 3}#
\end{upotreba}
\end{miditest}
\begin{miditest}
\begin{upotreba}{3}
#\naslovInt#
#\izlaz{Unesite broj elemenata niza:}\ulaz{3}#
#\izlaz{Unesite elemente niza:}#
#\ulaz{e}#
#\ulaz{k}#
#\ulaz{F}#
#\izlaz{Broj cifara je: 0}#
\end{upotreba}
\end{miditest}

\linkresenje{p.broj_cifara}
\end{Exercise}
\begin{Answer}[ref=p.broj_cifara]
\includecode{resenja/2_PredstavljanjePodataka/2.1_Nizovi/broj_cifara.c}
\end{Answer}

\begin{Exercise}[label=p.zbir_opsega_niza] 
 Napisati funkciju \kckod{int zbir(int a[], int n, int i, int j)} koja računa zbir elemenata niza celih brojeva $a$ dužine $n$ od pozicije $i$ do pozicije $j$. Napisati i program koji testira rad funkcije. Pretpostaviti da dužina niza neće biti veća od $100$. \\
\begin{miditest}
\begin{upotreba}{1}
#\naslovInt#
#\izlaz{Unesite broj elemenata niza:}\ulaz{5}#
#\izlaz{Unesite elemente niza:}\ulaz{11 5 6 48 8}#
#\izlaz{Unesite vrednosti za i i j:}\ulaz{0 2}#
#\izlaz{Zbir je: 22}#
\end{upotreba}
\end{miditest}
\begin{miditest}
\begin{upotreba}{2}
#\naslovInt#
#\izlaz{Unesite broj elemenata niza:}\ulaz{3}#
#\izlaz{Unesite elemente niza:}\ulaz{-2 8 1}#
#\izlaz{Unesite vrednosti za i i j:}\ulaz{8 12}#
#\izlaz{Greska: Nekorektne vrednosti granica!}#
\end{upotreba}
\end{miditest}
\begin{miditest}
\begin{upotreba}{3}
#\naslovInt#
#\izlaz{Unesite broj elemenata niza:}\ulaz{7}#
#\izlaz{Unesite elemente niza:}\ulaz{-2 5 9 11 6 -3 -4}#
#\izlaz{Unesite vrednosti za i i j:}\ulaz{2 5}#
#\izlaz{Zbir: 23}#
\end{upotreba}
\end{miditest}
  
\linkresenje{p.zbir_opsega_niza}
\end{Exercise}
\begin{Answer}[ref=p.zbir_opsega_niza]
\includecode{resenja/2_PredstavljanjePodataka/2.1_Nizovi/zbir_opsega_niza.c}
\end{Answer}

\begin{Exercise}[label=p.zbir_k_pozitivnih] 
 Napisati funkciju \kckod{float zbir\_pozitivnih(float a[], int n, int k)} koja izračunava zbir prvih $k$ pozitivnih elemenata realnog niza $a$ dužine $n$. Napisati i program koji testira rad funkcije. Pretpostaviti da dužina niza neće biti veća od $100$. \\
\begin{miditest}
\begin{upotreba}{1}
#\naslovInt#
#\izlaz{Unesite broj elemenata niza:}\ulaz{8}#
#\izlaz{Unesite elemente niza:}#
#\ulaz{2.34 1 -12.7 5.2 -8 -6.2 7 14.2}#
#\izlaz{Unesite vrednost za k:}\ulaz{3}#
#\izlaz{Zbir je: 8.54}#
\end{upotreba}
\end{miditest}
\begin{miditest}
\begin{upotreba}{2}
#\naslovInt#
#\izlaz{Unesite broj elemenata niza:}\ulaz{3}#
#\izlaz{Unesite elemente niza:}#
#\ulaz{-6.598 -8.14 -15}#
#\izlaz{Unesite vrednost za k:}\ulaz{4}#
#\izlaz{Zbir je: 0.00}#
\end{upotreba}
\end{miditest}
\begin{miditest}
\begin{upotreba}{3}
#\naslovInt#
#\izlaz{Unesite broj elemenata niza:}\ulaz{7}#
#\izlaz{Unesite elemente niza:}#
#\ulaz{-35.11 5.29 -1.98 12.1 12.2 -3.33 -4.17}#
#\izlaz{Unesite vrednost za k:}\ulaz{15}# 
#\izlaz{Zbir: 29.59}#
\end{upotreba}
\end{miditest}

\linkresenje{p.zbir_k_pozitivnih}
\end{Exercise}
\begin{Answer}[ref=p.zbir_k_pozitivnih]
\includecode{resenja/2_PredstavljanjePodataka/2.1_Nizovi/zbir_k_pozitivnih.c}
\end{Answer}

\begin{Exercise}[label=p.kvadriranje_parnih] 
 Napisati funkciju \kckod{void kvadriranje(float a[], int n)} koja kvadrira elemente realnog niza $a$ dužine $n$ koji se nalaze na parnim pozicijama. Napisati i program koji testira rad funkcije. Pretpostaviti da dužina niza neće biti veća od $100$. \\
\begin{miditest}
\begin{upotreba}{1}
#\naslovInt#
#\izlaz{Unesite broj elemenata niza:}\ulaz{8}#
#\izlaz{Unesite elemente niza:}#
#\ulaz{2.34 1 -12.7 5.2 -8 -6.2 7 14.2}#
#\izlaz{5.4756 1 161.29 5.2 64 -6.2 49 14.2}#
\end{upotreba}
\end{miditest}
\begin{miditest}
\begin{upotreba}{2}
#\naslovInt#
#\izlaz{Unesite broj elemenata niza:}\ulaz{3}#
#\izlaz{Unesite elemente niza:}#
#\ulaz{-6 -8.14 -15}#
#\izlaz{36 -8.14 225}#
\end{upotreba}
\end{miditest}
\begin{miditest}
\begin{upotreba}{3}
#\naslovInt#
#\izlaz{Unesite broj elemenata niza:}\ulaz{1}#
#\izlaz{Unesite elemente niza:}#
#\ulaz{-35.11}#
#\izlaz{1232.71}#
\end{upotreba}
\end{miditest}
\linkresenje{p.kvadriranje_parnih}
\end{Exercise}
\begin{Answer}[ref=p.kvadriranje_parnih]
\includecode{resenja/2_PredstavljanjePodataka/2.1_Nizovi/kvadriranje_parnih.c}
\end{Answer}

%TODO: resiti
\begin{Exercise}[label=vp.bez_resenja_9] 
Napisati funkciju \kckod{int blizu\_3(int a[],int n)} koja pronalazi i vraća indeks elementa niza koji je po vrednosti najbliži aritmetičkoj sredini onih elemenata niza koji su deljivi brojem $3$. Napisati i program koji testira rad funkcije. Pretpostaviti da dužina niza neće biti veća od $100$. \\
\begin{miditest}
\begin{upotreba}{1}
#\naslovInt#
#\izlaz{Unesite broj elemenata niza:}\ulaz{5}#
#\izlaz{Unesite elemente niza:}#
#\ulaz{1 2 3 4 5}#
#\izlaz{2}#
\end{upotreba}
\end{miditest}
\begin{miditest}
\begin{upotreba}{2}
#\naslovInt#
#\izlaz{Unesite broj elemenata niza:}\ulaz{5}#
#\izlaz{Unesite elemente niza:}#
#\ulaz{3 6 2 4 7}#
#\izlaz{3}#
\end{upotreba}
\end{miditest}
%\linkresenje{vp.bez_resenja_9}
\end{Exercise}
\begin{Answer}[ref=vp.bez_resenja_9]
%\includecode{resenja/2_PredstavljanjePodataka/2.1_Nizovi/bez_resenja_9.c}
\end{Answer}

%TODO: proveriti da li je u redu zadržati ovaj zadatak - preuzet je iz knjige Janicic&Maric
%TODO: resiti
\begin{Exercise}[label=pv.bez_resenja_5] 
Napisati funkcije za rad sa nizovima celih brojeva. 
\begin{description}
\item{a)} Napisati funkciju koja izbacuje poslednji element niza.
\item{b)} Napisati funkciju koja izbacuje prvi element niza. Zadatak rešiti na dva načina: čuvanjem redosleda elemenata i premeštanjem poslednjeg elementa niza na upražnjenu poziciju.
\item{c)} Napisati funkciju koja izbacuje element sa date pozicije $k$. 
\item{d)} Napisati funkciju koja izbacuje sva pojavljivanja datog elementa $x$ iz niza.
\item{e)} Napisati funkciju koja ubacuje dati element $x$ na kraj niza.
\item{f)} Napisati funkciju koja ubacuje dati element $x$ na početak niza.
\item{g)} Napisati funkciju koja ubacuje dati element $x$ na datu poziciju $k$. 
\end{description}
Napisati program koji testira rad zadatih funkcija. Sa standardnog ulaza učitati dimenziju niza (broj ne veći od $100$). U slučaju greške ispisati odgovarajuću poruku. \\
\begin{maxitest}
\begin{upotreba}{1}
#\naslovInt#
#\izlaz{Unesite dimenziju niza:}#
#\ulaz{8}#
#\ulaz{2 5 -2 16 33 19 8 11}#
#\izlaz{Niz posle izbacivanja poslednjeg elementa: 2 5 -2 16 33 5 8}#
#\izlaz{Niz nakon izbacivanja prvog elementa: 5 -2 16 33 5 8}#
#\izlaz{Unesite poziciju elementa za izbacivanje: }#
#\ulaz{3}#
#\izlaz{Niz nakon izbacivanja 3. elementa: 5 -2 16 5 8}#
#\izlaz{Unesite element cije pojavljivanje treba izbaciti: }#
#\ulaz{5}#
#\izlaz{Niz nakon izbacivanja elementa 5: -2 16 8}#
#\izlaz{Unesite element koji treba ubaciti u niz: }#
#\ulaz{19}#
#\izlaz{Niz nakon ubacivanja elementa 19 na kraj: -2 16 8 19}#
#\izlaz{Niz nakon ubacivanja elementa 19 na pocetak: 19 -2 16 8 19}#
#\izlaz{Unesite poziciju na koju treba ubaciti element: }#
#\ulaz{2}#
#\izlaz{Niz nakon ubacivanja elementa 19 na poziciju 2: 19 -2 19 16 8 19}#
\end{upotreba}
\end{maxitest}
%\linkresenje{pv.bez_resenja_5}
\end{Exercise}
\begin{Answer}[ref=pv.bez_resenja_5]
%\includecode{resenja/2_PredstavljanjePodataka/2.1_Nizovi/bez_resenja_5.c}
\end{Answer}


%TODO: proveriti da li je u redu zadržati ovaj zadatak - preuzet je iz knjige Janicic&Maric
%TODO: resiti i dodati test primere
\begin{Exercise}[label=pv.bez_resenja_6] 
Napisati funkcije za rad sa nizovima celih brojeva. 
\begin{description}
\item{a)} Napisati funkciju koja određuje dužinu najduže serije jednakih uzastopnih elemenata
u datom nizu brojeva.
\item{b)} Napisati funkciju koja određuje dužinu najvećeg neopadajućeg podniza datog niza celih
  brojeva. 
\item{c)} Napisati funkciju koja određuje da li se jedan niz javlja kao podniz uzastopnih elemenata drugog niza.
\item{d)} Napisati funkciju koja određuje da li se jedan niz javlja kao podniz elemenata drugog niza (elementi ne moraju da budu uzastopni, ali je redosled pojavljivanja isti).  
\item{e)} Napisati funkciju koja izbacuje višestruka pojavljivanja elemenata iz datog niza
  brojeva. Zadatak rešiti na dva načina: zadržavnjem prvog pojavljivanje elementa i zadržavanjem poslednjeg pojavljivanje elementa.  
\end{description}
%\linkresenje{pv.bez_resenja_6}
\end{Exercise}
\begin{Answer}[ref=pv.bez_resenja_6]
%\includecode{resenja/2_PredstavljanjePodataka/pv.bez_resenja_6.c}
\end{Answer}


%TODO resiti
\begin{Exercise}[label=vp.bez_resenja_2] 
Napisati funkciju koja iz zadatog niza izbacuje sve elemente koji su deljivi svojim indeksom. Niz reorganizovati tako da nema \emph{rupa} koje su nastale izbacivanjem elemenata. Povratna vrednost funkcije je nova dimenzija niza. Napisati i program koji učitava dimenziju niza (broj manji od $100$) i njegove elemente, a zatim ispisuje niz dobijen nakon poziva funkcije. U slučaju greške ispisati odgovarajuću poruku. \napomena{Element na nultoj poziciji niza zadržati jer nije dozvoljeno deljenje nulom.}
\begin{miditest}
\begin{upotreba}{1}
#\naslovInt#
#\izlaz{Unesite broj elemenata niza:}\ulaz{10}#
#\izlaz{Unesite elemente niza:}#
#\ulaz{4 2 1 6 7 10 8 2 16 27}#
#\izlaz{4 1 7 8 2 16}#
\end{upotreba}
\end{miditest}
%\linkresenje{vp.bez_resenja_2}
\end{Exercise}
\begin{Answer}[ref=vp.bez_resenja_2]
%\includecode{resenja/2_PredstavljanjePodataka/2.1_Nizovi/.....c}
\end{Answer}


% Mislim da ovaj zadatak mozemo izbaciti - ista funkcija se ocekuje u zadatku sa permutacijama

%\begin{Exercise}[label=p2.1_] 
%Napisati funkciju \kckod{void brojanje(int a[], int brojac[], int n)}
%čiji su argumenti $a$ i $brojac$ celobrojni nizovi
%dimenzije $n$. Vrednosti elemenata niza $a$ su između $0$ i $n-1$.
%Funkcija izračunava elemente niza $brojac$ tako da je
%$brojac[i]$ jednak broju pojavljivanja broja $i$ u nizu
%$a$. Program testirati učitavanjem dimenzije (broj ne veći od $100$) i elemenata niza $a$ i ispisom rezultujućeg niza $brojac$. 
%\linkresenje{p2.1_}
%\end{Exercise}
%\begin{Answer}[ref=p2.1_]
%\includecode{resenja/2_PredstavljanjePodataka/2.1_Nizovi/1_10.c}
%\end{Answer}


%TODO: resiti
\begin{Exercise}[label=vp.bez_resenja_4] 
Za celobrojni niz $a$ dimenzije $n$ kažemo da je \textit{permutacija} ako sadrži sve brojeve od $1$ do $n$.
\begin{description}
\item {a)} Napisati funkciju \kckod{void brojanje(int a[], int b[], int n)} koja na osnovu celobrojnog niza $a$ dimenzije $n$ formira niz $b$ tako što $i$-ti element niza $b$ odgovara broju pojavljivanja vrednosti $i$ u nizu $a$. \\
\item{b)} Napisati funkciju \kckod{int permutacija(int a[], int n)} koja proverava da li je zadati niz permutacija. Funkcija vraća vrednost $1$ ako je svojstvo ispunjeno, odnosno $0$ ako nije. \uputstvo{Koristiti funkciju $brojanje$ iz tačke a).} \\
\end{description}
Napisati program koji sa standardnog ulaza učitava dimenziju niza (broj manji od $100$) i elemente niza i ispisuje da li je uneti niz permutacija ili ne.  \\
\begin{miditest}
\begin{upotreba}{1}
#\naslovInt#
#\izlaz{Unesite broj elemenata niza:}\ulaz{5}#
#\izlaz{Unesite elemente niza:}#
#\ulaz{1 5 4 3 2}#
#\izlaz{Uneti niz je permutacija.}#
\end{upotreba}
\end{miditest}
\begin{miditest}
\begin{upotreba}{2}
#\naslovInt#
#\izlaz{Unesite broj elemenata niza:}\ulaz{6}#
#\izlaz{Unesite elemente niza:}#
#\ulaz{2 3 3 1 1 5}#
#\izlaz{Uneti niz nije permutacija.}#
\end{upotreba}
\end{miditest}
\linkresenje{vp.bez_resenja_4}
\end{Exercise}
\begin{Answer}[ref=vp.bez_resenja_4]
%\includecode{resenja/2_PredstavljanjePodataka/2.1_Nizovi/....c}
\end{Answer}

\section{Rešenja}
\shipoutAnswer
