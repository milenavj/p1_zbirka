\chapter{Napredni tipovi podataka}

\section{Nizovi}


\begin{Exercise}[label=v.parni_elementi] 
Napisati program koji učitava dimenziju niza, elemente niza i zatim ispisuje:
\skrati{1}
\begin{enumerate}
\setlength\itemsep{0em}
\item elemente niza koji se nalaze na parnim pozicijama.
\item parne elemente niza.
\end{enumerate}
Maksimalni broj elemenata niza je $100$.
U slučaju neispravnog unosa, ispisati odgovarajuću poruku o grešci. 

\skrati{1}
\begin{miditest}
\begin{upotreba}{1}
#\naslovInt#
#\izlaz{Unesite dimenziju niza:}#
#\ulaz{6}#
#\izlaz{Unesite elemente niza:}#
#\ulaz{1 8 2 -5 -13 75}#
#\izlaz{Elementi niza na parnim pozicijama:}#
#\izlaz{1 2 -13}#
#\izlaz{Parni elementi niza:}#
#\izlaz{8 2}#
\end{upotreba}
\end{miditest}
\begin{miditest}
\begin{upotreba}{2}
#\naslovInt#
#\izlaz{Unesite dimenziju niza:}#
#\ulaz{3}#
#\izlaz{Unesite elemente niza:}#
#\ulaz{11 81 -63}#
#\izlaz{Elementi niza na parnim pozicijama:}#
#\izlaz{11 -63}#
#\izlaz{Parni elementi niza:}#
#\izlaz{}#
\end{upotreba}
\end{miditest}

\skrati{1}
\begin{miditest}
\begin{upotreba}{3}
#\naslovInt#
#\izlaz{Unesite dimenziju niza: }#
#\ulaz{-4}#
#\izlaz{Greska: neispravan unos.}#
\end{upotreba}
\end{miditest}
\linkresenje{v.parni_elementi}
\end{Exercise}
\ifresenja
\begin{Answer}[ref=v.parni_elementi]
\includecode{resenja/2_NapredniTipoviPodataka/2.1_Nizovi/nizovi_01.c}
\end{Answer}
\fi


\begin{Exercise}[label=p.kvadriranje_elemenata]
Napisati program koji učitava dimenziju niza, elemente niza i zatim
menja uneti niz tako što kvadrira sve negativne elemente niza.
Maksimalni broj elemenata niza je $100$.
U slučaju neispravnog unosa, ispisati odgovarajuću poruku o grešci. 

\skrati{1}
\begin{miditest}
\begin{upotreba}{1}
#\naslovInt#
#\izlaz{Unesite dimenziju niza:}\ulaz{6}#
#\izlaz{Unesite elemente niza:}#
#\ulaz{12.34 -6 1 8 32.4 -16}#
#\izlaz{Rezultujuci niz:}#
#\izlaz{12.34 36 1 8 32.4 256}#
\end{upotreba}
\end{miditest}
\begin{miditest}
\begin{upotreba}{2}
#\naslovInt#
#\izlaz{Unesite dimenziju niza:}\ulaz{9}#
#\izlaz{Unesite elemente niza:}#
#\ulaz{-8.25 6 17 2 -1.5 1 -7 2.65 -125.2}#
#\izlaz{Rezultujuci niz:}#
#\izlaz{68.0625 6 17 2 2.25 1 49 2.65 15675.04}#
\end{upotreba}
\end{miditest}

\skrati{1}
\begin{miditest}
\begin{upotreba}{3}
#\naslovInt#
#\izlaz{Unesite dimenziju niza:}\ulaz{4}#
#\izlaz{Unesite elemente niza:}#
#\ulaz{9.53 5 1 4.89}#
#\izlaz{Rezultujuci niz:}#
#\izlaz{9.53 5 1 4.89}#
\end{upotreba}
\end{miditest}
\begin{miditest}
\begin{upotreba}{4}
#\naslovInt#
#\izlaz{Unesite dimenziju niza:}\ulaz{104}#
#\izlaz{Greska: neispravan unos.}#
\end{upotreba}
\end{miditest}
\linkresenje{p.kvadriranje_elemenata}
\end{Exercise}

\ifresenja
\begin{Answer}[ref=p.kvadriranje_elemenata]
\includecode{resenja/2_NapredniTipoviPodataka/2.1_Nizovi/nizovi_02.c}
\end{Answer}
\fi


\skrati{1}
\begin{Exercise}[label=v.skalarni_proizvod] 
Ako su $a = (a_1, \ldots, a_n)$ i $b = (b_1,\ldots, b_n)$ vektori
dimenzije $n$, njihov skalarni proizvod se definiše kao $a \cdot b = a_1\cdot b_1 +
\ldots + a_n\cdot b_n$. Napisati program koji računa skalarni proizvod
dva vektora. Vektori se zadaju kao celobrojni nizovi sa najviše $100$
elemenata. Program učitava dimenziju i elemente nizova, a na izlaz
ispisuje vrednost skalarnog proizvoda.
U slučaju neispravnog unosa, ispisati odgovarajuću poruku o grešci. 

\skrati{1}
\begin{miditest}
\begin{upotreba}{1}
#\naslovInt#
#\izlaz{Unesite dimenziju vektora:}\ulaz{5}#
#\izlaz{Unesite koordinate vektora a:}#
#\ulaz{8 -2 0 2 4}#
#\izlaz{Unesite koordinate vektora b:}#
#\ulaz{35 12 5 -6 -1}#
#\izlaz{Skalarni proizvod: 240}# 
\end{upotreba}
\end{miditest}
\begin{miditest}
\begin{upotreba}{2}
#\naslovInt#
#\izlaz{Unesite dimenziju vektora:}\ulaz{3}#
#\izlaz{Unesite koordinate vektora a:}#
#\ulaz{-1 0 1}#
#\izlaz{Unesite koordinate vektora b:}#
#\ulaz{5 5 5}#
#\izlaz{Skalarni proizvod: 0}# 
\end{upotreba}
\end{miditest}

\skrati{1}
\begin{miditest}
\begin{upotreba}{3}
#\naslovInt#
#\izlaz{Unesite dimenziju vektora:}\ulaz{0}#
#\izlaz{Greska: neispravan unos.}#
\end{upotreba}
\end{miditest}
\begin{miditest}
\begin{upotreba}{4}
#\naslovInt#
#\izlaz{Unesite dimenziju vektora:}\ulaz{1}#
#\izlaz{Unesite koordinate vektora a:}#
#\ulaz{-1}#
#\izlaz{Unesite koordinate vektora b:}#
#\ulaz{1}#
#\izlaz{Skalarni proizvod: -1}# 
\end{upotreba}
\end{miditest}
\linkresenje{v.skalarni_proizvod}
\end{Exercise}

\ifresenja
\begin{Answer}[ref=v.skalarni_proizvod]
\includecode{resenja/2_NapredniTipoviPodataka/2.1_Nizovi/nizovi_03.c}
\end{Answer}
\fi


\skrati{1}
\begin{Exercise}[label=p.pretraga_deljivih_sa_k] 
Napisati program koji učitava dimenziju niza, elemente niza, a potom i ceo broj $k$ i ispisuje indekse elemenata koji su
deljivi sa $k$. Maksimalni broj elemenata niza je $100$.
U slučaju neispravnog unosa, ispisati odgovarajuću poruku o grešci. 

\begin{miditest}
\begin{upotreba}{1}
#\naslovInt#
#\izlaz{Unesite dimenziju niza:}\ulaz{4}#
#\izlaz{Unesite elemente niza:}\ulaz{10 14 86 20}#
#\izlaz{Unesite broj k:}\ulaz{5}#
#\izlaz{Rezultat: 0 3}#
\end{upotreba}
\end{miditest}
\begin{miditest}
\begin{upotreba}{2}
#\naslovInt#
#\izlaz{Unesite dimenziju niza:}\ulaz{4}#
#\izlaz{Unesite elemente niza:}\ulaz{6 14 8 9}#
#\izlaz{Unesite broj k:}\ulaz{5}#
#\izlaz{U nizu nema elemenata koji su}#
#\izlaz{deljivi brojem 5.}#
\end{upotreba}
\end{miditest}

\begin{miditest}
\begin{upotreba}{3}
#\naslovInt#
#\izlaz{Unesite dimenziju niza:}\ulaz{6}#
#\izlaz{Unesite elemente niza:}\ulaz{8 9 11 -4 8 11}#
#\izlaz{Unesite broj k:}\ulaz{2}#
#\izlaz{Rezultat: 0 3 4}#
\end{upotreba}
\end{miditest}
\begin{miditest}
\begin{upotreba}{4}
#\naslovInt#
#\izlaz{Unesite dimenziju niza:}\ulaz{6}#
#\izlaz{Unesite elemente niza:}\ulaz{1 2 3 4 5 6}#
#\izlaz{Unesite broj k:}\ulaz{0}#
#\izlaz{Greska: neispravan unos.}#
\end{upotreba}
\end{miditest}

\linkresenje{p.pretraga_deljivih_sa_k}
\end{Exercise}

\ifresenja
\begin{Answer}[ref=p.pretraga_deljivih_sa_k]
\includecode{resenja/2_NapredniTipoviPodataka/2.1_Nizovi/nizovi_04.c}
\end{Answer}
\fi


\begin{Exercise}[label=autobusi]
  Autobusi su označeni rednim brojevima (počevši od $1$) i u nizu se
  čuva vreme putovanja svakog autobusa u minutima. Međutim, zbog
  radova na putu između Požege i Užica, svi autobusi koji saobraćaju
  na tom potezu (autobusi označeni rednim brojevima od $k$ do $t$)
  saobraćaju $m$ minuta duže. 
  Napisati program koji učitava broj autobusa $n$, $n$ celih brojeva
  koji označavaju vreme putovanja tih autobusa i vrednosti $k$, $t$ i $m$
  i ispisuje vreme putovanja svih autobusa nakon unetih izmena.
  Maksimalni broj autobusa je $200$.
U slučaju neispravnog unosa, ispisati odgovarajuću poruku o grešci. 

\begin{miditest}
\begin{upotreba}{1}
#\naslovInt#
#\izlaz{Unesite broj autobusa:}\ulaz{8}#
#\izlaz{Unesite vreme putovanja:}#
#\ulaz{24 78 13 124 56 90 205 45}#
#\izlaz{Unesite vrednosti k, t i m:}#
#\ulaz{3 6 23}#
#\izlaz{Vreme putovanja nakon izmena:}#
#\izlaz{24 78 36 147 79 113 205 45}#
\end{upotreba}
\end{miditest}
\begin{miditest}
\begin{upotreba}{2}
#\naslovInt#
#\izlaz{Unesite broj autobusa:}\ulaz{8}#
#\izlaz{Unesite vreme putovanja:}#
#\ulaz{24 78 13 124 56 90 205 45}#
#\izlaz{Unesite vrednosti k, t i m:}#
#\ulaz{3 15 3}#
#\izlaz{Greska: neispravan unos.}#
\end{upotreba}
\end{miditest}
\linkresenje{autobusi}
\end{Exercise}

\ifresenja
\begin{Answer}[ref=autobusi]
\includecode{resenja/2_NapredniTipoviPodataka/2.1_Nizovi/nizovi_05.c}
\end{Answer}
\fi


\begin{Exercise}[label=v.prebrojavanje_cifara] 
Napisati program koji za učitani ceo broj ispisuje broj pojavljivanja
svake od cifara u zapisu tog broja. \uputstvo{Za evidenciju broja
  pojavljivanja svake cifre pojedinačno, koristiti niz.}

\begin{maxitest}
\begin{upotreba}{1}
#\naslovInt#
#\izlaz{Unesite ceo broj:}\ulaz{2355623}#
#\izlaz{U zapisu broja 2355623, cifra 2 se pojaviljuje 2 puta}#
#\izlaz{U zapisu broja 2355623, cifra 3 se pojaviljuje 2 puta}#
#\izlaz{U zapisu broja 2355623, cifra 5 se pojaviljuje 2 puta}#
#\izlaz{U zapisu broja 2355623, cifra 6 se pojaviljuje 1 puta}#
\end{upotreba}
\end{maxitest}

\begin{maxitest}
\begin{upotreba}{2}
#\naslovInt#
#\izlaz{Unesite ceo broj:}\ulaz{-39902}#
#\izlaz{U zapisu broja -39902, cifra 0 se pojaviljuje 1 puta}#
#\izlaz{U zapisu broja -39902, cifra 2 se pojaviljuje 1 puta}#
#\izlaz{U zapisu broja -39902, cifra 3 se pojaviljuje 1 puta}#
#\izlaz{U zapisu broja -39902, cifra 9 se pojaviljuje 2 puta}#
\end{upotreba}
\end{maxitest}
\linkresenje{v.prebrojavanje_cifara}
\end{Exercise}

\ifresenja
\begin{Answer}[ref=v.prebrojavanje_cifara]
\includecode{resenja/2_NapredniTipoviPodataka/2.1_Nizovi/nizovi_06.c}
\end{Answer}
\fi


\begin{Exercise}[label=p.niz_karaktera_obrnuto] 
 Napisati program koji učitava karaktere sve do unosa karaktera \textit{*}, a zatim ih 
 ispisuje u redosledu suprotnom od redosleda čitanja.
 Maksimalni broj karaktera je $500$.

\begin{miditest}
\begin{upotreba}{1}
#\naslovInt#
#\izlaz{Unesite karakter:}\ulaz{a}#
#\izlaz{Unesite karakter:}\ulaz{8}#
#\izlaz{Unesite karakter:}\ulaz{5}#
#\izlaz{Unesite karakter:}\ulaz{Y}#
#\izlaz{Unesite karakter:}\ulaz{I}#
#\izlaz{Unesite karakter:}\ulaz{o}#
#\izlaz{Unesite karakter:}\ulaz{?}#
#\izlaz{Unesite karakter:}\ulaz{*}#
#\izlaz{? o I Y 5 8 a}#
\end{upotreba}
\end{miditest}
\begin{miditest}
\begin{upotreba}{2}
#\naslovInt#
#\izlaz{Unesite karakter:}\ulaz{g}#
#\izlaz{Unesite karakter:}\ulaz{g}#
#\izlaz{Unesite karakter:}\ulaz{2}#
#\izlaz{Unesite karakter:}\ulaz{2}#
#\izlaz{Unesite karakter:}\ulaz{)}#
#\izlaz{Unesite karakter:}\ulaz{)}#
#\izlaz{Unesite karakter:}\ulaz{*}#
#\izlaz{) ) 2 2 g g}#
\end{upotreba}
\end{miditest}

\linkresenje{p.niz_karaktera_obrnuto}
\end{Exercise}

\ifresenja
\begin{Answer}[ref=p.niz_karaktera_obrnuto]
\includecode{resenja/2_NapredniTipoviPodataka/2.1_Nizovi/nizovi_07.c}
\end{Answer}
\fi


\begin{Exercise}[label=v.prebrojavanje] 
Napisati program koji učitava karaktere sve do
kraja ulaza, a potom i izračunava koliko se puta u unetom tekstu pojavila svaka
od cifara, svako malo slovo i svako veliko slovo. Ispisati broj
pojavljivanja samo za karaktere koji su se u unetom tekstu pojavili
barem jednom. 
\uputstvo{Za evidenciju broja pojavljivanja cifara,
  malih i velih slova korisiti pojedinačne nizove.}

\begin{miditest}
\begin{upotreba}{1}
#\naslovInt#
#\izlaz{Unesite tekst:}#
#\ulaz{Mis je dobio grip.}#
#\izlaz{Karakter b se pojavljuje 1 puta}#
#\izlaz{Karakter d se pojavljuje 1 puta}#
#\izlaz{Karakter e se pojavljuje 1 puta}#
#\izlaz{Karakter g se pojavljuje 1 puta}#
#\izlaz{Karakter i se pojavljuje 3 puta}#
#\izlaz{Karakter j se pojavljuje 1 puta}#
#\izlaz{Karakter o se pojavljuje 2 puta}#
#\izlaz{Karakter p se pojavljuje 1 puta}#
#\izlaz{Karakter r se pojavljuje 1 puta}#
#\izlaz{Karakter s se pojavljuje 1 puta}#
#\izlaz{Karakter M se pojavljuje 1 puta}#
\end{upotreba}
\end{miditest}
\begin{miditest}
\begin{upotreba}{2}
#\naslovInt#
#\izlaz{Unesite tekst:}#
#\ulaz{Programiranje 1 je zanimljivo!!}#
#\izlaz{Karakter 1 se pojavljuje 1 puta}#
#\izlaz{Karakter a se pojavljuje 3 puta}#
#\izlaz{Karakter e se pojavljuje 2 puta}#
#\izlaz{Karakter g se pojavljuje 1 puta}#
#\izlaz{Karakter i se pojavljuje 3 puta}#
#\izlaz{Karakter j se pojavljuje 3 puta}#
#\izlaz{Karakter l se pojavljuje 1 puta}#
#\izlaz{Karakter m se pojavljuje 2 puta}#
#\izlaz{Karakter n se pojavljuje 2 puta}#
#\izlaz{Karakter o se pojavljuje 2 puta}#
#\izlaz{Karakter r se pojavljuje 3 puta}#
#\izlaz{Karakter v se pojavljuje 1 puta}#
#\izlaz{Karakter z se pojavljuje 1 puta}#
#\izlaz{Karakter P se pojavljuje 1 puta}#
\end{upotreba}
\end{miditest}

\linkresenje{v.prebrojavanje}
\end{Exercise}

\ifresenja
\begin{Answer}[ref=v.prebrojavanje]
\includecode{resenja/2_NapredniTipoviPodataka/2.1_Nizovi/nizovi_08.c}
\sstrana
\end{Answer}
\fi


\begin{Exercise}[label=brojanje_slova] 
Napisati program koji učitava jednu liniju teksta i
ispisuje koliko puta se pojavilo svako od slova
engleske abecede u unetom tekstu. Ne praviti razliku između malih i
velikih slova.

\begin{maxitest}
\begin{upotreba}{1}
#\naslovInt#
#\ulaz{Tasi, tasi, TaNaNa i SVILENA marama.....}#
#\izlaz{ a:9  b:0  c:0  d:0  e:1  f:0  g:0  h:0  i:4  j:0  k:0  l:1  m:2}#
#\izlaz{ n:3  o:0  p:0  q:0  r:1  s:3  t:3  u:0  v:1  w:0  x:0  y:0  z:0}#
\end{upotreba}
\end{maxitest}

\begin{maxitest}
\begin{upotreba}{2}
#\naslovInt#
#\ulaz{Mihailo Petrovic Alas (6 maj 1868 - 8 jun 1943)}#
#\izlaz{ a:4  b:0  c:1  d:0  e:1  f:0  g:0  h:1  i:3  j:2  k:0  l:2  m:2}#
#\izlaz{ n:1  o:2  p:1  q:0  r:1  s:1  t:1  u:1  v:1  w:0  x:0  y:0  z:0}#
\end{upotreba}
\end{maxitest}

\begin{maxitest}
\begin{upotreba}{3}
#\naslovInt#
#\ulaz{Alan Matison Tjuring (London, 23. jun 1912 - Cesir, 7. jun 1954) }#
#\izlaz{ a:3  b:0  c:1  d:1  e:1  f:0  g:1  h:0  i:3  j:3  k:0  l:2  m:1}#
#\izlaz{ n:7  o:3  p:0  q:0  r:2  s:2  t:2  u:3  v:0  w:0  x:0  y:0  z:0}#
\end{upotreba}
\end{maxitest}
\linkresenje{brojanje_slova}
\end{Exercise}

\ifresenja
\begin{Answer}[ref=brojanje_slova]
\includecode{resenja/2_NapredniTipoviPodataka/2.1_Nizovi/nizovi_09.c}
\end{Answer}
\fi


\skrati{2}
\begin{Exercise}[label=v.nizovi_funkcije_intro] 
Takmičari na Beogradskom maratonu su označeni rednim
brojevima počevši od 0. Vremena za koja su takmičari istrčali maraton izražena u minutima se zadaju nizom celih brojeva u kojem indeks elementa niza označava redni broj takmičara. Napisati sledeće funkcije za obradu navedenih podataka:
\begin{enumerate}
\setlength\itemsep{0em}
\item \kckod{void ucitaj(int a[], int n)} koja
  učitava elemente niza $a$ dimenzije $n$.
\item \kckod{void ispisi(int a[], int n)} koja
  ispisuje elemente niza $a$ dimenzije $n$.
\item \kckod{int suma(int a[], int n)} koja računa i
  vraća ukupno vreme trčanja svih takmičara.
\item \kckod{float prosek(int a[], int n)} koja
  računa i vraća prosečno vreme (aritmetičku sredinu) trčanja
  takmičara.
\item \kckod{int maksimum(int a[], int n)} koja
  izračunava i vraća najduže vreme trčanja takmičara.
\item \kckod{int pozicija\_minimum(int a[], int n)}
  koja vraća redni broj pobednika Beogradskog maratona, tj. onog
  takmičara koji je najkraće trčao. U slučaju da ima više takvih
  takmičara, vratiti onog sa najmanjim rednim brojem.
\end{enumerate}
Napisati program koji učitava podatke o rezultatima takmičara na maratonu i ispisuje
učitane podatke, ukupno, prosečno i maksimalno vreme trčanja, kao i redni broj pobednika
maratona. 
Maksimalni broj takmičara je $1000$.
U slučaju neispravnog unosa, ispisati odgovarajuću poruku o grešci. 

\skrati{2}
\begin{miditest}
\begin{upotreba}{1}
#\naslovInt#
#\izlaz{Unesite dimenziju niza:}#
#\ulaz{5}#
#\izlaz{Unesite elemente niza:}\ulaz{140 126 170 220 130}#
#\izlaz{Vreme trcanja takmicara: 140 126 170 220 130}#
#\izlaz{Ukupno vreme: 786}#
#\izlaz{Prosecno vreme trcanja: 157.20}#
#\izlaz{Maksimalno vreme trcanja: 220}#
#\izlaz{Indeks pobednika: 1}#
\end{upotreba}
\end{miditest}
\begin{miditest}
\begin{upotreba}{2}
#\naslovInt#
#\izlaz{Unesite dimenziju niza:}#
#\ulaz{-5}#
#\izlaz{Greska: neispravan unos.}#
\end{upotreba}
\end{miditest}
\linkresenje{v.nizovi_funkcije_intro}
\end{Exercise}

\ifresenja
\begin{Answer}[ref=v.nizovi_funkcije_intro]
\includecode{resenja/2_NapredniTipoviPodataka/2.1_Nizovi/nizovi_10.c}
\end{Answer}
\fi


\skrati{2}
\begin{Exercise}[label=p.broj_manjih_od_poslednjeg] 
 Napisati funkciju koja izračunava broj elemenata celobrojnog niza koji su
 manji od poslednjeg elementa niza. 
 Napisati program koji učitava dimenziju niza i elemente niza, a zatim ispisuje broj
 elemenata koji zadovoljavaju pomenuti uslov.
 Maksimalni broj elemenata niza je $100$.
 U slučaju neispravnog unosa, ispisati odgovarajuću poruku o grešci. 

\skrati{2}
\begin{miditest}
\begin{upotreba}{1}
#\naslovInt#
#\izlaz{Unesite dimenziju niza:}\ulaz{4}#
#\izlaz{Unesite elemente niza:}\ulaz{11 2 4 9}#
#\izlaz{Rezultat: 2}#
\end{upotreba}
\end{miditest}
\begin{miditest}
\begin{upotreba}{2}
#\naslovInt#
#\izlaz{Unesite dimenziju niza:}\ulaz{7}#
#\izlaz{Unesite elemente niza:}\ulaz{7 2 1 14 65 2 8}#
#\izlaz{Rezultat: 4}#
\end{upotreba}
\end{miditest}

\begin{miditest}
\begin{upotreba}{3}
#\naslovInt#
#\izlaz{Unesite dimenziju niza:}\ulaz{5}#
#\izlaz{Unesite elemente niza:}\ulaz{25 18 29 30 14}#
#\izlaz{Rezultat: 0}#
\end{upotreba}
\end{miditest}
\begin{miditest}
\begin{upotreba}{4}
#\naslovInt#
#\izlaz{Unesite dimenziju niza:}\ulaz{-45}#
#\izlaz{Greska: neispravan unos.}#
\end{upotreba}
\end{miditest}

\linkresenje{p.broj_manjih_od_poslednjeg}
\end{Exercise}

\ifresenja
\begin{Answer}[ref=p.broj_manjih_od_poslednjeg]
Pogledajte zadatak \ref{v.nizovi_funkcije_intro}.
\end{Answer}
\fi


\begin{Exercise}[label=p.broj_manjih_od_maksimuma] 
 Napisati funkciju koja izračunava broj parnih elemenata celobrojnog niza
 koji prethode maksimalnom elementu niza.
 Napisati program koji učitava dimenziju niza i elemente niza, a zatim ispisuje broj
 elemenata koji prethode maksimalnom elementu. Maksimalni broj elemenata niza je $100$.
U slučaju neispravnog unosa, ispisati odgovarajuću poruku o grešci. 

\begin{miditest}
\begin{upotreba}{1}
#\naslovInt#
#\izlaz{Unesite dimenziju niza:}\ulaz{4}#
#\izlaz{Unesite elemente niza:}\ulaz{11 2 4 9}#
#\izlaz{Rezultat: 0}#
\end{upotreba}
\end{miditest}
\begin{miditest}
\begin{upotreba}{2}
#\naslovInt#
#\izlaz{Unesite dimenziju niza:}\ulaz{7}#
#\izlaz{Unesite elemente niza:}\ulaz{7 2 1 14 65 2 8}#
#\izlaz{Rezultat: 2}#
\end{upotreba}
\end{miditest}

\begin{miditest}
\begin{upotreba}{3}
#\naslovInt#
#\izlaz{Unesite dimenziju niza:}\ulaz{5}#
#\izlaz{Unesite elemente niza:}\ulaz{25 18 29 30 14}#
#\izlaz{Rezultat: 1}#
\end{upotreba}
\end{miditest}
\begin{miditest}
\begin{upotreba}{4}
#\naslovInt#
#\izlaz{Unesite dimenziju niza:}\ulaz{105}#
#\izlaz{Greska: neispravan unos.}#
\end{upotreba}
\end{miditest}

\linkresenje{p.broj_manjih_od_maksimuma}
\end{Exercise}

\ifresenja
\begin{Answer}[ref=p.broj_manjih_od_maksimuma]
\includecode{resenja/2_NapredniTipoviPodataka/2.1_Nizovi/nizovi_12.c}
\end{Answer}
\fi


\begin{Exercise}[label=p.zbir_opsega_niza] 
 Napisati funkciju \kckod{int zbir(int a[], int n, int i, int j)} koja
 računa zbir elemenata niza celih brojeva $a$ dužine $n$ od pozicije
 $i$ do pozicije $j$.
 Napisati program koji učitava dimenziju niza, elemente niza i vrednosti $i$ i $j$ 
 i zatim ispisuje zbir u datom opsegu.
 Maksimalni broj elemenata niza je $100$.
 U slučaju neispravnog unosa, ispisati odgovarajuću poruku o grešci. 
 
\begin{miditest}
\begin{upotreba}{1}
#\naslovInt#
#\izlaz{Unesite dimenziju niza:}\ulaz{5}#
#\izlaz{Unesite elemente niza:}\ulaz{11 5 6 48 8}#
#\izlaz{Unesite vrednosti za i i j:}\ulaz{0 2}#
#\izlaz{Zbir je: 22}#
\end{upotreba}
\end{miditest}
\begin{miditest}
\begin{upotreba}{2}
#\naslovInt#
#\izlaz{Unesite dimenziju niza:}\ulaz{3}#
#\izlaz{Unesite elemente niza:}\ulaz{-2 8 1}#
#\izlaz{Unesite vrednosti za i i j:}\ulaz{1 12}#
#\izlaz{Greska: neispravan unos.}#
\end{upotreba}
\end{miditest}

\begin{miditest}
\begin{upotreba}{3}
#\naslovInt#
#\izlaz{Unesite dimenziju niza:}\ulaz{7}#
#\izlaz{Unesite elemente niza:}\ulaz{-2 5 9 11 6 -3 -4}#
#\izlaz{Unesite vrednosti za i i j:}\ulaz{2 5}#
#\izlaz{Zbir je: 23}#
\end{upotreba}
\end{miditest}
\begin{miditest}
\begin{upotreba}{4}
#\naslovInt#
#\izlaz{Unesite dimenziju niza:}\ulaz{4}#
#\izlaz{Unesite elemente niza:}\ulaz{9 5 7 6}#
#\izlaz{Unesite vrednosti za i i j:}\ulaz{2 2}#
#\izlaz{Zbir je: 7}#
\end{upotreba}
\end{miditest}
  
\linkresenje{p.zbir_opsega_niza}
\end{Exercise}

\ifresenja
\begin{Answer}[ref=p.zbir_opsega_niza]
\includecode{resenja/2_NapredniTipoviPodataka/2.1_Nizovi/nizovi_13.c}
\end{Answer}
\fi


\begin{Exercise}[label=p.zbir_k_pozitivnih] 
 Napisati funkciju \kckod{float zbir\_pozitivnih(float a[], int n, int
   k)} koja izračunava zbir prvih $k$ pozitivnih elemenata realnog
 niza $a$ dužine $n$. 
 Napisati program koji učitava dimenziju niza, elemente niza i broj $k$, a 
 zatim ispisuje zbir prvih $k$ pozitivnih elemenata niza.
 Maksimalni broj elemenata niza je $100$.
 U slučaju neispravnog unosa, ispisati odgovarajuću poruku o grešci. 

\begin{miditest}
\begin{upotreba}{1}
#\naslovInt#
#\izlaz{Unesite dimenziju niza:}\ulaz{8}#
#\izlaz{Unesite elemente niza:}#
#\ulaz{2.34 1 -12.7 5.2 -8 -6.2 7 14.2}#
#\izlaz{Unesite vrednost k:}\ulaz{3}#
#\izlaz{Zbir je: 8.54}#
\end{upotreba}
\end{miditest}
\begin{miditest}
\begin{upotreba}{2}
#\naslovInt#
#\izlaz{Unesite dimenziju niza:}\ulaz{3}#
#\izlaz{Unesite elemente niza:}#
#\ulaz{-6.598 -8.14 -15}#
#\izlaz{Unesite vrednost k:}\ulaz{4}#
#\izlaz{Zbir je: 0.00}#
\end{upotreba}
\end{miditest}

\begin{miditest}
\begin{upotreba}{3}
#\naslovInt#
#\izlaz{Unesite dimenziju niza:}\ulaz{7}#
#\izlaz{Unesite elemente niza:}#
#\ulaz{-35.11 5.29 -1.98 12.1 12.2 -3.33 -4.17}#
#\izlaz{Unesite vrednost k:}\ulaz{15}# 
#\izlaz{Zbir je: 29.59}#
\end{upotreba}
\end{miditest}
\begin{miditest}
\begin{upotreba}{4}
#\naslovInt#
#\izlaz{Unesite dimenziju niza:}\ulaz{3}#
#\izlaz{Unesite elemente niza:}#
#\ulaz{-0.11 5.29 -4.17}#
#\izlaz{Unesite vrednost k:}\ulaz{-15}# 
#\izlaz{Greska: neispravan unos.}#
\end{upotreba}
\end{miditest}

\linkresenje{p.zbir_k_pozitivnih}
\end{Exercise}

\ifresenja
\begin{Answer}[ref=p.zbir_k_pozitivnih]
\includecode{resenja/2_NapredniTipoviPodataka/2.1_Nizovi/nizovi_14.c}
\end{Answer}
\fi


\begin{Exercise}[label=p.razmena_min_max]
  Napisati funkciju koja menja niz tako što razmenjuje mesta najmanjem i najvećem elementu niza.
  Ukoliko se neki od ovih elemenata javlja više puta, uzeti u obzir prvo pojavljivanje.
  Napisati program koji učitava dimenziju niza, elemente niza, a zatim ispisuje
  izmenjeni niz. 
  Maksimalni broj elemenata niza je $100$. 
  U slučaju neispravnog unosa, ispisati odgovarajuću poruku o grešci. 

\begin{miditest}
\begin{upotreba}{1}
#\naslovInt#
#\izlaz{Unesite dimenziju niza:}\ulaz{5}#
#\izlaz{Unesite elemente niza:}\ulaz{8 -2 11 19 4}#
#\izlaz{Rezultujuci niz:}#
#\izlaz{8 19 11 -2 4}#
\end{upotreba}
\end{miditest}
\begin{miditest}
\begin{upotreba}{2}
#\naslovInt#
#\izlaz{Unesite dimenziju niza:}\ulaz{10}#
#\izlaz{Unesite elemente niza:}#
#\ulaz{46 -2 51 8 -5 66 2 8 3 14}#
#\izlaz{Rezultujuci niz:}#
#\izlaz{46 -2 51 8 66 -5 2 8 3 14}#
\end{upotreba}
\end{miditest}

\begin{miditest}
\begin{upotreba}{3}
#\naslovInt#
#\izlaz{Unesite dimenziju niza:}\ulaz{145}#
#\izlaz{Greska: neispravan unos.}#
\end{upotreba}
\end{miditest}
\linkresenje{p.razmena_min_max}
\end{Exercise}

\ifresenja
\begin{Answer}[ref=p.razmena_min_max]
\includecode{resenja/2_NapredniTipoviPodataka/2.1_Nizovi/nizovi_15.c}
\end{Answer}
\fi


\begin{Exercise}[label=v.nizovi_funkcije_razno] 
Napisati program koji vrši pretragu niza nadmorskih visina.
\begin{enumerate}
\item Napisati funkciju koja proverava da li niz sadrži zadati broj
  $m$. Povratna vrednost funkcije je $1$ ako je
  vrednost sadržana u nizu ili $0$ ako nije.
\item Napisati funkciju koja vraća vrednost prve pozicije na kojoj se
  nalazi element koji ima vrednost $m$ ili $-1$ ukoliko
  element nije u nizu.
\item Napisati funkciju koja vraća vrednost poslednje pozicije na
  kojoj se nalazi element koji ima vrednost $m$ ili $-1$
  ukoliko element nije u nizu.
\end{enumerate}
Program učitava podatke o nadmorskim visinama i ceo broj $m$, a zatim ispisuje da li 
u nizu postoji podatak o unetoj nadmorskoj visini. Ukoliko postoji, ispisuje i poziciju
prvog i poslednjeg pojavljivanja vrednosti $m$ u nizu. Pozicije se broje od $0$.
Maksimalni broj elemenata niza je $100$.
U slučaju neispravnog unosa, ispisati odgovarajuću poruku o grešci. 

\begin{miditest}
\begin{upotreba}{1}
#\naslovInt#
#\izlaz{Unesite dimenziju niza:}#
#\ulaz{7}#
#\izlaz{Unesite podatke:}#
#\ulaz{800 1100 -200 1400 -200 1100 800}#
#\izlaz{Unesite vrednost m: }#
#\ulaz{1100}#
#\izlaz{Nadmorska visina 1100 se nalazi medju podacima.}#
#\izlaz{Pozicija prvog pojavljivanja: 1}#
#\izlaz{Pozicija poslednjeg pojavljivanja: 5}#
\end{upotreba}
\end{miditest}
\begin{miditest}
\begin{upotreba}{2}
#\naslovInt#
#\izlaz{Unesite dimenziju niza:}#
#\ulaz{-5}#
#\izlaz{Greska: neispravan unos.}#
\end{upotreba}
\end{miditest}
\linkresenje{v.nizovi_funkcije_razno}
\end{Exercise}

\ifresenja
\begin{Answer}[ref=v.nizovi_funkcije_razno]
\includecode{resenja/2_NapredniTipoviPodataka/2.1_Nizovi/nizovi_16.c}
\end{Answer}
\fi


\begin{Exercise}[label=p.elementi_3_pojavljivanja] 
Marko skuplja sličice za Svetsko prvenstvo u fudbalu. Marko je
primetio da mu se neke sličice ponavljaju i rešio je da ih razmeni sa
drugarima. 
Napisati funkciju \kckod{int duplikati(int a[], int n, int b[])} koja od niza $a$ dimenzije $n$
formira niz b koji sadrži sve različite elemente niza $a$ koji se pojavljuju bar dva puta u nizu. Funkcija
kao povratnu vrednost vraća dimenziju niza $b$.
Napisati program koji učitava brojeve Markovih sličica i ispisuje sve duplikate.
Maksimalni broj elemenata niza je $600$.
U slučaju neispravnog unosa, ispisati odgovarajuću poruku o grešci. 

\begin{miditest}
\begin{upotreba}{1}
#\naslovInt#
#\izlaz{Unesite dimenziju niza:}\ulaz{8}#
#\izlaz{Unesite elemente niza a:}#
#\ulaz{4 11 4 6 8 4 6 6}#
#\izlaz{Elementi niza b: 4 6}#
\end{upotreba}
\end{miditest}
\begin{miditest}
\begin{upotreba}{2}
#\naslovInt#
#\izlaz{Unesite dimenziju niza:}\ulaz{13}#
#\izlaz{Unesite elemente niza a:}#
#\ulaz{8 26 7 2 1 1 7 2 2 2 7 5 1}#
#\izlaz{Elementi niza b: 7 2 1}#
\end{upotreba}
\end{miditest}

\begin{miditest}
\begin{upotreba}{3}
#\naslovInt#
#\izlaz{Unesite dimenziju niza:}\ulaz{2}#
#\izlaz{Unesite elemente niza a:}#
#\ulaz{9 5}#
#\izlaz{Elementi niza b: }#
\end{upotreba}
\end{miditest}
\begin{miditest}
\begin{upotreba}{4}
#\naslovInt#
#\izlaz{Unesite dimenziju niza:}\ulaz{0}#
#\izlaz{Greska: neispravan unos.}#
\end{upotreba}
\end{miditest}

\linkresenje{p.elementi_3_pojavljivanja}
\end{Exercise}
\ifresenja
\begin{Answer}[ref=p.elementi_3_pojavljivanja]
\includecode{resenja/2_NapredniTipoviPodataka/2.1_Nizovi/nizovi_17.c}
\end{Answer}
\fi


\begin{Exercise}[label=palindrom]
Palindrom je tekst koji se isto čita i sa leve i sa desne
strane. Napisati funkciju koja proverava da li je tekst zadat nizom karaktera palindrom (zanemariti razliku između malih i velikih slova). 
Napisati program koji učitava dužinu niza i niz karaktera, a zatim ispisuje da li je
uneti tekst palindrom. 
Maksimalni broj elemenata niza je $200$.
U slučaju neispravnog unosa, ispisati odgovarajuću poruku o grešci. 

\begin{miditest}
\begin{upotreba}{1}
#\naslovInt#
#\izlaz{Unesite dimenziju niza:}\ulaz{15}#
#\izlaz{Unesite elemente niza:}#
#\ulaz{AnaVoliMilovana}#
#\izlaz{Niz jeste palindrom.}#  
\end{upotreba}
\end{miditest}
\begin{miditest}
\begin{upotreba}{2}
#\naslovInt#
#\izlaz{Unesite dimenziju niza:}\ulaz{26}#
#\izlaz{Unesite elemente niza:}#
#\ulaz{Zanimljivo je programirati!}#
#\izlaz{Niz nije palindrom.}#  
\end{upotreba}
\end{miditest}

\begin{miditest}
\begin{upotreba}{3}
#\naslovInt#
#\izlaz{Unesite dimenziju niza:}\ulaz{1}#
#\izlaz{Unesite elemente niza:}#
#\ulaz{a}#
#\izlaz{Niz jeste palindrom.}#  
\end{upotreba}
\end{miditest}
\begin{miditest}
\begin{upotreba}{4}
#\naslovInt#
#\izlaz{Unesite dimenziju niza:}\ulaz{226}#
#\izlaz{Greska: neispravan unos.}# 
\end{upotreba}
\end{miditest}

\linkresenje{palindrom}
\end{Exercise}

\ifresenja
\begin{Answer}[ref=palindrom]
\includecode{resenja/2_NapredniTipoviPodataka/2.1_Nizovi/nizovi_18.c}
\end{Answer}
\fi


\begin{Exercise}[label=neopadajuce]
  Napisati funkciju koja proverava da li su elementi celobrojnog niza
  uređeni neopadajuće. 
  Napisati program koji učitava dimenziju niza, elemente niza, a zatim ispisuje
  da li je pomenuti uslov ispunjen. 
  Maksimalni broj elemenata niza je $300$.
U slučaju neispravnog unosa, ispisati odgovarajuću poruku o grešci. 

\begin{miditest}
\begin{upotreba}{1}
#\naslovInt#
#\izlaz{Unesite dimenziju niza:}\ulaz{7}#
#\izlaz{Unesite elemente niza:}\ulaz{-40 -8 -8 2 30 30 46}#
#\izlaz{Niz jeste uredjen neopadajuce.}#  
\end{upotreba}
\end{miditest}
\begin{miditest}
\begin{upotreba}{2}
#\naslovInt#
#\izlaz{Unesite dimenziju niza:}\ulaz{4}#
#\izlaz{Unesite elemente niza:}\ulaz{4 23 15 30}#
#\izlaz{Niz nije uredjen neopadajuce.}#  
\end{upotreba}
\end{miditest}

\begin{miditest}
\begin{upotreba}{3}
#\naslovInt#
#\izlaz{Unesite dimenziju niza:}\ulaz{1}#
#\izlaz{Unesite elemente niza:}\ulaz{5}#
#\izlaz{Niz jeste uredjen neopadajuce.}#  
\end{upotreba}
\end{miditest}
\begin{miditest}
\begin{upotreba}{4}
#\naslovInt#
#\izlaz{Unesite dimenziju niza:}\ulaz{304}#
#\izlaz{Greska: neispravan unos.}#  
\end{upotreba}
\end{miditest}

\linkresenje{neopadajuce}
\end{Exercise}

\ifresenja
\begin{Answer}[ref=neopadajuce]
\includecode{resenja/2_NapredniTipoviPodataka/2.1_Nizovi/nizovi_19.c}
\end{Answer}
\fi


\begin{Exercise}[label=najduzi_neopadajuci]
U celobrojnom nizu se čuvaju informacije o prodaji artikala jedne prodavnice. Svaki indeks niza označava jedan dan u mesecu, a elementi niza predstavljaju broj artikala koji se prodao tog dana. 
Napisati funkciju koja računa najdužu uzastopnu seriju dana za koju važi da broj
prodatih artikala nije opao.
Napisati program koji učitava broj dana u mesecu, broj prodatih artikala 
za svaki dan u mesecu i zatim ispisuje dužinu izračunate serije.
U slučaju neispravnog unosa, ispisati odgovarajuću poruku o grešci. 

\begin{miditest}
\begin{upotreba}{1}
#\naslovInt#
#\izlaz{Unesite dimenziju niza:}\ulaz{30}#
#\izlaz{Unesite broj prodatih artikala:}#
#\ulaz{89 171 112 67 119 36 181 157}#
#\ulaz{49 96 73 116 21 172}#
#\ulaz{140 0 23 71 157 135 11 166 21}#
#\ulaz{56 56 87 103 183 148 174}#
#\izlaz{Duzina najduzeg neopadajuceg}#
#\izlaz{prodavanja je 6.}#
\end{upotreba}
\end{miditest}
\begin{miditest}
\begin{upotreba}{2}
#\naslovInt#
#\izlaz{Unesite dimenziju niza:}\ulaz{31}#
#\izlaz{Unesite broj prodatih artikala:}#
#\ulaz{215 223 262 95 18 116 334 97}#
#\ulaz{146 146 19 314 270 115 21 40}#
#\ulaz{253 27 210 68 96 175 41 242}#
#\ulaz{98 163 8 218 107 102}#
#\izlaz{Duzina najduzeg neopadajuceg}#
#\izlaz{prodavanja je 3.}#
\end{upotreba}
\end{miditest}

\begin{miditest}
\begin{upotreba}{3}
#\naslovInt#
#\izlaz{Unesite dimenziju niza:}\ulaz{-5}#
#\izlaz{Greska: neispravan unos.}#
\end{upotreba}
\end{miditest}
\begin{miditest}
\begin{upotreba}{4}
#\naslovInt#
#\izlaz{Unesite dimenziju niza:}\ulaz{31}#
#\izlaz{Unesite broj prodatih artikala:}#
#\ulaz{-215 223 262 95 18 116 334 97}#
#\ulaz{146 146 19 314 -270 115 21 40}#
#\ulaz{253 27 210 68 96 175 41 242}#
#\ulaz{98 163 -8 218 107 102}#
#\izlaz{Greska: neispravan unos.}#
\end{upotreba}
\end{miditest}
\linkresenje{najduzi_neopadajuci}
\end{Exercise}

\ifresenja
\begin{Answer}[ref=najduzi_neopadajuci]
\includecode{resenja/2_NapredniTipoviPodataka/2.1_Nizovi/nizovi_20.c}
\end{Answer}
\fi


\skrati{3}
\begin{Exercise}[label=uzastopni_jednaki] 
Napisati funkciju koja određuje dužinu najduže serije jednakih
uzastopnih elemenata u datom nizu brojeva. 
Napisati program koji učitava dimenziju niza i elemente niza, a zatim ispisuje
dužinu najduže serije jednakih elemenata niza.
Maksimalni broj elemenata niza je $100$.
U slučaju neispravnog unosa, ispisati odgovarajuću poruku o grešci. 

\skrati{3}
\begin{miditest}
\begin{upotreba}{1}
#\naslovInt#
#\izlaz{Unesite dimenziju niza:}\ulaz{8}#
#\izlaz{Unesite elemente niza:}#
#\ulaz{9 -1 2 2 2 2 80 -200}#
#\izlaz{Duzina najduze serije je 4.}#
\end{upotreba}
\end{miditest}
\begin{miditest}
\begin{upotreba}{2}
#\naslovInt#
#\izlaz{Unesite dimenziju niza:}\ulaz{8}#
#\izlaz{Unesite elemente niza:}#
#\ulaz{9 9 0 -3 -3 -3 -3 72}#
#\izlaz{Duzina najduze serije je 4.}#
\end{upotreba}
\end{miditest}

\skrati{2}
\begin{miditest}
\begin{upotreba}{3}
#\naslovInt#
#\izlaz{Unesite dimenziju niza:}\ulaz{8}#
#\izlaz{Unesite elemente niza:}\ulaz{1 2 3 4 5 6 7 8}#
#\izlaz{Duzina najduze serije je 1.}#
\end{upotreba}
\end{miditest}
\begin{miditest}
\begin{upotreba}{4}
#\naslovInt#
#\izlaz{Unesite dimenziju niza:}\ulaz{108}#
#\izlaz{Greska: neispravan unos.}#
\end{upotreba}
\end{miditest}
\linkresenje{uzastopni_jednaki}
\end{Exercise}

\ifresenja
\begin{Answer}[ref=uzastopni_jednaki]
\includecode{resenja/2_NapredniTipoviPodataka/2.1_Nizovi/nizovi_21.c}
\end{Answer}
\fi


\skrati{2}
\begin{Exercise}[label=podniz] 
Napisati funkciju koja određuje da li se jedan niz javlja kao (uzastopni) podniz drugog niza. 
\begin{enumerate}[itemsep=0pt]
\item Niz $b$ je uzastopni podniz niza $a$ ako su elementi niza $b$ uzastopni elementi niza $a$. 
\item Niz $b$ je podniz niza $a$ ako je redosled pojavljivanja elemenata niza $b$ u nizu $a$ isti i ne nužno uzastopan. 
\end{enumerate}
Napisati program koji učitava dimenzije i elemente dvaju nizova, a zatim ispisuje da li je drugi niz podniz prvog niza. 
Maksimalni broj elemenata nizova $a$ i $b$ je $100$.
U slučaju neispravnog unosa, ispisati odgovarajuću poruku o grešci. 

\skrati{2}
\begin{miditest}
\begin{upotreba}{1}
#\naslovInt#
#\izlaz{Unesite dimenziju niza:}\ulaz{8}#
#\izlaz{Unesite elemente niza:}#
#\ulaz{-4 2 7 90 -22 15  14 7}#
#\izlaz{Unesite dimenziju niza:}\ulaz{4}#
#\izlaz{Unesite elemente niza:}\ulaz{90 -22 15 14}#
#\izlaz{Elementi drugog niza cine}#
#\izlaz{uzastopni podniz prvog niza.}#
#\izlaz{Elementi drugog niza cine}#
#\izlaz{podniz prvog niza.}#
\end{upotreba}
\end{miditest}
\begin{miditest}
\begin{upotreba}{2}
#\naslovInt#
#\izlaz{Unesite dimenziju niza:}\ulaz{8}#
#\izlaz{Unesite elemente niza:}#
#\ulaz{-4 2 7 90 -22 15  14 7}#
#\izlaz{Unesite dimenziju niza:}\ulaz{4}#
#\izlaz{Unesite elemente niza:}\ulaz{2 7 15 7}#
#\izlaz{Elementi drugog niza ne cine}#
#\izlaz{uzastopni podniz prvog niza.}#
#\izlaz{Elementi drugog niza cine}#
#\izlaz{podniz prvog niza.}#
\end{upotreba}
\end{miditest}

\skrati{2}
\begin{miditest}  
\begin{upotreba}{3}
#\naslovInt#
#\izlaz{Unesite dimenziju niza:}\ulaz{8}#
#\izlaz{Unesite elemente niza:}#
#\ulaz{-4 2 7 90 -22 15 14 7}#
#\izlaz{Unesite dimenziju niza:}\ulaz{4}#
#\izlaz{Unesite elemente niza:}\ulaz{90 -22 200 1}#
#\izlaz{Elementi drugog niza ne cine}#
#\izlaz{uzastopni podniz prvog niza.}#
#\izlaz{Elementi drugog niza ne}#
#\izlaz{cine podniz prvog niza.}#  
\end{upotreba}
\end{miditest}
\begin{miditest}
\begin{upotreba}{4}
#\naslovInt#
#\izlaz{Unesite dimenziju niza:}\ulaz{8}#
#\izlaz{Unesite elemente niza:}#
#\ulaz{-4 2 7 90 -22 15 14 7}#
#\izlaz{Unesite dimenziju niza:}\ulaz{1}#
#\izlaz{Unesite elemente niza:}\ulaz{90}#
#\izlaz{Elementi drugog niza cine}#
#\izlaz{uzastopni podniz prvog niza.}#
#\izlaz{Elementi drugog niza cine}#
#\izlaz{podniz prvog niza.}#
\end{upotreba}
\end{miditest}
\linkresenje{podniz}
\end{Exercise}

\ifresenja
\begin{Answer}[ref=podniz]
\includecode{resenja/2_NapredniTipoviPodataka/2.1_Nizovi/nizovi_22.c}
\end{Answer}
\fi


\skrati{3}
\begin{Exercise}[label=permutacija] 
Za celobrojni niz $a$ dimenzije $n$ kažemo da je \textit{permutacija}
ako sadrži sve brojeve od $1$ do $n$.
\skrati{3}
\begin{enumerate}
\setlength\itemsep{0em}
\item Napisati funkciju \kckod{void brojanje(int a[], int b[], int n)}
  koja na osnovu celobrojnog niza $a$ dimenzije $n$ formira niz $b$ dimenzije  $n$
  tako što $i$-ti element niza $b$ odgovara broju pojavljivanja
  vrednosti $i$ u nizu $a$.
\item Napisati funkciju \kckod{int permutacija(int a[], int n)} koja
  proverava da li je zadati niz permutacija. Funkcija vraća vrednost
  $1$ ako je svojstvo ispunjeno, odnosno $0$ ako
  nije. \uputstvo{Koristiti funkciju \kckod{brojanje}.}
\end{enumerate}
Napisati program koji učitava dimenziju niza i
elemente niza i ispisuje da li je uneti niz permutacija. 
Maksimalni broj elemenata niza je $100$.
U slučaju neispravnog unosa, ispisati odgovarajuću poruku o grešci. 

\skrati{3}
\begin{miditest}
\begin{upotreba}{1}
#\naslovInt#
#\izlaz{Unesite dimenziju niza:}\ulaz{5}#
#\izlaz{Unesite elemente niza:}\ulaz{1 5 4 3 2}#
#\izlaz{Uneti niz je permutacija.}#
\end{upotreba}
\end{miditest}
\begin{miditest}
\begin{upotreba}{2}
#\naslovInt#
#\izlaz{Unesite dimenziju niza:}\ulaz{6}#
#\izlaz{Unesite elemente niza:}\ulaz{2 3 3 1 1 5}#
#\izlaz{Uneti niz nije permutacija.}#
\end{upotreba}
\end{miditest}

\skrati{3}
\begin{miditest}
\begin{upotreba}{3}
#\naslovInt#
#\izlaz{Unesite dimenziju niza:}\ulaz{1}#
#\izlaz{Unesite elemente niza:}\ulaz{1}#
#\izlaz{Uneti niz je permutacija.}#
\end{upotreba}
\end{miditest}
\begin{miditest}
\begin{upotreba}{4}
#\naslovInt#
#\izlaz{Unesite dimenziju niza:}\ulaz{101}#
#\izlaz{Greska: neispravan unos.}#
\end{upotreba}
\end{miditest}
\linkresenje{permutacija}
\end{Exercise}

\ifresenja
\begin{Answer}[ref=permutacija]
\includecode{resenja/2_NapredniTipoviPodataka/2.1_Nizovi/nizovi_23.c}
\end{Answer}
\fi


\skrati{3}
\begin{Exercise}[label=p.brojevi_sa_istim_zapisima] 
 Napisati program koji učitava dva cela broja i 
 proverava da li se uneti brojevi zapisuju pomoću istih
 cifara. %\uputstvo{Zadatak rešiti korišćenjem nizova.}
 
\skrati{3} 
\begin{miditest}
\begin{upotreba}{1}
#\naslovInt#
#\izlaz{Unesite dva broja:}\ulaz{251 125}#
#\izlaz{Brojevi se zapisuju istim ciframa.}#
\end{upotreba}
\end{miditest}
\begin{miditest}
\begin{upotreba}{2}
#\naslovInt#
#\izlaz{Unesite dva broja:}\ulaz{8898 9988}#
#\izlaz{Brojevi se ne zapisuju istim ciframa.}#
\end{upotreba}
\end{miditest}

\skrati{3} 
\begin{miditest}
\begin{upotreba}{3}
#\naslovInt#
#\izlaz{Unesite dva broja:}\ulaz{-7391 1397}#
#\izlaz{Brojevi se zapisuju istim ciframa.}#
\end{upotreba}
\end{miditest} 
\begin{miditest}
\begin{upotreba}{4}
#\naslovInt#
#\izlaz{Unesite dva broja:}\ulaz{-1 1}#
#\izlaz{Brojevi se zapisuju istim ciframa.}#
\end{upotreba}
\end{miditest} 
\linkresenje{p.brojevi_sa_istim_zapisima}
\end{Exercise}

\ifresenja
\begin{Answer}[ref=p.brojevi_sa_istim_zapisima]
\includecode{resenja/2_NapredniTipoviPodataka/2.1_Nizovi/nizovi_24.c}
\end{Answer}
\fi


\skrati{3} 
\begin{Exercise}[label=v.nizovi_funkcije_pomeranja]
Napisati program koji vrši transformacije niza.
\skrati{2}
\begin{enumerate}[itemsep=0pt]
\item Napisati funkciju koja obrće elemente niza.     
\item Napisati funkciju koja rotira niz ciklično za jedno mesto ulevo.
\item Napisati funkciju koja rotira niz ciklično za $k$ mesta ulevo.
\end{enumerate}
\skrati{2}
Program učitava dimenziju niza, elemente niza i pozitivan ceo broj $k$, a zatim 
ispisuje niz koji se dobija nakon obrtanja početnog niza, niz koji se dobija
rotiranjem tako dobijenog niza za jedno mesto ulevo i niz koji se dobija rotiranjem novodobijenog niza 
za $k$ mesta ulevo. 
Maksimalni broj elemenata niza je $100$.
U slučaju neispravnog unosa, ispisati odgovarajuću poruku o grešci. 

\skrati{3}
\begin{miditest}
\begin{upotreba}{1}
#\naslovInt#
#\izlaz{Unesite dimenziju niza:}\ulaz{6}#
#\izlaz{Unesite elemente niza:}\ulaz{7 -3 11 783 26 -19}#
#\izlaz{Elementi niza nakon obrtanja:}#
#\izlaz{-17 28 785	13 -1 9}#
#\izlaz{Elementi niza nakon rotiranja za 1 mesto ulevo:}#
#\izlaz{28 785 13 -1 9 -17}#
#\izlaz{Unesite jedan pozitivan ceo broj:}\ulaz{3}#
#\izlaz{Elementi niza nakon rotiranja za 3 mesto ulevo:}#
#\izlaz{-1 9 -17 28	785	13}#
\end{upotreba}
\end{miditest}
\begin{miditest}
\begin{upotreba}{2}
#\naslovInt#
#\izlaz{Unesite dimenziju niza:}\ulaz{252}#
#\izlaz{Greska: neispravan unos.}#
\end{upotreba}
\end{miditest}
\linkresenje{v.nizovi_funkcije_pomeranja}
\end{Exercise}

\ifresenja
\begin{Answer}[ref=v.nizovi_funkcije_pomeranja]
\includecode{resenja/2_NapredniTipoviPodataka/2.1_Nizovi/nizovi_25.c}
\end{Answer}
\fi


\skrati{3}
\begin{Exercise}[label=v.ukrstanje_nizova]
Napisati funkciju \kckod{void ukrsti(int a[], int b[], int n, int c[])} 
koja formira niz $c$ koji se dobija naizmeničnim raspoređivanjem elemenata nizova $a$ i $b$, 
tj.~$c = [a_0, b_0, a_1, b_1, \ldots, a_{n-1}, b_{n-1}]$. 
Napisati program koji učitava dimenziju i elemente dvaju nizova i ispisuje niz koji se dobija
ukrštanjem unetih nizova. 
Maksimalni broj elemenata niza $a$ i $b$ je $100$.
U slučaju neispravnog unosa, ispisati odgovarajuću poruku o grešci. 

\skrati{3}
\begin{miditest}
\begin{upotreba}{1}
#\naslovInt#
#\izlaz{Unesite dimenziju nizova:}\ulaz{5}#
#\izlaz{Unesite elemente niza a:}\ulaz{2 -5 11 4 8}#
#\izlaz{Unesite elemente niza b:}\ulaz{3 3 9 -1 17}#
#\izlaz{Rezultujuci niz:}#
#\izlaz{2 3 -5 3 11 9 4 -1 8 17}#
\end{upotreba}
\end{miditest}
\begin{miditest}
\begin{upotreba}{2}
#\naslovInt#
#\izlaz{Unesite dimenziju nizova:}\ulaz{105}#
#\izlaz{Greska: neispravan unos.}#
\end{upotreba}
\end{miditest}
\linkresenje{v.ukrstanje_nizova}
\end{Exercise}

\ifresenja
\begin{Answer}[ref=v.ukrstanje_nizova]
\includecode{resenja/2_NapredniTipoviPodataka/2.1_Nizovi/nizovi_26.c}
\end{Answer}
\fi


\skrati{3}
\begin{Exercise}[label=p.nizovi_spajanje] 
Napisati funkciju \kckod{void spoji(int a[], int b[], int n, int c[])} koja od
nizova $a$ i $b$ dimenzije $n$ formira niz $c$ čija prva polovina odgovara
elementima niza $b$, a druga polovina elementima niza $a$, 
tj.~$c = [b_0, b_1, \ldots, b_{n-1}, a_0, a_1, \ldots, a_{n-1}]$.
Napisati program koji učitava dimenziju i elemente dvaju nizova i ispisuje niz koji se dobija
spajanjem unetih nizova na pomenuti način.
Maksimalni broj elemenata niza $a$ i $b$ je $100$.
U slučaju neispravnog unosa, ispisati odgovarajuću poruku o grešci. 

\skrati{3}
\begin{miditest}
\begin{upotreba}{1}
#\naslovInt#
#\izlaz{Unesite dimenziju nizova:}\ulaz{3}#
#\izlaz{Unesite elemente niza a:}\ulaz{4 -8 32}#
#\izlaz{Unesite elemente niza b:}\ulaz{5 2 11}#
#\izlaz{Rezultujuci niz:}#
#\izlaz{5 2 11 4 -8 32}#
\end{upotreba}
\end{miditest}
\begin{miditest}
\begin{upotreba}{2}
#\naslovInt#
#\izlaz{Unesite dimenziju nizova:}\ulaz{145}#
#\izlaz{Greska: neispravan unos.}#
\end{upotreba}
\end{miditest}
\linkresenje{p.nizovi_spajanje}
\end{Exercise}

\ifresenja
\begin{Answer}[ref=p.nizovi_spajanje]
\includecode{resenja/2_NapredniTipoviPodataka/2.1_Nizovi/nizovi_27.c}
\end{Answer}
\fi


\begin{Exercise}[difficulty=1, label=p.nizovi_spajanje_sortiranih]
 Napisati funkciju \kckod{void spoji\_sortirano(int a[], int b[], int n, int c[])} koja
 od nizova $a$ i $b$ dimenzije $n$ koji su uređeni neopadajuće 
 formira niz $c$ koji je uređen na isti način.
 Napisati program koji učitava dimenziju i elemente uređenih nizova $a$ i $b$ i
 ispisuje niz koji se dobija spajanjem ovih nizova na pomenuti način.
Maksimalni broj elemenata niza $a$ i $b$ je $100$.
U slučaju neispravnog unosa, ispisati odgovarajuću poruku o grešci. 

\skrati{2}
\begin{miditest}
\begin{upotreba}{1}
#\naslovInt#
#\izlaz{Unesite dimenziju nizova:}\ulaz{5}#
#\izlaz{Unesite elemente sortiranog niza:}#
#\ulaz{2 11 28 40 63}#
#\izlaz{Unesite elemente sortiranog niza:}#
#\ulaz{-19 -5 5 11 52}#
#\izlaz{Rezultujuci niz:}#
#\izlaz{-19 -5 2 5 11 11 28 40 52 63}#
\end{upotreba}
\end{miditest}
\begin{miditest}
\begin{upotreba}{2}
#\naslovInt#
#\izlaz{Unesite dimenziju nizova:}\ulaz{3}#
#\izlaz{Unesite elemente sortiranog niza:}#
#\ulaz{-2 4 8}#
#\izlaz{Unesite elemente sortiranog niza:}#
#\ulaz{6 15 19}#
#\izlaz{Rezultujuci niz:}#
#\izlaz{-2 4 6 8 15 19}#
\end{upotreba}
\end{miditest}

\skrati{2}
\begin{miditest}
\begin{upotreba}{3}
#\naslovInt#
#\izlaz{Unesite dimenziju nizova:}\ulaz{145}#
#\izlaz{Greska: neispravan unos.}#
\end{upotreba}
\end{miditest}
\linkresenje{p.nizovi_spajanje_sortiranih}
\end{Exercise}

\ifresenja
\begin{Answer}[ref=p.nizovi_spajanje_sortiranih]
\includecode{resenja/2_NapredniTipoviPodataka/2.1_Nizovi/nizovi_28.c}
\end{Answer}
\fi


\skrati{2}
\begin{Exercise}[label=vp.bez_resenja_7]
Napisati funkciju \kckod{void promeni\_redosled(int a[], int n)} koja menja redosled elementima
niza $a$ dimenzije $n$ tako da se parni elementi niza nalaze na početku niza, a neparni na kraju. 
Napisati program koji učitava dimenziju niza i elemente niza i ispisuje niz koji je izmenjen na
pomenuti način. 
Maksimalni broj elemenata niza je $100$.
U slučaju neispravnog unosa, ispisati odgovarajuću poruku o grešci. 
\napomena{Ne koristiti pomoćne nizove.}

\skrati{3}
\begin{miditest}
\begin{upotreba}{1}
#\naslovInt#
#\izlaz{Unesite dimenziju niza:}\ulaz{10}#
#\izlaz{Unesite elemente niza:}#
#\ulaz{-2 8 11 53 59 20 17 -8 3 14}#
#\izlaz{Rezultujuci niz:}#
#\izlaz{14 142 -6 -278 28 34 33 -69 -9 9}#
\end{upotreba}
\end{miditest}
\begin{miditest}
\begin{upotreba}{2}
#\naslovInt#
#\izlaz{Unesite dimenziju niza:}\ulaz{10}#
#\izlaz{Unesite elemente niza:}#
#\ulaz{9 142 -9 -278 -69 33 34 28 -6 14}#
#\izlaz{Rezultujuci niz:}#
#\izlaz{-2 8 14 -8 20 59 17 53 3 11}#
\end{upotreba}
\end{miditest}
\linkresenje{vp.bez_resenja_7}
\end{Exercise}

\ifresenja
\begin{Answer}[ref=vp.bez_resenja_7]
\includecode{resenja/2_NapredniTipoviPodataka/2.1_Nizovi/nizovi_29.c}
\end{Answer}
\fi


\skrati{3}
\begin{Exercise}[label=p.izbacivanje_prostih_elemenata] 
Napisati funkciju koja iz datog niza briše sve elemente koji su prosti brojevi. 
Funkcija kao povratnu vrednost treba da vrati broj elemenata niza nakon brisanja. 
Napisati program koji učitava dimenziju niza i elemente niza i ispisuje niz koji se dobija 
brisanjem pomenutih elemenata. 
Maksimalni broj elemenata niza je $100$.
U slučaju neispravnog unosa, ispisati odgovarajuću poruku o grešci. 
\napomena{Zadatak rešiti uz korišćenje pomoćnog niza.}

\skrati{2}
\begin{miditest}
\begin{upotreba}{1}
#\naslovInt#
#\izlaz{Unesite dimenziju niza:}\ulaz{5}#
#\izlaz{Unesite elemente niza:}\ulaz{11 5 6 48 8}#
#\izlaz{Rezultujuci niz: 6 48 8}#
\end{upotreba}
\end{miditest}
\begin{miditest}
\begin{upotreba}{2}
#\naslovInt#
#\izlaz{Unesite dimenziju niza:}\ulaz{4}#
#\izlaz{Unesite elemente niza:}\ulaz{11 5 19 21}#
#\izlaz{Rezultujuci niz: 21}#
\end{upotreba}
\end{miditest}

\skrati{2}
\begin{miditest}
\begin{upotreba}{3}
#\naslovInt#
#\izlaz{Unesite dimenziju niza:}\ulaz{5}#
#\izlaz{Unesite elemente niza:}\ulaz{12 18 9 31 7}#
#\izlaz{Rezultujuci niz: 12 18 9}#
\end{upotreba}
\end{miditest}
\begin{miditest}
\begin{upotreba}{4}
#\naslovInt#
#\izlaz{Unesite dimenziju niza:}\ulaz{5}#
#\izlaz{Unesite elemente niza:}\ulaz{-2 15 -11 8 7}#
#\izlaz{Rezultujuci niz: 15 8}#
\end{upotreba}
\end{miditest}
\linkresenje{p.izbacivanje_prostih_elemenata}
\end{Exercise}

\ifresenja
\begin{Answer}[ref=p.izbacivanje_prostih_elemenata]
\includecode{resenja/2_NapredniTipoviPodataka/2.1_Nizovi/nizovi_30.c}
\end{Answer}
\fi


\skrati{3}
\begin{Exercise}[label=p.izbacivanje_elemenata]
Napisati funkciju koja iz datog niza briše sve neparne elemente. Funkcija kao povratnu vrednost treba
da vrati broj elemenata niza nakon brisanja. 
Napisati program koji učitava dimenziju niza i elemente niza i ispisuje niz koji se dobija 
brisanjem neparnih elemenata.
Maksimalni broj elemenata niza je $100$.
U slučaju neispravnog unosa, ispisati odgovarajuću poruku o grešci. 
\napomena{Zadatak rešiti bez korišćenja pomoćnog niza.}
 
\skrati{2}
\begin{miditest}
\begin{upotreba}{1}
#\naslovInt#
#\izlaz{Unesite dimenziju niza:}\ulaz{4}#
#\izlaz{Unesite elemente niza:}#
#\ulaz{8 9 15 12}#
#\izlaz{Rezultujuci niz: 8 12}#
\end{upotreba}
\end{miditest}
\begin{miditest}
\begin{upotreba}{2}
#\naslovInt#
#\izlaz{Unesite dimenziju niza:}\ulaz{6}#
#\izlaz{Unesite elemente niza:}#
#\ulaz{21 5 3 22 19 188}#
#\izlaz{Rezultujuci niz: 22 188}#
\end{upotreba}
\end{miditest}

\skrati{2}
\begin{miditest}
\begin{upotreba}{3}
#\naslovInt#
#\izlaz{Unesite dimenziju niza:}\ulaz{4}#
#\izlaz{Unesite elemente niza:}\ulaz{133 129 121 101}#
#\izlaz{Rezultujuci niz:}#
\end{upotreba}
\end{miditest}
\begin{miditest}
\begin{upotreba}{4}
#\naslovInt#
#\izlaz{Unesite dimenziju niza:}\ulaz{8}#
#\izlaz{Unesite elemente niza:}#
#\ulaz{15 -22 -23 13 18 46 14 -31}#
#\izlaz{Rezultujuci niz: -22 18 46 14}#
\end{upotreba}
\end{miditest}
\linkresenje{p.izbacivanje_elemenata}
\end{Exercise}

\ifresenja
\begin{Answer}[ref=p.izbacivanje_elemenata]
\includecode{resenja/2_NapredniTipoviPodataka/2.1_Nizovi/nizovi_31.c}
\end{Answer}
\fi


\skrati{2}
\begin{Exercise}[label=v.brisanje_elemenata]
Napisati funkciju koja iz datog niza briše sve elemente koji nisu
deljivi svojom poslednjom cifrom. Izuzetak su elementi čija je poslednja
cifra nula. Funkcija kao povratnu vrednost treba
da vrati broj elemenata niza nakon brisanja. 
Napisati program koji učitava dimenziju niza i elemente niza i ispisuje niz koji se dobija  
brisanjem pomenutih elemenata. 
Maksimalni broj elemenata niza je $100$.
U slučaju neispravnog unosa, ispisati odgovarajuću poruku o grešci. 
\napomena{Zadatak rešiti bez korišćenja pomoćnog niza.}

\skrati{2}
\begin{miditest}
\begin{upotreba}{1}
#\naslovInt#
#\izlaz{Unesite dimenziju niza:}\ulaz{9}#
#\izlaz{Unesite elemente niza a:}#
#\ulaz{173 -25 23 7 17 25 34 61 -4612}#
#\izlaz{Rezultujuci niz: -25 7 25 61 -4612}#
\end{upotreba}
\end{miditest}
\begin{miditest}
\begin{upotreba}{2}
#\naslovInt#
#\izlaz{Unesite dimenziju niza:}\ulaz{0}#
#\izlaz{Greska: neispravan unos.}#
\end{upotreba}
\end{miditest}
\linkresenje{v.brisanje_elemenata}
\end{Exercise}

\ifresenja
\begin{Answer}[ref=v.brisanje_elemenata]
Pogledajte zadatke \ref{p.izbacivanje_prostih_elemenata} i \ref{p.izbacivanje_elemenata}.
\end{Answer}
\fi


\skrati{2}
\begin{Exercise}[label=deljivi_indeksom]
Napisati funkciju koja iz datog niza briše sve brojeve koji nisu deljivi svojim indeksom.
Ne razmatrati da li je u novom nizu, nakon brisanja i pomeranja, element deljiv svojim indeksom.
Funkcija kao povratnu vrednost treba da vrati broj elemenata niza nakon brisanja. 
Napisati program koji učitava dimenziju niza i elemente niza i ispisuje niz koji se dobija 
brisanjem pomenutih elemenata. 
Maksimalni broj elemenata niza je $700$. 
U slučaju neispravnog unosa, ispisati odgovarajuću poruku o grešci. 
\napomena{Nulti element niza treba zadržati jer nije dozvoljeno
  deljenje nulom. Zadatak rešiti bez korišćenja pomoćnog niza.}

\skrati{2}
\begin{miditest}
\begin{upotreba}{1}
#\naslovInt#
#\izlaz{Unesite dimenziju niza:}\ulaz{10}#
#\izlaz{Unesite elemente niza:}#
#\ulaz{4 2 1 6 7 8 10 2 16 3}#
#\izlaz{Rezultujuci niz: 4 2 6 16}#
\end{upotreba}
\end{miditest}
\begin{miditest}
\begin{upotreba}{2}
#\naslovInt#
#\izlaz{Unesite dimenziju niza:}\ulaz{10}#
#\izlaz{Unesite elemente niza:}#
#\ulaz{-8 5 10 6 7 10 8 2 16 27}#
#\izlaz{Rezultujuci niz: -8 5 10 6 10 16 27}#
\end{upotreba}
\end{miditest}
\linkresenje{deljivi_indeksom}
\end{Exercise}

\ifresenja
\begin{Answer}[ref=deljivi_indeksom]
\includecode{resenja/2_NapredniTipoviPodataka/2.1_Nizovi/nizovi_33.c}
\end{Answer}
\fi


\skrati{2}
\begin{Exercise}[label=p.unija_presek_razlika] 
Korišćenjem nizova moguće je predstaviti skupove podataka. Napisati
program koji demonstrira osnovne operacije nad skupovima (uniju,
presek i razliku). Pomoću dva niza predstaviti dva skupa celih
brojeva, a zatim ispisati njihovu uniju, presek i razliku. Maksimalni broj elemenata dva uneta niza je $500$.
U slučaju neispravnog unosa, ispisati odgovarajuću poruku o grešci. 
  
\skrati{2}
\begin{miditest}
\begin{upotreba}{1}
#\naslovInt#
#\izlaz{Unesite broj elemenata niza a:}\ulaz{5}#
#\izlaz{Unesite elemente niza a:}\ulaz{1 2 3 4 5}#
#\izlaz{Unesite broj elemenata niza b:}\ulaz{3}#
#\izlaz{Unesite elemente niza b:}\ulaz{5 4 9}#
#\izlaz{Unija: 1 2 3 4 5 9}#
#\izlaz{Presek: 4 5}#
#\izlaz{Razlika: 1 2 3}#
\end{upotreba}
\end{miditest}
\begin{miditest}
\begin{upotreba}{2}
#\naslovInt#
#\izlaz{Unesite broj elemenata niza a:}\ulaz{3}#
#\izlaz{Unesite elemente niza a:}\ulaz{11 4 -5}#
#\izlaz{Unesite broj elemenata niza b:}\ulaz{2}#
#\izlaz{Unesite elemente niza b:}\ulaz{18 9}#
#\izlaz{Unija: 11 4 -5 18 9}#
#\izlaz{Presek: }#
#\izlaz{Razlika: 11 4 -5}#
\end{upotreba}
\end{miditest}

\skrati{2}
\begin{miditest}
\begin{upotreba}{3}
#\naslovInt#
#\izlaz{Unesite broj elemenata niza a:}\ulaz{6}#
#\izlaz{Unesite elemente niza a:}\ulaz{12 7 9 12 5 1}#
#\izlaz{Greska: skup ne moze imati duplikate.}#
\end{upotreba}
\end{miditest}
\begin{miditest}
\begin{upotreba}{4}
#\naslovInt#
#\izlaz{Unesite broj elemenata niza a:}\ulaz{-2}#
#\izlaz{Greska: neispravan unos.}#
\end{upotreba}
\end{miditest}

\linkresenje{p.unija_presek_razlika}
\end{Exercise}

\ifresenja
\begin{Answer}[ref=p.unija_presek_razlika]
\includecode{resenja/2_NapredniTipoviPodataka/2.1_Nizovi/nizovi_34.c}
\sstrana
\end{Answer}
\fi


\skrati{2}
\begin{Exercise}[label=izbacivanje_ubacivanje_u_niz] 
Da bi opsluživanje klijenata bilo efikasno i udobno, prilikom ulaska u banku svaki klijent dobija redni broj opsluživanja. Redni brojevi se čuvaju
u nizu, počinju od vrednosti $1$ i iznova se generišu svakog radnog dana.  Postoje i specijalni klijenti (npr. oni koji podižu stambeni kredit) koji mogu dobiti i negativni redni broj da bi se razlikovali od uobičajenih klijenata. Pomozite radniku obezbeđenja da lakše prati redosled opsluživanja klijenata.

\begin{enumerate}[itemsep=0pt]
\item Napisati funkciju koja ubacuje redni broj klijenta $x$ na kraj niza (klijenta koji je poslednji došao).
\item Napisati funkciju koja ubacuje redni broj klijenta
  $x$ na početak niza (klijenta koji će biti prvi uslužen, na primer, lica sa posebnim potrebama, trudnice ili stara lica).
\item Napisati funkciju koja ubacuje redni broj klijenta
  $x$ na poziciju $k$ koju bira radnik obezbeđenja (manje prioritetna lica, recimo službena
  lica ili roditelji sa decom).
\item Napisati funkciju koja izbacuje prvi redni broj iz niza (redni broj usluženog
  klijenta).
\item Napisati funkciju koja izbacuje poslednji redni broj iz niza (redni broj klijenta
  koji je odustao jer je shvatio da ima mnogo klijenata ispred njega).
\item Napisati funkciju koja izbacuje redni broj iz niza sa pozicije $k$
  (redni broj klijenta koji je odustao jer je dugo čekao).
\end{enumerate}
Napisati program koji testira rad navedenih funkcija. Maksimalni broj klijenata
u jednom danu je $2000$.
U slučaju neispravnog unosa, ispisati odgovarajuću poruku o grešci. 

\begin{maxitest}
\begin{upotreba}{1}
#\naslovInt#
#\izlaz{Unesite trenutni broj klijenata:}\ulaz{8}#
#\izlaz{Unesite niz sa rednim brojevima klijenata:}\ulaz{2 5 -2 16 33 19 8 11}#
#\izlaz{Unesite klijenta kojeg treba ubaciti u niz:}\ulaz{35}#
#\izlaz{Niz nakon ubacivanja klijenta: 2 5 -2 16 33 19 8 11 35}#
#\izlaz{Unesite prioritetnog klijenta kojeg treba ubaciti u niz:}\ulaz{36}#
#\izlaz{Niz nakon ubacivanja klijenta: 36 2 5 -2 16 33 19 8 11 35}#
#\izlaz{Unesite prioritetnog klijenta kojeg treba ubaciti u niz i njegovu poziciju:}\ulaz{-6 2}#
#\izlaz{Niz nakon ubacivanja klijenta: 36 2 -6 5 -2 16 33 19 8 11 35}#  
#\izlaz{Niz nakon odlaska klijenta: 2 -6 5 -2 16 33 19 8 11 35}#  
#\izlaz{Niz nakon odlaska poslednjeg klijenta: 2 -6 5 -2 16 33 19 8 11}#
#\izlaz{Unesite redni broj klijenta koji je napustio red:}\ulaz{-2}#
#\izlaz{Niz nakon odlaska klijenta: 2 -6 5 16 33 19 8 11}#
\end{upotreba}
\end{maxitest}
\linkresenje{izbacivanje_ubacivanje_u_niz}
\end{Exercise}

\ifresenja
\begin{Answer}[ref=izbacivanje_ubacivanje_u_niz]
\includecode{resenja/2_NapredniTipoviPodataka/2.1_Nizovi/nizovi_35.c}
\end{Answer}
\fi


\ifresenja
\sstrana
\section{Rešenja}
\shipoutAnswer
\fi
