
\chapter{Predstavljanje podataka}

\section{Nizovi}



\begin{Exercise}[label=v2.1_01] 
Tekst
\linkresenje{v2.1_01}
\end{Exercise}
\begin{Answer}[ref=v2.1_01]
\includecode{resenja/2_PredstavljanjePodataka/2.1_Nizovi/1_01.c}
\end{Answer}

\begin{Exercise}[label=v2.1_02] 
Tekst
\linkresenje{v2.1_02}
\end{Exercise}
\begin{Answer}[ref=v2.1_02]
\includecode{resenja/2_PredstavljanjePodataka/2.1_Nizovi/1_02.c}
\end{Answer}

\begin{Exercise}[label=v2.1_03] 
Tekst
\linkresenje{v2.1_03}
\end{Exercise}
\begin{Answer}[ref=v2.1_03]
\includecode{resenja/2_PredstavljanjePodataka/2.1_Nizovi/1_03.c}
\end{Answer}

\begin{Exercise}[label=v2.1_04] 
Tekst
\linkresenje{v2.1_04}
\end{Exercise}
\begin{Answer}[ref=v2.1_04]
\includecode{resenja/2_PredstavljanjePodataka/2.1_Nizovi/1_04.c}
\end{Answer}

\begin{Exercise}[label=v2.1_05] 
Tekst
\linkresenje{v2.1_05}
\end{Exercise}
\begin{Answer}[ref=v2.1_05]
\includecode{resenja/2_PredstavljanjePodataka/2.1_Nizovi/1_05.c}
\end{Answer}

\begin{Exercise}[label=v2.1_06] 
Tekst
\linkresenje{v2.1_06}
\end{Exercise}
\begin{Answer}[ref=v2.1_06]
\includecode{resenja/2_PredstavljanjePodataka/2.1_Nizovi/1_06.c}
\end{Answer}

\begin{Exercise}[label=v2.1_07] 
Tekst
\linkresenje{v2.1_07}
\end{Exercise}
\begin{Answer}[ref=v2.1_07]
\includecode{resenja/2_PredstavljanjePodataka/2.1_Nizovi/1_07.c}
\end{Answer}

\begin{Exercise}[label=v2.1_08] 
Tekst
\linkresenje{v2.1_08}
\end{Exercise}
\begin{Answer}[ref=v2.1_08]
\includecode{resenja/2_PredstavljanjePodataka/2.1_Nizovi/1_08.c}
\end{Answer}

\begin{Exercise}[label=v2.1_09] 
Tekst
\linkresenje{v2.1_09}
\end{Exercise}
\begin{Answer}[ref=v2.1_09]
\includecode{resenja/2_PredstavljanjePodataka/2.1_Nizovi/1_09.c}
\end{Answer}

\begin{Exercise}[label=v2.1_10] 
Tekst
\linkresenje{v2.1_10}
\end{Exercise}
\begin{Answer}[ref=v2.1_10]
\includecode{resenja/2_PredstavljanjePodataka/2.1_Nizovi/1_10.c}
\end{Answer}

\begin{Exercise}[label=p2.1_] 
Sa standardnog ulaza se unosi dimenzija niza (broj manji od 100), a zatim i njegovi elementi. Napisati program koji kvadrira sve negativne elemente niza i ispisuje rezultujući niz. \\
\begin{miditest}
\begin{upotreba}{1}
#\naslovInt#
#\izlaz{Unesite broj elemenata niza:}\ulaz{6}#
#\izlaz{Unesite elemente niza:}#
#\ulaz{12.34 -6 1 8 32.4 -16}#
#\izlaz{12.34 36 1 8 32.4 256}#
\end{upotreba}
\end{miditest}
\begin{miditest}
\begin{upotreba}{2}
#\naslovInt#
#\izlaz{Unesite broj elemenata niza:}\ulaz{9}#
#\izlaz{Unesite elemente niza:}#
#\ulaz{-8.25 6 17 2 -1.5 1 -7 2.65 -125.2}#
#\izlaz{68.0625 6 17 2 2.25 1 49 2.65 15675.04}#
\end{upotreba}
\end{miditest}
\begin{miditest}
\begin{upotreba}{3}
#\naslovInt#
#\izlaz{Unesite broj elemenata niza:}\ulaz{4}#
#\izlaz{Unesite elemente niza:}#
#\ulaz{9.53 5 1 4.89}#
#\izlaz{9.53 5 1 4.89}#
\end{upotreba}
\end{miditest}
\linkresenje{p2.1_}
\end{Exercise}
\begin{Answer}[ref=p2.1_]
%\includecode{resenja/2_PredstavljanjePodataka/2.1_Nizovi/1_10.c}
\end{Answer}

\begin{Exercise}[label=p2.1_] 
Sa standardnog ulaza se učitava dimenzija niza (broj manji od 100), elemente niza i jedan ceo broj $k$. Napisati program koji štampa indekse elemenata koji su deljivi sa $k$. \\
\begin{miditest}
\begin{upotreba}{1}
#\naslovInt#
#\izlaz{Unesite dimenziju niza:}\ulaz{4}#
#\izlaz{Unesite elemente niza:}\ulaz{10 14 86 20}#
#\izlaz{Unesite broj k:}\ulaz{5}#
#\izlaz{0 3}#
\end{upotreba}
\end{miditest}
\begin{miditest}
\begin{upotreba}{2}
#\naslovInt#
#\izlaz{Unesite dimenziju niza:}\ulaz{4}#
#\izlaz{Unesite elemente niza:}\ulaz{6 14 8 9}#
#\izlaz{Unesite broj k:}\ulaz{5}#
#\izlaz{U nizu nema elemenata koji su deljivi brojem 5!}#
\end{upotreba}
\end{miditest}
\begin{miditest}
\begin{upotreba}{3}
#\naslovInt#
#\izlaz{Unesite dimenziju niza:}\ulaz{6}#
#\izlaz{Unesite elemente niza:}\ulaz{8 9 11 -4 8 11}#
#\izlaz{Unesite broj k:}\ulaz{2}#
#\izlaz{0 3 4}#
\end{upotreba}
\end{miditest}

\linkresenje{p2.1_}
\end{Exercise}
\begin{Answer}[ref=p2.1_]
%\includecode{resenja/2_PredstavljanjePodataka/2.1_Nizovi/1_10.c}
\end{Answer}

\begin{Exercise}[label=p2.1_] 
 Napisati program koji sa standardnog ulaza učitava dimenziju niza (broj manji od 100) i elemente niza, a zatim štampa niz u kojem su najveći i najmanji element niza razmenili mesta. \\
\begin{miditest}
\begin{upotreba}{1}
#\naslovInt#
#\izlaz{Unesite dimenziju niza:}\ulaz{5}#
#\izlaz{Unesite elemente niza:}\ulaz{8 -2 11 19 4}#
#\izlaz{8 19 11 -2 4}#
\end{upotreba}
\end{miditest}
\begin{miditest}
\begin{upotreba}{2}
#\naslovInt#
#\izlaz{Unesite dimenziju niza:}\ulaz{10}#
#\izlaz{Unesite elemente niza:}#
#\ulaz{46 -2 51 8 -5 66 2 8 3 14}#
#\izlaz{46 -2 51 8 66 -5 2 8 3 14}#
\end{upotreba}
\end{miditest}
\begin{miditest}
\begin{upotreba}{3}
#\naslovInt#
#\izlaz{Unesite dimenziju niza:}\ulaz{145}#
#\izlaz{Greska: pogresan unos!}#
\end{upotreba}
\end{miditest}

\linkresenje{p2.1_}
\end{Exercise}
\begin{Answer}[ref=p2.1_]
%\includecode{resenja/2_PredstavljanjePodataka/2.1_Nizovi/1_10.c}
\end{Answer}

\begin{Exercise}[label=p2.1_] 
 Napisati program koji učitava karaktere sa ulaza (najviše njih 100) sve do pojave karaktera \textit{*}, a zatim ih ispisuje u redosledu suprotnom od redosleda čitanja. \\
\begin{miditest}
\begin{upotreba}{1}
#\naslovInt#
#\izlaz{Unesite karakter:}\ulaz{a}#
#\izlaz{Unesite karakter:}\ulaz{8}#
#\izlaz{Unesite karakter:}\ulaz{5}#
#\izlaz{Unesite karakter:}\ulaz{Y}#
#\izlaz{Unesite karakter:}\ulaz{I}#
#\izlaz{Unesite karakter:}\ulaz{o}#
#\izlaz{Unesite karakter:}\ulaz{?}#
#\izlaz{Unesite karakter:}\ulaz{*}#
#\izlaz{? o I Y 5 8 a}#
\end{upotreba}
\end{miditest}
\begin{miditest}
\begin{upotreba}{2}
#\naslovInt#
#\izlaz{Unesite karakter:}\ulaz{g}#
#\izlaz{Unesite karakter:}\ulaz{g}#
#\izlaz{Unesite karakter:}\ulaz{2}#
#\izlaz{Unesite karakter:}\ulaz{2}#
#\izlaz{Unesite karakter:}\ulaz{)}#
#\izlaz{Unesite karakter:}\ulaz{)}#
#\izlaz{Unesite karakter:}\ulaz{*}#
#\izlaz{) ) 2 2 g g}#
\end{upotreba}
\end{miditest}
\begin{miditest}
\begin{upotreba}{3}
#\naslovInt#
#\izlaz{Unesite karakter:}\ulaz{U}#
#\izlaz{Unesite karakter:}\ulaz{4}#
#\izlaz{Unesite karakter:}\ulaz{a}#
#\izlaz{Unesite karakter:}\ulaz{u}#
#\izlaz{Unesite karakter:}\ulaz{*}#
#\izlaz{u a 4 U}#
\end{upotreba}
\end{miditest}

\linkresenje{p2.1_}
\end{Exercise}
\begin{Answer}[ref=p2.1_]
%\includecode{resenja/2_PredstavljanjePodataka/2.1_Nizovi/1_10.c}
\end{Answer}

\begin{Exercise}[label=p2.1_] 
 Napisati program koji za dva cela broja $x$ i $y$ koja se učitavaju sa standardnog ulaza proverava da li se zapisuju pomoću istih cifara. Napomena: iskoristiti niz za čuvanje broja pojavljivanja svake od cifara. \\
\begin{miditest}
\begin{upotreba}{1}
#\naslovInt#
#\izlaz{Unesite dva broja:}\ulaz{251 125}#
#\izlaz{Brojevi se zapisuju istim ciframa!}#
\end{upotreba}
\end{miditest}
\begin{miditest}
\begin{upotreba}{2}
#\naslovInt#
#\izlaz{Unesite dva broja:}\ulaz{8898 9988}#
#\izlaz{Brojevi se ne zapisuju istim ciframa!}#
\end{upotreba}
\end{miditest}
\begin{miditest}
\begin{upotreba}{3}
#\naslovInt#
#\izlaz{Unesite dva broja:}\ulaz{-7391 1397}#
#\izlaz{Brojevi se zapisuju istim ciframa!}#
\end{upotreba}
\end{miditest} 

\linkresenje{p2.1_}
\end{Exercise}
\begin{Answer}[ref=p2.1_]
%\includecode{resenja/2_PredstavljanjePodataka/2.1_Nizovi/1_10.c}
\end{Answer}

\begin{Exercise}[label=p2.1_] 
 Sa standardnog ulaza se učitava dimenzija niza (broj manji od 100), zatim i elementi dvaju nizova $a$ i $b$. Napisati program koji formira i ispisuje niz $c$ čiju prvu polovinu čine elementi niza $b$, a drugu polovinu elementi niza $a$. \\
\begin{miditest}
\begin{upotreba}{1}
#\naslovInt#
#\izlaz{Unesite broj n:}\ulaz{3}#
#\izlaz{Unesite elemente niza a:}\ulaz{4 -8 32}#
#\izlaz{Unesite elemente niza b:}\ulaz{5 2 11}#
#\izlaz{5 2 11 4 -8 32}#
\end{upotreba}
\end{miditest}
\begin{miditest}
\begin{upotreba}{2}
#\naslovInt#
#\izlaz{Unesite broj n:}\ulaz{4}#
#\izlaz{Unesite elemente niza a:}\ulaz{1 0 -1 0}#
#\izlaz{Unesite elemente niza b:}\ulaz{5 5 5 3}#
#\izlaz{5 5 5 3 1 0 -1 0}#
\end{upotreba}
\end{miditest}
\begin{miditest}
\begin{upotreba}{3}
#\naslovInt#
#\izlaz{Unesite dimenziju niza:}\ulaz{145}#
#\izlaz{Greska: pogresan unos!}#
\end{upotreba}
\end{miditest}

\linkresenje{p2.1_}
\end{Exercise}
\begin{Answer}[ref=p2.1_]
%\includecode{resenja/2_PredstavljanjePodataka/2.1_Nizovi/1_10.c}
\end{Answer}

\begin{Exercise}[label=p2.1_] 
 Sa standardnog ulaza se unosi dimenzija niza $a$ (broj manji od 100), a zatim i njegovi elementi. Napisati program koji od datog niza formira niz $b$ u koji ulaze elementi niza $a$ koji se pojavljuju tačno 3 puta. \\
\begin{miditest}
\begin{upotreba}{1}
#\naslovInt#
#\izlaz{Unesite broj elemenata niza:}\ulaz{8}#
#\izlaz{Unesite elemente niza a:}#
#\ulaz{4 11 4 6 8 4 6 6}#
#\izlaz{Elementi niza b: 4 6}#
\end{upotreba}
\end{miditest}
\begin{miditest}
\begin{upotreba}{2}
#\naslovInt#
#\izlaz{Unesite broj elemenata niza:}\ulaz{13}#
#\izlaz{Unesite elemente niza a:}#
#\ulaz{-8 26 7 2 1 1 7 2 2 2 7 5 1}#
#\izlaz{Elementi niza b: 7 1}#
\end{upotreba}
\end{miditest}
\begin{miditest}
\begin{upotreba}{3}
#\naslovInt#
#\izlaz{Unesite broj elemenata niza:}\ulaz{2}#
#\izlaz{Unesite elemente niza a:}#
#\ulaz{9 5}#
#\izlaz{Elementi niza b: }#
\end{upotreba}
\end{miditest}

\linkresenje{p2.1_}
\end{Exercise}
\begin{Answer}[ref=p2.1_]
%\includecode{resenja/2_PredstavljanjePodataka/2.1_Nizovi/1_10.c}
\end{Answer}

\begin{Exercise}[label=p2.1_] 
 Sa standardnog ulaza se, redom, učitavaju dimenzija i elementi dvaju nizova $a$ i $b$. Napisati program koji određuje njihovu uniju, presek i razliku (redosled prikaza elemenata nije bitan). Pretpostaviti da će nizovi imati manje od 100 elemenata. \\
\begin{miditest}
\begin{upotreba}{1}
#\naslovInt#
#\izlaz{Unesite broj elemenata niza a:}\ulaz{5}#
#\izlaz{Unesite elemente niza a:}\ulaz{2 8 1 5 2}#
#\izlaz{Unesite broj elemenata niza b:}\ulaz{3}#
#\izlaz{Unesite elemente niza b:}\ulaz{5 7 8}#
#\izlaz{Unija: 2 8 1 5 2 5 7 8}#
#\izlaz{Presek: 5}#
#\izlaz{Razlika: 2 1 2}#
\end{upotreba}
\end{miditest}
\begin{miditest}
\begin{upotreba}{2}
#\naslovInt#
#\izlaz{Unesite broj elemenata niza a:}\ulaz{3}#
#\izlaz{Unesite elemente niza a:}\ulaz{11 4 4}#
#\izlaz{Unesite broj elemenata niza b:}\ulaz{2}#
#\izlaz{Unesite elemente niza b:}\ulaz{18 9}#
#\izlaz{Unija: 11 4 4 18 9}#
#\izlaz{Presek: }#
#\izlaz{Razlika: 11 4 4}#
\end{upotreba}
\end{miditest}
\begin{miditest}
\begin{upotreba}{3}
#\naslovInt#
#\izlaz{Unesite broj elemenata niza a:}\ulaz{6}#
#\izlaz{Unesite elemente niza a:}\ulaz{12 7 9 12 5 1}#
#\izlaz{Unesite broj elemenata niza b:}\ulaz{4}#
#\izlaz{Unesite elemente niza b:}\ulaz{1 12 22 12}#
#\izlaz{Unija: 12 7 9 12 5 1 1 12 22 12}#
#\izlaz{Presek: 12 12 1}#
#\izlaz{Razlika: 7 9 5}#
\end{upotreba}
\end{miditest}

\linkresenje{p2.1_}
\end{Exercise}
\begin{Answer}[ref=p2.1_]
%\includecode{resenja/2_PredstavljanjePodataka/2.1_Nizovi/1_10.c}
\end{Answer}

\begin{Exercise}[label=p2.1_] 
 Napisati program koji učitava dimenziju niza (broj manji od 100) i elemente niza, a zatim formira i ispisuje niz koji se dobija izbacivanjem svih neparnih elemenata niza. Zadatak rešiti na dva načina: korišćenjem pomoćnog niza i transformacijom polaznog niza.\\
\begin{miditest}
\begin{upotreba}{1}
#\naslovInt#
#\izlaz{Unesite broj elemenata niza:}\ulaz{4}#
#\izlaz{Unesite elemente niza:}\ulaz{8 9 15 12}#
#\izlaz{8 12}#
\end{upotreba}
\end{miditest}
\begin{miditest}
\begin{upotreba}{2}
#\naslovInt#
#\izlaz{Unesite broj elemenata niza:}\ulaz{6}#
#\izlaz{Unesite elemente niza:}\ulaz{21 5 3 22 19 188}#
#\izlaz{22 188}#
\end{upotreba}
\end{miditest}
\begin{miditest}
\begin{upotreba}{3}
#\naslovInt#
#\izlaz{Unesite broj elemenata niza:}\ulaz{4}#
#\izlaz{Unesite elemente niza:}\ulaz{133 129 121 101}#
#\izlaz{}#
\end{upotreba}
\end{miditest}

\begin{maxitest}
\begin{upotreba}{4}
#\naslovInt#
#\izlaz{Unesite broj elemenata niza:}\ulaz{8}#
#\izlaz{Unesite elemente niza:}\ulaz{15 -22 -23 13 18 46 14 -31}#
#\izlaz{-22 18 46 14}#
\end{upotreba}
\end{maxitest}


\linkresenje{p2.1_}
\end{Exercise}
\begin{Answer}[ref=p2.1_]
%\includecode{resenja/2_PredstavljanjePodataka/2.1_Nizovi/1_10.c}
\end{Answer}

\begin{Exercise}[label=p2.1_] 
 Napisati program koji učitava dimenziju niza (broj manji od 100) i elemente niza, a zatim formira i ispisuje niz koji se dobija izbacivanjem svih elemenata koji su prosti brojevi. Zadatak rešiti na dva načina: korišćenjem pomoćnog niza i transformacijom polaznog niza. Napomena: brojeve -1 i 1 smatrati prostim. \\
\begin{miditest}
\begin{upotreba}{1}
#\naslovInt#
#\izlaz{Unesite broj elemenata niza:}\ulaz{5}#
#\izlaz{Unesite elemente niza:}\ulaz{11 5 6 48 8}#
#\izlaz{6 48 8}#
\end{upotreba}
\end{miditest}
\begin{miditest}
\begin{upotreba}{2}
#\naslovInt#
#\izlaz{Unesite broj elemenata niza:}\ulaz{4}#
#\izlaz{Unesite elemente niza:}\ulaz{11 5 19 21}#
#\izlaz{21}#
\end{upotreba}
\end{miditest}
\begin{miditest}
\begin{upotreba}{3}
#\naslovInt#
#\izlaz{Unesite broj elemenata niza:}\ulaz{5}#
#\izlaz{Unesite elemente niza:}\ulaz{12 18 9 31 7}#
#\izlaz{12 18 9}#
\end{upotreba}
\end{miditest}

\begin{miditest}
\begin{upotreba}{4}
#\naslovInt#
#\izlaz{Unesite broj elemenata niza:}\ulaz{3}#
#\izlaz{Unesite elemente niza:}\ulaz{-31 11 -19}#
#\izlaz{}#
\end{upotreba}
\end{miditest}
\begin{miditest}
\begin{upotreba}{5}
#\naslovInt#
#\izlaz{Unesite broj elemenata niza:}\ulaz{5}#
#\izlaz{Unesite elemente niza:}\ulaz{-2 15 -11 8 7}#
#\izlaz{15 8}#
\end{upotreba}
\end{miditest}


\linkresenje{p2.1_}
\end{Exercise}
\begin{Answer}[ref=p2.1_]
%\includecode{resenja/2_PredstavljanjePodataka/2.1_Nizovi/1_10.c}
\end{Answer}

\begin{Exercise}[label=p2.1_] 
 Napisati funkciju $int\ prebrojavanje(int\ a[],\ int\ n)$ koja izračunava broj elemenata niza celih brojeva $a$ dužine $n$ koji su manji od poslednjeg elementa niza. Napisati i program koji testira rad funkcije. Pretpostaviti da dužina niza neće biti veća od 100. \\
\begin{miditest}
\begin{upotreba}{1}
#\naslovInt#
#\izlaz{Unesite broj elemenata niza:}\ulaz{4}#
#\izlaz{Unesite elemente niza:}\ulaz{11 2 4 9}#
#\izlaz{2}#
\end{upotreba}
\end{miditest}
\begin{miditest}
\begin{upotreba}{2}
#\naslovInt#
#\izlaz{Unesite broj elemenata niza:}\ulaz{7}#
#\izlaz{Unesite elemente niza:}\ulaz{7 2 1 14 65 2 8}#
#\izlaz{4}#
\end{upotreba}
\end{miditest}
\begin{miditest}
\begin{upotreba}{3}
#\naslovInt#
#\izlaz{Unesite broj elemenata niza:}\ulaz{5}#
#\izlaz{Unesite elemente niza:}\ulaz{25 18 29 30 14}#
#\izlaz{0}#
\end{upotreba}
\end{miditest}

\linkresenje{p2.1_}
\end{Exercise}
\begin{Answer}[ref=p2.1_]
%\includecode{resenja/2_PredstavljanjePodataka/2.1_Nizovi/1_10.c}
\end{Answer}

\begin{Exercise}[label=p2.1_] 
 Napisati funkciju $int\ prebrojavanje(int\ a[],\ int\ n)$ koja izračunava broj parnih elemenata niza celih brojeva $a$ dužine $n$ koji prethode maksimalnom elementu niza. Napisati i program koji testira rad funkcije. Pretpostaviti da dužina niza neće biti veća od 100. \\
\begin{miditest}
\begin{upotreba}{1}
#\naslovInt#
#\izlaz{Unesite broj elemenata niza:}\ulaz{4}#
#\izlaz{Unesite elemente niza:}\ulaz{11 2 4 9}#
#\izlaz{0}#
\end{upotreba}
\end{miditest}
\begin{miditest}
\begin{upotreba}{2}
#\naslovInt#
#\izlaz{Unesite broj elemenata niza:}\ulaz{7}#
#\izlaz{Unesite elemente niza:}\ulaz{7 2 1 14 65 2 8}#
#\izlaz{2}#
\end{upotreba}
\end{miditest}
\begin{miditest}
\begin{upotreba}{3}
#\naslovInt#
#\izlaz{Unesite broj elemenata niza:}\ulaz{5}#
#\izlaz{Unesite elemente niza:}\ulaz{25 18 29 30 14}#
#\izlaz{1}#
\end{upotreba}
\end{miditest}

\linkresenje{p2.1_}
\end{Exercise}
\begin{Answer}[ref=p2.1_]
%\includecode{resenja/2_PredstavljanjePodataka/2.1_Nizovi/1_10.c}
\end{Answer}

\begin{Exercise}[label=p2.1_] 
 Napisati funkciju $int\ prebrojavanje\_cifre(char\ s[],\ int\ n)$ koja izračunava broj cifara u nizu karaktera $a$ dužine $n$. Napisati i program koji testira rad funkcije. Pretpostaviti da dužina niza neće biti veća od 100. \\
\begin{miditest}
\begin{upotreba}{1}
#\naslovInt#
#\izlaz{Unesite broj elemenata niza:}\ulaz{5}#
#\izlaz{Unesite elemente niza:}#
#\ulaz{4}#
#\ulaz{+}#
#\ulaz{A}#
#\ulaz{u}#
#\ulaz{8}#
#\izlaz{Broj cifara je: 2}#
\end{upotreba}
\end{miditest}
\begin{miditest}
\begin{upotreba}{2}
#\naslovInt#
#\izlaz{Unesite broj elemenata niza:}\ulaz{7}#
#\izlaz{Unesite elemente niza:}#
#\ulaz{J}#
#\ulaz{M}#
#\ulaz{a}#
#\ulaz{5}#
#\ulaz{5}#
#\ulaz{-}#
#\ulaz{2}#
#\izlaz{Broj cifara je: 3}#
\end{upotreba}
\end{miditest}
\begin{miditest}
\begin{upotreba}{3}
#\naslovInt#
#\izlaz{Unesite broj elemenata niza:}\ulaz{3}#
#\izlaz{Unesite elemente niza:}#
#\ulaz{e}#
#\ulaz{k}#
#\ulaz{F}#
#\izlaz{Broj cifara je: 0}#
\end{upotreba}
\end{miditest}

\linkresenje{p2.1_}
\end{Exercise}
\begin{Answer}[ref=p2.1_]
%\includecode{resenja/2_PredstavljanjePodataka/2.1_Nizovi/1_10.c}
\end{Answer}

\begin{Exercise}[label=p2.1_] 
 Napisati funkciju $int\ zbir(int\ a[],\ int\ n,\ int\ i,\ int\ j)$ koja računa zbir elemenata niza celih brojeva $a$ dužine $n$ od pozicije $i$ do pozicije $j$. Napisati i program koji testira rad funkcije. Pretpostaviti da dužina niza neće biti veća od 100. \\
\begin{miditest}
\begin{upotreba}{1}
#\naslovInt#
#\izlaz{Unesite broj elemenata niza:}\ulaz{5}#
#\izlaz{Unesite elemente niza:}\ulaz{11 5 6 48 8}#
#\izlaz{Unesite vrednosti za i i j:}\ulaz{0 2}#
#\izlaz{Zbir je: 22}#
\end{upotreba}
\end{miditest}
\begin{miditest}
\begin{upotreba}{2}
#\naslovInt#
#\izlaz{Unesite broj elemenata niza:}\ulaz{3}#
#\izlaz{Unesite elemente niza:}\ulaz{-2 8 1}#
#\izlaz{Unesite vrednosti za i i j:}\ulaz{8 12}#
#\izlaz{Greska: nekorektne vrednosti granica!}#
\end{upotreba}
\end{miditest}
\begin{miditest}
\begin{upotreba}{3}
#\naslovInt#
#\izlaz{Unesite broj elemenata niza:}\ulaz{7}#
#\izlaz{Unesite elemente niza:}\ulaz{-2 5 9 11 6 -3 -4}#
#\izlaz{Unesite vrednosti za i i j:}\ulaz{2 5}#
#\izlaz{Zbir: 23}#
\end{upotreba}
\end{miditest}
  
\linkresenje{p2.1_}
\end{Exercise}
\begin{Answer}[ref=p2.1_]
%\includecode{resenja/2_PredstavljanjePodataka/2.1_Nizovi/1_10.c}
\end{Answer}

\begin{Exercise}[label=p2.1_] 
 Napisati funkciju $float\ zbir\_pozitivnih(float\ a[],\ int\ n,\ int\ k)$ koja izračunava zbir prvih $k$ pozitivnih elemenata realnog niza $a$ dužine $n$. Napisati i program koji testira rad funkcije. Pretpostaviti da dužina niza neće biti veća od 100. \\
\begin{miditest}
\begin{upotreba}{1}
#\naslovInt#
#\izlaz{Unesite broj elemenata niza:}\ulaz{8}#
#\izlaz{Unesite elemente niza:}#
#\ulaz{2.34 1 -12.7 5.2 -8 -6.2 7 14.2}#
#\izlaz{Unesite vrednost za k:}\ulaz{3}#
#\izlaz{Zbir je: 8.54}#
\end{upotreba}
\end{miditest}
\begin{miditest}
\begin{upotreba}{2}
#\naslovInt#
#\izlaz{Unesite broj elemenata niza:}\ulaz{3}#
#\izlaz{Unesite elemente niza:}#
#\ulaz{-6.598 -8.14 -15}#
#\izlaz{Unesite vrednost za k:}\ulaz{4}#
#\izlaz{Zbir je: 0.00}#
\end{upotreba}
\end{miditest}
\begin{miditest}
\begin{upotreba}{3}
#\naslovInt#
#\izlaz{Unesite broj elemenata niza:}\ulaz{7}#
#\izlaz{Unesite elemente niza:}#
#\ulaz{-35.11 5.29 -1.98 12.1 12.2 -3.33 -4.17}#
#\izlaz{Unesite vrednost za k:}\ulaz{15}# 
#\izlaz{Zbir: 29.59}#
\end{upotreba}
\end{miditest}

\linkresenje{p2.1_}
\end{Exercise}
\begin{Answer}[ref=p2.1_]
%\includecode{resenja/2_PredstavljanjePodataka/2.1_Nizovi/1_10.c}
\end{Answer}

\begin{Exercise}[label=p2.1_] 
 Napisati funkciju $void\ kvadriranje(float\ a[],\ int\ n)$ koja kvadrira elemente realnog niza $a$ dužine $n$ koji se nalaze na parnim pozicijama. Napisati i program koji testira rad funkcije. Pretpostaviti da dužina niza neće biti veća od 100. \\
\begin{miditest}
\begin{upotreba}{1}
#\naslovInt#
#\izlaz{Unesite broj elemenata niza:}\ulaz{8}#
#\izlaz{Unesite elemente niza:}#
#\ulaz{2.34 1 -12.7 5.2 -8 -6.2 7 14.2}#
#\izlaz{5.4756 1 161.29 5.2 64 -6.2 49 14.2}#
\end{upotreba}
\end{miditest}
\begin{miditest}
\begin{upotreba}{2}
#\naslovInt#
#\izlaz{Unesite broj elemenata niza:}\ulaz{3}#
#\izlaz{Unesite elemente niza:}#
#\ulaz{-6 -8.14 -15}#
#\izlaz{36 -8.14 225}#
\end{upotreba}
\end{miditest}
\begin{miditest}
\begin{upotreba}{3}
#\naslovInt#
#\izlaz{Unesite broj elemenata niza:}\ulaz{1}#
#\izlaz{Unesite elemente niza:}#
#\ulaz{-35.11}#
#\izlaz{1232.71}#
\end{upotreba}
\end{miditest}
\linkresenje{p2.1_}
\end{Exercise}
\begin{Answer}[ref=p2.1_]
%\includecode{resenja/2_PredstavljanjePodataka/2.1_Nizovi/1_10.c}
\end{Answer}

\begin{Exercise}[label=p2.1_] 
\textbf{Filip-Janicic?} Napisati funkciju (i program koji je testira)
koja:
\begin{enumerate}
\item proverava da li dati niz sadrži dati broj;
\item pronalazi indeks prve pozicije na kojoj se u nizu nalazi dati
  broj (-1 ako niz ne sadrži broj).
\item pronalazi indeks poslednje pozicije na kojoj se u nizu nalazi
  dati broj (-1 ako niz ne sadrži broj).
\item izračunava zbir svih elemenata datog niza brojeva;
\item izračunava prosek (aritmetičku sredinu) svih elemenata datog
  niza brojeva;
\item izračunava najmanji element datog elemenata niza brojeva;
\item određuje poziciju najvećeg elementa u nizu brojeva (u slučaju
  više pojavljivanja najvećeg elementa, vratiti najmanju poziciju); 
\item proverava da li je dati niz brojeva uređen neopadajuće
\end{enumerate}
  \linkresenje{p2.1_}
\end{Exercise}
\begin{Answer}[ref=p2.1_]
%\includecode{resenja/2_PredstavljanjePodataka/2.1_Nizovi/1_10.c}
\end{Answer}

\begin{Exercise}[label=p2.1_] 
\textbf{Filip-Janicic?} Napisati funkciju (i program koji je testira)
koja: 
\begin{enumerate}
\item  izbacuje poslednji element niza; 
\item izbacuje prvi element niza (napisati varijantu u kojoj je bitno
  očuvanje redosleda elemenata i varijantu u kojoj nije bitno očuvanje
  redosleda); 
\item izbacuje element sa date pozicije k ; 
\item ubacuje element na kraj niza; 
\item ubacuje element na početak niza; 
\item ubacuje dati element x na datu poziciju k ; 
\item izbacuje sva pojavljivanja datog elementa x iz niza. 
\end{enumerate}
 Napomena: funkcija kao argument prima niz i broj njegovih trenutno
 popun- jenih elemenata, a vraća broj popunjenih elemenata nakon
 izvođenja zahtevane operacije.  \\
\linkresenje{p2.1_}
\end{Exercise}
\begin{Answer}[ref=p2.1_]
%\includecode{resenja/2_PredstavljanjePodataka/2.1_Nizovi/1_10.c}
\end{Answer}

\begin{Exercise}[label=p2.1_] 
\textbf{Filip-Janicic?} Napisati funkciju (i program koji je testira)
koja: 
\begin{enumerate}
\item  određuje dužinu najduže serije jednakih uzastopnih elemenata
u datom nizu brojeva; 
\item određuje dužinu najvećeg neopadajućeg podniza datog niza celih
  brojeva; 
\item određuje da li se jedan niz javlja kao podniz uzastopnih
  elemenata drugog; 
\item određuje da li se jedan niza javlja kao podniz elemenata drugog
  (elementi ne moraju da budu uzastopni, ali se redosled pojavljivanja
  poštuje); 
\item obrće dati niz brojeva; 
\item rotira sve elemente datog niza brojeva za k pozicija ulevo; 
\item rotira sve elemente datog niza brojeva za k pozicija udesno; 
\item izbacuje višestruka pojavljivanja elemenata iz datog niza
  brojeva (napisati varijantu u kojoj se zadržava prvo pojavljivanje i
  varijantu u kojoj se zadržava poslednje pojavljivanje).  
\item spaja dva niza brojeva koji su sortirani neopadajući u treći niz
  brojeva koji je sortiran neopadajući.  
\end{enumerate}
\linkresenje{p2.1_}
\end{Exercise}
\begin{Answer}[ref=p2.1_]
%\includecode{resenja/2_PredstavljanjePodataka/2.1_Nizovi/1_10.c}
\end{Answer}

\begin{Exercise}[label=p2.1_] 
Napisati funkciju \verb|int f3(int a[], int n, int b[], int m)| i
ispituje da li prvi sadr\v{z}i bar dva broja koji se pojavljuju u
drugom nizu. Povratna vrednost je dakle, 0, ili 1.  Testirati pozivom
u main-u. Maksimalna du\v zina niza je 100 elemenata.
\linkresenje{p2.1_}
\end{Exercise}
\begin{Answer}[ref=p2.1_]
%\includecode{resenja/2_PredstavljanjePodataka/2.1_Nizovi/1_10.c}
\end{Answer}

\begin{Exercise}[label=p2.1_] 
Napisati C funkciju koja u prosle\d enom nizu elimini\v se sve brojeve
koji nisu deljivi svojim indeksom (vrednost na indeksu 0 zadr\v zati,
jer nije dozvoljeno deljenje sa 0). Niz reorganizovati, tako da nema
\emph{rupa} koje su nastale eliminacijom elemenata. Kao rezultat
funkcije vratiti novu dimenziju niza.
\begin{miditest}
\begin{upotreba}{1}
#\naslovInt#
#\izlaz{Unesite broj elemenata niza:}\ulaz{10}#
#\izlaz{Unesite elemente niza:}#
#\ulaz{4 2 1 6 7 8 10 2 16 3}#
#\izlaz{4 2 6 16}#
\end{upotreba}
\end{miditest}
\linkresenje{p2.1_}
\end{Exercise}
\begin{Answer}[ref=p2.1_]
%\includecode{resenja/2_PredstavljanjePodataka/2.1_Nizovi/1_10.c}
\end{Answer}

\begin{Exercise}[label=p2.1_] 
Implementirati funkciju \verb|int min_max(int a[], int n)| koja
prihvata celobrojni niz, pronalazi indekse najmanjeg i najve\' ceg
elementa tog niza koriste\' ci samo jedan prolaz (jednu petlju), a
zatim kao povratnu vrednost vra\' ca manji od ta dva
indeksa.\\ Program testirati pozivom funkcije iz main programa i
ispisom rezultata na standardni izlaz, pri \v cemu korisnik sa
standardnog ulaza unosi niz duzine 10 elemenata.
\linkresenje{p2.1_}
\end{Exercise}
\begin{Answer}[ref=p2.1_]
%\includecode{resenja/2_PredstavljanjePodataka/2.1_Nizovi/1_10.c}
\end{Answer}

\begin{Exercise}[label=p2.1_] 
Napisati funkciju \verb|void brojanje(int a[], int brojac[], int N) |
\v ciji su argumenti \verb|a| i \verb|brojac| celobrojni nizovi
dimenzije N. Vrednosti elemenata niza \verb|a| su izmedu 0 i N -
1. Funkcija izra\v cunava elemente niza \verb|brojac| tako da je
\verb|brojac[i]| jednak broju pojavljivanja broja \verb|i| u nizu
\verb|a|.  Program testirati pozivom funkcije iz main programa -
korsnik u\v citava broj \verb|N| i potom niz \verb|a| du\v zine
\verb|N|, potom poziva funkciju i potom na standardnom izlazu izpisuje
dobijeni niz.
\linkresenje{p2.1_}
\end{Exercise}
\begin{Answer}[ref=p2.1_]
%\includecode{resenja/2_PredstavljanjePodataka/2.1_Nizovi/1_10.c}
\end{Answer}


\begin{Exercise}[label=p2.1_] 
Napisati funkciju \verb|int ind(int a[],int n)| koja kao povratnu
vrednost ima indeks onog elementa niza koji je po vrednosti najbli\v
zi srednjoj vrednosti onih elemenata niza brojeva koji su deljivi sa
3.\\ Program testirati pozivom funkcije iz main programa i ispisom
rezulata na standarni izlaz, pri \v cemu korisnik sa standardnog ulaza
unosi broj n, a zatim niz od n celih brojeva (maksimalna dimenzija
niza je 100 elemenata). \\
\begin{miditest}
\begin{upotreba}{1}
#\naslovInt#
#\izlaz{Unesite broj elemenata niza:}\ulaz{5}#
#\izlaz{Unesite elemente niza:}#
#\ulaz{1 2 3 4 5}#
#\izlaz{2}#
\end{upotreba}
\end{miditest}
\begin{miditest}
\begin{upotreba}{2}
#\naslovInt#
#\izlaz{Unesite broj elemenata niza:}\ulaz{5}#
#\izlaz{Unesite elemente niza:}#
#\ulaz{3 6 2 4 7}#
#\izlaz{3}#
\end{upotreba}
\end{miditest}
\linkresenje{p2.1_}
\end{Exercise}
\begin{Answer}[ref=p2.1_]
%\includecode{resenja/2_PredstavljanjePodataka/2.1_Nizovi/1_10.c}
\end{Answer}

\begin{Exercise}[label=p2.1_] 
Sa standardnog ulaza se unosi jedna linija teksta. Napisati program
koji prikazuje koliko puta se javilo svako od slova engleskog alfabeta
(ne praviti razliku izmedju velikih i malih slova). \\
\begin{maxitest}
\begin{upotreba}{1}
#\naslovInt#
#\ulaz{haHJjkL}#
#\izlaz{ a:1  b:0  c:0  d:0  e:0  f:0  g:0  h:2  i:0  j:2  k:1  l:1  m:0  n:0  o:0  p:0  q:0  r:0  s:0
   t:0  u:0  v:0  w:0  x:0  y:0  z:0}#
\end{upotreba}
\end{maxitest}
\begin{maxitest}
\begin{upotreba}{2}
#\naslovInt#
#\ulaz{DanaS j3 \_j\_utRo laBU78d}#
#\izlaz{a:3  b:1  c:0  d:2  e:0  f:0  g:0  h:2  i:0  j:2  k:0  l:1  m:0  n:1  o:1  p:0  q:0  r:1  s:1
  t:1  u:2  v:0  w:0  x:0  y:0  z:0}#
\end{upotreba}
\end{maxitest}
\begin{maxitest}
\begin{upotreba}{3}
#\naslovInt#
#\ulaz{Sao PaoLo 1998 \_JuZna Amerika90}#
#\izlaz{a:5  b:0  c:0  d:2  e:1  f:0  g:0  h:0  i:1  j:1  k:1  l:1  m:1  n:1  o:3  p:1  q:0  r:1  s:1
  t:0  u:1  v:0  w:0  x:0  y:0  z:0}#
\end{upotreba}
\end{maxitest}
\begin{maxitest}
\begin{upotreba}{4}
#\naslovInt#
#\ulaz{Ixxx kk 3yyy 4qqqq}#
#\izlaz{a:0  b:0  c:0  d:0  e:0  f:0  g:0  h:0  i:1  j:0  k:2  l:0  m:0  n:0  o:0  p:0  q:4  r:0  s:0
  t:0  u:0  v:0  w:0  x:3  y:3  z:0}#
\end{upotreba}
\end{maxitest}
\linkresenje{p2.1_}
\end{Exercise}
\begin{Answer}[ref=p2.1_]
%\includecode{resenja/2_PredstavljanjePodataka/2.1_Nizovi/1_10.c}
\end{Answer}

\begin{Exercise}[label=p2.1_] 
Napisati program koji sa standardnog ulaza u\v citava $50$ celih brojeva
i razdvaja ih na parne i neparne tako \v sto parne brojeve upisuje na
po\v cetak niza, a neparne na kraj niza. Ispisati niz dobijen na taj
na\v cin. Nije dozvoljeno koristiti dodatne nizove.
\linkresenje{p2.1_}
\end{Exercise}
\begin{Answer}[ref=p2.1_]
%\includecode{resenja/2_PredstavljanjePodataka/2.1_Nizovi/1_10.c}
\end{Answer}

\begin{Exercise}[label=p2.1_] 
\begin{itemize}
\item [(a)]
Napisati funkciju \verb"void brojanje(int a[], int brojac[], int N)"
\v ciji su argumenti \verb"a" i \verb"brojac" celobrojni nizovi dimenzije $N$.
Vrednosti elemenata niza \verb"a" su izme\d u $0$ i $N-1$. Funkcija izra\v cunava
elemente niza \verb"brojac" tako da je $i$-ti element \verb"brojac[i]" jednak
broju pojavljivanja broja $i$ u nizu \verb"a".

\item[(b)]
Za celobrojni niz \verb"a" dimenzije $N$ 
ka\v zemo da je \emph{permutacija} ako sadr\v zi sve brojeve
$i$: $0 \leq i \le N$.
Sastaviti funkciju \verb"int DaLiJePermutacija(int a[], int N)" koja vra\' ca
$1$ ako je niz \verb"a" permutacija, a $0$ ina\v ce. (koristiti funkciju
\verb"brojanje").
\end{itemize}
\linkresenje{p2.1_}
\end{Exercise}
\begin{Answer}[ref=p2.1_]
%\includecode{resenja/2_PredstavljanjePodataka/2.1_Nizovi/1_10.c}
\end{Answer}


\section{Rešenja}
\shipoutAnswer
