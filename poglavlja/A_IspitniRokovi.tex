\appendix
\chapter{Ispitni rokovi}

\section{Opšta grupa}

% I rok
\subsection{Praktični deo ispita,  januar 2019.}

\begin{Exercise}[label=A_o_1_1] 
Napisati program koji učitava četvorocifrene brojeve do unosa broja $0$, a zatim ispisuje one brojeve kojima je cifra desetica najveća cifra u zapisu. Ukoliko nema takvih brojeva među unetima, ispisati broj $0$. U slučaju greške, ispisati $-1$ na standardni izlaz za greške.\\

\begin{miditest}
\begin{upotreba}{1}
#\naslovInt#
#\naslovUlaz#
#\ulaz{9523 -8542 3232 -9999 -1121 1576 0}#
#\naslovIzlaz#
#\izlaz{3232 -9999 -1121 1576}#
\end{upotreba}
\end{miditest}
\begin{miditest}
\begin{upotreba}{2}
#\naslovInt#
#\naslovUlaz#
#\ulaz{9876 2258 -4579 4689 -5567 6630 1200 5204 0}#
#\naslovIzlaz#
#\izlaz{0}#
\end{upotreba}
\end{miditest}
\begin{miditest}
\begin{upotreba}{3}
#\naslovInt#
#\naslovUlaz#
#\ulaz{4596 1234 9631 -120 0}#
#\naslovIzlazZaGresku#
#\izlaz{4596 -1}#
\end{upotreba}
\end{miditest}

\linkresenje{A_o_1_1}
\end{Exercise}

\ifresenja
\begin{Answer}[ref=A_o_1_1]
\includecode{resenja/A_IspitniRokovi/rok_o_26.01.2019/1.c}
\end{Answer}
\fi

\begin{Exercise}[label=A_o_1_2] 
Napisati program koji pomaže korisniku  da ''šifruje'' svoju elektronsku adresu kako ne bi dobijao nepoželjne poruke. ''Šifrovanje'' adrese se vrši tako što se znak @ zameni sa $[AT]$. Elektronska adresa se učitava kao niska maksimalne dužine 100 karaktera sa standardnog ulaza,  a šifrovana adresa se ispisuje na standardni izlaz. U slučaju da elektronska adresa nije ispravno zadata ispisati $-1$ na standardni izlaz za greške. \\
\begin{minitest}
\begin{upotreba}{1}
#\naslovInt#
#\naslovUlaz#
#\ulaz{korisnik@gmail.com}#
#\naslovIzlaz#
#\izlaz{korisnik[AT]gmail.com}#
\end{upotreba}
\end{minitest}
\begin{minitest}
\begin{upotreba}{2}
#\naslovInt#
#\naslovUlaz#
#\ulaz{student@matf.bg.ac.rs}#
#\naslovIzlaz#
#\izlaz{student[AT]matf.bg.ac.rs}#
\end{upotreba}
\end{minitest}
\begin{minitest}
\begin{upotreba}{3}
#\naslovInt#
#\naslovUlaz#
#\ulaz{pogresnaadresayahoo.com}#
#\naslovIzlazZaGresku#
#\izlaz{-1}#
\end{upotreba}
\end{minitest}


\linkresenje{A_o_1_2}
\end{Exercise}

\ifresenja
\begin{Answer}[ref=A_o_1_2]
\includecode{resenja/A_IspitniRokovi/rok_o_26.01.2019/2.c}
\end{Answer}
\fi

\begin{Exercise}[label=A_o_1_3] 
Definisati strukturu \textit{Hemijski\_element} koja sadrži naziv elementa (nisku dužine najviše 20 karaktera), oznaku elementa (nisku dužine najviše 2 karaktera) i broj neutrona (ceo broj). Napisati program koji učitava podatke o hemijskim elementima do unosa reči \textbf{kraj}, a potom još jedan naziv elementa i na standardni izlaz ispisuje oznaku i broj neutrona tog elementa. Ukoliko element nije pronađen među učitanim podacima, ispisati -1. \\
\napomena{Pretpostaviti da neće biti uneto više od 120 elemenata, kao i da su podaci o hemijskim elementima ispravno zadati.} \\
\begin{miditest}
\begin{upotreba}{1}
#\naslovInt#
#\naslovUlaz#
#\ulaz{kalcijum Ca 20}#
#\ulaz{cink Zn 35}#
#\ulaz{fosfor P 16}#
#\ulaz{kraj}#
#\ulaz{fosfor}#
#\naslovIzlaz#
#\izlaz{P 16}#
\end{upotreba}
\end{miditest}
\begin{miditest}
\begin{upotreba}{2}
#\naslovInt#
#\naslovUlaz#
#\ulaz{nikl Ni 31}#
#\ulaz{bor B 6}#
#\ulaz{kripton Kr 48}#
#\ulaz{natrijum Na 12}#
#\ulaz{kraj}#
#\ulaz{hrom}#
#\naslovIzlazZaGresku#
#\izlaz{-1}#
\end{upotreba}
\end{miditest}
\begin{miditest}
\begin{upotreba}{3}
#\naslovInt#
#\naslovUlaz#
#\ulaz{litijum Li 4}#
#\ulaz{ugljenik C 6}#
#\ulaz{aluminijum Al 14}#
#\ulaz{srebro Ag 61}#
#\ulaz{gvozdje Fe 40}#
#\ulaz{brom Br 45}#
#\ulaz{kraj}#
#\ulaz{ugljenik}#
#\naslovIzlaz#
#\izlaz{C 6}#
\end{upotreba}
\end{miditest}

\linkresenje{A_o_1_3}
\end{Exercise}

\ifresenja
\begin{Answer}[ref=A_o_1_3]
\includecode{resenja/A_IspitniRokovi/rok_o_26.01.2019/3.c}
\end{Answer}
\fi


\begin{Exercise}[label=A_o_1_4] 
U datoteci \textit{pesme.txt} dat je ceo broj $n$ koji označava broj pesama, a potom i $n$ redova sa podacima o pesmama. U svakom redu naveden je naziv pesme i njen žanr (niske bez belina, dužine najviše 30 karaktera). Napisati program koji učitava podatke iz datoteke, a zatim, u zavisnosti od opcije koja se zadaje kao argument komandne linije, obrađuje podatke na sledeći način: 
\begin{itemize}
\item ukoliko je zadata opcija \textbf{-p}, učitava se sa standardnog ulaza jedan karakter i na standardni izlaz ispisuju svi nazivi pesama koji počinju zadatim karakterom;
\item ukoliko je zadata opcija \textbf{-z}, učitava se sa standardnog ulaza niska koja predstavlja žanr pesme i na standardni izlaz ispisuju nazivi svih pesama odabranog žanra.
\end{itemize}

Prilikom odabira pesama za ispis, zanemariti veličinu slova. U slučaju greške, ispisati $-1$ na standardni izlaz za greške. \\

\begin{miditest}
\begin{upotreba}{1}
#\naslovInt#
#\poziv{./a.out -p}#
#\naslovDat{pesme.txt}#
#\datoteka{7}#
#\datoteka{BohemianRhapsody rock}#
#\datoteka{RollingInTheDeep pop}#
#\datoteka{StairwayToHeaven rock}#
#\datoteka{BeatIt pop}#
#\datoteka{SoWhat jazz}#
#\datoteka{MyFunnyValentine jazz}#
#\datoteka{Smooth pop}#
#\naslovUlaz#
#\ulaz{S}#
#\naslovIzlaz#
#\izlaz{StairwayToHeaven}#
#\izlaz{SoWhat}#
#\izlaz{Smooth}#
\end{upotreba}
\end{miditest}
\begin{miditest}
\begin{upotreba}{2}
#\naslovInt#
 #\poziv{./a.out -z}#
 #\naslovDat{pesme.txt}#
#\datoteka{7}#
#\datoteka{BohemianRhapsody rock}#
#\datoteka{RollingInTheDeep pop}#
#\datoteka{StairwayToHeaven rock}#
#\datoteka{BeatIt pop}#
#\datoteka{SoWhat jazz}#
#\datoteka{MyFunnyValentine jazz}#
#\datoteka{Smooth pop}#
#\naslovUlaz#
#\ulaz{pop}#
#\naslovIzlaz#
#\izlaz{RollingInTheDeep}#
#\izlaz{BeatIt}#
#\izlaz{Smooth}#
\end{upotreba}
\end{miditest}\\
\begin{miditest}
\begin{upotreba}{3}
#\naslovInt#
#\poziv{./a.out -x}#
#\naslovIzlazZaGresku#
#\izlaz{-1}#
\end{upotreba}
\end{miditest}
\begin{miditest}
\begin{upotreba}{4}
#\naslovInt#
#\poziv{./a.out -p -z }#
#\naslovIzlazZaGresku#
#\izlaz{-1}#
\end{upotreba}
\end{miditest}
\begin{miditest}
\begin{upotreba}{5}
#\naslovInt#
#\poziv{./a.out }#
#\naslovIzlazZaGreske#
#\izlaz{-1}#
\end{upotreba}
\end{miditest}

\linkresenje{A_o_1_4}
\end{Exercise}

\ifresenja
\begin{Answer}[ref=A_o_1_4]
\includecode{resenja/A_IspitniRokovi/rok_o_26.01.2019/4.c}
\end{Answer}


% II rok
\subsection{Praktični deo ispita,  februar 2019.}

\begin{Exercise}[label=A_o_2_1] 
\item Napisati program koji učitava pozitivan četvorocifren broj $n$, a zatim na standardni izlaz ispisuje zbir onih cifara broja $n$ koje su po vrednosti veće od aritmetičke sredine svih cifara broja $n$. U slučaju greške, ispisati $-1$ na standardni izlaz za greške. \\

\begin{miditest}
\begin{upotreba}{1}
#\naslovInt#
#\naslovUlaz#
#\ulaz{1234}#
#\naslovIzlaz#
#\izlaz{7}#
\end{upotreba}
\end{miditest}
\begin{miditest}
\begin{upotreba}{2}
#\naslovInt#
#\ulaz{6745}#
#\naslovIzlaz#
#\izlaz{13}#
\end{upotreba}
\end{miditest}
\begin{miditest}
\begin{upotreba}{3}
#\naslovInt#
#\ulaz{100}#
#\naslovIzlazZaGreske#
#\izlaz{-1}#
\end{upotreba}
\end{miditest}
\begin{miditest}
\begin{upotreba}{4}
#\naslovInt#
#\ulaz{-1234}#
#\naslovIzlazZaGreske#
#\izlaz{-1}#
\end{upotreba}
\end{miditest}

\linkresenje{A_o_2_1}
\end{Exercise}

\ifresenja
\begin{Answer}[ref=A_o_2_1]
\includecode{resenja/A_IspitniRokovi/rok_o_09.02.2019/1.c}
\end{Answer}
\fi

\begin{Exercise}[label=A_o_2_2] 
Napisati program koji učitava nisku $s$ parne dužine od najviše 20 karaktera i na standardni izlaz ispisuje nisku koja se dobija nadovezivanjem karaktera prve polovine niske $s$ na drugu polovinu niske $s$. U slučaju greške, ispisati $-1$ na standardni izlaz za greške. \\
\begin{miditest}
\begin{upotreba}{1}
#\naslovInt#
#\naslovUlaz#
#\ulaz{Beograde}#
#\naslovIzlaz#
#\izlaz{radeBeog}#
\end{upotreba}
\end{miditest}
\begin{miditest}
\begin{upotreba}{2}
#\naslovInt#
#\naslovUlaz#
#\ulaz{matematika}#
#\naslovIzlaz#
#\izlaz{atikamatem}#
\end{upotreba}
\end{miditest}
\begin{miditest}
\begin{upotreba}{3}
#\naslovInt#
#\naslovUlaz#
#\ulaz{1234}#
#\naslovIzlaz#
#\izlaz{3412}#
\end{upotreba}
\end{miditest}
\begin{miditest}
\begin{upotreba}{4}
#\naslovInt#
#\naslovUlaz#
#\ulaz{abc1234}#
#\naslovIzlazZaGreske#
#\izlaz{-1}#
\end{upotreba}
\end{miditest}

\linkresenje{A_o_2_2}
\end{Exercise}

\ifresenja
\begin{Answer}[ref=A_o_2_2]
\includecode{resenja/A_IspitniRokovi/rok_o_09.02.2019/2.c}
\end{Answer}
\fi


\begin{Exercise}[label=A_o_2_3] 
Napisati program koji čita sadržaj datoteke \textit{ulaz.txt} i ispisuje na standardni izlaz sve niske datoteke koje predstavljaju cele brojeve. U slučaju greške, ispisati -1 na standardni izlaz za greške. \\ 
\begin{minitest}
\begin{upotreba}{1}
#\naslovInt#
#\poziv{./a.out}#
#\naslovDat{ulaz.txt}#
#\ulaz{123 ab1 2ab -23}#
#\naslovIzlaz#
#\izlaz{123 -23}#
\end{upotreba}
\end{minitest}
\begin{minitestt}
\begin{upotreba}{2}
#\naslovInt#
#\poziv{./a.out}#
#\naslovDat{ulaz.txt}#
#\ulaz{145as 25gf 265 478 65 -96}#
#\naslovIzlaz#
#\izlaz{265 478 65 -96}#
\end{upotreba}
\end{minitest}
\begin{minitest}
\begin{upotreba}{3}
#\naslovInt#
#\poziv{./a.out}#
#\naslovDat{ulaz.txt}#
#\ulaz{Ovde nema brojeva}#
#\naslovIzlaz#
#\izlaz{}#
\end{upotreba}
\end{minitest}
\begin{minitest}
\begin{upotreba}{4}
#\naslovInt#
#\poziv{./a.out}#
#\naslovDat{ulaz.txt ne postoji!}#
#\naslovIzlazZaGreske#
#\izlaz{-1}#
\end{upotreba}
\end{minitest}

\linkresenje{A_o_2_3}
\end{Exercise}

\ifresenja
\begin{Answer}[ref=A_o_2_3]
\includecode{resenja/A_IspitniRokovi/rok_o_09.02.2019/3.c}
\end{Answer}
\fi

\begin{Exercise}[label=A_o_2_4] 
\item Napisati program koji sa standardnog ulaza učitava podatke o osvajačima takmičenja. Za svako takmičenje se redom zadaju godina takmičenja (pozitivan ceo broj) i ime osvajača (niska od najviše $30$ karaktera bez belina). Program treba da ispiše:
\begin{itemize}
\item ako je navedena opcija -y kao prvi argument komandne linije, ime osvajača takmičenja za godinu koja se navodi kao drugi argument
 \item ako je navedena opcija -w kao prvi argument komandne linije, sve godine u kojima je takmičar čije se ime navodi kao drugi argument komande linije osvajao takmičenje.
\end{itemize}

U slučaju greške, ispisati $-1$ na standardni izlaz za greške. \\

\napomena{Podrazumevati da su ulazni podaci o takmičenjima ispravni. Broj osvajača nije unapred poznat.}\\

\begin{miditest}
\begin{upotreba}{1}
#\naslovInt#
#\poziv{./a.out -y 2016}#
#\naslovUlaz#
#\ulaz{2011 ManUtd}#
#\ulaz{2012 ManCity}#
#\ulaz{2013 ManUtd}#
#\ulaz{2014 ManCity}# 
#\ulaz{2015 Chelsea}#
#\ulaz{2016 Leicester}#
#\ulaz{2017 Chelsea}#
#\ulaz{2018 ManCity}#
#\naslovIzlaz#
#\izlaz{Leicester}#
\end{upotreba}
\end{miditest}
\begin{miditest}
\begin{upotreba}{2}
#\naslovInt#
#\poziv{./a.out -w RealMadrid}#
#\naslovUlaz#
#\ulaz{2011 Barcelona}#
#\ulaz{2012 Chelsea}#
#\ulaz{2013 BayernMunich}#
#\ulaz{2014 RealMadrid}#
#\ulaz{2015 Barcelona}#
#\ulaz{2016 RealMadrid}#
#\ulaz{2017 RealMadrid}#
#\ulaz{2018 RealMadrid}#
#\naslovIzlaz#
#\izlaz{2014 2016 2017 2018}#
\end{upotreba}
\end{miditest}
\begin{miditest}
\begin{upotreba}{3}
#\naslovInt#
#\poziv{./a.out -s 2001}#
#\naslovIzlazZaGreske#
#\izlaz{-1}#
\end{upotreba}
\end{miditest}
\begin{miditest}
\begin{upotreba}{4}
#\naslovInt#
#\poziv{./a.out -x}#
#\naslovIzlazZaGreske#
#\izlaz{-1}#
\end{upotreba}
\end{miditest}
\begin{miditest}
\begin{upotreba}{5}
#\naslovInt#
#\poziv{./a.out -s 2012 2000}#
#\naslovIzlazZaGreske#
#\izlaz{-1}#
\end{upotreba}
\end{miditest}
\begin{miditest}
\begin{upotreba}{6}
#\naslovInt#
#\poziv{./a.out -y 2005 -w RealMadrid}#
#\naslovIzlazZaGreske#
#\izlaz{-1}#
\end{upotreba}
\end{miditest}

\linkresenje{A_o_2_4}
\end{Exercise}

\ifresenja
\begin{Answer}[ref=A_o_2_4]
\includecode{resenja/A_IspitniRokovi/rok_o_09.02.2019/4.c}
\end{Answer}


\section{I smer}

\subsection{Praktični deo ispita,  xx.xx.2019.}



\section{Rešenja}
\shipoutAnswer