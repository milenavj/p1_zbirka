\appendix
\chapter{Ispitni zadaci}

\section{Testovi/Kolokvijumi}

\subsection{Programiranje 1, i-smer, kolokvijum}
\subsubsection{Grupa I}


\begin{Exercise}[label=v1.3_01] 
Napisati URM program koji izra\v cunava funkciju:
    $$ f(x,y)= \begin{cases}
                  2x - y & 2x \geq y \\
                  3y & \text{ina\v ce} \\
                  \end{cases}  $$
\linkresenje{v1.3_01}
\end{Exercise}
\begin{Answer}[ref=v1.3_01]
%\includecode{resenja/1_KontrolaToka/1.3_Petlje/1_01.c}
\end{Answer}


\begin{Exercise}[label=v1.3_01] 
Sa standardnog ulaza unose se jedan karakter (\textbf{p} ili \textbf{n}) i dva
pozitivna trocifrena broja. Na osnovu vrednosti unetog
karaktera izra\v cunati i ispisati na standardni izlaz:\\
\textbf{p} -  zbir cifara na parnim pozicijama unetih brojeva\\
\textbf{n} -  zbir cifara na neparnim pozicijama unetih brojeva\\
Cifre se broje sa desne strane, tako da cifri jedinice odgovara
pozicija 1.\\ U slu\v caju gre\v ske ( ukoliko karakter nije p ili n
ili nisu uneti pozitivni trocifreni brojevi )ispisati -1.
\begin{miditest}
\begin{upotreba}{1}
#\naslovInt#
#\ulaz{p 235 645}#
#\izlaz{8}#
\end{upotreba}
\end{miditest}
\begin{miditest}
\begin{upotreba}{2}
#\naslovInt#
#\ulaz{n 567 101}#
#\izlaz{14}#
\end{upotreba}
\end{miditest}
\begin{miditest}
\begin{upotreba}{3}
#\naslovInt#
#\ulaz{A 432 543}#
#\izlaz{-1}#
\end{upotreba}
\end{miditest}
\begin{miditest}
\begin{upotreba}{4}
#\naslovInt#
#\ulaz{p 102 1234}#
#\izlaz{-1}#
\end{upotreba}
\end{miditest}
\linkresenje{v1.3_01}
\end{Exercise}
\begin{Answer}[ref=v1.3_01]
%\includecode{resenja/1_KontrolaToka/1.3_Petlje/1_01.c}
\end{Answer}



\begin{Exercise}[label=v1.3_01] 
Sa standardnog ulaza u\v citava se pozitivan ceo broj i ceo broj {\tt
i} ($1\le i$).  Na standardni izlaz ispisati broj koji se dobija kada
se ukloni {\tt i}-ta cifra broja.  Cifre se broje sa desne strane,
tako da cifri jedinice odgovara pozicija 1.  Neispravan ulaz je kada
se unose negativan broj ili negativna vrednost ili nula za {\tt i} i u
tom slu\v caju na standardni izlaz ispisati -1.  Ukoliko broj nema
{\tt i}-tu cifru broj ostaje nepromenjen.
\begin{miditest}
\begin{upotreba}{1}
#\naslovInt#
#\ulaz{35243 2}#
#\izlaz{3523}#
\end{upotreba}
\end{miditest}
\begin{miditest}
\begin{upotreba}{2}
#\naslovInt#
#\ulaz{-14423 1}#
#\izlaz{-1}#
\end{upotreba}
\end{miditest}
\begin{miditest}
\begin{upotreba}{3}
#\naslovInt#
#\ulaz{1234 5}#
#\izlaz{1234}#
\end{upotreba}
\end{miditest}
\begin{miditest}
\begin{upotreba}{4}
#\naslovInt#
#\ulaz{523156 6}#
#\izlaz{23156}#
\end{upotreba}
\end{miditest}
\linkresenje{v1.3_01}
\end{Exercise}
\begin{Answer}[ref=v1.3_01]
%\includecode{resenja/1_KontrolaToka/1.3_Petlje/1_01.c}
\end{Answer}


\subsubsection{Grupa II}

\begin{Exercise}[label=v1.3_01] 
Napisati URM program koji izra\v cunava funkciju:
    $$ f(x,y,z)=4x + 2y + 3z $$
\linkresenje{v1.3_01}
\end{Exercise}
\begin{Answer}[ref=v1.3_01]
%\includecode{resenja/1_KontrolaToka/1.3_Petlje/1_01.c}
\end{Answer}


\begin{Exercise}[label=v1.3_01] 
Korisnik unosi 7 karaktera koji predstavljaju indeks studenta koji je
oblika OOGGBBB. OO je oznaka smera i moze biti mi, ma, mr, ms, mm,
mv. GG je oznaka godine upisa. BBB je oznaka broja koji moze biti
jednocifren, trocifren ili dvocifren sa vode\' cim nulama. Na osnovu
ovih podataka na standarni izlaz ispisati ime smera kome student
pripada i indeks u obliku broj/godina. U slu\v caju gre\v ske (
ukoliko OO kao oznaka smera nije ispravna ili ostali karakteri nisu
brojevi ) ispisati -1. Nazivi smerova su: mi - informatika, ma -
astronomija, mr - racunarstvo i informatika, ms - statistika, mm -
teorijska matematika, mp - primenjena matematika \\
\begin{miditest}
\begin{upotreba}{1}
#\naslovInt#
#\ulaz{mi11275}#
#\izlaz{informatika 275/2011}#
\end{upotreba}
\end{miditest}
\begin{miditest}
\begin{upotreba}{2}
#\naslovInt#
#\ulaz{mm98005}#
#\izlaz{teorijska matematika 5/1998}#
\end{upotreba}
\end{miditest}
\begin{miditest}
\begin{upotreba}{3}
#\naslovInt#
#\ulaz{mo23112}#
#\izlaz{-1}#
\end{upotreba}
\end{miditest}
\begin{miditest}
\begin{upotreba}{4}
#\naslovInt#
#\ulaz{ms12001}#
#\izlaz{statistika 1/2012}#
\end{upotreba}
\end{miditest}
\linkresenje{v1.3_01}
\end{Exercise}
\begin{Answer}[ref=v1.3_01]
%\includecode{resenja/1_KontrolaToka/1.3_Petlje/1_01.c}
\end{Answer}



\begin{Exercise}[label=v1.3_01] 
Dr\v zavna lutrija do\v sla je na ideju o novoj igri na sre\' cu.  Ova
igra na sre\' cu igra se tako \v sto se izvu\v ce jedan broj od 1000
do 9999, Nagrada koja se dobija ako ste pogodili izvu\v cen broj je
proizvod njegovih parnih cifara i samog broja.  Va\v s zadatak je da
na osnovu izu\v cenog broja izra\v cunate nagradu koja se dobija.  Kao
ulaz sigurno \'cete dobiti ispravan broj. Ako broj nema parnih cifara,
nagrada je sam taj broj. Na standardni izlaz ispi\v site nagradu. \\
\begin{miditest}
\begin{upotreba}{1}
#\naslovInt#
#\ulaz{1321}#
#\izlaz{2642}#
\end{upotreba}
\end{miditest}
\begin{miditest}
\begin{upotreba}{2}
#\naslovInt#
#\ulaz{3284}#
#\izlaz{210176}#
\end{upotreba}
\end{miditest}
\begin{miditest}
\begin{upotreba}{3}
#\naslovInt#
#\ulaz{1111}#
#\izlaz{1111}#
\end{upotreba}
\end{miditest}
\begin{miditest}
\begin{upotreba}{4}
#\naslovInt#
#\ulaz{2222}#
#\izlaz{35552}#
\end{upotreba}
\end{miditest}
\begin{miditest}
\begin{upotreba}{5}
#\naslovInt#
#\ulaz{6031}#
#\izlaz{0}#
\end{upotreba}
\end{miditest}
\begin{miditest}
\begin{upotreba}{1}
#\naslovInt#
#\ulaz{4321}#
#\izlaz{34568}#
\end{upotreba}
\end{miditest}
\linkresenje{v1.3_01}
\end{Exercise}
\begin{Answer}[ref=v1.3_01]
%\includecode{resenja/1_KontrolaToka/1.3_Petlje/1_01.c}
\end{Answer}


\subsubsection{Grupa III}


\begin{Exercise}[label=v1.3_01] 
Napisati URM program koji izra\v cunava funkciju:
    $$ f(x,y,z)= \begin{cases}
                  2 \cdot x + 2 \cdot y & x \leq z \\
                  z + 3 & \text{ina\v ce} \\
                  \end{cases}  $$
\linkresenje{v1.3_01}
\end{Exercise}
\begin{Answer}[ref=v1.3_01]
%\includecode{resenja/1_KontrolaToka/1.3_Petlje/1_01.c}
\end{Answer}


\begin{Exercise}[label=v1.3_01] 
Napisati C program koji sa standardnog ulaza u\v citava 4 velika slova
abedece i nenegativan ceo broj k. Program na standardni izlaz ispisuje
4 karaktera koji se dobijaju cikli\v ckim pomeranjem (u okviru
karakterske tabele) unetih karaktera za k mesta unapred. Na primer,
karakter A pomeren za 4 mesta unapred postaje E dok karakter Z pomeren
za 3 mesta unapred postaje C. U slu\v caju neispravnog ulaza ispisati
-1. Ulaz se smatra neispravnim ako neki od unetih karaktera ne
predstavlja veliko slovo abecede ili ako je broj k negativan,
pretpostaviti da se na ulazu uvek zadaje ta\v cno \v cetiri karaktera. \\
\begin{miditest}
\begin{upotreba}{1}
#\naslovInt#
#\ulaz{BABA 3}#
#\izlaz{EDED}#
\end{upotreba}
\end{miditest}
\begin{miditest}
\begin{upotreba}{2}
#\naslovInt#
#\ulaz{DEDA 26}#
#\izlaz{DEDA}#
\end{upotreba}
\end{miditest}
\begin{miditest}
\begin{upotreba}{3}
#\naslovInt#
#\ulaz{ZABC 53}#
#\izlaz{ABCD}#
\end{upotreba}
\end{miditest}
\begin{miditest}
\begin{upotreba}{4}
#\naslovInt#
#\ulaz{PERA -2}#
#\izlaz{-1}#
\end{upotreba}
\end{miditest}
\begin{miditest}
\begin{upotreba}{1}
#\naslovInt#
#\ulaz{abcd}#
#\izlaz{-1}#
\end{upotreba}
\end{miditest}
\linkresenje{v1.3_01}
\end{Exercise}
\begin{Answer}[ref=v1.3_01]
%\includecode{resenja/1_KontrolaToka/1.3_Petlje/1_01.c}
\end{Answer}


\begin{Exercise}[label=v1.3_01] 
Napisati C program koji sa standardnog ulaza u\v citava dva \v
cetvorocifrena, pozitivna, cela broja i proverava da li je broj koji
se dobija u\v ce\v sljavanjem unetih brojeva palindrom. Ako uneti
brojevi imaju cifre a1 a2 a3 a4 i b1 b2 b3 b4 tada su cifre u\v ce\v
sljanog broja a1 b1 a2 b2 a3 b3 a4 b4. Broj je palindrom ako se \v
cita isto sa obe strane. Ukoliko je broj palindrom ispisati na
standardni izlaz 1, ukoliko nije tada ispisati 0, a u slu\v caju
neispravnog ulaza ispisati -1, neispravnim ulazom smatraju se
negativni brojevi i brojevi sa brojem cifara manjim ili ve\' cim od 4.
\begin{center}
\begin{verbatim}
Primer 1:        Primer 2:        Primer 3:        Primer 4:
1234 5678        1342 2431        1234 4321        -1234 1234
0                1                1                -1
\end{verbatim}
\end{center}
\linkresenje{v1.3_01}
\end{Exercise}
\begin{Answer}[ref=v1.3_01]
%\includecode{resenja/1_KontrolaToka/1.3_Petlje/1_01.c}
\end{Answer}


\section{Kvalifikacioni zadaci}

\section{Ispitni rokovi}

\subsection{Programiranje 1, i--smer, Zavr\v{s}ni ispit, januar, 23.01.2016.}

\begin{Exercise}[label=v1.3_01] 
{\em (5 poena)}  Napisati URM program koji izra\v cunava funkciju:
    $$ f(x)= \begin{cases}
                  2(x - 1) & x \geq 1 \\
                  0 & \text{ina\v ce} \\
                  \end{cases}  $$
\linkresenje{v1.3_01}
\end{Exercise}
\begin{Answer}[ref=v1.3_01]
%\includecode{resenja/1_KontrolaToka/1.3_Petlje/1_01.c}
\end{Answer}


\subsubsection{Grupa I}

\begin{Exercise}[label=v1.3_01] 
{\em (4 poena)} Napisati C program koji sa standardnog ulaza učitava
pozitivan ceo broj \textbf{n} i na standardni izlaz ispisuje n-ti član
niza:

 $$ a_n= \begin{cases}
                  1 & n = 1 \\
                  3 & n = 2 \\
                  2a_{n-1} + 3a_{n-2} + 4 & n \geq 3 \\
                  \end{cases}  $$

Neispravnim ulazom se smatra broj manji ili jednak nuli i u tom
slučaju na standardni izlaz ispisati -1.  Dozvoljeno je kori\v
s\'cenje nizova. Maksimalna vrednost za {\bf n} je {\bf 2000}.
\begin{center}
\begin{verbatim}
Primer 1:       Primer 2:     Primer 3:     Primer 4:  
-123            1             4             10
-1              1             39            29523
\end{verbatim}
\end{center}
\linkresenje{v1.3_01}
\end{Exercise}
\begin{Answer}[ref=v1.3_01]
%\includecode{resenja/1_KontrolaToka/1.3_Petlje/1_01.c}
\end{Answer}


\begin{Exercise}[label=v1.3_01] 
{\em (7 poena)} 
Napisati funkciju
\begin{center} 
\begin{verbatim}void f3(char s[], char* c, int* br)
\end{verbatim}
\end{center}
   koja proverava koji karakter se najviše puta pojavio u niski \textit{s}.
   Taj karakter smešta u promenljivu \textbf{c}, a broj pojavljivanja karaktera u promenljivu \textbf{br}.
   Sa standardnog ulaza unosi se linija teksta (može sadr\v zati beline).
   Testirati rad funkcije \textit{f3} programom koji sa standardnog ulaza učitava nisku i na standarni izlaz ispisati koji karakter se najviše puta pojavio u okviru nje, kao i broj pojavljivanja datog karaktera. Ukoliko postoji više karaktera čiji broj pojavljivanja odgovara maksimalnom broju, ispisati onaj sa najmanjim kodom u ASCII tabeli.
Pretpostaviti da se na sistemu koristi {\tt ASCII} tabela.
\begin{center}
\begin{verbatim}
Primer 1:       Primer 2:     Primer 3:     Primer 4:  
abrakadabra     cvrcak        jorgovan99    s@rm@ ponek@d v@zno
a 5             c 2           9 2           @ 4        
\end{verbatim}
\end{center}
\linkresenje{v1.3_01}
\end{Exercise}
\begin{Answer}[ref=v1.3_01]
%\includecode{resenja/1_KontrolaToka/1.3_Petlje/1_01.c}
\end{Answer}



\begin{Exercise}[label=v1.3_01] 
{\em (7 poena)} Igra "Minesweeper" sastoji se od pravougaone table
   izdeljene na polja koja mogu biti bezbedna ili su na njima
   rasporedjene mine. Sa standardnog ulaza učitavaju se
   brojevi \textbf{n} i \textbf{m} koji označavaju dimenzije
   table. Nako toga unosi se broj \textbf{k} kojim se navodi koliko
   mina se nalazi na tabli i k pozicija (\textbf{i}, \textbf{j}) koja
   označavaju pozicije na tabli na kojima se nalaze mine (i-ti red,
   j-ta kolona). Korisnik zatim unosi koordinate \textbf{l}
   i \textbf{m} za koje se na standardni izlaz ispisuje broj koliko se
   mina nalazi na poljima susednim tom polju. Proveravaju se susedna
   polja u svih 8 pravaca.  Ukoliko je polje koje se proverava baš
   mina ispisati na standardni izlaz \textbf{MINA}. Maksimalna
   dimenzija table je 100x100. Ukoliko je neka od koordinata izvan
   dimenzija table ili su dimenzije table izvan dozvoljenih granica na
   standardni izlaz ispisati -1.


\begin{center}
\begin{verbatim}
Primer 1:           Primer 2:           Primer 3:         Primer 4:
Ulaz:    Izlaz:     Ulaz:    Izlaz:     Ulaz     Izlaz:   Ulaz:     Izlaz:
4 4      2          4 4      MINA       2 3      -1       101 10    -1
3                   2                   1                 1
0 1                 0 1                 -1 0              45 67
1 2                 1 2                 2 2               30 31
2 3                 2 3
2 2                 2 3
\end{verbatim}
\end{center}
\linkresenje{v1.3_01}
\end{Exercise}
\begin{Answer}[ref=v1.3_01]
%\includecode{resenja/1_KontrolaToka/1.3_Petlje/1_01.c}
\end{Answer}


\begin{Exercise}[label=v1.3_01] 
{\em (7 poena)} Služba gradskog prevoza želi da u svakom trenutku ima
evidenciju o opterećenju svojih linija. Na linijama saobraćaju
autobusi, trolejbusi i tramvaji. Maksimalni kapacitet autobusa je 25,
trolejbusa 20 a tramvaja 30 putnika. Broj linije je pozitivan ceo broj
manji od 1000.
\begin{description}
\item[a)] {\em (1 poen)}  Definisati strukturu kojim se opisuje vozilo. Svako vozilo zadato je svojim tipom (autobus, trolejbus, tramvaj), linijom na kojom saobraća i
   brojem putnika koji se u vozilu nalaze.
\item[b)] {\em (6 poena)}  Sa standardnog ulaza se učitava
   broj \textbf{n ($0 \le n \le 1000$), n vozila i broj linije}. Za
   zadati broj linije na standardni izlaz ispisati ukupan broj
   slobodnih mesta na toj liniji. Koristiti strukturu definisanu pod
   {\bf a)}. Neispravnim ulazom smatraju se negativan broj putnika,
   broj putnika veći od dozvoljenog kapaciteta za navedeni tip vozila,
   tip vozila sa nazivom različitim od navedena tri ili negativan broj
   linije. U tim slučajevima na standardni izlaz ispisati -1.
\end{description}
\begin{small}
\begin{center}
\begin{verbatim}
Primer 1:                      Primer 2:                       Primer 3:             
Ulaz:               Izlaz:     Ulaz:             Izlaz:        Ulaz              Izlaz:
4                   8          3                 -1            3                 0
autobus   27 18                AutobuS 65 23                   tramvaj 7  29
trolejbus 28 15                Kombi   1  10                   tramvaj 3  15
tramvaj   7  29                minibus 6  21                   tramvaj 12 12
autobus   27 24               6                                14
27
_________________________________________________________________________________________
Primer 4:                           Primer 5:
Ulaz:               Izlaz:          Ulaz:        Izlaz:
2                   -1              500          -1
autobus 26 20  
tramvaj 9  32
\end{verbatim}
\end{center}
\end{small}
\linkresenje{v1.3_01}
\end{Exercise}
\begin{Answer}[ref=v1.3_01]
%\includecode{resenja/1_KontrolaToka/1.3_Petlje/1_01.c}
\end{Answer}




% -------------------------------------------
%              GRUPA 2
% -------------------------------------------

\subsubsection{Grupa II}


\begin{Exercise}[label=v1.3_01] 
{\em (4 poena)}
Napisati C program koji za uneti niz  celobrojnog tipa i neparne 
dužine $n$ ispisuje po $k$ elemenata levo i desno od sredine 
niza (ne uključujući sredinu). Prvo se unosi $n$, zatim niz 
od $n$ elemenata, a na kraju i $k$.

Neispravnim ulazom se smatra niz parne ili negativne dužine, 
kao i $k$ koje je negativno ili veće od polovine dužine niza. 
U slučaju neispravnog ulaza ispisati -1 na standardni izlaz.

Smatrati da je maksimalna veličina niza 100 elemenata.
\begin{center}
\begin{verbatim}
Primer 1:       Primer 2:             Primer 3:     Primer 4:   Primer 5:
Ulaz:           Ulaz:                 Ulaz:         Ulaz:       Ulaz:
5               9                     6             3           5
1 2 3 4 5       9 8 7 6 5 4 3 2 1     1 2 3 4 5 6   1 2 3       10 9 8 7 6
2               1                     5             10          -6
Izlaz:          Izlaz:                Izlaz:        Izlaz:      Izlaz:
1 2 3 4         6 4                   -1            -1          -1
\end{verbatim}
\end{center}
\linkresenje{v1.3_01}
\end{Exercise}
\begin{Answer}[ref=v1.3_01]
%\includecode{resenja/1_KontrolaToka/1.3_Petlje/1_01.c}
\end{Answer}


\begin{Exercise}[label=v1.3_01] 
{\em (7 poena)}
Barkod kodira broj proizvoda dodajući mu kontrolnu cifru. Kontrolna
   cifra izračunava se kao poslednja cifra zbira jedinica u zapisu
   svake cifre broja proizvoda. Npr. broj 86012 kodira se kao 1000
   0110 0000 0001 0010 a kontrolna cifra je (1 + 1 + 1 + 1 + 1) mod 10
   = 5.

   Napisati funkciju 
   \begin{center} 
\begin{verbatim}
void kontrolna(char broj_proizvoda[], int *kont)   
\end{verbatim}
   \end{center} koja izračunava kontrolnu cifru broja proizvoda, koji
   se zadaje kao niska, i smešta ga u promenljivu kont. Niska može
   sadržati beline i druge karaktere, ali ih pri izračunavanju
   kontrolne cifre treba ignorisati, samo cifre uzeti u obzir.

   Napisati program koji sa standardnog ulaza učitava liniju teksta kojom je predstavljen broj proizvoda i testira funkciju kontrolna.
   Na standardni izlaz ispisati izračunatu kontrolnu cifru. Maksimalna dužina niske je 100 karaktera.

   Na sistemu se koristi {\tt ASCII} tabela. Ukoliko ne postoji ni jedna cifra u barkodu, onda je kontrolna cifra {\tt 0}.

\begin{center}
\begin{verbatim}
Primer 1:       Primer 2:     Primer 3:     Primer 4:  
Ulaz:           Ulaz:         Ulaz:         Ulaz:
86012           001-223-4     555 555-555   AB-- 123 --BA
Izlaz:          Izlaz:        Izlaz:        Izlaz:
5               6             8             4
\end{verbatim}
\end{center}
\linkresenje{v1.3_01}
\end{Exercise}
\begin{Answer}[ref=v1.3_01]
%\includecode{resenja/1_KontrolaToka/1.3_Petlje/1_01.c}
\end{Answer}


\begin{Exercise}[label=v1.3_01] 
{\em (7 poena)} Napisati program koji ispisuje prosek zbirova svih kolona
matrice čiji su elementi tipa \verb|double|.

Prvo se unosi broj redova matrice $n$, zatim broj kolona matrice
$m$, i onda $n$ redova sa po $m$ elemenata.

Maksimalna veli\v cina matrice je 100 $\times$ 100. Ukoliko je ulaz neispravan (za vrednosti $m$ i $n$) 
prekinuti rad programa i ispisati {\tt -1}.


\begin{center}
\begin{verbatim}
Primer 1:           Primer 2:           Primer 3:           Primer 4:
Ulaz:               Ulaz:               Ulaz:               Ulaz:
4 4                 3 2                 2 4                 3 3
0.2 0.4 0.7 1.3     1.23 4.56           0.1 0.2 0.3 0.4     1 0 0
1.5 1.7 2.2 2.5     0 1                 10.98 7.65 4.32 1   0 1 0
6.3 -1.2 4.4 5.6    7.89 1              Izlaz:              0 0 1
1.6 2.3 2.8 3.5     Izlaz:              6.2375              Izlaz:
Izlaz:              7.8400                                  1.000
8.9500                

\end{verbatim}
\end{center}
\linkresenje{v1.3_01}
\end{Exercise}
\begin{Answer}[ref=v1.3_01]
%\includecode{resenja/1_KontrolaToka/1.3_Petlje/1_01.c}
\end{Answer}

\begin{Exercise}[label=v1.3_01] 
{\em (7 poena)}
Profesor na jednom predmetu je uveo pravilo da njegov predmet
polo\v zio svako ko na ispitu osvoji broj poena koji je veći ili
jednak od proseka poena umanjenog za 10.

\begin{description}
\item[a)] {\em (1 poen)} Definisati strukturu kojom se opisuje svaki student sa indeksom ({\tt indeks-u-obliku-alas-naloga})
          i brojem poena koji je osvojio (ceo broj od 0 do 100).

\item[b)] {\em (6 poena)} Na ulazu ćete dobiti $n$ ($0 \le n \le 300$), broj studenata koji su polagali predmet,
i onda $n$ redova oblika
\begin{verbatim}
indeks-u-obliku-alas-naloga broj-poena-na-ispitu
\end{verbatim}
 Ispisati na standardni izlaz
indekse svih studenata koji su polozili ovaj predmet.  Koristiti strukturu definisanu pod {\bf a)}.

\end{description}

Smatrati da je indeks pravilno zapisan. U slu\v caju lo\v se vrednosti za $n$ ili lo\v se vrednosti za broj poena ispisati -1.
\begin{center}
\begin{verbatim}
Primer 1:           Primer 2:         Primer 3:        Primer 4:        Primer 5:
Ulaz:               Ulaz:             Ulaz:            Ulaz:            Ulaz:
4                   4                 6                4                3
mi12123 80          mr12345 91        mi00001 20       mi11110 100      mi05900 98
mi15512 70          ml54321 80        mi00002 32       mi11111 99       mi13034 120
mi15555 99          mv36925 29        mi00003 96       mi11112 98       mi11234 34
mi13333 40          mi14725 55        mi00004 52       mi11113 87       Izlaz:
Izlaz:              Izlaz:            mi00005 41       Izlaz:           -1
mi12123             mr12345           mi00006 15       mi11110
mi15512             ml54321           Izlaz:           mi11111
mi15555             mi14725           mi00003          mi11112
                                      mi00004          mi11113
                                      mi00005

\end{verbatim}
\end{center}
\linkresenje{v1.3_01}
\end{Exercise}
\begin{Answer}[ref=v1.3_01]
%\includecode{resenja/1_KontrolaToka/1.3_Petlje/1_01.c}
\end{Answer}


\subsection{Programiranje 1, i--smer, Zavr\v{s}ni ispit, februar, 11.02.2016.}

\begin{Exercise}[label=v1.3_01] 
Napisati URM program koji izra\v cunava funkciju:
    $$ f(x)= \begin{cases}
                  2(x - y) & x \geq y \\
                  0 & \text{ina\v ce} \\
                  \end{cases}  $$
\linkresenje{v1.3_01}
\end{Exercise}
\begin{Answer}[ref=v1.3_01]
%\includecode{resenja/1_KontrolaToka/1.3_Petlje/1_01.c}
\end{Answer}


\begin{Exercise}[label=v1.3_01] 
Sa standardnog ulaza se unose celi, nenegativni brojevi sve dok se ne
unese nula. Na standardni izlaz ispisati kvadrat razlike najvećeg i
najmanjeg od unetih brojeva. U slučaju neispravnog ulaza ispisati
-1. Ulaz se smatra neispravnim ukoliko je unet negativan broj ili
ukoliko nije unet ni jedan broj osim nule.


\begin{center}
\begin{verbatim}
Primer 1:       Primer 2:       Primer 3:     Primer 4:  
1 2 3 4 5 0     1 2 3 -4 5 0    1 1 1 1 0     0
16              -1              0             -1
\end{verbatim}
\end{center}
\linkresenje{v1.3_01}
\end{Exercise}
\begin{Answer}[ref=v1.3_01]
%\includecode{resenja/1_KontrolaToka/1.3_Petlje/1_01.c}
\end{Answer}


\begin{Exercise}[label=v1.3_01] 
    \textbf{a)} Napisati funkciju
    \begin{verbatim}
    void mutacije(char s1[], char s2[], int *br)   
    \end{verbatim}
    koja za navedene niske \textbf{s1} i \textbf{s2} iste dužine proverava na koliko mesta se karakteri niski razlikuju i rezultat upisuje u promenljivu \textbf{br}.
    Pri pore\dj enju ignorisati beline.
    \\
    \\
    \textbf{b)} Napisati program koji sa standardnog ulaza učitava dve DNK sekvence (niske karaktera A, T, C ili G) iste dužine i testira funkciju \textbf{mutacije} ispisujući vrednost promenljive \textbf{br} na standardni izlaz. Maksimalna du\v zina niski je 100 karaktera. U slučaju neispravnog ulaza ispisati -1. Ulaz se smatra neispravnim ukoliko neka od niski sadrži karakter koji ne pripada skupu \{A, T, C, G\} i nije belina
   ili je jedna niska duža od druge.
     


\begin{center}
\begin{verbatim}
Primer 1:       Primer 2:        Primer 3:        Primer 4:  

Ulaz:           Ulaz:            Ulaz:            Ulaz:
AGTC CGCT AGT   ATCG ATCG ATCG   AGTTGTTGT ATGX   AGGGATGGATGAG
AGTCC GC TAGT   ACCG ATGC ATCA   TTGTATGGA GGAT   TTGATGACGT

Izlaz:          Izlaz:           Izlaz:           Izlaz:
0               3                -1               -1
\end{verbatim}
\end{center}
\linkresenje{v1.3_01}
\end{Exercise}
\begin{Answer}[ref=v1.3_01]
%\includecode{resenja/1_KontrolaToka/1.3_Petlje/1_01.c}
\end{Answer}

\begin{Exercise}[label=v1.3_01] 
    Krtice su organizovano napale baštu šargarepa. Farmer je napravio
  pravougaonu mapu bašte dimenzija n x m, gde je znakom \textbf{X}
  označio polje na kome se nalazi krtičnjak, dok je netaknuta polja
  označio znakom \textbf{-} . Kako je bašta velika, farmer želi da bez
  mnogo muke izračuna broj krtičnjaka u proizvoljnom pravougaonom delu
  svoje bašte. Sa standardnog ulaza unose se dimenzije mape \textbf{n}
  i \textbf{m}, zatim mapa bašte sa oznakama krtičnjaka i netaknutih
  polja. Nakon toga farmer zadaje koordinate
  (\textbf{i1}, \textbf{j1}) i (\textbf{i2}, \textbf{j2}) koje
  označavaju gornji levi i donji desni ugao pravouganika za koji
  farmer pita koliko krtičnjaka je obuhvaćeno na mapi tim
  pravouganikom. Na standardni izlaz ispisati broj krtičnjaka u
  zadatom pravouganiku. Maksimalna dimenzija mape je 100 x 100. U
  slučaju neispravnih koordinata uglova pravouganika, neispravnih
  dimenzija mape ili oznaka na tabli van skupa \{ X, - \} na
  standardni izlaz ispisati -1.
     


\begin{center}
\begin{verbatim}
Primer 1:        Primer 2:        Primer 3:        Primer 4:  

Ulaz:            Ulaz:            Ulaz:            Ulaz:
4 4              4 4              4 4              4 4
- - X -          - - X K          - X - X          - X - X
X - - -          - X - -          X - X -          X - X -
- X - X          X - - -          - X - X          - X - X
X - X -          X X - X          X - X -          X - X -
0 1              1 2              3 4              0 0
2 2              3 4              1 2              3 3

Izlaz:           Izlaz:           Izlaz:           Izlaz:
2                -1               -1               8
\end{verbatim}
\end{center}
\linkresenje{v1.3_01}
\end{Exercise}
\begin{Answer}[ref=v1.3_01]
%\includecode{resenja/1_KontrolaToka/1.3_Petlje/1_01.c}
\end{Answer}

\begin{Exercise}[label=v1.3_01] 
    Vlasnik pekare želi da utvrdi koliko je isplativa prodaja njegovog
    najskupljeg peciva.
    
    \textbf{a)} Definisati strukturu \textbf{Pecivo} koja sadrži
    podatke o imenu peciva (najviše 50 karaktera) i ceni peciva
    (realan broj tipa double).  \\ \\ \textbf{b)} Sa standardnog ulaza
    se unosi broj \textbf{n} a zatim mesečni obračun sa n prodatih
    komada peciva, pri \v cemu je naziv peciva u jednom redu a cena u
    narednom. Na standardni izlaz ispisati ukupnu zaradu od prodaje
    najskupljeg peciva zaokruženu na dva decimalna mesta. U slučaju
    negativne cene peciva ili u slu\v caju da je n manje ili jednako
    nuli ispisati -1. Pretpostaviti da će samo jedna vrsta peciva
    imati maksimalnu cenu.
     


\begin{center}
\begin{verbatim}
Primer 1:              Primer 2:        Primer 3:        Primer 4:  

Ulaz:                  Ulaz:            Ulaz:            Ulaz:
5                      3                -1               5
burek sa mesom         mafin                             kroasan sa dzemom
100.50                 -50.03                            49.99
buhtla sa cokoladom    krofna                            kroasan sa dzemom
50.00                  56.00                             49.99
burek sa mesom         krofna                            kroasan sa dzemom
100.50                 56.00                             49.99
rol virsla                                               kroasan sa dzemom
75.00                                                    49.99
kroasan sa kremom                                        kroasan sa dzemom
60.00                                                    49.99

Izlaz:                 Izlaz            Izlaz:           Izlaz:
201.00                 -1               -1               249.95
\end{verbatim}
\end{center}
\linkresenje{v1.3_01}
\end{Exercise}
\begin{Answer}[ref=v1.3_01]
%\includecode{resenja/1_KontrolaToka/1.3_Petlje/1_01.c}
\end{Answer}



\subsection{1. Grupa, I smer, Programiranje 1 2015/2016, ispit, jun}

\begin{Exercise}[label=v1.3_01] 
Napisati URM program koji izra\v cunava funkciju:
    $$ f(x)= \begin{cases}
                  x - y + 2  & x + 2 \geq y \\
                  0 & \text{ina\v ce} \\
                  \end{cases}  $$
\linkresenje{v1.3_01}
\end{Exercise}
\begin{Answer}[ref=v1.3_01]
%\includecode{resenja/1_KontrolaToka/1.3_Petlje/1_01.c}
\end{Answer}


\begin{Exercise}[label=v1.3_01] 
Napisati program koji sa standardnog ulaza u\v citava prvo pozitivan
  ceo broj $n$ ( $0 < n \le 99$), a zatim i $n$ celih brojeva i izra\v
  cunava zbir parnih. Izra\v cunati zbir ispisati na standardni
  izlaz. U slu\v caju gre\v ske (za $n \leq 0$ ili $n \ge 100 $ ) na
  standardni izlaz ispisati -1.


\small
\begin{tabular}{ |l|l|l|l|l| }
\hline 
  Ulaz & 5 1 2 3 4 5  & 5 -1 -2 -3 -4 -5 & 3 10 -10 10 & -3 1 2 3 \\ \hline 
  Izlaz & 6 & -6 & 10 & -1 \\ \hline 
\end{tabular}
\normalsize
\linkresenje{v1.3_01}
\end{Exercise}
\begin{Answer}[ref=v1.3_01]
%\includecode{resenja/1_KontrolaToka/1.3_Petlje/1_01.c}
\end{Answer}

\begin{Exercise}[label=v1.3_01] 
Napisati funkciju \emph{void f(char s[], char c, int *prva, int*
  poslednja)} koja u datoj nisci {\em s} pronalizi indekse prvog i
  poslednjeg pojavljivanja datog karaktera {\em c} i dobijene
  vrednosti redom sme\v sta u promenljive {\em prva} i {\em
    poslednja}. Ukoliko se karakter ne pojavljuje u nisci, obe
  vrednosti postaviti na {\em -1}. 

  Potom napisati program koji sa standardnog ulaza u\v citava
  karaktersku nisku (du\v zine ne ve\' ce od 150 karaktera) i jedan
  karakter i nakon toga poziva funkciju {\em f}, a potom na standarni
  izlaz ispisuje indekse prvog i poslednjeg pojavljivanja datog
  karaktera u datoj nisci. Pretpostaviti da je ulaz u ispravnom
  formatu.

\small
\begin{tabular}{ |l|l|l|l|l| }
\hline 
  Ulaz & ucionica i  & ucionica u & ucionica o & ucionica p \\ \hline 
  Izlaz & 2 5 & 0 0 & 3 3 & -1 -1 \\ \hline 
\end{tabular}
\normalsize
\linkresenje{v1.3_01}
\end{Exercise}
\begin{Answer}[ref=v1.3_01]
%\includecode{resenja/1_KontrolaToka/1.3_Petlje/1_01.c}
\end{Answer}

\begin{Exercise}[label=v1.3_01] 
Sa standarnog ulaza se zadaje dimenzija kvadratne matrice $n$ (
  $0 < n \le 99$), a zatim elementi matrice koji su celi brojevi. Na
  standardni izlaz ispisati redni broj vrste koja ima najve\' ci zbir
  elemenata. U slu\v caju gre\v ske (za $n \leq 0$ ili $n \ge 100 $ )
  na standardni izlaz ispisati -1.


\iffalse
\small
\begin{tabular}{ |l|l|l|l|l| }
\hline 
  Ulaz & 
  \mlcell{3 \\ 1 2 3 \\ 4 5 6 \\ 7 8 9 }&
  \mlcell{5 \\ 1 0 0 0 0 \\ 1 1 0 0 0 \\ 1 1 1 0 0 \\ 1 1 1 1 0 \\ 1 1 1 1 1} & 
  \mlcell{5 \\ 1 0 -1 0 -1 \\ -1 0 -1 5 0 \\ 1 -1 -1 0 1 \\ 1 0 -3 0 -1 \\ 0 -1 0 -1 0} & 
  \mlcell{-3 \\ 1 2 3 \\ 4 5 6 \\ 7 8 9 }\\ 
  \hline 
  Izlaz &
  \mlcell{2} &  
  \mlcell{4} &  
  \mlcell{1}&  
  -1\\ 
  \hline 
\end{tabular}
\fi
\normalsize
\linkresenje{v1.3_01}
\end{Exercise}
\begin{Answer}[ref=v1.3_01]
%\includecode{resenja/1_KontrolaToka/1.3_Petlje/1_01.c}
\end{Answer}

\begin{Exercise}[label=v1.3_01] 
Definisati strukturu $Tacka$ za predstavljanje ta\v caka u ravni
  sa koordinatama tipa $double$. Sa standardnog ulaza se ucitava broj
  $n$ ( $1 < n \le 99$), zatim niz od $n$ ta\v caka tako sto se unosi prvo
  $x$, pa $y$ koordinata za svaku ta\v cku. Za zadate ta\v cke
  ispisati na standardni izlaz du\v zinu najdu\v ze du\v zi koja se
  mo\v ze obrazovati od neke dve ta\v cke iz u\v citanog
  niza. Rezultat ispisati na dve decimale. Du\v zina du\v zi izme\dj u
  ta\v caka $a(x1; y1)$ i $b(x2; y2)$ se ra\v cuna po
  formuli $$\sqrt{(x1-x2)^2+(y1-y2)^2}$$ U slu\v caju gre\v ske (za $n
  \leq 1$ ili $n \ge 100 $ ) na standardni izlaz ispisati -1.


\iffalse
\small
\begin{tabular}{ |l|l|l|l|l| }
\hline 
  Ulaz & 
  \mlcell{2 \\ 3 4 \\ 0 0 }&
  \mlcell{3 \\ 0.5 0.3  \\ -5 3 \\ 3 4 } & 
  \mlcell{4 \\ 1.2 1.2  \\ 1.2 1.2 \\ 1.2 1.2 \\ 1.2 1.2} & 
  \mlcell{1 \\ 0 0 }\\ 
  \hline 
  Izlaz &
  \mlcell{5.00} &  
  \mlcell{8.06} &  
  \mlcell{0.00}&  
  -1\\ 
  \hline 
\end{tabular}
\fi
\normalsize
\linkresenje{v1.3_01}
\end{Exercise}
\begin{Answer}[ref=v1.3_01]
%\includecode{resenja/1_KontrolaToka/1.3_Petlje/1_01.c}
\end{Answer}


\subsection{Praktični deo ispita, jun ...}

\section{Rešenja}
\shipoutAnswer
