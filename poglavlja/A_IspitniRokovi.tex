\appendix
\chapter{Ispitni rokovi}

\section{Opšta grupa}

% I rok
\subsection{Praktični deo ispita,  januar 2019.}

\begin{Exercise}[label=A_o_1_1] 
Napisati program koji učitava četvorocifrene brojeve do unosa broja $0$, a zatim ispisuje one brojeve kojima je cifra desetica najveća cifra u zapisu. Ukoliko nema takvih brojeva među unetima, ispisati broj $0$. U slučaju greške, ispisati $-1$ na standardni izlaz za greške.

\begin{miditest}
\begin{test}{1}
#\naslovUlaz#
#\ulaz{9523 -8542 3232 -9999 -1121 1576 0}#
#\naslovIzlaz#
#\izlaz{3232 -9999 -1121 1576}#
\end{test}
\end{miditest}
\begin{miditest}
\begin{test}{2}
#\naslovUlaz#
#\ulaz{4596 1234 9631 -120 0}#
#\naslovIzlazZaGresku#
#\izlaz{4596 -1}#
\end{test}
\end{miditest}

\begin{miditest}
\begin{test}{2}
#\naslovUlaz#
#\ulaz{9876 2258 -4579 4689 -5567 6630 1200 5204 0}#
#\naslovIzlaz#
#\izlaz{0}#
\end{test}
\end{miditest}

\linkresenje{A_o_1_1}
\end{Exercise}

\ifresenja
\begin{Answer}[ref=A_o_1_1]
\includecode{resenja/A_IspitniRokovi/rok_o_26.01.2019/1.c}
\end{Answer}
\fi

\begin{Exercise}[label=A_o_1_2] 
Napisati program koji pomaže korisniku  da ''šifruje'' svoju elektronsku adresu kako ne bi dobijao nepoželjne poruke. ''Šifrovanje'' adrese se vrši tako što se znak @ zameni sa $[AT]$. Elektronska adresa se učitava kao niska maksimalne dužine 100 karaktera sa standardnog ulaza,  a šifrovana adresa se ispisuje na standardni izlaz. U slučaju da elektronska adresa nije ispravno zadata ispisati $-1$ na standardni izlaz za greške. 

\begin{minitest}
\begin{test}{1}
#\naslovUlaz#
#\ulaz{korisnik@gmail.com}#
#\naslovIzlaz#
#\izlaz{korisnik[AT]gmail.com}#
\end{test}
\end{minitest}
\begin{minitest}
\begin{test}{2}
#\naslovUlaz#
#\ulaz{student@matf.bg.ac.rs}#
#\naslovIzlaz#
#\izlaz{student[AT]matf.bg.ac.rs}#
\end{test}
\end{minitest}
\begin{minitest}
\begin{test}{3}
#\naslovUlaz#
#\ulaz{pogresnaadresayahoo.com}#
#\naslovIzlazZaGresku#
#\izlaz{-1}#
\end{test}
\end{minitest}


\linkresenje{A_o_1_2}
\end{Exercise}

\ifresenja
\begin{Answer}[ref=A_o_1_2]
\includecode{resenja/A_IspitniRokovi/rok_o_26.01.2019/2.c}
\end{Answer}
\fi

\begin{Exercise}[label=A_o_1_3] 
Definisati strukturu \textit{Hemijski\_element} koja sadrži naziv elementa (nisku dužine najviše 20 karaktera), oznaku elementa (nisku dužine najviše 2 karaktera) i broj neutrona (ceo broj). Napisati program koji učitava podatke o hemijskim elementima do unosa reči \textbf{kraj}, a potom još jedan naziv elementa i na standardni izlaz ispisuje oznaku i broj neutrona tog elementa. Ukoliko element nije pronađen među učitanim podacima, ispisati -1. \\
\napomena{Pretpostaviti da neće biti uneto više od 120 elemenata, kao i da su podaci o hemijskim elementima ispravno zadati.} 

\begin{minitest}
\begin{test}{1}
#\naslovUlaz#
#\ulaz{kalcijum Ca 20}#
#\ulaz{cink Zn 35}#
#\ulaz{fosfor P 16}#
#\ulaz{kraj}#
#\ulaz{fosfor}#
#\naslovIzlaz#
#\izlaz{P 16}#
\end{test}
\end{minitest}
\begin{minitest}
\begin{test}{2}
#\naslovUlaz#
#\ulaz{nikl Ni 31}#
#\ulaz{bor B 6}#
#\ulaz{kripton Kr 48}#
#\ulaz{natrijum Na 12}#
#\ulaz{kraj}#
#\ulaz{hrom}#
#\naslovIzlazZaGresku#
#\izlaz{-1}#
\end{test}
\end{minitest}
\begin{minitest}
\begin{test}{3}
#\naslovUlaz#
#\ulaz{litijum Li 4}#
#\ulaz{ugljenik C 6}#
#\ulaz{aluminijum Al 14}#
#\ulaz{srebro Ag 61}#
#\ulaz{gvozdje Fe 40}#
#\ulaz{brom Br 45}#
#\ulaz{kraj}#
#\ulaz{ugljenik}#
#\naslovIzlaz#
#\izlaz{C 6}#
\end{test}
\end{minitest}

\linkresenje{A_o_1_3}
\end{Exercise}

\ifresenja
\begin{Answer}[ref=A_o_1_3]
\includecode{resenja/A_IspitniRokovi/rok_o_26.01.2019/3.c}
\end{Answer}
\fi


\begin{Exercise}[label=A_o_1_4] 
U datoteci \textit{pesme.txt} dat je ceo broj $n$ koji označava broj pesama, a potom i $n$ redova sa podacima o pesmama. U svakom redu naveden je naziv pesme i njen žanr (niske bez belina, dužine najviše 30 karaktera). Napisati program koji učitava podatke iz datoteke, a zatim, u zavisnosti od opcije koja se zadaje kao argument komandne linije, obrađuje podatke na sledeći način: 
\begin{itemize}
\item ukoliko je zadata opcija \textbf{-p}, učitava se sa standardnog ulaza jedan karakter i na standardni izlaz ispisuju svi nazivi pesama koji počinju zadatim karakterom;
\item ukoliko je zadata opcija \textbf{-z}, učitava se sa standardnog ulaza niska koja predstavlja žanr pesme i na standardni izlaz ispisuju nazivi svih pesama odabranog žanra.
\end{itemize}

Prilikom odabira pesama za ispis, zanemariti veličinu slova. U slučaju greške, ispisati $-1$ na standardni izlaz za greške. 

\begin{miditest}
\begin{test}{1}
#\poziv{./a.out -p}#
#\naslovDat{pesme.txt}#
#\datoteka{7}#
#\datoteka{BohemianRhapsody rock}#
#\datoteka{RollingInTheDeep pop}#
#\datoteka{StairwayToHeaven rock}#
#\datoteka{BeatIt pop}#
#\datoteka{SoWhat jazz}#
#\datoteka{MyFunnyValentine jazz}#
#\datoteka{Smooth pop}#
#\naslovUlaz#
#\ulaz{S}#
#\naslovIzlaz#
#\izlaz{StairwayToHeaven}#
#\izlaz{SoWhat}#
#\izlaz{Smooth}#
\end{test}
\end{miditest}
\begin{miditest}
\begin{test}{2}
#\poziv{./a.out -z}#
#\naslovDat{pesme.txt}#
#\datoteka{7}#
#\datoteka{BohemianRhapsody rock}#
#\datoteka{RollingInTheDeep pop}#
#\datoteka{StairwayToHeaven rock}#
#\datoteka{BeatIt pop}#
#\datoteka{SoWhat jazz}#
#\datoteka{MyFunnyValentine jazz}#
#\datoteka{Smooth pop}#
#\naslovUlaz#
#\ulaz{pop}#
#\naslovIzlaz#
#\izlaz{RollingInTheDeep}#
#\izlaz{BeatIt}#
#\izlaz{Smooth}#
\end{test}
\end{miditest}

\begin{minitest}
\begin{test}{3}
#\poziv{./a.out -x}#
#\naslovIzlazZaGresku#
#\izlaz{-1}#
\end{test}
\end{minitest}
\begin{minitest}
\begin{test}{4}
#\poziv{./a.out -p -z }#
#\naslovIzlazZaGresku#
#\izlaz{-1}#
\end{test}
\end{minitest}
\begin{minitest}
\begin{test}{5}
#\poziv{./a.out }#
#\naslovIzlazZaGresku#
#\izlaz{-1}#
\end{test}
\end{minitest}

\linkresenje{A_o_1_4}
\end{Exercise}

\ifresenja
\begin{Answer}[ref=A_o_1_4]
\includecode{resenja/A_IspitniRokovi/rok_o_26.01.2019/4.c}
\end{Answer}


% II rok
\subsection{Praktični deo ispita,  februar 2019.}

\begin{Exercise}[label=A_o_2_1] 
Napisati program koji učitava pozitivan četvorocifren broj $n$, a zatim na standardni izlaz ispisuje zbir onih cifara broja $n$ koje su po vrednosti veće od aritmetičke sredine svih cifara broja $n$. U slučaju greške, ispisati $-1$ na standardni izlaz za greške. 

\begin{miniminitest}
\begin{test}{1}
#\naslovUlaz#
#\ulaz{1234}#
#\naslovIzlaz#
#\izlaz{7}#
\end{test}
\end{miniminitest}
\begin{miniminitest}
\begin{test}{2}
#\naslovUlaz#
#\ulaz{6745}#
#\naslovIzlaz#
#\izlaz{13}#
\end{test}
\end{miniminitest}
\begin{miniminitest}
\begin{test}{3}
#\naslovUlaz#
#\ulaz{100}#
#\naslovIzlazZaGresku#
#\izlaz{-1}#
\end{test}
\end{miniminitest}
\begin{miniminitest}
\begin{test}{4}
#\naslovUlaz#
#\ulaz{-1234}#
#\naslovIzlazZaGresku#
#\izlaz{-1}#
\end{test}
\end{miniminitest}

\linkresenje{A_o_2_1}
\end{Exercise}

\ifresenja
\begin{Answer}[ref=A_o_2_1]
\includecode{resenja/A_IspitniRokovi/rok_o_09.02.2019/1.c}
\end{Answer}
\fi

\begin{Exercise}[label=A_o_2_2] 
Napisati program koji učitava nisku $s$ parne dužine od najviše 20 karaktera i na standardni izlaz ispisuje nisku koja se dobija nadovezivanjem karaktera prve polovine niske $s$ na drugu polovinu niske $s$. U slučaju greške, ispisati $-1$ na standardni izlaz za greške. 

\begin{miniminitest}
\begin{test}{1}
#\naslovUlaz#
#\ulaz{Beograde}#
#\naslovIzlaz#
#\izlaz{radeBeog}#
\end{test}
\end{miniminitest}
\begin{miniminitest}
\begin{test}{2}
#\naslovUlaz#
#\ulaz{matematika}#
#\naslovIzlaz#
#\izlaz{atikamatem}#
\end{test}
\end{miniminitest}
\begin{miniminitest}
\begin{test}{3}
#\naslovUlaz#
#\ulaz{1234}#
#\naslovIzlaz#
#\izlaz{3412}#
\end{test}
\end{miniminitest}
\begin{miniminitest}
\begin{test}{4}
#\naslovUlaz#
#\ulaz{abc1234}#
#\naslovIzlazZaGresku#
#\izlaz{-1}#
\end{test}
\end{miniminitest}

\linkresenje{A_o_2_2}
\end{Exercise}

\ifresenja
\begin{Answer}[ref=A_o_2_2]
\includecode{resenja/A_IspitniRokovi/rok_o_09.02.2019/2.c}
\end{Answer}
\fi


\begin{Exercise}[label=A_o_2_3] 
Napisati program koji čita sadržaj datoteke \textit{ulaz.txt} i ispisuje na standardni izlaz sve niske datoteke koje predstavljaju cele brojeve. U slučaju greške, ispisati -1 na standardni izlaz za greške. 

\begin{minitest}
\begin{test}{1}

#\poziv{./a.out}#
#\naslovDat{ulaz.txt}#
#\ulaz{123 ab1 2ab -23}#
#\naslovIzlaz#
#\izlaz{123 -23}#
\end{test}
\end{minitest}
\begin{minitest}
\begin{test}{2}

#\poziv{./a.out}#
#\naslovDat{ulaz.txt}#
#\ulaz{145as 25gf 265 478 65 -96}#
#\naslovIzlaz#
#\izlaz{265 478 65 -96}#
\end{test}
\end{minitest}
\begin{minitest}
\begin{test}{3}

#\poziv{./a.out}#
#\naslovDat{ulaz.txt}#
#\ulaz{Ovde nema brojeva}#
#\naslovIzlaz#
#\izlaz{}#
\end{test}
\end{minitest}

\begin{minitest}
\begin{test}{4}

#\poziv{./a.out}#
#\naslovDat{ulaz.txt ne postoji!}#
#\naslovIzlazZaGresku#
#\izlaz{-1}#
\end{test}
\end{minitest}

\linkresenje{A_o_2_3}
\end{Exercise}

\ifresenja
\begin{Answer}[ref=A_o_2_3]
\includecode{resenja/A_IspitniRokovi/rok_o_09.02.2019/3.c}
\end{Answer}
\fi

\begin{Exercise}[label=A_o_2_4] 
Napisati program koji sa standardnog ulaza učitava podatke o osvajačima takmičenja. Za svako takmičenje se redom zadaju godina takmičenja (pozitivan ceo broj) i ime osvajača (niska od najviše $30$ karaktera bez belina). Program treba da ispiše:
\begin{itemize}
\item ako je navedena opcija -y kao prvi argument komandne linije, ime osvajača takmičenja za godinu koja se navodi kao drugi argument
 \item ako je navedena opcija -w kao prvi argument komandne linije, sve godine u kojima je takmičar čije se ime navodi kao drugi argument komande linije osvajao takmičenje.
\end{itemize}
U slučaju greške, ispisati $-1$ na standardni izlaz za greške. 

\napomena{Podrazumevati da su ulazni podaci o takmičenjima ispravni. Broj osvajača nije unapred poznat.}

\begin{miditest}
\begin{test}{1}
#\poziv{./a.out -y 2016}#
#\naslovUlaz#
#\ulaz{2011 ManUtd}#
#\ulaz{2012 ManCity}#
#\ulaz{2013 ManUtd}#
#\ulaz{2014 ManCity}# 
#\ulaz{2015 Chelsea}#
#\ulaz{2016 Leicester}#
#\ulaz{2017 Chelsea}#
#\ulaz{2018 ManCity}#
#\naslovIzlaz#
#\izlaz{Leicester}#
\end{test}
\end{miditest}
\begin{miditest}
\begin{test}{2}
#\poziv{./a.out -w RealMadrid}#
#\naslovUlaz#
#\ulaz{2011 Barcelona}#
#\ulaz{2012 Chelsea}#
#\ulaz{2013 BayernMunich}#
#\ulaz{2014 RealMadrid}#
#\ulaz{2015 Barcelona}#
#\ulaz{2016 RealMadrid}#
#\ulaz{2017 RealMadrid}#
#\ulaz{2018 RealMadrid}#
#\naslovIzlaz#
#\izlaz{2014 2016 2017 2018}#
\end{test}
\end{miditest}

\begin{miditest}
\begin{test}{3}

#\poziv{./a.out -s 2001}#
#\naslovIzlazZaGresku#
#\izlaz{-1}#
\end{test}
\end{miditest}
\begin{miditest}
\begin{test}{4}

#\poziv{./a.out -x}#
#\naslovIzlazZaGresku#
#\izlaz{-1}#
\end{test}
\end{miditest}

\begin{miditest}
\begin{test}{5}

#\poziv{./a.out -s 2012 2000}#
#\naslovIzlazZaGresku#
#\izlaz{-1}#
\end{test}
\end{miditest}
\begin{miditest}
\begin{test}{6}

#\poziv{./a.out -y 2005 -w RealMadrid}#
#\naslovIzlazZaGresku#
#\izlaz{-1}#
\end{test}
\end{miditest}

\linkresenje{A_o_2_4}
\end{Exercise}

\ifresenja
\begin{Answer}[ref=A_o_2_4]
\includecode{resenja/A_IspitniRokovi/rok_o_09.02.2019/4.c}
\end{Answer}


\section{I smer}

% III rok
\subsection{Praktični deo ispita,  januar 2019.}

\begin{Exercise}[label=A_i_1_1] 
Napisati program koji učitava cele trocifrene brojeve sve do kraja ulaza i na standardni izlaz ispisuje one čije su cifre uređene strogo rastuće (cifre se čitaju sa leva na desno). U slučaju greške, ispisati $-1$ i prekinuti izvršavanje programa. 

\begin{minitest}
\begin{test}{1}
#\naslovUlaz#
#\ulaz{-532 236 100 -555 546}#
#\naslovIzlaz#
#\izlaz{236}#
\end{test}
\end{minitest}
\begin{miniminitest}
\begin{test}{2}
#\naslovUlaz#
#\ulaz{123 -123 321 -321}#
#\naslovIzlaz#
#\izlaz{123 -123}#
\end{test}
\end{miniminitest}
\begin{miniminitest}
\begin{test}{3}
#\naslovUlaz#
#\ulaz{258 695 -1234}#
#\naslovIzlaz#
#\izlaz{258 -1}#
\end{test}
\end{miniminitest}
\begin{miniminitest}
\begin{test}{4}
#\naslovUlaz#
#\ulaz{14}#
#\naslovIzlaz#
#\izlaz{-1}#
\end{test}
\end{miniminitest}

\linkresenje{A_i_1_1}
\end{Exercise}

\ifresenja
\begin{Answer}[ref=A_i_1_1]
\includecode{resenja/A_IspitniRokovi/rok_i_januar/1.c}
\end{Answer}
\fi

\begin{Exercise}[label=A_i_1_2] 
Napisati program koji sa standardnog ulaza učitava reč $s$ maksimalne dužine $20$ karaktera (bez belina), a zatim karakter koji predstavlja način modifikacije učitane niske:
 \begin{itemize}
\item ukoliko je učitan karakter $m$, sve karaktere reči $s$ koji su mala slova, pretvoriti u odgovarajuća velika 
\item ukoliko je učitan karakter $v$, sve karaktere reči $s$ koji su velika slova, pretvoriti u odgovarajuća mala
\item ukolko je učitan karakter $o$, ne menjati karaktere reči $s$
 \end{itemize}
Na standardni izlaz ispisati nisku nakon modifikacije. U slučaju greške, ispisati $-1$ na standardni izlaz i prekinuti izvršavanje programa. 

\begin{miniminitest}
\begin{test}{1}
#\naslovUlaz#
#\ulaz{sreca m}#
#\naslovIzlaz#
#\izlaz{SRECA}#
\end{test}
\end{miniminitest}
\begin{miniminitest}
\begin{test}{2}
#\naslovUlaz#
#\ulaz{IspiT v}#
#\naslovIzlaz#
#\izlaz{ISPIT}#
\end{test}
\end{miniminitest}
\begin{miniminitest}
\begin{test}{3}
#\naslovUlaz#
#\ulaz{Rec o}#
#\naslovIzlaz#
#\izlaz{Rec}#
\end{test}
\end{miniminitest}
\begin{miniminitest}
\begin{test}{4}
#\naslovUlaz#
#\ulaz{PROgram x}#
#\naslovIzlaz#
#\izlaz{-1}#
\end{test}
\end{miniminitest}

\linkresenje{A_i_1_2}
\end{Exercise}

\ifresenja
\begin{Answer}[ref=A_i_1_2]
\includecode{resenja/A_IspitniRokovi/rok_i_januar/2.c}
\end{Answer}
\fi


\begin{Exercise}[label=A_i_1_3] 
Napisati program za praćenje rezultata automobilske trke. Na takmičenju učestvuje $n$ ($n\geq 3$) takmičara u $m$ ($m\geq 2$) trka. Program prvo učitava broj takmičara i trka, a zatim za svakog od $n$ takmičara vreme u sekundama u svakoj od $m$ trka. Pretpostaviti da neće biti više od $100$ takmičara i $100$ trka. Vremena čuvati u matrici dimenzije $n \times m$ tako da element ($i$, $j$) predstavlja vreme koje je takmičar $i$ postigao u $j$-toj trci. Na standardni izlaz ispisati redne brojeve takmičara (brojeći ih od 0) koji su pobedili u trkama (bili najbrži), redom za svaku trku. Pretpostaviti da neće biti više takmičara sa istim prolaznim vremenom po trci. U slučaju greške, ispisati $-1$ na standardni izlaz i prekinuti izvršavanje programa. 

\begin{miniminitest}
\begin{test}{1}
#\naslovUlaz#
#\ulaz{3 3}#
#\ulaz{192.9 87.8 109.102}#
#\ulaz{181.2 92.1 102.4}#
#\ulaz{151.1 87.9 118.9}#
#\naslovIzlaz#
#\izlaz{2 0 1}#
\end{test}
\end{miniminitest}
\begin{minitest}
\begin{test}{2}
#\naslovUlaz#
#\ulaz{3 4}#
#\ulaz{51.3 184.94 121.7 99.51}#
#\ulaz{50.9 182.71 119.2 99.2}#
#\ulaz{51.2 192.11 122.9 100.1}#
#\naslovIzlaz#
#\izlaz{1 1 1 1}#
\end{test}
\end{minitest}
\begin{miniminitest}
\begin{test}{3}
#\naslovUlaz#
#\ulaz{4 3}#
#\ulaz{113.5 145.2 -14.5}#
#\naslovIzlaz#
#\izlaz{-1}#
\end{test}
\end{miniminitest}
\begin{miniminitest}
\begin{test}{4}
#\naslovUlaz#
#\ulaz{4 -3}#
#\naslovIzlaz#
#\izlaz{-1}#
\end{test}
\end{miniminitest}

\linkresenje{A_i_1_3}
\end{Exercise}

\ifresenja
\begin{Answer}[ref=A_i_1_3]
\includecode{resenja/A_IspitniRokovi/rok_i_januar/3.c}
\end{Answer}
\fi

\begin{Exercise}[label=A_i_1_4] 
Definisati strukturu sa nazivom $Kutija$ koja sadrži dužinu, širinu i visinu kutije u centimetrima (pozitivni celi brojevi). Napisati program koji učitava pozitivan ceo broj $n$ ($n \leq 100$), a zatim i podatke o $n$ kutija. Nakon toga, program treba da ispiše zapreminu kutije u koju se može smestiti svaka od preostalih $n-1$ kutija pojedinačno. Pretpostaviti da neće biti više takvih kutija, a ukoliko takva kutija ne postoji, ispisati $0$. U slučaju greške, ispisati $-1$ na standardni izlaz i prekinuti izvršavanje programa.  

\napomena{Da bi jedna kutija (sa celobrojnim dimenzijama) stala u drugu, svaka od dimenzija te kutije (dužina, širina i visina redom) mora biti manja barem 1 centimetar od odgovarajućih dimenzija druge kutije. Prilikom smeštanja jedne kutije u drugu nema obrtanja kutije.}

\begin{miniminitest}
\begin{test}{1}
#\naslovUlaz#
#\ulaz{4}#
#\ulaz{15 2 9}#
#\ulaz{185 27 12}#
#\ulaz{16 21 10}# 
#\ulaz{120 12 3}#
#\naslovIzlaz#
#\izlaz{59940}#
\end{test}
\end{miniminitest}
\begin{miniminitest}
\begin{test}{2}
#\naslovUlaz#
#\ulaz{3}#
#\ulaz{9 18 2}#
#\ulaz{21 5 3}#
#\ulaz{3 15 5}#
#\naslovIzlaz#
#\izlaz{0}#
\end{test}
\end{miniminitest}
\begin{miniminitest}
\begin{test}{3}
#\naslovUlaz#
#\ulaz{-3}#
#\naslovIzlaz#
#\izlaz{-1}#
\end{test}
\end{miniminitest}
\begin{miniminitest}
\begin{test}{4}
#\naslovUlaz#
#\ulaz{3}#
#\ulaz{1 2 3}#
#\ulaz{8 9 -5}#
#\naslovIzlaz#
#\izlaz{-1}#
\end{test}
\end{miniminitest}

\linkresenje{A_i_1_4}
\end{Exercise}

\ifresenja
\begin{Answer}[ref=A_i_1_4]
\includecode{resenja/A_IspitniRokovi/rok_i_januar/4.c}
\end{Answer}


% IV rok
\subsection{Praktični deo ispita, februar 2019.}

\begin{Exercise}[label=A_i_2_1] 
Napisati program koji učitava cele trocifrene brojeve sve do kraja ulaza i na standardni izlaz ispisuje one brojeve čija je cifra desetica jednaka aritmetičkoj sredini cifara stotina i jedinica. 
U slučaju greške, ispisati $-1$ na standardni izlaz i prekinuti izvršavanje programa. 

\begin{minitest}
\begin{test}{1}
#\naslovUlaz#
#\ulaz{543 236 100 -555 546}#
#\naslovIzlaz#
#\izlaz{543 -555}#
\end{test}
\end{minitest}
\begin{miniminitest}
\begin{test}{2}
#\naslovUlaz#
#\ulaz{402 -402 103 -103}#
#\naslovIzlaz#
#\izlaz{}#
\end{test}
\end{miniminitest}
\begin{miniminitest}
\begin{test}{3}
#\naslovUlaz#
#\ulaz{-1234}#
#\naslovIzlaz#
#\izlaz{-1}#
\end{test}
\end{miniminitest}
\begin{miniminitest}
\begin{test}{4}
#\naslovUlaz#
#\ulaz{14}#
#\naslovIzlaz#
#\izlaz{-1}#
\end{test}
\end{miniminitest}

\linkresenje{A_i_2_1}
\end{Exercise}

\ifresenja
\begin{Answer}[ref=A_i_2_1]
\includecode{resenja/A_IspitniRokovi/rok_i_februar/1.c}
\end{Answer}
\fi

\begin{Exercise}[label=A_i_2_2] 
Sa standardnog ulaza se učitava niska $s$ maksimalne dužine $30$ karaktera. Napisati program koji na standardni izlaz ispisuje dužinu najduže podniske niske $s$ čiji su karakteri uređeni strogo rastuće po ASCII kodovima čitajući sa leva na desno.  

\begin{miniminitest}
\begin{test}{1}
#\naslovUlaz#
#\ulaz{stolica}#
#\naslovIzlaz#
#\izlaz{2}#
\end{test}
\end{miniminitest}
\begin{miniminitest}
\begin{test}{2}
#\naslovUlaz#
#\ulaz{a12bcABc}#
#\naslovIzlaz#
#\izlaz{4}#
\end{test}
\end{miniminitest}
\begin{miniminitest}
\begin{test}{3}
#\naslovUlaz#
#\ulaz{PPPPPPPP}#
#\naslovIzlaz#
#\izlaz{1}#
\end{test}
\end{miniminitest}
\begin{miniminitest}
\begin{test}{4}
#\naslovUlaz#
#\ulaz{abcdefw}#
#\naslovIzlaz#
#\izlaz{7}#
\end{test}
\end{miniminitest}

\linkresenje{A_i_2_2}
\end{Exercise}

\ifresenja
\begin{Answer}[ref=A_i_2_2]
\includecode{resenja/A_IspitniRokovi/rok_i_februar/2.c}
\end{Answer}
\fi


\begin{Exercise}[label=A_i_2_3] 
Sa standardnog ulaza se učitava neparan prirodan broj $n$ ($n\leq101$), a zatim $n^{2}$ celih brojeva koje treba sačuvati u odgovarajućoj kvadratnoj matrici. Proveriti da li je suma elemenata na glavnoj dijagonali matrice neparna, i ako jeste, na standardni izlaz ispisati vrednost maksimalnog elementa glavne dijagonale. Ako to nije slučaj, ispisati vrednost minimalnog elementa glavne dijagonale. U slučaju greške, ispisati $-1$ na standardni izlaz i prekinuti izvršavanje programa. 

\begin{miniminitest}
\begin{test}{1}
#\naslovUlaz#
#\ulaz{3 }#
#\ulaz{15 6 7}#
#\ulaz{2 -4 -2}#
#\ulaz{3 2 6}#
#\naslovIzlaz#
#\izlaz{15}#
\end{test}
\end{miniminitest}
\begin{miniminitest}
\begin{test}{2}
#\naslovUlaz#
#\ulaz{5}#
#\ulaz{12 6 7 1 2}#
#\ulaz{2 -4 -2 2 0}#
#\ulaz{3 2 6 10 7}#
#\ulaz{3 2 6 12 5}#
#\ulaz{12 6 7 1 2}#
#\naslovIzlaz#
#\izlaz{-4}#
\end{test}
\end{miniminitest}
\begin{miniminitest}
\begin{test}{3}
#\naslovUlaz#
#\ulaz{4}#
#\naslovIzlaz#
#\izlaz{-1}#
\end{test}
\end{miniminitest}
\begin{miniminitest}
\begin{test}{4}
#\naslovUlaz#
#\ulaz{-7}#
#\naslovIzlaz#
#\izlaz{-1}#
\end{test}
\end{miniminitest}

\linkresenje{A_i_2_3}
\end{Exercise}

\ifresenja
\begin{Answer}[ref=A_i_2_3]
\includecode{resenja/A_IspitniRokovi/rok_i_februar/3.c}
\end{Answer}
\fi

\begin{Exercise}[label=A_i_2_4] 
Definisati strukturu sa nazivom $Student$ koja sadrži podatke o studentu: indeks studenta (pozitivan ceo broj), broj poena ostvaren na ispitu (nenegativan realan broj dvostruke tačnosti iz intervala $[0,100]$ ) i oznaku učionice u kojoj je student polagao ispit (niska iz skupa "704", "718", "rlab" i "bim").  Napisati program koji sa standardnog ulaza učitava prirodan broj $n$, a zatim podatke o $n$ studenata koji su polagali ispit iz Programiranja 1, redom, indeks, broj poena i oznaku učionice. Nakon podataka o studentima se učitava oznaka učionice za koju treba ispisati broj studenata iz te učionice koji su položili ispit. Oznaka učionice se zadaje kao niska od najviše $10$ karaktera.  Pretpostaviti da su podaci o studentima ispravni i da neće biti više od $100$ studenata. U slučaju greške ispisati $-1$ na standardni izlaz i prekinuti izvršavanje programa.  
Student je položio ispit ako je na istom ostvario bar 51 poen.  

\begin{minitest}
\begin{test}{1}
#\naslovUlaz#
#\ulaz{9}#
#\ulaz{20180001 98 704}#
#\ulaz{20180002 33 704}#
#\ulaz{20180003 7 718}#
#\ulaz{20180005 61.8 rlab}#
#\ulaz{20180006 50.5 bim}#
#\ulaz{20180007 55.6 718}#
#\ulaz{20180008 51 704}#
#\ulaz{20180009 30 rlab}#
#\ulaz{20180010 40.4 rlab}#
#\ulaz{704}#
#\naslovIzlaz#
#\izlaz{2}#
\end{test}
\end{minitest}
\begin{minitest}
\begin{test}{2}
#\naslovUlaz#
#\ulaz{7}#
#\ulaz{20180003 73 718}#
#\ulaz{20180005 60.8 rlab}#
#\ulaz{20180006 40.5 bim}#
#\ulaz{20180007 45.6 718}#
#\ulaz{20180008 19.9 704}#
#\ulaz{20180009 31.4 rlab}#
#\ulaz{20180010 49.4 rlab}#
#\ulaz{rlab}#
#\naslovIzlaz#
#\izlaz{1}#
\end{test}
\end{minitest}
\begin{minitest}
\begin{test}{3}
#\naslovUlaz#
#\ulaz{4}#
#\ulaz{20180001 98 704}#
#\ulaz{20180002 33 704}#
#\ulaz{20180003 73.5 718}#
#\ulaz{20180005 60.8 rlab}#
#\ulaz{bim}#
#\naslovIzlaz#
#\izlaz{0}#
\end{test}
\end{minitest}

\begin{minitest}
\begin{test}{4}
#\naslovUlaz#
#\ulaz{-4}#
#\naslovIzlaz#
#\izlaz{-1}#
\end{test}
\end{minitest}

\linkresenje{A_i_2_4}
\end{Exercise}

\ifresenja
\begin{Answer}[ref=A_i_2_4]
\includecode{resenja/A_IspitniRokovi/rok_i_februar/4.c}
\end{Answer}


\section{Rešenja}
\shipoutAnswer